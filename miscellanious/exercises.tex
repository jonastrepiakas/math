\documentclass[a4paper]{article}

\usepackage[margin=2.5cm]{geometry}
\usepackage[pdftex]{graphicx}
\usepackage[utf8]{inputenc}
\usepackage[T1]{fontenc}
\usepackage{textcomp}
\usepackage{babel}
\usepackage{amsmath, amssymb}
\usepackage[colorlinks=true,linkcolor=blue]{hyperref}
\usepackage{float}
\usepackage{mathrsfs}
%\usepackage{enumitem}

\newcommand{\incfig}[2][1]{%
\def\svgwidth{#1\columnwidth}
\import{./figures/}{#2.pdf_tex}
}


% figure support
\usepackage{import}
\usepackage{xifthen}
\pdfminorversion=7
\usepackage{pdfpages}
\usepackage{transparent}

\pdfsuppresswarningpagegroup=1

\setlength\parindent{0pt}

\newcommand{\qed}{\tag*{$\blacksquare$}}
\newcommand{\qedwhite}{\hfill \ensuremath{\Box}}

%Inequalities
\newcommand{\cycsum}{\sum_{\mathrm{cyc}}}
\newcommand{\symsum}{\sum_{\mathrm{sym}}}
\newcommand{\cycprod}{\prod_{\mathrm{cyc}}}
\newcommand{\symprod}{\prod_{\mathrm{sym}}}

%Linear Algebra

%Redeclaring Span and image
\DeclareMathOperator{\Span}{span}
\DeclareMathOperator{\Ima}{Im}
\DeclareMathOperator{\diag}{diag}
\DeclareMathOperator{\Ker}{Ker}

%Row operations
\newcommand{\elem}[1]{% elementary operations
\xrightarrow{\substack{#1}}%
}

\newcommand{\lelem}[1]{% elementary operations (left alignment)
\xrightarrow{\begin{subarray}{l}#1\end{subarray}}%
}

%SS
\DeclareMathOperator{\supp}{supp}
\DeclareMathOperator{\Var}{Var}

%NT
\DeclareMathOperator{\ord}{ord}

\DeclareMathAlphabet{\pazocal}{OMS}{zplm}{m}{n}
\newcommand{\unif}{\pazocal{U}}

\begin{document}
    


\section*{Ring Theory}
\subsection*{Dummit and Foote}
\subsubsection*{Section 7.6}
\textbf{Exercise 2:} Let $R$ be a finite Boolean ring with identity $1\neq 0$. Prove
that $R \cong \mathbb{Z} / 2\mathbb{Z} \times  \cdots \times
\mathbb{Z}/2\mathbb{Z}$.\\
\linebreak
\textit{Solution:} 
Let $R = \left\{0, a_1, \ldots, a_n \right\} $. By the previous problem,
\[
R \cong Ra_1 \times R (1-a_1)
\] 
Now, we showed that $a_1$ is identity for $Ra_1$ and $(1-a_1)$ is an identity
for $R(1-a_1)$. If $a_k \in Ra_1 \cap R (1-a_1)$ then
$a_k = a_k a_1 (1-a_1) = a_k \left( a_1-a_1^2 \right) = a_k \left( a_1 -a_1 \right) 
= 0$. Thus $Ra_1 \cap R(1-a_1) = 0$. Also,
both $Ra_1$ and $R(1-a_1)$ are nonempty, since
$a_1 = a_1 a_1 \in Ra_1$ and $1-a_1 = (1-a_1)^2 \in R (1-a_1)$ - and also it
always contains $0$, so we see that the separation, at least $2$ elements are
in each new set.\\
Summarizing, we see that under such a separation, the cardinality of each new
ideal has decreased by at least $1$ and is also greater than or equal to $2$.\\
Since each new ideal is also Boolean, we can choose any nonzero element and
repeat the process. Then after at most $n$ reiterations, we find a product of
ideals of $R$ of cardinality $2$ each (containing $0$ and some $a_k$ ). Each of
these is isomorphic to $\mathbb{Z} /2\mathbb{Z}$, hence we find
 \[
R \cong \mathbb{Z} /2\mathbb{Z} \times \cdots \times \mathbb{Z} /2\mathbb{Z}.
\] 







\subsection*{Ekstraopgaver - Alg2}

\subsubsection*{Ugeseddel 5}

\textbf{Ekstraopgave 1:} (a) Assume $a,b \in \mathbb{Z}$ are coprime. Let $u
\in \mathbb{Z}$ be the inverse to $b$ mod $a^2 + b^2$. Show that the map
\[
    \varphi  \colon \mathbb{Z}[i] \to \mathbb{Z}/ \left( a^2 + b^2 \right) \mathbb{Z}
    \quad x+yi \mapsto \left[ x-auy \right]_{\left( a^2 + b^2 \mathbb{Z} \right) }
\] 
is a surjective ring homomorphism with $\Ker \varphi = (a+bi)$.\\
\linebreak
\textit{Solution:} All except the kernel part is trivial.\\
Assume $\varphi (x+iy) = 0$. Then
$x - auy \equiv 0 \pmod{a^2 + b^2}$, so $
xb - ay \equiv 0 \pmod{a^2 + b^2}$. Thus
$xb - ay = a^2 t + b^2 t$ for some $t \in \mathbb{Z}$.
Now we find
$xb \equiv b^2 t \implies x \equiv bt \pmod{a}$, so for some $s \in
\mathbb{Z}$,
$x = bt + as$. Likewise, $-ay \equiv a^2 t \implies y \equiv -at \pmod{b}$, so
for some $r \in \mathbb{Z}$, $y = br - at$. Now putting it together, we have
\[
a^2 t + b^2 t = xb - ay = b^2 t + abs - abr + a^2 t \implies ab (s-r)
= 0 \implies s=r.
\] 
Thus we have
\[
x + iy = bt +ar + i (br - at) = bt +ar + ibr - iat
= (a+bi) (r - it).
\] 
Hence $x + iy \in (a+bi)$. The converse inclusion is trivial.\\
\linebreak
(b) We thus find by (a) that e.g.
\[
    \mathbb{Z}[i]/(5) = \mathbb{Z}[i] / ( (2+i)(2-i))
    \cong \mathbb{Z}[i] /(2+i) \times \mathbb{Z}[i]/(2-i)
    \cong \mathbb{Z}/5\mathbb{Z} \times \mathbb{Z}/5\mathbb{Z}.
\] 
Likewise, in general, for all primes $p \equiv 1 \pmod{4}$, we have
\[
    \mathbb{Z}[i]/(p) \cong \mathbb{Z}/p\mathbb{Z} \times \mathbb{Z}/p\mathbb{Z}
\] 
by Fermat's theorem on sums of squares - note: the coprime condition is
satisfies because $p$ is prime. Thus, we cannot do likewise for any $n$ that
can be written as the sum of two squares.




















\end{document}
