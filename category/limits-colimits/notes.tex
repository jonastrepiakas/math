\documentclass[reqno]{amsart}
\usepackage{amscd, amssymb, amsmath, amsthm}
\usepackage{graphicx}
\usepackage[colorlinks=true,linkcolor=blue]{hyperref}
\usepackage[utf8]{inputenc}
\usepackage[T1]{fontenc}
\usepackage{textcomp}
\usepackage{babel}
%% for identity function 1:
\usepackage{bbm}
%%For category theory diagrams:
\usepackage{tikz-cd}


\setlength\parindent{0pt}

\pdfsuppresswarningpagegroup=1

\newtheorem{theorem}{Theorem}[section]
\newtheorem{lemma}[theorem]{Lemma}
\newtheorem{proposition}[theorem]{Proposition}
\newtheorem{corollary}[theorem]{Corollary}
\newtheorem{conjecture}[theorem]{Conjecture}

\theoremstyle{definition}
\newtheorem{definition}[theorem]{Definition}
\newtheorem{example}[theorem]{Example}
\newtheorem{exercise}[theorem]{Exercise}
\newtheorem{problem}[theorem]{Problem}
\newtheorem{question}[theorem]{Question}

\theoremstyle{remark}
\newtheorem*{remark}{Remark}
\newtheorem*{note}{Note}
\newtheorem*{solution}{Solution}



%Inequalities
\newcommand{\cycsum}{\sum_{\mathrm{cyc}}}
\newcommand{\symsum}{\sum_{\mathrm{sym}}}
\newcommand{\cycprod}{\prod_{\mathrm{cyc}}}
\newcommand{\symprod}{\prod_{\mathrm{sym}}}

%Linear Algebra

\DeclareMathOperator{\Span}{span}
\DeclareMathOperator{\im}{im}
\DeclareMathOperator{\diag}{diag}
\DeclareMathOperator{\Ker}{Ker}
\DeclareMathOperator{\ob}{ob}
\DeclareMathOperator{\Hom}{Hom}
\DeclareMathOperator{\Mor}{Mor}
\DeclareMathOperator{\sk}{sk}
\DeclareMathOperator{\Vect}{Vect}
\DeclareMathOperator{\Set}{Set}
\DeclareMathOperator{\Group}{Group}
\DeclareMathOperator{\Ring}{Ring}
\DeclareMathOperator{\Ab}{Ab}
\DeclareMathOperator{\Top}{Top}
\DeclareMathOperator{\hTop}{hTop}
\DeclareMathOperator{\Htpy}{Htpy}
\DeclareMathOperator{\Cat}{Cat}
\DeclareMathOperator{\CAT}{CAT}
\DeclareMathOperator{\Cone}{Cone}
\DeclareMathOperator{\dom}{dom}
\DeclareMathOperator{\cod}{cod}
\DeclareMathOperator{\Aut}{Aut}
\DeclareMathOperator{\Mat}{Mat}
\DeclareMathOperator{\Fin}{Fin}
\DeclareMathOperator{\rel}{rel}
\DeclareMathOperator{\Int}{Int}
\DeclareMathOperator{\sgn}{sgn}
\DeclareMathOperator{\Homeo}{Homeo}
\DeclareMathOperator{\SHomeo}{SHomeo}
\DeclareMathOperator{\PSL}{PSL}
\DeclareMathOperator{\Bil}{Bil}
\DeclareMathOperator{\Sym}{Sym}
\DeclareMathOperator{\Skew}{Skew}
\DeclareMathOperator{\Alt}{Alt}
\DeclareMathOperator{\Quad}{Quad}
\DeclareMathOperator{\Sin}{Sin}
\DeclareMathOperator{\Supp}{Supp}
\DeclareMathOperator{\Char}{char}
\DeclareMathOperator{\Fun}{Fun}


%Row operations
\newcommand{\elem}[1]{% elementary operations
\xrightarrow{\substack{#1}}%
}

\newcommand{\lelem}[1]{% elementary operations (left alignment)
\xrightarrow{\begin{subarray}{l}#1\end{subarray}}%
}

%SS
\DeclareMathOperator{\supp}{supp}
\DeclareMathOperator{\Var}{Var}

%NT
\DeclareMathOperator{\ord}{ord}

%Alg
\DeclareMathOperator{\Rad}{Rad}
\DeclareMathOperator{\Jac}{Jac}

%Misc
\newcommand{\SL}{{\mathrm{SL}}}
\newcommand{\mobgp}{{\mathrm{PSL}_2(\mathbb{C})}}
\newcommand{\id}{{\mathrm{id}}}
\newcommand{\Mod}{{\mathrm{Mod}}}
\newcommand{\PMod}{{\mathrm{PMod}}}
\newcommand{\SMod}{{\mathrm{SMod}}}
\newcommand{\ud}{{\mathrm{d}}}
\newcommand{\Vol}{{\mathrm{Vol}}}
\newcommand{\Area}{{\mathrm{Area}}}
\newcommand{\diam}{{\mathrm{diam}}}
\newcommand{\End}{{\mathrm{End}}}


\newcommand{\reg}{{\mathtt{reg}}}
\newcommand{\geo}{{\mathtt{geo}}}

\newcommand{\tori}{{\mathcal{T}}}
\newcommand{\cpn}{{\mathtt{c}}}
\newcommand{\pat}{{\mathtt{p}}}

\let\Cap\undefined
\newcommand{\Cap}{{\mathcal{C}}ap}
\newcommand{\Push}{{\mathcal{P}}ush}
\newcommand{\Forget}{{\mathcal{F}}orget}




\begin{document}





\section{Preparation}

\subsection{Representability}

\begin{definition}[Initial and terminal objects]
    An object $c$ in a category $C$ is initial if
    and only if the functor
    $C \left( c,- \right) \colon C \to \Set$ is
    naturally isomorphic to the constant functor
    $* \colon C \to \Set$ that sends every object to
    the singleton set.
    That is, for any $d,e \in C $ and any
    $f \colon d \to e$, we have
    
    \begin{equation*}
    \begin{tikzcd}
        C(c,d) \ar[d, "f_*"] \ar[r, "\approx"] & \left\{ * \right\} 
        \ar[d] \\
        C(c,e) \ar[r, "\approx"] & \left\{ * \right\} 
    \end{tikzcd}
    \end{equation*}
    Since these are isomorphisms in set,
    we can conclude that
    $C(c,d)$ and $C(c,e)$ are singletons, and
    as these is only one map between the singleton
    sets $\left\{ * \right\} \to \left\{ * \right\} $, we
    conclude that
    $\Hom \left( C(c,d), C(c,e) \right) $ is also
    a singleton set.
    
    
\end{definition}

\begin{theorem}[Yoneda lemma]
    For any functor $F \colon C \to \Set$, whose
    domain $C$ is locally small and any object
     $c \in C$, there is a bijection
     \[
     \Hom \left( C(c,-),F \right) \cong Fc
     \] 
     that associates a natural transformation
     $\alpha \colon C\left( c,- \right) \implies F$ to the
     elements $\alpha_{c}\left( \mathbbm{1}_c \right) 
     \in Fc$. Moreover, this correspondence is natural
     in both $c$ and $F$.
\end{theorem}


\section{Limits and colimits}



    \begin{definition}[]
        A diagram of shame $J$ in a category $C$ is 
        a functor $F \colon J \to C$.
    \end{definition}

    \begin{definition}[]
        For any object $c \in C$ and any category
        $J$, the constant functor $c \colon J \to C$ 
        sends every object of $J$ to $c$ and every
        morphism in $J$ to the identity morphism
        $\mathbbm{1}_c$.
    \end{definition}

    \begin{definition}[Embedding]
        Recall that an embedding is a faithful functor that
        is injective on objects.\\
        A fully faithful functor that is injective on
        objects defines a full embedding. The domain
        then defines a full subcategory of the codomain.
    \end{definition}



    \begin{remark}[]
        The constant functors define an embedding
        $\Delta \colon C \to \Fun (J,C)$
        sending $c$ to the constant functor at
        $c$ and a morphism $f \colon c \to c'$ to
        the constant natural transformation, in
        which each component is defined to be the morphism
        $f$.
    \end{remark}

    \begin{definition}[Cones]
        A cone over a diagram $F \colon J \to C$ with
        summit or apex $c \in C$ is a natural
        transformation $\lambda \colon c \implies F$, so
        $\lambda \in \Fun (J,C)$, whose
        domain is the constant functor at $c$.
        The components
        $\left( \lambda_j \colon c \to Fj \right)_{j \in J}$ 
        of the natural transformation are
        called the legs of the cone.

        \begin{equation*}
        \begin{tikzcd}
            & c \ar[ld, "\lambda_j"'] \ar[dr, "\lambda_k"]&\\
            F_j \ar[rr, "Ff"'] & & F_k
        \end{tikzcd}
        \end{equation*}

    \end{definition}

    \begin{definition}[Cocones]
        Dually, a cone under $F$ with nadir $c$ is a natural
        transformation $\lambda \colon F \implies c$ whose
        legs are the components
        $\left( \lambda_j \colon F_j \to c \right)_{j 
        \in J}$.o
        So naturality asserts that for each
        morphism $f \colon j \to k$ of $J$, the triangle

        \begin{equation*}
        \begin{tikzcd}
            F_j \ar[rr, "Ff"] \ar[dr, "\lambda_j"'] & & F_k
            \ar[dl, "\lambda_k"] \\
                                                   & c &
        \end{tikzcd}
        \end{equation*}
       commutes in $C$. 
       Cones under a diagram are also called cocones.
    \end{definition}

    \begin{definition}[Limits and colimits I]
        For any diagram $F \colon J \to C$, there
        is a functor
        \[
        \Cone (-,F) \colon C^{op} \to \Set
        \] 
        that sends $c \in C$ to the
        set of cones over $F$ with
        summit $c$.

        A limit of $F$ is a representation
        for $\Cone \left( -,F \right) $.
    \end{definition}








    %\bibliography{../refs.bib}
\end{document}
