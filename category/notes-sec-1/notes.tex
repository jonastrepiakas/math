\documentclass[a4paper]{article}

\usepackage[margin=2.5cm]{geometry}
\usepackage[pdftex]{graphicx}
\usepackage[utf8]{inputenc}
\usepackage[T1]{fontenc}
\usepackage{textcomp}
\usepackage{babel}
\usepackage{amsmath, amssymb, amsthm}
\usepackage[colorlinks=true,linkcolor=blue]{hyperref}
\usepackage{float}
\usepackage{mathrsfs}
%\usepackage{enumitem}
%% for identity function 1:
\usepackage{bbm}
%%For category theory diagrams:
\usepackage{tikz-cd}
%%For code (e.g. python) in latex:
%\usepackage{listings}
%
%Usage: 
%\begin{lstlisting}[language=Python]
%\end{lstlisting}

\newcommand{\incfig}[2][1]{%
\def\svgwidth{#1\columnwidth}
\import{./figures/}{#2.pdf_tex}
}


\theoremstyle{plain}% default
\newtheorem{theorem}{Theorem}[section]
\newtheorem{lemma}[theorem]{Lemma}
\newtheorem{proposition}[theorem]{Proposition}
\newtheorem*{corollary}{Corollary}


\theoremstyle{definition}
\newtheorem{defn}{Definition}[section]
\newtheorem{example}{Example}[section]
\newtheorem{exercise}[example]{Exercise}


\theoremstyle{remark}
\newtheorem*{remark}{Remark}
\newtheorem*{note}{Note}






% figure support
\usepackage{import}
\usepackage{xifthen}
\pdfminorversion=7
\usepackage{pdfpages}
\usepackage{transparent}

\pdfsuppresswarningpagegroup=1

\setlength\parindent{0pt}

\newcommand{\qedwhite}{\hfill \ensuremath{\Box}}

%Inequalities
\newcommand{\cycsum}{\sum_{\mathrm{cyc}}}
\newcommand{\symsum}{\sum_{\mathrm{sym}}}
\newcommand{\cycprod}{\prod_{\mathrm{cyc}}}
\newcommand{\symprod}{\prod_{\mathrm{sym}}}

%Linear Algebra

\DeclareMathOperator{\Span}{span}
\DeclareMathOperator{\Ima}{Im}
\DeclareMathOperator{\diag}{diag}
\DeclareMathOperator{\Ker}{Ker}
\DeclareMathOperator{\ob}{ob}
\DeclareMathOperator{\Hom}{Hom}
\DeclareMathOperator{\sk}{sk}
\DeclareMathOperator{\Vect}{Vect}
\DeclareMathOperator{\Set}{Set}
\DeclareMathOperator{\Group}{Group}
\DeclareMathOperator{\Ring}{Ring}
\DeclareMathOperator{\Ab}{Ab}
\DeclareMathOperator{\Top}{Top}
\DeclareMathOperator{\hTop}{hTop}
\DeclareMathOperator{\Htpy}{Htpy}
\DeclareMathOperator{\Cat}{Cat}
\DeclareMathOperator{\CAT}{CAT}
\DeclareMathOperator{\Cone}{Cone}
\DeclareMathOperator{\dom}{dom}
\DeclareMathOperator{\cod}{cod}
\DeclareMathOperator{\Aut}{Aut}
\DeclareMathOperator{\Mat}{Mat}
\DeclareMathOperator{\Fin}{Fin}




%Row operations
\newcommand{\elem}[1]{% elementary operations
\xrightarrow{\substack{#1}}%
}

\newcommand{\lelem}[1]{% elementary operations (left alignment)
\xrightarrow{\begin{subarray}{l}#1\end{subarray}}%
}

%SS
\DeclareMathOperator{\supp}{supp}
\DeclareMathOperator{\Var}{Var}

%NT
\DeclareMathOperator{\ord}{ord}

%Alg
\DeclareMathOperator{\Rad}{Rad}
\DeclareMathOperator{\Jac}{Jac}

\DeclareMathAlphabet{\pazocal}{OMS}{zplm}{m}{n}
\newcommand{\unif}{\pazocal{U}}

\begin{document}



\subsection*{Section 1.1}

\textbf{iii.} For any category $C$ and any object $c \in C$, show that:\\
\begin{enumerate}
    \item There is a category $c /C$ whose objects are morphisms $f  \colon
        c \to x$ with domain $c$ and in which a morphism from
         $f  \colon c \to x$ to $g  \colon c \to y$ is a map
         $h  \colon x \to y$ between the codomains so that the triangle
         \begin{equation*}
         \begin{tikzcd}
             & c \ar[dl, "f"'] \ar[dr, "g"] &\\
             x \ar[rr, "h"'] & & y
         \end{tikzcd}
         \end{equation*}
         \textbf{commutes}, i.e., so that $g = hf$.
     \item There is a category $C /c$ whose objects are morphisms $f  \colon
         x \to c$ with codomain $c$ and in which a morphism from
         $f  \colon x \to c$ to $g  \colon y \to c$ is a map 
         $h  \colon x \to y$ between the domains so that the triangle
         \begin{equation*}
         \begin{tikzcd}
             x \ar[dr, "f"'] \ar[rr, "h"] & & y \ar[dl, "g"]\\
                                         & c &
         \end{tikzcd}
         \end{equation*}
         \textbf{commutes}, i.e., so that $f = gh$.
\end{enumerate}
The categories $c /C$ and $C / c$ are called the \textbf{slice categories} of
$C$ \textbf{under} and  \textbf{over} $c$, respectively.\\
\linebreak
\textit{Solution:} Each element has identity since
\begin{equation*}
\begin{tikzcd}
    & c \ar[dl, "f"'] \ar[dr, "f"] &\\
    x \ar[rr, "\mathbbm{1}"'] & & x
\end{tikzcd}
\end{equation*}
commutes for any $f  \colon c \to x \in \Hom \left( c /C \right) $.\\
Composites exist since if each of the smaller triangles in the diagram
\begin{equation*}
\begin{tikzcd}
    & c \ar[dl, "f"'] \ar[d, "g"] \ar[dr, "h"] &\\
    x \ar[r, "\alpha"'] & y \ar[r, "\beta"'] & z
\end{tikzcd}
\end{equation*}
commutes, then 
$h = \beta \cdot g = \beta \cdot  \left( \alpha \cdot f \right) 
= \left( \beta \cdot \alpha \right) \cdot f$, so
$\beta \cdot \alpha  \colon x \to z \in \Hom\left( c /C \right) $.\\
\begin{equation*}
\begin{tikzcd}
    & c \ar[dl, "f"'] \ar[d, "f"] \ar[dr, "g"] &\\
    x \ar[r, "\mathbbm{1}_x"'] & x \ar[r, "h"'] & y
\end{tikzcd}
\end{equation*}
so since the outer triangles commutes, $h \cdot \mathbbm{1}_x \in 
\Hom \left( c /C \right) $. And $\mathbbm{1}_y \cdot h$ similarly.\\
Now if  $\alpha  \colon x \to y, \beta  \colon y \to z, \gamma  \colon z \to w$ then
\begin{equation*}
\begin{tikzcd}
    & & c\ar[dll, "f"'] \ar[dl, "g"] \ar[dr, "h"'] \ar[drr, "j"]  & &\\
    x \ar[r, "\alpha"'] & y \ar[rr, "\beta"']& & z \ar[r, "\gamma"'] & w
\end{tikzcd}
\end{equation*}
and clearly $\left( \gamma \cdot  \beta \right) \cdot \alpha
= \gamma \cdot \left( \beta \cdot \alpha \right) $.\\
\linebreak







\subsection*{Section 1.3}

\textbf{Example 1.3.7.(v)} For a generic small category $C$, a functor
$C^{op} \to \Set$ is called a (set-valued) \textbf{presheaf} on
$C$. A typical example is the functor $\mathcal{O}(X)^{op}\to \Set$ whose
domain is the poset $\mathcal{O}(X)$ of open subsets of a topological space $X$
and whose value at $U \subset X$ is the set of continuous real-valued functions on
$U$. The action on morphisms is by restriction.\\
So suppose we have spaces $X$ and $Y$ and a continuous map $f  \colon X \to Y$
inducing a morphism $f^{-1}  \colon \mathcal{O}(Y) \to \mathcal{O}(X)$. Let
$F  \colon \mathcal{O}(X)^{op} \to \Set$ be the given functor. Now, an object
in $\mathcal{O}(X)^{op}$ is an open subset $U \subset X$, so $F(U)$ is mapped
to the set of continuous real-valued functions on $U$, $\mathcal{C}(U,\mathbb{R}) \subset
\Set$. A morphism  $U \to V$ means that $U \hookrightarrow V$ is a continuous
inclusion.  Now, if $U \hookrightarrow V$ then any function continuous on $V$
is continuous on $U$, so $F(V) \subset F(U)$. In particular, we define
$F(f  \colon U \to V) = F(V) \hookrightarrow F(U)$.\\
\linebreak
This map is functorial since if $U \stackrel{f}{\to } V \stackrel{g}{\to
} W $    then
$F(gf) = F(W) \hookrightarrow F(U) = F(W) \hookrightarrow F(V) \hookrightarrow
F(U) = F(f) F(g)$.\\
 \linebreak
 \textbf{Example vi.} Presheaves on the category $\Delta$, of finite non-empty
 ordinals and order-preserving maps, are called \textbf{simplicial sets}.
 $\Delta$ is also called the \textbf{simplex category}. The ordinal
 $n+1 = \left\{ 0, 1, \ldots, n \right\} $ may be throught of as a direct
 version of the topological $n$-simplex and, with this interpretation in mind,
 is typically denoted by "$\left[ n \right] $ " by algebraic topologists.\\
 \linebreak
 \textbf{Def.} A \textbf{monoid} is a set $M$ equipped with an associative
 binary operation $M \times M \to M$ and an identity $e \in M$ serving as
 a two-sided identity. In other words, a monoid is precisely a one-objects
 category.\\
 A commutative monoid is then one in which $\forall a,b \in M  \colon ab
 = ba$.\\
 \linebreak
We can also state this as saying that a \textbf{monoid} is an object $M \in
\Set$ together with a pair of morphisms $\mu  \colon M \times M \to M$ and
$\eta  \colon 1 \to M$ so that the following diagrams commute:

\begin{center}
\begin{tikzcd}
    M \times M \times M \ar[d, "\mu \times \mathbbm{1}_M"] \ar[r,
    "\mathbbm{1}_M \times \mu"] & M \times M \ar[d, "\mu"]\\
    M \times M \ar[r, "\mu"] & M
\end{tikzcd}
\begin{tikzcd}
    M \ar[r, "\eta \times \mathbbm{1}_{M}"] \ar[rd, "\mathbbm{1}_M"] &
    M \times M \ar[d, "\mu"] & M \ar[l, "\mathbbm{1}_M \times \eta"] \ar[dl,
    "\mathbbm{1}_M"]\\
                             & M &
\end{tikzcd}
\end{center}
The first diagram gives associativity, and the second gives the $\eta$ acts as
a two-sided identity. Note that $\eta  \colon 1 \to M$ identifies an object in
$M$ which we call $\eta$, so in the morphism  $M \to M \times M$ by
$\eta \times \mathbbm{1}_{M}$ we map an object $m \in M$ to
$\eta \times m \in M \times M$.\\
\linebreak







 \textbf{Example 1.3.2.(xi)} Any commutative monoid $M$ can be used to define
 a functor $M^{-}  \colon \Fin_{*} \to \Set$. Writing $n_+ \in \Fin_*$ for the
 set with $n$ non-basepoint elements, define $M^{n_+}$ to be
 $M^{n}$, the $n$-fold cartesian product of the set $M$ with itself. By
 convention, $M^{0_+}$ is a singleton set. For any based map
 $f  \colon m_+ \to n_+$, define the $i$ th component of the corresponding
 function
 $M^{f}  \colon M^{m} \to M^{n}$ by projecting from $M^{m}$ to the coordinates
 indexed by elements in the fiber $f^{-1}(i)$ and then multiplying these using
 the commutative monoid structure; if the fiber is empty, the function $M^{f}$ 
 inserts the unit element in the $i$ th coordinate. Note each of the sets
 $M^{n}$ itself has a basepoint, the $n$-tuple of unit elements, and each of
 the maps in the image of the functor are based. It follows that the functor
 $M^{-}$ lifts along the forgetful functor $U  \colon \Set_* \to \Set$.\\
 \linebreak
 There is a special property satisfied by this construction that allows one to
 extract the commutative monoid $M$ from the functor $\Fin_* \to \Set$. This
 observation was used by Segal to introduce a suitable notion of "commutative
 monoid" into algebraic topology.\\
 \linebreak
 See section 1.5, exercise 2 for continuation.
 
 


\subsection*{Section 1.3 - Problems}
\textbf{i.} What is a functor between groups, regarded as one-object
categories?\\
\linebreak
\textit{Solution:} 
Suppose $C, D$ are categories with one object each representing the groups
$G,H$, respectively. Then each morphism
represents an element of the groups.\\
Let $F  \colon C \to D$ be a functor. Then
$F$ maps the object to the object, and if $a,b$ are morphisms in $C$, then
$F \left( a b \right) = Fa Fb$. Thus $F$ is simply a group homomorphism between the
groups.\\
\linebreak
\textbf{ii.} What is a functor between preorders, regarded as categories?\\
\linebreak
\textit{Solution:} Suppose
$(P, \le )$ and $(Q,<)$ are preorder categories.\\
Let $F  \colon P \to Q$ be a functor. 
If $a,b \in P$ and $a \le b$ then $\exists \alpha  \colon a \to b$, and
$F(\alpha)  \colon Fa \to Fb$, so a functor between preorders is just any map
that preserves order.\\
\linebreak
\textbf{v:} What is the difference between a functor $C^{op} \to D$ and
a functor
$C \to D^{op}$? What is the difference between a functor $C \to D$ and
a functor
$C^{op} \to D^{op}$?\\
\linebreak

\textit{Solution:}
Suppose $F  \colon C^{op} \to D$ is a functor. 
We claim that $F^{op}  \colon C \to D^{op}$ defined by
being equal to $F$ on objects and for $\alpha \in \Hom\left( C \right) $, we
have
$F^{op} \left( \alpha \right) = \left( F \left( \alpha^{op} \right) \right)^{op} $.\\
This is a functor since for a composable pair
$\alpha, \beta \in \Hom(C)$, we have
$F^{op} \left( \alpha \beta \right) 
= F\left( \left( \alpha \beta \right)^{op} \right)^{op} 
= F \left( \beta^{op} \alpha^{op} \right)^{op}
= \left( F\left( \beta^{op} \right) F\left( \alpha^{op} \right) \right)^{op}
= F\left( \alpha^{op} \right)^{op} F\left( \beta^{op} \right)^{op}
= F^{op}(\alpha) F^{op} (\beta)$, and
$F^{op} \left( \mathbbm{1}_c \right) 
= F \left( \mathbbm{1}_c^{op} \right)^{op}
= \mathbbm{1}_{F c}^{op}
= \mathbbm{1}_{F^{op} c}^{op}$, so
$F^{op}$ is a covariant functor $C \to D^{op}$. Hence
each covariant  $F  \colon C^{op} \to D$ induces a covariant
$F^{op}  \colon C \to D^{op}$. Arguing by duality, we get that each functor 
$C^{op} \to D$ corresponds to a functor $C \to D^{op}$.\\
\linebreak
For a functor $F  \colon C \to D$, we get that if $\alpha  \colon a \to b$,
then
$F \alpha  \colon Fa \to Fb$. So
defining $F^{op}  \colon C^{op} \to D^{op}$ by
$F^{op} \left( \alpha^{op} \right) =
F \left( \alpha \right)^{op}$, we get that
$F^{op} \left( \alpha^{op} \beta^{op} \right) 
= F^{op} \left( \left( \beta \alpha \right)^{op} \right) 
= F\left(  \beta \alpha \right)^{op}
= F\left( \alpha \right)^{op} F\left( \beta \right)^{op}
= F^{op} \left( \alpha^{op} \right) F^{op} \left( \beta^{op} \right) $, and
$F^{op} \left( \mathbbm{1}_{c}^{op} \right) 
= F\left( \mathbbm{1}_c \right)^{op}
= \mathbbm{1}_{F c}^{op}
= \mathbbm{1}_{F^{op} c}^{op}$.\\
\linebreak


\textbf{vi (The comma category).} Given functors $F  \colon D \to C$ and
$G  \colon E \to C$, show that there is a category called the
\textbf{comma category} $F \downarrow G$, which has
\begin{itemize}
    \item as objects, triples $\left( d \in D, e \in E, 
        f  \colon Fd \to Ge \in C \right) $, and
    \item as morphisms $\left( d,e,f \right) 
        \to \left( d', e', f' \right) $, a pair of morphisms
        $\left( h  \colon d \to d', k  \colon e\to e' \right) $ so that the
        square
        \begin{equation*}
        \begin{tikzcd}
            Fd \ar[d, "Fh"] \ar[r, "f"] & Ge \ar[d, "Gk"]\\
            Fd' \ar[r, "f'"] & Ge'
        \end{tikzcd}
        \end{equation*}
        commutes in $C$, i.e., so that $f' \cdot  Fh = Gk \cdot  f$
\end{itemize}
Define a pair of projection functors 
$\dom  \colon F \downarrow G \to D$ and
$\cod  \colon F \downarrow G \to E$.\\
\linebreak
\textit{Solution:}
We must firstly check that each object has an identity morphism.\\
Now, since
each object in $D$ and $E$ has an identity, we have
that since
\begin{equation*}
\begin{tikzcd}
    Fd \ar[d, "F \mathbbm{1}_d = \mathbbm{1}_{Fd}"']  \ar[r, "f"] & Ge \ar[d, "G
    \mathbbm{1}_e = \mathbbm{1}_{Ge}"]\\
    Fd \ar[r, "f"'] & Ge
\end{tikzcd}
\end{equation*}
obviously commutes, the morphism
$\left( 1_{Fd}  \colon Fd \to Fd, \mathbbm{1}_{Ge}  \colon Ge \to Ge \right)
$ is the identity of the object\\
$\left( d \in D, e \in E, f  \colon Fd \to Ge \in C \right) $.\\
\linebreak
Now we must make sure that composition is defined.\\
Suppose $\left( d,e,f \right) \stackrel{\left( h,k \right) }{\to}  \left( d',e',f' \right) 
\stackrel{\left( h', k' \right) }{\to } \left( d'', e'', f'' \right) $. Then

\begin{equation*}
\begin{tikzcd}
    Fd \ar[d, "Fh"'] \ar[r, "f"] & Ge \ar[d, "Gk"]\\
    Fd' \ar[r, "f'"] \ar[d, "Fh'"'] & Ge' \ar[d, "Gk'"]\\
    Fd'' \ar[r, "f''"] & Ge''
\end{tikzcd}
\end{equation*}
Now commutativity of each inner square gives that the outer rectangle commutes,
hence
$f'' \cdot  Fh' \cdot  Fh = Gk' \cdot Gk \cdot f$, so the square
\begin{equation*}
\begin{tikzcd}
    Fd  \ar[d, "F \left( h' \cdot h \right)"'] \ar[r, "f"] & Ge \ar[d, "G \left( k' \cdot k \right)"] \\
    Fd'' \ar[r, "f''"'] & Ge''
\end{tikzcd}
\end{equation*}
commutes by functoriality of $F$ and $G$ and since
$h' \cdot h \in \Hom\left( d, d'' \right) $ and
$k' \cdot k \in \Hom \left( e,e'' \right) $.\\
\linebreak
Now suppose $\left(d, e, f \right) \stackrel{\left( h
 \colon d \to d', k  \colon e\to e' \right) }{\to } \left( d',e',f' \right) $.
 Then
$f' \cdot  \mathbbm{1}_{Fd'} \cdot Fh = \mathbbm{1}_{Ge'} \cdot Gk \cdot f$, so
$f' \cdot F\left( h \right)  =f' \cdot F\left( \mathbbm{1}_{d'} \cdot h \right) 
= G\left( \mathbbm{1}_{e'} \cdot k \right) \cdot f = G\left( k \right) \cdot
f$, so
$\left( \mathbbm{1}_{d'}, \mathbbm{1}_{e'} \right) \circ \left( h,k \right) 
= \left( h,k \right) $, and similarly,
$(h,k) \circ \left( \mathbbm{1}_d, \mathbbm{1}_e \right) $.\\
Associativity requires that
\[
    \left( h'', k'' \right) \circ \left( \left( h',k' \right) \circ \left( h,k
        \right) \right)
    = \left( \left( h'', k'' \right) \circ \left( h',k' \right)  \right) \circ
    \left( h,k \right).
\] 
This is true if
\[
F\left( h'' \cdot  \left( h' \cdot h \right)  \right) 
= F \left( \left( h'' \cdot h' \right) \cdot h \right) \quad \text{and} \quad
G\left( k'' \cdot \left( k' \cdot k \right)  \right) 
= G \left( \left( k'' \cdot k' \right) \cdot k \right) 
\] 
which is true since composition of morphisms in $C$ is associative as it is
a category.\\
\linebreak
Define
$\dom (d,e,f) = d$ and
$\dom \left( h  \colon d \to d', k  \colon e\to e' \right) =
h  \colon d\to d'$. Then
\[
    \dom \left( \left( h'  \colon d' \to d'', k'  \colon e' \to e'' \right) \cdot 
\left( h  \colon d \to d', k  \colon e \to e' \right) \right)
= h' \cdot h  \colon d \to d''
= \dom \left( h', k' \right) \cdot 
\dom \left( h,k \right) .
\]
and similarly for $\cod$.\\
\linebreak
\[
\dom \left( \mathbbm{1}_d, \mathbbm{1}_e \right) 
= \mathbbm{1}_d = \mathbbm{1}_{\dom \left( d,e,f \right) }, \quad \cod \left( \mathbbm{1}_d , \mathbbm{1}_e \right) 
= \mathbbm{1}_e = \mathbbm{1}_{\cod (d,e,f)}.
\] 


\textbf{vii.} Define functors to construct the slice categories $c /C$ and
$C /c$ of exercise 1.1.iii as special cases of comma categories constructed in
exercise 1.3.vi. What are the projection functors?\\
\linebreak
\textit{Solution:} Let $D$ be the single-object category $\mathbbm{1}$ and
$F$ be the functor sending the object to $c$.
Let $E = C$ and $G$ be the trivial functor on $C$. Then
an object consists of
$\left( c, e \in C, f  \colon c \to e \right) $, and morphisms consists
of $\left( h  \colon c \to c, k  \colon e \to e' \right) 
= \left( \mathbbm{1}_c , k  \colon e \to e' \right) $.\\
Then $\dom$ sends any object to the underlying single object in $D$ and
any morphism to the identity in $D$; and $\cod$ sends any object
$\left( d, e,f \right) $ to $e$ and
$\left( h, k \right) $ to $k$.\\
\linebreak

\textbf{viii.} Lemma 1.3.8 shows that functors preserve isos. Find an example
to demonstrate that functors need not \textbf{reflect isos:} i.e., find
a functor
$F  \colon C \to D$ and a morphism $f$ in $C$ so that $F f$ is an iso in $D$
but
$f$ is not an iso in $C$.\\
\linebreak
\textit{Solution:} Suppose simply
$C = \left\{ a,b \right\} $ with identities and
a morphism $f  \colon a \to b$ that is not iso. Then map
it all to the category $\mathbbm{1}$. Any morphism in the image is an iso as
there is only one, thus solving the problem.\\
\linebreak
\textbf{1.3.ix.} The operators $Z(-), C(-), \Aut(-)$ sending a group to its
center, commutator subgroup and automorphism group, respectively, all define
functors on the discrete category of groups (with only identity morphisms) to
$\Group$. Are they functorial in the isomorphisms of groups, i.e., do they
extend to functors $\Group_{\text{iso}} \to \Group$?\\
\textit{Solution:} Suppose $G \cong H$ by $\varphi$, so
$\varphi  \colon G \to H$. Then is
$Z(\varphi)  \colon Z(G) \to Z(H)$ functorial? 
If $h \in Z(G)$ then for all
$g \in Z(H), \varphi(h) \varphi(\varphi^{-1}g)) =
\varphi \left( h \varphi^{-1}(g) \right) 
= \varphi \left( \varphi^{-1}(g) h \right) 
= g \varphi(h)$, so $\varphi(h) \in Z(H)$ and
$Z(\varphi) $.\\
\linebreak




We indeed have that
if $G \cong H$ then $Z(G) \cong Z(H)$ by the restriction of $\varphi$, so
$Z(\varphi)  \colon Z(G) \to Z(H)$ by an iso. Furthermore,
$Z(\mathbbm{1}_G) = \mathbbm{1}_{Z(G)}$ and $Z(\psi \cdot \varphi) =
Z(\psi) \cdot Z(\varphi)$.\\
Now, for $C(-)$, if $G \cong H$ by $\varphi$ with inverse $\psi$ then
if $x = ghg^{-1}h^{-1}$, we have
$\varphi (x) =
\varphi (g) \varphi(h) \varphi(g)^{-1} \varphi(h)^{-1} \in C(H)$, and clearly
$\psi (\varphi(x)) = x \in C(G)$. So
$C(-)$ extends to a functor $\Group_{\text{iso}}\to  \Group$ and clearly
$C(\mathbbm{1}_G) = \mathbbm{1}_{C(G)}$.\\
For $\Aut(-)$, if $G \cong H$ by $\varphi$ with inverse $\psi$, then
$\Aut (\varphi) = \varphi \circ -$. 
Functoriality: $\Aut \left( \varphi \circ \psi \right) 
= \left( \varphi \circ \psi \right) \circ -
= \varphi \circ \left( \psi \circ - \right) 
= \Aut \left( \varphi \right) \circ \Aut \left( \psi \right) $.\\

So each extends to $\Group_{\text{iso}}\to \Group$.\\
\linebreak
Does it extend to $\Group_{\text{epi}}\to \Group$?\\
\linebreak
Functoriality of $Z(-)$. If $\varphi  \colon G \to H$ is a homomorphism such
that if $h \varphi = k \varphi$ then $h = k$ and similarly for $\psi
 \colon H \to J$. 
I don't believe it extends here.\\
\linebreak


\subsection*{Section 1.4}

\textbf{ii.} What is a natural transformation between a parallel pair of
functors between groups, regarded as one-object categories?\\
\linebreak
\textit{Solution:} 
Suppose $\alpha  \colon F \implies G$ is a natural transformation where
$F,G  \colon BG \to BH$. 
So
\begin{equation*}
\begin{tikzcd}
    BH \ar[d, "Ff"'] \ar[r, "\alpha"] & BH \ar[d, "Gf"]\\
    BH \ar[r, "\alpha"'] & BH
\end{tikzcd}
\end{equation*}
commutes. Hence
$\alpha'\cdot  Ff = Gf \cdot \alpha $ for all $f \in \Hom(BG)$. So a natural transformation between
parallel pairs of functors between groups as one-object categories is
an element $g \in \Hom(BH)$, so $g \in H$, such that
$g \cdot Ff = Gf \cdot g$, i.e., $Ff = g^{-1} Gf g$, so
$F(-) = g^{-1} G(-) g$.\\
\linebreak
\textbf{iii.} What is a natural transformation between a parallel pair of
functors between preorders, regarded as categories?\\
\linebreak
\textit{Solution:} Suppose $\alpha  \colon F \implies G$ and
$F,G  \colon P \to Q$ where $(P,\le ), (Q,<)$ are preorders. Then
for any $f  \colon c \to d$ in $P$, we have
\begin{equation*}
\begin{tikzcd}
    Fc \ar[d, "Ff"'] \ar[r, "\alpha_c"] & Gc \ar[d, "Gf"]\\
    Fd \ar[r, "\alpha_d"'] & Gd
\end{tikzcd}
\end{equation*}
commutes. What this means is that
$Fc < Fd < Gd$ and $Fc < Gc < Gd$, so
$\alpha_c$ is the map $Fc \to Gc$ and
$\alpha_d$ is the map $Fd \to Gd$.\\
So $\alpha_x \left( Fx \right) = Gx$ for all $x \in P$. Thus, for all
$x \in P$, $Fx < Gx$, so $F < G$ on all of $P$.\\
\linebreak
\textbf{iv.} In the notation of example 1.4.7, prove that distinct parallel
morphisms $f, g  \colon c\to d$ define distinct natural transformations
\[
f_*, g_*  \colon C\left( -,c \right) \implies C(-,d) \quad
\text{and} \quad f^{*},g^{*}  \colon C(d,-) \implies C(c,-)
\] 
by post- and pre-composition.\\
\linebreak
\textit{Solution:} We must show that $(f_*)_{x} \neq (g_*)_{x}$ for some $x$
and similarly with $f^{*},g^{*}$. Now, if
$h  \colon y\to x$ in $C$, then
\begin{equation*}
\begin{tikzcd}
    C(x,c) \ar[r, "f_*"] \ar[d, "h^{*}"] & C(x,d) \ar[d, "h^{*}"]\\
    C(y,c) \ar[r, "f_*"] & C(y,d)
\end{tikzcd}
\end{equation*}

If $f_*$ and $g_*$ define the same natural transformation, then
$(f_*)_{c} = \left( g_* \right)_c$, but
$\left( f_* \right)_{c} =
\mathbbm{1}_d f \left( \mathbbm{1}_c \right) 
= f(\mathbbm{1}_c) = f$ and
$(g_*)_c = g \left( \mathbbm{1}_c \right) = g $, so
$f = g$, hence $(f_*)_{c} \neq \left( g_* \right)_{c}$.\\
Similarly for $f^{*}$ and $g^{*}$.\\
\linebreak
\textbf{v.} Construct a canonical natural transformation $\alpha
 \colon F \dom \implies G \cod$ between the functors that form the boundary
 of the square
 \begin{equation*}
 \begin{tikzcd}
     F\downarrow G \ar[r, "\cod"] \ar[d, "\dom"'] & E \ar[d, "G"]\\
     D \ar[r, "F"'] \ar[ru, Rightarrow, "\alpha"] & C
 \end{tikzcd}
 \end{equation*}
 \textit{Solution:} 
 Suppose 
  \[
      (d,e,f) \stackrel{\left( h,k \right) }{\to } \left( d',e',f' \right) 
 \] 
so
$F \dom \left( d, e,f \right) 
= Fd$ while
$G \cod \left( d, e,f \right) 
= Ge$, so 
\begin{equation*}
\begin{tikzcd}
    Fd \ar[r, "\alpha_d"] \ar[d, "F h"'] & Ge \ar[d, "G k"]\\
    Fd' \ar[r, "\alpha_{d'}"'] & Ge'
\end{tikzcd}
\end{equation*}
so choosing $\alpha_d = f$ and $\alpha_{d'} = f'$, we get
that the above commutes by definition of the comma category.\\
\linebreak
\textbf{Definition.} A functor $F  \colon C \to D$ is
\begin{itemize}
    \item \textbf{full} if $\forall x,y \in C \colon C(x,y) \to D(Fx, Fy)$ is
        surjective.
    \item \textbf{faithful} if $\forall x,y \in C$, the map
        $C(x,y) \to D(Fx, Fy)$ is injective;
    \item and \textbf{essentially surjective on objects} if for every object $d
        \in D$, $\exists c \in C$ such that $d$ is isomorphic to $Fc$, i.e.,
        $d \cong Fc$.
\end{itemize}
\textbf{Remark:} A faithful functor that is injective on objects is called an
\textbf{embedding} and identifies the domain category as a subcategory of the
codomain; in this case, faithfulness implies that the functor is globally
injective on arrows. A full and faithful functor, called \textbf{fully
faithful}, that is injective-on-objects defines a \textbf{full embedding} of
the domain category into the codomain category. The domain then ddefines
a \textbf{full subcategory} of the codomain.\\
\linebreak
\textbf{Theorem. (Characterizing equivalences of categories)} A functor
defining an equivalence of categories is full, faithful and essentially
surjective on objects. Assuming the axiom of choice, any functor with these
properties defines an equivalence of categories.\\
\linebreak
\textbf{Proof.} 
Suppose $F  \colon C \to D, G  \colon D \to C$ and
$\varepsilon  \colon FG \cong \mathbbm{1}_{D}$ and
$ \eta  \colon GF \cong \mathbbm{1}_C$.\\
\linebreak
Essentially surjective: let $d \in D$, then
$FG d \cong d$ by $\varepsilon_{d}$, so $F$ is essentially surjective.\\
\linebreak
Full: Suppose $Fx, Fy \in D$ and
$d  \colon Fx \to Fy$ is in $D$. Then there exists $x', y' \in D$ such that
$Gx' \stackrel{\alpha}{\cong}x, Gy' \stackrel{\beta}{\cong} y$. Now
\begin{equation*}
\begin{tikzcd}
    Fx \ar[r, "\cong"] \ar[d, "d"] & FGx' \ar[r, "\varepsilon_{x'}"] \ar[d, "F
    \left( \beta \cdot d \cdot \alpha^{-1} \right) "] & x' \ar[d, dashed, "d'"] \\
    Fy \ar[r, "\cong"'] & FGy' \ar[r, "\varepsilon_{y'}"'] & y'
\end{tikzcd}
\end{equation*}
commutes, so there exists a unique map $d'  \colon x' \to y'$ such that the
above commutes, namely, $d' = \varepsilon_{y'} \cdot F(\beta) \cdot d \cdot 
F \cdot  \left( \alpha^{-1} \right) \cdot \varepsilon_{x'}^{-1} $ - unique since $F$
is a functor and hence preserves isomorphisms.\\
\linebreak
Faithful: Suppose $x,y \in C$, and $d  \colon x \to y$ in $C$. Suppose 
$\exists  \tilde{d}  \colon x \to y$ and
$F(d) = F(\tilde{d})$.\\
Then there exist $x', y' \in D  \mid Gx' \cong x, Gy' \cong y$, so
\begin{equation*}
\begin{tikzcd}
    x \ar[d, "d; \tilde{d}"'] \ar[r, "\eta_{x}"] & GF x \ar[d, " Gd = GF d = G\tilde{d}"]\\
    y \ar[r, "\eta_{y}"] & GFy
\end{tikzcd}
\end{equation*}
Now, since $Gd = G\tilde{d}$ and $\eta_x, \eta_y$ are isos, the map $x\to y$
making the diagram commute is uniquely determined to be 
$\eta_y^{-1} Gd \eta_x$, so $d = \tilde{d}$.\\
\linebreak
For the opposite direction, suppose $F  \colon C \to D$ is fully faithful and essentailly
surjective on objects. 
We wish to defined $G  \colon D \to C$ such that
$FG \cong \mathbbm{1}_D$ by $\varepsilon$ and $GF \cong \mathbbm{1}_C$ by
$\eta$.\\
Let $x,y \in D$. Then by essential surjectivity, there must exist $a,b \in C$
such that $Fa \cong x, Fb \cong y$. By fully faithfulness, we have that
$\Hom(a,b) \stackrel{F}{\to } \Hom(x,y)$ is bijective. Define
$G(x)$ to be any element such that $G(x) \cong a$ and similarly  $G(y) \cong
b$. Now for any  $d  \colon x \to y$, we have that there exists $d'  \colon
a \to b$ such that $F(d') = d$, so define $G(d) =q d' p$ where $p $ and $q$ are
isos making the domain and codomain match up with $G(x)$ and $G(y)$. This is
a functor since if $\overline{d}  \colon y \to z$ then
since $Gy = b$, and some $c \in C$ has $Gz \cong c$, we have that
there exists $d''  \colon b \to c$ such that
$G \overline{d} = r d'' p^{-1}$ where $r$ maps $Gz$ to $c$. So
$G \left( \overline{d} d  \right) 
= r d'' d' q = r d'' p^{-1}p d' q
= G(\overline{d}) G(d)$, and $G\left( \mathbbm{1}_x \right) 
= p^{-1} \mathbbm{1}_a p = \mathbbm{1}_{Gx}$.\\
\linebreak

Alternatively as in the book, we can use diagram chasing:\\
Suppose $F  \colon C \to D$ is fully faithful and essentially surjective on
objects. Then using essential surjectivity and the axiom of choice, we can, for
each $d \in D$, choose a $Gd \in C$ such that $\varepsilon_d  \colon FGd \cong
d$. Now, suppose $\alpha  \colon d \to d'$, then there is precisely one
morphism
$G d \to Gd'$ making the following commute:
\begin{equation*}
\begin{tikzcd}
    Gd \ar[d, dashed] \ar[r, "\cong"] & d \ar[d, "\alpha"] \\
    Gd' \ar[r, "\cong"] & d'
\end{tikzcd}
\end{equation*}
so we define
$G \alpha$ as this morphism.\\
\linebreak
We must then check functriality of $G$ :\\

\begin{equation*}
\begin{tikzcd}
    FG d \ar[r, "\varepsilon_d"] \ar[d, "
    FG \mathbbm{1}_d \text{ or } F \mathbbm{1}_{Gd}"'] & d \ar[d, "\mathbbm{1}_d"]\\
    FG d \ar[r, "\varepsilon_d"] & d
\end{tikzcd}
\end{equation*}
By $\varepsilon$ being a natural transformation, $FG \mathbbm{1}_d$ makes the
diagram commute, and since $F$ is a functor, 
$F \mathbbm{1}_{Gd} = \mathbbm{1}_{FG d}$, so
clearly $F \mathbbm{1}_{Gd} = \varepsilon_d^{-1} \mathbbm{1}_d \varepsilon$,
hence, since only one morphism makes the diagram commute,
 $FG \mathbbm{1}_d = F \mathbbm{1}_{Gd}$ and since
 $F$ is faithful, $G \mathbbm{1}_d = \mathbbm{1}_{Gd}$.\\
 \linebreak
We must now check that $G \left( d' d \right) 
= G(d') G(d)$ if $d  \colon a \to b$ and $d'  \colon b \to c$. But
\begin{equation*}
\begin{tikzcd}
    FG a \ar[d, "FG(d'd) \text{ or } F(G(d') G(d))"'] \ar[r, "\varepsilon_a"]
    & a \ar[d, "d'd"] \\
    FG c \ar[r, "\varepsilon_{c}"'] & c
\end{tikzcd}
\end{equation*}
so again by the same arguments, $G(d'd) = G(d') G(d)$. 
By full and faithfulness of  $F$, we may define the iso $\eta_{c} 
 \colon c \to GF c$ by specifying
 isos $F \eta_c  \colon Fc \to FGFc$. Define
 $F \eta_c$ to be $\varepsilon_{Fc}^{-1}$. For any $f  \colon c \to c'$, the
 outer triangle
 \begin{equation*}
 \begin{tikzcd}
     Fc \ar[d, "Ff"'] \ar[r, "F \eta_c"] & FGFc \ar[d, "FGF f"] \ar[r,
     "\varepsilon_{Fc}"] & Fc \ar[d, "Ff"] \\
     Fc' \ar[r, "F \eta_{c'}"'] & FGFc' \ar[r, "\varepsilon_{Fc'}"'] & Fc'
 \end{tikzcd}
 \end{equation*}
 
 commutes. The right-hand square commutes by naturality of $\varepsilon$. This
 implies that the left-hand square commutes. Faithfulness of $F$ tells us that
 $\eta_{c'} \cdot  f = GFf \cdot \eta_{c}$, so $\eta$ is a natural
 transformation.\\
 \linebreak
\textbf{Cor. } For any field $\mathbb{F}$, the categories $\Mat_{\mathbb{F}}$
and
$\Vect_{\mathbb{F}}^{\text{fd}}$ are equivalent.
We have
\begin{equation*}
\begin{tikzcd}
    \Mat_{\mathbb{F}} \ar[r, "\mathbb{F}^{(-)}"] & \ar[l, "H"]
    \Vect_{\mathbb{F}}^{\text{basis}} \ar[r, "U"] & \ar[l, "C"]
    \Vect_{\mathbb{F}}^{\text{fd}} 
\end{tikzcd}
\end{equation*}
Here $U$ is the forgetful functor, $\mathbb{F}^{(-)}$ is the functor sending
$n$ to the vector space $\mathbb{F}^{n}$, equipped with the standard basis. The
functor $H$ carries a vector space to its dimension and a linear map
$\varphi  \colon V \to W$ to the matrix expressing the action of $\varphi$ on
the chosen basis of $V$ with respect to the chosen basis of $W$.
The functor $C$ is defined by choosing a basis for each vector space.\\
\linebreak
Now,
$\mathbb{F}^{(-)}$ is essentially surjective on objects since given a vector
space $V$ with a basis $\mathcal{B}$, it is isomorphic to $\mathbb{F}^{m}$ for
some $m$ and hence
$(V, \mathcal{B}) \cong F(m)$.\\
It is full, since for any $n,m \in \mathbb{Z}_{+}$, we have
that if $\varphi  \colon \mathbb{F}^{n} \to \mathbb{F}^{m}$ is linear, there
exists a map $\varphi'  \colon n \to m$ in $\Mat_{\mathbb{F}}$ such that
$\mathbb{F}^{(-)}(\varphi') = \varphi$ by definition.\\
It is faithful, since a linear map is uniquely determined on the basis. Thus
$\Mat_{\mathbb{F}}$ is equivalent to $\Vect_{\mathbb{F}}^{\text{basis}}$.\\
Similarly, it is clear that $U$ is essentially surjective, since any
finite-dimensional vector space has a basis. It is fully faithful since
any linear map between vector spaces is uniquely determined by its values on
the basis.\\
Hence $\Vect_{\mathbb{F}}^{\text{basis}}$ is equivalent to
$\Vect_{\mathbb{F}}^{\text{fd}}$.\\
\linebreak
\textbf{Def.} A category is \textbf{connected} if any pair of objects can be
connected by a finite zig-zag of morphisms.\\
\linebreak
\textbf{Def.} The automorphism group of an object $c \in C$ is 
$\Hom(c,c)$ in $C$.\\
\linebreak

\textbf{Prop.} Any connected groupoid is equivalent, as a category, to the
automorphism group of any of its objects.\\
\linebreak
\textit{Proof:} Choose an object $c \in C$ where $C$ is a connected groupoid.
Let
$\Aut(c) = \Hom(c,c)$ be the automorphism group represented by a single-object
category. Then  define a functor
$F  \colon \Aut(c) \to C$ by mapping the object to $c$ and any morphism in
$\Aut(c)$ to the corresponding morphism in $C$. This is clearly fully faithful
as it is bijective, and it is essentially surjective on objects since $C$ is
connected.\\
\linebreak
\textbf{Cor.} In a path-connected space $X$, any choice of basepoint $x \in X$
yields an isomorphic fundamental group $\pi_1 (X,x)$.\\
\linebreak
\textit{Proof:} Any space $X$ has a fundamental groupoid $\Pi_1 (X)$ whose
objects are points in $X$ and whose morphisms are endpoint-preserving homotopy
classes of paths in $X$. Picking any point $x$, the group of automorphisms of
the object $x \in \Pi_1 (X)$ is exactly the fundamental group $\pi_1 (X,x)$.
Now the previous proposition implies that any pair of automorphism groups are
equivalent, as categories, to the fundamental groupoid
\[
\pi_1 (X, x) \hookrightarrow \Pi_1(X) \hookleftarrow \pi_1 (X,x')
\] 
and thus to each other. An equivalence between $1$-object categories is an
isomorphism of categories.\\
\linebreak


\textbf{Recall:} an \textbf{isomorphism of categories} is given by a pair of
inverse functors $F  \colon C \to D$ and $G  \colon D \to C$ such that
$FG = \mathbbm{1}_{D}$ and $GF = \mathbbm{1}_{C}$, i.e., the identity functors
of $D$ and $C$, respectively. An iso induces a bijection between the objects of
$C$ and $D$ and likewise for morphisms.\\
\linebreak

And since isomorphisms of groups regarded as $1$-object categories is exactly
isomorphisms of groups in the usual sense, we get that all of the fundamental
groups defined by choosing a basepoint in a path-connected space are
isomorphic.\\
\linebreak

\textbf{Def.} A category $C$ is \textbf{skeletal} if it contains just one
object in each isomorphism class. The \textbf{skeleton} 
$\sk C$ of a category $C$ is the unique (up to isomorphism) skeletal category
that is equivalent to $C$.\\
\linebreak
An equivalence between skeletal categories is necessarily and isomorphism of
categories. Suppose that $\sk C $ is isomorphic to $\sk D$ by $F  \colon
\sk C \to \sk D$ and $G  \colon \sk D \to \sk C$.\\
Since the inclusion  $\iota_C  \colon \sk C \to C$ and
$\iota_D  \colon \sk D \to D$ define functors that are fully faithful and essentially surjective on
object, they are equivalences of categories, so
$C \cong \sk C \cong \sk D \cong D$ and hence $C$ and $D$ are equivalent.\\
Thus, two categories are equivalence if and only if their skeletons are
isomorphic - so in particular, a category is always equivalent to its skeleton.\\
\linebreak
\textbf{Example.} The skeleton of a connected groupoid is the group of
automorphisms of any of its objects.\\
\textbf{Example.} The skeleton of $\Vect_{\Bbbk}^{\text{fd}}$ is the
category $\Mat_{\Bbbk}$.\\
\textbf{Def.} $\Fin$ is the category of finite sets where morphisms are
functions between sets. $\Fin_{\text{iso}}$ is the maximal subgroupoid of
$\Fin$.\\
\linebreak
\textbf{Example.} The skeleton of $\Fin_{\text{iso}}$ is the category whose
objects are positive integers and with $\Hom (n,n) = \Sigma_n$, the group of
permutations of $n$ elements. The hom-sets between distinct natural numbers are
all empty.\\
\linebreak
\textbf{Example. (a categorification of sum orbit satiblizers)} Let $X  \colon
BG \to \Set$ be a left $G$-set. Its \textbf{translation groupoid} 
$T_G X$ has elements of $X$ as objects. A morphism $g  \colon x \to y$ is an
element $g \in G$ so that $g \cdot x = y$. The objects in the skeleton
$\sk T_G X$ are the connected components in the translation groupoid. These are
precisely the \textbf{orbit} of the group action, which partition $X$ in
precisely this manner.\\
\linebreak
Consider $x \in X$ as a representative of its orbit $O_x$. Because the
translation groupoid is equivalent to its skeleton, we must have
\[
\Hom_{\sk T_G X} \left( O_x, O_x \right) \cong
\Hom_{T_G X} (x,x) =: G_x
\] 
the set of automorphisms of $x$. So $\Hom_{T_G X}(x,x)$ is the
\textbf{stabilizer} $G_x$ of $x$ wrt. the $G$-action.\\
This means that any pair of elements in the same orbit must have isomorphic
stabilizers. In summary, the skeleton of the translation groupoid, as
a category, is the disjoint union of the stabilizer groups, indexed by the
orbits of the action of $G$ on $X$.\\
Thus $\sk T_G X = \bigcup_{O_x} G_x$ as a category.
The set of morphisms in the translation groupoid with domain $x$ is isomorphic
to $G$. This set may be expressed as a disjoint union of hom-sets
$\bigcup_{y \in O_x} \Hom_{T_GX}(x,y)$. Now, for any
$g \in \Hom_{T_G X}(x,x)$, choose any $h  \colon x \to y$. Then
$\Hom_{T_G X}(x,y) = h \Hom_{T_GX}(x,x)$. The $\supset$ is clear, and
for any $j  \colon x \to y$, $j = h h^{-1} j \in h \Hom_{T_G X}(x,x)$, so
$\subset $ is shown. Now, we claim that the map
$g \mapsto hg$ is injective. Suppose $hg = hg'$. Then
$g = g'$ by multiplying by $h^{-1}$ on both sides. Hence
$\left| h G_x \right| = \left| G_x \right| $. Thus
\[
\left| G \right| = 
\left| \bigcup_{y \in O_x}  \Hom_{T_G X}(x,y) \right| 
= \sum_{y \in O_x} \left| \Hom_{T_GX}(x,y) \right| 
= \sum_{y \in O_x} \left| G_x \right| \\
= \left| O_x \right| \left| G_x \right| .
\] 
We define some concepts up to equivalence. We say a category is
\textbf{essentially small} if it is equivalent to a small category, or,
equivalently, if its skeleton is small. A category is \textbf{essentially
discrete} if it is equivalent to a discrete category.\\
\linebreak
The following are invariant under equivalence:
\begin{itemize}
    \item Smallness of categories
    \item Being a groupoid - i.e., any category equivalent to a groupoid is
        a groupoid.
    \item If $C \simeq D$ then $C^{op} \simeq D^{op}$.
    \item If $C \simeq D$ and $A \simeq B$ then
        $C \times A \simeq D \times B$.
    \item An arrow in $C$ is an isomorphism iff its image under an equivalence
        $C \stackrel{\simeq}{\to } D$ is an isomorphism.
\end{itemize}
 


\subsection*{Section 1.5 - Problems}
\textbf{ii.} Segal defined a category $\Gamma$ as follows:
\begin{center}
    $\Gamma$ is the category whose objects are all finite sets, and whose
    morphisms from $S$ to $T$ are maps $\theta  \colon S \to P(T)$ such that
    $\theta (\alpha)$ and $\theta (\beta)$ are disjoint when
    $\alpha \neq \beta$. The composite of $\theta  \colon S \to P(T)$ and
    $\varphi  \colon T \to P(U)$ is $\psi  \colon S \to P(U)$ where
    $\psi (\alpha) = \bigcup_{\beta \in \theta(\alpha)} \varphi(\beta)$.
\end{center}
Prove that $\Gamma$ is equivalent to the opposite of the category $\Fin_*$ of
finite pointed sets. In particular, the functors introduced in Example
1.3.2.(xi) defined presheaves on $\Gamma$.\\
\linebreak
\textit{Solution:} Can we define a functor 
$F  \colon \Gamma \to \Fin_*^{op}$ that is fully faithful and essentially
surjective on objects?\\
\linebreak

\textbf{iv.} Show taht fully faithful functors $F  \colon C \to D$ both
\textbf{reflect} and \textbf{create isomorphisms}. I.e., show that
\begin{enumerate}
    \item If $f$ is a morphism in $C$ such that $Ff$ is a morphism in $D$, then
        $f$ is an isomorphism.
    \item If $x$ and $y$ are object in $C$ so that $Fx \cong Fy$ in $D$, then
        $x \cong y$ in $C$.
\end{enumerate}
\textit{Solution:} Suppose $f  \colon c \to d$ in $C$ so that
$Ff  \colon Fc \to Fd$ is an iso in $D$. Then there exists
$g  \colon Fd \to Fc$ such that $Ff g = \mathbbm{1}_{Fd}$ and $g Ff
= \mathbbm{1}_{Fc}$. But since $F$ is full, the map
$F_*  \colon \Hom(d,c) \to \Hom(Fd, Fc) \ni g$ is surjective, so $\exists g'
\in 
\Hom(d,c)$ such taht $Fg' = g$. Then
$F \left( g' f \right) = \mathbbm{1}_{Fc}$ and
$F \left( f g' \right) = \mathbbm{1}_{Fd}$, so
since $F$ is faithful, $g'f = \mathbbm{1}_{c}$ and
$fg' = \mathbbm{1}_{d}$, hence $g'$ is the inverse of $f$.\\
\linebreak
For (ii), suppose $Fx \cong Fy$ with $d  \colon Fx \to Fy$ and $e  \colon Fy
\to Fx$ being the isos. Since $F$ is full, there exist
$d'  \colon x\to y$ and $e'  \colon y \to x$ such that
$Fd' = d, Fe' = e$. Now
$\mathbbm{1}_{Fy} = de = Fd' Fe' = F\left( d' e' \right)$, and
$\mathbbm{1}_{Fx} = ed = F\left( e' d' \right) $, so since
$F$ is faithful, $e'd' = \mathbbm{1}_x$ and
$d'e' = \mathbbm{1}_y$.\\
\linebreak
\textbf{v.} Find an example to show that a faithful functor need not reflect
isomorphisms.\\
\linebreak
\textit{Solution:} Suppose we let $C = \mathbbm{1}$ and
$\ob D = \left\{ a,b \right\} $ and $\Hom D = \left\{ a\to b, b\to a,
\mathbbm{1}_a, \mathbbm{1}_b \right\} $. Let
$F 0 = a, F 1 = b$ and $F \left( 0 \to 1 \right) = a\to b$. Then $F $ does not
reflect $a\to b$ even though it is an isomorphism.\\
\linebreak
\textbf{vii.} Let $G$ be a connected groupoid and let $\Aut G$ be the group of
automorphisms at any of its objects. The inclusion $B\Aut G \hookrightarrow G$
defines an equivalence of categories. Construct an inverse equivalence $G \to
B \Aut G$.\\
\linebreak
\textit{Solution:} The inclusion $B\Aut G \hookrightarrow G$ defines an
equivalence. Suppose
$F \iota \simeq \mathbbm{1}_{B \Aut G}$ and
$\iota F \simeq \mathbbm{1}_{G}$, so
\begin{equation*}
\begin{tikzcd}
    F \iota  * \ar[d, "F \iota g"] \ar[r] & * \ar[d, "g"]\\
    F \iota * \ar[r] & *
\end{tikzcd}
\end{equation*}

\begin{equation*}
\begin{tikzcd}
    \iota F x \ar[d, "\iota F f"] \ar[r] & x \ar[d, "f"] \\
    \iota F y \ar[r] & y
\end{tikzcd}
\end{equation*}
Letting $\varepsilon  \colon \iota F \implies \mathbbm{1}_{G}$, we get that
$\iota F (f) = \varepsilon_y^{-1} \cdot f \cdot  \varepsilon_x$, so since the
inclusion is faithful, there is precisely one morphism 
 $\iota^{-1} \left( \varepsilon_y^{-1} \cdot f \cdot \varepsilon_x \right)
 $ for $F(f)$.
 Then
 \begin{equation*}
 \begin{tikzcd}
     * \ar[d, "\iota^{-1} \left( \varepsilon_y^{-1} g \varepsilon_x (\iota g) \right)
     "'] \ar[r] & * \ar[d, "g"]\\
     * \ar[r] & *
 \end{tikzcd}
 \end{equation*}
 

\textbf{ix:} Show that any category that is equivalent to a locally small
category is locally small.\\
\linebreak
\textit{Proof:} Suppose $ C \simeq D$ with $F  \colon C\to D$ and $G  \colon
D \to C$ and $FG \simeq \mathbbm{1}_{D}, GF \simeq \mathbbm{1}_{C}$ and $C$ is locally
small. Then let $x',y' \in D$. By essential surjectivity on objects, $\exists 
x,y \in C$ such that $Fx \cong x', Fy \cong y'$. Claim:
$\left| \Hom(x',y') \right| \le  \left| \Hom(x,y) \right| $.
Let $f  \colon x' \to y'$. Let $\varepsilon  \colon Fx \to x',
\eta  \colon Fy \to y'$ be isomorphisms. Then
$\eta^{-1} \cdot f \cdot \varepsilon  \colon Fx \to Fy$, som we claim
$L = \eta^{-1}_* \circ \varepsilon^{*}$ is injective.\\
If $L(f) = L(g)$ then
$\eta^{-1} f \varepsilon = \eta^{-1} g \varepsilon$ so by isomorphism,
$f = g$. Thus
 $L  \colon \Hom(x',y') \to \Hom(Fx,Fy)$ is injective, so
 $\left| \Hom(x',y') \right| \le \left| \Hom(Fx, Fy) \right| 
 = \left| \Hom(x,y) \right| $ where the last equality follows as $F$ is fully
 faithful.\\
 \linebreak
 \textbf{x.} Characterize the categories that are equivalent to discrete
 categories. A category that is connected and essentially discrete is called
 \textbf{chaotic}.\\
 \linebreak
 \textit{Solution:} Suppose $C$ is a category that is equivalent to a discrete
 category $D$. Then there exists a fully faithful and essentially surjective on
 objects functor $F  \colon C \to D$. Now, for any $x,y \in C$, we have
 $\left| \Hom(x,y) \right| = \left| \Hom(Fx, Fy) \right| 
 = \delta_{Fx,Fy}$ since a discrete category only has identity morphisms.
 Clearly then $\left| \Hom(x,y) \right| =
 \left| \Hom(y,x) \right| $ and $F$ is faithful and
 if $f  \colon x\to y, g  \colon y \to x$ then
 $F(gf) \in \Hom(Fx,Fx) = \left\{ \mathbbm{1}_{Fx} \right\} $, so
 $gf = \mathbbm{1}_{x}$ and $fg = \mathbbm{1}_{y}$, so $x \cong y$. Thus,
 $C$ is a groupoid, and since any groupoid has a discrete skeleton, we find
 that the categories equivalent to discrete categories are precisely all
 groupoids.\\
 \linebreak
 Therefore if a category is a connected groupoid, it is chaotic - this simply
 means that $\left| \Hom(x,y) \right| =1$ for all $x,y \in C$ and
 that each morphism is an isomorphism. 




 \subsection*{Section 1.6}

\textbf{Def.} A \textbf{diagram} in a category $C$ is a functor $F  \colon
J \to C$ whose domain is referred to as the \textbf{indexing category} of the
diagram.\\
\linebreak
Formally, a diagram is just a functor, but in practice a functor is referred to
as a diagram when the indexing category is smaller than the target category.
There are, however, instances, such as lemma 3.7.1, where it is convenient to
consider arbitrary functors as diagrams.\\
\linebreak
Functors preserve commutative diagrams.\\
\linebreak
\textbf{Lemma.} Suppose $f_1, \ldots, f_n$ is a composable sequence - a "path"
- of morphisms in a category. If the composite $f_k f_{k-1} \ldots f_{i+1} f_i$
equals $g_m \ldots g_1$ for another composable sequence of morphisms $g_1,
\ldots, g_m$ then $f_n \ldots f_1 = f_{n} \ldots f_{k+1} g_m \ldots g_1 f_{i-1}
\ldots f_1$.\\
\linebreak
Lemma 1.6.11 and transitivity of equality imply that commutativity of an entire
diagram may be checked by establishing commutativity of each minimal subdiagram
in the directed graph.\\
\linebreak
\textbf{Lemma.} For any commutative square $\beta \alpha = \delta \gamma$ in
which each of the morphisms is an isomorphism, the inverse define a commutative
square $\alpha^{-1} \beta^{-1} = \gamma^{-1} \delta^{-1}$.\\
\linebreak
\textbf{Def.} An object $i \in C$ is \textbf{initial} if for every $c \in C$
there
exists a unique morphism $i \to c$. Dually, an object $t \in C$ is
\textbf{terminal} if for every $c \in C$ there exists a unique morphism
$c \to t$.\\
\linebreak
\textbf{Examples.}\\
\begin{enumerate}
    \item $\varnothing \in \Set$ is initial, since for any set $A \in \Set$,
        there exists precisely one subset of $\varnothing \times A$, namely the
        empty set. Thus the empty map is the only morphism from  $\varnothing$ 
        to any set in $\Set$.
    \item In $\Top$, the empty and singleton spaces are, respectively, initial
        and terminal.
\end{enumerate}

\textbf{Def.} A \textbf{concrete category} is a category $C$ equipped with
a faithful functor $U  \colon C \to \Set$\\
\linebreak
Because faithful functors reflect identifications between parallel morphisms:\\
\textbf{Lemma.} If $U  \colon C \to D$ is faithful, then any diagram in $C$
whose image commutes in $D$ also commutes in $C$.\\
\linebreak
This implies that a diagram in a concrete category commutes if and only if the
induces diagram of underlying sets commutes.\\
\linebreak




\textbf{Lemma} Consider morphisms with the indicated sources and targets
\begin{equation*}
\begin{tikzcd}
    a \ar[r,"f"] \ar[d, "g"] & b \ar[d, "h"] \ar[r, "j"] & c \ar[d, "l"]\\
    a' \ar[r, "k"] & b' \ar[r, "m"] & c'
\end{tikzcd}
\end{equation*}
and suppose that the outer rectangle commutes. This data defines a commutative
rectangle if either:
\begin{enumerate}
    \item the right-hand square commutes and $m$ is a monomorphism; or
    \item the left-hand square commutes and $f$ is an epimorphism.
\end{enumerate}


\section*{Section 1.6 - Problems}

\textbf{i.} Show that any map from a terminal obj in a category to an initial
one is an iso. An object that is both initial and terminal is called
a \textbf{zero object}.\\
\linebreak
\textit{Solution:}
Suppose $f  \colon t \to i$ where $t$ is terminal and $i$ is initial. By
definition, there exists precisely one morphism
$g  \colon i \to t$. Now, $fg \in \Hom (i,i) =
\left\{ \mathbbm{1}_i \right\} $ and
$gf \in \Hom(t,t) = \left\{ \mathbbm{1}_t \right\} $, so
$f$ is an isomorphism.\\
\linebreak
\textbf{iii} Show that any faithful functor reflects monomorphisms.\\
\linebreak
\textit{Solution:} Suppose $F  \colon C \to D$ is faithful and
$F f  \colon Fc \to Fd$ is a monomorphism in $D$. If $f$ is not a monomorphism,
there exist $h,k  \colon d \to e$ such that
$hf = kf$ yet $h \neq k$. But then
$F(h) F(f) = F(k) F(f)$ so since $F(f)$ is monic, we have
$F(h) = F(k)$, but $F(h), F(k) \in \Hom(Fd, Fe)$ and since
$F$ is faithful, i.e., $F_*$ is injective, so $h = k$.\\
\linebreak
Now, if we flip all arrow in the above proof, $f$ becomes a map
$d \to c$ and $h,k  \colon e \to d$, so $fh = fk$. 
Anywhere where a property of $F f$ being a monomorphism was used would become
the property that $Ff$ is an epimorphism - since the graphs representing
monomorphisms and epimorphisms are duals - see terminology section.\\
\linebreak









\section*{Terminology}

\textbf{Def.} A morphism $f  \colon x \to y$ in a category is
\begin{enumerate}
    \item a \textbf{monomorphism} if for any parallel morphism $h, k  \colon
        w \to k$, $fh = fk$ implies $h = k$ 
    \item an \textbf{epimorphism} if for any parallel morphism $h,k
         \colon y \to z$, $hf = kf$ implies that $h = k$.
\end{enumerate}
So
\begin{center}
\begin{tikzcd}
    \bullet \ar[dr, "h"] & &\\
                         & \bullet \ar[r, "f"] & \bullet\\
    \bullet \ar[ru, "k"] & &
\end{tikzcd}
\begin{tikzcd}
    & & \bullet\\
    \bullet \ar[r, "f"] & \bullet \ar[ru, "h"] \ar[rd, "k"] & \\
                        & & \bullet
\end{tikzcd}
\end{center}
where the left shows $f$ as a monomorphism and right one shows $f$ as an
epimorphism.\\
\linebreak
\textbf{Def.} The category $\Top^2$ consists of objects being ordered pairs
$(X,A)$ where $X$ is a topological space and $A$ is a subspace of $X$.\\
A morphism $f  \colon (X,A) \to (Y,B)$ is an ordered pair
$(f,f')$ where $f \colon X \to Y$ is continuous and
$f i = j f'$ where $i$ and $j$ are inclusions,
\begin{equation*}
\begin{tikzcd}
    A \ar[d, "f'"'] \ar[r, "i", hookrightarrow] & X \ar[d, "f"]\\
    B \ar[r, "j"', hookrightarrow] & Y
\end{tikzcd}
\end{equation*}
and composition is coordinatewise. $\Top^2$ is called the category of pairs of
topological spaces.\\
$\Top_*$ is a subcategory of $\Top^2$.\\
\linebreak
\textbf{Rotman 0.12.} Given a category $C$, show that the following construction
gives a category $M$. First, an object of $M$ is a morphism of $C$. Next, if
$f,g \in \ob M$, say, $f  \colon A \to B$ and
$g  \colon C \to D$, then a morphism in $M$ is an ordered pair
$(h,k)$ of morphisms in $C$ such that the diagram
\begin{equation*}
\begin{tikzcd}
    A \ar[r, "f"] \ar[d, "h"] & B \ar[d,"k"]\\
    C \ar[r, "g"] & D
\end{tikzcd}
\end{equation*}
commutes. Define composition coordinatewise:
\[
    (h',k') \circ (h,k) = (h' \circ h, k' \circ k).
\] 

\textit{Solution:} Suppose we have an object
$f  \colon A \to B \in \ob M$. Then since the following commutes:
\begin{equation*}
\begin{tikzcd}
    A \ar[d, "\mathbbm{1}_A"] \ar[r, "f"] & B \ar[d, "\mathbbm{1}_B"]\\
    A \ar[r, "f"] & B
\end{tikzcd}
\end{equation*}
we have that $(\mathbbm{1}_{A}, \mathbbm{1}_B)
 \colon f \to f$.\\
 This is an identity for $f$ since for any
 map $(h,k)  \colon f \to g$, where $g  \colon C \to D$, we have
 \begin{equation*}
 \begin{tikzcd}
     A \ar[dd, bend right, dashed, "h \circ \mathbbm{1}_A"'] \ar[d, "\mathbbm{1}_A"] \ar[r, "f"]
     & B \ar[d, "\mathbbm{1}_B"'] \ar[dd, bend left, dashed,
     " k \circ \mathbbm{1}_B"]\\
    A \ar[d, "h"] \ar[r, "f"] & B \ar[d, "k"']\\
    C \ar[r, "g"] & D
 \end{tikzcd}
 \end{equation*}
 gives that the outer rectangle commutes, so
 $(h,k) \circ (\mathbbm{1}_{A}, \mathbbm{1}_{B}) =
 (h,k)$ and similarly for the other order.\\
 \linebreak
 Associativity follows from associativity in $C$.\\
 Composition is also well-defined from composition in $C$.\\
 \linebreak
 \textbf{Def. (Rotman)} A \textbf{congruence} on a category
 $C$ is an equivalence relation $\sim$ on the class
 $\bigcup_{(A,B)} \Hom(A,B)$ of all morphisms in $C$ such that:
 \begin{enumerate}
     \item $f \in \Hom(A,B)$ and $f \sim f'$ implies
         $f' \in \Hom(A,B);$
     \item $f \sim f', g \sim g'$, and the composite $g \circ f$ exists imply
         that
         \[
         g \circ f \sim g' \circ f'.
         \] 
 \end{enumerate}

 \textbf{Theorem 0.4 (Quotient category)} Let $C$ be a category with congruence
 $\sim$, and let $\left[ f \right] $ denote the equivalence class of a morphism
 $f$. Define $C'$ as follows:
 \begin{align*}
     \ob C' &= \ob C\\
     \Hom_{C'}(A,B) &= \left\{ \left[ f \right]  \colon
     f \in \Hom_{C} (A,B) \right\} \\
     \left[ g \right] \circ \left[ f \right] 
                    &= \left[ g \circ f \right].
 \end{align*}
 Then $C'$ is a category.\\
 \linebreak
 \textit{Proof:} We must check that composition is associative:
$\left[ g \right] \circ \left( \left[ f \right] \circ \left[ h \right] 
\right) = \left[ g \right] \circ \left[ f \circ h \right] 
\left[ g \circ \left( f \circ h \right)  \right]
\left[ \left( g \circ f \right) \circ h \right] 
= \left[ g \circ f \right] \left[ h \right] 
= \left( \left[ g \right] \left[ f \right]  \right) \left[ h \right] $, where
associativity inside is inherited from $C$.\\
\linebreak
We must check that this is well defined.\\
If $\left[ g \right] = \left[ g' \right] $ and
$\left[ f \right] = \left[ f' \right] $, then
$g,g'$ have the same domain and codomain and likewise for $f$ and
$f'$.\\
Now $g \circ f \sim g' \circ f'$ by assumption on congruence, so
$\left[ g \circ f \right] = \left[ g' \circ f' \right] $.\\
So it is well-defined.\\
\linebreak
For identity, choose any object $A \in C'$, then
$\left[ \mathbbm{1}_A \right] $ is an identity since
if  $f  \colon A \to B$ then
$\left[ f \right] \left[ \mathbbm{1}_A \right] 
= \left[ f \circ \mathbbm{1}_A \right] 
= \left[ f \right] $ and
if $g  \colon C \to A$ then
$\left[ \mathbbm{1}_A \right] \left[ g \right] 
= \left[ \mathbbm{1}_A \circ g \right] 
\left[ g \right] $.\\
\linebreak
The category $C'$ just constructed is called a
\textbf{quotient category} of $C$ ; one usually denotes
$\Hom_{C'} (A,B)$ by $\left[ A,B \right] $.\\
\linebreak
The most important quotient category for us is the \textbf{homotopy
category}.\\
We will see it later, but for a lesser example:\\
\linebreak
Let $C = \Group$ and $f,f' \in \Hom(G,H)$. Define
$f \sim f'$ if there exists $a \in H$ such that
$f(x) = a f'(x) a^{-1}$ for all $x \in G$.\\
\linebreak
Equivalence: $f \sim f$ is clear by $a = e$.\\
$f \sim f'$ then $f(x) = a f'(x) a^{-1}$ so
$f'(x) = a^{-1} f(x) \left( a^{-1} \right)^{-1}$, so
$f' \sim f$. If
$f \sim f' \sim f''$ then
$f(x) = a f'(x) a^{-1} = 
a b f''(x) b^{-1} a^{-1}
= (ab) f''(x) (ab)^{-1}$, so
$f \sim f''$.\\
\linebreak
To see congruence, 
if $f \in \Hom(G,H)$ and
$f \sim f'$ then
$f'(x) = a f(x) a^{-1}$. Suppose
$g, g' \in G$. Then
$f'(g g') = a f( g g') a^{-1}
= a f(g) a^{-1} a f(g') a^{-1}
= f'(g) f'(g')$, so
$f' \in \Hom(G,H)$. If
$f \sim f', g \sim g'$ and  $g \circ f$ is well defined, so say,
$f  \colon G \to H, g  \colon H \to J$. Then
$f'(x) = af(x) a^{-1}, g'(x) = bg(x)b^{-1}$, so
$g' \circ f' (x)
= b g(f'(x)) b^{-1}
= b g \left( a f(x) a^{-1} \right) b^{-1}
= b g(a) g(f(x)) g(a)^{-1} b^{-1}
= (bg(a)) (g \circ f)(x) (bg(a))^{-1}$ and
$b g(a) \in J$, so
$\left[ g' \circ f' \right] \sim \left[ g \circ f \right] $.\\
\linebreak
Thus the quotient category is defined. If
$G$ and $H$ are groups, then $\left[ G , H \right] $ is the set of all 
conjugacy classes $\left[ f \right] $ where
$f  \colon G \to H$ is a homomorphism.\\
\linebreak
\textbf{Rotman 0.17.} Let $C$ and $A$ be categories, and let
$\sim$ be a congruence on $C$. If
$T  \colon C \to A$ is a functor with $T(f) = T(g)$ whenever
$f \sim g$, then $T$ defines a functor $T'  \colon C' \to A$ 
(where $C'$ is the quotient category) by
$T'(X) = T(X)$ for every object $X$ and
$T'\left( \left[ f \right]  \right) 
= T(f)$ for every morphism $f$.\\
\linebreak
\textit{Proof:} We must show that $T'$ is a functor.\\
Since $T'$ matches $X$ on objects, we have
$T'(X) \in \ob A$.\\
Now, if $f  \colon X \to Y$ then
$T'\left( \left[ f \right]  \right) = 
T \left( f \right)   \colon TX (T'X) \to TY (T'Y)$.\\
\linebreak
Suppose $f  \colon x \to y$ and
$g  \colon y \to z$ then
$T' \left( \left[ f \right] \left[ g \right]  \right) 
= T'\left( \left[ fg \right]  \right) 
= T(fg) = T(f) T(g)
= T'\left( \left[ f \right]  \right) 
T'\left( \left[ g \right]  \right) $.\\
And 
$T' \left( \left[ \mathbbm{1}_A \right]  \right) 
= T(\mathbbm{1}_A) = 
\mathbbm{1}_{T(A)}
= \mathbbm{1}_{T'(A)}$.\\
\linebreak
Hence $T'$ is a functor.\\
\linebreak
\textbf{Rotman 0.20.}\\
(i) If $X$ is a topological space, show that $C(X)$, the set of all continuous
real-valued functions on $X$, is a commutative ring with $1$ under pointwise
operations:
\[
f+g   \colon x \mapsto f(x) + g(x) \quad \text{and} \quad 
f \cdot g  \colon x \mapsto f(x) g(x)
\] 
for all $x \in X$.\\
(ii) Show that $X \mapsto C(X)$ gives a contravariant functor
$\Top \to \Ring$.\\
\linebreak
\textit{Solution:} 
(i) Trivial.\\
(ii) Let $F  \colon \Top \to \Ring$ by
$F(X) = C(X)$. We let $F$ act on morphisms
as follows: suppose
$f  \colon X \to Y$ is a continuous map, then
$F(f)  \colon F(Y) \to F(X) = C(Y) \to C(X)$ by
$F(f) = f^*$.\\
\linebreak
This is a functor since
if $f  \colon X \to Y$ and $g  \colon Y \to Z$ then
$F(gf) = (gf)^* = f^* g^* = F(f) F(g)
$ 
and $F(\mathbbm{1}_X) = 
\mathbbm{1}_{X}^{*} =
\mathbbm{1}_{C(X)}= \mathbbm{1}_{F(X)}$ because
for any $g  \colon X \to \mathbb{R} \in C(X)$ we have
$\mathbbm{1}_{X}^{*}(g) = 
g \circ \mathbbm{1}_X
= g$, so $\mathbbm{1}_{X}^{*}= \mathbbm{1}_{C(X)}$.\\
\linebreak
It is easy to show that 
homotopy is a congruence on the category $\Top$, so
it follows from theorem 0.4 (Quotient category) that there is a quotient
category whose objects are topological spaces $X$ and whose $\Hom$ sets
are $\Hom (X,Y) = \left[ X,Y \right] $ and whose composition is 
$\left[ g \right] \circ \left[ f \right] 
= \left[ g \circ f \right] $ where 
$\left[ X,Y \right] $ denotes the family of all homotopy classes 
\[
\left[ f \right] = \left\{ \text{continuous }g
 \colon X \to Y  \colon g \sim f\right\} 
\] 
of continuous maps $X \to Y$.\\
\linebreak
\textbf{Def.} The quotient category just described is called the
\textbf{homotopy category}, and it is denoted by
$\hTop$

\subsubsection*{A few philosophical remarks}
One might expect that the functor $C  \colon \Top \to \Ring$ of
exercise 0.20 is as valuable as the homology functors. Indeed, a
theorem of Gelfand and Kolmogoroff states that for $X,Y$ compact Hausdorff,
$C(X)$ and $C(Y)$ isomorphic as rings implies that
$X$ and $Y$ are homeomorphic. However, a less accurate translation of a problem
from topology to algebra is usually more interesting than a very accurate one
- i.e., often a loss of information is valuable. The functor $C$ is not as
useful as other functors precisely because of the theorem of Gelfand and
Kolmogoroff: the translated problem has no loss and is exactly as complicated
as the original one and hence cannot be any easier to solve (one can hope only
that the change in viewpoint is helpful).\\
\linebreak
Now, all the functors $T  \colon \Top \to A$ that we shall construct, where $A$
is some "algebraic" category, will have the property that
$f \sim g$ implies $T(f) = T(g)$ - thus there is a loss of information, we
collapse to the equivalence classes. This fact, aside from a natural wish to
identify homotopic maps, makes homotopy valuable because it guarantees that the
algebraic problem in $A$ arising from a topological problem
via $T$ is simpler than the original problem. Furthermore, exercise
0.17 shows that every such functor gives a functor
$\hTop \to A$, and so the homotopy category is quite fundamental.\\
\linebreak
What are the isomorphisms in $\hTop$? (Rotman uses equivalence for isomorphism
in categories).\\
\linebreak
Suppose we have $X,Y \in \ob \hTop$ and
morphisms $\left[ f \right] \in \left[ X,Y \right] ,
\left[ g \right] \in \left[ Y,X \right] $ such that
$\left[ f \right] \left[ g \right] = \left[ \mathbbm{1}_{Y} \right] ,
\left[ g \right] \left[ f \right] = \left[ \mathbbm{1}_{X} \right] $.
Thus $\left[ f \right] $ is an isomorphism and any 
isomorphism is of this form. So\\
\linebreak
\textbf{Def. (Homotopy equivalence - isomorphisms in $\hTop$)} 
A continuous map $f  \colon X \to Y$ is a homotopy equivalence
if there is a continuous map $g  \colon Y \to X$ with
$g \circ f \sim \mathbbm{1}_{X}$ and $f \circ g \sim \mathbbm{1}_{Y}$. Two
spaces $X$ and $Y$ have the \textbf{same homotopy type} if there is a homotopy
equivalence $f  \colon X \to Y$. That is, $X$ and $Y$ are isomorphic,
$X \cong Y$, if there
is a homotopy equivalence $f  \colon X \to Y$.\\
\linebreak








\end{document}
