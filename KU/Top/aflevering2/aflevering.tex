\documentclass[a4paper]{article}

\usepackage[margin=2.5cm]{geometry}
\usepackage[pdftex]{graphicx}
\usepackage[utf8]{inputenc}
\usepackage[T1]{fontenc}
\usepackage{textcomp}
\usepackage{babel}
\usepackage{amsmath, amssymb}
\usepackage[colorlinks=true,linkcolor=blue]{hyperref}
\usepackage{float}
\usepackage{mathrsfs}
\usepackage{enumitem}


\newcommand{\incfig}[2][1]{%
\def\svgwidth{#1\columnwidth}
\import{./figures/}{#2.pdf_tex}
}


% figure support
\usepackage{import}
\usepackage{xifthen}
\pdfminorversion=7
\usepackage{pdfpages}
\usepackage{transparent}

\pdfsuppresswarningpagegroup=1

\setlength\parindent{0pt}

\newcommand{\qed}{\tag*{$\blacksquare$}}
\newcommand{\qedwhite}{\hfill \ensuremath{\Box}}

%Inequalities
\newcommand{\cycsum}{\sum_{\mathrm{cyc}}}
\newcommand{\symsum}{\sum_{\mathrm{sym}}}
\newcommand{\cycprod}{\prod_{\mathrm{cyc}}}
\newcommand{\symprod}{\prod_{\mathrm{sym}}}

%Linear Algebra

%Redeclaring Span and image
\DeclareMathOperator{\Span}{span}
\DeclareMathOperator{\Ima}{Im}
\DeclareMathOperator{\diag}{diag}

%Row operations
\newcommand{\elem}[1]{% elementary operations
\xrightarrow{\substack{#1}}%
}

\newcommand{\lelem}[1]{% elementary operations (left alignment)
\xrightarrow{\begin{subarray}{l}#1\end{subarray}}%
}

%SS
\DeclareMathOperator{\supp}{supp}
\DeclareMathOperator{\Var}{Var}

%NT
\DeclareMathOperator{\ord}{ord}

\DeclareMathAlphabet{\pazocal}{OMS}{zplm}{m}{n}
\newcommand{\unif}{\pazocal{U}}


\title{Assignment 2}
\author{Jonas Trepiakas - hvn548@alumni.ku.dk}
\date{}

\begin{document}
\maketitle
\newpage
    \textbf{Homework 4:} Let $X$ and $Y$ be topological spaces and $f  \colon
    X \to  Y$ a continuous function. Let $Z := \left\{ (x, f(x))  \mid 
    x \in X \right\} \subseteq X \times Y$ be equipped with the subspace
    topology coming from the product topology on $X \times Y$. Show that $Z$ is
    homeomorphic to $X$.\\
    \linebreak
    \textit{Solution:} We check that lemma 6.6 is satisfied:\\
    Let $\varphi  \colon X \to Z$ by $\varphi (x) =
    (x, f(x))$. This is clearly a well
    defined bijection. Let $A \subseteq Z$ be open. If $A =\varnothing$, then
    $\varphi^{-1}(A) = \varnothing$ which is open in $X$ by definition of
    a topology, so assume $A$ is nonempty and let 
    $x \in \varphi^{-1} (A)$. Then $A = Z \cap (U \times V)$ where $U$ is open
    in $X$ and $V$ is open in $Y$ by definition 4.1 for the subspace topology
    and definition 4.6 for the product topology.
    Now, since $f$ is continuous and $x \times f(x) \in Z \cap \left( U \times
    V\right)$ implies that $f(x) \in V$, we have that
    $f^{-1}(V)$ is a neighborhood of $x$, so since the intersection of a finite
    collection of open sets is open by definition of a topological space,
    $S = U \cap f^{-1}(V)$ is a neighborhood
    of $x$. Now, if $x' \in S$ then $f(x') \in V$ and $x' \in U$, so
    $\varphi(x') = x' \times f(x') \in Z \cap (U \times V) = A$, thus
    $x \in S \subseteq \varphi^{-1}(A)$.\\
    Since $x$ was arbitrary, we can for any $x \in A$, find an open
    neighborhood of $x$, $S_x$, as above. Then we have
    \[
    A = \bigcup_{x \in A} \left\{ x \right\} 
    \subseteq \bigcup_{x \in A} S_x
    \stackrel{\forall x \in A \colon S_x \subseteq A}{\subseteq} A
    \] 
    Thus $A = \bigcup_{x \in A} S_x$ and by definition 2.1, $A$ is open.\\
    \linebreak
    Now, we check the second part of Lemma 6.6: if $U \subseteq X$ is open,
    then $\varphi(U)$ is open in $Z$.\\
    \linebreak
    Let $U \subseteq X$ be open (if $U =\varnothing$, $(\varphi^{\circ -1})^{-1}(U)
    = \varnothing$ which is open, so assume $U$ is nonempty)
    and let $x \times f(x) \in \varphi(U)$.
    Then $x \in U$ and hence $f(x) \in Y$, so
    $x \times f(x) \in \left( U \times Y \right) \cap Z$. We now claim
    $\left( U \times Y \right) \cap Z \subseteq \varphi(U)$. Let
    $x \times y \in \left( U \times Y \right) \cap Z$. Since
    $x \times y \in Z$, $f(x) = y$, and since $x \times y \in U \times Y$, we
    have $x \in U$, so $x \times y = x \times f(x) = \varphi(x) \in
    \varphi(U)$.\\
    Now, since $(U \times Y) \cap Z$ was chosen for any $x \in U$ independent
    of $x$, we have
    \[
    \varphi(U) = \bigcup_{x \in U} \left\{ x \times f(x) \right\} 
    \subseteq \bigcup_{x \in U} \left( U \times Y \right) \cap Z
    \subseteq \varphi(U)
    \] 
    hence $\varphi(U) = \left( U\times Y \right) \cap Z$.
    Since $U$ is open in $X$ and $Y$ is open in $Y$ by
    definition of a topology, we have that $\left( U \times Y \right) \cap Z$ 
    is open in $Z$ by definition of a subspace topology, hence
    $\varphi(U)$ is open in $Z$.\\
    Hence $\varphi$ is a homeomorphism by Lemma 6.6.\\
    \linebreak
    \textbf{Homework 5:} Let $X = \mathbb{R}$ be equipped with the topology
    \[
        \mathcal{T} = \{\varnothing\} \cup \left\{ U \subseteq X
         \mid \# (X \backslash U) < \infty \right\} 
    .\] 
    \begin{enumerate}[label=(\roman*)]
        \item Prove or disprove that $(X, \mathcal{T})$ is Hausdorff.
        \item Prove or disprove that $(X, \mathcal{T})$ is connected.
        \item Find the closure of the sets $A = [0,1] = \left\{ t \in \mathbb{R}
             \mid 0 \le t \le 1 \right\} , B = (0,1)
             = \left\{ t \in \mathbb{R} \mid 0 < t < 1 \right\} ,
             C = \left\{ 2,4,6,8 \right\} $ in $X$ with respect to
             $\mathcal{T}$.
         \item Suppose that $f \colon X \to X$ is a bijection. Prove that
             $f$ is a homeomorphism.
    \end{enumerate}
    \textit{Solution:}\\
    (i) We claim that $(X, \mathcal{T})$ is not Hausdorff: We show that
    condition (b) in theorem 5.19 is not satisfied: Let $x_n$ be the
    sequence $x_n = \frac{1}{n}$. Let $x$ be any point of $\mathbb{R}$. Then
    for any neighborhood $U$ of $x$, its complement is finite and can thus contain
    at most a finite subcollection of $\{x_n\}_{n \in \mathbb{N}}$. Let $N \in
    \mathbb{N}$ be the maximum of the
    indices in this subcollection, the for all $n> N$, we have
    that $x_n$ is not in $U^{c}$, so $x_n \in \left( U^{c} \right)^{c} = U$
    so since $U$ was arbitrary, $x_n$ converges to $x$ by definition 5.14, 
    so since $x_n$ converges to every point in $\mathbb{R}$, we get
    contraposition of
    theorem 5.19.(b) that $X$ is not Hausdorff.\\
    \linebreak
    (ii) If $X$ could be separated as $X = U \cup V$ where $U,V$ are open
    disjoint nonempty subsets of $X$, then $V = X \backslash U$, so
    $\# V = \# X \backslash U < \infty$ by definition of $V$ being open, hence
    $\# X \backslash V = \# X - \# V = \infty$, where the second equality
    follows from theorem 468 in the dismat book, so $V$ is not open. Thus no
    such separation exists, so $X$ is connected by definition 8.1.\\
    \linebreak
    (iii) We saw in $(i)$ that for any $x \in \mathbb{R}$, any neighborhood of
    $x$ intersects $\left\{ \frac{1}{n}  \mid n \in \mathbb{N} \right\}
    \subseteq (0,1) \subseteq [0,1]$, thus by proposition 5.9, 
    $\mathbb{R} \subseteq \overline{(0,1)}, \overline{[0,1]} \subseteq
    \mathbb{R}$ hence
    $\mathbb{R} = \overline{(0,1)}, \overline{[0,1]} $. Now, since
    $\# \mathbb{R} \backslash \left( \mathbb{R}\backslash \{2,4,6,8\} \right) 
    = \# \{2,4,6,8\} = 4 < \infty$, we find that $\mathbb{R} \backslash
    \{2,4,6,8\}$ is open, hence its complement 
    $\mathbb{R} \backslash \left( \mathbb{R} \backslash \{2,4,6,8\} \right) =
        \{2,4,6,8\}$ is closed by definition, so $\overline{\{2,4,6,8\}} = \{2,4,6,8\}$ 
        by corollary 5.13 and proposition 5.12.\\
        \linebreak
        (iv) Since $f$ is a bijection, $f^{\circ -1}$ is too, so we have that for any subset
        $U \subseteq X$, \# $f^{-1}(U) = \# U$. Now, for any nonempty open set
        $U \subseteq X$, we have $\mathbb{R} \backslash U$ is finite and
        \[
        \mathbb{R} \backslash f^{-1}(U) = f^{-1}(X) \backslash f^{-1}(U)
        \stackrel{\text{week 1, 4.(v)}}{=} f^{-1}\left( X \backslash U \right) 
        .\] 
        And thus
        $\# \mathbb{R} \backslash f^{-1}(U) = \# f^{-1}(X \backslash U) = \#
        X \backslash U < \infty$, so $f^{-1}(U)$ is open. Now, if
        $U = \varnothing$ then $f^{-1}(U) = \varnothing$, which is open in $X$.
        Thus for any open set $U \subseteq X$, $f^{-1}(U)$ is open in $X$, so
        $f$ is continuous.
        \\
        Now, by replacing all $f$ in the above by $f^{\circ -1}$ (the inverse
        of $f$ ), we find that
        $f^{\circ -1}$ is continuous too. Hence $f$ is a homeomorphism.\\
        \linebreak
        \textbf{Homework 6:} Consider the following topological spaces
        $(X, \mathcal{T})$ and the given sequences $\left( a_n \right)_{n \in
        \mathbb{N}}$ in them. In each case either prove that the sequence does
        not converge, or find all the points the sequence converges to.\\
        \begin{enumerate}[label=(\roman*)]
            \item $X = \mathbb{R}$, $T$ generated by all the half-open
                intervals $[x,y), x,y \in \mathbb{R}, x < y$.
                \begin{enumerate}
                    \item $a_{n} = \frac{1}{n}, n \in \mathbb{N}$.
                    \item $a_n = -\frac{1}{n}, n \in \mathbb{N}$.   
                \end{enumerate}
            \item $X = \left\{ 1,2,3,4,5 \right\} $, $\mathcal{T}=
                \left\{ \varnothing, X, \{1\}, \{2,3\}, \{1,2,3\}, \{2,3,4,5\}
                \right\} $.
                \begin{enumerate}
                    \item $a_n = 4, n \in \mathbb{N}$.
                    \item $a_n = 2$ when $n \in \mathbb{N}$ is even, and
                        $a_n = 3$ when $n \in \mathbb{N}$ is odd.
                \end{enumerate}
            \item $X = \mathbb{R}$, $\mathcal{T} = \{\varnothing, \mathbb{R}\}
                \cup \left\{ (t,\infty)  \mid t \in \mathbb{R} \right\} $.
                \begin{enumerate}
                    \item $a_n = \frac{1}{n}, n \in \mathbb{N}$.
                    \item $a_n = n, n \in \mathbb{N}$.
                \end{enumerate}
        \end{enumerate}
   \textit{Solution:}\\
   (i):\\
   (a): We claim $a_n \to 0$ as its only limit point. For any open set
   $U$ containing $0$, there exists a basis element $[x,y)$ such that $0\in [x,y)
   \subseteq U$ by definition 3.1 and definition 3.7. Now we have $x\le 0<y$.
   Thus by the Archimedean property, we can find $N \in
   \mathbb{N}$ such that for all $n\ge N$, we have
   $\frac{1}{y} < n$ hence since $n$ is positive, $x\le 0 < \frac{1}{n}<y$, so
   $\frac{1}{n} \in [x,y) \subseteq U$ and thus by definition 5.14,
   $a_n$ converges to $0$. Assume $a_n$ also converges to $z \neq 0$. If $z < 0$, then
   $[z-1, 0)$ contains $z$ but does not contain any points of the sequence
   $a_n$, so $z$ is not a limit point of $a_n$. If $z>0$ then
   there exists $N \in \mathbb{N}$ such that
   $N < \frac{1}{z} \le N+1$, and thus $z \in [\frac{1}{N+1},\frac{1}{N})$
   since
   inversion is continuous on $(0,\infty)$.
   Now, assume $\frac{1}{k} \in \left[ \frac{1}{N+1},\frac{1}{N} \right) $ for
   a $k \in \mathbb{N}$. Then $k=N+1$, and so for all $n>N+1$, 
   $\frac{1}{n} \not\in \left[ \frac{1}{N+1},\frac{1}{N} \right)$, so $z$ is
   not a limit point of $a_n$. Thus $0$ is the only limit point of the
   sequence.\\
   \linebreak
   (b) We claim that $a_n$ has no limit points in $\mathbb{R}_{l}$ and thus
   does not converge.\\
   Firstly, since for any $z\ge 0$, $[z,z+1)$ is a neighborhood of $z$ that does not intersect the
   sequence since it contains only non-negative points of $\mathbb{R}$ while all
   points of the sequence are negative, the 
   sequence does not converge to any $z\ge 0$.
   Assume $a_n \to z$ with $z<0$.\\
   There exists $N \in \mathbb{N}$ such
   that $-(N+1) < \frac{1}{z} \le  -N$, so $z \in \left[- \frac{1}{N},
   -\frac{1}{N+1}  \right) $ by continuity of inversion on $(-\infty, 0)$, which, as in (a), only contains $-\frac{1}{N}$ 
   of the sequence, so for all $n>N$, $a_n = -\frac{1}{n} \not\in \left[
       -\frac{1}{N}, -\frac{1}{N+1}\right)$, hence $a_n$ does not converge to
       $z$.\\
       \linebreak
       (ii):\\
       (a) We claim that $a_n = 4, n \in \mathbb{N}$ converges to $4$ and $5$.
       Let $U$ be any neighborhood of either point. Then since the only
       neighborhoods containing $4$ or $5$ are $X$ and $\{2,3,4,5\}$ which both
       contain $4 = a_n, \forall n \in \mathbb{N}$, we have that $a_n$ is in
       $U$ for all $n\ge 1$, so by definition 5.14, $a_n$ converges to $4$ and
       $5$.\\
       For the points  $1,2,3$, we have $1,2,3 \in \{1,2,3\}$ which does not
       contain  $4=a_n$ for all $n$, and thus the condition in definition 5.14
       is not satisfied, so $a_n$ does not converge to $1,2$ or $3$.\\
       \linebreak
       (b) We claim that $a_n$ converges to the points $2,3,4$ and $5$.\\
       We note that for any neighborhood of any of the points $2,3,4$ and $5$, 
       $\{2,3\}$ is contained in the neighborhood, and thus for all $n\ge 1$,
       $a_n$ is in the neighborhood, so $a_n$ converges to $2,3,4$ and $5$ by
       definition 5.14.\\
       Now, for the point $1$, take the neighborhood $\{1\}$. If there exists
       $N \in \mathbb{N}$ such that for all $n\ge N$, $a_n \in \{1\}$ then
       $a_n = 1$ which is a contradiction. Hence $a_n$ does not converge to
       $1$.\\
       \linebreak
       (iii):\\
       (a) We claim that $a_n$ converges to all points in $\mathbb{R}_{0}^{-}
       = \left\{ x \in \mathbb{R}  \mid x\le 0 \right\} $.\\
       Let $x \in \mathbb{R}_{0}^{-}$ and $U$ be any neighborhood of $x$.
       Assume first that $U \neq X$. Then $U$ is of the form
       $(t,\infty)$ with $t<x \le 0$. Now, since for all $n\ge 1$, $a_n
       > 0$, we have $a_n \in (t,\infty)$, so the condition in definition 5.14
       is satisfied with $N=1$. If $U = X$, $a_n$ is similarly in $X$ for any
       $n\ge 1$. Thus $a_n$ converges to $x$.\\
       We claim that $a_n$ does not converge to any $x >0$. Fix any $x >0$,
       then
       $x \in (\frac{x}{2},\infty)$. Now, by the Archimedean property,
       there exists $N \in \mathbb{N}$ such
       that $\frac{2}{x} < N$, and thus for all $n\ge N$, we have
       $a_n = \frac{1}{n} < \frac{x}{2}$, so for all $n \ge N$, $a_n \not\in 
       (\frac{x}{2},\infty)$, and thus $a_n$ does not satisfy the condition in
       definition 5.14, so it does not converge to any $x >0$.\\
       \linebreak
       (b) We claim $a_n$ converges to all points in $\mathbb{R}$.\\
       Fix any $x \in \mathbb{R}$ and take any neighborhood U of $x$. Assume
       first $U\neq X$, then $U$ is of the form $(t, \infty)$.
       Then by the Archimedean property,
       there exists $N \in \mathbb{N}$ such that $x < N$, so we
       find that for all $n\ge N$, we have
       $t < x < N \le n = a_n$, and thus for all $n\ge N$, $a_n \in
       (t,\infty)$. Now, if $U=X$, then $a_n$ is trivially in $U$ for all $n\ge
       1$.
       Thus $a_n$ converges to $x$ by definition 5.14.




















\end{document}
