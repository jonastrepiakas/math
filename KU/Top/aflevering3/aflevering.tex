\documentclass[a4paper]{article}

\usepackage[margin=2.5cm]{geometry}
\usepackage[pdftex]{graphicx}
\usepackage[utf8]{inputenc}
\usepackage[T1]{fontenc}
\usepackage{textcomp}
\usepackage{babel}
\usepackage{amsmath, amssymb}
\usepackage[colorlinks=true,linkcolor=blue]{hyperref}
\usepackage{float}
\usepackage{mathrsfs}
\usepackage{enumitem}

\newcommand{\incfig}[2][1]{%
\def\svgwidth{#1\columnwidth}
\import{./figures/}{#2.pdf_tex}
}


% figure support
\usepackage{import}
\usepackage{xifthen}
\pdfminorversion=7
\usepackage{pdfpages}
\usepackage{transparent}

\pdfsuppresswarningpagegroup=1

\setlength\parindent{0pt}

\newcommand{\qed}{\tag*{$\blacksquare$}}
\newcommand{\qedwhite}{\hfill \ensuremath{\Box}}

%Inequalities
\newcommand{\cycsum}{\sum_{\mathrm{cyc}}}
\newcommand{\symsum}{\sum_{\mathrm{sym}}}
\newcommand{\cycprod}{\prod_{\mathrm{cyc}}}
\newcommand{\symprod}{\prod_{\mathrm{sym}}}

%Linear Algebra

%Redeclaring Span and image
\DeclareMathOperator{\Span}{span}
\DeclareMathOperator{\Ima}{Im}
\DeclareMathOperator{\diag}{diag}

%Row operations
\newcommand{\elem}[1]{% elementary operations
\xrightarrow{\substack{#1}}%
}

\newcommand{\lelem}[1]{% elementary operations (left alignment)
\xrightarrow{\begin{subarray}{l}#1\end{subarray}}%
}

%SS
\DeclareMathOperator{\supp}{supp}
\DeclareMathOperator{\Var}{Var}

%NT
\DeclareMathOperator{\ord}{ord}

\DeclareMathAlphabet{\pazocal}{OMS}{zplm}{m}{n}
\newcommand{\unif}{\pazocal{U}}


\title{Assignment 3}
\author{Jonas Trepiakas - hvn548@alumni.ku.dk}
\date{}

\begin{document}
\maketitle
\newpage
    \textbf{Homework 7:} Let $X$ and $Y$ be topological spaces. We say that $X$ 
    has property (*) if every compact subset of $X$ is closed in $X$. Prove:
    \begin{enumerate}[label=(\roman*)]
        \item If $X$ has property (*), then $X$ is $T_1$.
        \item If $X$ has property (*) and $f  \colon Y \to X$ is a continuous
            map, then for every compact subset $A \subseteq Y$ we have that
            $f(A) \subseteq X$ is closed in $X$.
        \item A finite topological space has property (*) if and only if its
            topology is the discrete topology.
    \end{enumerate}

\textit{Solution:}\\
(i) Let $x \in X$ be any element of $X$. Equip $\left\{ x \right\} $ with the
subspace topology from $X$. Let $\mathcal{A}$ be any open covering of $\left\{
x \right\} $ - such a covering exists since e.g. $\mathcal{A} = \left\{ X \cap
\{x\} \right\} $ is an open covering since $X$ is open in $X$.
Now since $\bigcup_{A \in \mathcal{A}} A = \left\{ x \right\} $, we can choose
an $A \in \mathcal{A}$ such that $x \in A$. But since
$\bigcup_{A \in \mathcal{A}} A = \{x\} \subseteq A \subseteq \bigcup_{A \in
\mathcal{A}} \{x\}$, we have that $\{x\} = A$, so $\{A\}$ is a finite open
subcovering of $\{x\}$. Hence $\{x\}$ is compact with the subspace topology. By
assumption, $X$ has property (*), so $\{x\}$ is closed. Since $x \in X$ was
arbitrary, all one-point sets are closed in $X$, i.e. $X$ is $T_1$.\\
\linebreak
(ii) Let $A \subseteq Y$ be compact. If $f  \colon Y \to X$ is continuous, then
$f|_{A}  \colon A \to X$ under the restriction is continuous since
$f|_{A}^{-1} (U) = A \cap f^{-1}(U)$ is open in $A$ since $f$ is continuous for any open set $U
\subseteq X$. Thus by proposition 9.12, $f(A) = f|_{A}(A)$ is compact in $X$.
Assuming $X$ has property (*), we thus find that $f(A)$ is closed in $X$.\\
\linebreak
(iii) Assume $X$ is a finite topological space with property (*). We saw in (i)
that $X$ is $T_1$. We further claim that the one-point sets are open.
Let $x \in X$. Then $\bigcup_{y \in X -\{x\}} \{y\}$ is a finite (since $X$ is
finite) union of closed sets (since $X$ is $T_1$ ). Hence by proposition
5.2.(c), $\bigcup_{y \in X -\{x\}} \{y\}$ is closed, so by definition 5.1, its
complement $X - \bigcup_{y \in X - \{x\}} \{y\} = X - \left( X - \{x\} \right)
= \{x\}$ is open. Since $x$ was arbitrary, all one-point sets are also open.\\
Now let $A$ be any nonempty subset of $X$. Then $A = \bigcup_{a \in A} \{a\}$ 
is a finite union of simultaneously open and closed sets, so it is open by
definition 2.1.(b). Thus the topology on $X$ is the power set of $X$, so by
example 2.5, the topology on $X$ is the discrete topology.\\
\linebreak
Conversely, if $X$ is a finite topological space equipped with the discrete
topology, then every set is closed and open by example 5.4, so in particular,
if $A$ is a subset of $X$, then $A = \bigcup_{a \in A} \{a\}$ which is a finite
union of closed sets and hence closed by proposition 5.2. So all subsets of $X$
are closed and thus, in particular, all compact subsets of $X$ are closed in
$X$. So $X$ has property (*).\\
\linebreak
\textbf{Homework 8:} Let $X = \mathbb{N}$ with the discrete topology, and let
$X^{+} = X \cup \{\infty \}$ denote the one-point compactification of $X$. Let
$Y$ be another topological space. Given a sequence $\left( y_n \right)_{n \in
\mathbb{N}}$ in $Y$ and a point $y \in Y$, define $f \colon X^{+} \to Y$ by
$f(n) = y_n$ when $n \in X$ and $f(\infty) = y$.\\
Show that the sequence $(y_n)_{n \in \mathbb{N}}$ converges to $y$ in $Y$ if
and only if $f \colon X^{+} \to Y$ is continuous.\\
\linebreak
\textit{Solution:} Assume that $(y_n)_{n \in \mathbb{N}}$ converges to $y$ in
$Y$. Let $U \subseteq Y$ be any open set. Then $f^{-1}(U) = f^{-1}\left( U
\cap \left( (y_n)_{n \in \mathbb{N}}\cup y \right) \right) $. Thus it suffices
to show that $f$ is continuous when restricting the range to the subspace
$A = \left\{ y_n  \mid n \in \mathbb{N} \right\} \cup \{y\}$. 
Let $U$ be an open set in $A$. If $y \not\in U$, then $f^{-1}(U) \subseteq X$,
and since the open sets of $X$ are open in $X^{+}$ since $X$ is open in $X^{+}$, and all subsets of $X$ are
open in $X$ since it is equipped with the discrete topology, we have
$f^{-1}(U)$ is open in $X^{+}$.\\
If $y \in U$, then $f^{-1} (U) = \{\infty\} \cup f^{-1}\left( U - \{y\} \right)
$. We claim that the complement of $f^{-1}(U)$ is closed in $X^{+}$.
The complement is contained in $X$ since $\infty \in f^{-1}(U)$, so it is in
particular closed by example 5.4. Since the complement of $f^{-1}(U)$ is closed
in $X^{+}$, $f^{-1}(U)$ is open in $X^{+}$ by definition 5.1.\\
Thus $f \colon X^{+} \to Y$ is continuous.\\
\linebreak
Assume conversely that $f \colon X^{+} \to Y$ is continuous. Let $U$ be any
neighborhood of $y$. Then $f^{-1}(U)$ is open in $X^{+}$ and in particular
contains $\infty$. Thus its complement is contained in $X$, and since $X$ is
equipped with the discrete topology, example 5.4 gives that the complement of
$f^{-1}(U)$ is closed. We further claim that the complement of $f^{-1}(U)$ is
bounded above by an element $M$.\\
Let $A \subseteq X$ be any infinite subset of $X$. Then
$\bigcup_{a \in A} \{a\}$ is an open covering of $A$ with no finite subcover,
so $A$ is not compact. Thus any infinite subset of $X$ is not compact.\\
 However, since all one-point sets are open, $X$ is
Hausdorff. Thus all closed subsets of $X$ are compact by theorem 9.9. Since
the complement of $f^{-1}(U)$ is a closed subspace of $X$, it can thus not be infinite. Thus by
the total ordering of $\mathbb{N}$, there exists a maximal element $M \in \mathbb{N}$ in the
complement of
$f^{-1}(U)$. Since
$f^{-1}\left( Y-U \right) = f^{-1}(Y) - f^{-1}(U) = X^{+} - f^{-1}(U)$, we thus
have that $\mathbb{N} \cap [M+1, \infty) \subseteq f^{-1}(U)$, since
$X^{+} = f^{-1}(U) \cup X^{+} - f^{-1}(U) = f^{-1}(U) \cup f^{-1}\left( Y-U
\right) $. Thus for all $n \ge M+1$, $y_n = f(N+1) \in f \left( 
\mathbb{N} \cap [M +1, \infty) \right) \subseteq U$.\\
Since $U$ was arbitrary, $y_n$ converges to $y$ by definition 5.14.\\
\linebreak
\textbf{Homework 9:} Investigate which of the following statements are true in
general, and accordingly either give a proof or find a counterexample:

\begin{enumerate}[label=(\roman*)]
    \item If $(X,\mathcal{T}_1)$ is a connected space, and $\mathcal{T}_2$ is
        another topology on $X$ which is coarser than $\mathcal{T}_1$, then
        $(X, \mathcal{T}_2)$ is also connected.
    \item If $(X, \mathcal{T}_1)$ is a path-connected space, and
        $\mathcal{T}_2$ is another topology on $X$ which is coarser than
        $\mathcal{T}_1$, then $(X, \mathcal{T}_2)$ i s also path-connected.
    \item If $(X, \mathcal{T}_1)$ is a path-connected space, and
        $\mathcal{T}_2$ is another topology on $X$ which is finer than
        $\mathcal{T}_1$, then $(X, \mathcal{T}_2)$ is also path-connected.
    \item Suppose $A_n, n \in \mathbb{N}$ is a sequence of connected subsets of
        $\mathbb{R}^2$ (where $\mathbb{R}^2$ is equipped with the standard
        topology) satisfying $A_n \supseteq A_{n+1}$ for all $n \in
        \mathbb{N}$. Then $\bigcap_{n \in \mathbb{N}} A_n$ is connected.
\end{enumerate}
\textit{Solution:}\\
(i) This is true. Assume there is a separation $U,V$ of $X$ where $U$ and $V$
are disjoint nonempty open subsets of $X$ whose union is $X$. Then since
$\mathcal{T}_2$ is coarser than $\mathcal{T}_1$, $U$ and $V$ are also in
$\mathcal{T}_1$ and thus would also form a separation of $X$. However, $(X,
\mathcal{T}_1)$ is connected, so no such separation exists. Thus $(X,
\mathcal{T}_2)$ is connected.\\
\linebreak
(ii) This is true. Let $x $ and $y$ be any points of $X$. Then by assumption there exists
a path, i.e. a continuous function, $f \colon [a,b] \to X$ where $X$ is equipped with the topology
$\mathcal{T}_1$ and $f(a) = x$ and $f(b) = y$. Since a continuous function is
also continuous when the range is equipped with any coarser topology by
proposition 2.12, this path
is also a path when $X$ is equipped with the topology $\mathcal{T}_2$. Hence
$(X, \mathcal{T}_2)$ is path connected.\\
\linebreak
(iii) This is false. For example, the function $f \colon [0,1] \to \mathbb{R}$ 
given by $f(t) = x + t(y-x)$ is a path (continuous since polynomial, example
1.12) connecting any points  $x,y$ of
$\mathbb{R}$ when $\mathbb{R}$ is equipped with the standard topology. 
Now let $\mathbb{R}_d$ denote $\mathbb{R}$ equipped with the discrete topology.
Let $t \in [0,1]$. The set $\{f(t)\}$ is open in $\mathbb{R}_d$, but
$f^{-1} \left( \{f(t)\} \right) = \{t\}$ is not open in $[0,1]$. 
For example
the complement of $\{t\}$ is not closed in $[0,1]$ since it has $t$ as
a limit point; or alternatively, if $\{t\}$ were open, it would contain a basis
element (proposition 3.2), but any basis element is an interval by definition
of the order topology, and since $t$ has no immediate
successor or precessor in $[0,1]$, it thus contains a point
different from $t$; since this point is not in $\{t\}$, $\{t\}$ cannot be
open.\\
\linebreak
(iv) This is false. Let $A_n$ be the sets
\[
    A_n = \mathbb{R}_- \times 0 \cup \mathbb{R}_+ \times 0 \cup 
    \left\{ x \times y  \mid y \in (0, \frac{1}{n}) \right\}.
\] 
We will show that $A_n$ is path connected by showing that each $A_n$ is
star-shaped domain with respect to the point $(0, \frac{1}{2n})$. Then, since
path-connectedness is an equivalence relation by the section following
definition 8.12, it will follow that there exists
a path between any points of $A_n$ and hence that it is path-connected and then
since every path-connected space is connected by theorem 8.13, it follows that
$A_n$ is connected.\\
For any point $x \in A_n$, define $\gamma_x  \colon [0,1] \to A_n$ to be the function
\[
\gamma_x (t) = \left(0,\frac{1}{2n} \right) t + (1-t) x.
\] 
For all $t \in [0,1]$, this is clearly contained in $A_n$ since for $t=0$,
$\gamma_x (0) = x \in A_x$ and for all other $t$, the $y$-coordinate of
$\gamma_x (t)$ is in $(0,\frac{1}{n})$, so in particular
$\gamma_x (t) \in \mathbb{R} \times \left( 0, \frac{1}{n} \right) 
\subset A_n$. The function is furthermore continuous since it's a polynomial in
each coordinate,
$\mathbb{R} \to \mathbb{R}^2$ (example 1.12), and therefore also from the
subspace $[0,1]$. Thus it is a path. 
Now by the section following definition 8.12, the reverse path
$\tilde{\gamma}_x  \colon [0,1] \to A_n$ by $\tilde{\gamma}_x (t) = \gamma_x
(1-t)$ is a path. Thus for any points $x,y \in A_n$, we can form the path
$\gamma (t) = \begin{cases}
    \gamma_x (2t) & t \le \frac{1}{2}\\
    \tilde{\gamma}_y (2t-1) & t>\frac{1}{2}
\end{cases}$ 
which is a path connecting $x$ and $y$ by the section following definition
8.12.\\
Thus we conclude $A_n$ to be connected for all $n \in \mathbb{N}$ and clearly
$A_{n+1} \subseteq A_n$ for all $n$, but
$\bigcap_{n \in \mathbb{N}} A_n = \mathbb{R}_- \times 0 \cup \mathbb{R}_+
\times 0$
which is clearly separable by $\left\{ x \times y  \mid x >0 \right\} 
\cap \mathbb{R}_- \times 0$ and $\left\{ x \times y  \mid
    x<0 \right\} \cap \mathbb{R}_+ \times 0
    $, where $\{x \times y  \mid x >0\}$ and $\{x \times y  \mid x < 0\}$
    are clearly open since for any point $x \times y$ in these
    sets, the ball of radius $x$ is contained in the sets. For each set, taking the union
    over these balls, we find that they are each open.



    































\end{document}
