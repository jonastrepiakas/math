\documentclass[a4paper]{article}

\usepackage[margin=2.5cm]{geometry}
\usepackage[pdftex]{graphicx}
\usepackage[utf8]{inputenc}
\usepackage[T1]{fontenc}
\usepackage{textcomp}
\usepackage{babel}
\usepackage{amsmath, amssymb}
\usepackage[colorlinks=true,linkcolor=blue]{hyperref}
\usepackage{float}
\usepackage{mathrsfs}


\newcommand{\incfig}[2][1]{%
\def\svgwidth{#1\columnwidth}
\import{./figures/}{#2.pdf_tex}
}


% figure support
\usepackage{import}
\usepackage{xifthen}
\pdfminorversion=7
\usepackage{pdfpages}
\usepackage{transparent}

\pdfsuppresswarningpagegroup=1

\setlength\parindent{0pt}

\newcommand{\qed}{\tag*{$\blacksquare$}}
\newcommand{\qedwhite}{\hfill \ensuremath{\Box}}

%Inequalities
\newcommand{\cycsum}{\sum_{\mathrm{cyc}}}
\newcommand{\symsum}{\sum_{\mathrm{sym}}}
\newcommand{\cycprod}{\prod_{\mathrm{cyc}}}
\newcommand{\symprod}{\prod_{\mathrm{sym}}}

%Linear Algebra

%Redeclaring Span and image
\DeclareMathOperator{\Span}{span}
\DeclareMathOperator{\Ima}{Im}
\DeclareMathOperator{\diag}{diag}

%Row operations
\newcommand{\elem}[1]{% elementary operations
\xrightarrow{\substack{#1}}%
}

\newcommand{\lelem}[1]{% elementary operations (left alignment)
\xrightarrow{\begin{subarray}{l}#1\end{subarray}}%
}

%SS
\DeclareMathOperator{\supp}{supp}
\DeclareMathOperator{\Var}{Var}

\DeclareMathAlphabet{\pazocal}{OMS}{zplm}{m}{n}
\newcommand{\unif}{\pazocal{U}}

\usepackage{enumitem}

\title{Assignment 1}
\author{Jonas Trepiakas - hvn548@alumni.ku.dk}
\date{}



\begin{document}
\maketitle
\newpage
     \textbf{Homework 1:} Decide which of the following $\mathcal{B}$ are
     a basis for some topology on the respective given sets $X$.
    \begin{enumerate}[label=(\roman*)]
    \item $X= \mathbb{N}, \mathcal{B} = \left\{ U \subseteq \mathbb{N}  \mid 
        10 \le \# \left( \mathbb{N} \backslash U \right) <\infty \right\} $;
    \item $X = \mathbb{N}, \mathcal{B} = \left\{ U \subseteq \mathbb{N}
         \mid \# \left( \mathbb{N} \backslash U \right) \le 10 \right\} $ ;
     \item $X = \mathbb{R}^2, \mathcal{B} = \left\{ 
         B_r (0)  \mid r > 0\right\} $ where $B_r (0)$ denotes the open disk of
         radius $r$ around $0$ in $\mathbb{R}^2$ with respect to the Euclidean
         metric.
    \end{enumerate}
    \textit{Solution:} We will freely use that if $A \subseteq B$, the
    inclusion function is injective and thus $\#A \le \# B$.\\
    (i) We claim $\mathcal{B}$ is a basis for a topology on $X$.
    We check the requirements in theorem 3.6:\\
    We have that any $U \in \mathcal{B}$ is by definition a subset of $X$.
    Now,\\
    (a) Let $U_x = \mathbb{N} \backslash \{x, x+1, x+2, \ldots, x+9\}$ 
    for $x \in \mathbb{N}$. Then
    $U_x \subseteq \mathbb{N}$ and $\# \left( \mathbb{N} \backslash U_x \right) 
    = \# \{x, x+1, x+2, \ldots, x+9\} = 10$, so $U_x \in \mathcal{B}$, and
    \[
    X = \bigcup_{x \in X} \left\{ x \right\} 
    \subseteq \bigcup_{x \in X} U_x \subseteq \bigcup_{U \in \mathcal{B}} U \subseteq X
    .\] 
    Hence $\bigcup_{U \in \mathcal{B}} U = \bigcup_{x \in X} U_x = X$.\\
    (b) Let $U, V \in \mathcal{B}$ and let $x \in U \cap V$ (if $U \cap
    V = \varnothing$, the requirement is satisfied trivially).
    We have 
    \[
    \# \mathbb{N} \backslash \left( U \cap V \right) 
    \stackrel{\text{De Morgan}}{=} \# \mathbb{N} \backslash U \cup \mathbb{N} \backslash V \ge \# \mathbb{N}
\backslash U \ge 10\]
and
\[
\# \mathbb{N} \backslash \left( U \cap V \right) 
\stackrel{\text{De Morgan}}{=} \# \mathbb{N} \backslash
U \cup \mathbb{N} \backslash V \stackrel{\text{PIE}}{\le} \# \mathbb{N} \backslash U + \#  \mathbb{N}
\backslash V < \infty
,\] 
where PIE stands for the principle of inclusion-exclusion (sætning 469,
dismat).
Thus $U \cap V \in \mathcal{B}$ and the condition is satisfies with $W = U\cap
V$ since
$x \in U \cap V \subseteq U \cap V$.\\
Now the claim follows by theorem 3.6.\\
\linebreak
(ii) We claim $\mathcal{B}$ is not a basis for a topology on $\mathbb{N}$.\\
Let $U = \left\{ x \in \mathbb{N}  \mid x \ge 11 \right\} ,
V = \left\{ x \in \mathbb{N}  \mid x \ge 21 \lor x \le 10 \right\} $. Then
\[
\# \mathbb{N} \backslash U = \# \left\{ 1, 2, \ldots, 10 \right\} = 10,
\qquad \# \mathbb{N} \backslash V = \# \left\{ 11, 12, \ldots, 20 \right\} = 10
.\] 
Thus $U, V \in \mathcal{B}$. Now,
\[
U \cap V = \left\{ x \in \mathbb{N}  \mid x \ge 11 \land \left( x\ge 21 \lor
x \le 10 \right)  \right\} = \left\{ x \in \mathbb{N}  \mid x \ge 21 \right\}.
.\] 
Then $21 \in U \cap V$, and assume $\exists W \in \mathcal{B}$ such that
$21 \in W \subseteq U \cap V = \left\{ x \in \mathbb{N}  \mid x \ge 21 \right\}
$. Then $W \cap \left( U \cap V \right)^{c} = \varnothing$, so
\[
\left\{ 1, 2, \ldots, 20 \right\} \subseteq \mathbb{N} \backslash W
.\] 
Hence $20 = \# \left\{ 1, \ldots, 20 \right\} \le \# \mathbb{N}\backslash W$,
so
$W \not\in \mathcal{B}$ in contradiction with the assumption.\\
Now the claim follows by theorem 3.6.\\
\linebreak
(iii) We claim $\mathcal{B}$ is a basis for a topology on $\mathbb{R}^2 = X$.\\
Any element in $\mathcal{B}$ is by definition a subset of $X$. Now,
we claim $\bigcup_{r \in \mathbb{N}} B_r (0) = \mathbb{R}^2$: since
$B_r (0) \subseteq \mathbb{R}^2$, $\bigcup_{r \in \mathbb{N}} B_r (0) \subseteq
\mathbb{R}^2$ trivially.\\
Now assume $x \in \mathbb{R}^2$. Then $|x| \in \mathbb{R}$, so there exists
$N \in \mathbb{N}$ such that $|x| < N$ (take e.g. $N = \left\lceil |x|
\right\rceil +1$ ). Then $x \in B_N (0)$ by definition and thus $x \in
\bigcup_{r \in \mathbb{N}} B_r (0) $. Hence $\mathbb{R}^2 \subseteq 
\bigcup_{r \in \mathbb{N}} B_r (0)$. Thus the claim follows.\\
Now we have
\[
\mathbb{R}^2 = \bigcup_{r \in \mathbb{N}} B_r (0)
\subseteq \bigcup_{U \in \mathcal{B}} U \subseteq \mathbb{R}^2
.\] 
Thus $\bigcup_{U \in \mathcal{B}} U = \mathbb{R}^2$, so (a) in theorem 3.6 is
satisfied.\\
\linebreak
Now let $B_r(0), B_s(0) \in \mathcal{B}$. We claim
$B_r (0) \cap B_s(0) = B_{\min\{r,s\}}(0)$:\\
Let $x \in B_r(0) \cap B_s(0)$, then $x \in B_r(0)$ and $x \in B_s(0)$ so
by definition  $|x| < r$ and $|x| < s$, hence, since $\min\{r,s\} \in \{r,s\}$,
$|x| < \min \{r,s\}$, so  $x \in B_{\min \{r,s\}}(0)$.\\
Conversely, if $x \in B_{\min \{r,s\}}(0)$ then 
$|x| < \min\{r,s\} \le r,s$, so by transitivity (or simply equality if $r=s$ ),
$|x| < r$ and $|x| < s$, hence
$x \in B_r(0)$ and $x \in B_s(0)$ so $x \in B_r(0) \cap B_s(0)$.\\
Now condition (b) in theorem 3.6 is satisfied by choosing $W
= B_{\min\{r,s\}}(0)$ (since $r,s > 0$ also $\min\{r,s\} >0$ so it is in
$\mathcal{B}$) since for any $x \in B_r(0) \cap B_s(0), x \in
B_{\min\{r,s\}}(0) \subseteq B_r(0) \cap B_s(0)$.\\
Now the claim follows by theorem 3.6.\\
\linebreak
\textbf{Homework 2}. Let $X$ be a non-empty set and fix a point $x_0 \in X$.
Define
\[
    \mathcal{B} := \left\{ \{x_0\} \right\} \cup \left\{ \{x_0, x\}  \mid 
    x \in X \backslash \{x_0\} \right\} 
.\] 
(i) Prove that $\mathcal{B}$ is the basis for some topology on $X$ which we
will denote by $\mathcal{T}$.\\
\linebreak
\textit{Solution:} We check the conditions in theorem 3.6:\\
All sets in $\mathcal{B}$ are subsets of $X$.\\
\linebreak
Assume first $X \backslash \{x_0\} \neq \varnothing$.\\
Now, let $U_x = \{x_0, x\}$ for $x \in X \backslash \{x_0\}$, then
$x \in U_x \in \mathcal{B}$, so
\[
    X = \bigcup_{x \in X} \{x\} \subseteq \bigcup_{x \in X \backslash \{x_0\}}
    U_x \subseteq \bigcup_{U \in \mathcal{B}} U \subseteq X
\] 
where the first inclusion follows by the above and because $x_0 \in U_x$ for
any $x \in X \backslash \{x_0\}$.
Thus $\bigcup_{U \in \mathcal{B}} U = X$, so (a) is satisfied.\\
Let  $U,V \in \mathcal{B}$. If either $U$ or $V$ are $\{x_0\}$, then
$U\cap V = \{x_0\}$. If $U = \{x_0, x\}, V = \{x_0, y\}$ for $x,y \in
X \backslash \{x_0\}$ with $x \neq y$ then also $U \cap V = \{x_0\}$. In these
cases, setting $W = \{x_0\}$, $W \in \mathcal{B}$ and satisfies the condition
(b) of theorem 3.6: $x_0 \in W \subseteq U \cap V$.\\
If $x=y$ above, then $U\cap V = U \in \mathcal{B}$, hence for all
$x \in U\cap V = U$, $x \in U \subseteq U\cap V$, so $W=U$ satisfies (b).\\
This exhausts all cases, hence (b) is checked and it follows that
$\mathcal{B}$ is a basis for some topology on $X$ by theorem 3.6.\\
\linebreak
If $X \backslash \{x_0\} = \varnothing$, then $\mathcal{B}= \{X\}$ which
clearly is a basis for the trivial topology on $X$ since $\mathcal{B} \subseteq
\{\varnothing, X\}$ and for any $x \in X$ and every neighborhood $U$ of $x$,
$U$ must be $X$ and thus the condition in definition 3.1 is satisfied with $U'
= X \in \mathcal{B}$ since $x \in U' \subseteq U$.\\
\linebreak
(ii) Find all subsets of $X$ which are simultaneously open and closed with
respect to $\mathcal{T}$.\\
\linebreak
\textit{Solution:} If $A \subseteq X$ is simultaneously open and closed with
respect to $\mathcal{T}$, then $A \in \mathcal{T}$ and $A^{c} = X \backslash A \in
\mathcal{T}$. Since $X \backslash \varnothing = X \in \mathcal{T}, X \backslash
X = \varnothing \in \mathcal{T}$, we have that $X, \varnothing$ are
simultaneously open and closed.\\
\linebreak
Assume now $A$ is neither $X$ nor $\varnothing$ - in particular we are dealing
with the case $X \backslash \{x_0\} \neq \varnothing$ from (i).\\
Thus by proposition 3.2 and theorem 3.6.(a),
$A = \bigcup_{U \in S} U, A^{c} = \bigcup_{U \in T} U$ where
$S,T \subsetneq \mathcal{B}$ and $S,T \neq \varnothing$. Then
\[
\varnothing = A \cap A^{c} = \bigcup_{U \in S} U \cap \bigcup_{V \in T} V
\stackrel{(7.52) \text{ dismat}}{=} \bigcup_{U \in S, V \in T}  \underbrace{U \cap V}_{\text{contains }x_0}
.\] 
The last equality follows by using the distribute law for sets ((7.52) in the
dismat book twice).\\
We saw in (i) that any intersection of basis elements contains $x_0$, hence the
union above contains $x_0$ and hence cannot equal the empty set. Thus, no such
$A$ can exist. So the only subsets of $X$ that are simultaneously open and closed with
respect to $\mathcal{T}$ are $X$ and $\varnothing$.\\
\linebreak
(iii) Let $Y = X \backslash \{x_0\}$. Prove that the subspace topology on $Y$
obtained from $(X, \mathcal{T})$ equals the discrete topology on $Y$.\\
\linebreak
\textit{Solution:} We note that any topology on $Y$ is coarser than the
discrete topology on $Y$ (Remark 2.11), hence we must only show that the discrete topology on
$Y$ is coarser than the subspace topology on $Y$ obtained from $(X,
\mathcal{T})$.\\
Let $A$ be any open set in the discrete topology on
$Y$; let $y \in A$. Since $A \subseteq X \backslash \{x_0\}$, $A \cap \{x_0\}
= \varnothing$, so $y \in \{x_0, y\} \cap Y \subseteq A$, and since
$\{x_0, y\}$ is a basis element in $X$, it is open in the topology generated by
the basis by definition 3.7, hence
$\{x_0, y\} \cap Y$ is open in the subspace
topology on $Y$ by definition 4.1. Letting $A_y = \{x_0, y\} \cap Y$, we thus
get
\[
    A = \bigcup_{y \in A} \{y\} \subseteq \bigcup_{y \in A} A_y
    \subseteq A
.\] 
Hence $A = \bigcup_{y \in A} A_y$ which is open as it is the union of open
sets.
So $A$ is open in the subspace topology.\\
\linebreak
\textbf{Homework 3}. Let $(X, <)$ be a totally ordered set. Assume that $X$ has
a maximal element with respect to $<$ which we call $x_{max}$. Let
\[
    \mathcal{S} = \left\{ (x, x_{max}]  \mid x \in X \backslash \{x_{max}\} \right\} 
.\] 
Under what condition does $\mathcal{S}$ define a subbasis for some topology on
$X?$ In the case that it defines a subbasis, compare, if possible, the topology
generated by $\mathcal{S}$ with the order topology on $X$.\\
\linebreak
\textit{Solution:} By lemma 3.11, $\mathcal{S}$ is a subbasis for some topology
on $X$ if it satisfies (a) of definition 3.10.\\
Assume $X$ has no minimal element. Then let $x \in X$. We then have
$x \in (y, x_{max}] \in \mathcal{S}$ where $y \in X$ with $y<x$ (such an
element exists since otherwise $x$ would be a minimal element). Since $x \in X$
was arbitrary, we can for any $x \in X$ find $S_x \in \mathcal{S}$ such that
$x \in S_x$.
Hence
\[
    X = \bigcup_{x \in X} \{x\} \subseteq \bigcup_{x \in X} S_x
    \subseteq \bigcup_{S \in \mathcal{S}} S \subseteq X
.\] 
So (a) of definition 3.10 is satisfied.\\
\linebreak
Assume that $X$ has a minimal element, call it $x_{min}$. Then $x_{min} \in X$
but there does not exist $x \in X \backslash \{x_{max}\}$ such that
$x_{min} \in (x, x_{max}]$ since if such an $x$ existed, $x < x_{min}$ by
definition of intervals, and this contradicts the minimality of $x_{min}$. Thus
$x \not\in \bigcup_{S \in \mathcal{S}} S$, so $\bigcup_{S \in \mathcal{S}}
S \neq X$ and thus (a) of definition 3.10 is not satisfied; so $\mathcal{S}$ is
not a subbasis for some topology $\mathcal{T}$ on $X$.\\
Hence $\mathcal{S}$ defines a subbasis for some topology on $X$ if and only if
$X$ does not have a minimal element.\\
\linebreak
Assume now that $\mathcal{S}$ does define a subbasis for some topology on
$X$.\\
Then $X$ has no minimal element, so the order topology on $X$ is generated by
the basis containing all intervals of the form $(x,y)$ with $x,y \in X, x<y$
and all intervals of the for $(x, x_{max}], x \in X \backslash \{x_{max}\}$
(basis by proposition 4.13). Hence $\mathcal{S}$ is contained in this basis,
and thus $\mathcal{S}$ is contained in the order topology on $X$. Hence, since
topologies are closed under finite intersections, the basis generated by the
subbasis $\mathcal{S}$ is also contained in the order topology, and again,
since topologies are closed under finite intersections and arbitrary unions,
the topology generated by $\mathcal{S}$ (the topology generated by the basis
generated by $\mathcal{S}$ ) is contained in the order topology.\\
\linebreak
Thus, the order topology on $X$ is finer than the topology generated by
$\mathcal{S}$ on $X$. We claim it is strictly finer:
since $X$ has no minimal element and is totally
ordered, we can choose $x \in X$ such that $x < x_{max}$. Since $x$ is not
minimal, we can choose $y \in X$ such that $y < x < x_{max}$. Again, since $y$
is not minimal, we can choose $z \in X$ with $z < y < x < x_{max}$. Now
$y \in (z,x)$ which is open in the order topology.\\
\linebreak
We claim that $\mathcal{S}$
is, in fact, even a basis for the topology it generates. For this it suffices
to show that the basis it generates is equal to $\mathcal{S}$.
Let $(x_1, x_{max}], \ldots, (x_n, x_{max}]$ be a finite collection of elements
from $\mathcal{S}$, then since $X$ is totally ordered, we can order $x_1,
\ldots, x_n$, and assume without loss of generality that
$x_1 \le x_2 \le \ldots \le x_n$. Then
\[
    \bigcap_{i=1}^{n} (x_i, x_{max}] = (x_n, x_{max}] \in \mathcal{S}
.\] 
Hence $\mathcal{S}$ is closed under finite intersections and is thus equal to
the basis it generates.\\
Thus if the topologies were equal, there would exist a basis element $(a,
x_{max}]$ of the topology generated by $\mathcal{S}$ such that
$y \in (a, x_{max}] \subseteq (z,x)$ (by definition 3.1);
however then $x_{max} < x < x_{max}$
which is a contradiction. Hence the order topology is strictly finer than the
topology generated by $\mathcal{S}$.




























\end{document}
