\documentclass[reqno]{amsart}
\usepackage{amscd, amssymb, amsmath, amsthm}
\usepackage{graphicx}
\usepackage[colorlinks=true,linkcolor=blue]{hyperref}
\usepackage[utf8]{inputenc}
\usepackage[T1]{fontenc}
\usepackage{textcomp}
\usepackage{babel}
%% for identity function 1:
\usepackage{bbm}
%%For category theory diagrams:
\usepackage{tikz-cd}


\setlength\parindent{0pt}

\pdfsuppresswarningpagegroup=1

\newtheorem{theorem}{Theorem}[section]
\newtheorem{lemma}[theorem]{Lemma}
\newtheorem{proposition}[theorem]{Proposition}
\newtheorem{corollary}[theorem]{Corollary}
\newtheorem{conjecture}[theorem]{Conjecture}

\theoremstyle{definition}
\newtheorem{definition}[theorem]{Definition}
\newtheorem{example}[theorem]{Example}
\newtheorem{exercise}[theorem]{Exercise}
\newtheorem{problem}[theorem]{Problem}
\newtheorem{question}[theorem]{Question}

\theoremstyle{remark}
\newtheorem*{remark}{Remark}
\newtheorem*{note}{Note}
\newtheorem*{solution}{Solution}



%Inequalities
\newcommand{\cycsum}{\sum_{\mathrm{cyc}}}
\newcommand{\symsum}{\sum_{\mathrm{sym}}}
\newcommand{\cycprod}{\prod_{\mathrm{cyc}}}
\newcommand{\symprod}{\prod_{\mathrm{sym}}}

%Linear Algebra

\DeclareMathOperator{\Span}{span}
\DeclareMathOperator{\im}{im}
\DeclareMathOperator{\diag}{diag}
\DeclareMathOperator{\Ker}{Ker}
\DeclareMathOperator{\ob}{ob}
\DeclareMathOperator{\Hom}{Hom}
\DeclareMathOperator{\Mor}{Mor}
\DeclareMathOperator{\sk}{sk}
\DeclareMathOperator{\Vect}{Vect}
\DeclareMathOperator{\Set}{Set}
\DeclareMathOperator{\Group}{Group}
\DeclareMathOperator{\Ring}{Ring}
\DeclareMathOperator{\Ab}{Ab}
\DeclareMathOperator{\Top}{Top}
\DeclareMathOperator{\hTop}{hTop}
\DeclareMathOperator{\Htpy}{Htpy}
\DeclareMathOperator{\Cat}{Cat}
\DeclareMathOperator{\CAT}{CAT}
\DeclareMathOperator{\Cone}{Cone}
\DeclareMathOperator{\dom}{dom}
\DeclareMathOperator{\cod}{cod}
\DeclareMathOperator{\Aut}{Aut}
\DeclareMathOperator{\Mat}{Mat}
\DeclareMathOperator{\Fin}{Fin}
\DeclareMathOperator{\rel}{rel}
\DeclareMathOperator{\Int}{Int}
\DeclareMathOperator{\sgn}{sgn}
\DeclareMathOperator{\Homeo}{Homeo}
\DeclareMathOperator{\SHomeo}{SHomeo}
\DeclareMathOperator{\PSL}{PSL}
\DeclareMathOperator{\Bil}{Bil}
\DeclareMathOperator{\Sym}{Sym}
\DeclareMathOperator{\Skew}{Skew}
\DeclareMathOperator{\Alt}{Alt}
\DeclareMathOperator{\Quad}{Quad}
\DeclareMathOperator{\Sin}{Sin}
\DeclareMathOperator{\Supp}{Supp}
\DeclareMathOperator{\Char}{char}


%Row operations
\newcommand{\elem}[1]{% elementary operations
\xrightarrow{\substack{#1}}%
}

\newcommand{\lelem}[1]{% elementary operations (left alignment)
\xrightarrow{\begin{subarray}{l}#1\end{subarray}}%
}

%SS
\DeclareMathOperator{\supp}{supp}
\DeclareMathOperator{\Var}{Var}

%NT
\DeclareMathOperator{\ord}{ord}

%Alg
\DeclareMathOperator{\Rad}{Rad}
\DeclareMathOperator{\Jac}{Jac}

%Misc
\newcommand{\SL}{{\mathrm{SL}}}
\newcommand{\mobgp}{{\mathrm{PSL}_2(\mathbb{C})}}
\newcommand{\id}{{\mathrm{id}}}
\newcommand{\Mod}{{\mathrm{Mod}}}
\newcommand{\PMod}{{\mathrm{PMod}}}
\newcommand{\SMod}{{\mathrm{SMod}}}
\newcommand{\ud}{{\mathrm{d}}}
\newcommand{\Vol}{{\mathrm{Vol}}}
\newcommand{\Area}{{\mathrm{Area}}}
\newcommand{\diam}{{\mathrm{diam}}}
\newcommand{\End}{{\mathrm{End}}}


\newcommand{\reg}{{\mathtt{reg}}}
\newcommand{\geo}{{\mathtt{geo}}}

\newcommand{\tori}{{\mathcal{T}}}
\newcommand{\cpn}{{\mathtt{c}}}
\newcommand{\pat}{{\mathtt{p}}}

\let\Cap\undefined
\newcommand{\Cap}{{\mathcal{C}}ap}
\newcommand{\Push}{{\mathcal{P}}ush}
\newcommand{\Forget}{{\mathcal{F}}orget}



\title{Assignment 1}
\author{Jonas Trepiakas - KU-Id: hvn548}
\date{}


\begin{document}

\maketitle

    \begin{exercise}[H1.1]
        \begin{proof}
            \[
            f * e (n) = \sum_{d  \mid n} f(d) e (\frac{n}{d})
            = \sum_{d  \mid n} f(d) \delta_{\frac{n}{d},1}
            = f(n)
            \] 
            and since the sets
            $\left\{ d \colon d \mid n \right\} $ and
            $\left\{ \frac{n}{d}  \colon
            d  \mid n\right\} $ are equal, we have
            \[
            g * f = 
            \sum_{d \mid n} g(d) f\left( \frac{n}{d} \right) 
            =
            \sum_{d  \mid n} g\left( \frac{n}{d} \right) 
            f \left( \frac{n}{\frac{n}{d}} \right) 
            = f * g (n)
            \] 
        \end{proof}
    \end{exercise}


    \begin{exercise}[H1.2]
        \begin{proof}
            \[
            \mu * 1 (n)
            = \sum_{d \mid n}
            \mu (d) 1 \left( \frac{n}{d} \right) 
            = \sum_{d \mid n} \mu(d)
            \] 
            If $n = p$ is a prime, we trivially
            have
            $\left\{ d \colon d \mid n \right\} 
            = \left\{ 1,p \right\} $, so
            $\sum_{d \mid n} \mu (d) =
            1 -1 = 0 = e(p)$, so it is true for
            $n$ a prime.

            Now, if $n = p^{\alpha}$, then
            \[
            \mu * 1 (n) = 
            \sum_{d  \mid n} \mu (d) 1\left( \frac{n}{d} \right) 
            = \mu (p) + \mu (1)
            = 0
            \] 
            since in all other terms, $\mu$ is evaluated at
            a non-squarefree integer.


            Lastly, for
            $n = p_1^{\alpha_1} \cdots p_{k}^{\alpha_{k}}$, it
            reduces to the previous case because
            $\mu$ is only non-zero on squarefree integers, so
            \begin{align*}
                \mu * 1 (n)
                &= \sum_{d  \mid \frac{n}{p_1^{\alpha_1-1} 
                \cdots p_k^{\alpha_k-1}}}
                \mu (d) = 0
            \end{align*}
            since the sets
            $\left\{ d \colon
            d  \mid 
        \frac{n}{p_1^{\alpha_1 - 1} \cdots
    p_k^{\alpha_k -1}}\right\} $ and
    $\left\{ d \colon
    d  \mid p_1 \cdots p_k \right\} $ are equal.

    Thus, indeed, $\mu * 1 = e$.
        \end{proof}
    \end{exercise}

    \begin{exercise}[H1.3]
        We claim that the set of arithmetic functions
        with Dirichlet convolution as a binary operation is
        an abelian semigroup.
        For this, if $f,g \colon \mathbb{N}  \to \mathbb{C}$, then
        clearly $f * g \colon \mathbb{N} \to \mathbb{C}$ too.
        Also,
        $f *g(n) = \sum_{ab = n} f(a) g(b) = 
        \sum_{ba = n} g(b) f(a) = g*f(n)$ by commutativity
        of multiplication in $\mathbb{C}$.
        Lastly,
        \[
            \left( f*g \right) *h(n)
            = \sum_{ab = n} f*g(a) h(b)
            = \sum_{ab=n} \sum_{cd = a} f(c) g(d) h(b)
            = \sum_{cdb = n} f(c) g(d) h(b)
        \] 
        and
        \[
        f * \left( g*h \right) (n)
        = \sum_{ab = n} f(a) g*h(b)
        = \sum_{ab = n} \sum_{cd = b} f(a) g(c) h(d)
        = \sum_{acd = n} f(a) g(c) h(d)
        \] 
        (all of this is just Theorem 5.1.4 in the book for Introduction to Number Theory
        by Risager).

        Now, if
        $f = 1 * g$ then
        $\mu * f = \mu * \left( 1 * g \right) 
        = \left( \mu * 1 \right) * g
        = e * g
        = g * e = g$ by the above together with
        H1.1.
        Likewise, if $g = \mu * f$, then
        $1 * g = 1 * \left( \mu * f \right) 
        = \left( 1 * \mu \right) * f
        = \left( \mu * 1 \right) * f
        = e * f = f * e = f$ again.
    \end{exercise}

    \begin{exercise}[H1.4]
        We have
        \begin{align*}
            \sum_{n=1}^{\infty} \left| 
            \frac{f(n)}{n^{s}}\right| 
            &\le 
            \sum_{n=1}^{\infty } \frac{C n^{k}}{n^{\sigma}}\\
            &\le \sum_{n=1}^{\infty}
            \frac{C}{n^{\sigma - k}}\\
            &< \infty
        \end{align*}
        as $\sigma - k > 1$. Thus the series
        converges absolutely.
    \end{exercise}

    \begin{exercise}[H1.5]
        \begin{proof}
            We know that
            $L_f$ converges absolutely for
            $\sigma > 1 + k_f$ and
            $L_g$ converges absolutely for
            $\sigma > 1 + k_g$.
            Now,
            \begin{align*}
                \sum_{n=1}^{\infty }
                \left| \frac{\sum_{d  \mid n} f(d)
                g(\frac{n}{d})}{n^{s}}
                \right| 
                &\le \sum_{n=1}^{\infty} \sum_{d \mid n}
                \frac{C_f C_g d^{k_f} 
                \left( \frac{n}{d} \right)^{k_g}}{n^{\sigma}}\\
                &= \sum_{n=1}^{\infty} C \frac{1}{n^{\sigma -
                k_g}}
            \end{align*}
            and by symmetric, likewise for a different
            constant  $C$ with $k_f$ replacing $k_g$.
            Hence the sum defining
             $L_{f*g}(s)$ is absolutely convergent
             for $\sigma > \min \left\{ 1+k_f, 1+k_g \right\} $,
             and in this half-plane, 
             \[
             L_f(s) L_g(s) = 
             \lim_{n \to \infty}
             \sum_{k=1}^{n} \sum_{t=1}^{n} \frac{f(k)}{k^{s}}
             \frac{g(t)}{t^{s}}
             = \lim_{n \to \infty} 
             \sum_{k=1}^{n} \sum_{d  \mid k} 
             \frac{f(d) g(\frac{n}{d})}{k^{s}}
             = L_{f*g}(s)
             \] 
        \end{proof}
    \end{exercise}

    \begin{exercise}[H1.6]
        We have
        $L_{1}(s) L_{\mu}(s) = 
        L_{1 * \mu}(s) =
        L_{e}(s)
        = 1$, but
        $L_1(s) = \zeta(s)$ and
        $L_{\mu}(s) = \sum_{n=1}^{\infty} \frac{\mu(n)}{n^{s}}$, 
        so the result follows from H1.5 since
        $1(n) = O(1)$ and
        $\mu (n) = O(1)$, so
        $L_{1*\mu}$ is absolutely convergent
        for $\sigma > 1$ with 
        $L_{1} (s) L_{\mu}(s) = L_{1 * \mu}(s)$.
    \end{exercise}


    \begin{exercise}[H 1.7]
        \begin{proof}
            For $f(n) = n^{w}$, we have
            $\sigma_w (n) = f * 1 (n)$.
            The abscissa of convergence for
            $1$ is $1$ and for $f$ it is
            $1 + \Re (w)$. For  $\sigma > 
            \min \left\{ 1, 1 + \Re (w) \right\} $, we have
            $\sum_{n=1}^{\infty} \frac{\sigma_w (n)}{n^{s}}
            = L_{\sigma_w}(s)
            = L_{f}(s) L_{1}(s)$. Now
            $L_1 (s) = \zeta(s)$, and
            \[
            L_f(s) = 
            \sum_{n=1}^{\infty} \frac{n^{w}}{n^{s}}
            = \sum_{n=1}^{\infty}
            \frac{1}{n^{s-w}}
            = \zeta(s-w).
            \] 
            Thus
            $\sum_{n=1}^{\infty} \frac{\sigma_w(n)}{n^{s}}
            = \zeta(s-w) \zeta(s)$
        \end{proof}
    \end{exercise}





    %\bibliography{../refs.bib}
\end{document}
