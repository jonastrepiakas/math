\documentclass[reqno]{amsart}
\usepackage{amscd, amssymb, amsmath, amsthm}
\usepackage{graphicx}
\usepackage[colorlinks=true,linkcolor=blue]{hyperref}
\usepackage[utf8]{inputenc}
\usepackage[T1]{fontenc}
\usepackage{textcomp}
\usepackage{babel}
%% for identity function 1:
\usepackage{bbm}
%%For category theory diagrams:
\usepackage{tikz-cd}

%\usepackage[backend=biber]{biblatex}
%\addbibresource{.bib}


\setlength\parindent{0pt}

\pdfsuppresswarningpagegroup=1

\newtheorem{theorem}{Theorem}[section]
\newtheorem{lemma}[theorem]{Lemma}
\newtheorem{proposition}[theorem]{Proposition}
\newtheorem{corollary}[theorem]{Corollary}
\newtheorem{conjecture}[theorem]{Conjecture}

\theoremstyle{definition}
\newtheorem{definition}[theorem]{Definition}
\newtheorem{example}[theorem]{Example}
\newtheorem{exercise}[theorem]{Exercise}
\newtheorem{problem}[theorem]{Problem}
\newtheorem{question}[theorem]{Question}

\theoremstyle{remark}
\newtheorem*{remark}{Remark}
\newtheorem*{note}{Note}
\newtheorem*{solution}{Solution}



%Inequalities
\newcommand{\cycsum}{\sum_{\mathrm{cyc}}}
\newcommand{\symsum}{\sum_{\mathrm{sym}}}
\newcommand{\cycprod}{\prod_{\mathrm{cyc}}}
\newcommand{\symprod}{\prod_{\mathrm{sym}}}

%Linear Algebra

\DeclareMathOperator{\Span}{span}
\DeclareMathOperator{\im}{im}
\DeclareMathOperator{\diag}{diag}
\DeclareMathOperator{\Ker}{Ker}
\DeclareMathOperator{\ob}{ob}
\DeclareMathOperator{\Hom}{Hom}
\DeclareMathOperator{\Mor}{Mor}
\DeclareMathOperator{\sk}{sk}
\DeclareMathOperator{\Vect}{Vect}
\DeclareMathOperator{\Set}{Set}
\DeclareMathOperator{\Group}{Group}
\DeclareMathOperator{\Ring}{Ring}
\DeclareMathOperator{\Ab}{Ab}
\DeclareMathOperator{\Top}{Top}
\DeclareMathOperator{\hTop}{hTop}
\DeclareMathOperator{\Htpy}{Htpy}
\DeclareMathOperator{\Cat}{Cat}
\DeclareMathOperator{\CAT}{CAT}
\DeclareMathOperator{\Cone}{Cone}
\DeclareMathOperator{\dom}{dom}
\DeclareMathOperator{\cod}{cod}
\DeclareMathOperator{\Aut}{Aut}
\DeclareMathOperator{\Mat}{Mat}
\DeclareMathOperator{\Fin}{Fin}
\DeclareMathOperator{\rel}{rel}
\DeclareMathOperator{\Int}{Int}
\DeclareMathOperator{\sgn}{sgn}
\DeclareMathOperator{\Homeo}{Homeo}
\DeclareMathOperator{\SHomeo}{SHomeo}
\DeclareMathOperator{\PSL}{PSL}
\DeclareMathOperator{\Bil}{Bil}
\DeclareMathOperator{\Sym}{Sym}
\DeclareMathOperator{\Skew}{Skew}
\DeclareMathOperator{\Alt}{Alt}
\DeclareMathOperator{\Quad}{Quad}
\DeclareMathOperator{\Sin}{Sin}
\DeclareMathOperator{\Supp}{Supp}
\DeclareMathOperator{\Char}{char}
\DeclareMathOperator{\Teich}{Teich}
\DeclareMathOperator{\GL}{GL}
\DeclareMathOperator{\tr}{tr}
\DeclareMathOperator{\codim}{codim}
\DeclareMathOperator{\coker}{coker}
\DeclareMathOperator{\Diff}{Diff}
\DeclareMathOperator{\Bun}{Bun}
\DeclareMathOperator{\Sm}{Sm}



%Row operations
\newcommand{\elem}[1]{% elementary operations
\xrightarrow{\substack{#1}}%
}

\newcommand{\lelem}[1]{% elementary operations (left alignment)
\xrightarrow{\begin{subarray}{l}#1\end{subarray}}%
}

%SS
\DeclareMathOperator{\supp}{supp}
\DeclareMathOperator{\Var}{Var}

%NT
\DeclareMathOperator{\ord}{ord}

%Alg
\DeclareMathOperator{\Rad}{Rad}
\DeclareMathOperator{\Jac}{Jac}

%Misc
\newcommand{\SL}{{\mathrm{SL}}}
\newcommand{\mobgp}{{\mathrm{PSL}_2(\mathbb{C})}}
\newcommand{\id}{{\mathrm{id}}}
\newcommand{\MCG}{{\mathrm{MCG}}}
\newcommand{\PMCG}{{\mathrm{PMCG}}}
\newcommand{\SMCG}{{\mathrm{SMCG}}}
\newcommand{\ud}{{\mathrm{d}}}
\newcommand{\Vol}{{\mathrm{Vol}}}
\newcommand{\Area}{{\mathrm{Area}}}
\newcommand{\diam}{{\mathrm{diam}}}
\newcommand{\End}{{\mathrm{End}}}


\newcommand{\reg}{{\mathtt{reg}}}
\newcommand{\geo}{{\mathtt{geo}}}

\newcommand{\tori}{{\mathcal{T}}}
\newcommand{\cpn}{{\mathtt{c}}}
\newcommand{\pat}{{\mathtt{p}}}

\let\Cap\undefined
\newcommand{\Cap}{{\mathcal{C}}ap}
\newcommand{\Push}{{\mathcal{P}}ush}
\newcommand{\Forget}{{\mathcal{F}}orget}




\begin{document}



\section{Theory}


    Recall that
    \begin{definition}[Dirichlet Series]
        Let $f$ be an arithmetic function. Then the corresponding
        Dirichlet series is defined, for
        $s \in \mathbb{C}$, by
        \[
        D_f (s) = \sum_{n=1}^{\infty} \frac{f(n)}{n^{s}}.
        \] 
    \end{definition}


    \begin{lemma}[]\label{Lemma:02329}
        \[
        0 \le 3 + 4 \cos \theta + \cos 2\theta = 
        2 \left( 1+ \cos \theta \right)^2
        \] 
    \end{lemma}

    \begin{lemma}[]
        Let $\sigma > 1$. Then
        \[
        \Re \left( -3 \frac{\zeta '}{\zeta} (\sigma)-
        4 \frac{\zeta'}{\zeta}(\sigma + it)-
    \frac{\zeta'}{\zeta}(\sigma + 2 i t)\right) \ge 0
        \] 
    \end{lemma}

    For the proof of the lemma, one shows that
    \[
    \Re \left( \frac{1}{n^{s}} \right) =
    \frac{1}{n^{\sigma}} \cos \left( t \log n \right) , \quad
    s = \sigma + it \tag{$A_1$} \label{A-1}
    \] 

    \begin{proof}
        \[
        \Re \left( - \frac{\zeta'}{\zeta}(s) \right) 
        = \Re \sum_{n=1}^{\infty} \frac{\Lambda (n)}{n^{s}}
        = \sum_{n=1}^{\infty} \frac{\Lambda(n)}{n^{\sigma}}
        \cos \left( t \log n \right) .
        \] 
        Hence
        \begin{align*}
        \Re \left( -3 \frac{\zeta '}{\zeta} (\sigma)-
        4 \frac{\zeta'}{\zeta}(\sigma + it)-
    \frac{\zeta'}{\zeta}(\sigma + 2 i t)\right) 
    &= \sum_{n=1}^{\infty} \frac{\Lambda(n)}{n^{\sigma}}
    \left[ 3 + 4 \cos \left( t \log n \right) +
    \cos \left( 2t \log n \right) \right] 
    \stackrel{\eqref{Lemma:02329}}{\ge} 0
        \end{align*}
    \end{proof}


\newpage

\section{Week 1}

\begin{exercise}[E1.1. Abel summation]
    Let $\left\{ a_n \right\}_{n \in \mathbb{N} }\subset \mathbb{C}$ 
    and $f \colon \left[ 1,x \right] \to \mathbb{C}$ be
    $C^{1}$. Define $A(t) = \sum_{n \le t} a_n$. Then for
    $x>1$, we have
    \[
    \sum_{n\le x} a_n f(n) = A(x) f(x) -
    \int_{1}^{x} A(t) f'(t) dt. 
    \] 
\end{exercise}



\section{Week 2}

Let $\psi (x) :=
\sum_{n \le x} \Lambda (n)$.

\begin{exercise}[E2.6]
    Show that
    \[
    \theta(x) := 
    \sum_{p \le x} \log p = \psi (x) + O\left( x^{\frac{1}{2}}
    \log^2 x \right) 
    \] 
\end{exercise}

\begin{exercise}[E2.7]
    Show that
    \[
    \pi(x) = \frac{\psi (x)}{\log x}+ 
    \int_{2}^{x} \frac{\psi (t)}{t \log^2t}dt +
    O\left( x^{\frac{1}{2}} \log x \right) .
    \] 
\end{exercise}

\begin{proof}
    By Abel summation, we first find that
    \[
    \theta(x) :=
    \sum_{p \le x} \log p = 
    \pi(x) \log x - \int_{2}^{x} \frac{\pi(t)}{t}dt
    \] 
    and from the previous exercise, we now find
    that
    \[
    \pi(x) = \frac{\psi (x)}{\log x} +
    \frac{1}{\log x} \int_{2}^{x}\frac{\pi(t)}{t}dt 
    + O \left( x^{\frac{1}{2}} \log x \right) 
    \] 
    The result follows if we
    can show that
    \[
    \frac{1}{\log x} \int_{2}^{x} \frac{\pi(t)}{t}dt
    = \int_{2}^{x} \frac{\psi (t)}{t \log^2 t}dt 
    + O\left( x^{\frac{1}{2}} \log x \right).
    \] 
    Now $\psi (t) \le \pi(t) \log t$, so
    \begin{align*}
        \left| \int_{2}^{x} 
        \frac{\psi (t)}{t \log^2 t} - 
        \frac{\pi(t)}{t \log x}dt \right| 
        &\le 
        \left| \int_{2}^{x} \frac{\pi(t)}{t \log t} -
        \frac{\pi(t)}{t \log x} dt\right| \\
        &= \left| 
        \int_{2}^{x} \frac{\pi(t)}{t} 
        \frac{\log \left( \frac{x}{t} \right) }{\log x 
        \log t} dt\right| \\
    \end{align*}
\end{proof}

\section{Week 3}

\begin{exercise}[E3.1]
    Let $m \ge 0$. Show that
    \[
    \sum_{n \le x} \log^{m} n = 
    x \log^{m} x + O \left( x \log^{m-1} x \right) .
    \] 
\end{exercise}

\begin{proof}
    Let $a_n = 1$ for all $n$. Then
    $A(x) = \left\lfloor x \right\rfloor$, so
    \begin{align*}
    \sum_{n \le x} \log^{m}n 
    &= 
    \left\lfloor x \right\rfloor \log^{m} x
    - \int_{1}^{x} m \left\lfloor t \right\rfloor \frac{1}{t}
    \log^{m-1} t dt\\
    &= x \log^{m}x
    - \left( x - \left\lfloor x \right\rfloor  \right) \log^{m} x
    -m \int_{1}^{x} \frac{\left\lfloor  t \right\rfloor }{t}
    \log^{m-1}(t) dt
    \end{align*}
    Thus we must show that
    \begin{align*}
    \left| \left( x - \left\lfloor x \right\rfloor  \right) \log^{m} x
    +m \int_{1}^{x} \frac{\left\lfloor  t \right\rfloor }{t}
    \log^{m-1}(t) dt  \right| \le 
    C x \log^{m-1}x
    \end{align*}
    But $\frac{\left\lfloor  t \right\rfloor }{t}
    \log^{m-1} (t) \le \log^{m-1}(x)$ giving that
    the right hand term is $O\left( x \log^{m-1} x \right) $.
    For the left hand term,
    it suffices to show that
    $\left( x - \left\lfloor x  \right\rfloor  \right) 
    \log x \le x$, but this is clear
    since $x - \left\lfloor x \right\rfloor \le 1$ and
    $ \log x  \le x$.
\end{proof}

\begin{exercise}[E3.2]
    Let $d(n) = \sum_{d  \mid n} 1$. Show
    $d(n) \le 2 \sqrt{n} $.
    If we consider the set $D \subset \mathbb{N} $ of positive
    divisors of $n$, then we can define a bijection
    $D \to D$ by
    $k \mapsto \frac{n}{k}$. Suppose
    now that $d(n) > 2 \sqrt{n} $. 
    Suppose $d  \mid n$ and
    $d \ge  \sqrt{n} $. Then
    since $\frac{d}{n} \cdot d = n$, we must have
    $\frac{d}{n} \le  \sqrt{n} $. This implies that
    under this bijection, either the source or
    target lies in
    $\left\{ 1,\ldots, \left\lfloor \sqrt{n}  \right\rfloor \right\} $.
    Hence $d(n) = \left| D \right| \le 
    2 \left| \left\{ 1, \ldots,
    \left\lfloor \sqrt{n}  \right\rfloor \right\}  \right| 
    \le 2 \sqrt{n} $.
\end{exercise}

\begin{exercise}[E3.3]
    Prove that for every $\varepsilon > 0$, there
    exists a constant $C_{\varepsilon}$ such that
    $d(n) \le C_{\varepsilon} n^{\varepsilon}$.\\
    Hint:
    \begin{enumerate}
        \item Show that $d(n_1n_2) = d(n_1)d(n_2)$ if
            $\left( n_1,n_2 \right) =1$.
        \item Show that
            \[
            \frac{d(n)}{n^{\varepsilon}}
            = \prod_{p^{\alpha} \mid  \mid   n} \frac{\alpha+1}{
            p^{\alpha \varepsilon}}
            \] 
            where $p^{\alpha} \mid  \mid n$ means that
            $\alpha$ is a positive integer,
            $p^{\alpha} \mid n$ and
            $p^{\alpha+1}\not| n$.
        \item Split the product in $2$. Into the product
            over those primes $p < 2^{\frac{1}{\varepsilon}}$ and
            the product over the rest. Show that the
            second product is bounded by $1$.
        \item Show that the factors in the first product
            are less than $1 + \left( \varepsilon \log 2 \right)^{-1}$.
    \end{enumerate}
\end{exercise}

\begin{proof}
    We follow the hint:\\
    (1) Suppose $\left( n_1,n_2 \right) =1$. 
    Let $D$ be the set of divisors of $n_1n_2$, 
    $D_1 $ the set of divisors of $n_1$ and
    $D_2$ the set of divisors of $n_2$.
    Suppose $d_1 \in D_1, d_2 \in D_2$.
    Then $d_1 a = n_1, d_2 b = n_2$, so
    $d_1d_2 ab = n_1n_2$, hence
    $d_1d_2 \in D$. We thus obtain a map
    $D_1 \times D_2 \to D$ sending
    $\left( d_1,d_2 \right) \mapsto d_1d_2$. We claim this is
    a bijection. Suppose
    $d_1d_2 = d_1' d_2'$.
    If $d_1  \mid d_2'$, then
    $d_1 = 1$, in which case,
    $d_1' = 1$, and thus $d_2 = d_2'$.
    Suppose thus that $d_1 \neq 1$, so
    $d_1 \not| d_2'$.
    Then since $\left( d_1', d_2' \right) =1$, we have
    $d_1  \mid d_1'$. Similarly,
    $d_1'  \mid d_1$. So 
    $d_1 = d_1'$. And again $d_2 = d_2'$. This gives injectivity.
    For surjectivity, if
    $d  \mid n_1n_2$, then consider
    $d_1 := \frac{d}{\left( n_2,d \right) }$ and
    $d_2 := \frac{d}{\left( n_1,d \right) }$.
    Then $d_1 d_2 = d$ and $d_1 \in D_1, d_2 \in D_2$.\\
    (2) Clearly,
    $n^{\varepsilon} = \prod_{p^{\alpha} \mid  \mid 
    n} p^{\alpha \varepsilon}$. It thus suffices to show that
    $\prod_{p^{\alpha} \mid  \mid n} \left( \alpha+1 \right) 
    = d(n)$. But if we factorize $n$ as
    $n = p_1^{\alpha_1} \cdots p_m^{\alpha_m}$, then
    it is clear that the divisors corresponds precisely to
    tuples
    $\left( a_1, \ldots, a_m \right) $ with
    $0 \le a_i \le \alpha_i$. There
    are precisely $\alpha_1 + 1$ choices for each
    $a_i$, giving
    $\left( \alpha_1 +1 \right) \cdots
    \left( \alpha_m +1 \right) = d(n)$ which indeed
    is what we wanted to show.\\
    (3) We can split the product as
    \[
    \frac{d(n)}{n^{\varepsilon}} = 
    \underbrace{\prod_{\substack{p^{\alpha} \mid  \mid n \\ p <
    2^{\frac{1}{\varepsilon}}}}
\frac{\alpha+1}{p^{\alpha \varepsilon}}}_{A}
    \cdot 
    \underbrace{\prod_{\substack{p^{\alpha} \mid  \mid n \\ p \ge 
    2^{\frac{1}{\varepsilon}}}}
\frac{\alpha+1}{p^{\alpha \varepsilon}}}_{B}
    \] 
    We claim that $B \le 1$. Indeed

    \[
    \prod_{\substack{p^{\alpha} \mid  \mid n \\ p \ge 
    2^{\frac{1}{\varepsilon}}}}
    \frac{\alpha+1}{p^{\alpha \varepsilon}}
    \le 
    \prod_{\substack{p^{\alpha} \mid  \mid n \\ p \ge 
    2^{\frac{1}{\varepsilon}}}}
    \underbrace{\frac{\alpha+1}{2^{\alpha}}}_{\le 1}
    \le 1
    \] 

    (4) For the factors in the first product, we have
    $\alpha = \left\lfloor \frac{\log n}{\log p}
    \right\rfloor$ and
    $\log p < \frac{1}{\varepsilon} \log 2$, and
    $\alpha \le \frac{\log n}{\log p}$, so
    $\frac{\log p}{\log n} \le \frac{1}{\alpha}$
    \[
    \varepsilon^2 \log p < \varepsilon \log 2
    \] 


    \[
        \frac{\alpha+1}{p^{\alpha \varepsilon}}
        \le \frac{\log n +\log p}{p^{\alpha \varepsilon} \log p}
        \le 1 + \frac{1}{\varepsilon \log 2} = 
        \frac{\varepsilon \log 2 + 1}{\varepsilon \log 2}
    \] 

    What we want to bound is
    \[
\prod_{\substack{p^{\alpha} \mid  \mid n \\ p <
    2^{\frac{1}{\varepsilon}}}}
\frac{\alpha+1}{p^{\alpha \varepsilon}}
\] 
Note here that $p$ is bounded and
as $\alpha$ increases, we should expect the denominator to
take over. However, while $\alpha$ is small,
we might have some large terms since $p^{\varepsilon}$ might
be large. All our terms are however bounded by
$p^{\varepsilon}$ by the looks of it? Then we
would get that the product is
the product is bounded by
$\prod_{p < 2^{\frac{1}{\varepsilon}}} 
\frac{\log n}{\log p} \frac{1}{p^{\varepsilon}}$


\end{proof}

\begin{exercise}[E3.4]
    Show that
    \[
    \sum_{n=1}^{\infty} \frac{d(n)}{n^{s}}
    \] 
    is absolutely convergent in
    $\Re (s) > 1$.
\end{exercise}

\begin{proof}
    Fix some $s = \sigma + it \in \mathbb{C}$ with
    $\sigma > 1$. Then choosing an $\varepsilon > 0$ with
    $1 + \varepsilon < \sigma$, we have that
    $d(n) \le C_{\varepsilon} n^{\varepsilon}$, so
    \[
    \sum \left| \frac{d(n)}{n^{s}} \right| 
    \le \sum C_{\varepsilon} \frac{n^{\varepsilon}}{n^{\sigma}}
    \le C_{\varepsilon} \sum \frac{1}{n^{\sigma - \varepsilon}} < 
    \infty.
    \] 
\end{proof}

\begin{exercise}[E3.5]
    Show that the average order of
    $d(n)$ is $\log n$, i.e., that
    \[
    \frac{1}{x} \sum_{n \le x} d(n) = \log x +
    o\left( \log x \right) .
    \] 
    Hint: Show that
    \[
    \sum_{n \le x} d(n) = \sum_{a \le x}
    \left[ \frac{x}{a} \right] 
    \] 
    where $\left[ b \right] $ is the integer
    part of $b$.
\end{exercise}


\begin{proof}
    We follow the hint.
    For each $n \in \mathbb{N} $, let
    $D_n$ denote the set of positive divisors of
    $n$. Then we want to find
    $\left| D_1 \cup \ldots \cup D_{\left[ x \right] } \right|  $.
    Now, $\left[ \frac{x}{a} \right] $ is precisely
    the amount of multiples of $a$ smaller than or equal to
    $x$, i.e., the amount of numbers in between $1$ and
    $x$ which have $a$ as a divisor.
    Hence the right hand side indeed counts the
    number of divisors of the numbers less than or equal to
     $x$ which is precisely the left hand side.
     Now, recall also the bound
     \[
     \log x + \frac{1}{x} \le \sum_{a \le x}\frac{1}{a}
     \le \log x + 1
     \] 
     so \[
     1 + \frac{1}{x \log x} \le \frac{1}{\log x}
     \sum_{a \le x} \frac{1}{a} \le 1 + \frac{1}{\log x}.
     \] 
     In particular, taking the limit as
     $x \mapsto \infty$, the outer
     functions tend to $1$, so
     \[
     \lim_{x \to \infty} \frac{1}{\log x} \sum_{a \le x}\frac{1}{x}
     = 1.
     \] 
     In particular,
     \[
     \frac{1}{x \log x}\sum_{n\le x} d(n)
     \le \frac{1}{\log x} \sum_{a \le x} \frac{1}{a}
     \to 1, \quad x \to \infty.
     \] 
     For a lower bound, we have
     \[
     \frac{1}{\log x}  
     \sum_{a\le x} \frac{1}{a} -
     \frac{1}{x\log x} \sum_{a\le x}\frac{1}{a}
     = \frac{1}{\log x} \sum_{a \le x} \frac{x-1}{ax}
     \le \frac{1}{\log x} \sum_{a\le x} \left[ \frac{x}{a} \right] 
     \] 
     But
     \[
     \frac{1}{x} + \frac{1}{x^2 \log x}
     \le \frac{1}{x \log x} \sum_{a\le x} \frac{1}{a}
     \le \frac{1}{x} + \frac{1}{x \log x}
     \] 
     so letting $x \to \infty$, 
     \[
     \lim_{x \to \infty} \frac{1}{x \log x} \sum_{a \le x}
     \frac{1}{a} = 0
     \] 
     Hence also
     \[
     1 \le \lim_{x \to \infty} \frac{1}{x \log x}
     \sum_{n \le x} d(n) \le 1
     \] 
     giving the desired result.
\end{proof}

\begin{exercise}[E3.6]
    Let  \[
    \chi_4 (n) = 
    \begin{cases}
        (-1)^{\frac{n-1}{2}},& n \text{ odd}\\
        0,& n \text{ even}
    \end{cases}.
    \] 
    Show that $\chi_4$ is a Dirichlet character
    modulo $4$ and find
    $L\left( 1, \chi_4 \right) $. Use the value
    to give (yet another) proof- based on the
    irrationality of $\pi$ - that there
    are infinitely many primes.
    Hint: Remember (or prove by playing around with
    $\arctan (1)$ ) that
    \[
    \pi = 4 \sum_{n=1}^{\infty} \frac{(-1)^{n-1}}{2n-1}.
    \] 
\end{exercise}

\begin{proof}
    We must check $3$ criteria for
    $\chi_4$ to be a Dirichlet character mod $4$.\\
    (i) It must be $4$-periodic. Now
    if $n$ is even, then
    $n+4$ is even, so then
    $\chi_4(n+4) = 0 = \chi_4 (n)$.\\
    If $n$ is odd, then so is $n+4$, so
    \[
    \chi_4(n+4) = 
    \left( -1 \right)^{\frac{n+4-1}{2}}
    = \left( -1 \right)^{\frac{n-1}{2} + 2}
    = \left( -1 \right)^{\frac{n-1}{2}}
    = \chi_4 (n).
    \] 
    So $\chi_4$ is $4$-periodic.\\
    (ii) We must check that
    $\chi_4(n) = 0$ if and only if
    $\left( n,4 \right) \neq 1$.
    Now, $\chi_4(n) = 0$ if and only if $n$ is even
    if and only if $\left( n,4 \right) \in 
    \left\{ 2,4 \right\} $ if and only if
    $\left( n,4 \right) \neq 1$.\\
    (iii) We must check that
    $\chi_4$ is multiplicative.
    Indeed, if either
    $n$ or $m$ is even, then
    \[
    \chi_4 (nm) = 0 = \chi(n) \chi(m).
    \] 
    If both $n,m$ are odd, then
    \[
    \chi_4 \left( nm \right) 
    = \left( -1 \right)^{\frac{nm-1}{2}}
    = 
    \begin{cases}
        -1,& nm \equiv 3 \pmod{4}\\
        1,& nm \equiv 1 \pmod{4}
    \end{cases}
    \] 
    Now, if $n$ and $m$ are both
    equivalent to $3$ mod $4$, then
    their product is equivalent to
    $1$ mod 4, which works out.
    If only one is equivalent to $3$ mod $4$, then
    $nm$ is also, so it checks out, and
    similarly, if both are equivalent to $1$ mod $4$, then
    so is their product.
    Now, by definition,
    \[
    L(1,\chi_4) :=
    \sum_{n=1}^{\infty} \frac{\chi_4(n)}{n}
    = \sum_{n=0}^{\infty} \frac{\left( -1 \right)^{n}}{2n+1}
    = \arctan(1)
    = \frac{\pi}{4}
    \] 
    Now,
    since  $\chi_4 \neq \chi_0^{4}$, 
    we know that
    $L\left( s, \chi_4 \right) $ is convergent and
    analytic for $\Re (s) > 0$. In particular, it is
    continuous at $s = 1$. But for
    $\Re(s) > 1$, we know that
    $L(s, \chi_4) = 
    \prod_p \left( 1 - \chi_4(p)p^{-s} \right)^{-1}$, so
    by continuity,
    \[
    \frac{\pi}{4} = L\left( 1,\chi_4 \right) 
    = \prod_p \left( 1- \chi_4(p) p^{-1} \right)^{-1}
    \] 
    Now, all the terms in the product are
    rational, so by irrationality of $\pi$, this
    forces there to be infinitely
    many primes.
\end{proof}


\begin{exercise}[E3.7]
    Let $\left\{ a_n \right\} $ be a sequence of complex numbers
    satisfying that $\sum_{n \le x} a_n = O\left( x^{\delta}
    \right) $ for some $\delta > 0$. Prove that
    \[
    \sum_{n=1}^{\infty} \frac{a_n}{n^{s}} = s
    \int_{1}^{\infty} \sum_{n\le t} a_n \frac{1}{t^{s+1}} dt 
    \] 
    for $\Re(s) > \delta$, and that the sum converges
    to an analytic function in this region.
\end{exercise}

\begin{proof}
    Let $f(x) = x^{s}$. Then
    \[
    \sum_{n\le x} \frac{a_n}{n^{s}}
    = \sum_{n\le x}a_n \frac{1}{x^{s}} + s
    \int_{1}^{x} \sum_{n\le t} a_n \frac{1}{t^{s-1}} dt
    \] 
    when $s \neq 1$. But
    $\left| \sum_{n \le x}a_n \right| 
    \le C x^{\delta} $, so
    \[
    \left| \sum_{n\le x} a_n \frac{1}{x^{s}} \right|
    \le C x^{\delta - \sigma} \to 0, \quad x \to \infty
    \] 
    as $\delta - \sigma < 0$.
    Thus
    \[
    \sum_{n=1}^{\infty} \frac{a_n}{n^{s}} = 
    s \int_{1}^{\infty} \sum_{n\le t} a_n \frac{1}{t^{s+1} }dt . 
    \] 
\end{proof}


\section{Week 4}

\begin{exercise}[E4.1]
    Let $K \ge 0$. Prove that
    \[
    \log \left( K \left| t \right| +4 \right) =
    O\left( \log \left( \left| t \right| +4 \right)  \right) 
    \] 
    for $t \in \mathbb{R}$. Let
    $c_1, c_2, c_3 > 0$. Prove that there exists a constant
    $c_4$ such that for all $t \in \mathbb{R}$,
    \[
    c_1 + c_2 \log\left( \left| t \right| +4 \right) +
    c_3 \log\left( \left| 2t \right| +4 \right) 
    \le c_4 \log \left( \left| t \right| +4 \right) .
    \] 
\end{exercise}

\begin{proof}
    If $0 \le K \le 1$, then
    $\log \left( K \left| t \right| +4 \right) \le 
    \log \left( \left| t \right| +4 \right) $ by
    monotonicity of $\log$.
    So assume  $K > 1$. Then
    $\log \left( K \left| t \right| +4 \right) 
    = \log K + \log \left( \left| t \right|  + \frac{4}{K} \right) 
    \le \log K + \log \left( \left| t \right| +4 \right) $.
    Now $\log \left( \left| t \right| +4 \right) > 1$, so
    there exists some $C$ such that
    $C \log \left( \left| t \right| +4 \right) \ge \log K$.
    Hence
    $\log \left( K \left| t \right| +4 \right) =
    O \left( \log \left( \left| t \right| +4 \right)  \right) $.
    Since
    $c_1 + c_2 \log\left( \left| t \right| +4 \right) +
    c_3 \log \left( \left| 2t \right| +4 \right) $ is a
    sum of terms that are all
    $O\left( \log \left( \left| t \right| +4 \right)  \right) $, so
    is their sum, so the conclusion holds.
\end{proof}


\begin{exercise}[E4.2]
    Let $f(s)$ be a complex polynomial of degree $n$ with
    complex zeroes $z_1, z_2, \ldots, z_n$. Show that
    \[
    \frac{f'}{f}(z) = \sum_{i=1}^{n} \frac{1}{z- z_i}.
    \] 
    Consider how Lemma 6.3 is a generalization of this.
\end{exercise}

\begin{proof}
    Firstly, $f'$ is entire, so
    $\frac{f'}{f}$ is holomorphic on 
    $\mathbb{C} - \left\{ z_1, \ldots, z_n \right\} $.
    Now, by Theorem 6.1 in KomAn, there
    exist unique functions
    $g_i$ holomorphic on
    $\mathbb{C} - \left\{ z_1,\ldots,z_n \right\} $ such that
    $g_i(z_i) \neq 0$ and
     \[
     f(z) = \left( z-z_i \right)^{n_i} g_i(z)
     \] 
     where $n_i$ is the multiplicity of $z_i$.
     In particular,
     $f'(z) = 
     n_i (z-z_i)^{n_i-1}g_i(z) + (z-z_i)^{n_i} g_i'(z)$ which
     has $z_i$ a zero of order
     $n_i -1$. Hence
     $\frac{f'}{f}$ has $z_i$ as a simple pole.
     Applying the partial fraction decomposition
     to $\frac{f'}{f}$ (theorem 6.12 in KomAn), we get
     that
     \[
     \frac{f'}{f}(z) = 
     \sum_{i=1}^{n} \frac{c_i}{z- z_i}
     \] 
     for certain constants $c_i$.
     Now $\lim_{z \to z_i}  (z-z_i) \frac{f'}{f}(z)
     = n_i$. 
     Now, $f$ is of degree $n$ with $n$ distinct zeroes, so
     $n_i$ must be $1$ for each $i$.\\
     
     Now let us recall Lemma 6.3:
     \begin{lemma}[6.3]
         Let $f \colon B \to \mathbb{C}$ be analytic,
         $B \subset \mathbb{C}$ open, and assume
         \begin{enumerate}
             \item $\left\{ z  \mid \left| z \right| \le 1
                 \right\} \subset B$ 
             \item $\left| f(z) \right| \le M$ when 
                 $\left| z \right| \le 1$ 
             \item $f(0) \neq 0$.
         \end{enumerate}
         Let $0 < r < R < 1$. Then for
         $\left| z \right| < r$,
         \[
         \frac{f'}{f}(z) = 
         \sum_{\substack{f(z_k) = 0 \\ \left| z_k \right| \le R}} 
         \frac{1}{z-z_k} + 
         O \left( \log \frac{M}{\left| f(0) \right| } \right) 
         \] 
     \end{lemma}

         Note here that $f$ is not required to be
         a polynomial. However, since $f$ is holomorphic
         in $B$, it has an analytic representation on $B$, so
         essentially, Lemma 6.3 generalizes the
         representation to analytic functions. 
\end{proof}


\begin{exercise}[E4.3]
    Show that the Riemann zeta function $\zeta (s)$ has no
    zeroes for $\frac{1}{2} \le s < 1$.
\end{exercise}

\begin{proof}
    Recall that for $\sigma > 0$ and $s\neq 1$, we have
    \[
    \zeta (s) = \frac{s}{s-1} - s \int_{1}^{\infty} 
    \left( u - \left[ u \right]  \right) u^{-s-1}du.
    \] 
    For $s \in [\frac{1}{2},1)$, 
    $\frac{s}{s-1} \le  -1$. So we
    wish to show that
    \[
    s \int_{1}^{\infty} \left( u - \left[ u \right]  \right) 
    u^{-s-1} du > -1
    \] 
    But
    \[
        s \int_{1}^{\infty} 
        \left( u - \left[ u \right]  \right) u^{-s-1}du
    \] 
    is positive since the inner function and $s$ are
    both positive on $[1, \infty)$.

\end{proof}

\begin{exercise}[E4.4]
    Let $\chi$ be a Dirichlet character modulo $q$. Find the
    Dirichlet series representation for
    $L'(s, \chi) / L(s,\chi)$. Let $\chi_0$ be the trivial
    Dirichlet character modulo $q$. Prove that for
    $\sigma > 1, t \in \mathbb{R}$,
    \[
    R := \Re \left( -3 \frac{L'(\sigma, \chi_0)}{L(\sigma, \chi_0)}-
    4 \frac{L' (\sigma + it , \chi)}{L(\sigma + it, \chi)}-
\frac{L' (\sigma + i 2t, \chi^2) }{L(\sigma + i 2 t, \chi^2)}\right) 
\ge 0.
    \] 
\end{exercise}


\begin{proof}
    We want to represent
    $\frac{L' (s, \chi)}{L (s, \chi)}$ as a Dirichlet series.
    We imitate the idea for $\frac{\zeta'}{\zeta}$.

    \begin{align*}
        \frac{L'(s, \chi)}{L (s, \chi)}
        &= \frac{d}{ds} \log \left( L (s, \chi) \right) \\
        &=  - \sum_p \frac{d}{ds}
        \log \left( 1- \frac{\chi(p)}{p^{s}} \right) \\
        &= - \sum_p \frac{d}{ds} 
        \sum_{k=1}^{\infty} (-1)^{k+1} \left( - \frac{\chi(p)}{p^{s}}
        \right)^{k}\\
        &= \sum_p \sum_{k=1}^{\infty} \frac{d}{ds} 
        \left( \frac{\chi (p)}{p^{s}} \right)^{k}\\
        &= \sum_p \sum_{k=1}^{\infty} \chi(p)^{k} (-k \log p)
        p^{-sk}\\
        &= - \sum_p \sum_{k=1}^{\infty} k \log p \left( 
        \frac{\chi(p)}{p^{s}} \right)^{k}
    \end{align*}
    Thus
    We want to find 
    $\Re \left( \left( \frac{\chi(p)}{p^{s}} \right)^{k} \right) $.
    We have
    \begin{align*}
      \Re \left( \left( \frac{\chi(p)}{p^{s}} \right)^{k} \right) 
      &= \frac{1}{2} \left[ \left( \frac{\chi(p)}{p^{s}} \right)^{k} 
      + \left( \frac{\overline{\chi(p)}}{\overline{p^{s}}} \right)^{k}
  \right] \\
      &= 
    \end{align*}
    
    \[
    \Re \left( - \frac{L'(s,\chi)}{L(s,\chi)} \right) 
    = \sum_p \sum_{k=1}^{\infty} k \log p 
    \cos \left( tk \log p \right) .
    \] 
    So

\end{proof}


\begin{exercise}[E4.5]
    Let $\zeta(s)$ be the Riemann zeta function. Let
    $K$ be a compact subset of 
    $\left\{ s \in \mathbb{C}  \mid  
    \Re (s) > 0 \right\} $. Assume that
    $1 \in K$ and that $K$ does not contain any zeroes
    of $\zeta$. Show that
    \[
    - \frac{\zeta'}{\zeta}(s) = \frac{1}{s-1} + O(1)
    \] 
    for $s \in K - \left\{ 1 \right\} $. Show that there
    exists a constant $c >0$ such that for
    $0 < \delta < 1$,
    \[
    - \frac{\zeta'}{\zeta}(1+\delta) < \frac{1}{\delta} + c.
    \] 
\end{exercise}

\begin{proof}
    Since $1$ is a simple pole of
    $\frac{\zeta'}{\zeta}$ and 
    $K$ has no other zeroes of $\zeta$ and hence
    neither of $\zeta'$, we have that
    \[
    - (s-1) \frac{\zeta'}{\zeta}(s)
    \] 
    is holomorphic on $K$, hence bounded as $K$ is compact.
    Thus
    \[
    - \frac{\zeta'}{\zeta}(s) = \frac{1}{s-1} + O(1)
    \] 
    for $s \in K -\left\{ 1 \right\} $. Thus for small
    $0 < \delta < 1$ such that
    $1 + \delta \in K - \left\{ 1 \right\} $,
    \[
    - \frac{\zeta'}{\zeta}\left( 1+ \delta \right) 
    < \frac{1}{\delta} + c
    \] 
    for some $c > 0$.
\end{proof}


\begin{exercise}[E4.6]
    Use partial summation (Abel summation)
    to show that for $\sigma > 1$,
    \[
    - \frac{\zeta'}{\zeta}(s) = 
    s \int_{1}^{\infty} \frac{\psi (x)}{x^{s+1}} dx 
    \] 
    where $\psi (x) = \sum_{n\le x}
    \Lambda (n)$, and $\Lambda$ is the von Mangoldt function.
\end{exercise}

\begin{proof}
    Recall that
    \[
    - \frac{\zeta'}{\zeta}(s) = 
    \sum_{n=1}^{\infty} \frac{\Lambda (n)}{n^{s}}
    \] 
    for $ \sigma = \Re (s) > 1$.

    Let $f(x) = \frac{1}{x^{s}}$ and
    $a_n = \Lambda (n)$.
    Partial summation gives
    \[
    \sum_{n\le x} \frac{\Lambda (n)}{n^{s}}
    = \underbrace{\sum_{n\le x} \Lambda(n)}_{\psi (x)}
    \frac{1}{x^{s}}
    +s \int_{1}^{x}  \underbrace{\sum_{n\le t} \Lambda (n)}_{\psi (t)} \frac{1}{t^{s+1}} dt
    \] 
    By the prime number theorem,
    \[
    \psi (x) = x + O
    \left( \frac{x}{e^{c' \sqrt{\log x} }} \right) 
    \] 
    so
    \[
    \frac{\psi (x)}{x^{s}} \to 0, \quad x \to \infty
    \] 
    Thus
    \[
    - \frac{\zeta'}{\zeta} (s)
    = s \int_{1}^{\infty} \frac{\psi (t)}{t^{s+1}}dt 
    \] 
    for $\sigma > 1$.
\end{proof}



\section{Week 5}

\begin{exercise}[E5.1]
    Show that
    \[
    x \exp \left( -c \sqrt{\log x}  \right) 
    = O_m \left( \frac{x}{\log^{m} x} \right) 
    \] 
    for every $m$, and that
    \[
    x^{1-\varepsilon} = O_{\varepsilon}
    \left( x \exp \left( -c \sqrt{\log x}  \right)  \right) 
    \] 
    for every $\varepsilon > 0$. Discuss what this
    means for the quality of the error-term in the
    prime number theorem.
\end{exercise}


\begin{proof}
    \[
    \frac{\log^{m} x}{e^{c \sqrt{\log x} }}
    = \frac{ \sqrt{\log x}^{2m}}{e^{c \sqrt{\log x} }}
    \] 
    Now
    \begin{lemma}[]
        For any $a > 0$ and any  $b > 1$,
        \[
        \frac{x^{a}}{b^{x}} \to 0, \quad x \to \infty.
        \] 
    \end{lemma}
        Let $v = \sqrt{\log x} $. Then
        the above reads
        $\frac{v^{2m}}{e^{cv}}$.
        Assuming $c > 0$, we find that for
        $v \to \infty$,
        $\frac{v^{2m}}{e^{cv}} \to 0$.
        So in fact,
        \[
        x \exp \left( -c \sqrt{\log x}  \right) 
        = o \left( \frac{x}{\log^{m} x} \right) 
        \] 
        Now
        \[
        x^{1-\varepsilon} = x
        x^{-\varepsilon} = 
        x e^{- \log (x) \varepsilon}
        \le x e^{- c \sqrt{\log x} }.
        \] 
        Recall that we proved the following version of
        the prime number theorem:
        \begin{theorem}[Prime number theorem]
            There exists a $c' > 0$ such that
            \[
            \psi (x) =
            x + O \left( x
            \exp \left( -c' \sqrt{\log x}  \right) \right) 
            \] 
        \end{theorem}
        So by the above,
        \[
        \psi (x) = x +
        O_m \left( \frac{x}{\log^{m}(x)} \right) 
        \] 
        So essentially, the error
        term is smaller than
        $\frac{x}{\log^{m}(x)}$ for any $x$ but still
        larger than $x^{1-\varepsilon}$ for any
        $\varepsilon > 0$.
\end{proof}

\begin{exercise}[E5.2]
    Prove that the following two statements are equivalent:
    \begin{enumerate}
        \item There exists a $c >0$ such that
            \[
            \psi (x) = x + O\left( x \exp
            \left( -c \sqrt{\log x}  \right) \right) 
            \] 
        \item There exists a $c > 0$ such that
            \[
            \pi(x) = li(x) + O \left( x \exp
            \left( - c \sqrt{\log x}  \right) \right) 
            \] 
            where $li (x) = \int_{2}^{x} \frac{1}{\log t} dt $.
    \end{enumerate}
\end{exercise}

\begin{proof}
    Suppose $(1)$ is true. Then
    \begin{align*}
        \pi(x) 
        &=
    \frac{\psi (x)}{\log x} + \int_{2}^{x} 
    \frac{\psi (t)}{t \log^2 t} dt + O \left( x^{\frac{1}{2}}
    \log x\right)\\
        &= \frac{x}{\log x} + O\left( \frac{x}{\log x}
        \exp\left( -c \sqrt{\log x}  \right) \right) 
        + \int_{2}^{x} \frac{1}{\log^2 t} 
        + O\left( \frac{1}{\log^2 t \exp\left( c \sqrt{\log t} 
        \right) } \right) dt
        + O\left( x^{\frac{1}{2}} \log x \right) 
    \end{align*}
    Now
    \[
    \int_{2}^{x} \frac{1}{\log^2 t} dt
    = - \frac{t}{\log t} \bigg|_{2}^{x} 
    + li(x) 
    \] 
    giving
    \[
    \pi(x) = li(x) + \frac{2}{\log 2}
    + O\left( \frac{x}{\log x} e^{-c \sqrt{\log x} } \right) 
    + \int_{2}^{x} 
    O\left( \frac{e^{- c \sqrt{\log t} }}{\log^2 t } \right) dt
    + O \left( x^{\frac{1}{2}} \log x \right) 
    \] 
    All the middle terms apart from the
    last two are clearly
    $O \left( x e^{-c \sqrt{\log x} } \right) $.
    To take care of the last term, we use the lemma:
    \begin{lemma}[]
        For any $a > 0$,
        \[
        \frac{\log x}{x^{a}} \to 0, \quad x \to \infty
        \] 
    \end{lemma}
    Hence
    $x^{\frac{1}{2}} \log x 
    = O\left( x^{\frac{3}{4}} \right) 
    = O\left( x e^{-c' \sqrt{\log x} } \right) $.\\

    For the last part
    \begin{align*}
        \int_{2}^{x} O \left( \frac{e^{-c \sqrt{\log t} }}{\log^2 t}
        \right)dt
        &\le 
    \end{align*}


Note that the derivative of
$x e^{-c \sqrt{\log x} }$ is 
\[
e^{-c \sqrt{\log x} } - c \frac{d}{dx} \left[ \sqrt{\log x} 
\right] e^{-c \sqrt{\log x} }
= e^{-c \sqrt{\log x} }
-c \frac{1}{2} \frac{1}{x} \frac{1}{\sqrt{\log x} }
e^{-c \sqrt{\log x } }
\] 
But as
$x \to \infty$, this grows
faster than
$\frac{e^{-c} \sqrt{\log x} }{\log^2 x}$, which
is what we wanted.\\
\linebreak
Now we want to show that
$(2)$ implies $(1)$. So assume
there exists a $c> 0$ such that
\[
\pi(x) = li(x) + O\left( x \exp\left( -c \sqrt{\log x} 
\right) \right) .
\] 
Then recall that
\[
\psi (x) = 
\pi(x) \log x - \int_{2}^{x} \frac{\pi(t)}{t} dt
- O\left( x^{\frac{1}{2}} \log^2 x \right) 
\] 
So
\begin{align*}
    \psi (x)
    &= li(x) \log x + \log x O\left( xe^{-c \sqrt{\log x} } \right) 
    - \int_{2}^{x} \frac{li (t)}{t} dt
    - \int_{2}^{x} O\left( e^{-c \sqrt{\log t} } \right)dt
    - O\left( x^{\frac{1}{2}} \log^2 x \right) \\
\end{align*}
Now, by repeated integration by parts, we get
\begin{align*}
    li(x) 
    &= \frac{t}{\log t} \bigg|_{2}^{x}
    + \int_{2}^{x} \frac{1}{\log^2 t} dt\\
    &= \frac{t}{\log t}\bigg|_{2}^{x}
    + \left[ \frac{t}{\log^2 t} \bigg|_{2}^{x}
    + 2 \int_{2}^{x} \frac{1}{\log^3 t}dt \right] \\
    &= \frac{t}{\log t} + \frac{t}{\log^2 t}\bigg|_{2}^{x}
    + 2 \left[ \frac{t}{\log^3 t} \bigg|_{2}^{x}
    +  3 \int_{2}^{x} \frac{1}{\log^{4} t} dt  \right] \\
    &= x \sum_{r=1}^{k-1} \frac{(r-1)!}{\log^{r} x}
    + (k-1)! \int_{2}^{x}  \frac{1}{\log^{k} t} dt 
\end{align*}

\end{proof}




\begin{exercise}[E5.3]
    Let $f$ be a Schwartz function on the real line, and
    let $\hat{f}$ be its Fourier transform. Show that
    \[
        \sum_{n \in \mathbb{Z}} f \left( \frac{v+n}{t} \right) 
    = \sum_{n \in \mathbb{Z}} \left| t \right| 
    \hat{f}\left( nt \right) e^{2 \pi i n v}.
\]
\end{exercise}

\begin{proof}
    For a Schwartz function $f$, we know from the
    Poisson summation formula that
    \[
    \sum_{n \in \mathbb{Z}} f(n)
    = \sum_{n \in \mathbb{Z}} \hat{f}(n),
    \] 
    where
    \[
    \hat{f}(y) = \int_{-\infty}^{\infty} 
    e^{-2 \pi i xy} f(x) dx
    \] 
    Define
    $g(x) = f\left( \frac{v+x}{t} \right) $. Then
    $g$ is also a Schwartz function, so
    \[
    \sum_{n\in \mathbb{Z}} g(n) = 
    \sum_{n \in \mathbb{Z}}
    \int_{-\infty}^{\infty} e^{-2 \pi i n x}
    f\left( \frac{v+x}{t} \right) dx
    \] 
    Let
    $z = \frac{v+x}{t}$. Then
    $dz = \frac{1}{\left| t \right|} dx$, so
    
    \[
    \sum_{n \in \mathbb{Z}}
    f\left( \frac{v+n}{t} \right) 
    = \sum_{n \in \mathbb{Z}}g(n)
    = \sum_{n \in \mathbb{Z}}
    \left| t \right|  e^{-2 \pi i n \left( tz-v \right) }
    f(z) dz
    = \sum_{n \in \mathbb{Z}}
    \left| t \right| \hat{f}(nt) e^{2\pi i n v}
    \] 
\end{proof}


\begin{exercise}[E5.4]
    Let $\theta > \frac{1}{2}$. Prove that if
    for every $\varepsilon > 0$, 
    $\psi (x) = x + O\left( x^{\theta + \varepsilon} \right) $,
    then the Riemann zeta function has
    no zeroes in $\Re (s) > \theta$. 
    (It turns out that this is in fact
    an 'if and only if statement'). Think about
    what this implies for the Riemann hypothesis. Compare
    with the zerofree region provided by
    Theorem 6.6.
\end{exercise}

\begin{proof}
    By the explicit formula, if we simply let
    $x$ range among $\mathbb{R} - \mathbb{Z}$, then
    we have
    \[
    O\left( x^{\theta + \varepsilon} \right) 
    = \lim_{T \to \infty} 
    \sum_{\substack{\zeta(\rho) = 0\\ \left| \im \rho \right| 
    \le T}} \frac{x^{\rho}}{\rho }
    + \frac{\zeta'}{\zeta}(0) + \frac{1}{2} 
    \log \left( 1 - \frac{1}{x^2} \right) ,
    \] 
    however, if there is a  $\rho $ with
    $\Re (\rho) > \theta$, then choosing
    $\varepsilon$ such that
    $\theta < \varepsilon < \Re (\rho)$, we get
    that the right hand side grows faster, giving
    a contradiction.\\
    \linebreak
    Hence the Riemann hypothesis can be
    reformulated as saying that
    for any $\varepsilon > 0$,
    \[
    \psi (x) = x + O\left( x^{\frac{1}{2}+ \varepsilon} \right) .
    \] 
    Now, any $\theta$ would, of course, be a very strong improvement
    combined with the zero-free region. This is because
    the zero-free region tapers off as the imaginary
    part grows in size, while
    finding a $\theta$ such that the
    above holds would imply, as shown, that
    we can shrink the critical strip to
    a narrower strip.
\end{proof}

\begin{exercise}[E5.5]
    Let $p_n$ be the $n$ th prime. Show that
    \[
    \frac{1}{N} \sum_{n=1}^{N} \frac{p_{n+1}- p_n}{\log p_n}
    \to 1
    \] 
    as $N \to \infty$, and discuss how to interpret
    this as a statement about the average
    spacing between adjacent primes.
\end{exercise}


\begin{proof}
    By Abel summation, we have
    \[
    \sum_{n\le x} \frac{p_n}{\log p_n}
    = \sum_{n\le x} p_n \frac{1}{\log x}
    - \int_{1}^{x} \sum_{n\le t} p_n \frac{1}{\log t}dt 
    \] 
    And similarly
    \[
    \sum_{n\le x} \frac{p_{n+1}}{\log p_n}
    = \sum_{n\le x} p_{n+1} \frac{1}{\log x}
    - \int_{1}^{x} \sum_{n\le t} p_{n+1} \frac{1}{\log t}dt 
    \] 
    Hence
    \begin{align*}
    \sum_{n\le x} \frac{p_{n+1} - p_n}{\log p_n}
    &= \frac{1}{\log x} \sum_{n\le x} p_{n+1} -p_n
    - \int_{1}^{x} \frac{1}{\log t} \sum_{n \le t}(p_{n+1}-p_n) dt\\
    &= \frac{p_{\left[ x \right] +1} - 2}{\log x} 
    - \int_{1}^{x} \frac{p_{\left[ t \right] +1} - 2}{\log t} dt
    \end{align*}
    Now
    $\frac{p_n}{n \log n} \to 1$ as $n \to \infty$, so
    $\frac{p_{n+1}}{n \log n}
    = \frac{p_{n+1}}{(n+1) \log(n+1)} \frac{(n+1) \log(n+1)}{
    n \log n}
    \to 1$ as $n \to \infty$. So we will get
    the result if we can show that
    \[
    \lim_{n \to \infty} \frac{1}{n} 
    \int_{1}^{n} \frac{p_{\left[ t \right] +1}-2}{\log t} dt 
    = 0.
    \] 

    By the PNT, we have
    \[
    p_n \sim n \log n.
    \] 
    So
    \[
    \lim_{N \to \infty} \frac{1}{n} \sum_{n \ge N}
    \frac{p_{n+1}-p_n}{\log p_n}
    = \lim_{N\to \infty}
    \sum_{n\ge N} \frac{(1+\frac{1}{n})\log(n+1) - \log n}{\log n + 
    \log \log n}
    = 
    \] 
\end{proof}


\section{Assignment 1}


    \begin{exercise}[H1.1]
        \begin{proof}
            \[
            f * e (n) = \sum_{d  \mid n} f(d) e (\frac{n}{d})
            = \sum_{d  \mid n} f(d) \delta_{\frac{n}{d},1}
            = f(n)
            \] 
            and since the sets
            $\left\{ d \colon d \mid n \right\} $ and
            $\left\{ \frac{n}{d}  \colon
            d  \mid n\right\} $ are equal, we have
            \[
            g * f = 
            \sum_{d \mid n} g(d) f\left( \frac{n}{d} \right) 
            =
            \sum_{d  \mid n} g\left( \frac{n}{d} \right) 
            f \left( \frac{n}{\frac{n}{d}} \right) 
            = f * g (n)
            \] 
        \end{proof}
    \end{exercise}


    \begin{exercise}[H1.2]
        \begin{proof}
            \[
            \mu * 1 (n)
            = \sum_{d \mid n}
            \mu (d) 1 \left( \frac{n}{d} \right) 
            = \sum_{d \mid n} \mu(d)
            \] 
            If $n = p$ is a prime, we trivially
            have
            $\left\{ d \colon d \mid n \right\} 
            = \left\{ 1,p \right\} $, so
            $\sum_{d \mid n} \mu (d) =
            1 -1 = 0 = e(p)$, so it is true for
            $n$ a prime.\\
            \linebreak
            Suppose now that
            $n = p_1 \cdots p_s$, so
            $\mu (n) = (-1)^{s}$. 
            We need to find out how many
            elements the set
            $D_k = 
            \left\{ d  \mid n \colon
            d \text{ is a product of k distinct primes}\right\} $
            has.
            But this is simply the same as choosing
            an unordered set of $k$ elements from a set
            of  $s$ elements. There
            are precisely
            $\begin{pmatrix} s\\ k \end{pmatrix} $ ways to do
            so. Since for each $d \in D_k$,
            we have $\mu (d) = (-1)^{k}$, we find
            that
            \[
            \sum_{d  \mid n}\mu (d)
            = \sum_{k=1}^{s} \begin{pmatrix} s\\k \end{pmatrix} 
            (-1)^{k}
            = (1-1)^{s}
            = 0.
            \] 
            Then, in particular,
            \[
            \sum_{d  \mid n} \mu(d)
            \] 


            Lastly, for
            $n = p_1^{\alpha_1} \cdots p_{k}^{\alpha_{k}}$, it
            reduces to the previous case because
            $\mu$ is only non-zero on squarefree integers, so
            \begin{align*}
                \mu * 1 (n)
                &= \sum_{d  \mid \frac{n}{p_1^{\alpha_1-1} 
                \cdots p_k^{\alpha_k-1}}}
                \mu (d) = 0
            \end{align*}
            since the sets
            $\left\{ d \colon
            d  \mid 
        \frac{n}{p_1^{\alpha_1 - 1} \cdots
    p_k^{\alpha_k -1}}\right\} $ and
    $\left\{ d \colon
    d  \mid p_1 \cdots p_k \right\} $ are equal.

    Thus, indeed, $\mu * 1 = e$.
        \end{proof}
    \end{exercise}

    \begin{exercise}[H1.3]
        We claim that the set of arithmetic functions
        with Dirichlet convolution as a binary operation is
        an abelian semigroup.
        For this, if $f,g \colon \mathbb{N}  \to \mathbb{C}$, then
        clearly $f * g \colon \mathbb{N} \to \mathbb{C}$ too.
        Also,
        $f *g(n) = \sum_{ab = n} f(a) g(b) = 
        \sum_{ba = n} g(b) f(a) = g*f(n)$ by commutativity
        of multiplication in $\mathbb{C}$.
        Lastly,
        \[
            \left( f*g \right) *h(n)
            = \sum_{ab = n} f*g(a) h(b)
            = \sum_{ab=n} \sum_{cd = a} f(c) g(d) h(b)
            = \sum_{cdb = n} f(c) g(d) h(b)
        \] 
        and
        \[
        f * \left( g*h \right) (n)
        = \sum_{ab = n} f(a) g*h(b)
        = \sum_{ab = n} \sum_{cd = b} f(a) g(c) h(d)
        = \sum_{acd = n} f(a) g(c) h(d)
        \] 
        (all of this is just Theorem 5.1.4 in the book for Introduction to Number Theory
        by Risager).

        Now, if
        $f = 1 * g$ then
        $\mu * f = \mu * \left( 1 * g \right) 
        = \left( \mu * 1 \right) * g
        = e * g
        = g * e = g$ by the above together with
        H1.1.
        Likewise, if $g = \mu * f$, then
        $1 * g = 1 * \left( \mu * f \right) 
        = \left( 1 * \mu \right) * f
        = \left( \mu * 1 \right) * f
        = e * f = f * e = f$ again.
    \end{exercise}

    \begin{exercise}[H1.4]
        We have
        \begin{align*}
            \sum_{n=1}^{\infty} \left| 
            \frac{f(n)}{n^{s}}\right| 
            &\le 
            \sum_{n=1}^{\infty } \frac{C n^{k}}{n^{\sigma}}\\
            &\le \sum_{n=1}^{\infty}
            \frac{C}{n^{\sigma - k}}\\
            &< \infty
        \end{align*}
        as $\sigma - k > 1$. Thus the series
        converges absolutely.
    \end{exercise}

    \begin{exercise}[H1.5]
        \begin{proof}
            We know that
            $L_f$ converges absolutely for
            $\sigma > 1 + k_f$ and
            $L_g$ converges absolutely for
            $\sigma > 1 + k_g$.
            Assume without loss of generality that
            $k_g > k_f$.
            Now,
            \begin{align*}
                \sum_{n=1}^{\infty }
                \left| \frac{\sum_{d  \mid n} f(d)
                g(\frac{n}{d})}{n^{s}}
                \right| 
                &\le \sum_{n=1}^{\infty} \sum_{d \mid n}
                \frac{C_f C_g d^{k_f} 
                \left( \frac{n}{d} \right)^{k_g}}{n^{\sigma}}\\
                &= \sum_{n=1}^{\infty} C_f C_g
                \sum_{d \mid n} d^{k_f - k_g} \frac{1}{n^{\sigma -
                k_g}}
            \end{align*}
            Now, by E3.2, we have
            $d(n) \le 2 \sqrt{n} $, so
            since
            $\sum_{d \mid n} d^{k_f -k_g}
            \le \sum_{d \mid n} 1 = 
            d(n) \le 2 \sqrt{n}$, we have
            \begin{align*}
                \sum_{n=1}^{\infty} C_f C_g
                \sum_{d \mid n} d^{k_f - k_g} \frac{1}{n^{\sigma -
                k_g}}
                &\le 
                \sum_{n=1}^{\infty} C_f C_g
                2\sqrt{n}  \frac{1}{n^{\sigma -
                k_g}}\\
                &= 2C_f C_g \sum_{n=1}^{\infty} 
                \frac{1}{n^{\sigma -\left( k_g + \frac{1}{2}
                \right) }}
            \end{align*}

            Hence the sum defining
             $L_{f*g}(s)$ is absolutely convergent
             for $\sigma > 
             k_g + \frac{3}{2}$,
             and in this half-plane, 
             \[
             L_f(s) L_g(s) = 
             \sum_{k=1}^{\infty} \sum_{t=1}^{\infty}
             \frac{f(k)}{k^{s}}
             \frac{g(t)}{t^{s}}
             = 
             \sum_{r=1}^{\infty} \sum_{d  \mid r} 
             \frac{f(d) g(\frac{n}{d})}{r^{s}}
             = L_{f*g}(s)
             \] 
        \end{proof}
    \end{exercise}

    \begin{exercise}[H1.6]
        We have that
        when $L_1$ and $L_{\mu}$ are absolutely convergent,
        and satisfy the bounds
        from H1.5,
        we can use Cauchy summation to get
        $L_{1}(s) L_{\mu}(s) = 
        L_{1 * \mu}(s) =
        L_{e}(s)
        = 1$ which is absolutely
        convergent everywhere; but
        $L_1(s) = \zeta(s)$ and
        $L_{\mu}(s) = \sum_{n=1}^{\infty} \frac{\mu(n)}{n^{s}}$, 
        so
        the result follows in whenever all sums
        are absolutely convergent. Hence
        the desired equality extends (by the identity
        theorem) to all
        of $\Re (s) > 1$ since
        $\sum_{n=1}^{\infty} \frac{\mu(n)}{n^{s}} $ 
        converges to a holomorphic function in this
        half-plane (being the uniform limit of a 
        series of holomorphic
        functions).
    \end{exercise}


    \begin{exercise}[H 1.7]
        \begin{proof}
            For $f(n) = n^{w}$, we have
            $\sigma_w (n) = f * 1 (n)$.
            The abscissa of convergence for
            $1$ is $1$ and for $f$ it is
            $1 + \Re (w)$. In some
            halfplane, we have
            $\sum_{n=1}^{\infty} \frac{\sigma_w (n)}{n^{s}}
            = L_{\sigma_w}(s)
            = L_{f}(s) L_{1}(s)$. Now
            $L_1 (s) = \zeta(s)$, and
            \[
            L_f(s) = 
            \sum_{n=1}^{\infty} \frac{n^{w}}{n^{s}}
            = \sum_{n=1}^{\infty}
            \frac{1}{n^{s-w}}
            = \zeta(s-w).
            \] 
            Thus
            $\sum_{n=1}^{\infty} \frac{\sigma_w(n)}{n^{s}}
            = \zeta(s-w) \zeta(s)$
            in some right half-plane.
        \end{proof}
    \end{exercise}


\section{Assignment 2}

\begin{exercise}[H2.1]
    Show that
    \[
    \sum_{p\le x} \frac{1}{p}
    = \log \log (x) + O(1),
    \] 
    where the sum is over primes less than
    $x$.
\end{exercise}

\begin{proof}
    As is the custom, we of course start by Abel summation:
    \[
    \sum_{p\le x} \frac{1}{p}
    = \pi(x) \frac{1}{x}
    +\int_{1}^{x} \frac{\pi(t)}{t^2} dt
    \] 
    Now applying the PNT, we get

    \[
    \pi(x) \frac{1}{x} + 
    \int_{1}^{x} \frac{\pi(t)}{t^2} dt
    = \frac{1}{\log x} +
    O\left( e^{-c \sqrt{\log x} } \right) 
    + \int_{1}^{x} \frac{1}{t \log t}dt
    + \int_{1}^{x} O\left( t^2 e^{-c \sqrt{\log t} } \right) dt 
    \] 
    Since
    \[
    \int_{1}^{x} \frac{1}{t \log t}dt
    = \log \log t  \bigg|_{1}^{x}
    \] 
    we have what we needed.

\end{proof}

\begin{exercise}[H2.2]
    This exercise gives a different proof that
    $\zeta(s)$ has no zeros on
    $\Re(s) = 1$.
    \begin{enumerate}
        \item Prove that for
            $\sigma > 1, t \in \mathbb{R}$,
            \[
            \Re \left( 3 \log \zeta (\sigma)
            + 4 \log \zeta \left( \sigma + it)
        + \log \zeta \left( \sigma+ 2it \right) \right) \right) 
        \ge 0.
            \] 
        \item Prove that
            $\left| \zeta(\sigma)^3
            \zeta \left( \sigma + it \right)^{4} 
            \zeta\left( \sigma + 2it \right) \right| \ge 1 $.
        \item Prove that if $\zeta \left( 1+ it_0 \right) =0$,
            then
            $\left| \zeta(\sigma)^3
            \zeta\left( \sigma + it_0 \right)^{4}
            \zeta \left( \sigma + 2it_0 \right) \right| 
            \to 0$ as $\sigma \to 1$.
        \item Conclude that $\zeta \left( 1+it \right) \neq 0
            $ for every $t \neq 0$.
    \end{enumerate}
\end{exercise}

\begin{proof}
    i) 
\end{proof}



    %\printbibliography
\end{document}
