\documentclass[reqno]{amsart}
\usepackage{amscd, amssymb, amsmath, amsthm}
\usepackage{graphicx}
\usepackage[colorlinks=true,linkcolor=blue]{hyperref}
\usepackage[utf8]{inputenc}
\usepackage[T1]{fontenc}
\usepackage{textcomp}
\usepackage{babel}
%% for identity function 1:
\usepackage{bbm}
%%For category theory diagrams:
\usepackage{tikz-cd}

%\usepackage[backend=biber]{biblatex}
%\addbibresource{.bib}


\setlength\parindent{0pt}

\pdfsuppresswarningpagegroup=1

\newtheorem{theorem}{Theorem}[section]
\newtheorem{lemma}[theorem]{Lemma}
\newtheorem{proposition}[theorem]{Proposition}
\newtheorem{corollary}[theorem]{Corollary}
\newtheorem{conjecture}[theorem]{Conjecture}

\theoremstyle{definition}
\newtheorem{definition}[theorem]{Definition}
\newtheorem{example}[theorem]{Example}
\newtheorem{exercise}[theorem]{Exercise}
\newtheorem{problem}[theorem]{Problem}
\newtheorem{question}[theorem]{Question}

\theoremstyle{remark}
\newtheorem*{remark}{Remark}
\newtheorem*{note}{Note}
\newtheorem*{solution}{Solution}



%Inequalities
\newcommand{\cycsum}{\sum_{\mathrm{cyc}}}
\newcommand{\symsum}{\sum_{\mathrm{sym}}}
\newcommand{\cycprod}{\prod_{\mathrm{cyc}}}
\newcommand{\symprod}{\prod_{\mathrm{sym}}}

%Linear Algebra

\DeclareMathOperator{\Span}{span}
\DeclareMathOperator{\im}{im}
\DeclareMathOperator{\diag}{diag}
\DeclareMathOperator{\Ker}{Ker}
\DeclareMathOperator{\ob}{ob}
\DeclareMathOperator{\Hom}{Hom}
\DeclareMathOperator{\Mor}{Mor}
\DeclareMathOperator{\sk}{sk}
\DeclareMathOperator{\Vect}{Vect}
\DeclareMathOperator{\Set}{Set}
\DeclareMathOperator{\Group}{Group}
\DeclareMathOperator{\Ring}{Ring}
\DeclareMathOperator{\Ab}{Ab}
\DeclareMathOperator{\Top}{Top}
\DeclareMathOperator{\hTop}{hTop}
\DeclareMathOperator{\Htpy}{Htpy}
\DeclareMathOperator{\Cat}{Cat}
\DeclareMathOperator{\CAT}{CAT}
\DeclareMathOperator{\Cone}{Cone}
\DeclareMathOperator{\dom}{dom}
\DeclareMathOperator{\cod}{cod}
\DeclareMathOperator{\Aut}{Aut}
\DeclareMathOperator{\Mat}{Mat}
\DeclareMathOperator{\Fin}{Fin}
\DeclareMathOperator{\rel}{rel}
\DeclareMathOperator{\Int}{Int}
\DeclareMathOperator{\sgn}{sgn}
\DeclareMathOperator{\Homeo}{Homeo}
\DeclareMathOperator{\SHomeo}{SHomeo}
\DeclareMathOperator{\PSL}{PSL}
\DeclareMathOperator{\Bil}{Bil}
\DeclareMathOperator{\Sym}{Sym}
\DeclareMathOperator{\Skew}{Skew}
\DeclareMathOperator{\Alt}{Alt}
\DeclareMathOperator{\Quad}{Quad}
\DeclareMathOperator{\Sin}{Sin}
\DeclareMathOperator{\Supp}{Supp}
\DeclareMathOperator{\Char}{char}
\DeclareMathOperator{\Teich}{Teich}
\DeclareMathOperator{\GL}{GL}
\DeclareMathOperator{\tr}{tr}
\DeclareMathOperator{\codim}{codim}
\DeclareMathOperator{\coker}{coker}
\DeclareMathOperator{\Diff}{Diff}
\DeclareMathOperator{\Bun}{Bun}
\DeclareMathOperator{\Sm}{Sm}



%Row operations
\newcommand{\elem}[1]{% elementary operations
\xrightarrow{\substack{#1}}%
}

\newcommand{\lelem}[1]{% elementary operations (left alignment)
\xrightarrow{\begin{subarray}{l}#1\end{subarray}}%
}

%SS
\DeclareMathOperator{\supp}{supp}
\DeclareMathOperator{\Var}{Var}

%NT
\DeclareMathOperator{\ord}{ord}

%Alg
\DeclareMathOperator{\Rad}{Rad}
\DeclareMathOperator{\Jac}{Jac}

%Misc
\newcommand{\SL}{{\mathrm{SL}}}
\newcommand{\mobgp}{{\mathrm{PSL}_2(\mathbb{C})}}
\newcommand{\id}{{\mathrm{id}}}
\newcommand{\MCG}{{\mathrm{MCG}}}
\newcommand{\PMCG}{{\mathrm{PMCG}}}
\newcommand{\SMCG}{{\mathrm{SMCG}}}
\newcommand{\ud}{{\mathrm{d}}}
\newcommand{\Vol}{{\mathrm{Vol}}}
\newcommand{\Area}{{\mathrm{Area}}}
\newcommand{\diam}{{\mathrm{diam}}}
\newcommand{\End}{{\mathrm{End}}}


\newcommand{\reg}{{\mathtt{reg}}}
\newcommand{\geo}{{\mathtt{geo}}}

\newcommand{\tori}{{\mathcal{T}}}
\newcommand{\cpn}{{\mathtt{c}}}
\newcommand{\pat}{{\mathtt{p}}}

\let\Cap\undefined
\newcommand{\Cap}{{\mathcal{C}}ap}
\newcommand{\Push}{{\mathcal{P}}ush}
\newcommand{\Forget}{{\mathcal{F}}orget}




\begin{document}



\section{Theory}


    Recall that
    \begin{definition}[Dirichlet Series]
        Let $f$ be an arithmetic function. Then the corresponding
        Dirichlet series is defined, for
        $s \in \mathbb{C}$, by
        \[
        D_f (s) = \sum_{n=1}^{\infty} \frac{f(n)}{n^{s}}.
        \] 
    \end{definition}


    \begin{lemma}[]\label{Lemma:02329}
        \[
        0 \le 3 + 4 \cos \theta + \cos 2\theta = 
        2 \left( 1+ \cos \theta \right)^2
        \] 
    \end{lemma}

    \begin{lemma}[]\label{Lemma:39292}
        Let $\sigma > 1$. Then
        \[
        \Re \left( -3 \frac{\zeta '}{\zeta} (\sigma)-
        4 \frac{\zeta'}{\zeta}(\sigma + it)-
    \frac{\zeta'}{\zeta}(\sigma + 2 i t)\right) \ge 0
        \] 
    \end{lemma}

    For the proof of the lemma, one shows that
    \[
    \Re \left( \frac{1}{n^{s}} \right) =
    \frac{1}{n^{\sigma}} \cos \left( t \log n \right) , \quad
    s = \sigma + it \tag{$A_1$} \label{A-1}
    \] 

    \begin{proof}
        \[
        \Re \left( - \frac{\zeta'}{\zeta}(s) \right) 
        = \Re \sum_{n=1}^{\infty} \frac{\Lambda (n)}{n^{s}}
        = \sum_{n=1}^{\infty} \frac{\Lambda(n)}{n^{\sigma}}
        \cos \left( t \log n \right) .
        \] 
        Hence
        \begin{align*}
        \Re \left( -3 \frac{\zeta '}{\zeta} (\sigma)-
        4 \frac{\zeta'}{\zeta}(\sigma + it)-
    \frac{\zeta'}{\zeta}(\sigma + 2 i t)\right) 
    &= \sum_{n=1}^{\infty} \frac{\Lambda(n)}{n^{\sigma}}
    \left[ 3 + 4 \cos \left( t \log n \right) +
    \cos \left( 2t \log n \right) \right] 
    \stackrel{\eqref{Lemma:02329}}{\ge} 0
        \end{align*}
    \end{proof}


\newpage

\section{Week 1}

\begin{exercise}[E1.1. Abel summation]
    Let $\left\{ a_n \right\}_{n \in \mathbb{N} }\subset \mathbb{C}$ 
    and $f \colon \left[ 1,x \right] \to \mathbb{C}$ be
    $C^{1}$. Define $A(t) = \sum_{n \le t} a_n$. Then for
    $x>1$, we have
    \[
    \sum_{n\le x} a_n f(n) = A(x) f(x) -
    \int_{1}^{x} A(t) f'(t) dt. 
    \] 
\end{exercise}



\section{Week 2}

Let $\psi (x) :=
\sum_{n \le x} \Lambda (n)$.

\begin{exercise}[E2.6]
    Show that
    \[
    \theta(x) := 
    \sum_{p \le x} \log p = \psi (x) + O\left( x^{\frac{1}{2}}
    \log^2 x \right) 
    \] 
\end{exercise}

\begin{exercise}[E2.7]
    Show that
    \[
    \pi(x) = \frac{\psi (x)}{\log x}+ 
    \int_{2}^{x} \frac{\psi (t)}{t \log^2t}dt +
    O\left( x^{\frac{1}{2}} \log x \right) .
    \] 
\end{exercise}

\begin{proof}
    By Abel summation, we first find that
    \[
    \theta(x) :=
    \sum_{p \le x} \log p = 
    \pi(x) \log x - \int_{2}^{x} \frac{\pi(t)}{t}dt
    \] 
    and from the previous exercise, we now find
    that
    \[
    \pi(x) = \frac{\psi (x)}{\log x} +
    \frac{1}{\log x} \int_{2}^{x}\frac{\pi(t)}{t}dt 
    + O \left( x^{\frac{1}{2}} \log x \right) 
    \] 
    The result follows if we
    can show that
    \[
    \frac{1}{\log x} \int_{2}^{x} \frac{\pi(t)}{t}dt
    = \int_{2}^{x} \frac{\psi (t)}{t \log^2 t}dt 
    + O\left( x^{\frac{1}{2}} \log x \right).
    \] 
    Now $\psi (t) \le \pi(t) \log t$, so
    \begin{align*}
        \left| \int_{2}^{x} 
        \frac{\psi (t)}{t \log^2 t} - 
        \frac{\pi(t)}{t \log x}dt \right| 
        &\le 
        \left| \int_{2}^{x} \frac{\pi(t)}{t \log t} -
        \frac{\pi(t)}{t \log x} dt\right| \\
        &= \left| 
        \int_{2}^{x} \frac{\pi(t)}{t} 
        \frac{\log \left( \frac{x}{t} \right) }{\log x 
        \log t} dt\right| \\
    \end{align*}
\end{proof}

\section{Week 3}

\begin{exercise}[E3.1]
    Let $m \ge 0$. Show that
    \[
    \sum_{n \le x} \log^{m} n = 
    x \log^{m} x + O \left( x \log^{m-1} x \right) .
    \] 
\end{exercise}

\begin{proof}
    Let $a_n = 1$ for all $n$. Then
    $A(x) = \left\lfloor x \right\rfloor$, so
    \begin{align*}
    \sum_{n \le x} \log^{m}n 
    &= 
    \left\lfloor x \right\rfloor \log^{m} x
    - \int_{1}^{x} m \left\lfloor t \right\rfloor \frac{1}{t}
    \log^{m-1} t dt\\
    &= x \log^{m}x
    - \left( x - \left\lfloor x \right\rfloor  \right) \log^{m} x
    -m \int_{1}^{x} \frac{\left\lfloor  t \right\rfloor }{t}
    \log^{m-1}(t) dt
    \end{align*}
    Thus we must show that
    \begin{align*}
    \left| \left( x - \left\lfloor x \right\rfloor  \right) \log^{m} x
    +m \int_{1}^{x} \frac{\left\lfloor  t \right\rfloor }{t}
    \log^{m-1}(t) dt  \right| \le 
    C x \log^{m-1}x
    \end{align*}
    But $\frac{\left\lfloor  t \right\rfloor }{t}
    \log^{m-1} (t) \le \log^{m-1}(x)$ giving that
    the right hand term is $O\left( x \log^{m-1} x \right) $.
    For the left hand term,
    it suffices to show that
    $\left( x - \left\lfloor x  \right\rfloor  \right) 
    \log x \le x$, but this is clear
    since $x - \left\lfloor x \right\rfloor \le 1$ and
    $ \log x  \le x$.
\end{proof}

\begin{exercise}[E3.2]
    Let $d(n) = \sum_{d  \mid n} 1$. Show
    $d(n) \le 2 \sqrt{n} $.
    If we consider the set $D \subset \mathbb{N} $ of positive
    divisors of $n$, then we can define a bijection
    $D \to D$ by
    $k \mapsto \frac{n}{k}$. Suppose
    now that $d(n) > 2 \sqrt{n} $. 
    Suppose $d  \mid n$ and
    $d \ge  \sqrt{n} $. Then
    since $\frac{d}{n} \cdot d = n$, we must have
    $\frac{d}{n} \le  \sqrt{n} $. This implies that
    under this bijection, either the source or
    target lies in
    $\left\{ 1,\ldots, \left\lfloor \sqrt{n}  \right\rfloor \right\} $.
    Hence $d(n) = \left| D \right| \le 
    2 \left| \left\{ 1, \ldots,
    \left\lfloor \sqrt{n}  \right\rfloor \right\}  \right| 
    \le 2 \sqrt{n} $.
\end{exercise}

\begin{exercise}[E3.3]
    Prove that for every $\varepsilon > 0$, there
    exists a constant $C_{\varepsilon}$ such that
    $d(n) \le C_{\varepsilon} n^{\varepsilon}$.\\
    Hint:
    \begin{enumerate}
        \item Show that $d(n_1n_2) = d(n_1)d(n_2)$ if
            $\left( n_1,n_2 \right) =1$.
        \item Show that
            \[
            \frac{d(n)}{n^{\varepsilon}}
            = \prod_{p^{\alpha} \mid  \mid   n} \frac{\alpha+1}{
            p^{\alpha \varepsilon}}
            \] 
            where $p^{\alpha} \mid  \mid n$ means that
            $\alpha$ is a positive integer,
            $p^{\alpha} \mid n$ and
            $p^{\alpha+1}\not| n$.
        \item Split the product in $2$. Into the product
            over those primes $p < 2^{\frac{1}{\varepsilon}}$ and
            the product over the rest. Show that the
            second product is bounded by $1$.
        \item Show that the factors in the first product
            are less than $1 + \left( \varepsilon \log 2 \right)^{-1}$.
    \end{enumerate}
\end{exercise}

\begin{proof}
    We follow the hint:\\
    (1) Suppose $\left( n_1,n_2 \right) =1$. 
    Let $D$ be the set of divisors of $n_1n_2$, 
    $D_1 $ the set of divisors of $n_1$ and
    $D_2$ the set of divisors of $n_2$.
    Suppose $d_1 \in D_1, d_2 \in D_2$.
    Then $d_1 a = n_1, d_2 b = n_2$, so
    $d_1d_2 ab = n_1n_2$, hence
    $d_1d_2 \in D$. We thus obtain a map
    $D_1 \times D_2 \to D$ sending
    $\left( d_1,d_2 \right) \mapsto d_1d_2$. We claim this is
    a bijection. Suppose
    $d_1d_2 = d_1' d_2'$.
    If $d_1  \mid d_2'$, then
    $d_1 = 1$, in which case,
    $d_1' = 1$, and thus $d_2 = d_2'$.
    Suppose thus that $d_1 \neq 1$, so
    $d_1 \not| d_2'$.
    Then since $\left( d_1', d_2' \right) =1$, we have
    $d_1  \mid d_1'$. Similarly,
    $d_1'  \mid d_1$. So 
    $d_1 = d_1'$. And again $d_2 = d_2'$. This gives injectivity.
    For surjectivity, if
    $d  \mid n_1n_2$, then consider
    $d_1 := \frac{d}{\left( n_2,d \right) }$ and
    $d_2 := \frac{d}{\left( n_1,d \right) }$.
    Then $d_1 d_2 = d$ and $d_1 \in D_1, d_2 \in D_2$.\\
    (2) Clearly,
    $n^{\varepsilon} = \prod_{p^{\alpha} \mid  \mid 
    n} p^{\alpha \varepsilon}$. It thus suffices to show that
    $\prod_{p^{\alpha} \mid  \mid n} \left( \alpha+1 \right) 
    = d(n)$. But if we factorize $n$ as
    $n = p_1^{\alpha_1} \cdots p_m^{\alpha_m}$, then
    it is clear that the divisors corresponds precisely to
    tuples
    $\left( a_1, \ldots, a_m \right) $ with
    $0 \le a_i \le \alpha_i$. There
    are precisely $\alpha_1 + 1$ choices for each
    $a_i$, giving
    $\left( \alpha_1 +1 \right) \cdots
    \left( \alpha_m +1 \right) = d(n)$ which indeed
    is what we wanted to show.\\
    (3) We can split the product as
    \[
    \frac{d(n)}{n^{\varepsilon}} = 
    \underbrace{\prod_{\substack{p^{\alpha} \mid  \mid n \\ p <
    2^{\frac{1}{\varepsilon}}}}
\frac{\alpha+1}{p^{\alpha \varepsilon}}}_{A}
    \cdot 
    \underbrace{\prod_{\substack{p^{\alpha} \mid  \mid n \\ p \ge 
    2^{\frac{1}{\varepsilon}}}}
\frac{\alpha+1}{p^{\alpha \varepsilon}}}_{B}
    \] 
    We claim that $B \le 1$. Indeed

    \[
    \prod_{\substack{p^{\alpha} \mid  \mid n \\ p \ge 
    2^{\frac{1}{\varepsilon}}}}
    \frac{\alpha+1}{p^{\alpha \varepsilon}}
    \le 
    \prod_{\substack{p^{\alpha} \mid  \mid n \\ p \ge 
    2^{\frac{1}{\varepsilon}}}}
    \underbrace{\frac{\alpha+1}{2^{\alpha}}}_{\le 1}
    \le 1
    \] 

    (4) For the factors in the first product, we have
    $\alpha = \left\lfloor \frac{\log n}{\log p}
    \right\rfloor$ and
    $\log p < \frac{1}{\varepsilon} \log 2$, and
    $\alpha \le \frac{\log n}{\log p}$, so
    $\frac{\log p}{\log n} \le \frac{1}{\alpha}$
    \[
    \varepsilon^2 \log p < \varepsilon \log 2
    \] 


    \[
        \frac{\alpha+1}{p^{\alpha \varepsilon}}
        \le \frac{\log n +\log p}{p^{\alpha \varepsilon} \log p}
        \le 1 + \frac{1}{\varepsilon \log 2} = 
        \frac{\varepsilon \log 2 + 1}{\varepsilon \log 2}
    \] 

    What we want to bound is
    \[
\prod_{\substack{p^{\alpha} \mid  \mid n \\ p <
    2^{\frac{1}{\varepsilon}}}}
\frac{\alpha+1}{p^{\alpha \varepsilon}}
\] 
Note here that $p$ is bounded and
as $\alpha$ increases, we should expect the denominator to
take over. However, while $\alpha$ is small,
we might have some large terms since $p^{\varepsilon}$ might
be large. All our terms are however bounded by
$p^{\varepsilon}$ by the looks of it? Then we
would get that the product is
the product is bounded by
$\prod_{p < 2^{\frac{1}{\varepsilon}}} 
\frac{\log n}{\log p} \frac{1}{p^{\varepsilon}}$


\end{proof}

\begin{exercise}[E3.4]
    Show that
    \[
    \sum_{n=1}^{\infty} \frac{d(n)}{n^{s}}
    \] 
    is absolutely convergent in
    $\Re (s) > 1$.
\end{exercise}

\begin{proof}
    Fix some $s = \sigma + it \in \mathbb{C}$ with
    $\sigma > 1$. Then choosing an $\varepsilon > 0$ with
    $1 + \varepsilon < \sigma$, we have that
    $d(n) \le C_{\varepsilon} n^{\varepsilon}$, so
    \[
    \sum \left| \frac{d(n)}{n^{s}} \right| 
    \le \sum C_{\varepsilon} \frac{n^{\varepsilon}}{n^{\sigma}}
    \le C_{\varepsilon} \sum \frac{1}{n^{\sigma - \varepsilon}} < 
    \infty.
    \] 
\end{proof}

\begin{exercise}[E3.5]
    Show that the average order of
    $d(n)$ is $\log n$, i.e., that
    \[
    \frac{1}{x} \sum_{n \le x} d(n) = \log x +
    o\left( \log x \right) .
    \] 
    Hint: Show that
    \[
    \sum_{n \le x} d(n) = \sum_{a \le x}
    \left[ \frac{x}{a} \right] 
    \] 
    where $\left[ b \right] $ is the integer
    part of $b$.
\end{exercise}


\begin{proof}
    We follow the hint.
    For each $n \in \mathbb{N} $, let
    $D_n$ denote the set of positive divisors of
    $n$. Then we want to find
    $\left| D_1 \cup \ldots \cup D_{\left[ x \right] } \right|  $.
    Now, $\left[ \frac{x}{a} \right] $ is precisely
    the amount of multiples of $a$ smaller than or equal to
    $x$, i.e., the amount of numbers in between $1$ and
    $x$ which have $a$ as a divisor.
    Hence the right hand side indeed counts the
    number of divisors of the numbers less than or equal to
     $x$ which is precisely the left hand side.
     Now, recall also the bound
     \[
     \log x + \frac{1}{x} \le \sum_{a \le x}\frac{1}{a}
     \le \log x + 1
     \] 
     so \[
     1 + \frac{1}{x \log x} \le \frac{1}{\log x}
     \sum_{a \le x} \frac{1}{a} \le 1 + \frac{1}{\log x}.
     \] 
     In particular, taking the limit as
     $x \mapsto \infty$, the outer
     functions tend to $1$, so
     \[
     \lim_{x \to \infty} \frac{1}{\log x} \sum_{a \le x}\frac{1}{x}
     = 1.
     \] 
     In particular,
     \[
     \frac{1}{x \log x}\sum_{n\le x} d(n)
     \le \frac{1}{\log x} \sum_{a \le x} \frac{1}{a}
     \to 1, \quad x \to \infty.
     \] 
     For a lower bound, we have
     \[
     \frac{1}{\log x}  
     \sum_{a\le x} \frac{1}{a} -
     \frac{1}{x\log x} \sum_{a\le x}\frac{1}{a}
     = \frac{1}{\log x} \sum_{a \le x} \frac{x-1}{ax}
     \le \frac{1}{\log x} \sum_{a\le x} \left[ \frac{x}{a} \right] 
     \] 
     But
     \[
     \frac{1}{x} + \frac{1}{x^2 \log x}
     \le \frac{1}{x \log x} \sum_{a\le x} \frac{1}{a}
     \le \frac{1}{x} + \frac{1}{x \log x}
     \] 
     so letting $x \to \infty$, 
     \[
     \lim_{x \to \infty} \frac{1}{x \log x} \sum_{a \le x}
     \frac{1}{a} = 0
     \] 
     Hence also
     \[
     1 \le \lim_{x \to \infty} \frac{1}{x \log x}
     \sum_{n \le x} d(n) \le 1
     \] 
     giving the desired result.
\end{proof}

\begin{exercise}[E3.6]
    Let  \[
    \chi_4 (n) = 
    \begin{cases}
        (-1)^{\frac{n-1}{2}},& n \text{ odd}\\
        0,& n \text{ even}
    \end{cases}.
    \] 
    Show that $\chi_4$ is a Dirichlet character
    modulo $4$ and find
    $L\left( 1, \chi_4 \right) $. Use the value
    to give (yet another) proof- based on the
    irrationality of $\pi$ - that there
    are infinitely many primes.
    Hint: Remember (or prove by playing around with
    $\arctan (1)$ ) that
    \[
    \pi = 4 \sum_{n=1}^{\infty} \frac{(-1)^{n-1}}{2n-1}.
    \] 
\end{exercise}

\begin{proof}
    We must check $3$ criteria for
    $\chi_4$ to be a Dirichlet character mod $4$.\\
    (i) It must be $4$-periodic. Now
    if $n$ is even, then
    $n+4$ is even, so then
    $\chi_4(n+4) = 0 = \chi_4 (n)$.\\
    If $n$ is odd, then so is $n+4$, so
    \[
    \chi_4(n+4) = 
    \left( -1 \right)^{\frac{n+4-1}{2}}
    = \left( -1 \right)^{\frac{n-1}{2} + 2}
    = \left( -1 \right)^{\frac{n-1}{2}}
    = \chi_4 (n).
    \] 
    So $\chi_4$ is $4$-periodic.\\
    (ii) We must check that
    $\chi_4(n) = 0$ if and only if
    $\left( n,4 \right) \neq 1$.
    Now, $\chi_4(n) = 0$ if and only if $n$ is even
    if and only if $\left( n,4 \right) \in 
    \left\{ 2,4 \right\} $ if and only if
    $\left( n,4 \right) \neq 1$.\\
    (iii) We must check that
    $\chi_4$ is multiplicative.
    Indeed, if either
    $n$ or $m$ is even, then
    \[
    \chi_4 (nm) = 0 = \chi(n) \chi(m).
    \] 
    If both $n,m$ are odd, then
    \[
    \chi_4 \left( nm \right) 
    = \left( -1 \right)^{\frac{nm-1}{2}}
    = 
    \begin{cases}
        -1,& nm \equiv 3 \pmod{4}\\
        1,& nm \equiv 1 \pmod{4}
    \end{cases}
    \] 
    Now, if $n$ and $m$ are both
    equivalent to $3$ mod $4$, then
    their product is equivalent to
    $1$ mod 4, which works out.
    If only one is equivalent to $3$ mod $4$, then
    $nm$ is also, so it checks out, and
    similarly, if both are equivalent to $1$ mod $4$, then
    so is their product.
    Now, by definition,
    \[
    L(1,\chi_4) :=
    \sum_{n=1}^{\infty} \frac{\chi_4(n)}{n}
    = \sum_{n=0}^{\infty} \frac{\left( -1 \right)^{n}}{2n+1}
    = \arctan(1)
    = \frac{\pi}{4}
    \] 
    Now,
    since  $\chi_4 \neq \chi_0^{4}$, 
    we know that
    $L\left( s, \chi_4 \right) $ is convergent and
    analytic for $\Re (s) > 0$. In particular, it is
    continuous at $s = 1$. But for
    $\Re(s) > 1$, we know that
    $L(s, \chi_4) = 
    \prod_p \left( 1 - \chi_4(p)p^{-s} \right)^{-1}$, so
    by continuity,
    \[
    \frac{\pi}{4} = L\left( 1,\chi_4 \right) 
    = \prod_p \left( 1- \chi_4(p) p^{-1} \right)^{-1}
    \] 
    Now, all the terms in the product are
    rational, so by irrationality of $\pi$, this
    forces there to be infinitely
    many primes.
\end{proof}


\begin{exercise}[E3.7]
    Let $\left\{ a_n \right\} $ be a sequence of complex numbers
    satisfying that $\sum_{n \le x} a_n = O\left( x^{\delta}
    \right) $ for some $\delta > 0$. Prove that
    \[
    \sum_{n=1}^{\infty} \frac{a_n}{n^{s}} = s
    \int_{1}^{\infty} \sum_{n\le t} a_n \frac{1}{t^{s+1}} dt 
    \] 
    for $\Re(s) > \delta$, and that the sum converges
    to an analytic function in this region.
\end{exercise}

\begin{proof}
    Let $f(x) = x^{s}$. Then
    \[
    \sum_{n\le x} \frac{a_n}{n^{s}}
    = \sum_{n\le x}a_n \frac{1}{x^{s}} + s
    \int_{1}^{x} \sum_{n\le t} a_n \frac{1}{t^{s-1}} dt
    \] 
    when $s \neq 1$. But
    $\left| \sum_{n \le x}a_n \right| 
    \le C x^{\delta} $, so
    \[
    \left| \sum_{n\le x} a_n \frac{1}{x^{s}} \right|
    \le C x^{\delta - \sigma} \to 0, \quad x \to \infty
    \] 
    as $\delta - \sigma < 0$.
    Thus
    \[
    \sum_{n=1}^{\infty} \frac{a_n}{n^{s}} = 
    s \int_{1}^{\infty} \sum_{n\le t} a_n \frac{1}{t^{s+1} }dt . 
    \] 
\end{proof}


\section{Week 4}

\begin{exercise}[E4.1]
    Let $K \ge 0$. Prove that
    \[
    \log \left( K \left| t \right| +4 \right) =
    O\left( \log \left( \left| t \right| +4 \right)  \right) 
    \] 
    for $t \in \mathbb{R}$. Let
    $c_1, c_2, c_3 > 0$. Prove that there exists a constant
    $c_4$ such that for all $t \in \mathbb{R}$,
    \[
    c_1 + c_2 \log\left( \left| t \right| +4 \right) +
    c_3 \log\left( \left| 2t \right| +4 \right) 
    \le c_4 \log \left( \left| t \right| +4 \right) .
    \] 
\end{exercise}

\begin{proof}
    If $0 \le K \le 1$, then
    $\log \left( K \left| t \right| +4 \right) \le 
    \log \left( \left| t \right| +4 \right) $ by
    monotonicity of $\log$.
    So assume  $K > 1$. Then
    $\log \left( K \left| t \right| +4 \right) 
    = \log K + \log \left( \left| t \right|  + \frac{4}{K} \right) 
    \le \log K + \log \left( \left| t \right| +4 \right) $.
    Now $\log \left( \left| t \right| +4 \right) > 1$, so
    there exists some $C$ such that
    $C \log \left( \left| t \right| +4 \right) \ge \log K$.
    Hence
    $\log \left( K \left| t \right| +4 \right) =
    O \left( \log \left( \left| t \right| +4 \right)  \right) $.
    Since
    $c_1 + c_2 \log\left( \left| t \right| +4 \right) +
    c_3 \log \left( \left| 2t \right| +4 \right) $ is a
    sum of terms that are all
    $O\left( \log \left( \left| t \right| +4 \right)  \right) $, so
    is their sum, so the conclusion holds.
\end{proof}


\begin{exercise}[E4.2]
    Let $f(s)$ be a complex polynomial of degree $n$ with
    complex zeroes $z_1, z_2, \ldots, z_n$. Show that
    \[
    \frac{f'}{f}(z) = \sum_{i=1}^{n} \frac{1}{z- z_i}.
    \] 
    Consider how Lemma 6.3 is a generalization of this.
\end{exercise}

\begin{proof}
    Firstly, $f'$ is entire, so
    $\frac{f'}{f}$ is holomorphic on 
    $\mathbb{C} - \left\{ z_1, \ldots, z_n \right\} $.
    Now, by Theorem 6.1 in KomAn, there
    exist unique functions
    $g_i$ holomorphic on
    $\mathbb{C} - \left\{ z_1,\ldots,z_n \right\} $ such that
    $g_i(z_i) \neq 0$ and
     \[
     f(z) = \left( z-z_i \right)^{n_i} g_i(z)
     \] 
     where $n_i$ is the multiplicity of $z_i$.
     In particular,
     $f'(z) = 
     n_i (z-z_i)^{n_i-1}g_i(z) + (z-z_i)^{n_i} g_i'(z)$ which
     has $z_i$ a zero of order
     $n_i -1$. Hence
     $\frac{f'}{f}$ has $z_i$ as a simple pole.
     Applying the partial fraction decomposition
     to $\frac{f'}{f}$ (theorem 6.12 in KomAn), we get
     that
     \[
     \frac{f'}{f}(z) = 
     \sum_{i=1}^{n} \frac{c_i}{z- z_i}
     \] 
     for certain constants $c_i$.
     Now $\lim_{z \to z_i}  (z-z_i) \frac{f'}{f}(z)
     = n_i$. 
     Now, $f$ is of degree $n$ with $n$ distinct zeroes, so
     $n_i$ must be $1$ for each $i$.\\
     
     Now let us recall Lemma 6.3:
     \begin{lemma}[6.3]
         Let $f \colon B \to \mathbb{C}$ be analytic,
         $B \subset \mathbb{C}$ open, and assume
         \begin{enumerate}
             \item $\left\{ z  \mid \left| z \right| \le 1
                 \right\} \subset B$ 
             \item $\left| f(z) \right| \le M$ when 
                 $\left| z \right| \le 1$ 
             \item $f(0) \neq 0$.
         \end{enumerate}
         Let $0 < r < R < 1$. Then for
         $\left| z \right| < r$,
         \[
         \frac{f'}{f}(z) = 
         \sum_{\substack{f(z_k) = 0 \\ \left| z_k \right| \le R}} 
         \frac{1}{z-z_k} + 
         O \left( \log \frac{M}{\left| f(0) \right| } \right) 
         \] 
     \end{lemma}

         Note here that $f$ is not required to be
         a polynomial. However, since $f$ is holomorphic
         in $B$, it has an analytic representation on $B$, so
         essentially, Lemma 6.3 generalizes the
         representation to analytic functions. 
\end{proof}


\begin{exercise}[E4.3]
    Show that the Riemann zeta function $\zeta (s)$ has no
    zeroes for $\frac{1}{2} \le s < 1$.
\end{exercise}

\begin{proof}
    Recall that for $\sigma > 0$ and $s\neq 1$, we have
    \[
    \zeta (s) = \frac{s}{s-1} - s \int_{1}^{\infty} 
    \left( u - \left[ u \right]  \right) u^{-s-1}du.
    \] 
    For $s \in [\frac{1}{2},1)$, 
    $\frac{s}{s-1} \le  -1$. So we
    wish to show that
    \[
    s \int_{1}^{\infty} \left( u - \left[ u \right]  \right) 
    u^{-s-1} du > -1
    \] 
    But
    \[
        s \int_{1}^{\infty} 
        \left( u - \left[ u \right]  \right) u^{-s-1}du
    \] 
    is positive since the inner function and $s$ are
    both positive on $[1, \infty)$.

\end{proof}

\begin{exercise}[E4.4]
    Let $\chi$ be a Dirichlet character modulo $q$. Find the
    Dirichlet series representation for
    $L'(s, \chi) / L(s,\chi)$. Let $\chi_0$ be the trivial
    Dirichlet character modulo $q$. Prove that for
    $\sigma > 1, t \in \mathbb{R}$,
    \[
    R := \Re \left( -3 \frac{L'(\sigma, \chi_0)}{L(\sigma, \chi_0)}-
    4 \frac{L' (\sigma + it , \chi)}{L(\sigma + it, \chi)}-
\frac{L' (\sigma + i 2t, \chi^2) }{L(\sigma + i 2 t, \chi^2)}\right) 
\ge 0.
    \] 
\end{exercise}


\begin{proof}
    We want to represent
    $\frac{L' (s, \chi)}{L (s, \chi)}$ as a Dirichlet series.
    We imitate the idea for $\frac{\zeta'}{\zeta}$.

    \begin{align*}
        \frac{L'(s, \chi)}{L (s, \chi)}
        &= \frac{d}{ds} \log \left( L (s, \chi) \right) \\
        &=  - \sum_p \frac{d}{ds}
        \log \left( 1- \frac{\chi(p)}{p^{s}} \right) \\
        &= - \sum_p \frac{d}{ds} 
        \sum_{k=1}^{\infty} (-1)^{k+1} \left( - \frac{\chi(p)}{p^{s}}
        \right)^{k}\\
        &= \sum_p \sum_{k=1}^{\infty} \frac{d}{ds} 
        \left( \frac{\chi (p)}{p^{s}} \right)^{k}\\
        &= \sum_p \sum_{k=1}^{\infty} \chi(p)^{k} (-k \log p)
        p^{-sk}\\
        &= - \sum_p \sum_{k=1}^{\infty} k \log p \left( 
        \frac{\chi(p)}{p^{s}} \right)^{k}
    \end{align*}
    Thus
    We want to find 
    $\Re \left( \left( \frac{\chi(p)}{p^{s}} \right)^{k} \right) $.
    We have
    \begin{align*}
      \Re \left( \left( \frac{\chi(p)}{p^{s}} \right)^{k} \right) 
      &= \frac{1}{2} \left[ \left( \frac{\chi(p)}{p^{s}} \right)^{k} 
      + \left( \frac{\overline{\chi(p)}}{\overline{p^{s}}} \right)^{k}
  \right] \\
      &= 
    \end{align*}
    
    \[
    \Re \left( - \frac{L'(s,\chi)}{L(s,\chi)} \right) 
    = \sum_p \sum_{k=1}^{\infty} k \log p 
    \cos \left( tk \log p \right) .
    \] 
    So

\end{proof}


\begin{exercise}[E4.5]
    Let $\zeta(s)$ be the Riemann zeta function. Let
    $K$ be a compact subset of 
    $\left\{ s \in \mathbb{C}  \mid  
    \Re (s) > 0 \right\} $. Assume that
    $1 \in K$ and that $K$ does not contain any zeroes
    of $\zeta$. Show that
    \[
    - \frac{\zeta'}{\zeta}(s) = \frac{1}{s-1} + O(1)
    \] 
    for $s \in K - \left\{ 1 \right\} $. Show that there
    exists a constant $c >0$ such that for
    $0 < \delta < 1$,
    \[
    - \frac{\zeta'}{\zeta}(1+\delta) < \frac{1}{\delta} + c.
    \] 
\end{exercise}

\begin{proof}
    Since $1$ is a simple pole of
    $\frac{\zeta'}{\zeta}$ and 
    $K$ has no other zeroes of $\zeta$ and hence
    neither of $\zeta'$, we have that
    \[
    - (s-1) \frac{\zeta'}{\zeta}(s)
    \] 
    is holomorphic on $K$, hence bounded as $K$ is compact.
    Thus
    \[
    - \frac{\zeta'}{\zeta}(s) = \frac{1}{s-1} + O(1)
    \] 
    for $s \in K -\left\{ 1 \right\} $. Thus for small
    $0 < \delta < 1$ such that
    $1 + \delta \in K - \left\{ 1 \right\} $,
    \[
    - \frac{\zeta'}{\zeta}\left( 1+ \delta \right) 
    < \frac{1}{\delta} + c
    \] 
    for some $c > 0$.
\end{proof}


\begin{exercise}[E4.6]
    Use partial summation (Abel summation)
    to show that for $\sigma > 1$,
    \[
    - \frac{\zeta'}{\zeta}(s) = 
    s \int_{1}^{\infty} \frac{\psi (x)}{x^{s+1}} dx 
    \] 
    where $\psi (x) = \sum_{n\le x}
    \Lambda (n)$, and $\Lambda$ is the von Mangoldt function.
\end{exercise}

\begin{proof}
    Recall that
    \[
    - \frac{\zeta'}{\zeta}(s) = 
    \sum_{n=1}^{\infty} \frac{\Lambda (n)}{n^{s}}
    \] 
    for $ \sigma = \Re (s) > 1$.

    Let $f(x) = \frac{1}{x^{s}}$ and
    $a_n = \Lambda (n)$.
    Partial summation gives
    \[
    \sum_{n\le x} \frac{\Lambda (n)}{n^{s}}
    = \underbrace{\sum_{n\le x} \Lambda(n)}_{\psi (x)}
    \frac{1}{x^{s}}
    +s \int_{1}^{x}  \underbrace{\sum_{n\le t} \Lambda (n)}_{\psi (t)} \frac{1}{t^{s+1}} dt
    \] 
    By the prime number theorem,
    \[
    \psi (x) = x + O
    \left( \frac{x}{e^{c' \sqrt{\log x} }} \right) 
    \] 
    so
    \[
    \frac{\psi (x)}{x^{s}} \to 0, \quad x \to \infty
    \] 
    Thus
    \[
    - \frac{\zeta'}{\zeta} (s)
    = s \int_{1}^{\infty} \frac{\psi (t)}{t^{s+1}}dt 
    \] 
    for $\sigma > 1$.
\end{proof}



\section{Week 5}

\begin{exercise}[E5.1]
    Show that
    \[
    x \exp \left( -c \sqrt{\log x}  \right) 
    = O_m \left( \frac{x}{\log^{m} x} \right) 
    \] 
    for every $m$, and that
    \[
    x^{1-\varepsilon} = O_{\varepsilon}
    \left( x \exp \left( -c \sqrt{\log x}  \right)  \right) 
    \] 
    for every $\varepsilon > 0$. Discuss what this
    means for the quality of the error-term in the
    prime number theorem.
\end{exercise}


\begin{proof}
    \[
    \frac{\log^{m} x}{e^{c \sqrt{\log x} }}
    = \frac{ \sqrt{\log x}^{2m}}{e^{c \sqrt{\log x} }}
    \] 
    Now
    \begin{lemma}[]
        For any $a > 0$ and any  $b > 1$,
        \[
        \frac{x^{a}}{b^{x}} \to 0, \quad x \to \infty.
        \] 
    \end{lemma}
        Let $v = \sqrt{\log x} $. Then
        the above reads
        $\frac{v^{2m}}{e^{cv}}$.
        Assuming $c > 0$, we find that for
        $v \to \infty$,
        $\frac{v^{2m}}{e^{cv}} \to 0$.
        So in fact,
        \[
        x \exp \left( -c \sqrt{\log x}  \right) 
        = o \left( \frac{x}{\log^{m} x} \right) 
        \] 
        Now
        \[
        x^{1-\varepsilon} = x
        x^{-\varepsilon} = 
        x e^{- \log (x) \varepsilon}
        \le x e^{- c \sqrt{\log x} }.
        \] 
        Recall that we proved the following version of
        the prime number theorem:
        \begin{theorem}[Prime number theorem]
            There exists a $c' > 0$ such that
            \[
            \psi (x) =
            x + O \left( x
            \exp \left( -c' \sqrt{\log x}  \right) \right) 
            \] 
        \end{theorem}
        So by the above,
        \[
        \psi (x) = x +
        O_m \left( \frac{x}{\log^{m}(x)} \right) 
        \] 
        So essentially, the error
        term is smaller than
        $\frac{x}{\log^{m}(x)}$ for any $x$ but still
        larger than $x^{1-\varepsilon}$ for any
        $\varepsilon > 0$.
\end{proof}

\begin{exercise}[E5.2]
    Prove that the following two statements are equivalent:
    \begin{enumerate}
        \item There exists a $c >0$ such that
            \[
            \psi (x) = x + O\left( x \exp
            \left( -c \sqrt{\log x}  \right) \right) 
            \] 
        \item There exists a $c > 0$ such that
            \[
            \pi(x) = li(x) + O \left( x \exp
            \left( - c \sqrt{\log x}  \right) \right) 
            \] 
            where $li (x) = \int_{2}^{x} \frac{1}{\log t} dt $.
    \end{enumerate}
\end{exercise}

\begin{proof}
    Suppose $(1)$ is true. Then
    \begin{align*}
        \pi(x) 
        &=
    \frac{\psi (x)}{\log x} + \int_{2}^{x} 
    \frac{\psi (t)}{t \log^2 t} dt + O \left( x^{\frac{1}{2}}
    \log x\right)\\
        &= \frac{x}{\log x} + O\left( \frac{x}{\log x}
        \exp\left( -c \sqrt{\log x}  \right) \right) 
        + \int_{2}^{x} \frac{1}{\log^2 t} 
        + O\left( \frac{1}{\log^2 t \exp\left( c \sqrt{\log t} 
        \right) } \right) dt
        + O\left( x^{\frac{1}{2}} \log x \right) 
    \end{align*}
    Now
    \[
    \int_{2}^{x} \frac{1}{\log^2 t} dt
    = - \frac{t}{\log t} \bigg|_{2}^{x} 
    + li(x) 
    \] 
    giving
    \[
    \pi(x) = li(x) + \frac{2}{\log 2}
    + O\left( \frac{x}{\log x} e^{-c \sqrt{\log x} } \right) 
    + \int_{2}^{x} 
    O\left( \frac{e^{- c \sqrt{\log t} }}{\log^2 t } \right) dt
    + O \left( x^{\frac{1}{2}} \log x \right) 
    \] 
    All the middle terms apart from the
    last two are clearly
    $O \left( x e^{-c \sqrt{\log x} } \right) $.
    To take care of the last term, we use the lemma:
    \begin{lemma}[]
        For any $a > 0$,
        \[
        \frac{\log x}{x^{a}} \to 0, \quad x \to \infty
        \] 
    \end{lemma}
    Hence
    $x^{\frac{1}{2}} \log x 
    = O\left( x^{\frac{3}{4}} \right) 
    = O\left( x e^{-c' \sqrt{\log x} } \right) $.\\

    For the last part
    \begin{align*}
        \int_{2}^{x} O \left( \frac{e^{-c \sqrt{\log t} }}{\log^2 t}
        \right)dt
        &\le 
    \end{align*}


Note that the derivative of
$x e^{-c \sqrt{\log x} }$ is 
\[
e^{-c \sqrt{\log x} } - c \frac{d}{dx} \left[ \sqrt{\log x} 
\right] e^{-c \sqrt{\log x} }
= e^{-c \sqrt{\log x} }
-c \frac{1}{2} \frac{1}{x} \frac{1}{\sqrt{\log x} }
e^{-c \sqrt{\log x } }
\] 
But as
$x \to \infty$, this grows
faster than
$\frac{e^{-c} \sqrt{\log x} }{\log^2 x}$, which
is what we wanted.\\
\linebreak
Now we want to show that
$(2)$ implies $(1)$. So assume
there exists a $c> 0$ such that
\[
\pi(x) = li(x) + O\left( x \exp\left( -c \sqrt{\log x} 
\right) \right) .
\] 
Then recall that
\[
\psi (x) = 
\pi(x) \log x - \int_{2}^{x} \frac{\pi(t)}{t} dt
- O\left( x^{\frac{1}{2}} \log^2 x \right) 
\] 
So
\begin{align*}
    \psi (x)
    &= li(x) \log x + \log x O\left( xe^{-c \sqrt{\log x} } \right) 
    - \int_{2}^{x} \frac{li (t)}{t} dt
    - \int_{2}^{x} O\left( e^{-c \sqrt{\log t} } \right)dt
    - O\left( x^{\frac{1}{2}} \log^2 x \right) \\
\end{align*}
Now, by repeated integration by parts, we get
\begin{align*}
    li(x) 
    &= \frac{t}{\log t} \bigg|_{2}^{x}
    + \int_{2}^{x} \frac{1}{\log^2 t} dt\\
    &= \frac{t}{\log t}\bigg|_{2}^{x}
    + \left[ \frac{t}{\log^2 t} \bigg|_{2}^{x}
    + 2 \int_{2}^{x} \frac{1}{\log^3 t}dt \right] \\
    &= \frac{t}{\log t} + \frac{t}{\log^2 t}\bigg|_{2}^{x}
    + 2 \left[ \frac{t}{\log^3 t} \bigg|_{2}^{x}
    +  3 \int_{2}^{x} \frac{1}{\log^{4} t} dt  \right] \\
    &= x \sum_{r=1}^{k-1} \frac{(r-1)!}{\log^{r} x}
    + (k-1)! \int_{2}^{x}  \frac{1}{\log^{k} t} dt 
\end{align*}

\end{proof}




\begin{exercise}[E5.3]
    Let $f$ be a Schwartz function on the real line, and
    let $\hat{f}$ be its Fourier transform. Show that
    \[
        \sum_{n \in \mathbb{Z}} f \left( \frac{v+n}{t} \right) 
    = \sum_{n \in \mathbb{Z}} \left| t \right| 
    \hat{f}\left( nt \right) e^{2 \pi i n v}.
\]
\end{exercise}

\begin{proof}
    For a Schwartz function $f$, we know from the
    Poisson summation formula that
    \[
    \sum_{n \in \mathbb{Z}} f(n)
    = \sum_{n \in \mathbb{Z}} \hat{f}(n),
    \] 
    where
    \[
    \hat{f}(y) = \int_{-\infty}^{\infty} 
    e^{-2 \pi i xy} f(x) dx
    \] 
    Define
    $g(x) = f\left( \frac{v+x}{t} \right) $. Then
    $g$ is also a Schwartz function, so
    \[
    \sum_{n\in \mathbb{Z}} g(n) = 
    \sum_{n \in \mathbb{Z}}
    \int_{-\infty}^{\infty} e^{-2 \pi i n x}
    f\left( \frac{v+x}{t} \right) dx
    \] 
    Let
    $z = \frac{v+x}{t}$. Then
    $dz = \frac{1}{\left| t \right|} dx$, so
    
    \[
    \sum_{n \in \mathbb{Z}}
    f\left( \frac{v+n}{t} \right) 
    = \sum_{n \in \mathbb{Z}}g(n)
    = \sum_{n \in \mathbb{Z}}
    \left| t \right|  e^{-2 \pi i n \left( tz-v \right) }
    f(z) dz
    = \sum_{n \in \mathbb{Z}}
    \left| t \right| \hat{f}(nt) e^{2\pi i n v}
    \] 
\end{proof}


\begin{exercise}[E5.4]
    Let $\theta > \frac{1}{2}$. Prove that if
    for every $\varepsilon > 0$, 
    $\psi (x) = x + O\left( x^{\theta + \varepsilon} \right) $,
    then the Riemann zeta function has
    no zeroes in $\Re (s) > \theta$. 
    (It turns out that this is in fact
    an 'if and only if statement'). Think about
    what this implies for the Riemann hypothesis. Compare
    with the zerofree region provided by
    Theorem 6.6.
\end{exercise}

\begin{proof}
    By the explicit formula, if we simply let
    $x$ range among $\mathbb{R} - \mathbb{Z}$, then
    we have
    \[
    O\left( x^{\theta + \varepsilon} \right) 
    = \lim_{T \to \infty} 
    \sum_{\substack{\zeta(\rho) = 0\\ \left| \im \rho \right| 
    \le T}} \frac{x^{\rho}}{\rho }
    + \frac{\zeta'}{\zeta}(0) + \frac{1}{2} 
    \log \left( 1 - \frac{1}{x^2} \right) ,
    \] 
    however, if there is a  $\rho $ with
    $\Re (\rho) > \theta$, then choosing
    $\varepsilon$ such that
    $\theta < \varepsilon < \Re (\rho)$, we get
    that the right hand side grows faster, giving
    a contradiction.\\
    \linebreak
    Hence the Riemann hypothesis can be
    reformulated as saying that
    for any $\varepsilon > 0$,
    \[
    \psi (x) = x + O\left( x^{\frac{1}{2}+ \varepsilon} \right) .
    \] 
    Now, any $\theta$ would, of course, be a very strong improvement
    combined with the zero-free region. This is because
    the zero-free region tapers off as the imaginary
    part grows in size, while
    finding a $\theta$ such that the
    above holds would imply, as shown, that
    we can shrink the critical strip to
    a narrower strip.
\end{proof}

\begin{exercise}[E5.5]
    Let $p_n$ be the $n$ th prime. Show that
    \[
    \frac{1}{N} \sum_{n=1}^{N} \frac{p_{n+1}- p_n}{\log p_n}
    \to 1
    \] 
    as $N \to \infty$, and discuss how to interpret
    this as a statement about the average
    spacing between adjacent primes.
\end{exercise}


\begin{proof}
    By Abel summation, we have

    \[
    \sum_{n\le x}
    \frac{p_{n+1}- p_n}{\log p_n}
    = 
    \frac{p_{\left[ x \right] +1}-2}{\log x}
    - \int_{1}^{x} \frac{p_{\left[ t \right] +1}-2}{\log t}dt 
    \] 


    \[
    \sum_{n\le x} \frac{p_n}{\log p_n}
    = \sum_{n\le x} p_n \frac{1}{\log x}
    - \int_{1}^{x} \sum_{n\le t} p_n \frac{1}{\log t}dt 
    \] 
    And similarly
    \[
    \sum_{n\le x} \frac{p_{n+1}}{\log p_n}
    = \sum_{n\le x} p_{n+1} \frac{1}{\log x}
    - \int_{1}^{x} \sum_{n\le t} p_{n+1} \frac{1}{\log t}dt 
    \] 
    Hence
    \begin{align*}
    \sum_{n\le x} \frac{p_{n+1} - p_n}{\log p_n}
    &= \frac{1}{\log x} \sum_{n\le x} p_{n+1} -p_n
    - \int_{1}^{x} \frac{1}{\log t} \sum_{n \le t}(p_{n+1}-p_n) dt\\
    &= \frac{p_{\left[ x \right] +1} - 2}{\log x} 
    - \int_{1}^{x} \frac{p_{\left[ t \right] +1} - 2}{\log t} dt
    \end{align*}
    Now
    $\frac{p_n}{n \log n} \to 1$ as $n \to \infty$, so
    $\frac{p_{n+1}}{n \log n}
    = \frac{p_{n+1}}{(n+1) \log(n+1)} \frac{(n+1) \log(n+1)}{
    n \log n}
    \to 1$ as $n \to \infty$. So we will get
    the result if we can show that
    \[
    \lim_{n \to \infty} \frac{1}{n} 
    \int_{1}^{n} \frac{p_{\left[ t \right] +1}-2}{\log t} dt 
    = 0.
    \] 

    By the PNT, we have
    \[
    p_n \sim n \log n.
    \] 
    So
    \[
    \lim_{N \to \infty} \frac{1}{n} \sum_{n \ge N}
    \frac{p_{n+1}-p_n}{\log p_n}
    = \lim_{N\to \infty}
    \sum_{n\ge N} \frac{(1+\frac{1}{n})\log(n+1) - \log n}{\log n + 
    \log \log n}
    = 
    \] 
\end{proof}


\section{Week 6}

\begin{exercise}[E6.1]
    Show, using Corollary 10.3, that
    $\zeta(s)$ admits meromorphic continuation
    to $s \in \mathbb{C}$. Show that
    $\zeta\left( -2n \right) =0$ for
    $n \in \mathbb{N} $ (we call these
    zeroes \textit{trivial}), and that
    all the non-trivial zeroes of
    $\zeta(s)$ lie in
    $0 < \Re(s) < 1$.
\end{exercise}

\begin{proof}
    Here is Corollary 10.3:
    \begin{corollary}[10.3]
        The function
        \[
        \xi(s) = \frac{1}{2} s(s-1) \zeta(s)
        \Gamma \left( \frac{s}{2} \right) 
        \pi^{-\frac{s}{2}}
        \] 
        is entire, and
        $\xi(s) = \xi (1-s)$ for all $s$.
    \end{corollary}
    Hence we can define $\zeta$ as
    \[
    \zeta(s) = \frac{ 2 \pi^{\frac{s}{2}} \xi(s)}{s (s-1)
    \Gamma (\frac{s}{2})}
    \] 
    But using this, we see that
    as a product and quotient of meromorphic functions, 
    $\zeta$ is also meromorphic. In particular,
    we know that $\Gamma (s)$ has poles
    at $0, -1,-2,-3,\ldots$, which implies that
    $\zeta$ has zeroes at
    $-2, -4, -6, \ldots$ (to see this formally, use
    Theorem 6.1 in KomAn). These are the
    trivial zeroes. Now, by the Euler product
    of $\zeta$, it has no zeroes 
    in $\Re(s) > 1$. We also claim that
    $\zeta$ has no other zeroes apart from the
    nontrivial zeroes on $\Re(s) < 0$.
    Indeed, the poles from $\Gamma$ has already been
    accounted for, and
    note that $\xi$ has no zeroes in
    $\Re(s) < 0$ since then it would, by
    its functional equation, have zeroes in
    $\Re(s) > 0$, thus giving a zero
    in $\Re(s) > 0$ to $\zeta$ also. 
    Furthermore, it has no
    zeroes on $\Re(s) = 1$ by the zero-free region, and
    by symmetry of the functional equation stemming
    from the symmetry of $\xi$, it also has no
    zeroes on $\Re(s) = 0$.
\end{proof}

\begin{exercise}[E6.2]
    Find all poles of $\zeta(s)$, and show that if
    $\rho$ is a non-trivial zero
    of $\zeta (s)$ then
    $1- \rho $ and $\overline{\rho }$ are zeros
    of $\zeta(s)$.
\end{exercise}

\begin{proof}
    Since $\xi$ is analytic, all poles must
    stem from zeroes of
    $s(s-1) \Gamma (\frac{s}{2})$. Now,
    $\Gamma$ has no zeroes by a theorem, hence
    it does not give rise to any poles, so the only
    possible poles are
    $0$ and $1$. We know that it has
    a pole at $1$ by its series representation/Landau's lemma, and
    thus by the functional equation of
    $\xi$, since
    $\xi(1) = \xi(0)$ and
    $\xi(1) \neq 0$ since $(s-1)$ cancels with the
    pole from $\xi$, then also
    $\xi(0)\neq 0$, so
    $s$ must cancel. Now, $\Gamma$ has a simple pole at $0$, hence
    this cancels with  $s$.
    Since $\xi$ has no zeroes, $\zeta$ can therefore not
    have a pole at $0$.\\
    Concluding, $\zeta$ only has a single pole at $1$ which
    is simple.\\
    \linebreak
    Now, if $\rho$ is a non-trivial zero,
    then it lies in
    $0 < \Re (s) < 1$, so also
    $1 - \rho \in 0 < \Re (s) < 1$. 
    In particular, since $\rho$ lies in the critical
    strip, it does not arise from a pole
    of $\Gamma$, hence it must arise from a zero
    of $\xi (s)$, and thus also
    $\xi \left( 1- \rho  \right) = \xi(\rho) = 0$ by the
    functional equation. This shows that
    $1- \rho $ is then also a zero. 
    Now, note that
    on $\zeta$ is at least real valued
    on $\left( 0,1 \right) $ since
    on here, $\zeta$ has the representation
    \[
    \zeta(s) = \frac{s}{s-1} + 
    s \int_{1}^{\infty} \left( \left[ t \right] - t \right) 
    t^{-s-1} dt
    \] 
    which in particular, is real valued on
    $(0,1)$.
    Thus $\zeta$ is reflection invariant everywhere by
    exercise 6.3 in KomAn. I.e.,
    $\overline{\zeta(s)}
    = \zeta\left( \overline{s} \right) $. 
    So if $\rho $ is a zero, then
    so is $\overline{\rho }$.
\end{proof}

Recall that
$\theta (u) = 
\sum_{n \in \mathbb{Z}}
e^{-\pi n^2 u}$ for $\Re (u) > 0$.

\begin{exercise}[E6.3]
    Show that $\theta (u) = 1 + O \left( e^{-\pi u} \right) $ 
    for $u \in (1, \infty)$.
\end{exercise}

\begin{proof}
    We have
    $\theta (u) - 1 = 2 \sum_{n =1}^{\infty}
    e^{- \pi n^2 u}
    \le 2 \sum_{n=1}^{\infty}\frac{1}{2^{n-1}} e^{-\pi u}
    = 4 e^{- \pi u}$ since
    $\frac{1}{e^{\pi u (n^2 -1)}}
    \le \frac{1}{2^{n-1}}$ for $u>1$ and
    $n>1$.
\end{proof}

\begin{exercise}[E6.4]
    Show that for $u$ real,
    $\theta (u) \sim 
    \sqrt{u}^{-1} $ when $u \to 0$.
\end{exercise}

\begin{proof}
    Saying that $\theta(u) \sim 
    \sqrt{u}^{-1} $ as $u \to 0$ is the statement that
    $\sqrt{u} \theta (u) \to 1$ as $u \to 0$. But
    recall that
    $\theta (\frac{1}{u}) = 
    \sqrt{u} \theta (u)$. So the statement is
    precisely that
    $\theta \left( \frac{1}{u} \right) \to 1$ as
    $u \to 0$. Now
    \[
    \theta \left( \frac{1}{u} \right) 
    =1+ 2 \sum_{n=1}^{\infty} e^{-\pi n^2 \frac{1}{u}}
    \] 
    and
    \[
    \lim_{u \to 0} \theta \left( \frac{1}{u} \right) 
    = 1+ 2 \sum_{n=1}^{\infty} \lim_{u \to 0}
    e^{- \pi n^2 \frac{1}{u}}
    = 1.
    \] 
    Interchange of limit with sum here
    is possible because of the dominated convergence theorem
    (note that sums are just integrals with respect to the
    counting measure on $\mathbb{N} $ ).
\end{proof}


\begin{exercise}[E6.5]
    Show that if $\Re(s) > 1$, then
    $\int_{0}^{\infty} \left( \theta(u)-1 \right) 
    u^{\frac{s}{2}-1} du$ is convergent.
\end{exercise}

\begin{proof}
    As a function of $s$,
    $\left( \theta (u)-1 \right) 
    u^{\frac{s}{2}-1}$ is holomorphic for
    $\Re (s) > 1$, so by theorem 4.20 in KomAn,
    \[
    \int_{a}^{b} \left( \theta (u)-1 \right) u^{\frac{s}{2}-1}du
    \] 
    is convergent and holomorphic for
    $a,b \ge  1$. Now, since
    $\theta (\frac{1}{u}) = \sqrt{u} \theta(u)$, it
    follows that $\theta$ is also convergent
    on $(0,1]$.
    It remains to show that the integral is convergent when we take
    $a\to 0+$ and $b \to \infty$.
    Let $a > 1$. Then
    \[
    \int_{a}^{b} \left( \theta (u)-1 \right) 
    u^{\frac{s}{2}-1} du
    =\int_{a}^{b} O \left( e^{-\pi u} u^{\frac{s}{2}-1} \right) du 
    \] 
    But
    $O \left( e^{-\pi u}u^{\frac{s}{2}-1} \right) 
    = O \left( \frac{1}{u^2} \right) $, so
    \[
    \int_{a}^{b} O \left( e^{-\pi u}u^{\frac{s}{2}-1} \right) du
    \le - \frac{1}{u} \bigg|_{a}^{b}
    = \frac{1}{a}- \frac{1}{b}
    \to \frac{1}{a}, \quad \text{as } b\to \infty
    \] 
    Since the integral is also a monotone function, it converges.
    Secondly, we must show that
    \[
    \lim_{a \to 0+} \int_{a}^{\infty}  
    \left( \theta (u)-1 \right) u^{\frac{s}{2}-1} du
    \] 
    exists.

    But $\theta (u) \sim \sqrt{u}^{-1} $ as $u \to 0$, so
    for some $\delta > 0$, we have
    \[
    \left| \int_{0}^{\delta} \left( \theta(u)-1 \right) 
    u^{\frac{s}{2}-1} du \right| 
    \le \left| \int_{0}^{\delta} 
    \left( \frac{1}{\sqrt{u} } - 1 \right) 
    u^{\frac{s}{2}-1} du\right| + \varepsilon
    < \infty
    \] 
    because $\Re (s) > 1$.
    Hence the integral converges.
\end{proof}

\begin{exercise}[E6.6]
    Find the value $\zeta (0)$.
\end{exercise}

\begin{solution}
    We have
    \[
    \zeta(0) = 
    \frac{\xi (s)}{-\frac{1}{2} s(1-s) \pi^{-\frac{s}{2}}
    \Gamma(\frac{s}{2})} \bigg|_{s=0}
    \] 
    Now, the pole at $0$ for $\Gamma$ cancels with $s$, so
    since $s \Gamma(s) = \Gamma(s+1)$, we have
    $\Gamma\left( \frac{s}{2} \right) 
    = \Gamma \left( \frac{s}{2}+1 \right) \frac{2}{s}$, so
    we find
    \[
    \zeta(0) = 
    \frac{\xi (0)}{-\frac{1}{2} \Gamma(1) 2}
    = - \frac{\xi(0)}{\Gamma(1)} = -\xi(0)
    = - \xi (1)
    = \frac{1}{2} s (1-s) \pi^{-\frac{s}{2}}\Gamma(\frac{s}{2})
    \zeta(s) \bigg|_{s=1}
    \] 
    Now
    in this case, $(1-s)$ cancels with the pole of
    $\zeta$ at $1$. 
    Now, around $1$, we have
    \[
        (1-s)\zeta(s) = 
        -s + s (1-s) 
        \int_{1}^{\infty} \left( \left[ t \right] -t \right) 
        t^{-s-1} dt
    \] 
    so $(1-s)\zeta(s) \bigg|_{s=1} = -1$.
    Hence
    \[
    \zeta(0) = -\frac{1}{2} \pi^{-\frac{1}{2}}
    \Gamma(\frac{1}{2})
    \] 
    But $\Gamma(\frac{1}{2}) = \pi^{\frac{1}{2}}$, so
    \[
    \zeta(0) = - \frac{1}{2}.
    \] 
    \qed
\end{solution}

\begin{exercise}[E6.7]
    The table
    \begin{table}[htpb]
        \centering
        \caption{caption}
        \label{tab:label}
        \begin{tabular}{c | c c c c c c c c c c c c c c}
            n & 1 & 2 & 3 & 4 & 5 & 6 & 7 & 8 & 9 & 10 & 11 & 12
              & 13 & 14 \\
              \hline
            $\chi(n)$ & 1 & 0 & -1 & 0 & -1 & 0 & 0 & 0 & 1 & 0 & 1
            & 0 & -1 & 0
        \end{tabular}
    \end{table}
    defines a Dirichlet character 
    $\chi$ mod $14$. Find the
    conductor $d$ of $\chi$ and find a primitive
    $\chi_1$ mod $d$ such that
    $\chi = \chi_1 \chi_0^{14}$.

\end{exercise}

\begin{solution}
    The conductor is the smallest pseudo-period.
    Recall that the conductor must divide the period, so
    $d  \mid 14$. Hence there are $4$ possibilities for
    the value of $d$: $1, 2, 7, 14$.
    However, since  $\chi$ takes both the values
    $1$ and $-1$, it cannot be $1$.
    Now, also $1 \equiv 3 \pmod{2}$ and
    $(3,14) = 1$, however,
    $\chi(1) = 1 \neq -1 = \chi(3)$, so also
    $2$ is not a pseudo-period. Now, for any
    two numbers equivalent modulo $7$ in the list, one
    is even while the other is odd, hence
    the product cannot be relatively prime to $14$.
    This implies that the condition for being
    a pseudo-prime is trivially satisfied, so
    $7$ is the conductor.
    Next, we must find a primitive
    $\chi_1$ mod $7$ such that
    $\chi = \chi_1 \chi_0^{14}$.
    Recall that the definition of $\chi_1$ is essentially
    forced to be
    $\chi_1 (n) = \chi(n)$ whenever
    $(n,14) = 1$ which is for
    $1,3$ and $5$ modulo $7$.
    For $0,2,4,6$ modulo  $7$, we let
    $\chi_1 (n) = \chi(n+7k)$ such that
    $(n+7k, 14) = 1$, so either
    $k = 0$ or $k= 1$.
    Hence $\chi_1 (0) = 0,
    \chi_1 (2) = \chi(9) = 1,
    \chi_1(4) = \chi (11) = 1,
    \chi_1(6) = \chi(13) = -1$, and then
    extend this $7$-periodically. 
    Then $\chi = \chi_1 \chi_0^{14}$.
    \qed
\end{solution}

\begin{exercise}[E6.9]
    Show that a character $\chi$ mod $q$ is real,
    i.e., has real values, if and only if
    $\chi^2 = \chi_0^{q}$. We call a character
    \textit{quadratic} if
    $\chi^2 = \chi_0^{q}$, but
    $\chi \neq \chi_0^{q}$. Show that if
    $\chi$ mod $q$ is real and $\chi(-1) = 1$, then the
    Gauss symbol $\tau (\chi)$ is also real.
    What happens if $\chi(-1) = -1$?
\end{exercise}

\begin{solution}
    We have
    $\chi \overline{\chi} = \chi_0^{q}$ since
    $\overline{\chi}$ forms the inverse of
    $\chi$ in the group of Dirichlet characters modulo
    $q$. If $\chi$ is real, then
    clearly $\chi^2 = \chi \overline{\chi} = 
    \chi_0^{q}$. Conversely, if
    $\chi^2 = \chi_0^{q} = \chi \overline{\chi}$, then
    since these form a group, taking the inverse on both sides,
    we get $\chi = \overline{\chi}$, hence
    $\chi$ is real-valued.\\
    Now, recall that
    \[
    \overline{\tau (\chi)} = \chi(-1)
    \tau\left( \overline{\chi} \right) ,
    \] 
    so if
    $\chi$ is real and $\chi(-1) = 1$, then
    we simply get
    $\overline{\tau(\chi)} = 
    \tau(\chi)$, so
    $\tau(\chi)$ is real. If
    $\chi(-1) = -1$, we get
    $\overline{\tau(\chi)} = 
    - \tau(\chi)$ which means that
    $\tau(\chi)$ is purely imaginary.
\end{solution}


\section{Week 7}

\begin{exercise}[E7.1]
    Let $\chi \mod q$ be the Legende symbol. I.e.,
     \[
    \chi(m) =
    \begin{cases}
        0, & \text{if } 5  \mid m\\
        1, & \text{if } m \equiv a^2 \mod 5 \\
        -1,& \text{if } m \not \equiv a^2 
        \mod 5
    \end{cases}
    \] 
    Show that $\chi$ is primitive and calculate
    the Gauss sum $\tau(\chi)$.
\end{exercise}

\begin{solution}
    A character is primitive if and only if
    its conductor is its period. Here the
    period of $\chi$ is $5$.
    Since the conductor is a divisor of the period, the
    conductor is either $1$ or $5$. Since
    $\chi$ takes both values $1$ and $-1$, its
    conductor must be $5$, hence $\chi$ is primitive.\\
    To calculate the Gauss sum, we have
    \begin{align*}
    \tau(\chi) =
    \sum_{a=1}^{5} \chi(a) e\left( \frac{a}{5} \right) 
    &= e(\frac{1}{5}) - e(\frac{2}{5}) -
    e\left( \frac{3}{5} \right) +
    e\left( \frac{4}{5} \right) \\
    &= -1 - 2 \left[ e^{2 \pi i \frac{2}{5}}
    + e^{2\pi i \frac{3}{5}}\right] \\
    &= -1 - 4 \cos \left( \frac{4 \pi }{5} \right)\\
    &=\sqrt{5} 
    \end{align*}
\end{solution}

\begin{exercise}[E7.2]
    Let $\chi$ be a primitive Dirichlet character
    modulo $q > 1$. Show that
    \[
    L \left( \frac{1}{2}, \chi \right) =
    O \left( q^{\frac{1}{4}} \sqrt{\log q}  \right) .
    \] 
\end{exercise}

\begin{proof}
    We have
    \begin{align*}
        L \left( \frac{1}{2},\chi \right) 
        &= \sum_{n =1}^{\infty} \frac{\chi(n)}{\sqrt{n} }\\
        &= \sum_{n\le x} \frac{\chi(n)}{\sqrt{n} }
        + \sum_{n>x} \frac{\chi(n)}{\sqrt{n} }
    \end{align*}

    Firstly, we have
    \[
    \sum_{n\le x} \frac{\chi(n)}{\sqrt{n} }
    = \sum_{n\le x}\chi(n) \frac{1}{\sqrt{x} } + \frac{1}{2} 
    \int_{1}^{x} \sum_{n\le t}\chi(n) t^{-\frac{3}{2}}  dt
    \] 
    Now, using Polya-Vinogradov, we have
    \[
    \sum_{n\le x} \chi(n) = O\left( \sqrt{q} 
    \log q \right) 
    \] 
    so for $x$ sufficiently large, we have
    $\sum_{n\le x} \chi(n) \frac{1}{\sqrt{x} }
    \le q^{\frac{1}{4}} \sqrt{\log q} $.
    Now
    \[
    - \sqrt{q}  \log q
    \frac{1}{\sqrt{t} } \bigg|_{1}^{x}
    =
    \sqrt{q} \log q - \sqrt{q}  \log q \frac{1}{\sqrt{x} }
    \] 


    Alternatively,
    \begin{align*}
        \sum_{n>x} \frac{\chi(n)}{\sqrt{n} }
        &= \sum_{n\le x} \chi(n) \frac{1}{\sqrt{x} }
        \bigg|_{x}^{\infty} + \frac{1}{2}
        \int_{x}^{\infty} \sum_{n\le t} \chi(n) t^{-\frac{3}{2}} dt\\
        &=  - \sum_{n\le x}\chi(n) \frac{1}{\sqrt{x} }
        -  O\left( \sqrt{q} \log q \right)
        \left[ \frac{1}{\sqrt{t} } \right]_{x}^{\infty}\\
        &= O \left( \sqrt{q}  \log q \frac{1}{\sqrt{x} } \right)  
    \end{align*}

    \[
    \sum_{n\le x} \frac{\chi(n)}{\sqrt{n} }
    \le 
    \int_{1}^{x} \frac{\chi \left( \left[ t \right]  \right) }{
    \sqrt{t-1} } dt 
\] 




\end{proof}

\begin{exercise}[E7.3]
    Let $f$ be a sufficiently nice real function on
    $\mathbb{R}$, e.g., a Schwartz function. Show that
    the Fourier transform of $f'$ is
    $2\pi i y  \hat{f}(y)$. Let
    $u > 0$. Show that the Fourier transform of
    $xe^{- \pi u \left( qx \right)^2}$ is 
    \[
    - \frac{iy}{\left( u^{\frac{1}{2}}q \right)^3 } 
    e^{-\pi u^{-1} \left( \frac{y}{q} \right)^2}.
    \] 
\end{exercise}

\begin{proof}
    \begin{align*}
        \hat{f'}(y) 
        &= \int_{-\infty}^{\infty} 
        e^{-2 \pi i x y} f'(x) dx \\
        &= \underbrace{e^{- 2 \pi i xy}}_{
        \text{modulus }1}f(x)  \bigg|_{-\infty}^{\infty}
        + \left( 2\pi i y \right) 
        \underbrace{\int_{-\infty}^{\infty}  
        e^{-2 \pi i xy}f(x) dx}_{= \hat{f}(y)}.
    \end{align*}

    Now,
    let 
    $f(x) = - \frac{1}{2\pi u q^2} e^{-\pi u \left( qx \right)^2}$.
    Then $f'(x) = x e^{-\pi u \left( qx \right)^2}$.
    So by the above
    $\hat{f'}(y) = 
    2 \pi i y \hat{f}(y)$.
    Now
    \begin{align*}
    \hat{f'}(y) 
    &=-\frac{2 \pi i y}{2\pi u q^2} \int_{-\infty}^{\infty}  
    e^{- 2 \pi i x y} e^{-\pi u \left( qx \right)^2} dx\\
    &= - \frac{iy}{u q^2}
    e^{-\pi \left( \frac{y}{q \sqrt{u}  } \right)^2}
    \int_{-\infty}^{\infty} 
    e^{- \pi \left[ u \left( qx \right)^2 + 2i xy -
    \left(  \frac{y}{q \sqrt{u}  } \right)^2  \right] } dx
    \end{align*}
    Let
    $w = \left( \sqrt{u} qx + \frac{iy}{q \sqrt{u} } \right) $.
    Then $dw = \sqrt{u} q dx$, so

    \begin{align*}
     - \frac{iy}{u q^2}
    e^{-\pi \left( \frac{y}{q \sqrt{u}  } \right)^2}
    \int_{-\infty}^{\infty} 
    e^{- \pi \left[ u \left( qx \right)^2 + 2i xy -
    \left(  \frac{y}{q \sqrt{u}  } \right)^2  \right] } dx
    &=
     - \frac{iy}{u^{\frac{3}{2}} q^3}
    e^{-\pi \left( \frac{y}{q \sqrt{u}  } \right)^2}
    \underbrace{\int_{-\infty}^{\infty} 
    e^{- \pi w^2} dw}_{=1}\\
    &= \frac{- iy}{\left( u^{\frac{1}{2}}q \right)^3}
    e^{- \pi u^{-1} \left( \frac{y}{q} \right)^2}
    \end{align*}

\end{proof}

\begin{exercise}[E7.4]
    Let $\chi$ be an odd primitive character
    (meaning $\chi(-1) = -1$ ). Define
    \[
    \vartheta_{\chi}(u)
    := \sum_{n \in \mathbb{Z}} n \chi(n)
    e^{-\pi n^2 u}.
    \] 
    Show that
    \[
    \vartheta_{\chi}(u)
    = \frac{\tau(\chi)}{iq^2 u ^{\frac{3}{2}}}
    \vartheta_{\overline{\chi}} 
    \left( \frac{1}{q^2 u} \right) 
    \] 
\end{exercise}

\begin{proof}
    Recall from Theorem \ref{Poisson-summation-character} that
    \[
    \sum_{n \in \mathbb{Z}}\chi(n) f\left( \frac{n}{q} \right) 
    = \tau(\chi) \sum_{n \in \mathbb{Z}}
    \overline{\chi(n)} \hat{f}(n).
    \] 
    Letting $f(x) = 
    qx e^{- \pi (qx)^2 u}$, we find that
    \begin{align*}
    \vartheta_{\chi}(u)
    = \sum_{n \in \mathbb{Z}} \chi(n)
    f \left( \frac{n}{q} \right) 
    &= \tau(\chi) \sum_{n \in \mathbb{Z}}
    \overline{\chi(n)}
    \left( - \frac{in q}{\left( u^{\frac{1}{2}}q \right)^3}
    e^{- \pi u^{-1} \left( \frac{n}{q} \right)^2 }\right) \\
    &= \frac{\tau(\chi)}{i q^2 u^{\frac{3}{2}}}
    \sum_{n \in \mathbb{Z}} \overline{\chi(n)}
    n e^{- \pi u^{-1} \left( \frac{n}{q} \right)^2}\\
    &=  \frac{\tau(\chi)}{i q^2 u^{\frac{3}{2}}}
    \vartheta_{\overline{\chi}}\left( \frac{1}{q^2 u} \right) 
    \end{align*}
\end{proof}

\begin{exercise}[E7.5]
    Show that for
    $\Re (s) > 1$, we have
    \[
    2 \pi^{\frac{-(s+1)}{2}} \Gamma\left( \frac{s+1}{2} \right) 
    L \left( s, \chi \right) 
    = \int_{0}^{\infty} 
    \vartheta_{\chi}(u) u^{\frac{s+1}{2}} \frac{du}{u}.
    \] 
\end{exercise}

\begin{proof}
    a
\end{proof}



\section{Assignment 1}


    \begin{exercise}[H1.1]
        \begin{proof}
            \[
            f * e (n) = \sum_{d  \mid n} f(d) e (\frac{n}{d})
            = \sum_{d  \mid n} f(d) \delta_{\frac{n}{d},1}
            = f(n)
            \] 
            and since the sets
            $\left\{ d \colon d \mid n \right\} $ and
            $\left\{ \frac{n}{d}  \colon
            d  \mid n\right\} $ are equal, we have
            \[
            g * f = 
            \sum_{d \mid n} g(d) f\left( \frac{n}{d} \right) 
            =
            \sum_{d  \mid n} g\left( \frac{n}{d} \right) 
            f \left( \frac{n}{\frac{n}{d}} \right) 
            = f * g (n)
            \] 
        \end{proof}
    \end{exercise}


    \begin{exercise}[H1.2]
        \begin{proof}
            \[
            \mu * 1 (n)
            = \sum_{d \mid n}
            \mu (d) 1 \left( \frac{n}{d} \right) 
            = \sum_{d \mid n} \mu(d)
            \] 
            If $n = p$ is a prime, we trivially
            have
            $\left\{ d \colon d \mid n \right\} 
            = \left\{ 1,p \right\} $, so
            $\sum_{d \mid n} \mu (d) =
            1 -1 = 0 = e(p)$, so it is true for
            $n$ a prime.\\
            \linebreak
            Suppose now that
            $n = p_1 \cdots p_s$, so
            $\mu (n) = (-1)^{s}$. 
            We need to find out how many
            elements the set
            $D_k = 
            \left\{ d  \mid n \colon
            d \text{ is a product of k distinct primes}\right\} $
            has.
            But this is simply the same as choosing
            an unordered set of $k$ elements from a set
            of  $s$ elements. There
            are precisely
            $\begin{pmatrix} s\\ k \end{pmatrix} $ ways to do
            so. Since for each $d \in D_k$,
            we have $\mu (d) = (-1)^{k}$, we find
            that
            \[
            \sum_{d  \mid n}\mu (d)
            = \sum_{k=1}^{s} \begin{pmatrix} s\\k \end{pmatrix} 
            (-1)^{k}
            = (1-1)^{s}
            = 0.
            \] 
            Then, in particular,
            \[
            \sum_{d  \mid n} \mu(d)
            \] 


            Lastly, for
            $n = p_1^{\alpha_1} \cdots p_{k}^{\alpha_{k}}$, it
            reduces to the previous case because
            $\mu$ is only non-zero on squarefree integers, so
            \begin{align*}
                \mu * 1 (n)
                &= \sum_{d  \mid \frac{n}{p_1^{\alpha_1-1} 
                \cdots p_k^{\alpha_k-1}}}
                \mu (d) = 0
            \end{align*}
            since the sets
            $\left\{ d \colon
            d  \mid 
        \frac{n}{p_1^{\alpha_1 - 1} \cdots
    p_k^{\alpha_k -1}}\right\} $ and
    $\left\{ d \colon
    d  \mid p_1 \cdots p_k \right\} $ are equal.

    Thus, indeed, $\mu * 1 = e$.
        \end{proof}
    \end{exercise}

    \begin{exercise}[H1.3]
        We claim that the set of arithmetic functions
        with Dirichlet convolution as a binary operation is
        an abelian semigroup.
        For this, if $f,g \colon \mathbb{N}  \to \mathbb{C}$, then
        clearly $f * g \colon \mathbb{N} \to \mathbb{C}$ too.
        Also,
        $f *g(n) = \sum_{ab = n} f(a) g(b) = 
        \sum_{ba = n} g(b) f(a) = g*f(n)$ by commutativity
        of multiplication in $\mathbb{C}$.
        Lastly,
        \[
            \left( f*g \right) *h(n)
            = \sum_{ab = n} f*g(a) h(b)
            = \sum_{ab=n} \sum_{cd = a} f(c) g(d) h(b)
            = \sum_{cdb = n} f(c) g(d) h(b)
        \] 
        and
        \[
        f * \left( g*h \right) (n)
        = \sum_{ab = n} f(a) g*h(b)
        = \sum_{ab = n} \sum_{cd = b} f(a) g(c) h(d)
        = \sum_{acd = n} f(a) g(c) h(d)
        \] 
        (all of this is just Theorem 5.1.4 in the book for Introduction to Number Theory
        by Risager).

        Now, if
        $f = 1 * g$ then
        $\mu * f = \mu * \left( 1 * g \right) 
        = \left( \mu * 1 \right) * g
        = e * g
        = g * e = g$ by the above together with
        H1.1.
        Likewise, if $g = \mu * f$, then
        $1 * g = 1 * \left( \mu * f \right) 
        = \left( 1 * \mu \right) * f
        = \left( \mu * 1 \right) * f
        = e * f = f * e = f$ again.
    \end{exercise}

    \begin{exercise}[H1.4]
        We have
        \begin{align*}
            \sum_{n=1}^{\infty} \left| 
            \frac{f(n)}{n^{s}}\right| 
            &\le 
            \sum_{n=1}^{\infty } \frac{C n^{k}}{n^{\sigma}}\\
            &\le \sum_{n=1}^{\infty}
            \frac{C}{n^{\sigma - k}}\\
            &< \infty
        \end{align*}
        as $\sigma - k > 1$. Thus the series
        converges absolutely.
    \end{exercise}

    \begin{exercise}[H1.5]
        \begin{proof}
            We know that
            $L_f$ converges absolutely for
            $\sigma > 1 + k_f$ and
            $L_g$ converges absolutely for
            $\sigma > 1 + k_g$.
            Assume without loss of generality that
            $k_g > k_f$.
            Now,
            \begin{align*}
                \sum_{n=1}^{\infty }
                \left| \frac{\sum_{d  \mid n} f(d)
                g(\frac{n}{d})}{n^{s}}
                \right| 
                &\le \sum_{n=1}^{\infty} \sum_{d \mid n}
                \frac{C_f C_g d^{k_f} 
                \left( \frac{n}{d} \right)^{k_g}}{n^{\sigma}}\\
                &= \sum_{n=1}^{\infty} C_f C_g
                \sum_{d \mid n} d^{k_f - k_g} \frac{1}{n^{\sigma -
                k_g}}
            \end{align*}
            Now, by E3.2, we have
            $d(n) \le 2 \sqrt{n} $, so
            since
            $\sum_{d \mid n} d^{k_f -k_g}
            \le \sum_{d \mid n} 1 = 
            d(n) \le 2 \sqrt{n}$, we have
            \begin{align*}
                \sum_{n=1}^{\infty} C_f C_g
                \sum_{d \mid n} d^{k_f - k_g} \frac{1}{n^{\sigma -
                k_g}}
                &\le 
                \sum_{n=1}^{\infty} C_f C_g
                2\sqrt{n}  \frac{1}{n^{\sigma -
                k_g}}\\
                &= 2C_f C_g \sum_{n=1}^{\infty} 
                \frac{1}{n^{\sigma -\left( k_g + \frac{1}{2}
                \right) }}
            \end{align*}

            Hence the sum defining
             $L_{f*g}(s)$ is absolutely convergent
             for $\sigma > 
             k_g + \frac{3}{2}$,
             and in this half-plane, 
             \[
             L_f(s) L_g(s) = 
             \sum_{k=1}^{\infty} \sum_{t=1}^{\infty}
             \frac{f(k)}{k^{s}}
             \frac{g(t)}{t^{s}}
             = 
             \sum_{r=1}^{\infty} \sum_{d  \mid r} 
             \frac{f(d) g(\frac{n}{d})}{r^{s}}
             = L_{f*g}(s)
             \] 
        \end{proof}
    \end{exercise}

    \begin{exercise}[H1.6]
        We have that
        when $L_1$ and $L_{\mu}$ are absolutely convergent,
        and satisfy the bounds
        from H1.5,
        we can use Cauchy summation to get
        $L_{1}(s) L_{\mu}(s) = 
        L_{1 * \mu}(s) =
        L_{e}(s)
        = 1$ which is absolutely
        convergent everywhere; but
        $L_1(s) = \zeta(s)$ and
        $L_{\mu}(s) = \sum_{n=1}^{\infty} \frac{\mu(n)}{n^{s}}$, 
        so
        the result follows in whenever all sums
        are absolutely convergent. Hence
        the desired equality extends (by the identity
        theorem) to all
        of $\Re (s) > 1$ since
        $\sum_{n=1}^{\infty} \frac{\mu(n)}{n^{s}} $ 
        converges to a holomorphic function in this
        half-plane (being the uniform limit of a 
        series of holomorphic
        functions).
    \end{exercise}


    \begin{exercise}[H 1.7]
        \begin{proof}
            For $f(n) = n^{w}$, we have
            $\sigma_w (n) = f * 1 (n)$.
            The abscissa of convergence for
            $1$ is $1$ and for $f$ it is
            $1 + \Re (w)$. In some
            halfplane, we have
            $\sum_{n=1}^{\infty} \frac{\sigma_w (n)}{n^{s}}
            = L_{\sigma_w}(s)
            = L_{f}(s) L_{1}(s)$. Now
            $L_1 (s) = \zeta(s)$, and
            \[
            L_f(s) = 
            \sum_{n=1}^{\infty} \frac{n^{w}}{n^{s}}
            = \sum_{n=1}^{\infty}
            \frac{1}{n^{s-w}}
            = \zeta(s-w).
            \] 
            Thus
            $\sum_{n=1}^{\infty} \frac{\sigma_w(n)}{n^{s}}
            = \zeta(s-w) \zeta(s)$
            in some right half-plane.
        \end{proof}
    \end{exercise}


\section{Assignment 2}

\begin{exercise}[H2.1]
    Show that
    \[
    \sum_{p\le x} \frac{1}{p}
    = \log \log (x) + O(1),
    \] 
    where the sum is over primes less than
    $x$.
\end{exercise}

\begin{proof}
    As is the custom, we of course start by Abel summation:
    \[
    \sum_{p\le x} \frac{1}{p}
    = \pi(x) \frac{1}{x}
    +\int_{1}^{x} \frac{\pi(t)}{t^2} dt
    \] 
    Now applying the PNT, we get

    \[
    \pi(x) \frac{1}{x} + 
    \int_{1}^{x} \frac{\pi(t)}{t^2} dt
    = \frac{1}{\log x} +
    O\left( e^{-c \sqrt{\log x} } \right) 
    + \int_{1}^{x} \frac{1}{t \log t}dt
    + \int_{1}^{x} O\left( t^2 e^{-c \sqrt{\log t} } \right) dt 
    \] 
    Since
    \[
    \int_{1}^{x} \frac{1}{t \log t}dt
    = \log \log t  \bigg|_{1}^{x}
    \] 
    we have what we needed.

\end{proof}

\begin{exercise}[H2.2]
    This exercise gives a different proof that
    $\zeta(s)$ has no zeros on
    $\Re(s) = 1$.
    \begin{enumerate}
        \item Prove that for
            $\sigma > 1, t \in \mathbb{R}$,
            \[
            \Re := \Re \left( 3 \log \zeta (\sigma)
            + 4 \log \zeta \left( \sigma + it)
        + \log \zeta \left( \sigma+ 2it \right) \right) \right) 
        \ge 0.
            \] 
        \item Prove that
            $\left| \zeta(\sigma)^3
            \zeta \left( \sigma + it \right)^{4} 
            \zeta\left( \sigma + 2it \right) \right| \ge 1 $.
        \item Prove that if $\zeta \left( 1+ it_0 \right) =0$,
            then
            $\left| \zeta(\sigma)^3
            \zeta\left( \sigma + it_0 \right)^{4}
            \zeta \left( \sigma + 2it_0 \right) \right| 
            \to 0$ as $\sigma \to 1$.
        \item Conclude that $\zeta \left( 1+it \right) \neq 0
            $ for every $t \neq 0$.
    \end{enumerate}
\end{exercise}

\begin{proof}
    (1) We want to make use of
    Lemma \ref{Lemma:02329} or Lemma
    \ref{Lemma:39292}.

    We have
    \[
    \log \zeta \left( s \right) 
    = \sum_{n=1}^{\infty} \frac{\Lambda (n)}{\log n}n^{-s}
    \] 
    for $\sigma = \Re (s) > 1$.
    Thus
    $\Re \left( \log \zeta (s) \right) 
    = \sum_{n=1}^{\infty} \frac{\Lambda (n)}{\log n}
    \frac{1}{n^{\sigma}} \cos \left( t \log n \right) $ by
    \eqref{A-1}.
    Hence
    \begin{align*}
        \Re 
        &= \sum_{n=1}^{\infty} 
        \frac{\Lambda (n)}{n^{\sigma} \log n}
        \left[ 3 + 4 \cos (t \log n) +
        \cos (2t \log n )\right]
        \ge 0
    \end{align*}
    since all the terms are positive.\\
    \linebreak
    (2)
    Let
    \[
    X :=  \zeta (\sigma)^3 
    \zeta\left( \sigma + it  \right)^{4} 
    \zeta \left( \sigma + 2 it  \right) 
    \] 
    Then
    $\Re \log X \ge 0$, hence
    $\left| X \right|  = \left| e^{\log X} \right|  =
    \left| e^{\Re \log X} \right| 
      \ge 1$.\\
      \linebreak
      (3) Suppose
      $\zeta \left( 1+ it_0  \right) =0$.
      Since
      $X$ is continuous as a function of
      $\sigma$ with fixed $t_0$
      for $\sigma > 1$, we find that
      if
      \[
      X_{t_0} (\sigma) = \zeta(\sigma)^3
      \zeta\left( \sigma + it_0  \right)^{4}
      \zeta \left( \sigma + 2 it_0  \right) 
      \] 
      then the claim is
      $\lim_{\sigma \to 1+}X_{t_0}(\sigma) = 0$.
      Recall that $\zeta$ is meromorphic
      on $\mathbb{C}$ and has only one
      pole which is at $1$ and is simply.
      So in particular,
      $(s-1) \zeta(s)$ is holomorphic around
      $1$, so let
      $g(s) = (s-1) \zeta(s)$.
      In particular then, if
      $t_0 \neq 0$, we then
      $1+ it_0 $ is not a pole, so
      if we write
      $\zeta \left( s \right) 
      = (s- \left( 1+it_0  \right))
      h(s)$ near $1+it_0 $, then we get
      that for $\sigma$ close to $1$, we have
      \begin{align*}
      X_{t_0}(\sigma) 
      &= 
      \frac{g(\sigma)^3}{(\sigma-1)^3}
      \left( \left( \sigma + it_0  \right) 
      - \left( 1+it_0  \right)  \right)^{4}
      h(\sigma + it_0) 
      \zeta \left( \sigma + 2 it_0 \right) \\
      &= \left( \sigma-1 \right) g(\sigma)^3
      h\left( \sigma+it_0 \right) 
      \zeta\left( \sigma+ 2it_0 \right) \to
      0, \quad \sigma \to 1+
      \end{align*}

      (4) Since $X 
      \left( \sigma + it \right) \ge 1$ for
      all $\sigma > 1$, we thus cannot
      have that
      $X\left( 1+it \right) = 0$
      since continuity would break.

\end{proof}

\section{Assignment 3}

\begin{definition}[$k$-almost prime]
    For a number
    $n = p_1^{a_1} \cdots
    p_m^{a_m}$, let
    $\Omega (n) = \sum_{i=1}^{m} a_i$.
    Then $n$ is called a
    $k$-almost prime if
    $\Omega (n) = k$.
    Let $P_k$ denote the
    set of $k$-almost primes.
    Then define
    $\pi_k(x) = 
    \# \left( \left\{ 1, \ldots, \left[ x \right]  \right\} 
    \cap P_k \right)$, i.e., the
    number of $k$-almost primes
    less than or equal to $x$.
\end{definition}

\begin{exercise}[H3.1]
    \begin{enumerate}
        \item Show that
            \[
            \pi_2(x) = 
            \sum_{p \le \sqrt{x} }
            \pi\left( \frac{x}{p} \right) +
            O\left( \frac{x}{\log^2 x} \right) 
            \] 
        \item Show that the sum in (1) is
            \[
            \sum_{p\le \sqrt{x} }
            \frac{x}{p \log \left( \frac{x}{p} \right) }
            + O \left( \frac{x \log \log x}{\log^2 x} \right) 
            \] 
        \item Use PNT and summation by parts to show
            that the above sum is
            \[
            x \int_{2}^{\sqrt{x} } 
            \frac{1}{u \log \left( \frac{x}{u} \right) 
            \log u} du +
            O \left( \frac{x}{\log x} \right) .
            \] 
        \item Show that
             \[
            \pi_2 (x) =
            x \frac{\log \log x}{\log x}
            + O \left( \frac{x}{\log x} \right) .
            \] 
    \end{enumerate}
\end{exercise}

\begin{proof}
    (1) Suppose
    $n \in \left\{ 1, \ldots, \left[ x \right]  \right\} 
    \cap P_2$, so
    $n = p_1 p_2$ where we don't necessarily have
    that $p_1 \neq p_2$. Suppose without loss of generaliy
    that $p_2 \ge p_1$. Then, in particular,
    $\frac{n}{p_2} = p_1 \le \sqrt{x} $.
    For suppose for the contrary that
    $p_1 > \sqrt{x} $. Then
    $p_2 > \sqrt{x} $, so
    $x < p_1p_2 = n \le x$, contradiction. 

    Using this, we see that
    \[
    \pi_2 (x) = 
    \sum_{p_1 \le \sqrt{x}}
    \sum_{p_1 < p_2 \le \frac{x}{p_1}} 1
    =
    \sum_{p \le \sqrt{x} }
    \pi \left(\frac{x}{p} \right) - \pi(p)
    \le  \sum_{p \le \sqrt{x} }
    \pi\left( \frac{x}{p} \right) 
    + O \left( \sum_{p \le \sqrt{x} }
    \pi(p) \right) 
    \] 
    And
    \[
    \sum_{p \le \sqrt{x} }\pi(p)
    \le \pi\left( \sqrt{x}  \right)^2
    = O \left( \frac{x}{\log^2 \left( \sqrt{x}  \right) } \right) 
    = O \left( \frac{x}{\log^2 x} \right) 
    \] 
    since $\log^2 \left( x^{\frac{1}{2}} \right) 
    = \frac{1}{4} \log^2 x$, where
    we also used $\pi(x) = O \left( \frac{x}{\log x} \right) $.\\
    \linebreak
    (2) 
    By the PNT, 
    $\pi(x) = \frac{x}{\log x} + O\left( \frac{x}{\log^2 x} \right) $,
    so
    \[
    \sum_{p \le \sqrt{x} } \pi \left( \frac{x}{p} \right) 
    = \sum_{p \le \sqrt{x} } 
    \frac{\frac{x}{p}}{\log \left( \frac{x}{p} \right) }
    + O \left( \frac{\frac{x}{p}}{\log^2 \left( \frac{x}{p}
    \right) } \right) 
    \] 
    Now
    \[
    \sum_{p\le \sqrt{x} }
    \frac{x}{p \log^2 \left( \frac{x}{p} \right) }
    \le 
    \sum_{p\le \sqrt{x} }
    \frac{x}{p \log^2 \left( \sqrt{x}  \right) }
    \le \frac{x}{4 \log^2 x} O \left( \log \log \sqrt{x}  \right) 
    = O \left( \frac{x \log \log x}{\log^2 x} \right),
    \] 
    giving
    \[
    \sum_{p \le \sqrt{x} }\pi\left(\frac{x}{p}\right)
    =
    \sum_{p\le \sqrt{x} } \frac{x}{p \log\left( \frac{x}{p} \right) }
    + O \left( \frac{x \log \log x}{\log^2 x} \right).
    \] 
    (3) 
    Let
    $f(y) = \frac{1}{y \log(\frac{x}{y})}$ and
    $a_n = \delta_{n, \text{ prime}}$.
    Then
    \[
    f'(y) = - \frac{1}{y^2 \log\left( \frac{x}{y} \right) }
    - \frac{y}{xy \log^2 \left( \frac{x}{y} \right) }
    =- \frac{1 }{y^2 \log \left( \frac{x}{y} \right) }
    - \frac{1}{x \log^2 \left( \frac{x}{y} \right) }.
    \] 
    \begin{align*}
        \sum_{p\le \sqrt{x} } \frac{x}{p
        \log \left( \frac{x}{p} \right) }
        &= \frac{\pi\left( \sqrt{x}  \right) x}{\sqrt{x} 
        \log \left( \sqrt{x}  \right) }
        - x \int_{1}^{\sqrt{x} } 
        \pi(t) 
        \left[ \frac{- 1 }{t^2 \log\left( \frac{x}{t} \right) }
        - \frac{1}{x \log^2 \left( \frac{x}{t} \right) }\right] dt
    \end{align*}
    Firstly,
    \[
    \frac{\pi\left( \sqrt{x}  \right) x}{\sqrt{x} 
    \log \left( \sqrt{x}  \right) }
    = \frac{x}{4 \log^2 x}
    + O\left( \frac{x}{\log^3 x} \right) 
    = O \left( \frac{x}{\log x} \right) .
    \] 
    Secondly,
    \[
    x \int_{1}^{\sqrt{x} } 
    \frac{\pi(t)}{t^2 \log\left( \frac{x}{t} \right) } dt
    = x
\int_{1}^{\sqrt{x} } \frac{1}{
    t \log t \log \left( \frac{x}{t} \right) } dt
    \] 
    which is precisely the term we want. 
    It only remains to show that
    \[
    x \int_{1}^{\sqrt{x} } \frac{\pi(t)}{x
    \log^2 \left( \frac{x}{t} \right) } dt
    = O\left( \frac{x}{\log x} \right) .
    \] 
    I.e.
    
    \[
     \int_{1}^{\sqrt{x} } \frac{\pi(t)}{x
    \log^2 \left( \frac{x}{t} \right) } dt
    = O\left( \frac{1}{\log x} \right) .
    \] 
    Now
    \[
    \int_{2}^{\sqrt{x} } 
    \frac{\pi(t)}{x \log^2 \left( \frac{x}{t} \right) }
    dt = 
    \int_{2}^{\sqrt{x} } \frac{t}{x \log t
    \log^2 \left( \frac{x}{t} \right) } dt
    + \int_{2}^{\sqrt{x} } 
    O\left( \frac{t}{x \log^2 (t) \log^2
    \left( \frac{x}{t} \right) } \right) dt
    \] 
    The derivative of
    $\frac{1}{\log x}$ is
    $- \frac{1}{x \log^2 x}$.
    Now
    \[
    \left| \int_{2}^{\sqrt{x} } \frac{t}{x
    \log t \log^2 \left( \frac{x}{t} \right) }
    dt \right|
    \le 
    \left| \int_{2}^{\sqrt{x} } 
    \frac{1}{t \log^3 t}dt \right|
    \le 
    \left| \int_{2}^{\sqrt{x} } 
    \frac{1}{t \log^2 t} dt\right| 
    \le \left| \frac{1}{\log t} \bigg|_{2}^{\sqrt{x} }  \right| 
    = O \left( \frac{1}{\log x} \right) 
    \] 
    Furthermore, the
    second term is even
    smaller in size, hence
    also $O \left( \frac{1}{\log x} \right) $.\\
    \linebreak
    (4) Putting the above together, we find that
    \begin{align*}
        \pi_2(x) 
        &= \sum_{p \le \sqrt{x} } \pi\left( \frac{x}{p} \right) 
        + O \left( \frac{x}{\log^2 x} \right) \\
        &=
        \sum_{p\le \sqrt{x} } \frac{x}{p \log\left( 
        \frac{x}{p} \right) } +
        O\left( \frac{x \log \log x}{\log^2 x} \right) 
        + O \left( \frac{x}{\log^2 x} \right) \\
        &= x \int_{2}^{\sqrt{x} } 
        \frac{1}{t \log \left( \frac{x}{t} \right) \log t}dt
        + O \left( \frac{x}{\log x} \right) 
        + O \left( \frac{x \log \log x}{\log^2 x} \right) 
        + O \left( \frac{x}{\log^2 x} \right)\\
        &= x \int_{2}^{\sqrt{x} } \frac{1}{t \log \left( \frac{x}{t}
        \right) \log t} dt + O \left( \frac{x}{\log x} \right) . 
    \end{align*}

    Now, setting
    $\log t = v$, we get
    $dv = \frac{1}{t} dt$, so
    \begin{align*}
        x \int_{2}^{\sqrt{x} } 
        \frac{1}{t \log \left( \frac{x}{t} \right) \log t} dt
        &= x 
        \int_{\log 2}^{\log \sqrt{x} } 
        \frac{1}{v \left( \log x
        -v\right) } dv\\
        &= \frac{x}{\log x}
        \int_{\log 2}^{\log \sqrt{x} } 
        \frac{1}{v} + \frac{1}{\log x - v} dv\\
        &= \frac{x}{\log x} \left[ \log \log \sqrt{x} 
        - \log \log 2 \right] 
        - \frac{x}{\log x}
        \left[ \log \left( \log x - \log \sqrt{x}  \right) -
        \log \left( \log x - \log 2 \right) \right] \\
        &= \frac{x}{\log x} \left[ 
        \log \log \frac{x}{2} - \log \log 2 \right] 
        = x \frac{\log \log x}{\log x} + 
        O \left( \frac{x}{\log x} \right) .
    \end{align*}
    This completes the proof.
\end{proof}

\begin{exercise}[H3.2]
    Show that for $x$ sufficiently large, there
    are more primes in the interval
    $(1, x]$ than in the interval
    $(x, 2x]$.
\end{exercise}

\begin{proof}
    What this is saying is essentially that
    for $x$ sufficiently large,
    $\pi(x) > \pi(2x) - \pi(x)$, i.e.,
    $2 \pi(x) > \pi(2x)$. By the prime number theorem, we
    have
    \[
    \pi(x) = li(x) + O\left( \frac{x}{e^{c \sqrt{\log x} }} \right) 
    \] 
    So
    \[
    2\pi(x) - \pi(2x)
    = 2 li(x) - li(2x)
    + O \left( \frac{x}{e^{c' \sqrt{\log x} }} \right) 
    \] 
    Now
    \[
    2 li(x) - li(2x)
    = 2 \int_{2}^{x} \frac{1}{\log t}dt
    - \int_{2}^{2x} \frac{1}{\log t}dt
    = \int_{2}^{x} \frac{1}{\log t}dt
    - \int_{x}^{2x} \frac{1}{\log t} dt 
    \] 

    Now,
    $\log t \le t$, so
    $\int_{2}^{x} \frac{1}{\log t} dt
    \ge \int_{2}^{x} \frac{1}{t}dt
    = \log x - \log 2$.

    In a weaker form, the prime number theorem says that
    \[
    \pi(x) = \frac{x}{\log x} + O\left( \frac{x}{\log^2 x} \right) .
    \] 
    Using this, we obtain
    \begin{align*}
    2 \pi(x) - \pi(2x)
    &= 2x \left[ \frac{1}{\log x} - \frac{1}{\log 2x} \right] 
    + 2x\left[ O \left( \frac{1}{\log^2 x} \right) 
    - O \left( \frac{1}{\log^2 2x} \right)  \right]\\
    &= 2 \log 2 \frac{x}{\log 2x \log x} + \ldots
    \end{align*}

    Now
    \[
    \frac{1}{\log x} - \frac{1}{\log 2x}
    = \frac{\log 2}{\log x \log 2x}
    \] 
    So
    $\frac{1}{\log 2x} = \frac{1}{\log x} + 
    O \left( \frac{1}{\log^2 x} \right) $.
    Thus
    \[
    2 \log 2 \frac{x}{\log 2x \log x}
    = 2 \log 2 \frac{x}{\log^2 x}
    + O \left( \frac{x}{\log^3 x} \right) .
    \] 
    Now
    \[
    \left| \left| \frac{1}{\log^2 2x} \right| -
    \left| \frac{1}{\log^2 x} \right| \right| 
    \le \left| \frac{1}{\log^2 x}- \frac{1}{\log^2 2x} \right| 
    =
    \left| \frac{2 \log 2 \log x + \log^2 2}{\log^2 x \log^2 2x}
    \right| 
    \le \left| \frac{2 \log 2 \log x + \log^2 2}{
    \log^{4} x} \right| 
    \] 
    Hence
    also the last term is
    $O \left( \frac{x}{\log^3 x} \right) $, so
    we get
    \[
    2 \pi(x) - 2 \pi(x) = 
    2 \log 2 \frac{x}{\log^2 x} + 
    O \left( \frac{x}{\log^3 x} \right) .
    \] 
    Now, both
    $\frac{x}{\log^2 x}$ and
    $\frac{x}{\log^3 x}$ go to
    $\infty$ as $x \to \infty$, however,
    $\frac{x}{\log^2 x}$ grows faster, so
    we can find a $N \in \mathbb{N} $ such that
    $2 \log 2 \frac{N}{\log^2 N}
    > \frac{N}{\log^3 N} + 1$. In particular,
    this implies that
    there is at least one more
    prime in
    $(x,2x]$ than in $(1,x]$.
\end{proof}

\section{Assignment 4}

\begin{exercise}[H4.1]
    Let $f \in \mathcal{S}(\mathbb{R})$ be a 
    Schwartz-function on the real axis and
    let $\hat{f}(y)  =
    \int_{\mathbb{R}} f(t) e^{-2\pi i t y}dt$ be its
    Fourier transform. Let
    $F(x) = \sum_{m \in \mathbb{Z}}
    f(x+m)$.
    \begin{enumerate}
        \item Show that the sum defining $F(x)$ is convergent
            and that
            \[
            F(x) = \sum_{m \in \mathbb{Z}}
            \hat{f}(m) e^{2\pi i mx}.
            \] 
            Let now $\chi$ be a primitive Dirichlet character
            modulo $q$.
        \item Show that
            \[
            \sum_{m \in \mathbb{Z}}
            \chi(m) f\left( \frac{m}{q} \right) 
            = \sum_{a=1}^{q}\chi(a)
            F\left( \frac{a}{q} \right) .
            \] 
        \item Show that
            \[
            \sum_{a=1}^{q} \chi(a) F(\frac{a}{q})
            = \tau(\chi) \sum_{m \in \mathbb{Z}}
            \overline{\chi(m)} \hat{f}(m).
            \] 
        \item Wrap up 2) and 3) by stating in full a theorem
            that can be considered a character version
            of the Poisson summation formula.\\
            Let $\theta_{\chi}(u) = 
            \sum_{n \in \mathbb{Z}} \chi(n)
            e^{- \pi n^2 u}$.
        \item Show that
            \[
            \theta_{\chi}(u) = 
            \frac{\tau (\chi)}{q u^{\frac{1}{2}}}
            \theta_{\overline{\chi}} \left( \frac{1}{q^2 u} \right) .
            \] 
    \end{enumerate}
\end{exercise}

\begin{proof}
    (1) If $f \in \mathcal{S}(\mathbb{R})$, then also
    $f\left( x+m \right) \in \mathcal{S}(\mathbb{R})$,
    so letting $g (m) := f(x+m)$, we get
    by Poisson's summation formula that
    \[
    \sum_{m \in \mathbb{Z}} g(x)
    = \sum_{m \in \mathbb{Z}} \hat{g}(x)
    \] 
    where $
    \hat{g}(y) = \int_{-\infty}^{\infty} 
    e^{-2 \pi i x y }g(x) dx$. Thus
    \[
    F(x) = \sum_{m \in \mathbb{Z}}f(x+m)
    = \sum_{m \in \mathbb{Z}}g(m)
    = \sum_{m \in \mathbb{Z}}
    \hat{g}(m) 
    = \sum_{m \in \mathbb{Z}}
    \int_{-\infty}^{\infty} e^{-2 \pi i t m}
    f(x+t) dt
    \] 
    Let now
    $w = x+t$. Then
    $dw = dt$, so
    \[
    \sum_{m \in \mathbb{Z}}
    \int_{-\infty}^{\infty} e^{-2\pi i t m}
    f(x+t) dt
    = \sum_{m \in \mathbb{Z}}
    \int_{-\infty}^{\infty} 
    e^{-2 \pi i (w-x) m} f(w) dw
    = \sum_{m \in \mathbb{Z}} \hat{f}(m) e^{2 \pi i m x}.
    \] 
    (2) Since $
    \chi$ is $q$-periodic, we have
    \begin{align*}
        \sum_{m \in \mathbb{Z}}\chi(m) f\left( \frac{m}{q} \right) 
        &= \sum_{a=1}^{q}
        \chi (a) \sum_{k \in \mathbb{Z}}
        f \left( \frac{a}{q} + k \right) \\
        &= \sum_{a=1}^{q}
        \chi(a) F \left( \frac{a}{q} \right) .
    \end{align*}

    (3) Since $\chi$ is primitive, we know that
    \[
    \chi(n) \tau (\overline{\chi}) =
    c_{\overline{\chi}}(n) \tag{$\mathcal{A}_1$} \label{mcalA-1}
    \] 

    \begin{align*}
        \sum_{a=1}^{q}\chi(a) F\left( \frac{a}{q} \right) 
        &= \sum_{a=1}^{q} \chi (a)
        \sum_{m \in \mathbb{Z}} \hat{f}(m) e^{2\pi i m \frac{a}{q}}\\
        &= \sum_{m \in \mathbb{Z}}
        \hat{f}(m) \sum_{a=1}^{q}
        \chi(a)  e^{2 \pi i m \frac{a}{q}}\\
        &= \sum_{m \in \mathbb{Z}} \hat{f}(m)
        c_{\chi}(m)\\
        &\stackrel{\eqref{mcalA-1}}{=} 
        \tau (\chi) \sum_{m \in \mathbb{Z}}
         \overline{\chi}(m) \hat{f}(m)
    \end{align*}

    (4) 

    \begin{theorem}[Poisson summation formula, character
        version]\label{Poisson-summation-character}
    If $f \in \mathcal{S}(\mathbb{R})$, then
    \[
    \sum_{m \in \mathbb{Z}} 
    \chi(m) f \left( \frac{m}{q} \right) =
    \tau(\chi) \sum_{m \in \mathbb{Z}}
    \overline{\chi(m)} \hat{f}(m).
    \] 
    \end{theorem}

    (5) Let
    $\theta_{\chi}(u) = 
    \sum_{m \in \mathbb{Z}}\chi(m) e^{-\pi m^2 u}$.
    Let $f_u(x) = e^{- \pi u (qx)^2}$, so
    $\theta_{\chi}(u) = 
    \sum_{m \in \mathbb{Z}} \chi(m)
    f_u (\frac{m}{q})$.\\
    \linebreak
    Then
    \begin{align*}
        \theta_{\chi}(u)
        &= \tau (\chi) 
        \sum_{m \in \mathbb{Z}}
        \overline{\chi}(m)
        \int_{-\infty}^{\infty} 
        e^{- \pi u (qt)^2}
        e^{-2 \pi i t m} dt\\
        &= \tau (\chi) 
        \int_{-\infty}^{\infty} 
        e^{- \pi u (qt)^2}
        \sum_{m \in \mathbb{Z}}\overline{\chi}(m)
        e^{- 2 \pi i t m } dt\\
        &= 
        \tau(\chi) 
        \int_{-\infty}^{\infty} e^{-\pi u (qt)^2}
        \sum_{a=1}^{q}
        \overline{\chi}(a)
        \sum_{m \in \mathbb{Z}}
        e^{-2 \pi i t q \left( \frac{a}{q} + m \right) } dt
    \end{align*}
    Let
    $z = qt$, so
    $dz = q dt$, then
    \begin{align*}
        \tau(\chi) 
        \int_{-\infty}^{\infty} e^{-\pi u (qt)^2}
        \sum_{a=1}^{q}
        \overline{\chi}(a)
        \sum_{m \in \mathbb{Z}}
        e^{-2 \pi i t q \left( \frac{a}{q} + m \right) } dt
        &= \tau(\chi) \frac{1}{q}
        \int_{-\infty}^{\infty} e^{- \pi u z^2}
        \sum_{a=1}^{q} \overline{\chi}(a)
        \sum_{m \in \mathbb{Z}}
        e^{- 2 \pi i z \left( \frac{a}{q}+m \right) } dz\\
        &= \frac{\tau(\chi)}{q}
        \int_{-\infty}^{\infty} 
        e^{- \pi u z^2} \sum_{m \in \mathbb{Z}}
        \overline{\chi}(m) e^{- 2 \pi i z \frac{m}{q}}dz
    \end{align*}
    Now let
    $w = \sqrt{u} z $, so
    $dw = \sqrt{u}  dz$. Then
    \begin{align*}
        \frac{\tau(\chi)}{q}
        \int_{-\infty}^{\infty} 
        e^{- \pi u z^2} \sum_{m \in \mathbb{Z}}
        \overline{\chi}(m) e^{- 2 \pi i z \frac{m}{q}}dz
        &= 
        \frac{\tau(\chi)}{\sqrt{u} q}
        \int_{-\infty}^{\infty} 
        e^{- \pi w^2} \sum_{m \in \mathbb{Z}}
        \overline{\chi}(m) e^{-2 \pi i \frac{wm}{\sqrt{u} q }} 
        dw
    \end{align*}

    
    Let
    $v = \frac{t}{\sqrt{u} }$. Then
    $dv = \frac{1}{\sqrt{u} } dt$, so
    \begin{align*}
        \frac{1}{\sqrt{u} }
        \sum_{m \in \mathbb{Z}}\chi(m)
        \int_{-\infty}^{\infty} 
        e^{- \frac{\pi}{u} t^2 - 2 \pi i t m}dt
        &= \sum_{m \in \mathbb{Z}}
        \chi(m) \int_{-\infty}^{\infty} 
        e^{-\pi v^2 - 2 \pi i v \sqrt{u} m} dv\\
        &= \int_{-\infty}^{\infty} 
        e^{- \pi v^2} \sum_{m \in \mathbb{Z}}
        \chi(m) e^{-2 \pi i v \sqrt{u}  m} dv\\
        &= \int_{-\infty}^{\infty} e^{- \pi v^2}
    \end{align*}








\end{proof}



    %\printbibliography
\end{document}
