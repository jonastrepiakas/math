\documentclass[reqno]{amsart}
\usepackage{amscd, amssymb, amsmath, amsthm}
\usepackage{graphicx}
\usepackage[colorlinks=true,linkcolor=blue]{hyperref}
\usepackage[utf8]{inputenc}
\usepackage[T1]{fontenc}
\usepackage{textcomp}
\usepackage{babel}
%% for identity function 1:
\usepackage{bbm}
%%For category theory diagrams:
\usepackage{tikz-cd}

%\usepackage[backend=biber]{biblatex}
%\addbibresource{.bib}


\setlength\parindent{0pt}

\pdfsuppresswarningpagegroup=1

\newtheorem{theorem}{Theorem}[section]
\newtheorem{lemma}[theorem]{Lemma}
\newtheorem{proposition}[theorem]{Proposition}
\newtheorem{corollary}[theorem]{Corollary}
\newtheorem{conjecture}[theorem]{Conjecture}

\theoremstyle{definition}
\newtheorem{definition}[theorem]{Definition}
\newtheorem{example}[theorem]{Example}
\newtheorem{exercise}[theorem]{Exercise}
\newtheorem{problem}[theorem]{Problem}
\newtheorem{question}[theorem]{Question}

\theoremstyle{remark}
\newtheorem*{remark}{Remark}
\newtheorem*{note}{Note}
\newtheorem*{solution}{Solution}



%Inequalities
\newcommand{\cycsum}{\sum_{\mathrm{cyc}}}
\newcommand{\symsum}{\sum_{\mathrm{sym}}}
\newcommand{\cycprod}{\prod_{\mathrm{cyc}}}
\newcommand{\symprod}{\prod_{\mathrm{sym}}}

%Linear Algebra

\DeclareMathOperator{\Span}{span}
\DeclareMathOperator{\im}{im}
\DeclareMathOperator{\diag}{diag}
\DeclareMathOperator{\Ker}{Ker}
\DeclareMathOperator{\ob}{ob}
\DeclareMathOperator{\Hom}{Hom}
\DeclareMathOperator{\Mor}{Mor}
\DeclareMathOperator{\sk}{sk}
\DeclareMathOperator{\Vect}{Vect}
\DeclareMathOperator{\Set}{Set}
\DeclareMathOperator{\Group}{Group}
\DeclareMathOperator{\Ring}{Ring}
\DeclareMathOperator{\Ab}{Ab}
\DeclareMathOperator{\Top}{Top}
\DeclareMathOperator{\hTop}{hTop}
\DeclareMathOperator{\Htpy}{Htpy}
\DeclareMathOperator{\Cat}{Cat}
\DeclareMathOperator{\CAT}{CAT}
\DeclareMathOperator{\Cone}{Cone}
\DeclareMathOperator{\dom}{dom}
\DeclareMathOperator{\cod}{cod}
\DeclareMathOperator{\Aut}{Aut}
\DeclareMathOperator{\Mat}{Mat}
\DeclareMathOperator{\Fin}{Fin}
\DeclareMathOperator{\rel}{rel}
\DeclareMathOperator{\Int}{Int}
\DeclareMathOperator{\sgn}{sgn}
\DeclareMathOperator{\Homeo}{Homeo}
\DeclareMathOperator{\SHomeo}{SHomeo}
\DeclareMathOperator{\PSL}{PSL}
\DeclareMathOperator{\Bil}{Bil}
\DeclareMathOperator{\Sym}{Sym}
\DeclareMathOperator{\Skew}{Skew}
\DeclareMathOperator{\Alt}{Alt}
\DeclareMathOperator{\Quad}{Quad}
\DeclareMathOperator{\Sin}{Sin}
\DeclareMathOperator{\Supp}{Supp}
\DeclareMathOperator{\Char}{char}
\DeclareMathOperator{\Teich}{Teich}
\DeclareMathOperator{\GL}{GL}
\DeclareMathOperator{\tr}{tr}
\DeclareMathOperator{\codim}{codim}
\DeclareMathOperator{\coker}{coker}
\DeclareMathOperator{\Diff}{Diff}
\DeclareMathOperator{\Bun}{Bun}
\DeclareMathOperator{\Sm}{Sm}



%Row operations
\newcommand{\elem}[1]{% elementary operations
\xrightarrow{\substack{#1}}%
}

\newcommand{\lelem}[1]{% elementary operations (left alignment)
\xrightarrow{\begin{subarray}{l}#1\end{subarray}}%
}

%SS
\DeclareMathOperator{\supp}{supp}
\DeclareMathOperator{\Var}{Var}

%NT
\DeclareMathOperator{\ord}{ord}

%Alg
\DeclareMathOperator{\Rad}{Rad}
\DeclareMathOperator{\Jac}{Jac}

%Misc
\newcommand{\SL}{{\mathrm{SL}}}
\newcommand{\mobgp}{{\mathrm{PSL}_2(\mathbb{C})}}
\newcommand{\id}{{\mathrm{id}}}
\newcommand{\MCG}{{\mathrm{MCG}}}
\newcommand{\PMCG}{{\mathrm{PMCG}}}
\newcommand{\SMCG}{{\mathrm{SMCG}}}
\newcommand{\ud}{{\mathrm{d}}}
\newcommand{\Vol}{{\mathrm{Vol}}}
\newcommand{\Area}{{\mathrm{Area}}}
\newcommand{\diam}{{\mathrm{diam}}}
\newcommand{\End}{{\mathrm{End}}}


\newcommand{\reg}{{\mathtt{reg}}}
\newcommand{\geo}{{\mathtt{geo}}}

\newcommand{\tori}{{\mathcal{T}}}
\newcommand{\cpn}{{\mathtt{c}}}
\newcommand{\pat}{{\mathtt{p}}}

\let\Cap\undefined
\newcommand{\Cap}{{\mathcal{C}}ap}
\newcommand{\Push}{{\mathcal{P}}ush}
\newcommand{\Forget}{{\mathcal{F}}orget}




\begin{document}

\section{Week 1}

\begin{exercise}[E1.1. Abel summation]
    Let $\left\{ a_n \right\}_{n \in \mathbb{N} }\subset \mathbb{C}$ 
    and $f \colon \left[ 1,x \right] \to \mathbb{C}$ be
    $C^{1}$. Define $A(t) = \sum_{n \le t} a_n$. Then for
    $x>1$, we have
    \[
    \sum_{n\le x} a_n f(n) = A(x) f(x) -
    \int_{1}^{x} A(t) f'(t) dt. 
    \] 
\end{exercise}



\section{Week 2}

Let $\psi (x) :=
\sum_{n \le x} \Lambda (n)$.

\begin{exercise}[E2.6]
    Show that
    \[
    \theta(x) := 
    \sum_{p \le x} \log p = \psi (x) + O\left( x^{\frac{1}{2}}
    \log^2 x \right) 
    \] 
\end{exercise}

\begin{exercise}[E2.7]
    Show that
    \[
    \pi(x) = \frac{\psi (x)}{\log x}+ 
    \int_{2}^{x} \frac{\psi (t)}{t \log^2t}dt +
    O\left( x^{\frac{1}{2}} \log x \right) .
    \] 
\end{exercise}

\begin{proof}
    By Abel summation, we first find that
    \[
    \theta(x) :=
    \sum_{p \le x} \log p = 
    \pi(x) \log x - \int_{2}^{x} \frac{\pi(t)}{t}dt
    \] 
    and from the previous exercise, we now find
    that
    \[
    \pi(x) = \frac{\psi (x)}{\log x} +
    \frac{1}{\log x} \int_{2}^{x}\frac{\pi(t)}{t}dt 
    + O \left( x^{\frac{1}{2}} \log x \right) 
    \] 
    The result follows if we
    can show that
    \[
    \frac{1}{\log x} \int_{2}^{x} \frac{\pi(t)}{t}dt
    = \int_{2}^{x} \frac{\psi (t)}{t \log^2 t}dt 
    + O\left( x^{\frac{1}{2}} \log x \right).
    \] 
    Now $\psi (t) \le \pi(t) \log t$, so
    \begin{align*}
        \left| \int_{2}^{x} 
        \frac{\psi (t)}{t \log^2 t} - 
        \frac{\pi(t)}{t \log x}dt \right| 
        &\le 
        \left| \int_{2}^{x} \frac{\pi(t)}{t \log t} -
        \frac{\pi(t)}{t \log x} dt\right| \\
        &= \left| 
        \int_{2}^{x} \frac{\pi(t)}{t} 
        \frac{\log \left( \frac{x}{t} \right) }{\log x 
        \log t} dt\right| \\
    \end{align*}
\end{proof}

\section{Week 3}

\begin{exercise}[E3.1]
    Let $m \ge 0$. Show that
    \[
    \sum_{n \le x} \log^{m} n = 
    x \log^{m} x + O \left( x \log^{m-1} x \right) .
    \] 
\end{exercise}

\begin{proof}
    Let $a_n = 1$ for all $n$. Then
    $A(x) = \left\lfloor x \right\rfloor$, so
    \begin{align*}
    \sum_{n \le x} \log^{m}n 
    &= 
    \left\lfloor x \right\rfloor \log^{m} x
    - \int_{1}^{x} m \left\lfloor t \right\rfloor \frac{1}{t}
    \log^{m-1} t dt\\
    &= x \log^{m}x
    - \left( x - \left\lfloor x \right\rfloor  \right) \log^{m} x
    -m \int_{1}^{x} \frac{\left\lfloor  t \right\rfloor }{t}
    \log^{m-1}(t) dt
    \end{align*}
    Thus we must show that
    \begin{align*}
    \left| \left( x - \left\lfloor x \right\rfloor  \right) \log^{m} x
    +m \int_{1}^{x} \frac{\left\lfloor  t \right\rfloor }{t}
    \log^{m-1}(t) dt  \right| \le 
    C x \log^{m-1}x
    \end{align*}
    But $\frac{\left\lfloor  t \right\rfloor }{t}
    \log^{m-1} (t) \le \log^{m-1}(x)$ giving that
    the right hand term is $O\left( x \log^{m-1} x \right) $.
    For the left hand term,
    it suffices to show that
    $\left( x - \left\lfloor x  \right\rfloor  \right) 
    \log x \le x$, but this is clear
    since $x - \left\lfloor x \right\rfloor \le 1$ and
    $ \log x  \le x$.
\end{proof}

\begin{exercise}[E3.2]
    Let $d(n) = \sum_{d  \mid n} 1$. Show
    $d(n) \le 2 \sqrt{n} $.
    If we consider the set $D \subset \mathbb{N} $ of positive
    divisors of $n$, then we can define a bijection
    $D \to D$ by
    $k \mapsto \frac{n}{k}$. Suppose
    now that $d(n) > 2 \sqrt{n} $. 
    Suppose $d  \mid n$ and
    $d \ge  \sqrt{n} $. Then
    since $\frac{d}{n} \cdot d = n$, we must have
    $\frac{d}{n} \le  \sqrt{n} $. This implies that
    under this bijection, either the source or
    target lies in
    $\left\{ 1,\ldots, \left\lfloor \sqrt{n}  \right\rfloor \right\} $.
    Hence $d(n) = \left| D \right| \le 
    2 \left| \left\{ 1, \ldots,
    \left\lfloor \sqrt{n}  \right\rfloor \right\}  \right| 
    \le 2 \sqrt{n} $.
\end{exercise}

\begin{exercise}[E3.3]
    Prove that for every $\varepsilon > 0$, there
    exists a constant $C_{\varepsilon}$ such that
    $d(n) \le C_{\varepsilon} n^{\varepsilon}$.\\
    Hint:
    \begin{enumerate}
        \item Show that $d(n_1n_2) = d(n_1)d(n_2)$ if
            $\left( n_1,n_2 \right) =1$.
        \item Show that
            \[
            \frac{d(n)}{n^{\varepsilon}}
            = \prod_{p^{\alpha} \mid  \mid   n} \frac{\alpha+1}{
            p^{\alpha \varepsilon}}
            \] 
            where $p^{\alpha} \mid  \mid n$ means that
            $\alpha$ is a positive integer,
            $p^{\alpha} \mid n$ and
            $p^{\alpha+1}\not| n$.
        \item Split the product in $2$. Into the product
            over those primes $p < 2^{\frac{1}{\varepsilon}}$ and
            the product over the rest. Show that the
            second product is bounded by $1$.
        \item Show that the factors in the first product
            are less than $1 + \left( \varepsilon \log 2 \right)^{-1}$.
    \end{enumerate}
\end{exercise}

\begin{proof}
    We follow the hint:\\
    (1) Suppose $\left( n_1,n_2 \right) =1$. 
    Let $D$ be the set of divisors of $n_1n_2$, 
    $D_1 $ the set of divisors of $n_1$ and
    $D_2$ the set of divisors of $n_2$.
    Suppose $d_1 \in D_1, d_2 \in D_2$.
    Then $d_1 a = n_1, d_2 b = n_2$, so
    $d_1d_2 ab = n_1n_2$, hence
    $d_1d_2 \in D$. We thus obtain a map
    $D_1 \times D_2 \to D$ sending
    $\left( d_1,d_2 \right) \mapsto d_1d_2$. We claim this is
    a bijection. Suppose
    $d_1d_2 = d_1' d_2'$.
    If $d_1  \mid d_2'$, then
    $d_1 = 1$, in which case,
    $d_1' = 1$, and thus $d_2 = d_2'$.
    Suppose thus that $d_1 \neq 1$, so
    $d_1 \not| d_2'$.
    Then since $\left( d_1', d_2' \right) =1$, we have
    $d_1  \mid d_1'$. Similarly,
    $d_1'  \mid d_1$. So 
    $d_1 = d_1'$. And again $d_2 = d_2'$. This gives injectivity.
    For surjectivity, if
    $d  \mid n_1n_2$, then consider
    $d_1 := \frac{d}{\left( n_2,d \right) }$ and
    $d_2 := \frac{d}{\left( n_1,d \right) }$.
    Then $d_1 d_2 = d$ and $d_1 \in D_1, d_2 \in D_2$.\\
    (2) Clearly,
    $n^{\varepsilon} = \prod_{p^{\alpha} \mid  \mid 
    n} p^{\alpha \varepsilon}$. It thus suffices to show that
    $\prod_{p^{\alpha} \mid  \mid n} \left( \alpha+1 \right) 
    = d(n)$. But if we factorize $n$ as
    $n = p_1^{\alpha_1} \cdots p_m^{\alpha_m}$, then
    it is clear that the divisors corresponds precisely to
    tuples
    $\left( a_1, \ldots, a_m \right) $ with
    $0 \le a_i \le \alpha_i$. There
    are precisely $\alpha_1 + 1$ choices for each
    $a_i$, giving
    $\left( \alpha_1 +1 \right) \cdots
    \left( \alpha_m +1 \right) = d(n)$ which indeed
    is what we wanted to show.\\
    (3) We can split the product as
    \[
    \frac{d(n)}{n^{\varepsilon}} = 
    \underbrace{\prod_{\substack{p^{\alpha} \mid  \mid n \\ p <
    2^{\frac{1}{\varepsilon}}}}
\frac{\alpha+1}{p^{\alpha \varepsilon}}}_{A}
    \cdot 
    \underbrace{\prod_{\substack{p^{\alpha} \mid  \mid n \\ p \ge 
    2^{\frac{1}{\varepsilon}}}}
\frac{\alpha+1}{p^{\alpha \varepsilon}}}_{B}
    \] 
    We claim that $B \le 1$. Indeed

    \[
    \prod_{\substack{p^{\alpha} \mid  \mid n \\ p \ge 
    2^{\frac{1}{\varepsilon}}}}
    \frac{\alpha+1}{p^{\alpha \varepsilon}}
    \le 
    \prod_{\substack{p^{\alpha} \mid  \mid n \\ p \ge 
    2^{\frac{1}{\varepsilon}}}}
    \underbrace{\frac{\alpha+1}{2^{\alpha}}}_{\le 1}
    \le 1
    \] 

    (4) For the factors in the first product, we have
    $\alpha = \left\lfloor \frac{\log n}{\log p}
    \right\rfloor$ and
    $\log p < \frac{1}{\varepsilon} \log 2$, and
    $\alpha \le \frac{\log n}{\log p}$, so
    $\frac{\log p}{\log n} \le \frac{1}{\alpha}$
    \[
    \varepsilon^2 \log p < \varepsilon \log 2
    \] 


    \[
        \frac{\alpha+1}{p^{\alpha \varepsilon}}
        \le \frac{\log n +\log p}{p^{\alpha \varepsilon} \log p}
        \le 1 + \frac{1}{\varepsilon \log 2} = 
        \frac{\varepsilon \log 2 + 1}{\varepsilon \log 2}
    \] 

    What we want to bound is
    \[
\prod_{\substack{p^{\alpha} \mid  \mid n \\ p <
    2^{\frac{1}{\varepsilon}}}}
\frac{\alpha+1}{p^{\alpha \varepsilon}}
\] 
Note here that $p$ is bounded and
as $\alpha$ increases, we should expect the denominator to
take over. However, while $\alpha$ is small,
we might have some large terms since $p^{\varepsilon}$ might
be large. All our terms are however bounded by
$p^{\varepsilon}$ by the looks of it? Then we
would get that the product is
the product is bounded by
$\prod_{p < 2^{\frac{1}{\varepsilon}}} 
\frac{\log n}{\log p} \frac{1}{p^{\varepsilon}}$


\end{proof}

\begin{exercise}[E3.4]
    Show that
    \[
    \sum_{n=1}^{\infty} \frac{d(n)}{n^{s}}
    \] 
    is absolutely convergent in
    $\Re (s) > 1$.
\end{exercise}

\begin{proof}
    Fix some $s = \sigma + it \in \mathbb{C}$ with
    $\sigma > 1$. Then choosing an $\varepsilon > 0$ with
    $1 + \varepsilon < \sigma$, we have that
    $d(n) \le C_{\varepsilon} n^{\varepsilon}$, so
    \[
    \sum \left| \frac{d(n)}{n^{s}} \right| 
    \le \sum C_{\varepsilon} \frac{n^{\varepsilon}}{n^{\sigma}}
    \le C_{\varepsilon} \sum \frac{1}{n^{\sigma - \varepsilon}} < 
    \infty.
    \] 
\end{proof}

\begin{exercise}[E3.5]
    Show that the average order of
    $d(n)$ is $\log n$, i.e., that
    \[
    \frac{1}{x} \sum_{n \le x} d(n) = \log x +
    o\left( \log x \right) .
    \] 
    Hint: Show that
    \[
    \sum_{n \le x} d(n) = \sum_{a \le x}
    \left[ \frac{x}{a} \right] 
    \] 
    where $\left[ b \right] $ is the integer
    part of $b$.
\end{exercise}


\begin{proof}
    We follow the hint.
    For each $n \in \mathbb{N} $, let
    $D_n$ denote the set of positive divisors of
    $n$. Then we want to find
    $\left| D_1 \cup \ldots \cup D_{\left[ x \right] } \right|  $.
    Now, $\left[ \frac{x}{a} \right] $ is precisely
    the amount of multiples of $a$ smaller than or equal to
    $x$, i.e., the amount of numbers in between $1$ and
    $x$ which have $a$ as a divisor.
    Hence the right hand side indeed counts the
    number of divisors of the numbers less than or equal to
     $x$ which is precisely the left hand side.
     Then we find that
     \[
     \left| \frac{1}{x}\sum_{n\le x} d(n) - \log x \right| 
     \le  \left|  \sum_{a \le x} \frac{1}{a} - \log x \right|
     \] 
\end{proof}

    %\printbibliography
\end{document}
