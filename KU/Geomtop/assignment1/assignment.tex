\documentclass[reqno]{amsart}
\usepackage{amscd, amssymb, amsmath, amsthm}
\usepackage{graphicx}
\usepackage[colorlinks=true,linkcolor=blue]{hyperref}
\usepackage[utf8]{inputenc}
\usepackage[T1]{fontenc}
\usepackage{textcomp}
\usepackage{babel}
%% for identity function 1:
\usepackage{bbm}
%%For category theory diagrams:
\usepackage{tikz-cd}


\setlength\parindent{0pt}

\pdfsuppresswarningpagegroup=1

\newtheorem{theorem}{Theorem}[section]
\newtheorem{lemma}[theorem]{Lemma}
\newtheorem{proposition}[theorem]{Proposition}
\newtheorem{corollary}[theorem]{Corollary}
\newtheorem{conjecture}[theorem]{Conjecture}

\theoremstyle{definition}
\newtheorem{definition}[theorem]{Definition}
\newtheorem{example}[theorem]{Example}
\newtheorem{exercise}[theorem]{Exercise}
\newtheorem{problem}[theorem]{Problem}
\newtheorem{question}[theorem]{Question}

\theoremstyle{remark}
\newtheorem*{remark}{Remark}
\newtheorem*{note}{Note}
\newtheorem*{solution}{Solution}



%Inequalities
\newcommand{\cycsum}{\sum_{\mathrm{cyc}}}
\newcommand{\symsum}{\sum_{\mathrm{sym}}}
\newcommand{\cycprod}{\prod_{\mathrm{cyc}}}
\newcommand{\symprod}{\prod_{\mathrm{sym}}}

%Linear Algebra

\DeclareMathOperator{\Span}{span}
\DeclareMathOperator{\im}{im}
\DeclareMathOperator{\diag}{diag}
\DeclareMathOperator{\Ker}{Ker}
\DeclareMathOperator{\ob}{ob}
\DeclareMathOperator{\Hom}{Hom}
\DeclareMathOperator{\Mor}{Mor}
\DeclareMathOperator{\sk}{sk}
\DeclareMathOperator{\Vect}{Vect}
\DeclareMathOperator{\Set}{Set}
\DeclareMathOperator{\Group}{Group}
\DeclareMathOperator{\Ring}{Ring}
\DeclareMathOperator{\Ab}{Ab}
\DeclareMathOperator{\Top}{Top}
\DeclareMathOperator{\hTop}{hTop}
\DeclareMathOperator{\Htpy}{Htpy}
\DeclareMathOperator{\Cat}{Cat}
\DeclareMathOperator{\CAT}{CAT}
\DeclareMathOperator{\Cone}{Cone}
\DeclareMathOperator{\dom}{dom}
\DeclareMathOperator{\cod}{cod}
\DeclareMathOperator{\Aut}{Aut}
\DeclareMathOperator{\Mat}{Mat}
\DeclareMathOperator{\Fin}{Fin}
\DeclareMathOperator{\rel}{rel}
\DeclareMathOperator{\Int}{Int}
\DeclareMathOperator{\sgn}{sgn}
\DeclareMathOperator{\Homeo}{Homeo}
\DeclareMathOperator{\SHomeo}{SHomeo}
\DeclareMathOperator{\PSL}{PSL}
\DeclareMathOperator{\Bil}{Bil}
\DeclareMathOperator{\Sym}{Sym}
\DeclareMathOperator{\Skew}{Skew}
\DeclareMathOperator{\Alt}{Alt}
\DeclareMathOperator{\Quad}{Quad}
\DeclareMathOperator{\Sin}{Sin}
\DeclareMathOperator{\Supp}{Supp}
\DeclareMathOperator{\Char}{char}
\DeclareMathOperator{\Teich}{Teich}
\DeclareMathOperator{\GL}{GL}
\DeclareMathOperator{\tr}{tr}
\DeclareMathOperator{\codim}{codim}


%Row operations
\newcommand{\elem}[1]{% elementary operations
\xrightarrow{\substack{#1}}%
}

\newcommand{\lelem}[1]{% elementary operations (left alignment)
\xrightarrow{\begin{subarray}{l}#1\end{subarray}}%
}

%SS
\DeclareMathOperator{\supp}{supp}
\DeclareMathOperator{\Var}{Var}

%NT
\DeclareMathOperator{\ord}{ord}

%Alg
\DeclareMathOperator{\Rad}{Rad}
\DeclareMathOperator{\Jac}{Jac}

%Misc
\newcommand{\SL}{{\mathrm{SL}}}
\newcommand{\mobgp}{{\mathrm{PSL}_2(\mathbb{C})}}
\newcommand{\id}{{\mathrm{id}}}
\newcommand{\MCG}{{\mathrm{MCG}}}
\newcommand{\PMCG}{{\mathrm{PMCG}}}
\newcommand{\SMCG}{{\mathrm{SMCG}}}
\newcommand{\ud}{{\mathrm{d}}}
\newcommand{\Vol}{{\mathrm{Vol}}}
\newcommand{\Area}{{\mathrm{Area}}}
\newcommand{\diam}{{\mathrm{diam}}}
\newcommand{\End}{{\mathrm{End}}}


\newcommand{\reg}{{\mathtt{reg}}}
\newcommand{\geo}{{\mathtt{geo}}}

\newcommand{\tori}{{\mathcal{T}}}
\newcommand{\cpn}{{\mathtt{c}}}
\newcommand{\pat}{{\mathtt{p}}}

\let\Cap\undefined
\newcommand{\Cap}{{\mathcal{C}}ap}
\newcommand{\Push}{{\mathcal{P}}ush}
\newcommand{\Forget}{{\mathcal{F}}orget}



\title{Assignment 1}
\author{Jonas Trepiakas}
\date{}

\begin{document}

\maketitle
    
\begin{problem}[1]
    Classify all topological $1$-manifolds up to homeomorphism.
\end{problem}

\begin{proof}
    Let $M$ be a $1$-manifold. Since components of a manifold
    are themselves manifolds, we may assume $M$ is connected.
    We also only consider manifolds without boundary.
    The claim is that $M$ is homeomorphic to
    $S^{1}$ or $\mathbb{R}$.\\
    Suppose
    $\left( U, \varphi  \right) $ is a maximal chart
    in the sense that it is not contained in any 
    other chart. If $U$ covers all of $M$, then
    $M \cong \varphi (U) $ is a connected open subset
    of $\mathbb{R}$ which is an interval, hence
    homeomorphic to $\mathbb{R}$. So suppose
    $U$ does not cover all of $M$. 

    Since $M$ is locally-compact (as it is
    locally Euclidean) and Hausdorff, $M$ has a 
    one-point compactification. 

    We must check that the one-point compactification of a
    $1$-manifold is still a $1$-manifold. This we
    will not show for now.\\
    

    Suppose $\left( U, \varphi  \right) $ is a
    maximal chart for $M$ which is now compact. Compact
    connected
    subsets of $\mathbb{R}$ are closed bounded intervals
    which are $1$-manifolds with boundary, so
    $U$ cannot cover all of $M$. Since
    $U$ is open, it cannot be closed also since 
    closed and open sets would give a contradiction to
    the connectedness assumption. Hence
    there exists a point $x \in M -U$ which is a limit
    point of $U$. Let $\left( V, \psi  \right) $ be a 
    connected
    chart about $x$. Then $U \cap V$ has at least one
    component.
    Let $W \subset U \cap V$ be the component containing $x$.
    Then $\varphi (W), \psi (W)$ are open
    intervals in $\mathbb{R}$, hence homeomorphic.
    Let $f \colon \mathbb{R} \to \mathbb{R}$ be
    the homeomorphism sending
    $\psi (W) $ to $\varphi (W)$. Then
    $\varphi $ and $f \circ \psi $ agree
    on $W$, so we can extend $\varphi $ to a chart
    $\left( \varphi \cup  f \circ \psi ,
    U \cup B \right) $ where
    $B = B\left( x, \varepsilon \right) 
    \subset V$ is some small open ball, defined by
    \[
    \varphi \cup f \circ \psi (x)
    = 
    \begin{cases}
        \varphi (x),& x \in U\\
        f \circ \psi (x),& x \in B
    \end{cases}.
    \] 
    If $U \cup  B$ yields a larger open set,
    this contradicts maximality of $\left( 
    \varphi , U \right) $, hence not such
    ball $B$ can exist. 

    But then the intersection
    $V \cap U$ must have at least two components.
    If it have two components, then
    we can easily define a map
    $S^{1} \to V \cup  U = M$ such that the
    bottom hemisphere is send
    into $U$ and the top is sent into $V$ and
    some neighborhoods about $\left( 1,0 \right) $ and
    $\left( -1,0 \right) $ are sent to
    the two components of $U \cap V$, giving
    a homeomorphism $M \cong S^{1}$.
    To see that there cannot be more than two
    components, we cite the graph argument that Milnor gives
    on page 56 in his Topology from a differentiable viewpoint.

    Now as the non-compact connected manifolds
    have a one-point compactification by the above, we
    can obtain any non-compact one-manifold by
    deleting a point from $S^{1}$ which indeed
    gives a space homeomorphic to
    $S^{1} - \left\{ 1,0 \right\} \cong \mathbb{R}$.

\end{proof}

\begin{problem}[2]
    Show that the following spaces are topological manifolds
    \begin{enumerate}
        \item $\mathbb{R}\mathbb{P}^{n}, n \in \mathbb{N} $,
        \item $\mathbb{C}\mathbb{P}^{n}, n \in \mathbb{N} $,
        \item Stiefel manifolds $V_d \left( \mathbb{R}^{n} \right) $:
            for $d \le n$, let
            $V_d \left( \mathbb{R}^{n} \right) $ be the
            set of $d$-frames in $\mathbb{R}^{n}$, i.e.,
            the collection of linearly independent
            vectors $v_1, \ldots, v_d \in \mathbb{R}^{n}$.
            This inherits a topology as a subspace
            of $\mathbb{R}^{nd}$.
        \item Stiefel manifolds: for $d\le n$, let
            $\tilde{V_d}\left( \mathbb{R}^{n} \right) $ 
            be the collection of orthonormal frames in
            $\mathbb{R}^{n}$ (with respect to the standard
            inner product on $\mathbb{R}^{n}$ ). Again,
            this inherits a topology as a subspace of
            $\mathbb{R}^{nd}$.
        \item $\GL_n \left( \mathbb{R} \right) /
            B_n\left( \mathbb{R} \right) $, where
            $B_n(\mathbb{R})$ consists of invertible
            upper triangular matrices.
        \item Show whether each of the above examples is compact or not.
    \end{enumerate}
\end{problem}


\begin{proof}
    (1) \textit{Hausdorff:} let $\overline{x},\overline{y}
    \in \mathbb{R}\mathbb{P}^{n}$ be distinct. We have
    $\mathbb{R}\mathbb{P}^{n} := S^{n} / \sim$ where
    $a,b \in S^{n}$ are identified if and only if
    $a = \pm b$. Hence
    $x \neq \pm y$. Take some
    neighborhood $U_x$ of $x$ which is disjoint from
    $-x,y,-y$ by Hausdorffness. Now
    let $U = U_y \cup -U_y$ where
    $-U_y = \left\{ -u  \mid u \in U_y \right\} $ which
    is homeomorphic to $U$ since the antipodal map is
    a homeomorphism. Since
    $ \mathbb{R}^{n+1}$ is a regular space,
    $S^{n}$ is also regular, so there exist 
    disjoint open sets $V_y,W_y$ such that
    $\overline{U}\subset W_y$ and $y \in V_y$.
    Similarly, there exist open sets
    $V_{-y}, W_{-y}$ such that
    $\overline{U} \subset W_{-y}$ and
    $-y \in V_{-y}$. Let
    $V' = V_y \cap -V_{-y}$, and then
    $V = V' \cup -V'$. Then
    $V \cap U \subset  V' \cup -V'
    \cap \left( W_{y} \cap W_{-y} \right) 
    = \varnothing$
    since $V' \subset V_y$ and
    $-V' \subset V_{-y}$.

    Furthermore, $V$ and $U$ are saturated with respect to
    the quotient map. I.e.,
    $V = \pi^{-1}\left( \pi(V) \right) $ and
    $U = \pi^{-1}\left( \pi(U) \right) $, so
    $\pi(V),\pi(U)$ are open sets in $\mathbb{R}\mathbb{P}^{n}$,
    and they form disjoint neighborhoods around
    $\overline{x}$ and $\overline{y}$.

    \textit{Second-countable:} Suppose
    we take the collection
    $\mathcal{B}$ consisting of
    open sets 
    \[
        \pi \left( B(x,\frac{1}{n}) \cup 
    B\left( -x, \frac{1}{n} \right) \cap S^{n} \right)
\]
with
    $n \in \mathbb{N} $ and $
    x \in \mathbb{Q}^{n+1} \cap S^{n}$. Then 
    $\mathcal{B}$ is countable and consists of
    open sets in $\mathbb{R}\mathbb{P}^{n}$. Now for
    any open set $U \subset \mathbb{R}\mathbb{P}^{n}$ containing
    some $\overline{x} \in U$, we can take
    some
    $x \in \mathbb{Q}^{n+1} \cap S^{n}$ and some
    $n \in \mathbb{N} $ such that
    $ \left( B\left( x,\frac{1}{n} \right) 
    \cup B\left( -x, \frac{1}{n} \right) \right)
    \cap S^{n} \subset 
    \pi^{-1}(U)$, hence
    $U$ contains some open set $V \in \mathcal{B}$ such that
    $x \in V \subset U$. So $\mathcal{B}$ is a countable
    basis for $\mathbb{R}\mathbb{P}^{n}$.


    \textbf{Different way of showing Hausdorff and
    second-countable if we are using a different
definition of $\mathbb{R}\mathbb{P}^{n}$:}
If we are using the definition of
$\mathbb{R}\mathbb{P}^{n}$ as the space of lines
in $\mathbb{R}^{n+1}$, we can show Hausdorffness
in a different way:

    \begin{proof}
        Let $\left[ x \right] , \left[ y \right]
        \in \mathbb{R}\mathbb{P}^{n}$ be distinct.
        Then $\frac{x}{\|x\|} \neq \frac{y}{\|y\|}$. Define
        $\rho  \colon \mathbb{R}^{n+1} - \left\{ 0 \right\} \to
        S^{n}$ by $\rho (x) = \frac{x}{\|x\|}$. Since $S^{n}$ is Hausdorff, we can find
        open sets $U,V$ around $\frac{x}{\|x\|}$ and $\frac{y}{\|y\|}$, respectively.
        Since $\rho$ is continuous, $\rho^{-1}(U)$ and $\rho^{-1}(V)$ are open and
        contain $x$ and $y$. We claim that they are saturated with respect to
        $\pi  \colon \mathbb{R}^{n+1} - \left\{ 0 \right\} \to
        \mathbb{R}\mathbb{P}^{n}$. Suppose
        $w \in \pi^{-1}\left( \pi \left( \rho^{-1}(U) \right)  \right) $. Then
        $\left[ w \right] \in \pi \left( \rho^{-1}(U) \right) $. So there
        exists $u \in U$ such that $\pi \left( \rho^{-1}\left( u \right)  \right)
        = \left[ w \right] $. Then clearly, $u = \frac{w}{\|w\|}$, so
        $w = u \|w\| \in \rho^{-1}(U)$. Hence
        $\pi \left( \rho^{-1}(U) \right) $ and
        $\pi\left( \rho^{-1}(V) \right) $ are open sets around
        $\left[ x \right] $ and $\left[ y \right] $, respectively.
        Now, define an action on  $S^{n}$ by
        $\mathbb{Z}_2 \times S^{n} \to S^{n}$ by
        $\left( 0, x \right) =x$ and
        $\left( 1,x \right) = -x$. Then
        the orbit space is second countable since the quotient map is
        open, so as the orbit space is $\mathbb{R}\mathbb{P}^{n}$, we are done.\\
        Alternatively, for the Hausdorff condition, we can also say that
        $\mathbb{Z}_2$ acts properly on $S^{n}$ and since if a group $G$ acts properly
        on a Hausdorff space $X$ then $X /G$ is Hausdorff, we conclude that
        $\mathbb{R}\mathbb{P}^{n}$ is Hausdorff.
        \end{proof}

    \textit{Locally-Euclidean:} Take
    a point $
    \left[ x \right] =\left[ x_1,\ldots,x_n, x_{n+1} \right]
    \in \mathbb{R}\mathbb{P}^{n}$. Then
    some $x_i$ is non-zero. Let
    $U_i := \left\{ 
    \left[ z_1, \ldots, z_{n+1} \right] \colon
z_i \neq 0 \right\} \subset \mathbb{R}\mathbb{P}^{n}$.
So $\left[ x \right] \in U_i$. Now define the map
$\varphi_i \colon U_i \to \mathbb{R}^{n}$ by
$\varphi \left( \left[ z_1, \ldots, z_{n+1} \right]  \right) 
= \left( \frac{z_1}{z_i}, \ldots,
    \hat{z_i}, \ldots,
\frac{z_{n+1}}{z_i}\right) $ where
$\hat{z_i}$ means that this coordinate is excluded.
This is well-defined since $\frac{z_j}{z_i}=
\frac{\lambda z_j}{\lambda z_i}$ for all
$\lambda \neq 0$. In fact, this is a 
homeomorphism with inverse
$\psi_{i} \colon \mathbb{R}^{n}\to U_i$ given by
$\psi \left( y_1, \ldots, y_n \right) 
= \left[y_1, \ldots, y_{i-1}, 1, y_{i},\ldots,
y_n \right] $. Hence we choose
$\left( U_i, \varphi_i \right) $ to be one
chart for $\mathbb{R}\mathbb{P}^{n}$. The collection
$U_1, \ldots, U_{n+1}$ covers $\mathbb{R}\mathbb{P}^{n}$, hence
the collection of charts
$\left\{ U_i, \varphi_i \right\} $ for
$i = 1,\ldots, n+1$, gives an atlas for
$\mathbb{R}\mathbb{P}^{n}$.\\
\linebreak
(2) \textit{Hausdorff:} The Hausdorff condition
 for $\mathbb{C}\mathbb{P}^{n}$ is checked
 in the same way as for $\mathbb{R}\mathbb{P}^{n}$, noting
 that $\mathbb{C}^{n+1}$ is a regular space.

 \textit{Second-countable:} 
This is also checked similarly as for
$\mathbb{R}\mathbb{P}^{n}$ where
the collection $\mathcal{B}$ above is now taken
for $x \in \mathbb{Q}[i]^{n+1}\cap S^{n}$.

\textit{Locally-Euclidean:} Take the same
open sets $U_i$ as for $\mathbb{R}\mathbb{P}^{n}$, now
as subsets of $\mathbb{C}\mathbb{P}^{n}$, and define
$\varphi_i$ and $\psi_i$ in the same way. It is again
clear that $\varphi_i$ and $\psi_i$ are inverses of each
other, hence $\varphi_i \colon U_i \cong \mathbb{C}^{n}
\cong \mathbb{R}^{2n}$. So composing the charts
$\varphi_i$ with the isomorphism
$\mathbb{C}^{n} \cong \mathbb{R}^{2 n}$ gives
the desired map - denote this also by
$\varphi_i$. Then
the collection  $\left( U_i, \varphi_i \right),i
= 1,\ldots, n+1$ will
again define an atlas for $\mathbb{C}\mathbb{P}^{n}$.\\
\linebreak
(3) \textit{Hausdorff:} Let
$\left( v_1, \ldots, v_d \right),
\left( w_1, \ldots, w_d \right) \in V_d\left( \mathbb{R}^{n}
\right) $ be distinct $d$-frames.
Then there exists $i$ such that
$v_i \neq w_i$, hence a coordinate
$j$ such that $v_{ij} \neq w_{ij}$. Take
disjoint neighborhoods
$V,W$ containing $v_{ij}$ and $w_{ij}$, respectively,
in $\mathbb{R}$. Then
$\mathbb{R}^{ij-1} \times V \times 
\mathbb{R}^{nd-ij} \cap
V_d\left( \mathbb{R}^{n} \right) $ and
$\mathbb{R}^{ij-1} \times W \times 
\mathbb{R}^{nd-ij} 
\cap V_d\left( \mathbb{R}^{n} \right) $ are disjoint neighborhoods
of $\left( v_1, \ldots, v_d \right) $ and
$\left( w_1, \ldots, w_d \right) $, respectively.\\
\linebreak
\textit{Second-countable:}
Every subspace of a second-countable space is
second-countable, and as $\mathbb{R}^{nd}$ is
second-countable, so is $V_d \left( \mathbb{R}^{n} \right) $.\\
\linebreak
\textit{Locally-Euclidean:} 
Consider some $d$-frame $\left( v_1, \ldots, v_d \right) $ 
as an $n\times d$ matrix. Since the columns
are linearly independent, there
exists some $d\times d$ submatrix
with non-vanishing determinant. By continuity of
of the determinant function, this
$d \times d$ matrix has an open
neighborhood in $M_{d}\left( \mathbb{R}^{n} \right) $ 
on which the determinant function is non-vanishing. 
Extend this neighborhood to one
on $\mathbb{R}^{nd}$ by choosing
$\mathbb{R}$ for the other coordinates and then
taking the product of these sets to get an
open neighborhood of
$\left( v_1, \ldots, v_d \right) $.
On this neighborhood, the corresponding matrices 
have a $d \times d$ submatrix with non-vanishing determinant
which means the $d$ columns are linearly independent. Thus this
open set is in fact contained in
$V_d \left( \mathbb{R}^{n} \right) $ and hence
also an open set in $V_d \left( \mathbb{R}^{n} \right) $
as we use the subspace topology. 
Naturally, the chart on this
open set we choose to simply be
the one sending $\left( w_1, \ldots, w_d \right) $ to
its coordinates in $\mathbb{R}^{nd}$. This
is a bijective open map which is
continuous since on any open set contained
in the open set on $\mathbb{R}^{nd}$ which is
the image of the chart, we still have 
the $d\times d$ submatrix on which the determinant
function is non-vanishing, hence still linearly independent
columns. Since a bijective, open continuous map is a homeomorphism,
this gives a chart for
$\left( v_1, \ldots, v_d \right) $.\\
\linebreak
(d) \textit{Hausdorff:} any orthonormal frame
is also linearly independent, so we can use the same
open sets intersecting with
$\tilde{V_d}\left( \mathbb{R}^{n} \right) $ as for 
$V_d \left( \mathbb{R}^{n} \right) $ above.\\
\linebreak
\textit{Second-countable:} Every subspace of a second-countable
space is second-countable, and as $\mathbb{R}^{nd}$ is
second-countable, so is $\tilde{V_d}\left( \mathbb{R}^{n}
\right) $.\\
\linebreak
\textit{Locally-Euclidean:} 
Suppose we are given an orthonormal frame
$\left( v_1,\ldots, v_d \right) 
\in \tilde{V_d}\left( \mathbb{R}^{n} \right) $.
This information can be expressed by saying that
$A^{T} A = I_{d}$ where
$A$ is the matrix whose $i$th column is $v_i$. But this
is equivalent to saying that the entries of $A$ satisfy
$\sum_{r} \alpha_{ri} \alpha_{rj} = 0$ when
$i \neq j$ and
$\sum_{r} \alpha_{ri}^2 = 1$ which is the solution set
of a map
$\mathbb{R}^{nd} \to 
\mathbb{R}^{d} \times 
\mathbb{R}^{\frac{n \left( n-1 \right) }{2}}$ given by
$\left( x_1, \ldots,x_{nd} \right) \mapsto 
\left( \sum_{r} x_{r 1}^2 -1, 
\sum_{r} x_{r 2}^2 -1 ,\ldots,
\sum_{r} x_{r d}^2 - 1 ,
\sum_{r} x_{r 1} x_{r 2}, \ldots,
\sum_{r} x_{r n-1} x_{r n}\right) $.

\[
nd - \left( d + \frac{n(n-1)}{2} \right) 
= 
\] 

(6) Since  $\mathbb{R}\mathbb{P}^n$ and
$\mathbb{C}\mathbb{P}^{n}$ are quotients of
$S^{n}$ which is
compact in $\mathbb{R}^{n+1}$ and
$\mathbb{C}^{n+1}$, and quotient maps are continuous maps hence
preserve compactness, we find that
$\mathbb{R}\mathbb{P}^{n}$ and $\mathbb{C}\mathbb{P}^{n}$ 
are compact.

For Stiefel manifolds, note that we showed that
$V_d \left( \mathbb{R}^{n} \right) $ 
is an open subset of $\mathbb{R}^{nd}$ which is Hausdorff.
Compact subsets of a Hausdorff space are closed,
so $V_d \left( \mathbb{R}^{n} \right) $ would be
closed and open, implying that  $\mathbb{R}^{nd}$ is
not connected, which is a contradiction. Thus
$\mathbb{R}^{nd}$ cannot be compact.

For the Stiefel manifold of
orthonormal frames, we expressed this as the
zero set of a continuous map between Euclidean spaces,
hence it is a closed subset of
$\mathbb{R}^{nd}$. Furthermore, this set
is also bounded since each entry
of the $n \times d$ matrix is bounded
by $1$ as each column must have norm $1$. 
Closed bounded subsets of $\mathbb{R}^{nd}$ are compact,
hence $\tilde{V_d}\left( \mathbb{R}^{n} \right) $ is 
compact.\\
\linebreak
Suppose
we take the flag
$1 \subset 
E_1 := \Span\left( e_1 \right) \subset 
E_2 := \Span \left( e_1,e_2 \right)\subset \ldots
\subset 
\Span E_{n-1} := \left( e_1, \ldots, e_{n-1} \right) 
\subset \mathbb{R}^{n} := E_n$.
Then any $A \in \GL_n(\mathbb{R})$ sends this
flag to some other flag, and
$B_n(\mathbb{R})$ is precisely the
stabilizer of the flag under this action.
Thus 
$\GL_n(\mathbb{R}) / B_n(\mathbb{R})$ can be
identified with all possible flags
of length  $n$ in $\mathbb{R}^{n}$.

However, for any flag we can choose
a basis for each space in the flag
of orthonormal vectors. Now, putting these
vectors in the columns of a matrix, we obtain
a matrix in $O(n)$. Two such matrices represent the
same flag if there is a matrix sending the
$i$ th column of one flag to the $i$ th column of another
flag. Thus $\GL_n \left( \mathbb{R} \right) /
B_n(\mathbb{R})$ can also be seen as a quotient
of $O(n)$ which is compact as a
closed bounded subset of $\mathbb{R}^{n^2}$, and so
$\GL_n(\mathbb{R})/ B_n(\mathbb{R})$ is also compact.

\end{proof}





\begin{problem}[3]
    \begin{enumerate}
        \item Show that $\mathbb{R}^{m}$ is homeomorphic
            to $\mathbb{R}^{n}$ if and only if
            $m = n$.
        \item Show that the dimension of a connected
            topological manifold is unique, i.e.,
            such a manifold $M$ cannot have dimensions
            $m$ and $n$ with $m \neq n$.
    \end{enumerate}
\end{problem}


\begin{proof}
    (1) 
    If $m = n$, then $\mathbb{R}^{m} \cong \mathbb{R}^{n}$ 
    by the identity.

    Suppose $m\neq n$ and $\mathbb{R}^{m} \cong
     \mathbb{R}^{n}$. Then also
     the one point compactifications are homeomorphic, so
     $S^{m} \cong S^{n}$.
     Suppose $m>n\ge 0$. Then
     $ \mathbb{Z} \cong H_m \left( S^{m} \right) 
     \cong H_m \left( S^{n} \right) 
     \cong 0$ which is a contradiction.\\
     \linebreak
     (2) Suppose a manifold  $M$ has dimensions
     $m$ and $n$ with $m\neq n$.
     Let $p \in M$ and choose
     two charts $\left( U,\varphi  \right) ,
     \left( V, \psi  \right) $ centered at
     $0$ such that
     $\varphi (U) \subset \mathbb{R}^{n}$ 
     and $\psi (V) \subset \mathbb{R}^{m}$,
     and assume $n > m$. We can take some
     $B\left( 0,\varepsilon \right) 
     \subset \varphi (U)$ and take its preimage under
     $\varphi $ to replace $U$ with an open set
     whose image under $\varphi $ is
     $B\left( 0,\varepsilon \right) $, so assume without
     loss of generality that
     $\varphi (U) = B\left( 0,\varepsilon \right) $.
     Then $\psi \circ \varphi^{-1} \colon
     B\left( 0,\varepsilon \right) \to 
     \psi \left( U \cap V \right) \subset 
     \mathbb{R}^{m}$ is a homeomorphism.
     Restricting to 
     $S^{n-1} \cong
     \partial \overline{B\left( 0, \frac{\varepsilon}{2}
     \right) }$, we get an embedding
     $S^{n-1} \hookrightarrow \mathbb{R}^{m}$.
     But by the standard inclusion
     $S^{m} \hookrightarrow S^{n-1}$, we get
     an injective composite map
     $S^{m} \hookrightarrow S^{n-1} \hookrightarrow
     \mathbb{R}^{m}$, which is a contradiction
     by the Borsuk-Ulam theorem.
\end{proof}

\begin{problem}[4]
    \begin{enumerate}
        \item Let $M$ and $N$ be two topological
            manifolds. Show that
            $M \times N$ is again a topological manifold
            of dimension $\dim M + \dim N$.
        \item Let $M$ and $N$ be two topological
            manifolds of the same dimension. Show that
            $M \sqcup N$  is again a topological
            manifold.
        \item Show that the connected components of a
            manifold are again manifolds; in other
            words, every manifold is written as a disjoint
            union of a collection of connected manifolds.
            Can this collection be uncountable?
    \end{enumerate}
\end{problem}

\begin{proof}
    (1) \textit{Hausdorff:} The product of two Hausdorff spaces
    is Hausdorff in the product topology, so
    as $M,N$ are manifolds hence Hausdorff, so is
    $M \times N$.
    More explicitly, let $\left( m,n \right) ,
    \left( m',n' \right) \in M \times N$ be distinct, so
    assume wlog. that $m \neq m'$. Then
    take $U, V$ distinct open neighborhoods of
    $m$ and $m'$ respectively, and let $W, W'$ be any neighborhoods
    of $n$ and $n'$, respectively. In the product topology,
    $U \times W$ and $V \times W'$ are neighborhoods of
    $\left( m,n \right) $ and $\left( m',n' \right) $, respectively,
    and $U \times W \cap V \times W' =
    \left( U \cap V \right) \times \left( W \cap W' \right) 
    = \varnothing$ as $U \cap V$ is empty.\\
    \linebreak

    \textit{Second-countable:} 
    Likewise, the finite product of
    second-countable spaces is countable, so
    as $M,N$ are manifolds hence second-countable, so
    is $M \times N$.

    We will also show it. Suppose
    $\mathcal{B}$ and $\mathcal{B}'$ are countable
    bases for $M$ and $N$, respectively. 
    We claim that
    $\mathcal{B} \times \mathcal{B}'
    = \left\{ U \times V \mid 
    U \in \mathcal{B}, V \in \mathcal{B}'\right\} $ is
    a countable basis for
    $M \times N$. Firstly, it
    is countable since the product of a finite collection
    of countable sets is countable. Now,
    suppose $W$ is an open set in
    $M \times N$ and let
    $\left( m,n \right)  \in W$. Because sets of the form
    $U \times V$ for $U$ open in $M$ and $V$ open in
    $N$ form a basis for the product topology, we
    can find such $U$ and $V$ such that
    $\left( m,n \right) \in U \times V
    \subset W$. But now we can find
    $U_{\mathcal{B}} \in \mathcal{B}$ and
    $V_{\mathcal{B}'} \in \mathcal{B}'$ such that
    $m \in U_{\mathcal{B}} \subset U$ and
    $n \in V_{\mathcal{B}'} \subset V$ by assumption of
    $\mathcal{B}$ and $\mathcal{B}'$ being bases. 
    Then
    $\left( m,n \right) \in 
    U_{\mathcal{B}} \times V_{\mathcal{B}'}
    \subset W$, showing that
    $\mathcal{B} \times \mathcal{B}'$ is a countable basis for
    the product topology on $M \times N$.\\
    \linebreak
    
    
    
    \textit{Locally-Euclidean:}

    Let $\left( p,q \right) \in 
    M \times N$ be arbitrary with
    $m = \dim M$ and
     $n = \dim N$. Choose
    charts $\left( U,\varphi  \right) $ and
    $\left( V, \psi  \right) $ for
    $p$ and $q$, respectively.
    Then define a chart
    $\varphi  \times \psi \colon
    U \times V \to \mathbb{R}^{m+n}$ by sending
    $\varphi \times \psi \left( u,v \right) 
    = \left( \varphi (u)_1, \ldots,
    \varphi (u)_m, \psi (v)_1, \ldots,
\psi (v)_n \right) $ where
$\varphi (u)_i$ is the $i$ th coordinate of
$\varphi (u)$ and
$\psi (v)_j$ is the $j$ th coordinate of
$\psi (v)$. Taking the
product topology on $M \times N$, 
$\varphi \times \psi $ becomes a homeomorphism 
of the open set
$U \times V$ onto its image in $\mathbb{R}^{m+n}$ which
is an open set. This is
in particular because we have
inverse maps $\varphi^{-1} \colon
\varphi (U) \to U$ and
$\psi^{-1} \colon \psi (V) \to V$, so by
definition,
$\left( \varphi^{-1} \times \psi^{-1} \right) 
\circ \left( \varphi  \times \psi  \right) 
= \left( \varphi^{-1} \circ \varphi  \right) \times 
\left( \psi^{-1} \circ \psi  \right) 
= \id \times \id$ and
$\left( \varphi \times \psi  \right) 
\circ \left( \varphi^{-1} \times \psi^{-1} \right) 
= \id \times  \id$ likewise.\\
\linebreak
(2) We can define
$M \sqcup N := M \times \left\{ 1 \right\} \cup 
N \times \left\{ 0 \right\} $.
Suppose $M$ and $N$ are $n$-dimensional manifolds. Let
$\tilde{p} \in M \sqcup N$, then either
$\tilde{p} = 
\left( p,0 \right) $ with
$p \in N$ or $\tilde{p}= \left( p,1 \right)$ with
$p \in M$. Suppose without loss of generality that
$p \in N$. Take some chart $\left( U, \varphi  \right) $ 
around $p$ in $N$. Then
$\tilde{U} :=U \times [0,\frac{1}{2}) \cap
M \sqcup N$ is an open set around
$\tilde{p}$ in $M \sqcup N$.
Define a chart $\tilde{\varphi}\colon
\tilde{U} \to \mathbb{R}^{n}$ by
$\tilde{\varphi }\left( p,0 \right) 
= \varphi (p)$.
For some open set $V \subset \mathbb{R}^{n}$, we then
have
$\tilde{\varphi}^{-1}(V) =
\varphi^{-1}(V) \times \left\{ 0 \right\} 
= \varphi ^{-1}(V) \times [0, \frac{1}{2}) \cap
M \sqcup N$ which is clearly open.
Likewise, take some open subset
$W \times \left\{ 0 \right\}  \subset \tilde{U}$. Then
$\tilde{\varphi }\left( W \times \left\{ 0 \right\}  \right) 
= \varphi (W)$ which is open since
$\varphi $ is a homeomorphism on
$U$ onto its open image. As
$\varphi $ is bijective, so is
$\tilde{\varphi }$, so
$\tilde{\varphi }$ is a homeomorphism onto its
open image, so $\left( \tilde{U},\tilde{\varphi } \right) $ 
is a chart around $\tilde{p}$ in
$M \sqcup N$. As $\tilde{p}$ was arbitrary,
we see that $M \sqcup N$ is a topological
$n$ manifold.\\
\linebreak
\textit{Hausdorff:} The Hausdorff condition is
clear: if $\tilde{p}, \tilde{q} \in M \sqcup N$ are distinct
points which are
both either in  $M$ or $N$, suppose without loss of generality
both are in $N$, then $\tilde{p} = \left( p,0 \right) ,
\tilde{q} = \left( q,0 \right) $, so we can take
disjoint open neighborhoods $U,V$ around
$p$ and $q$ respectively using that $N$ is a manifold,
and then create open sets
$\tilde{U} = U \times [0,\frac{1}{2}) \cap
M \sqcup N,
\tilde{V} = V \times [0, \frac{1}{2})
\cap M \sqcup N$ which are still
disjoint and open in $M \sqcup N$ containing
$\tilde{p}$ and $\tilde{q}$, respectively.
If instead $\tilde{p}$ and $\tilde{q}$ are in, say,
$N$ and $M$, respectively, then
$\tilde{p} = \left( p,0 \right) $ and
$\tilde{q} = \left( q,1 \right) $, so
take open neighborhoods $U,V$ around $p$ and $q$, respectively.
Then
$\tilde{U} = U \times [0, \frac{1}{2}) \cap
M \sqcup N,
\tilde{V} = V \times (\frac{1}{2},1] \cap
M \sqcup N$  are disjoint open neighborhoods
of $\tilde{p}$ and $\tilde{q}$, respectively.\\
\linebreak
\textit{Second-countable:}
Let $\mathcal{B}, \mathcal{B}'$ be countable bases for
$N$ and $M$, respectively. Then
$\mathcal{B} \times \left\{ 0 \right\} \cup 
\mathcal{B}' \times \left\{ 1 \right\} $ gives
a countable basis for $M \sqcup N$.\\
\linebreak
(3) Suppose $M$ is a manifold and
$N \subset M$ is a connected component. 
We will show that $N$ is again a manifold in two ways: one
will be to show that $N$ is open in $M$, hence an open
submanifold. Secondly, we will prove it from definitions.\\
\linebreak

Since $M$ has a basis of coordinate balls,
$M$ is locally path-connected. Now for a general
locally path-connected topological space, we have
that path components are the same as components and
that the components are
are open in the topological space. (See Lee proposition
A.43). 
Therefore, $N$ is path-connected and open in
$M$.

\begin{lemma}[]
    Open subsets $U$ of a topological $m$-manifold
    $M$ inherit the structure of a topological
    $m$-manifold.
\end{lemma}

\begin{proof}
    For Hausdorffness, let $p,q \in U$ and choose disjoint
    open sets $V,W$ in $M$ around $p$ and $q$, respectively.
    Then $V \cap U$ and $W \cap U$ define open sets
    around $p$ and $q$ in $U$ in the subspace topology.

    For second-countability, let
    $\mathcal{B}$ be a countable basis for
    $M$. Then $\mathcal{B} \cap U
    = \left\{ V \cap U  \mid V \in  \mathcal{B} \right\} $ 
    is a countable basis for $U$.

    Lastly, define the
    collection
    $\mathcal{A} = \left\{ 
    \left( V \cap U , \varphi|_{U} \right) \colon
\left( V, \varphi  \right) \text{ is a chart on
 } M \right\} $. For any
 $u \in U$, there exists some chart
 $\left( V, \varphi  \right) $ for $u$ in
 $M$. Then $\left( V \cap U, \varphi|_{U} \right) $ 
 is a chart in $\mathcal{A}$. Now
 since each $\varphi|_{U}$ is a restriction
 onto an open set, 
 $\varphi|_{U}$ is still a homeomorphism onto
 its image which is open in 
 $\mathbb{R}^{m}$, hence
 $\mathcal{A}$ defines an atlas for $U$.
\end{proof}

This shows that $N \subset M$ is a manifold.\\
\linebreak


Now for the alternative direct checking of definitions:\\
Let $p \in N$. Since $M$ is a manifold, there
exists a chart $\left( U, \varphi  \right) $ around
$p$. Then there exists a ball
$B\left( \varphi (p), \varepsilon \right) \subset 
\varphi (U)$ since $\varphi (U)$ is open. Now
$\varphi^{-1}\left( B \left( \varphi (p),\varepsilon \right) 
\right) $ is a connected open subset of $U$ containing $p$ 
since homeomorphisms are in particular continuous hence preserve
connectivity. Thus we in particular have that
if we let $\tilde{U}:= \varphi^{-1}\left( 
B \left( \varphi (p), \varepsilon \right) \right) $, then
$\left( \tilde{U}, \varphi|_{\tilde{U}} \right) $ 
is a chart in $M$ hence also
in $N$ with $N$ given the subspace topology.
Thus every point $p \in N$ has a chart 
$U_p$ contained in $N$. Hausdorffness can be checked
as follows: if $p,q \in N$, then there
exist disjoint neighborhoods $V,W$ around $p$ and $q$, respectively.
Now  $V \cap U_p$ and $W \cap U_q$ are open disjoint neighborhoods
of $p$ and $q$, respectively, in $N$. Alternatively,
with $N$ inheriting the subspace topology, Hausdorffness
is inherited from $M$.

For second-countability, $M$ has a countable basis
$\mathcal{B}$, so since $N$ has the subspace topology,
$\mathcal{B} \cap N = 
\left\{ U \cap N  \mid  U \in \mathcal{B} \right\} $ 
is a countable basis for $N$.

Hence $N$ is a topological manifold of the
same dimension as $M$.\\
\linebreak
To answer whether an uncountable collection of
connected manifolds can be a manifold, we note that
given an uncountable collection of connected manifolds, 
second-countability would imply that
there exists a countable basis, suppose
$\mathcal{B}$ is such a basis.
Now let $\bigsqcup_{i \in I} M_i$ be the uncountable union of
connected manifolds. For each $i$, choose
an $x_i \in M_i$, and let
$\left( U_i, \varphi_i \right) $ be a chart around
$x_i$ contained in $M_i$ (whose existence
follows from the previous exercise). By assumption
of $\mathcal{B}$ being a basis, there
exists some $B_i \in \mathcal{B}$ such that
$x_i \in B_i \subset U_i$. But this defines an
injective map
$I \to \mathcal{B}$ where $I$ was assumed to be uncountable
and $\mathcal{B}$ countable - this is a contradiction.

\end{proof}







    %\bibliography{../refs.bib}
\end{document}
