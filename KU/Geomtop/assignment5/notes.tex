\documentclass[reqno]{amsart}
\usepackage{amscd, amssymb, amsmath, amsthm}
\usepackage{graphicx}
\usepackage[colorlinks=true,linkcolor=blue]{hyperref}
\usepackage[utf8]{inputenc}
\usepackage[T1]{fontenc}
\usepackage{textcomp}
\usepackage{babel}
%% for identity function 1:
\usepackage{bbm}
%%For category theory diagrams:
\usepackage{tikz-cd}

\usepackage[backend=biber]{biblatex}
\addbibresource{notes.bib}

\setlength\parindent{0pt}

\pdfsuppresswarningpagegroup=1

\newtheorem{theorem}{Theorem}[section]
\newtheorem{lemma}[theorem]{Lemma}
\newtheorem{proposition}[theorem]{Proposition}
\newtheorem{corollary}[theorem]{Corollary}
\newtheorem{conjecture}[theorem]{Conjecture}

\theoremstyle{definition}
\newtheorem{definition}[theorem]{Definition}
\newtheorem{example}[theorem]{Example}
\newtheorem{exercise}[theorem]{Exercise}
\newtheorem{problem}[theorem]{Problem}
\newtheorem{question}[theorem]{Question}

\theoremstyle{remark}
\newtheorem*{remark}{Remark}
\newtheorem*{note}{Note}
\newtheorem*{solution}{Solution}
\newtheorem*{slogan}{Slogan}



%Inequalities
\newcommand{\cycsum}{\sum_{\mathrm{cyc}}}
\newcommand{\symsum}{\sum_{\mathrm{sym}}}
\newcommand{\cycprod}{\prod_{\mathrm{cyc}}}
\newcommand{\symprod}{\prod_{\mathrm{sym}}}

%Linear Algebra

\DeclareMathOperator{\Span}{span}
\DeclareMathOperator{\im}{im}
\DeclareMathOperator{\diag}{diag}
\DeclareMathOperator{\Ker}{Ker}
\DeclareMathOperator{\ob}{ob}
\DeclareMathOperator{\Hom}{Hom}
\DeclareMathOperator{\Mor}{Mor}
\DeclareMathOperator{\sk}{sk}
\DeclareMathOperator{\Vect}{Vect}
\DeclareMathOperator{\Set}{Set}
\DeclareMathOperator{\Group}{Group}
\DeclareMathOperator{\Ring}{Ring}
\DeclareMathOperator{\Ab}{Ab}
\DeclareMathOperator{\Top}{Top}
\DeclareMathOperator{\hTop}{hTop}
\DeclareMathOperator{\Htpy}{Htpy}
\DeclareMathOperator{\Cat}{Cat}
\DeclareMathOperator{\CAT}{CAT}
\DeclareMathOperator{\Cone}{Cone}
\DeclareMathOperator{\dom}{dom}
\DeclareMathOperator{\cod}{cod}
\DeclareMathOperator{\Aut}{Aut}
\DeclareMathOperator{\Mat}{Mat}
\DeclareMathOperator{\Fin}{Fin}
\DeclareMathOperator{\rel}{rel}
\DeclareMathOperator{\Int}{Int}
\DeclareMathOperator{\sgn}{sgn}
\DeclareMathOperator{\Homeo}{Homeo}
\DeclareMathOperator{\SHomeo}{SHomeo}
\DeclareMathOperator{\PSL}{PSL}
\DeclareMathOperator{\Bil}{Bil}
\DeclareMathOperator{\Sym}{Sym}
\DeclareMathOperator{\Skew}{Skew}
\DeclareMathOperator{\Alt}{Alt}
\DeclareMathOperator{\Quad}{Quad}
\DeclareMathOperator{\Sin}{Sin}
\DeclareMathOperator{\Supp}{Supp}
\DeclareMathOperator{\Char}{char}
\DeclareMathOperator{\Teich}{Teich}
\DeclareMathOperator{\GL}{GL}
\DeclareMathOperator{\tr}{tr}
\DeclareMathOperator{\codim}{codim}
\DeclareMathOperator{\coker}{coker}
\DeclareMathOperator{\Diff}{Diff}
\DeclareMathOperator{\Bun}{Bun}
\DeclareMathOperator{\Sm}{Sm}
\DeclareMathOperator{\Fr}{Fr}
\DeclareMathOperator{\Cob}{Cob}




%Row operations
\newcommand{\elem}[1]{% elementary operations
\xrightarrow{\substack{#1}}%
}

\newcommand{\lelem}[1]{% elementary operations (left alignment)
\xrightarrow{\begin{subarray}{l}#1\end{subarray}}%
}

%SS
\DeclareMathOperator{\supp}{supp}
\DeclareMathOperator{\Var}{Var}

%NT
\DeclareMathOperator{\ord}{ord}

%Alg
\DeclareMathOperator{\Rad}{Rad}
\DeclareMathOperator{\Jac}{Jac}

%Misc
\newcommand{\SL}{{\mathrm{SL}}}
\newcommand{\mobgp}{{\mathrm{PSL}_2(\mathbb{C})}}
\newcommand{\id}{{\mathrm{id}}}
\newcommand{\MCG}{{\mathrm{MCG}}}
\newcommand{\PMCG}{{\mathrm{PMCG}}}
\newcommand{\SMCG}{{\mathrm{SMCG}}}
\newcommand{\ud}{{\mathrm{d}}}
\newcommand{\Vol}{{\mathrm{Vol}}}
\newcommand{\Area}{{\mathrm{Area}}}
\newcommand{\diam}{{\mathrm{diam}}}
\newcommand{\End}{{\mathrm{End}}}


\newcommand{\reg}{{\mathtt{reg}}}
\newcommand{\geo}{{\mathtt{geo}}}

\newcommand{\tori}{{\mathcal{T}}}
\newcommand{\cpn}{{\mathtt{c}}}
\newcommand{\pat}{{\mathtt{p}}}

\let\Cap\undefined
\newcommand{\Cap}{{\mathcal{C}}ap}
\newcommand{\Push}{{\mathcal{P}}ush}
\newcommand{\Forget}{{\mathcal{F}}orget}


\title{Assignment 5}
\author{Jonas Trepiakas}
\date{}


\begin{document}
\maketitle

\tableofcontents


I will start the assignment by definitions, lemmas and theorems
that we will use. I have put the problems at the back of the
assignment.

\section{Results and Theory}
\subsubsection{Coordinate bundles and fibre bundles}



\begin{lemma}[]\label{bundle-equiv-in-terms-of-maps}
    \cite[Lemma 2.8]{Steenrod}
    Let $\mathcal{B},\mathcal{B}'$ be coordinate bundles
    having the space base space, fibre and group. Then
    they are equivalent if and only if
    there exist continuous maps
    \[
    \overline{g}_{kj} \colon V_j \cap V_{k}' \to G
    \] 
    such that
    \begin{align*}
        \overline{g}_{ki}(x) 
        &= \overline{g}_{kj}(x) g_{ji}(x)\\
        \overline{g}_{lj}(x) 
        &= g_{lk}'(x) \overline{g}_{kj}(x)
    \end{align*}.
\end{lemma}


\subsubsection{Construction of a bundle from coordinate
transformations}
\begin{definition}[]
    Let $G$ be a topological group and $X$ a space.
    By a \textit{system of coordinate transformations in
    $X$ with values in $G$} is meant an indexed covering
    $\left\{ V_j \right\} $ of $X$ by open sets and
    a collection of continuous maps
    \[
    g_{ji} \colon V_{i} \cap V_j \to G
    \] 
    such that
     \[
    g_{kj}(x) g_{ji}(x) = g_{ki}(x).
    \] 
\end{definition}

\begin{remark}[]
    We have so far seen that any bundle over
    $X$ with group $G$ determines such a set of
    coordinate transformations. We now state a converse.
\end{remark}

\begin{theorem}[Existence]\label{Existence-theorem}
    \cite[Thm 3.2]{Steenrod}
    If $G$ is a topological transformation group of
    $Y$, and $\left\{ V_j \right\} , 
    \left\{ g_{ij} \right\} $ is a system
    of coordinate transformations in the space
    $X$, then there exists a bundle $\mathcal{B}$ with
    base space $X$, fibre $Y$, group $G$ and
    coordinate transformations
    $\left\{ g_{ij} \right\} $. Furthermore,
    any such bundles are equivalent.
\end{theorem}

\subsubsection{The Principal Bundle and the Principal Map}

\begin{definition}[Principal $G$-bundle]
    A bundle
    $\mathcal{B} = 
    \left\{ B,p,X,Y,G \right\} $ is called
    a principal bundle if $Y = G$ and
    $G$ operates on $Y $ by left translations.
\end{definition}

\begin{definition}[Associated prinicipal bundle]
    Let $\mathcal{B} = \left\{ B,p,X,Y,G \right\} $ be
    an arbitrary bundle. The
    \textit{associated principal bundle}
    $\tilde{B}$ of $\mathcal{B}$ is the bundle given
    by the construction/existence theorem using the
    same base space, the same $\left\{ V_j \right\} $,
    the same $\left\{ g_{ji} \right\} $ and
    the same group $G$ as for $\mathcal{B}$, but
    replacing $Y$ by $G$ and allowing $G$ to operate
    on itself by left translations.
\end{definition}

\begin{theorem}[Equivalence theorem]\label{Equivalence-theorem}
    \cite[Thm 10.3]{Steenrod}
    Two bundles having the same base space, fibre
    and group are equivalent if and only if their
    associated principal bundles are equivalent.
\end{theorem}

\begin{proof}
    By Lemma \ref{bundle-equiv-in-terms-of-maps}, 
    equivalence of bundles is purely a property
    of the coordinate transformations.
\end{proof}




\begin{definition}[Manifold bundle]
    Let $M$ be a smooth manifold. A
    manifold bundle over $M$ with structure group
    $G$ is a fiber
    bundle
    $W \to E \to M$ with
    structure group $G$ such that
    $E$ is a manifold and $E \to M$ is continuous.\\
    We say a manifold bundle over $M$ is a smooth
    manifold bundle if it is
    a smooth fiber bundle as well as
    a manifold bundle and $G$ acts by diffeomorphisms
    on $M$.
\end{definition}


    \begin{definition}[Associated bundles]
        Let $M$ be a smooth manifold, and
        fix a manifold bundle
        $E \stackrel{\xi}{\to } M$ with fibre a smooth
        manifold $W$ and structure group
        $G \le \Homeo (W)$. Given another smooth manifold
        $W'$ such that there exists an injective group
        homomorphism $\iota \colon G \hookrightarrow
        \Homeo\left( W' \right) $, the associated
        $W'$-manifold bundle of $\xi$ is defined
        as follows. Let
        $\left\{ U_{\alpha}, \varphi_{\alpha} \right\}_{\alpha}$ 
        be a cover of $M$ by open neighborhoods together with
        trivializations
        $\varphi_{\alpha}$ of $\xi$. Transition maps
        $\varphi_{\alpha} \varphi_{\beta}^{-1}$ give rise
        to transition function
        $g_{\alpha \beta} \colon U_{\alpha} \cap
        U_{\beta} \to G \le \Homeo(W)$ satisfying the
        cocycle condition. We define the associated
        $W'$-manifold by gluing trivializations
        $U_{\alpha} \times W'$ along transition maps
        \[
        \iota \circ g_{\alpha\beta} \colon U_{\alpha}
        \cap U_{\beta} \to G 
        \stackrel{\iota}{\to } \Homeo\left( W' \right) .
        \] 
    \end{definition}

    \begin{definition}[Structure group reduction]
        Fix a manifold bundle
        $\xi \colon E \to M$ over a smooth manifold
        $M$, with fibre a smooth manifold $W$ and structure
        group $G$. Given a subgroup $H \le G$, $\xi$ 
        is said to admit a structure group reduction to
        $H$ if it is isomorphic to a bundle so that all transition
        maps $g_{\alpha \beta} \colon U_{\alpha} \cap
        U_{\beta} \to G$ take values in $H$.
    \end{definition}


    \subsubsection{Associated frame bundles and structure
    group reductions}


    \begin{problem}[]
    For a rank $d$ vector bundle
    $\xi \colon E \to M$ over a smooth manifold, we 
    define the associated frame bundle $\Fr \left( \xi \right) $ 
    as the associated $\GL_d \left( \mathbb{R} \right) $-bundle.
        \begin{enumerate}
            \item For $M$ a smooth $d$-dimensional
                manifold, we define its frame
                bundle $\Fr (M)$ as the associated
                frame bundle of its tangent bundle
                $TM$. Show that
                $\Fr(M) \to M$ is a principal 
                $\GL_d \left( \mathbb{R} \right) $-bundle.
            \item Show that a manifold is
                orientable if and only if its
                frame bundle $\Fr (M)$ admits a
                $\GL_d^{+} (\mathbb{R})$ reduction of
                its structure group, where
                $\GL_d^{+}(\mathbb{R})$ is the subgroup
                of the general linear group consisting
                of invertible matrices with positive determinant.
            \item Show that a structure bundle reduction of
                the frame bundle
                $\Fr(M)$ to the orthogonal group
                $O(n) \le \GL_d(\mathbb{R})$ corresponds
                to a choice of a bundle metric on the tangent
                bundle  $TM$ of $M$.
        \end{enumerate}
    \end{problem}



    \subsubsection{The Induced Bundle}




    \begin{definition}[Induced Bundle]
        Suppose $\mathcal{B}', X$ and $\eta$ are
        as before. Form the product space
        $X \times B'$ and let
        $p \colon X \times B' \to X, h\colon
        X \times B' \to B'$ be the natural projections.
        Define $B = X \times_{X'} B' :=
        \left\{ (x,b') \in X \times B'  \mid 
        \eta (x) = p'(b') \right\} $ to be the
        fibered product.\\
        We want to give
        $\left[ p \colon B \to X \right] $ a fibre
        bundle structure (by giving it a coordinate bundle structure).
        Define
        $V_j = \eta^{-1} (V_j')$ and
        set
         \[
        \varphi_j(x,y) = \left( x, 
        \varphi_j'\left( \eta(x),y \right) \right).
        \] 
        Let's give these maps some motivation.
        For these to be trivializations,
        we want
        $\varphi_j$ to be homeomorphisms
        $p^{-1}(V_j) \cap B = p|_B^{-1}\left( V_j \right) 
        \cong V_j \times Y$.
        Now,  $\varphi_j$ simply maps
        $x$ to $x$ in the first coordinate, but
        $\varphi_j'$ by assumption maps
        $V_j' \times Y$ homeomorphically onto
        $p'^{-1}(V_j')$. Hence in particular,
        $\varphi_j'\left( \eta(x), y \right) 
        \in p'^{-1}(V_j') \subset B'$. So
        $\left( x, \varphi_j' \left( \eta(x), y \right)  \right) 
        \in B$ if and only if
        $\eta (x) = p' \left( 
        \varphi_j' \left( \eta(x), y \right) \right) $, but
        this is true by assumption.
        Furthermore,
        $\left( x, \varphi_j' \left( \eta(x),y \right)  \right) 
        \in X \times B'$, so applying $p$, we get
        $p \left( x, \varphi_j' \left( \eta(x),y \right)  \right) 
        = x$ which is in
        $V_j$ when $x \in V_j$.
        Hence putting things together,
        $\varphi_j$ maps
        $V_j \times Y$ to
        $p^{-1}\left( V_j \right) \cap B$.
        We, in fact, want to show that 
        $\varphi_j$ is a homeomorphism
        of these spaces. For this, simply note that
        the map
        $\left( u,v \right) \mapsto 
         \left( u, \pi_2 \circ \varphi_j'^{-1}(v) \right) $ 
         is an inverse.\\
         Lastly, let for
         $x \in V_i \cap V_j$,
         $g_{ij}(x) = \varphi_{i,x}^{-1} \varphi_{j,x}
         = p_i \varphi_{j,x}$


         Note then that
         \begin{align*}
             g_{ij}(x) y
             &= p_i \varphi_{j,x}(y)\\
             &= p_i \left( x, \varphi_j' \left( 
             \eta (x), y\right)  \right) \\
             &= p_i' \varphi_j' \left( \eta(x), y \right) \\
             &= g_{ij}' \left( \eta(x) \right) y
         \end{align*}
 
         So the clutching functions
         are simply
         $g_{ij}' \circ \eta$ which are indeed
         continuous.

    \end{definition}

    \begin{theorem}[Equivalence
        Theorem/pullbacks of fibre bundles with the same
        fibre and group exist
        ]\label{Equivalence-Theorem-Induced-Bundles}
        Let $\mathcal{B},\mathcal{B}'$ be two bundles having
        the same fibre and group and
        $h \colon \mathcal{B} \to \mathcal{B}'$ a bundle
        map. Let
        $\eta \colon X \to X'$  be the induced map of
        base spaces. Then the induced
        bundle $\eta^{*}\mathcal{B}'$ is equivalent to
        $\mathcal{B}$, and there
        is an equivalence
        $h_0 \colon \mathcal{B} \to \eta^{*}\mathcal{B}'$ such
        that $h$ is the composite
        $h = h^{*} \circ h_0$ where
        $h^{*} \colon \eta^{*}\mathcal{B}' \to 
        \mathcal{B}'$ is the induced map:

        \begin{equation*}
        \begin{tikzcd}
            B \ar[dr, "\exists", dashed]
            \ar[drr, bend left] \ar[ddr, bend right] & & \\
              & X \times_{X'}B'
            \cong \eta^{*}B' \ar[r] \ar[d]
            \ar[dr, phantom, "\lrcorner", very near start]
              & B' \ar[d] \\
              & X \ar[r, "\eta"] & X'
        \end{tikzcd}
        \end{equation*}
    \end{theorem}





    \begin{note}
        A "Bundle Theory" is also called a 
        Cartesian Fibration over
        $\Sm$.
    \end{note}



    \begin{definition}[Bundle Theory]
    A bundle theory is a functor from some
    arbitrary category $\mathcal{B}$ to $\Sm$ subject to the
    following conditions.\\
    Given a map $f \colon M \to N$ between smooth manifolds in
    $\Sm$, there exists a map
    $f^{*} \colon \mathcal{B}(N) \to \mathcal{B}(M)$.\\
    The solid arrows in the diagram below, the
    dashed lifts are in bijection and the diagram commutes.
    \begin{equation*}
    \begin{tikzcd}
        B' \ar[rr, bend left]
        \ar[d] \ar[r, dashed, "\psi "] & f^{*}B \ar[r]
        \ar[dr, phantom, "\exists", very near start]
        \ar[d]& B \ar[d]\\
        N \ar[r, dashed, "\varphi "] \ar[rr, bend right]
                         & P \ar[r, "f"] & M
    \end{tikzcd}
    \end{equation*}
    In the sense that given
    $\varphi $, there exists
    a $\psi $, everything commutes and
    composite map above is mapped under the functor
    to the composite map below. 

    Furthermore, it is required to satisfy gluing (the cocycle condition).
    I describe this in the solution of the problem below.\\
    

    A bundle $B \to M$ is called locally trivial if
    for each point $x \in M$, there exists
    a neighborhood $x \in U \stackrel{i}{\hookrightarrow} M$
    and there exists a bundle $B' \to *$ 
    and a pullback along $\pi \colon
    U \to *$ for $B'$ such that
    there exists an isomorphism
    $i^{*}B \cong \pi^{*}B'$.
    \end{definition}
    
    \section{Problems}

    \subsection{Principal $G$-bundles}

    Let $G$ be a discrete group. Consider the category
    $\Sm^{G}$ where objects are smooth manifolds equipped with a
    free, fixed point free action by $G$ which is
    properly discontinuous: the exists a cover
    $\left\{ U_\alpha \right\}_{\alpha \in A}$ of $M$ so that
    $\left\{ g \cdot U_{\alpha} \right\} $ are pairwise
    disjoint for all $\alpha \in A$ and $g \in G$.
    Furthermore, morphisms are smooth maps
    which are $G$-equivariant: 
    $f \colon M \to N$ is such that
    $f \left( g \cdot  x \right) = g\cdot f(x)$ for all
    $g \in G$ and $x \in M$.

    \begin{problem}[]
        \begin{enumerate}
            \item Show that for $M \in \Sm^{G}$, the quotient
                $M / G$ admits a structure of a smooth
                manifold so that the map
                $M \to M /G$ is a local diffeomorphism.
            \item Check that the association 
                $M \mapsto M / G$ defines a functor
                $\Sm^{G} \to \Sm$, and show that this defines
                a locally trivial bundle theory on smooth
                manifolds.
        \end{enumerate}
    \end{problem}

    \begin{proof}
        (1) (I will assume
        that $G$ acts by homeomorphisms on
        $M$ ) Using the covering space quotient theorem 
        (theorem 12.14 in Lee's book on Topological Manifolds), 
        we find that $M \to M /G$ is a covering space.
        To construct a smooth structure on
        $M /G$, let $p \in M /G$ and
        $U$ an evenly covered open neighborhood of
        $p$. Then
        $U$ splits into homeomorphic copies
        $\sqcup U_{\alpha}$ in $M$ with
        $\pi|_{U_{\alpha}} \colon U_{\alpha} \cong U$ homeomorphisms.
        For $\tilde{p} \in 
        U_{\alpha}$, choose a smooth chart
        $\left( V_{\tilde{p}}, \varphi_{\tilde{p}}  \right) $
        contained in $U_{\alpha}$.
        Since $\tilde{p} = g \cdot  p$ for some
        $g$, we may as well denote these charts as
        $\left( V_{g,p}, \psi_{g,p} \right) $. Now consider
        the charts
        $\left( \pi|_{g} (V_{g,p}),
        \psi_{g,p} \circ \left( \pi|_{g} \right)^{-1} \right) $.
        On an overlap
        the transition functions have the form
         \[
        \psi_{g,p} \circ \left( \pi|_g \right)^{-1}
        \left( \psi_{g',p'} \circ \left( \pi|_{g'} \right)^{-1}
        \right)^{-1}
        = \psi_{g,p} \circ \left( \pi|_{g} \right)^{-1}
        \pi|_{g'} \circ \psi_{g',p'}^{-1}
        = \psi_{g,p} \circ \psi_{g',p'}^{-1}
        \] 
        on the overlap, which is smooth by assumption.
        Hence we indeed obtain a smooth structure on
        $M /G$. In particular, the map
        $\pi \colon M \to M /G$ has coordinate form
        \[
            \left( \psi_{g,p} \circ
            \pi|_{g}^{-1} \right)  \pi \circ \psi_{g,p}^{-1}
            = \id
        \] 
        which is a diffeomorphism. So $\pi$ is a local
        diffeomorphism when we equip 
        $M /G$ with this smooth structure.\\
        \linebreak
        (2) Define the functor
        $F \colon \Sm^{G} \to \Sm$ 
        sending $M \mapsto M /G$ with the
        smooth structure defined in the first part of the
        exercise. Here,
        since
        maps $f \colon M \to N$ in $\Sm^{G}$ are
        $G$-equivariant, they, in particular,
        descend to smooth maps
        $\overline{f} \colon M / G \to N /G$, and we
        let $F(f) = \overline{f}$.
        Then indeed $F(\id_M) = \overline{\id_M} =
        \id_{M /G}$ and
        if $f \colon M \to N$ and
        $g \colon N \to P$, then
        $F \left( g \circ f \right) 
        = \overline{g \circ f}$.
        But by pasting the two squares
        \begin{equation*}
        \begin{tikzcd}
            M \ar[r, "f"] \ar[d] & N \ar[r, "g"]\ar[d] & P \ar[d]\\
            M /G \ar[r, "\overline{f}"] & N /G \ar[r, "\overline{g}"]
                                        & P /G
        \end{tikzcd}
        \end{equation*}
        we find that
        $\overline{g \circ f} = \overline{g} \circ \overline{f}$.
        So
        $F\left( g \circ f \right) 
        = F(g) \circ F(f)$.\\
        This shows that $F$ is indeed a functor.\\
        We want to show that this defines
        a bundle theory on $\Sm$.
        So suppose we have some
        $N \in \Sm^{G}$ and
        $f \colon M \to N /G$ in
        $\Sm$. Now, the quotient map
        $N \to N /G$ is a submersion (show this), so
        the pullback 
        along $f$ exists in $\Sm$, giving
        \begin{equation*}
        \begin{tikzcd}
            f^{*} N \ar[r] \ar[d]
            \ar[dr, phantom, "\lrcorner", very near start] 
            & N \ar[d] \\
            M \ar[r] & N /G
        \end{tikzcd}
        \end{equation*}
        Lastly, we must then show that
        $f^{*}N$ is in $\Sm^{G}$.
        For this, note that
        the induced bundle
        $f^{*}N$ is precisely the pullback
        which is equivalent as a fibre bundle to
        $M \times_{N /G} N$.
        But this inherits a natural action of $G$ given by
        $g \cdot 
        \left( m,n \right) = \left( m, g\cdot n \right) $.
        Choosing the same cover
        $\left\{ U_{\alpha} \right\} $ for
        $N$ as given in the condition of it being in
        $\Sm^{G}$, i.e., $\left\{ g \cdot U_{\alpha} \right\} $ 
        being disjoint for all $g$ and $\alpha$,
        the neighborhoods 
        $M \times U_{\alpha} \cap f^{*}N$ then satisfy the
        same conditions under this action of $G$. Lastly,
        the map
        $f^{*}N \cong M \times_{N /G} N \to 
        N$ given by the projection to the $N$ component which
        is the top map in the pullback diagram is
        naturally $G$-equivariant.
        This shows that the above diagram indeed can be made.

        Now suppose we have some 
        $P \in \Sm^{G}$ and a bundle map
        $P \to N$ giving the solid part of the diagram
        \begin{equation*}
        \begin{tikzcd}
            P \ar[rr, bend left] \ar[d] 
            \ar[r, dashed]& M \times_{N /G} N
            \ar[r] \ar[d] & N \ar[d]\\
            P /G \ar[r] & M \ar[r] & N /G
        \end{tikzcd}
        \end{equation*}
        where the map $P \to N$ descends to the composite map
        $P /G \to M \to N /G$ on the bottom.\\
        
        We then want to show that the dashed map
        exists. Let
        $p \colon P \to P /G$ and
        $q \colon f^{*}N \cong M \times_{N /G} N
        \to M$ be the projection. Let
        $k \colon P \to N$ be the map
        on the top. Let
        $f \colon P /G \to M$ be the map on the bottom.
        Define a map
        $h \colon P \to 
        M \times_{N /G}N$ by
        $h(x) =
        \left( f (p(x)), k(p) \right) $.
        Then if
        $l \colon M \to N /G$ denotes the map on the bottom,
        $l \circ f\left( p(x) \right) 
        = \pi \left( k(p) \right) $ where
        $\pi \colon N \to N /G$. By definition then
        $h(x) \in M \times_{N /G} N$.
        Furthermore,
        \[
        h \left( g \cdot x \right) 
        = \left( f\left( p \left( g\cdot x \right)  \right) ,
        k\left( g\cdot x \right) \right) 
        = \left( f \left( p \left( x \right)  \right) ,
        g \cdot k(x)\right) 
        = g\cdot \left( f \left( p(x) \right) ,
        k(x)\right) =
        g\cdot h(x),
        \] 
        so $h$ is $G$-equivariant.\\
        \linebreak
        Next we must check that the bundle theory
        is locally trivial. That is, we
        must check that
        for any $M \in \Sm^{G}$ and
        any point $x \in M /G$, there
        exists an open neighborhood
        $U$ about $x$ such that if we let
        $\pi \colon U \to *$ be the unique map and
        $i \colon U \to M /G$ the
        open embedding, there
        exists a manifold $N \in \Sm^{G}$ such that
        $N /G \cong *$, and such that
        the pullbacks are isomorphic:
        $i^{*}M \cong \pi^{*} N$.

        Note that these pullbacks are really
        \begin{equation*}
        \begin{tikzcd}
            U \times_{M /G}M
            \cong i^{*}M \ar[r] \ar[d] & M \ar[d,"p"] \\
            U \ar[r] & M /G
        \end{tikzcd}
        \end{equation*}
        But clearly if
        $\left( u,m \right) \in 
        U \times_{M /G} M$, then
        essentially $\overline{m} = u$, so
        $U \times_{M /G }M \cong p^{-1}(U)$, and
        \begin{equation*}
        \begin{tikzcd}
            U \times N
            \cong U \times_{*} N \ar[r] \ar[d] & N \ar[d] \\
            U \ar[r] & *
        \end{tikzcd}
        \end{equation*}
        So we find that the condition is indeed equivalent
        to the usual one: the existence of
        a neighborhood $U$ about $x$ and a homeomorphism
        $p^{-1}(U) \cong U \times N$.
        In this case, suppose
        $x \in M /G$ and simply choose one of the
        $U_{\alpha}$ such that 
        $x \in p\left( U_{\alpha} \right) $. Note that
        this is open in $M /G$ since
        the $g \cdot  U_{\alpha}$ are pairwise disjoint
        and $g$ acts by homeomorphisms ($G$ is discrete and
        each $g$ has $g^{-1}$ as inverse).
        Choosing  $U = p\left( U_{\alpha} \right) $, we get
        $p^{-1}(U) = \sqcup_{g \in G}U_{\alpha}
        \cong U_{\alpha} \times G
        \cong U \times G$  where
        $G \in \Sm^{G}$ is precisely $G$ considered as
        a smooth manifold with the trivial charts 
        $g \mapsto *$, at
        each $g \in G$. Indeed then
        $G /G \cong *$, so
        this satisfies the condition above.
        I.e., the functor
        $\Sm^{G} \to \Sm$ is locally trivial.\\
        \linebreak
        Lastly, we must check gluing. Namely that
        for $M \in \Sm^{G}$ and
        some open coordinate neighborhoods
        $U_i,U_j,U_k \subset M /G$, with coordinate
        maps
        $g_{ij} \colon U_i \cap U_j \to G,
        g_{jk} \colon U_{j} \cap U_k \to G$ and
        $g_{ki} \colon U_{k} \cap U_i \to G$, the
        maps satisfy
        $g_{ik} (x) = g_{ij}(x) g_{jk}(x)$ for
        $x \in U_i \cap U_j \cap U_k$.
        As we saw above,
        $p^{-1}\left( U_i \right) 
        = U_i \times G $, and we shall call this
        coordinate function $\varphi_i \colon
        U_i \times G \to p^{-1}(U_i)$.
        Let $g_{ij} (x) = 
        \varphi_{i,x}^{-1} \varphi_{j,x}$ where
        $\varphi_{i,x}(y) = \varphi_{i}(x,y)$ is the function
        considered only as a function of $y$. But then
        the condition
        $g_{ij}(x)  g_{jk}(x) = g_{ik}(x)$ follows
        trivially.\\
        \linebreak
        This completes the proof that the functor
        we constructed $\Sm^{G}\to \Sm$ is indeed
        a bundle theory over $\Sm$. 
    \end{proof}
        

    \subsection{Change of fibres of bundles}

    \begin{problem}[Change of fibres of bundles]
        Let $W_0$ and $W_1$ be two smooth manifolds, and
        let $G$ be a group which we assume as a simultaneous
        subgroup of both $\Homeo (W_0)$ and
        $\Homeo(W_1)$, i.e., we have injective group homomorphisms
        $\iota_0 \colon G \hookrightarrow
        \Homeo(W_0)$ and
        $\iota_1 \colon G \hookrightarrow (W_1)$. Given a
        fixed smooth manifold $M$, construct a bijection
        $\Bun_G^{W_0}(M) \to \Bun_G^{W_1}(M)$, where
        $\Bun_G^{W_i}(M)$ denotes the set of
        isomorphism classes of manifold bundles with fibre
        $W_i$ and structure group $G$ over the base
        space $M$.
    \end{problem}

    \begin{proof}
        Let
        $\mathcal{B} = \left\{ B,p,M,W_0,G \right\} 
        \in \Bun_G^{W_0}$.
        By Theorem \ref{Equivalence-theorem}, the
        bundle $\mathcal{B}$ is equivalent
        to its associated principal bundle
        $\tilde{\mathcal{B}} = 
        \left\{ B,p,M,G,G \right\} $ which thus
        represents the same isomorphism class.
        But by
        assumption, $G$ embeds into
        $\Homeo(W_1)$, so by Theorem 
        \ref{Existence-theorem}, also
        $\tilde{B}$ is equivalent to
        $\left\{ B,p,M,W_1,G \right\} =:
        \mathcal{B}'$ which has the
        same coordinate transformations. Thus
        $\tilde{\mathcal{B}}$ and $\tilde{\mathcal{B}'}$
        are equivalent.
        Now, seeing as equivalence of bundles is purely determined
        by their base space, fibre, structure group and
        coordinate transformations by Lemma
        \ref{bundle-equiv-in-terms-of-maps},
        this gives an injective map
        $\Bun_G^{W_0} \to \Bun_G^{W_1}$.
        We can
        simply use the existence theorem directly. Seeing
        as we can do the exact same thing to obtain an
        injective map
        $\Bun_{G}^{W_1} \to \Bun_G^{W_0}$, we obtain
        a bijection by Schröder-Bernstein.
    \end{proof}

    \subsection{Associated frame bundles and structure group
    reductions}
    I couldn't figure this one out in time.
    \begin{problem}[Associated frame bundles and structure
        group reductions]
        For a rank $d$ vector bundle $\xi \colon E
        \to M$ over a smooth manifold, we define the
        associated frame bundle $\Fr \left( \xi \right) $ as
        the associated $\GL_d \left( \mathbb{R} \right) $-bundle.
        \begin{enumerate}
            \item For $M$ a smooth $d$-dimensional manifold,
                we define its frame bundle $\Fr(M)$ as the
                associated frame bundle of its tangent
                bundle $TM$. Show that $\Fr (M) \to M$ is
                a principal $\GL_d (\mathbb{R})$-bundle.
        \end{enumerate}
    \end{problem}
    


    \subsection{Invertible Cobordisms and Boundaries of Compact
    Manifolds}
    \begin{problem}[Invertible cobordisms and boundaries of
        compact manifolds]
        Let
        $W_0 \colon M_0 \rightsquigarrow \varnothing $ and
        $W_1 \colon M_1 \rightsquigarrow \varnothing$ be
        two compact $d$-dimensional smooth cobordisms
        from compact $\left( d-1 \right) $-dimensional
        smooth manifolds $M_0$ and $M_1$ to the
        empty manifold, viewed as a
        $\left( d-1 \right) $-manifold. In other words,
        we have a smooth embedding
        $M_i \times \mathbb{R} \hookrightarrow W_i$ satisfying
        that $M_i \times (-\infty, 0]$ is
        closed, and such that
        their complement
        $W_i - \left( M_i \times \mathbb{R} \right) $
        is compact. We define
        $\Int \left( W_i \right) $ to
        be the complement of the image of
        $M_i \times (- \infty, t]$ for some $t \in \mathbb{R}$
        (and hence any $t \in \mathbb{R}$), and observe
        that $\Int \left( W_i \right) $ is again a
        smooth manifold, being an open subset of $W_i$.
        \begin{enumerate}
            \item Assume that in the situation of the
                above, $\Int (W_0)$ is diffeomorphic
                to $\Int \left( W_1 \right) $. Show that
                $M_0$ and $M_1$ are invertibly cobordant,
                i.e., there exists a cobordism
                $M_0 \rightsquigarrow M_1$ which is invertible
                in the category $\Cob_d$.
            \item Let $W$ be a smooth, open
                (i.e., non-compact) $d$-manifold. We define
                a compact closure of $W$ to be a compact
                cobordism $W' \colon
                M \rightsquigarrow \varnothing$ such that
                $W$ is diffeomorphic to
                $\Int (W')$. Assume that $W$ admits
                a compact closure $W' \colon M \rightsquigarrow
                \varnothing$. Show that the set of
                compact closures of $W$ up to isomorphism
                of their interiors is in bijection with the
                set of invertible cobordisms over $M$.
        \end{enumerate}
    \end{problem}

    \begin{proof}
        (1)

         Saying that
        $M_0 \rightsquigarrow M_1$ is invertible
        in $\Cob_d$ is precisely saying that
        there exists a cobordism
        $M_1 \rightsquigarrow M_0$ such that the
        composite cobordism
        $M_0 \rightsquigarrow M_1\rightsquigarrow M_0$
        is equivalent to the trivial cobordism
        $M_0 \rightsquigarrow M_0$. We
        will do this using the usual definition
        of cobordisms with boundaries. Then
        the problem is equivalently to show that
        we can find coborisms
        $M_0 \rightsquigarrow M_1\rightsquigarrow M_0$
        such that
        the composite
        is a product cobordism - i.e., has Morse number
        $0$.
        In this case, we are dealing with closed
        compact manifolds
        $W_0, W_1$ such that
        $\partial W_0 \cong M_0$ and
        $\partial W_1 \cong M_1$. Furthermore,
        the boundaries have closed collar
        neighborhoods
        $\partial W_i \times I$, and removing some open/usual collar
        neighborhoods of these boundaries $\partial W_i \times
        [0,1)$
        leaves us with compact spaces which are, by
        assumption, diffeomorphic.
        Now, take
        the cobordism $W_0$ and choose
        a collar neighborhood of $\partial W_0 $:
        $M_0 \times [0,1]$, where
        $M_0$ is identified with $M_0 \times 0$ in $W_0$.
        By assumption, there is a diffeomorphism
        $W_0 - \left( M_0 \times [0,1] \right)
        \cong W_1 - \left( M_1 \times [0,1] \right) $.
        Now, the diffeomorphism
        extends to the closure of the interiors
        which is also $M_i$ since the collar is a cylinder, so
        we obtain a diffeomorphism
        $h \colon M_0 \times 1 \cong M_1 \times 1$. Without
        loss of generality, we can reparametrize, to get the
        diffeomorphism
        $h \colon M_0 \times 1 \to M_1 \times 0$ since
        the boundaries of the interiors must map to each other.
        Now we can glue the collars by gluing the cobordisms
        they represent using theorem 1.4 in Milnor's book
        on h-cobordisms to get a cobordism
        $c_h $ which is the manifold
        $M_0 \times [0,1] \cup_h M_1 \times [0,1]$.
        This indeed now gives a cobordism
        $M_0 \rightsquigarrow M_1$. We can likewise
        obtain the cobordism
        $M_1  \rightsquigarrow M_0 $ which is also obtained
        by gluing
        $M_1 \times [0,1]$ with $M_0\times  [0,1]$ along
        $M_1 \times 1$ and $M_0 \times 0$. Denote this
        cobordism by $c_{h'}$.
        We claim that $c_{h} c_{h'} = \id_{M_0}$. That is, that
        $c_{h} c_{h'}$ is a product cobordism/trivial cobordism
        of $M_0$. One way to see this is
        by using theorem 1.6 in Milnor's book on $h$-cobordisms
        which says that
        $c_{h} c_{h'} = c_{h' h} = c_{\id_{M_0}}$ which indeed
        is the trivial cobordism.
        Alternatively, each collar neighborhood has no
        critical values, so
        $c_h$ and $c_{h'}$ both have Morse number $0$, and
        then corollary 3.8 in Milnor's book on
        $h$-cobordisms gives that
        $c_h c_{h'}$ also has Morse number $0$, hence
        is trivial by theorem 3.4 in the same book.
    \end{proof}


\printbibliography
\end{document}
