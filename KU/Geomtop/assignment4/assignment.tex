\documentclass[reqno]{amsart}
\usepackage{amscd, amssymb, amsmath, amsthm}
\usepackage{graphicx}
\usepackage[colorlinks=true,linkcolor=blue]{hyperref}
\usepackage[utf8]{inputenc}
\usepackage[T1]{fontenc}
\usepackage{textcomp}
\usepackage{babel}
%% for identity function 1:
\usepackage{bbm}
%%For category theory diagrams:
\usepackage{tikz-cd}


\setlength\parindent{0pt}
\setcounter{section}{1}

\pdfsuppresswarningpagegroup=1

\newtheorem{theorem}{Theorem}[section]
\newtheorem{lemma}[theorem]{Lemma}
\newtheorem{proposition}[theorem]{Proposition}
\newtheorem{corollary}[theorem]{Corollary}
\newtheorem{conjecture}[theorem]{Conjecture}

\theoremstyle{definition}
\newtheorem{definition}[theorem]{Definition}
\newtheorem{example}[theorem]{Example}
\newtheorem{exercise}[theorem]{Exercise}
\newtheorem{problem}[theorem]{Problem}
\newtheorem{question}[theorem]{Question}

\theoremstyle{remark}
\newtheorem*{remark}{Remark}
\newtheorem*{note}{Note}
\newtheorem*{solution}{Solution}



%Inequalities
\newcommand{\cycsum}{\sum_{\mathrm{cyc}}}
\newcommand{\symsum}{\sum_{\mathrm{sym}}}
\newcommand{\cycprod}{\prod_{\mathrm{cyc}}}
\newcommand{\symprod}{\prod_{\mathrm{sym}}}

%Linear Algebra

\DeclareMathOperator{\Span}{span}
\DeclareMathOperator{\im}{im}
\DeclareMathOperator{\diag}{diag}
\DeclareMathOperator{\Ker}{Ker}
\DeclareMathOperator{\ob}{ob}
\DeclareMathOperator{\Hom}{Hom}
\DeclareMathOperator{\Mor}{Mor}
\DeclareMathOperator{\sk}{sk}
\DeclareMathOperator{\Vect}{Vect}
\DeclareMathOperator{\Set}{Set}
\DeclareMathOperator{\Group}{Group}
\DeclareMathOperator{\Ring}{Ring}
\DeclareMathOperator{\Ab}{Ab}
\DeclareMathOperator{\Top}{Top}
\DeclareMathOperator{\hTop}{hTop}
\DeclareMathOperator{\Htpy}{Htpy}
\DeclareMathOperator{\Cat}{Cat}
\DeclareMathOperator{\CAT}{CAT}
\DeclareMathOperator{\Cone}{Cone}
\DeclareMathOperator{\dom}{dom}
\DeclareMathOperator{\cod}{cod}
\DeclareMathOperator{\Aut}{Aut}
\DeclareMathOperator{\Mat}{Mat}
\DeclareMathOperator{\Fin}{Fin}
\DeclareMathOperator{\rel}{rel}
\DeclareMathOperator{\Int}{Int}
\DeclareMathOperator{\sgn}{sgn}
\DeclareMathOperator{\Homeo}{Homeo}
\DeclareMathOperator{\SHomeo}{SHomeo}
\DeclareMathOperator{\PSL}{PSL}
\DeclareMathOperator{\Bil}{Bil}
\DeclareMathOperator{\Sym}{Sym}
\DeclareMathOperator{\Skew}{Skew}
\DeclareMathOperator{\Alt}{Alt}
\DeclareMathOperator{\Quad}{Quad}
\DeclareMathOperator{\Sin}{Sin}
\DeclareMathOperator{\Supp}{Supp}
\DeclareMathOperator{\Char}{char}
\DeclareMathOperator{\Teich}{Teich}
\DeclareMathOperator{\GL}{GL}
\DeclareMathOperator{\tr}{tr}
\DeclareMathOperator{\codim}{codim}
\DeclareMathOperator{\Bun}{Bun}
\DeclareMathOperator{\Diff}{Diff}


%Row operations
\newcommand{\elem}[1]{% elementary operations
\xrightarrow{\substack{#1}}%
}

\newcommand{\lelem}[1]{% elementary operations (left alignment)
\xrightarrow{\begin{subarray}{l}#1\end{subarray}}%
}

%SS
\DeclareMathOperator{\supp}{supp}
\DeclareMathOperator{\Var}{Var}

%NT
\DeclareMathOperator{\ord}{ord}

%Alg
\DeclareMathOperator{\Rad}{Rad}
\DeclareMathOperator{\Jac}{Jac}

%Misc
\newcommand{\SL}{{\mathrm{SL}}}
\newcommand{\mobgp}{{\mathrm{PSL}_2(\mathbb{C})}}
\newcommand{\id}{{\mathrm{id}}}
\newcommand{\MCG}{{\mathrm{MCG}}}
\newcommand{\PMCG}{{\mathrm{PMCG}}}
\newcommand{\SMCG}{{\mathrm{SMCG}}}
\newcommand{\ud}{{\mathrm{d}}}
\newcommand{\Vol}{{\mathrm{Vol}}}
\newcommand{\Area}{{\mathrm{Area}}}
\newcommand{\diam}{{\mathrm{diam}}}
\newcommand{\End}{{\mathrm{End}}}


\newcommand{\reg}{{\mathtt{reg}}}
\newcommand{\geo}{{\mathtt{geo}}}

\newcommand{\tori}{{\mathcal{T}}}
\newcommand{\cpn}{{\mathtt{c}}}
\newcommand{\pat}{{\mathtt{p}}}

\let\Cap\undefined
\newcommand{\Cap}{{\mathcal{C}}ap}
\newcommand{\Push}{{\mathcal{P}}ush}
\newcommand{\Forget}{{\mathcal{F}}orget}




\begin{document}
\begin{definition}[Fiber Bundle]
    Let $K$ be a topological group acting on a
    Hausdorff space $F$ as a group of homeomorphisms.
    Let $X$ and $B$ be Hausdorff spaces. By a 
    \textit{fiber bundle} over a base space
    $B$ with total space $X$, fiber $F$ and structure
    group $K$, we mean a bundle map
    $p \colon X \to B$ together with a maximal
    chart atlas
    $\Phi$ over $B$. Explicitly, $\Phi$ is a collection
    of trivializations
    $\varphi  \colon U
    \times F \to p^{-1}(U)$ such that
    \begin{enumerate}
        \item each point of $B$ has a neighborhood
            over which there is a chart in
            $\Phi$
        \item if $ \varphi \colon U \times F
            \to p^{-1} (U) $ is in
            $\Phi$ and  $V \subset U$, then the
            restriction
            $\varphi |_{V \times F}$ is also in
            $\Phi$.
        \item If $\varphi , \psi  \in \Phi$ are
            charts over $U$ then there exists a map
            $\theta \colon U \to K$ such that
            $\psi \left( u, y \right) 
            = \varphi \left( u, \theta(u) (y) \right) $ 
        \item the set $\Phi$ is maximal among the collections
            satisfying the (1),(2) and (3)
    \end{enumerate}
    The fiber bundle is called smooth if all the spaces
    are smooth manifolds and all maps involved are smooth.
\end{definition}

\begin{definition}[Manifold bundle]
    Let $M$ be a smooth manifold. A
    manifold bundle over $M$ with structure group
    $G$ is a fiber
    bundle
    $W \to E \to M$ with
    structure group $G$ such that
    $E$ is a manifold and $E \to M$ is continuous.\\
    We say a manifold bundle over $M$ is a smooth
    manifold bundle if it is 
    a smooth fiber bundle as well as
    a manifold bundle and $G$ acts by diffeomorphisms
    on $M$.
\end{definition}

\begin{problem}[Manifold bundles over $S^{1}$]
    We fix a smooth manifold $M$. The aim of this exercise
    is to study smooth manifold bundles over $S^{1}$ with
    fiber $M$.
    \begin{enumerate}
        \item Let $f \in \Diff (M)$, and consider the
            mapping torus
            \[
            T(f) := \left( M \times \left[ 0,1 \right]  \right) 
            / \sim
            \] 
            where $\sim$ identifies 
            $\left( x,0 \right) $ with
            $\left( f(x),1 \right) $ for all $x \in M$.
            Show that the projection map to the
            second factor yields a 
            smooth manifold bundle
            \[
            M \to T(f) \to S^{1}.
            \] 
        \item Show that if $f$ and $g$ are isotopic diffeomorphisms,
            the bundles $T(f) \to S^{1}$ and
            $T(g) \to S^{1}$ are isomorphic bundles.
        \item Show that the map
            \[
            \pi_0 \Diff(M) \to \Bun_M\left( S^{1} \right)
            \] 
            by
            \[
            \left[ f \right] \mapsto \left[ T(f) \right] 
            \] 
            from the mapping class group of $M$ to the
            set of isomorphism classes of
            $M$-manifold bundles over $S^{1}$ is bijective.
    \end{enumerate}
\end{problem}


\begin{problem}[2]
    Show that the following spaces admit the structure of
    smooth manifolds.
    \begin{enumerate}
        \item $O(n)$, the set of orthogonal matrices
            of degree $n\times  n$, topologized as
            a subspace of $\mathbb{R}^{n^2}$.
        \item $SO(n)$, the set of orthogonal matrices of
            degree $n \times n$ with determinant $1$.
        \item $\SL_n(\mathbb{R})$, the set
            of $\left( n\times n \right) $-matrices with
            determinant $1$.
    \end{enumerate}
\end{problem}

\begin{solution}
    (1) The orthogonal group is the zero set
    $\mathbb{V} (I)$ of the ideal
    $I = \left( \left\{ f_{ij} \right\}  \right) $ where
    \[
    f_{i,j} = \sum_{k=1}^{n} x_{ki} x_{kj}
    \quad \text{for }i \neq j \quad \text{and} \quad
    f_{ii} = \sum_{k=1}^{n} x_{ki}^2 -1
    \] 
    So defining a function
    $\varphi \colon 
    \mathbb{R}^{n^2} \to 
    \mathbb{R}^{n^2}$ by
    $\varphi \left( \left( x_{ij} \right)  \right) 
    = \left( \left( f_{ij} \right)  \right) $, then
    since $ \left( f_{ij} \right) $ is symmetric,
    we may modify this map so that
    $\varphi \left( x_{ij} \right) 
    = \left( \left( f_{ij} \right)_{i\ge j} \right) $ so
    $\varphi  \colon \mathbb{R}^{n^2} \to 
    \mathbb{R}^{\frac{n (n+1)}{2}}$.

    We can also write this map as
    $\varphi (A) = A^{t}A - I$.
    Then we find that
     \[
    \varphi '(A) = 
    \frac{d}{dt}|_{t=0} \varphi (A+ tX)
    = \frac{d}{dt}|_{t=0}
    \left( A+tX \right) \left( A+tX \right)^{t} - I
    =\frac{d}{dt}|_{t=0} A X^{t} t + A^{t}Xt
    = AX^{t} + XA^{t}
    \] 
    Now if
    $A \in \varphi^{-1}(0)$ and
    $B \in \mathbb{R}^{\frac{n(n+1)}{2}}$ represents a symmetric
    matrix, then
     \[
    \varphi '(\frac{1}{2}BA) = 
    \frac{1}{2} \left( A A^{t}B^{t} +
    BA A^{t} \right) 
    = \frac{1}{2} (B+B) = B
    \] 
    so $\varphi '$ is surjective, hence has full rank.
    Therefore, by the rank lemma (Lemma 5.9 in JB)
    $O(n) = \varphi^{-1}(0)$ is a smooth submanifold of
    $\mathbb{R}^{n^2}$ of dimension
    $n^2 - \frac{n(n+1)}{2}
    = \frac{n(n-1)}{2}$.
\end{solution}

\begin{problem}[3]
    Fix a manifold $M$ and consider the
    set $\Vect (M)$ of all isomorphism classes
    of finite dimensional real vector bundles over
     $M$.
     \begin{enumerate}
         \item For $E,E' \in \Vect(M)$, construct a 
             vector bundle $E \oplus E'$ over
             $M$ which fiberwise is obtained
             by applying the direct sum
             $V \oplus V'$. Formulate a
             universal property of $E \oplus E'$.
         \item For $E,E' \in \Vect(M)$, construct a
             vector bundle $E \otimes E'$ over
             $M$ which fiberwise is obtained
             by applying the tensor product
             $V \otimes V'$.
         \item Let $E \in \Vect(M)$ and fix
             $E' \subset E$ a subbundle of
             $E$, that is a vector bundle together with a
             map of bundles
             \begin{equation*}
             \begin{tikzcd}
                 V' \ar[rr, hookrightarrow]
                 \ar[d] & & V\ar[d] \\
                 E \ar[rr] \ar[dr] && E' \ar[dl]\\
                           & M &
             \end{tikzcd}
             \end{equation*}
             that induces linear injective maps
             on fibres. Construct a vector
             bundle $E / E'$ which fiberwise is given by
             taking the quotient vector space
             $V / V'$.
             
     \end{enumerate}
\end{problem}


\begin{solution}
    We will use
    the approach of Bröcker and Jänich by
    constructing pre-vector bundles with the desired properties.\\
    (1) (3pts) 
    We define
    $E \oplus E' = \bigcup_{p \in M} E_{p} \oplus E_{p}'$ 
    where $E_p$ and  $E_p'$ are the fibers at $p$.
    Now take $\pi \colon E \oplus E' \to M$ to be
    the projection $(e_p, e_p') \mapsto 
    p$.
    The vector space structure on
    $\left( E \oplus E' \right)_p =
    \pi^{-1}(p) = E_{p} \oplus E_{p}'$ is the
    precisely the direct sum of the vector space structures
    of $E_p$ and $E_p'$.

    For the pre-bundle atlas $\mathcal{B}$,
    let $ \mathcal{B}_E, \mathcal{B}_{E'}$ be
    bundle atlases for
     $E$ and $E'$, respectively.
     Then
     for  $\left( f_{\alpha}, U_{\alpha} \right) 
     \in \mathcal{B}_E$ and
     $\left( g_{\beta}, V_{\beta}
     \right) \in \mathcal{B}_{E'}$, let
     $\left( f_{\alpha} \oplus g_{\beta},
     U_{\alpha} \cap V_{\beta} \right) 
     \in \mathcal{B}$ where
     \[
     f_{\alpha} \oplus g_{\beta}
     \colon \pi^{-1} \left( U_{\alpha} \cap V_{\beta} \right) 
     \to U_{\alpha} \cap V_{\beta} \times \mathbb{R}^{n}
     \times \mathbb{R}^{m}
     \] 
     sending
     $\left( e_p, e_{p}' \right)  \mapsto
     \left( p, f_{\alpha}(e_p),
     g_{\beta}(e_p') \right) $
     is a bijective map
     which sends each fiber 
     $\left( E \oplus E' \right)_p$
     linearly and isomorphically
     onto
     $\left\{ p \right\} \times 
     \mathbb{R}^{n} \oplus \mathbb{R}^{m} 
     \cong \mathbb{R}^{n} \oplus \mathbb{R}^{m}$.
     Furthermore, the transition functions
     are of the form
     \[
     f_{\alpha}\oplus g_{\beta}
     \circ (f_{\alpha'} \oplus g_{\beta'})^{-1}
     (p, f_{\alpha'}(e_p),g_{\beta'}(e_p')) = 
     \left( p, f_\alpha(e_p), g_{\beta}(e_p') \right) 
     \] 
     which is continuous since each coordinate function
     is of the form
     $\id, f_{\alpha} \circ f_{\alpha'}^{-1}$ or
     $g_{\beta}\circ g_{\beta'}^{-1}$ which are
     assumed to be
     continuous.
     As for the universal property, 
     $E \oplus E'$ is the product of
     $E$ and $E'$ in
     $\Vect (M)$, so the usual universal property of
     products applies.\\
     \linebreak
     (2)(3pts) 
     Define
     $E \otimes E' :=
     \bigcup_{p \in M }
     E_p \otimes E_{p}'
     $ and
     $\pi$ the standard projection.
     Let $\mathcal{B}_E, \mathcal{B}_{E'}$ be
     bundle atlases for
     $E$ and  $E'$ respectively.
     Then, recalling that
     $\mathbb{R}^{n} \otimes \mathbb{R}^{m} 
     \cong \mathbb{R}^{nm}$ and using this identification,
     we get 
     for $\left( f_{\alpha},U_{\alpha} \right) 
     \in \mathcal{B}_E$ and
     $\left( g_{\beta}, V_{\beta} \right) \in 
     \mathcal{B}_{E'}$, the map
     $f_{\alpha} \otimes
     g_{\beta} \colon
     \pi^{-1}\left( 
     U_{\alpha} \cap V_{\beta}\right)  \to 
     U_{\alpha} \cap V_{\beta}
     \times \mathbb{R}^{nm}$ given by
     \[
     f_{\alpha}\otimes g_{\beta}
     \left( e_p \otimes e_{p}' \right) 
     = \left( p, f_{\alpha}(e_p) \otimes
     g_{\beta}\left( e_p' \right) \right) 
     \] 
     on simple tensors, and we extend this
     linearly over the fiber.\\
     The linearity then becomes automatic. 
     To see that this is an isomorphism,
     suppose
     \[
         (p,0) = 
     f_{\alpha} \otimes g_{\beta}
     \left( e_p \otimes e_p' \right) 
     = (p, f_{\alpha}(e_p) \otimes g_{\beta}(e_p'))
     \] 
     so either
     $f_{\alpha}(e_p) = 0$ or
     $g_{\beta}(e_p') = 0$. But then since
     $f_{\alpha}$ and $g_{\beta}$ are
     isomorphisms on $\pi^{-1}(U_{\alpha})$ and
     $\pi^{-1}(V_{\beta})$, respectively, this
     implies that either
     $e_p = 0$ or $e_{p}' = 0$, so
     $e_p \otimes e_p' = 0$.

     The transition maps then take on the form
     $\id$ and
     $f_{\alpha '} \circ f_{\alpha}^{-1} 
     \otimes g_{\beta' } \circ g_{\beta}^{-1}$
     which are continuous.\\
     \linebreak
     (3) (3pts) 
     Let
     $E / E' =
     \bigcup_{p \in M} E_p / E_p'$ and
     $\pi$ the standard projection. Here
     $E_p / E_p'$ is well-defined since
     $E_p'$ is a subspace of $E_p$ for all $p$ by
     assumption.
     Suppose
     $ \mathcal{B}_{E}, \mathcal{B}_{E'}$ are
     bundle atlases for
     $E$ and $E'$, respectively.
     Define for $\left( f_{\alpha},U_{\alpha} \right) 
     \in \mathcal{B}_E$ and
     $\left( g_{\beta}, V_{\beta} \right) 
     \in \mathcal{B}_{E'}$, 
     $\overline{f_{\alpha, \beta}} \colon
     \pi^{-1}(U_{\alpha} \cap V_{\beta})
     \to U_{\alpha} \cap V_{\beta}
     \times \frac{\mathbb{R}^{n}}{\mathbb{R}^{m}}
     \cong U_{\alpha} \cap V_{\beta}
     \times \mathbb{R}^{n-m}$ 
     by
     $x + E_{p}' \mapsto 
     \left( p, \overline{f_{\alpha}(x)} \right) =
     \left( p, f_{\alpha}(x) + g_{\beta}(V_{\beta}) \right) $.
     The transition maps are continuous as either the
     projection or the quotient of a continuous transition map.
\end{solution}




    %\bibliography{../refs.bib}
\end{document}
