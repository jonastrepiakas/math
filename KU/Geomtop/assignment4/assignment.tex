\documentclass[reqno]{amsart}
\usepackage{amscd, amssymb, amsmath, amsthm}
\usepackage{graphicx}
\usepackage[colorlinks=true,linkcolor=blue]{hyperref}
\usepackage[utf8]{inputenc}
\usepackage[T1]{fontenc}
\usepackage{textcomp}
\usepackage{babel}
%% for identity function 1:
\usepackage{bbm}
%%For category theory diagrams:
\usepackage{tikz-cd}


\setlength\parindent{0pt}
\setcounter{section}{1}

\pdfsuppresswarningpagegroup=1

\newtheorem{theorem}{Theorem}[section]
\newtheorem{lemma}[theorem]{Lemma}
\newtheorem{proposition}[theorem]{Proposition}
\newtheorem{corollary}[theorem]{Corollary}
\newtheorem{conjecture}[theorem]{Conjecture}

\theoremstyle{definition}
\newtheorem{definition}[theorem]{Definition}
\newtheorem{example}[theorem]{Example}
\newtheorem{exercise}[theorem]{Exercise}
\newtheorem{problem}[theorem]{Problem}
\newtheorem{question}[theorem]{Question}

\theoremstyle{remark}
\newtheorem*{remark}{Remark}
\newtheorem*{note}{Note}
\newtheorem*{solution}{Solution}



%Inequalities
\newcommand{\cycsum}{\sum_{\mathrm{cyc}}}
\newcommand{\symsum}{\sum_{\mathrm{sym}}}
\newcommand{\cycprod}{\prod_{\mathrm{cyc}}}
\newcommand{\symprod}{\prod_{\mathrm{sym}}}

%Linear Algebra

\DeclareMathOperator{\Span}{span}
\DeclareMathOperator{\im}{im}
\DeclareMathOperator{\diag}{diag}
\DeclareMathOperator{\Ker}{Ker}
\DeclareMathOperator{\ob}{ob}
\DeclareMathOperator{\Hom}{Hom}
\DeclareMathOperator{\Mor}{Mor}
\DeclareMathOperator{\sk}{sk}
\DeclareMathOperator{\Vect}{Vect}
\DeclareMathOperator{\Set}{Set}
\DeclareMathOperator{\Group}{Group}
\DeclareMathOperator{\Ring}{Ring}
\DeclareMathOperator{\Ab}{Ab}
\DeclareMathOperator{\Top}{Top}
\DeclareMathOperator{\hTop}{hTop}
\DeclareMathOperator{\Htpy}{Htpy}
\DeclareMathOperator{\Cat}{Cat}
\DeclareMathOperator{\CAT}{CAT}
\DeclareMathOperator{\Cone}{Cone}
\DeclareMathOperator{\dom}{dom}
\DeclareMathOperator{\cod}{cod}
\DeclareMathOperator{\Aut}{Aut}
\DeclareMathOperator{\Mat}{Mat}
\DeclareMathOperator{\Fin}{Fin}
\DeclareMathOperator{\rel}{rel}
\DeclareMathOperator{\Int}{Int}
\DeclareMathOperator{\sgn}{sgn}
\DeclareMathOperator{\Homeo}{Homeo}
\DeclareMathOperator{\SHomeo}{SHomeo}
\DeclareMathOperator{\PSL}{PSL}
\DeclareMathOperator{\Bil}{Bil}
\DeclareMathOperator{\Sym}{Sym}
\DeclareMathOperator{\Skew}{Skew}
\DeclareMathOperator{\Alt}{Alt}
\DeclareMathOperator{\Quad}{Quad}
\DeclareMathOperator{\Sin}{Sin}
\DeclareMathOperator{\Supp}{Supp}
\DeclareMathOperator{\Char}{char}
\DeclareMathOperator{\Teich}{Teich}
\DeclareMathOperator{\GL}{GL}
\DeclareMathOperator{\tr}{tr}
\DeclareMathOperator{\codim}{codim}
\DeclareMathOperator{\coker}{coker}
\DeclareMathOperator{\Bun}{Bun}
\DeclareMathOperator{\Diff}{Diff}


%Row operations
\newcommand{\elem}[1]{% elementary operations
\xrightarrow{\substack{#1}}%
}

\newcommand{\lelem}[1]{% elementary operations (left alignment)
\xrightarrow{\begin{subarray}{l}#1\end{subarray}}%
}

%SS
\DeclareMathOperator{\supp}{supp}
\DeclareMathOperator{\Var}{Var}

%NT
\DeclareMathOperator{\ord}{ord}

%Alg
\DeclareMathOperator{\Rad}{Rad}
\DeclareMathOperator{\Jac}{Jac}

%Misc
\newcommand{\SL}{{\mathrm{SL}}}
\newcommand{\mobgp}{{\mathrm{PSL}_2(\mathbb{C})}}
\newcommand{\id}{{\mathrm{id}}}
\newcommand{\MCG}{{\mathrm{MCG}}}
\newcommand{\PMCG}{{\mathrm{PMCG}}}
\newcommand{\SMCG}{{\mathrm{SMCG}}}
\newcommand{\ud}{{\mathrm{d}}}
\newcommand{\Vol}{{\mathrm{Vol}}}
\newcommand{\Area}{{\mathrm{Area}}}
\newcommand{\diam}{{\mathrm{diam}}}
\newcommand{\End}{{\mathrm{End}}}


\newcommand{\reg}{{\mathtt{reg}}}
\newcommand{\geo}{{\mathtt{geo}}}

\newcommand{\tori}{{\mathcal{T}}}
\newcommand{\cpn}{{\mathtt{c}}}
\newcommand{\pat}{{\mathtt{p}}}

\let\Cap\undefined
\newcommand{\Cap}{{\mathcal{C}}ap}
\newcommand{\Push}{{\mathcal{P}}ush}
\newcommand{\Forget}{{\mathcal{F}}orget}


\title{Assignment 4}
\author{Jonas Trepiakas}
\date{}

\begin{document}
\maketitle
\begin{definition}[Fiber Bundle]
    Let $K$ be a topological group acting on a
    Hausdorff space $F$ as a group of homeomorphisms.
    Let $X$ and $B$ be Hausdorff spaces. By a 
    \textit{fiber bundle} over a base space
    $B$ with total space $X$, fiber $F$ and structure
    group $K$, we mean a bundle map
    $p \colon X \to B$ together with a maximal
    chart atlas
    $\Phi$ over $B$. Explicitly, $\Phi$ is a collection
    of trivializations
    $\varphi  \colon U
    \times F \to p^{-1}(U)$ such that
    \begin{enumerate}
        \item each point of $B$ has a neighborhood
            over which there is a chart in
            $\Phi$
        \item if $ \varphi \colon U \times F
            \to p^{-1} (U) $ is in
            $\Phi$ and  $V \subset U$, then the
            restriction
            $\varphi |_{V \times F}$ is also in
            $\Phi$.
        \item If $\varphi , \psi  \in \Phi$ are
            charts over $U$ then there exists a map
            $\theta \colon U \to K$ such that
            $\psi \left( u, y \right) 
            = \varphi \left( u, \theta(u) (y) \right) $ 
        \item the set $\Phi$ is maximal among the collections
            satisfying the (1),(2) and (3)
    \end{enumerate}
    The fiber bundle is called smooth if all the spaces
    are smooth manifolds and all maps involved are smooth.
\end{definition}

\begin{definition}[Manifold bundle]
    Let $M$ be a smooth manifold. A
    manifold bundle over $M$ with structure group
    $G$ is a fiber
    bundle
    $W \to E \to M$ with
    structure group $G$ such that
    $E$ is a manifold and $E \to M$ is continuous.\\
    We say a manifold bundle over $M$ is a smooth
    manifold bundle if it is 
    a smooth fiber bundle as well as
    a manifold bundle and $G$ acts by diffeomorphisms
    on $M$.
\end{definition}

\begin{problem}[Manifold bundles over $S^{1}$]
    We fix a smooth manifold $M$. The aim of this exercise
    is to study smooth manifold bundles over $S^{1}$ with
    fiber $M$.
    \begin{enumerate}
        \item Let $f \in \Diff (M)$, and consider the
            mapping torus
            \[
            T(f) := \left( M \times \left[ 0,1 \right]  \right) 
            / \sim
            \] 
            where $\sim$ identifies 
            $\left( x,0 \right) $ with
            $\left( f(x),1 \right) $ for all $x \in M$.
            Show that the projection map to the
            second factor yields a 
            smooth manifold bundle
            \[
            M \to T(f) \to S^{1}.
            \] 
        \item Show that if $f$ and $g$ are isotopic diffeomorphisms,
            the bundles $T(f) \to S^{1}$ and
            $T(g) \to S^{1}$ are isomorphic bundles.
        \item Show that the map
            \[
            \pi_0 \Diff(M) \to \Bun_M\left( S^{1} \right)
            \] 
            by
            \[
            \left[ f \right] \mapsto \left[ T(f) \right] 
            \] 
            from the mapping class group of $M$ to the
            set of isomorphism classes of
            $M$-manifold bundles over $S^{1}$ is bijective.
    \end{enumerate}
\end{problem}


\begin{problem}[2]
    Show that the following spaces admit the structure of
    smooth manifolds.
    \begin{enumerate}
        \item $O(n)$, the set of orthogonal matrices
            of degree $n\times  n$, topologized as
            a subspace of $\mathbb{R}^{n^2}$.
        \item $SO(n)$, the set of orthogonal matrices of
            degree $n \times n$ with determinant $1$.
        \item $\SL_n(\mathbb{R})$, the set
            of $\left( n\times n \right) $-matrices with
            determinant $1$.
    \end{enumerate}
\end{problem}

\begin{solution}
    (1) (2pts) The orthogonal group is the zero set
    $\mathbb{V} (I)$ of the ideal
    $I = \left( \left\{ f_{ij} \right\}  \right) $ where
    \[
    f_{i,j} = \sum_{k=1}^{n} x_{ki} x_{kj}
    \quad \text{for }i \neq j \quad \text{and} \quad
    f_{ii} = \sum_{k=1}^{n} x_{ki}^2 -1
    \] 
    So defining a function
    $\varphi \colon 
    \mathbb{R}^{n^2} \to 
    \mathbb{R}^{n^2}$ by
    $\varphi \left( \left( x_{ij} \right)  \right) 
    = \left( \left( f_{ij} \right)  \right) $, then
    since $ \left( f_{ij} \right) $ is symmetric,
    we may modify this map so that
    $\varphi \left( x_{ij} \right) 
    = \left( \left( f_{ij} \right)_{i\ge j} \right) $ so
    $\varphi  \colon \mathbb{R}^{n^2} \to 
    \mathbb{R}^{\frac{n (n+1)}{2}}$.

    We can also write this map as
    $\varphi (A) = A^{t}A - I$.
    Then we find that
     \[
    \varphi '(A) = 
    \frac{d}{dt}|_{t=0} \varphi (A+ tX)
    = \frac{d}{dt}|_{t=0}
    \left( A+tX \right) \left( A+tX \right)^{t} - I
    =\frac{d}{dt}|_{t=0} A X^{t} t + A^{t}Xt
    = AX^{t} + XA^{t}
    \] 
    Now if
    $A \in \varphi^{-1}(0)$ and
    $B \in \mathbb{R}^{\frac{n(n+1)}{2}}$ represents a symmetric
    matrix, then
     \[
    \varphi '(\frac{1}{2}BA) = 
    \frac{1}{2} \left( A A^{t}B^{t} +
    BA A^{t} \right) 
    = \frac{1}{2} (B+B) = B
    \] 
    so $\varphi '$ is surjective, hence has full rank.
    Therefore, by the rank lemma (Lemma 5.9 in JB)
    $O(n) = \varphi^{-1}(0)$ is a smooth submanifold of
    $\mathbb{R}^{n^2}$ of dimension
    $n^2 - \frac{n(n+1)}{2}
    = \frac{n(n-1)}{2}$.\\
    \linebreak
    (2) (1pt)
    The determinant function defines a continuous
    function on $M_n(\mathbb{R})$, hence also
    on $O(n)$. On $O(n)$, it takes values in
    $\left\{ 1,-1 \right\} $, so in particular,
    $O(n) =
    \det|_{O(n)}^{-1}(-\infty, 0) 
    \sqcup \det|_{O(n)}^{-1}\left( 0,\infty \right) $ 
    has two components, each of which is a manifold by
    a previous problem sheet's problem. But
    $\det|_{O(n)}^{-1}(0, \infty) = 
    SO(n)$, so $SO(n)$ is also a smooth manifold, and in
    particular also of the same dimension as
    $O(n)$.\\
    \linebreak
    (3) (2pts) The determinant function is smooth and
    for example $I$ has determinant $1$. Now let
    $I_t$ be $I$ plus the matrix with a $t$ in the
    $1,1$ entry and $0$ elsewhere. Then
    $\det(I+tI) = (1+t)^{n} $, so
    $ \frac{d}{dt}|_{t=0}  \det
    \left( I + tI \right) 
    = n(1+t)^{n-1}|_{t=0}
    = n \neq 0$, hence
     $\det$ has full rank at $I$, so
     $1$ is a regular value, so
     $\det^{-1}(1)$ is a smooth
     submanifold of $M_n(\mathbb{R})$ by lemma 5.9 in BJ.
\end{solution}

\begin{problem}[3]
    Fix a manifold $M$ and consider the
    set $\Vect (M)$ of all isomorphism classes
    of finite dimensional real vector bundles over
     $M$.
     \begin{enumerate}
         \item For $E,E' \in \Vect(M)$, construct a 
             vector bundle $E \oplus E'$ over
             $M$ which fiberwise is obtained
             by applying the direct sum
             $V \oplus V'$. Formulate a
             universal property of $E \oplus E'$.
         \item For $E,E' \in \Vect(M)$, construct a
             vector bundle $E \otimes E'$ over
             $M$ which fiberwise is obtained
             by applying the tensor product
             $V \otimes V'$.
         \item Let $E \in \Vect(M)$ and fix
             $E' \subset E$ a subbundle of
             $E$, that is a vector bundle together with a
             map of bundles
             \begin{equation*}
             \begin{tikzcd}
                 V' \ar[rr, hookrightarrow]
                 \ar[d] & & V\ar[d] \\
                 E \ar[rr] \ar[dr] && E' \ar[dl]\\
                           & M &
             \end{tikzcd}
             \end{equation*}
             that induces linear injective maps
             on fibres. Construct a vector
             bundle $E / E'$ which fiberwise is given by
             taking the quotient vector space
             $V / V'$.
         \item Let $E$ and $E'$ be vector bundles over
             $M$, and assume we are given a bundle morphism
             $f \colon E \to E'$ such that the map on
             the fibres $V_p \to V_p'$ has constant
             rank for all $p \in M$. Construct a vector
             bundle $\ker f$ over $M$ which fiberwise
             is obtained by taking $\ker \left( V_p
             \to V_p'\right) $.
     \end{enumerate}
\end{problem}


\begin{solution}
    We will use
    the approach of Bröcker and Jänich by
    constructing pre-vector bundles with the desired properties.\\
    (1) (3pts) 
    We define
    $E \oplus E' = \bigcup_{p \in M} E_{p} \oplus E_{p}'$ 
    where $E_p$ and  $E_p'$ are the fibers at $p$.
    Now take $\pi \colon E \oplus E' \to M$ to be
    the projection $(e_p, e_p') \mapsto 
    p$.
    The vector space structure on
    $\left( E \oplus E' \right)_p =
    \pi^{-1}(p) = E_{p} \oplus E_{p}'$ is the
    precisely the direct sum of the vector space structures
    of $E_p$ and $E_p'$.

    For the pre-bundle atlas $\mathcal{B}$,
    let $ \mathcal{B}_E, \mathcal{B}_{E'}$ be
    bundle atlases for
     $E$ and $E'$, respectively.
     Then
     for  $\left( f_{\alpha}, U_{\alpha} \right) 
     \in \mathcal{B}_E$ and
     $\left( g_{\beta}, V_{\beta}
     \right) \in \mathcal{B}_{E'}$, let
     $\left( f_{\alpha} \oplus g_{\beta},
     U_{\alpha} \cap V_{\beta} \right) 
     \in \mathcal{B}$ where
     \[
     f_{\alpha} \oplus g_{\beta}
     \colon \pi^{-1} \left( U_{\alpha} \cap V_{\beta} \right) 
     \to U_{\alpha} \cap V_{\beta} \times \mathbb{R}^{n}
     \times \mathbb{R}^{m}
     \] 
     sending
     $\left( e_p, e_{p}' \right)  \mapsto
     \left( p, 
         \pi_{\mathbb{R}^{n}}
         \circ f_{\alpha}(e_p),
     \pi_{\mathbb{R}^{m}} \circ g_{\beta}(e_p') \right) $
     is a bijective map
     which sends each fiber 
     $\left( E \oplus E' \right)_p$
     linearly and isomorphically
     onto
     $\left\{ p \right\} \times 
     \mathbb{R}^{n} \oplus \mathbb{R}^{m} 
     \cong \mathbb{R}^{n} \oplus \mathbb{R}^{m}$.
     Furthermore, the transition functions
     are of the form
     \begin{align*}
     f_{\alpha}\oplus g_{\beta}
     \circ (f_{\alpha'} \oplus g_{\beta'})^{-1}
     (p, \pi_{\mathbb{R}^{n}} \circ 
     f_{\alpha'}(e_p), \pi_{\mathbb{R}^{m}} \circ
     g_{\beta'}(e_p')) 
     &= 
     \left( p, \pi_{\mathbb{R}^{n}}\circ
     f_\alpha(e_p), \pi_{\mathbb{R}^{m}} \circ 
 g_{\beta}(e_p') \right) \\
     &= \left( p, \tau 
     \left( \pi_{\mathbb{R}^{n}} \circ
 f_{\alpha'}(e_p), \pi_{\mathbb{R}^{m}}\circ
g_{\beta'}(e_p') \right) \right) 
     \end{align*}
     where
     $\tau = \tau_1 \oplus \tau_2$ where
     $\tau_1 \colon
     U_{\alpha} \cap U_{\alpha'} \to 
     \GL_n (\mathbb{R})$ and
     $\tau_2 \colon V_{\beta} \cap
     V_{\beta'} \to \GL_m(\mathbb{R})$ are the
     transition functions for
     the trivializations $\left( f_{\alpha},
     f_{\alpha'}\right) $ and
     $\left( g_{\beta}, g_{\beta'} \right) $, respecitively.
     
     Since each $\tau_1$ and $\tau_2$ is assumed to be continuous,
     $\tau$ is also. Hence $E \oplus E'$ is a vector bundle.\\
     As for the universal property, 
     $E \oplus E'$ is the product of
     $E$ and $E'$ in
     $\Vect (M)$, so the usual universal property of
     products applies.\\
     \linebreak
     (2)(3pts) 
     Define
     $E \otimes E' :=
     \bigcup_{p \in M }
     E_p \otimes E_{p}'
     $ and
     $\pi$ the standard projection.
     Let $\mathcal{B}_E, \mathcal{B}_{E'}$ be
     bundle atlases for
     $E$ and  $E'$ respectively.
     Then, recalling that
     $\mathbb{R}^{n} \otimes \mathbb{R}^{m} 
     \cong \mathbb{R}^{nm}$ and using this identification,
     we get 
     for $\left( f_{\alpha},U_{\alpha} \right) 
     \in \mathcal{B}_E$ and
     $\left( g_{\beta}, V_{\beta} \right) \in 
     \mathcal{B}_{E'}$, the map
     $f_{\alpha} \otimes
     g_{\beta} \colon
     \pi^{-1}\left( 
     U_{\alpha} \cap V_{\beta}\right)  \to 
     U_{\alpha} \cap V_{\beta}
     \times \mathbb{R}^{nm}$ given by
     \[
     f_{\alpha}\otimes g_{\beta}
     \left( e_p \otimes e_{p}' \right) 
     = \left( p, f_{\alpha}(e_p) \otimes
     g_{\beta}\left( e_p' \right) \right) 
     \] 
     on simple tensors, and we extend this
     linearly over the fiber.\\
     The linearity then becomes automatic. 
     To see that this is an isomorphism,
     suppose
     \[
         (p,0) = 
     f_{\alpha} \otimes g_{\beta}
     \left( e_p \otimes e_p' \right) 
     = (p, f_{\alpha}(e_p) \otimes g_{\beta}(e_p'))
     \] 
     so either
     $f_{\alpha}(e_p) = 0$ or
     $g_{\beta}(e_p') = 0$. But then since
     $f_{\alpha}$ and $g_{\beta}$ are
     isomorphisms on $\pi^{-1}(U_{\alpha})$ and
     $\pi^{-1}(V_{\beta})$, respectively, this
     implies that either
     $e_p = 0$ or $e_{p}' = 0$, so
     $e_p \otimes e_p' = 0$. Surjectivity is inherited from
     that of $f_{\alpha}$ and $g_{\beta}$.

     The transition maps then take on the form
     $\id$ and
     $\tau_1 \otimes \tau_2 $
     which is continuous.\\
     \linebreak
     (3) (3pts) 
     \begin{definition}[Subbundle]
         A bundle 
         $\left( E',p',B' \right) $ is a subbundle of
         $\left( E,p,B \right) $ provided
         $E'$ is a subspace of $E$, $B'$ is a subspace
         of $B$ and $p' = p|_{E'} \colon E' \to B'$.
     \end{definition}

     \begin{definition}[Vector subbundles]
         A vector subbundle
         $\left( E',p',B' \right) $ of
         $\left( E,p,B \right) $ is
         a subbundle together with a map
         $E' \to E$ which also
         induces linear injective maps on fibres
         $V' \hookrightarrow V$:
         \begin{equation*}
         \begin{tikzcd}
             V' \ar[d] \ar[rr, hookrightarrow] & & V \ar[d]\\
             E' \ar[rr] \ar[dr] && E \ar[dl]\\
                        & M &
         \end{tikzcd}
         \end{equation*}
         
     \end{definition}
     With this in mind, we let
     $E / E' =
     \bigcup_{p \in M} E_p / E_p'$ and
     $\pi$ the standard projection. Here
     $E_p / E_p'$ is well-defined since
     $E_p'$ is a subspace of $E_p$ for all $p$ by
     assumption.
     Suppose
     $ \mathcal{B}_{E}, \mathcal{B}_{E'}$ are
     bundle atlases for
     $E$ and $E'$, respectively.
     Let $\left( f_{\alpha}, U_{\alpha} \right) 
     \in \mathcal{B}_E$ and
     $\left( g_{\beta}, V_{\beta} \right) 
     \in \mathcal{B}_{E'}$.
     Let $\varphi  \colon E' \to E$ be the
     map in the definition of a sub-vector bundle.
     Then $\varphi $ induces an injective linear map
     $\id \times \tilde{\varphi }$. Following through fiberwise, we
     find that we get an induced isomorphism $\psi_{\alpha \beta
     }$ such that
     the following diagram commutes
     \begin{equation*}
     \begin{tikzcd}
         p'^{-1}\left( U_{\alpha}\cap
         V_{\beta} \right) \ar[rr, "\varphi "] \ar[d,
         "f_{\alpha}"', "\cong"] && 
         p^{-1}\left( U_{\alpha}\cap
     V_{\beta} \right) \ar[d, "g_{\beta}", "\cong"']
         \ar[r, "\coker"]& 
     \pi^{-1}\left( U_{\alpha} \cap V_{\beta} \right)
     \ar[ddl, "\psi_{\alpha \beta} ", "\cong"', bend left]\\
         U_{\alpha} \cap V_{\beta} \times \mathbb{R}^{n}
         \ar[rr, hookrightarrow, dashed, "
         \id \times \tilde{\varphi }"']
         \ar[drr, "\id \times 0"']&&
         U_{\alpha} \cap V_{\beta} \times \mathbb{R}^{m}
         \ar[d, "\id \times \coker \tilde{\varphi }"] & \\
                                       & &
                                       U_{\alpha}\cap
                                       V_{\beta} \times 
                                       \mathbb{R}^{m-n}
                                       &
     \end{tikzcd}
     \end{equation*}
     In particular, since $\tilde{\varphi }$ is
     linear, we get a global frame on
     $\mathbb{R}^{m-n}$ which transfers back to a local
     frame on $\pi^{-1}\left( U_{\alpha}\cap V_{\beta} \right) $.
     The transition functions then become linear isomorphisms
     at each point of
     $U_{\alpha} \cap V_{\beta}$, which are
     smooth.
     The collection
     $\left\{ \left( U_{\alpha} \cap
     V_{\beta} , \psi_{\alpha \beta} \right)  \right\} $ 
     then gives a pre-bundle atlas for
     $E / E'$. So we have a pre-vector bundle.\\
     \linebreak
     (4) (3 pts) (I will write the proof here, but
     I am not sure whether I should receive credit for it
     since I only understood how to solve it after reading
     the proof in Lee's book, and I am not sure I can
     find a much more different approach.
     The approach that Lee uses 
     is given in Theorem 10.34 in his book. In any case,
     here it is:)\\
     Suppose $f$ has constant rank $r$.\\
     Let $\ker f = \bigcup_{p \in M} \ker \left( 
     V_p \to V_p' \right) $ and
     $\pi$ the natural projection onto $M$.
     Let $\mathcal{B}_E$ and
     $\mathcal{B}_{E'}$ be bundle atlases for
     $E$ and $E'$, respectively. 
     Let $p \in M$ and choose some smooth
     local frame $\left( \sigma_i \right) $ for
     $E$ over some open neighborhood  $U$ of $p$.
      Then since $f$ has constant rank $r$, by rearranging, we
      may assume that
       $f \circ \sigma_1 (p), \ldots,
       f \circ \sigma_r (p)$ form a basis
       for $\left( \im f \right)|_{E_p}$.
       By continuity of the determinant, they
       remain linearly independent in some
       neighborhood $V$ of $p$. Hence
       $f \circ \sigma_1, \ldots, f \circ \sigma_r$ 
       form a local frame over $V$.
       By the local frame condition, this implies that
       $\im f$ is a smooth vector bundle. 
       Let now $V' = \Span \left( \sigma_1, \ldots,
       \sigma_r\right) $.
       Since $f|_{V'}$ is bijective,
       it is a smooth vector bundle isomorphism
       (prop 10.26 in Lee). So let
       $f|_V'^{-1} \colon \im f|_{V'} \to V'$ be the inverse.
       Let
       $\psi \colon V \to V$ be the map
       $\psi (v) = v - f_{V'}^{-1} \circ f_V (v)$.
       This is a smooth bundle morphism, and
       we claim that
       $\ker \psi  = V'$.
       If $v \in V$ then
       $\psi (v) = v$, and if
       $\psi (v) = 0$ then
       $f|_V'^{-1} \circ f(v) = 0$, so
       $v \in V'$.

       Now
       $\ker f|_V$ and $V'$ span
       $E_V$, so since
       $\ker f|_{V} \subset \im \psi $, and
       $V' \subset \ker \psi $, we must have
       $\im \psi = \ker f|_{V}$, and
       $\psi|_{ker f|_{V}} = \id $, hence
       $\ker f|_{V}$ is a smooth vector bundle over
       $V$. Since we have smooth local frames at each point, these
       glue together using the local frame criterion for
       subbundles to give a vector bundle
       $\ker f$.

    
       








\end{solution}


\begin{problem}[4]
    Let $\left( M, \cdot  \right) $ be a commutative
    monoid. The \textit{group completion} of
    $M$ is defined as
    \[
    M^{grp} := F(M) / \left( xy - (x+y) \right) 
    \] 
    where $\left( F(M), + \right) $ is the free
    abelian group on the underlying set
    of $M$.
    \begin{enumerate}
        \item Show that
            $\left( - \right)^{grp}$ defines a functor
            from the category of
            commutative monoids to the category of
            abelian groups, such that for any
            abelian group $A$, any monoid morphism
            $M \to A$ factors through
            $M^{grp}$.
        \item Let $\Vect_d \left( S^{n} \right) $ denote the
            set of isomorphism classes of vector
            bundles of rank $d$ over $S^{n}$. Show that the
            clutching map defines a bijection
            \[
            \left[ S^{n-1},
            \GL_d \left( \mathbb{R} \right) \right] 
            \to \Vect_{d}\left( S^{n} \right) 
            \] 
            where $\left[ S^{n-1}, \GL_d\left( \mathbb{R}
            \right) \right] $ denotes the set of homotopy classes
            of continuous maps.
    \end{enumerate}

    \begin{proof}
        (1) (2 pts)

        Firstly, we must show that
     $M^{grp}$ is abelian.
     It suffices to show that the generating elements
     of $F(M)$ commute in $M^{grp}$. Let
     $x,y \in M$ be such generators. Then
     $x+y = xy$ since this is the relation being modded
     out by. Now $M$ was assumed to be a \textit{commutative} monoid,
     so $xy = yx$ in $M$. Hence
     since also $yx = y+x$, we have
     $x+y = xy = yx = y+x$, so
     $M^{grp}$ has commuting generating elements, hence
     it is abelian.
     Now if $f \colon M \to N$ is a monoid morphism,
     then $f^{grp} \colon M^{grp} \to 
     N^{grp}$ is defined on
     generators by $x\mapsto f(x)$, and then
     we extend it linearly on
     $F(M)$ and lastly take the quotient.
     We must check that this is a group homomorphism.
     By the definition of $f^{grp}$, we have that
     it distributes over sums
     of generating elements, so
     $f^{grp} (\sum x_i) = 
     \overline{\sum f(x_i)} = 
     \sum \overline{f(x_i)}
     = \sum f^{grp}(x_i)$. Hence since any
     $x,y \in M^{grp}$ can be written as such, we have
     $f^{grp}\left( \sum_i x_i +\sum_j y_j \right) 
     = \sum_i f^{grp}(x_i) + \sum_j f^{grp}(y_j)
     = f^{grp}\left( \sum_i x_i \right) 
     + f^{grp}\left( \sum_j y_j \right) $.
     Furthermore, if
     $\id \colon M \to M$ is the identity, then
     $f^{grp}$ simply takes the form
     $f^{grp}(x) = x$ on all generating elements and
     extends linearly, hence $f^{grp}$ is also the
     identity on $M^{grp}$. 
     Now if $f \colon M \to N$ and $g \colon N \to S$, then
     $\left( g \circ f \right)^{grp}$ sends a generating
     element
     $x$ to $g\circ f(x)$ which uniquely determines it.
     Conversely,
     $g^{grp} \circ f^{grp}$ sends
     $x$ to $g^{grp}\left( f^{grp}(x) \right) 
     = g^{grp}\left( f(x) \right) 
     = g\circ f(x)$ and extends linearly onto all of
     $M^{grp}$. Hence
     the functions agree.
     So $\left( - \right)^{grp}$ is indeed a functor.

     Let now $\varphi  \colon M \to A$ be a monoid
     morphism with $A$ an abelian group.
     Then since
     $\varphi (xy) = \varphi (x) \varphi (y)
     = \varphi (y) \varphi (x) = \varphi (yx)$, we define a map
     $\tilde{\varphi }\colon
     M^{grp} \to A$ by
     $\tilde{\varphi }(x) = \varphi (x)$ and extend linearly.
     This is well-defined on generators since
     $\tilde{\varphi }(x+y)
     = \varphi (x) \varphi (y) = 
     \varphi (y) \varphi (x)
     = \tilde{\varphi }(y+x)$. 
     Now we have
     $\varphi (xy) = 
     \varphi (x) \varphi (y)
     = \tilde{\varphi }(x+y)
     =\tilde{\varphi }\left( \pi(x) + \pi(y) \right)
     = \tilde{\varphi }\left( \pi(xy) \right) $ 
     for $\pi \colon M \to M^{grp}$ the quotient map.
     Thus $\varphi = \tilde{\varphi }\circ \pi$.
    \end{proof}
\end{problem}


    %\bibliography{../refs.bib}
\end{document}
