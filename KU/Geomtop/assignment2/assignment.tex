\documentclass[reqno]{amsart}
\usepackage{amscd, amssymb, amsmath, amsthm}
\usepackage{graphicx}
\usepackage[colorlinks=true,linkcolor=blue]{hyperref}
\usepackage[utf8]{inputenc}
\usepackage[T1]{fontenc}
\usepackage{textcomp}
\usepackage{babel}
%% for identity function 1:
\usepackage{bbm}
%%For category theory diagrams:
\usepackage{tikz-cd}


\setlength\parindent{0pt}

\pdfsuppresswarningpagegroup=1

\newtheorem{theorem}{Theorem}[section]
\newtheorem{lemma}[theorem]{Lemma}
\newtheorem{proposition}[theorem]{Proposition}
\newtheorem{corollary}[theorem]{Corollary}
\newtheorem{conjecture}[theorem]{Conjecture}

\theoremstyle{definition}
\newtheorem{definition}[theorem]{Definition}
\newtheorem{example}[theorem]{Example}
\newtheorem{exercise}[theorem]{Exercise}
\newtheorem{problem}[theorem]{Problem}
\newtheorem{question}[theorem]{Question}

\theoremstyle{remark}
\newtheorem*{remark}{Remark}
\newtheorem*{note}{Note}
\newtheorem*{solution}{Solution}



%Inequalities
\newcommand{\cycsum}{\sum_{\mathrm{cyc}}}
\newcommand{\symsum}{\sum_{\mathrm{sym}}}
\newcommand{\cycprod}{\prod_{\mathrm{cyc}}}
\newcommand{\symprod}{\prod_{\mathrm{sym}}}

%Linear Algebra

\DeclareMathOperator{\Span}{span}
\DeclareMathOperator{\im}{im}
\DeclareMathOperator{\diag}{diag}
\DeclareMathOperator{\Ker}{Ker}
\DeclareMathOperator{\ob}{ob}
\DeclareMathOperator{\Hom}{Hom}
\DeclareMathOperator{\Mor}{Mor}
\DeclareMathOperator{\sk}{sk}
\DeclareMathOperator{\Vect}{Vect}
\DeclareMathOperator{\Set}{Set}
\DeclareMathOperator{\Group}{Group}
\DeclareMathOperator{\Ring}{Ring}
\DeclareMathOperator{\Ab}{Ab}
\DeclareMathOperator{\Top}{Top}
\DeclareMathOperator{\hTop}{hTop}
\DeclareMathOperator{\Htpy}{Htpy}
\DeclareMathOperator{\Cat}{Cat}
\DeclareMathOperator{\CAT}{CAT}
\DeclareMathOperator{\Cone}{Cone}
\DeclareMathOperator{\dom}{dom}
\DeclareMathOperator{\cod}{cod}
\DeclareMathOperator{\Aut}{Aut}
\DeclareMathOperator{\Mat}{Mat}
\DeclareMathOperator{\Fin}{Fin}
\DeclareMathOperator{\rel}{rel}
\DeclareMathOperator{\Int}{Int}
\DeclareMathOperator{\sgn}{sgn}
\DeclareMathOperator{\Homeo}{Homeo}
\DeclareMathOperator{\SHomeo}{SHomeo}
\DeclareMathOperator{\PSL}{PSL}
\DeclareMathOperator{\Bil}{Bil}
\DeclareMathOperator{\Sym}{Sym}
\DeclareMathOperator{\Skew}{Skew}
\DeclareMathOperator{\Alt}{Alt}
\DeclareMathOperator{\Quad}{Quad}
\DeclareMathOperator{\Sin}{Sin}
\DeclareMathOperator{\Supp}{Supp}
\DeclareMathOperator{\Char}{char}
\DeclareMathOperator{\Teich}{Teich}
\DeclareMathOperator{\GL}{GL}
\DeclareMathOperator{\tr}{tr}
\DeclareMathOperator{\codim}{codim}


%Row operations
\newcommand{\elem}[1]{% elementary operations
\xrightarrow{\substack{#1}}%
}

\newcommand{\lelem}[1]{% elementary operations (left alignment)
\xrightarrow{\begin{subarray}{l}#1\end{subarray}}%
}

%SS
\DeclareMathOperator{\supp}{supp}
\DeclareMathOperator{\Var}{Var}

%NT
\DeclareMathOperator{\ord}{ord}

%Alg
\DeclareMathOperator{\Rad}{Rad}
\DeclareMathOperator{\Jac}{Jac}

%Misc
\newcommand{\SL}{{\mathrm{SL}}}
\newcommand{\mobgp}{{\mathrm{PSL}_2(\mathbb{C})}}
\newcommand{\id}{{\mathrm{id}}}
\newcommand{\MCG}{{\mathrm{MCG}}}
\newcommand{\PMCG}{{\mathrm{PMCG}}}
\newcommand{\SMCG}{{\mathrm{SMCG}}}
\newcommand{\ud}{{\mathrm{d}}}
\newcommand{\Vol}{{\mathrm{Vol}}}
\newcommand{\Area}{{\mathrm{Area}}}
\newcommand{\diam}{{\mathrm{diam}}}
\newcommand{\End}{{\mathrm{End}}}


\newcommand{\reg}{{\mathtt{reg}}}
\newcommand{\geo}{{\mathtt{geo}}}

\newcommand{\tori}{{\mathcal{T}}}
\newcommand{\cpn}{{\mathtt{c}}}
\newcommand{\pat}{{\mathtt{p}}}

\let\Cap\undefined
\newcommand{\Cap}{{\mathcal{C}}ap}
\newcommand{\Push}{{\mathcal{P}}ush}
\newcommand{\Forget}{{\mathcal{F}}orget}


\title{Assignment 2}
\author{Jonas Trepiakas}
\date{}


\begin{document}
\maketitle
    \begin{problem}[1]
        Given a topological manifold $M$ of dimension
        $d \in \mathbb{N} $, we define a smooth
        atlas on $M$ as a collection of charts
        $\left( U_{\alpha}, \varphi_{\alpha} \right)_{\alpha
        \in A}$, where $U_{\alpha} \subset M$ is open
        and $\varphi_{\alpha} \colon U_{\alpha}
        \stackrel{\cong}{\to }\mathbb{R}^{d}$ is a homeomorphism,
        such that the transition maps fit into diagrams
        \begin{equation*}
        \begin{tikzcd}
           &  U_{\alpha}\cap U_{\beta} 
            \ar[dl, "\varphi_{\alpha}"'] \ar[dr, "
            \varphi_{\beta}"]& \\
           \varphi_{\alpha}\left( U_{\alpha}\cap U_{\beta} \right) 
            \ar[rr,"\varphi_{\beta} \varphi_{\alpha}^{-1}"'] &&
            \varphi_{\beta} \left( U_{\alpha} \cap
            U_{\beta} \right) 
        \end{tikzcd}
        \end{equation*}
        where the lower map $\varphi_{\beta}\varphi_{\alpha}^{-1}$ 
        is a smooth map between open subsets
        of $\mathbb{R}^{d}$.
        \begin{enumerate}
            \item (2.5 pts) Show that each smooth manifold (as defined in the
                lecture) admits a smooth atlas.
            \item (2.5 pts) Show that any topological manifold equipped
                with a smooth atlas admits the structure of a
                smooth manifold (as defined in the
                lecture)
        \end{enumerate}
    \end{problem}

    \begin{proof}
        We recall the definition given in the lecture:

        \begin{definition}[]
            For a topological space
            $X$, we let $C_K^{0}(X)$ denote the
            continuous functions on $X$ with support
            contained in $K$. 
        \end{definition}

        \begin{definition}[Smooth manifold]
            A smooth $n$-manifold is a topological $n$-manifold
            $M$ together with an $\mathbb{R}$-sub-algebra
            $C^{\infty}(M) \subset C^{0}(M)$ such that
            for every point $p \in M$, there exists
            a chart
            $i \colon \mathbb{R}^{n} \hookrightarrow 
            M$ sending $0 \mapsto p$ which is an open topological
            embedding, such that for all compact subsets
            $K \subset \mathbb{R}^{n}$, 
            $i^{*} \colon C_K^{\infty}(M)
            \cong C_K^{\infty}\left( \mathbb{R}^{n} \right) $ 
            and
            $i^{*} \colon
            C_K^{0}(M) \to C_K^{0}\left( \mathbb{R}^{n} \right) $
            are $\mathbb{R}$-algebra isomorphisms
            where $C_K^{\infty}(M) =
            C^{\infty}(M) \cap C_K^{0}(M)$ such that
            \begin{equation*}
            \begin{tikzcd}
                C_{K}^{\infty}(M) \ar[r, dashed, "\cong", "i^{*}"'] 
                \ar[d, hookrightarrow] &
                C_{K}^{\infty}\left( \mathbb{R}^{n} \right) 
                \ar[d, hookrightarrow] \\
                C_K^{0}(M) \ar[r, "\cong"] & C_K^{0}(\mathbb{R}^{n})
            \end{tikzcd}
            \end{equation*}
            commutes and such that
            $C^{\infty}(M)$ admits countable
            locally finite sums.
        \end{definition}

        (1) Suppose we are given a smooth
        $n$-manifold as defined in the definition
        above. Thus our data consists 
        of a topological manifold
        $M$ and an $R$-sub-algebra $C^{\infty}(M)
        \subset C^{0}(M)$.\\

        Let $p \in M$ be a point. By assumption, there
        exists a topological embedding $i_p \colon
        \mathbb{R}^{n} \hookrightarrow M$ sending
        $0 \mapsto p$.  For each $p \in M$, let
        $U_p := i_p \left( \mathbb{R}^{n} \right) $ and
        $\varphi_p = i_p^{-1}$. Then
        $\left\{ \left( U_p, \varphi_p \right)  \right\}_{p
        \in M}$ gives an atlas for
        $M$. Now take any two charts
        $\left( U_p, \varphi_p \right) ,
        \left( U_q, \varphi_q \right) $ such that
        $U_p \cap U_q \neq \varnothing$.
        We must show that
        $\varphi_{q} \circ \varphi_{p}^{-1}
        \colon
        \varphi_p \left( U_p \cap U_q \right) 
        \to \varphi_q \left( U_p \cap U_q \right) $ is
        smooth as a function
        between open subsets
        of  $\mathbb{R}^{n}$. Smoothness is a local property, so
        it suffices to check it locally at each point
        $x \in \varphi_p \left( U_p \cap U_q \right) $. Let
        $x$ be such a point. Then we can find an
        open ball $B\left( x, \varepsilon \right) 
        \subset \varphi_p \left( U_p \cap U_q \right) $, hence
        also the compact ball
        $\overline{B \left( x, \frac{\varepsilon}{2} \right) }
        \subset \varphi_p \left( U_p \cap U_q \right) $.
        Let $K = \overline{B\left( x, \frac{\varepsilon}{2}
        \right) }$.
        Let $f \colon
        \mathbb{R}^{n} \to \mathbb{R}$ be
        a smooth bump function with
        support
        in $B \left( x, \frac{\varepsilon}{2} \right) $ and
        which has value $1$ on some small open
        ball around $x$.\\
        So
        $f \in C_K^{\infty}\left( \mathbb{R}^{n} \right) $.
        Then $\left( \varphi_q \right)^*
        (f) \in C_K^{\infty}(M)$, where
        by this expression, we really mean a function
        which agrees with
        $f \circ \varphi_q$ on
        $\varphi_q^{-1}\left( B\left( x, \varepsilon \right) 
        \right) $ and
        has value $0$ outside of
        $\varphi_q^{-1}\left( B\left( x, \varepsilon \right) 
        \right) $. This ensures
        that $\left( \varphi_q \right)^{*}(f)$ really
        is defined on all of $M$.
        
        
        Since $\varphi_p^{-1} =
        i$, we know that if
        $\left( \varphi_q \right)^* \left( f\cdot 
        \pi_j \right) 
        =  (f \cdot \pi_j) \circ \varphi_q \in C_K^{\infty}(M)$
        for all $j$, then
        $\varphi_q \circ \varphi_p^{-1}$ is smooth around
        $x$.
        Now,
        $\pi_j \cdot f$ is a product of two
        functions in
        $C^{\infty}\left( \mathbb{R}^{n} \right) $, and
        since $f$ has support in $K$, the product is
        in $C_{K}^{\infty}\left( \mathbb{R}^{n} \right) $.
        Hence
        $\left( \varphi_q \right)^* 
        \left( \pi_j \cdot  f \right) 
        \in C_{K}^{\infty}(M)$, and thus
        $i_p^{*} \left( \varphi_q \right)^{*}
        \left( \pi_j \cdot f \right) \in C_{K}^{\infty}\left(
        \mathbb{R}^{n} \right) $ and
        agrees with $\varphi_q \circ \varphi_p^{-1}$ in in
        a neighborhood of $x$. Therefore,
        $\varphi_q \circ \varphi_p^{-1}$ is smooth in a
        neighborhood of $x$.
        As $x$ was arbitrary, this shows
        that $\varphi_q \circ \varphi_p^{-1}$ is smooth
        on all of
       $\varphi_p \left( U_p \cap U_q \right) $.
        Thus
        $\left\{ \left( U_p, \varphi_p \right)  \right\}_{
        p \in M}$ gives a smooth atlas which induces
        a smooth structure by taking the maximal atlas.\\
        To see that $C^{\infty}(M)$ corresponds to the
        smooth functions on $M$ in the smooth atlas
        structure, note that a function
        $f \in C^{\infty}(M)$ is locally
        smooth in the usual sense since
        for any compact set  $K$, 
        $C_K^{\infty}(M) \cong C_{K}^{\infty}\left( \mathbb{R}^{n}
        \right) $, where we can make use of this isomorphism by
        composing $f$ with a bump function vanishing
        outside some small
        open set contained in $K$. Now using a partition
        of unity, we can write $f$ as the countable sum
        of smooth functions each with support in
        an open set that is part of a locally finite
        collection covering $M$ - I have
        included a more explicit description
        of this in part (2) of this problem at the end.\\
        \linebreak
        (2) Next, suppose
        we are given a topological manifold
        with a smooth atlas as given through charts
        $i_{\alpha} 
        \colon \mathbb{R}^{n} \to 
        M$. Smoothness of this chart amounts to
        $i_{\beta}^{-1} \circ i_{\alpha}$ being smooth
        on $i_{\alpha}^{-1} 
        \left( i_{\alpha}(\mathbb{R}^{n}) \cap
        i_{\beta}\left( \mathbb{R}^{n} \right) \right) $.
        Let $C^{\infty}(M)$ be the 
        $\mathbb{R}$-algebra generated by
        all the smooth function using the
        smooth atlas definition.
        Now,
        precomposition is $\mathbb{R}$-linear.
        Furthermore, since $i_{\alpha}$ is continuous,
        $i_{\alpha}^{*}$ is indeed a linear map
        $C_{K}^{0}(M) \to C_{K}^{0}(\mathbb{R}^{n})$.
        As $i_{\alpha}$ is a homeomorphism, it has
        a continuous inverse, so
        simply applying
        $(-)^{*}$ to both compositions, we see
        that $i_{\alpha}^{*}$ must be an isomorphism
        (note that here again, 
        $\left( \varphi_q \right)^{*} (f)$ for
        $f \in C_K^{0}(\mathbb{R}^{n})$ is again
        defined by taking a product with a bump function
        and extending it by $0$ to the rest of
        $M$ just as in the first part of this problem).
        Now, if we
        have a smooth map
        $f \colon M \to \mathbb{R}$ with
        support in $K$, then by definition,
        $f \circ i_{\alpha} \colon
        \mathbb{R}^{n} \to \mathbb{R}$ is smooth, so
        $i_{\alpha}^{*}$ indeed restricts
        to a map
        $C_{K}^{\infty}(M) \to C_{K}^{\infty}(\mathbb{R}^{n})$.
        Now, given a map
        $f \in C_K^{\infty}(\mathbb{R}^{n})$, the
        map
        $f \circ i_{\alpha}^{-1}$ which we
        can extend using a bump function to all
        of $M$ is smooth since precomposing it
         with $i_{\alpha}$ gives $f$ as a coordinate
         representation which is smooth by assumption.\\
        So completely equivalently, 
        $\left( i_{\alpha}^{-1} \right)^{*}$ gives
        a map $C_K^{\infty}(\mathbb{R}^{n}) \to 
        C_K^{\infty}(M)$ where we again extend using a
        bump function, and these
        maps are inverses to each other. 

        It remains to show that
        $C^{\infty}(M)$ admits countable locally
        finite sums. But this follows from smooth
        manifolds in our smooth atlas definition admitting
        smooth partitions of unity. Since a
        partition of unity adds up to $1$, any
        smooth function $f$ in our smooth atlas
        definition is the countable sum of
        functions where we weigh $f$ against
        each element of the partition of unity such that
        only finitely many terms are nonzero
        around each point. But around each
        points by construction $f$ is weighted such that
        it has support in a compact set, hence
        this term is smooth in Robert's definition too.
        Now using that $C^{\infty}(M)$ admits countable
        locally finite sums (which this sum 
        using a partition of unity is), we obtain that
        $f$ is in $C^{\infty}(M)$.
        To be more precise, 
        choose for each point $p \in M$ a coordinate ball
        $\left( U_p, \varphi_p \right) $ such that
        $p \in U_p$ and $\varphi_p (U_p) =
        B(p, \varepsilon)$. 
        Let
        $X_p = \varphi_p^{-1}
        \left( B \left( p, \frac{\varepsilon}{3} \right)  \right) $
        which is an open set contained in the compact set
        $K_p = \varphi_p^{-1}\left( 
        \overline{B \left( p, \frac{\varepsilon}{2} \right) }
    \right) $. Then
    $\left\{ X_p \right\}_{p \in M} $ is an open covering
    of $M$, so by second countability, we may take
    a countable subcovering  which we will denote
     $\mathcal{X}$ indexed
     by a set $A$. Now take a
     smooth partition of unity subordinate to
     $\mathcal{X}$ (theorem 2.23 in Lee).
     This gives a sum
     \[
     \sum_{\alpha \in A} \psi_{\alpha}(x) = 1
     \] 
     for all $x \in M$ such that the family
     of supports is locally finite,
     $\supp \psi_{\alpha} \subset 
     X_{\alpha}$ and
     $0 \le \psi_{\alpha}(x) \le 1$ for
     all $\alpha \in A$ and $x \in M$. Now for a function
     $f$ that is smooth on $M$ in the smooth atlas sense,
     $\psi_{\alpha} \cdot f$ is smooth in Robert's
     sense, so
     $f$ being the countable sum of functions which 
     are nonzero only on finitely many 
     $X_{\alpha}$ is thus also smooth, so
     $f \in C^{\infty}(M)$.



    \end{proof}

    \begin{problem}[2 (Some examples)]
        Find a smooth structure on the following topological
        manifolds
        \begin{enumerate}
            \item $S^{n}, n \in \mathbb{N} $ 
            \item $\mathbb{R}\mathbb{P}^{n}, n \in \mathbb{N} $ 
            \item $\mathbb{C}\mathbb{P}^{n}, n \in \mathbb{N} $ 
            \item $\GL_n \left( \mathbb{R} \right) ,n
                \in \mathbb{N} $ 
            \item The Grassmannian
                $\text{Gr}_d \left( \mathbb{R}^{n} \right) ,
                d \le n$.
        \end{enumerate}
    \end{problem}

    \begin{proof}
        \begin{enumerate}
            \item We must find a smooth atlas (i.e., 
                charts such that the transition functions
                are all smooth).
                Choose the standard topological atlas for
                $S^{n}$ : 
                $\left( U_{i}^{\pm},
                \varphi_i^{\pm} \right) $ 
                given by
                \[
                U_i^{\pm} = 
                \left\{ \left( x_1,\ldots,
                x_{n+1} \right) \in S^{n}  \mid 
            \pm x_i > 0 \right\} 
                \] 
                and
                $\varphi_{i}^{\pm} \colon
                U_{i}^{\pm} \to \mathbb{R}^{n}$ is given by
                forgetting the $i$ th coordinate:
                \[
                    \left( x_1, \ldots, x_{i-1},
                    x_{i}, x_{i+1} , \ldots, x_{n+1}\right) 
                    \mapsto 
                    \left( x_1, \ldots, x_{i-1}, x_{i+1},
                    x_{n+1}\right) .
                \] 
                Since each coordinate of
                $\varphi_{i}^{\pm}$ is
                a projection, it is continuous, and
                since it has the inverse
                \[
                    \left( x_1, \ldots, x_{n} \right) 
                    \mapsto 
                    \left( x_1, \ldots, x_{i-1},
                    \sqrt{1 - \sum_j x_j^2} ,
                 x_i, \ldots, x_n \right) 
                \] 
                Essentially, this is just the homeomorphism
                of the hemispheres of
                $S^{n}$ with $D^{n}$. Clearly, each
                point of $S^{n}$ is in some open hemisphere,
                so this collection of charts
                is an atlas. We need to check that
                it's a smooth atlas. 
                All the transition functions will
                be of the form
                \[
                    \left( x_1, \ldots, x_n \right) 
                    \mapsto 
                    \left( x_1, \ldots,
                    x_{i-1}, 
                \sqrt{1 - \sum_j x_j^2} , x_i,
            \ldots, \hat{x_j}, \ldots x_n \right) 
                \] 
                which has smooth coordinate functions, hence
                is smooth.
            \item We check that the standard atlas
                on $\mathbb{R}\mathbb{P}^{n}$ is smooth.
                That is, let
                $U_i \subset \mathbb{R}\mathbb{P}^{n}$ be
                the subset
                 \[
                \left\{ \left[ x_1, \ldots,
                x_{n+1} \right] \colon x_i \neq 0 \right\} 
                \] 
                and
                $\varphi_i \colon
                U_i \to \mathbb{R}^{n}$ given by
                \[
                \varphi_i 
                \left( \left[ x_1, \ldots, x_{n+1} \right]  \right) 
                = \left( \frac{x_1}{x_i}, \ldots,
                \frac{x_{n+1}}{x_i}\right).
                \] 
                We have already previously checked that
                these are charts. Now 
                $U_i \cap U_j$ consists of all
                points whose $i$ th and $j$ th coordinates
                are nonzero. The transition functions will then
                have the form
                \[
                    \left( x_1, \ldots, x_n \right) 
                    \mapsto 
                    \left( \frac{x_1}{x_k}, \ldots
                    , 
                \frac{x_{i-1}}{x_k}, \frac{1}{x_k},
            \frac{x_i}{x_k}, \ldots , 
        \frac{x_n}{x_k}\ldots\right) 
                \] 
                where
                $x_k$ will be $x_j$ if 
                $j<i$ and
                $x_{j+1}$ if $i<j$. In either case,
                this is smooth since
                we are looking at an open domain
                where $x_k$ is nonzero everywhere.
            \item 
                We take the same maps
                and subsets as for $\mathbb{R}\mathbb{P}^{n}$,
                now interpreted over $\mathbb{C}$.

                So we get
                \begin{align*}
                \varphi_j
                \left( \left[ 
                a_1 + b_1 i, \ldots, a_{n+1}+ b_{n+1}i
        \right] \right) 
        &=
        \left( \frac{a_1+i b_1}{a_j + i b_j},\ldots,
        \frac{a_n + i b_n}{a_j + i b_j}\right) 
                        \end{align*}
                Now we also have the homeomorphism
                $\psi \colon \mathbb{C}^{n} \cong \mathbb{R}^{2n}$
                \[
                    \left( z_1, \ldots, z_n \right) 
                    \mapsto 
                    \left( \Re z_1, \Im z_1, \ldots,
                    \Re z_n, \Im z_n\right) .
                \] 
            The transition map will then
            be of the form
            \begin{align*}
                \left( x_1, y_1, x_2, y_2, \ldots,
                x_n, y_n \right) 
                &\mapsto 
                \left( x_1+iy_1, \ldots,
                x_n + iy_n\right) \\
                &\mapsto 
                \left[ x_1+iy_1, \ldots
                , x_{j-1}+ i y_{j-1}, 1,
            x_{j} + i y_{j}, \ldots, x_n + i y_{n}\right] \\
                &\mapsto 
                \left( \frac{x_1 + iy_1}{x_k + iy_k},
                \ldots, \frac{1}{x_k + i y_k}, \ldots,
            \frac{x_n + iy_n}{x_k + i y_k}\right) \\
                &\mapsto 
                \frac{1}{\|z_k\|^2}
                \left( \Re z_{1 k}, \Im
                z_{1 k}, 
            \Re z_{2 k}, \Im z_{2 k}, \ldots,
        \Re z_{n k}, \Im z_{nk}\right) 
            \end{align*}
                Where we let
                $z_k = x_k +i y_k $ and
                \[
                    z_{kj} = \left( x_k + i y_k  \right) 
                    \left( x_j - i y_j \right) 
                    = x_k x_j + y_k y_j +
                    i \left( y_k x_j - x_k y_j \right) .
                \] 
            Then, clearly, each
            $\Re z_{jk}, \Im z_{jk}$ is smooth, so
            the transition map is smooth since
            $\|z_k\|$ is nonzero in the intersection
            as $z_k \neq 0$.
        \item For $\GL_n (\mathbb{R})$, we simply identify
            a matrix
             \[
                 \left( \alpha_{ij} \right)  
                 \mapsto 
             \left( \alpha_{11}, \alpha_{12},
             \ldots, \alpha_{1n},
         \alpha_{21}, \alpha_{22}, \ldots, 
 \alpha_{n(n-1)}, \alpha_{nn} \right) \in 
\mathbb{R}^{n^2}\]
Since the determinant is continuous, we take a chart
about $\left( \alpha_{ij} \right) $ to be
the preimage of some open
neighborhood of its identified point in
$\mathbb{R}^{n^2}$ such that the determinant is
nonzero in this neighborhood. But since
the same chart map is used for every chart
domain, the transition maps will all be identity maps
on some open subsets of
$\mathbb{R}^{n^2}$ which are smooth, hence
this atlas is trivially a smooth atlas.
\item 
    \textbf{I did not finish this in time}
For the Grassmannian, we first define some charts.
Recall that $\text{Gr}_d \left( \mathbb{R}^{n} \right)$ inherits
the quotient topology from
$V_d \left( \mathbb{R}^{n} \right) \to 
\text{Gr}_d \left( \mathbb{R}^{n} \right) $ - or alternatively
as the quotient $\tilde{V_d}(\mathbb{R}^{n}) \to 
\text{Gr}_d \left( \mathbb{R}^{n} \right) $.
Now let $X \in 
\text{Gr}_d\left( \mathbb{R}^{n} \right)$
and define an open neighborhood
of $X$ given by
\[
\mathcal{U}_X = \left\{ P \in \text{Gr}_d(\mathbb{R}^{n})
\colon P \cap X^{\perp} = \left\{ 0 \right\} \right\} .
\] 

Given a $d$-dimensional subspace
$P \in \mathcal{U}_X$, take a point
$\left( x,y \right) \in 
P$ where $x \in X$ and $y \in X^{\perp}$. Then
if there exists a $y' \in X^{\perp}$ such that also
$\left( x,y' \right) \in P$, then
as $P$ is a subspace, $\left( 0, y-y' \right) 
\in P$, but this contradicts trivial intersection with
$X^{\perp}$ unless $y = y'$. 

Choosing a linear isomorphism
$P \cong \mathbb{R}^{d}$, we then obtain a 
well-defined map
$T \colon \mathcal{U}_X \to \Hom \left( \mathbb{R}^{d}
\to \mathbb{R}^{n-d} \right) $ where
$P \in \mathcal{U}_X$ is sent to
the map sending $x \in \mathbb{R}^{d} \mapsto y
\in \mathbb{R}^{n-d}$. In particular, this
shows that
$\mathcal{U}_X \cong \mathbb{R}^{k(n-k)}$.

Let now $j \colon \mathbb{R}^{n} = 
X \oplus X^{\perp} \to X$ be the projection
to the subspace $X$. Fix some orthonormal basis
$\left\{ x_1, \ldots, x_d \right\} $ for
$X$. Given $Y \in \mathcal{U}_X$, 
we have that any $z \in Y$ can be written as
\[
    z = \underbrace{\sum_{i=1}^{k} \left<z , x_i \right> x_i}_{:=x
    \in X}+ 
\underbrace{\left( z - \sum_{i=1}^{k} \left<z, x_i  \right> x_i
\right)}_{:=y \in X^{\perp}}
\] 

        \end{enumerate}
    \end{proof}





    \begin{problem}[3]
        \begin{enumerate}
            \item Let $M$ and $N$ be two smooth
                manifolds, and let
                $f \colon M \to N$ be a smooth
                embedding which is a homeomorphism
                onto its image. Show that
                $f$ is actually a diffeomorphism onto
                its image.
            \item Let $M$ and $N$ be two smooth,
                connected compact manifolds of the
                same dimension. Assume that we
                have an embedding
                $f \colon M \to N$. Show that
                $f$ is a diffeomorphism.
        \end{enumerate}
    \end{problem}

    \begin{proof}
        (1) To start with, by theorem 5.31 in Lee's book, there
        is a unique topological and smooth structure on
        $f(M)$ with respect to which $f$ is a smooth
        embedding, namely the slice charts.

        Let $f \colon M \to N$ be a smooth
        embedding which is a homeomorphism onto
        its image. 
        A smooth embedding is an injective smooth
        immersion. So locally, $f$ can be represented
        as a map $\mathbb{R}^{m} \to 
        \mathbb{R}^{n}$ by
        $x \mapsto (x, 0)$. Choosing the
        restriction of these charts
        to the first $m$ coordinates, we obtain smooth
        charts
        for $f(M)$ which are given by
        $\left( x,0 \right) \mapsto x$ on
        the restriction of the charts on
        $M$ giving this local representation to
        $f(M)$. I.e.,
        if $\left( U, \varphi  \right) $ 
        is a chart for $M$ such that
        $f$ has a local representation with this
        chart as
        $x \mapsto (x,0)$, then we let
        $V = U \cap f(M)$,
        $\hat{V} = 
        \pi \circ \varphi (V)$ where
        $\pi \colon \mathbb{R}^{n} \to 
        \mathbb{R}^{m}$ is the
        projection onto the first $m$ coordinates,
        and
        let $\psi =
        \pi \circ \varphi|_{V} : 
        V \to \hat{V}$. Then
        $\left( V, \psi  \right) $ will
        be the smooth charts for $f(M)$.
        Compatibility
        of these charts is inherited from
        $N$. This gives $f(M)$ the structure of
        an $m$-manifold. Furthermore, with respect to this
        structure, $f$ becomes locally of the form
        $x \mapsto (x,0) \mapsto x$, i.e., the identity.
        So $f \colon
        M \to f(M)$ is a local diffeomorphism. Note
        also that $f$ was assumed to be injective, and
        since we restricted the codomain, it is now also
        surjective. So $f$ is a bijective local diffeomorphism,
        which is a diffeomorphism.

        (2) 


        Compact subsets of a Hausdorff space are
        closed, so since $M$ is compact,
        $f(M)\subset N$ is compact, hence closed.
        However, $f$ is also an embedding, hence
        a homeomorphism onto its image, so as
        $M$ is open, $f(M)$ is open. As
        $N$ is connected and $f(M) \neq \varnothing$, we must
        have $f(M) = N$. Now part (1) establishes the
        result.
    \end{proof}


    \begin{problem}[4 (Bump Functions)]
        Let $M$ be a smooth manifold. Fix $K$ a compact
        subset of $M$, and $U$ an open neighborhood
        of $K$. Show that there exists a bump
        function $b_{K, U} \colon M \to \mathbb{R}$, i.e.,
        a smooth map such that $b_{K,U}$ is
        $1$ on $K$, and $0$ on the complement of
        $U$.
    \end{problem}

    \begin{proof}
        \textbf{I did not finish this one in time}
        We saw in class that in the situation where
        we have
        $K \subset U \subset \mathbb{R}^{n}$, then
        $\varphi_{\varepsilon} * \chi_U$ is a smooth
        function which is
        $1$ on $K \subset U$ and
        $0$ outside of $U$. We will denote
        this function
        $b_{K, U, \mathbb{R}^{n}}$.
        

        Assume that the existence of bump functions has
        already been shown for $\mathbb{R}^{n}$ - i.e., the
        situation above where $M = \mathbb{R}^{n}$.
        Now suppose

    \end{proof}



    \begin{problem}[5]
        \begin{enumerate}
            \item Show that there is no smooth surjective map
                $f \colon \mathbb{R}^{n} \to 
                \mathbb{R}^{m}$ whenever
                $n < m$.
            \item Let $M$ be a connected compact manifold
                of dimension $d$, and fix a smooth
                map $f \colon M \to \mathbb{R}^{d+1}$.
                Show that there is a point
                $p \in \mathbb{R}^{d+1}$ and a
                line in $\mathbb{R}^{d+1}$ through
                $p$ that meets $f(M)$ in finitely
                many points.
        \end{enumerate}
    \end{problem}

    \begin{proof}
        (1) Suppose $f \colon \mathbb{R}^{n} \to 
        \mathbb{R}^{m}$ is a smooth surjective map.
        As $n < m$, we have that
        $Df$ has rank at most $n$ at all points, so
        all points of  $\mathbb{R}^{n}$ are critical values
        of $f$. By Sard's theorem then,
        $f\left( \mathbb{R}^{n} \right) $ has
        measure zero in $\mathbb{R}^{m}$. This in particular implies,
        that no open set can be contained in
        $f\left( \mathbb{R}^{n} \right) $ since any
        open subset of $\mathbb{R}^{m}$ has Lebesgue measure
         greater than $0$. But then $f\left( \mathbb{R}^{n} \right) $ 
         cannot be surjective, giving us a contradiction.\\
         \linebreak
         (2) As formulated, this problem is false:
         for example, we can take the map
         $f \colon (0,1) \to \mathbb{R}^2$ by
         $f(x) = e^{-\frac{1}{x^2}} e^{2 \pi i \frac{1}{x}}$.
         This extends smoothly to
         $f(0) = 0$ and $f(1) = 
         \lim_{x \to 1-}f(x)$. Then connecting
         $f(0)$ and $f(1)$ with a smooth arc, we obtain
         by gluing, a smooth map
         $S^{1} \to \mathbb{R}^2$ which contradicts
         the problem statement.
     \end{proof}



    %\bibliography{../refs.bib}
\end{document}
