\documentclass[reqno]{amsart}
\usepackage{amscd, amssymb, amsmath, amsthm}
\usepackage{graphicx}
\usepackage[colorlinks=true,linkcolor=blue]{hyperref}
\usepackage[utf8]{inputenc}
\usepackage[T1]{fontenc}
\usepackage{textcomp}
\usepackage{babel}
%% for identity function 1:
\usepackage{bbm}
%%For category theory diagrams:
\usepackage{tikz-cd}


\setlength\parindent{0pt}

\pdfsuppresswarningpagegroup=1

\newtheorem{theorem}{Theorem}[section]
\newtheorem{lemma}[theorem]{Lemma}
\newtheorem{proposition}[theorem]{Proposition}
\newtheorem{corollary}[theorem]{Corollary}
\newtheorem{conjecture}[theorem]{Conjecture}

\theoremstyle{definition}
\newtheorem{definition}[theorem]{Definition}
\newtheorem{example}[theorem]{Example}
\newtheorem{exercise}[theorem]{Exercise}
\newtheorem{problem}[theorem]{Problem}
\newtheorem{question}[theorem]{Question}

\theoremstyle{remark}
\newtheorem*{remark}{Remark}
\newtheorem*{note}{Note}
\newtheorem*{solution}{Solution}



%Inequalities
\newcommand{\cycsum}{\sum_{\mathrm{cyc}}}
\newcommand{\symsum}{\sum_{\mathrm{sym}}}
\newcommand{\cycprod}{\prod_{\mathrm{cyc}}}
\newcommand{\symprod}{\prod_{\mathrm{sym}}}

%Linear Algebra

\DeclareMathOperator{\Span}{span}
\DeclareMathOperator{\im}{im}
\DeclareMathOperator{\diag}{diag}
\DeclareMathOperator{\Ker}{Ker}
\DeclareMathOperator{\ob}{ob}
\DeclareMathOperator{\Hom}{Hom}
\DeclareMathOperator{\Mor}{Mor}
\DeclareMathOperator{\sk}{sk}
\DeclareMathOperator{\Vect}{Vect}
\DeclareMathOperator{\Set}{Set}
\DeclareMathOperator{\Group}{Group}
\DeclareMathOperator{\Ring}{Ring}
\DeclareMathOperator{\Ab}{Ab}
\DeclareMathOperator{\Top}{Top}
\DeclareMathOperator{\hTop}{hTop}
\DeclareMathOperator{\Htpy}{Htpy}
\DeclareMathOperator{\Cat}{Cat}
\DeclareMathOperator{\CAT}{CAT}
\DeclareMathOperator{\Cone}{Cone}
\DeclareMathOperator{\dom}{dom}
\DeclareMathOperator{\cod}{cod}
\DeclareMathOperator{\Aut}{Aut}
\DeclareMathOperator{\Mat}{Mat}
\DeclareMathOperator{\Fin}{Fin}
\DeclareMathOperator{\rel}{rel}
\DeclareMathOperator{\Int}{Int}
\DeclareMathOperator{\sgn}{sgn}
\DeclareMathOperator{\Homeo}{Homeo}
\DeclareMathOperator{\SHomeo}{SHomeo}
\DeclareMathOperator{\PSL}{PSL}
\DeclareMathOperator{\Bil}{Bil}
\DeclareMathOperator{\Sym}{Sym}
\DeclareMathOperator{\Skew}{Skew}
\DeclareMathOperator{\Alt}{Alt}
\DeclareMathOperator{\Quad}{Quad}
\DeclareMathOperator{\Sin}{Sin}
\DeclareMathOperator{\Supp}{Supp}
\DeclareMathOperator{\Char}{char}
\DeclareMathOperator{\Teich}{Teich}
\DeclareMathOperator{\GL}{GL}
\DeclareMathOperator{\tr}{tr}
\DeclareMathOperator{\codim}{codim}


%Row operations
\newcommand{\elem}[1]{% elementary operations
\xrightarrow{\substack{#1}}%
}

\newcommand{\lelem}[1]{% elementary operations (left alignment)
\xrightarrow{\begin{subarray}{l}#1\end{subarray}}%
}

%SS
\DeclareMathOperator{\supp}{supp}
\DeclareMathOperator{\Var}{Var}

%NT
\DeclareMathOperator{\ord}{ord}

%Alg
\DeclareMathOperator{\Rad}{Rad}
\DeclareMathOperator{\Jac}{Jac}

%Misc
\newcommand{\SL}{{\mathrm{SL}}}
\newcommand{\mobgp}{{\mathrm{PSL}_2(\mathbb{C})}}
\newcommand{\id}{{\mathrm{id}}}
\newcommand{\MCG}{{\mathrm{MCG}}}
\newcommand{\PMCG}{{\mathrm{PMCG}}}
\newcommand{\SMCG}{{\mathrm{SMCG}}}
\newcommand{\ud}{{\mathrm{d}}}
\newcommand{\Vol}{{\mathrm{Vol}}}
\newcommand{\Area}{{\mathrm{Area}}}
\newcommand{\diam}{{\mathrm{diam}}}
\newcommand{\End}{{\mathrm{End}}}


\newcommand{\reg}{{\mathtt{reg}}}
\newcommand{\geo}{{\mathtt{geo}}}

\newcommand{\tori}{{\mathcal{T}}}
\newcommand{\cpn}{{\mathtt{c}}}
\newcommand{\pat}{{\mathtt{p}}}

\let\Cap\undefined
\newcommand{\Cap}{{\mathcal{C}}ap}
\newcommand{\Push}{{\mathcal{P}}ush}
\newcommand{\Forget}{{\mathcal{F}}orget}


\title{Assignment 2}
\author{Jonas Trepiakas}
\date{}


\begin{document}
\maketitle
    \begin{problem}[1]
        Given a topological manifold $M$ of dimension
        $d \in \mathbb{N} $, we define a smooth
        atlas on $M$ as a collection of charts
        $\left( U_{\alpha}, \varphi_{\alpha} \right)_{\alpha
        \in A}$, where $U_{\alpha} \subset M$ is open
        and $\varphi_{\alpha} \colon U_{\alpha}
        \stackrel{\cong}{\to }\mathbb{R}^{d}$ is a homeomorphism,
        such that the transition maps fit into diagrams
        \begin{equation*}
        \begin{tikzcd}
           &  U_{\alpha}\cap U_{\beta} 
            \ar[dl, "\varphi_{\alpha}"'] \ar[dr, "
            \varphi_{\beta}"]& \\
           \varphi_{\alpha}\left( U_{\alpha}\cap U_{\beta} \right) 
            \ar[rr,"\varphi_{\beta} \varphi_{\alpha}^{-1}"'] &&
            \varphi_{\beta} \left( U_{\alpha} \cap
            U_{\beta} \right) 
        \end{tikzcd}
        \end{equation*}
        where the lower map $\varphi_{\beta}\varphi_{\alpha}^{-1}$ 
        is a smooth map between open subsets
        of $\mathbb{R}^{d}$.
        \begin{enumerate}
            \item (2.5 pts) Show that each smooth manifold (as defined in the
                lecture) admits a smooth atlas.
            \item (2.5 pts) Show that any topological manifold equipped
                with a smooth atlas admits the structure of a
                smooth manifold (as defined in the
                lecture)
        \end{enumerate}
    \end{problem}

    \begin{proof}
        We recall the definition given in the lecture:

        \begin{definition}[]
            For a topological space
            $X$, we let $C_K^{0}(X)$ denote the
            continuous functions on $X$ with support
            contained in $K$. 
        \end{definition}

        \begin{definition}[Smooth manifold]
            A smooth $n$-manifold is a topological $n$-manifold
            $M$ together with an $\mathbb{R}$-sub-algebra
            $C^{\infty}(M) \subset C^{0}(M)$ such that
            for every point $p \in M$, there exists
            a chart
            $i \colon \mathbb{R}^{n} \hookrightarrow 
            M$ sending $0 \mapsto p$ which is an open topological
            embedding, such that for all compact subsets
            $K \subset \mathbb{R}^{n}$, 
            $i^{*} \colon C_K^{\infty}(M)
            \cong C_K^{\infty}\left( \mathbb{R}^{n} \right) $ 
            and
            $i^{*} \colon
            C_K^{0}(M) \to C_K^{0}\left( \mathbb{R}^{n} \right) $
            are $\mathbb{R}$-algebra isomorphisms
            where $C_K^{\infty}(M) =
            C^{\infty}(M) \cap C_K^{0}(M)$ such that
            \begin{equation*}
            \begin{tikzcd}
                C_{K}^{\infty}(M) \ar[r, dashed, "\cong", "i^{*}"'] 
                \ar[d, hookrightarrow] &
                C_{K}^{\infty}\left( \mathbb{R}^{n} \right) 
                \ar[d, hookrightarrow] \\
                C_K^{0}(M) \ar[r, "\cong"] & C_K^{0}(\mathbb{R}^{n})
            \end{tikzcd}
            \end{equation*}
            commutes and such that
            $C^{\infty}(M)$ admits countable
            locally finite sums.
        \end{definition}

        (1) Suppose we are given a smooth
        $n$-manifold as defined in the definition
        above. Thus our data consists 
        of a topological manifold
        $M$ and an $R$-sub-algebra $C^{\infty}(M)
        \subset C^{0}(M)$.\\

        Let $p \in M$ be a point. By assumption, there
        exists a topological embedding $i_p \colon
        \mathbb{R}^{n} \hookrightarrow M$ sending
        $0 \mapsto p$.  For each $p \in M$, let
        $U_p := i_p \left( \mathbb{R}^{n} \right) $ and
        $\varphi_p = i_p^{-1}$. Then
        $\left\{ \left( U_p, \varphi_p \right)  \right\}_{p
        \in M}$ gives an atlas for
        $M$. Now take any two charts
        $\left( U_p, \varphi_p \right) ,
        \left( U_q, \varphi_q \right) $ such that
        $U_p \cap U_q \neq \varnothing$.
        We must show that
        $\varphi_{q} \circ \varphi_{p}^{-1}
        \colon
        \varphi_p \left( U_p \cap U_q \right) 
        \to \varphi_q \left( U_p \cap U_q \right) $ is
        smooth as a function
        between open subsets
        of  $\mathbb{R}^{n}$. Smoothness is a local property, so
        it suffices to check it locally at each point
        $x \in \varphi_p \left( U_p \cap U_q \right) $. Let
        $x$ be such a point. Then we can find an
        open ball $B\left( x, \varepsilon \right) 
        \subset \varphi_p \left( U_p \cap U_q \right) $, hence
        also the compact ball
        $\overline{B \left( x, \frac{\varepsilon}{2} \right) }
        \subset \varphi_p \left( U_p \cap U_q \right) $.
        Let $K = \overline{B\left( x, \frac{\varepsilon}{2}
        \right) }$.
        Let $f \colon
        \mathbb{R}^{n} \to \mathbb{R}$ be
        a smooth bump function with
        support
        in $B \left( x, \frac{\varepsilon}{2} \right) $ and
        which has value $1$ on some small open
        ball around $x$.\\
        So
        $f \in C_K^{\infty}\left( \mathbb{R}^{n} \right) $.
        Then $\left( \varphi_q \right)^*
        (f) \in C_K^{\infty}(M)$.
        
        
        Since $\varphi_p^{-1} =
        i$, we know that if
        $\left( \varphi_q \right)^* \left( f\cdot 
        \pi_j \right) 
        =  (f \cdot \pi_j) \circ \varphi_q \in C_K^{\infty}(M)$
        for all $j$, then
        $\varphi_q \circ \varphi_p^{-1}$ is smooth around
        $x$.
        Now,
        $\pi_j \cdot f$ is a product of two
        functions in
        $C^{\infty}\left( \mathbb{R}^{n} \right) $, and
        since $f$ has support in $K$, the product is
        in $C_{K}^{\infty}\left( \mathbb{R}^{n} \right) $.
        Hence
        $\left( \varphi_q \right)^* 
        \left( \pi_j \cdot  f \right) 
        \in C_{K}^{\infty}(M)$, and thus
        $i_p^{*} \left( \varphi_q \right)^{*}
        \left( \pi_j \cdot f \right) \in C_{K}^{\infty}\left(
        \mathbb{R}^{n} \right) $ and
        agrees with $\varphi_q \circ \varphi_p^{-1}$ in in
        a neighborhood of $x$. Therefore,
        $\varphi_q \circ \varphi_p^{-1}$ is smooth in a
        neighborhood of $x$.
        As $x$ was arbitrary, this shows
        that $\varphi_q \circ \varphi_p^{-1}$ is smooth
        on all of
       $\varphi_p \left( U_p \cap U_q \right) $.
        Thus
        $\left\{ \left( U_p, \varphi_p \right)  \right\}_{
        p \in M}$ gives a smooth atlas which induces
        a smooth structure by taking the maximal atlas.



    \end{proof}





    \begin{problem}[3]
        \begin{enumerate}
            \item Let $M$ and $N$ be two smooth
                manifolds, and let
                $f \colon M \to N$ be a smooth
                embedding which is a homeomorphism
                onto its image. Show that
                $f$ is actually a diffeomorphism onto
                its image.
            \item Let $M$ and $N$ be two smooth,
                connected compact manifolds of the
                same dimension. Assume that we
                have an embedding
                $f \colon M \to N$. Show that
                $f$ is a diffeomorphism.
        \end{enumerate}
    \end{problem}

    \begin{proof}
        (1) A smooth embedding is an
        injective smooth immersion. By the rank theorem,
        this is the same as an injective smooth map
        whose differential is injective. 

        Note that a bijective local diffeomorphism
        is a diffeomorphism, so it suffices
        to show that $f$ is a local diffeomorphism.

        For this, note that
        since $f$ is assumed to be a homeomorphism onto
        its image, its image is also an
        $m$-submanifold, hence the tangent spaces have the same
        dimension, so as the differential is injective,
        it is an isomorphism. But
        since the differential is an isomorphism, it
        in particular has non-vanishing determinant, so
        by the inverse function theorem, there
        exists some small neighborhood
        of every point in $M$ which is mapped diffeomorphically
        into some neighborhood in
        $f(M)$, and as $f(M)$ is open in $N$, the image
        of the neighborhood is also open.
        Thus $f$ is a local diffeomorphism.\\
        \linebreak
        (2) 


        Compact subsets of a Hausdorff space are
        closed, so since $M$ is compact,
        $f(M)\subset N$ is compact, hence closed.
        However, $f$ is also an embedding, hence
        a homeomorphism onto its image, so as
        $M$ is open, $f(M)$ is open. As
        $N$ is connected and $f(M) \neq \varnothing$, we must
        have $f(M) = N$. Now part (1) establishes the
        result.
    \end{proof}


    \begin{problem}[5]
        \begin{enumerate}
            \item Show that there is no smooth surjective map
                $f \colon \mathbb{R}^{n} \to 
                \mathbb{R}^{m}$ whenever
                $n < m$.
            \item Let $M$ be a connected compact manifold
                of dimension $d$, and fix a smooth
                map $f \colon M \to \mathbb{R}^{d+1}$.
                Show that there is a point
                $p \in \mathbb{R}^{d+1}$ and a
                line in $\mathbb{R}^{d+1}$ through
                $p$ that meets $f(M)$ in finitely
                many points.
        \end{enumerate}
    \end{problem}

    \begin{proof}
        (1) Suppose $f \colon \mathbb{R}^{n} \to 
        \mathbb{R}^{m}$ is a smooth surjective map.
        As $n < m$, we have that
        $Df$ has rank at most $n$ at all points, so
        all points of  $\mathbb{R}^{n}$ are critical values
        of $f$. By Sard's theorem then,
        $f\left( \mathbb{R}^{n} \right) $ has
        measure zero in $\mathbb{R}^{m}$. This in particular implies,
        that no open set can be contained in
        $f\left( \mathbb{R}^{n} \right) $ since any
        open subset of $\mathbb{R}^{m}$ has Lebesgue measure
         greater than $0$. But then $f\left( \mathbb{R}^{n} \right) $ 
         cannot be surjective, giving us a contradiction.\\
         \linebreak
         (2) As before, since
         $f$ is smooth and $M$ is of dimension $d$, all
         points of $M$ are critical points of
         $f$, so $f(M)$ has measure zero in 
         $\mathbb{R}^{d+1}$. But furthermore,
         $M$ is compact and connected, so
         $f(M)$ is a compact connected subset of
         $\mathbb{R}^{d+1}$. That is,
         there exists $K > 0$ such that
         $f(M)$ is a closed connected measure zero subset
         of $\overline{B \left( 0,K \right) }$.
     \end{proof}



    %\bibliography{../refs.bib}
\end{document}
