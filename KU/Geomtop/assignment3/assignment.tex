\documentclass[reqno]{amsart}
\usepackage{amscd, amssymb, amsmath, amsthm}
\usepackage{graphicx}
\usepackage[colorlinks=true,linkcolor=blue]{hyperref}
\usepackage[utf8]{inputenc}
\usepackage[T1]{fontenc}
\usepackage{textcomp}
\usepackage{babel}
%% for identity function 1:
\usepackage{bbm}
%%For category theory diagrams:
\usepackage{tikz-cd}


\setlength\parindent{0pt}

\pdfsuppresswarningpagegroup=1

\newtheorem{theorem}{Theorem}[section]
\newtheorem{lemma}[theorem]{Lemma}
\newtheorem{proposition}[theorem]{Proposition}
\newtheorem{corollary}[theorem]{Corollary}
\newtheorem{conjecture}[theorem]{Conjecture}

\theoremstyle{definition}
\newtheorem{definition}[theorem]{Definition}
\newtheorem{example}[theorem]{Example}
\newtheorem{exercise}[theorem]{Exercise}
\newtheorem{problem}[theorem]{Problem}
\newtheorem{question}[theorem]{Question}

\theoremstyle{remark}
\newtheorem*{remark}{Remark}
\newtheorem*{note}{Note}
\newtheorem*{solution}{Solution}



%Inequalities
\newcommand{\cycsum}{\sum_{\mathrm{cyc}}}
\newcommand{\symsum}{\sum_{\mathrm{sym}}}
\newcommand{\cycprod}{\prod_{\mathrm{cyc}}}
\newcommand{\symprod}{\prod_{\mathrm{sym}}}

%Linear Algebra

\DeclareMathOperator{\Span}{span}
\DeclareMathOperator{\im}{im}
\DeclareMathOperator{\diag}{diag}
\DeclareMathOperator{\Ker}{Ker}
\DeclareMathOperator{\ob}{ob}
\DeclareMathOperator{\Hom}{Hom}
\DeclareMathOperator{\Mor}{Mor}
\DeclareMathOperator{\sk}{sk}
\DeclareMathOperator{\Vect}{Vect}
\DeclareMathOperator{\Set}{Set}
\DeclareMathOperator{\Group}{Group}
\DeclareMathOperator{\Ring}{Ring}
\DeclareMathOperator{\Ab}{Ab}
\DeclareMathOperator{\Top}{Top}
\DeclareMathOperator{\hTop}{hTop}
\DeclareMathOperator{\Htpy}{Htpy}
\DeclareMathOperator{\Cat}{Cat}
\DeclareMathOperator{\CAT}{CAT}
\DeclareMathOperator{\Cone}{Cone}
\DeclareMathOperator{\dom}{dom}
\DeclareMathOperator{\cod}{cod}
\DeclareMathOperator{\Aut}{Aut}
\DeclareMathOperator{\Mat}{Mat}
\DeclareMathOperator{\Fin}{Fin}
\DeclareMathOperator{\rel}{rel}
\DeclareMathOperator{\Int}{Int}
\DeclareMathOperator{\sgn}{sgn}
\DeclareMathOperator{\Homeo}{Homeo}
\DeclareMathOperator{\SHomeo}{SHomeo}
\DeclareMathOperator{\PSL}{PSL}
\DeclareMathOperator{\Bil}{Bil}
\DeclareMathOperator{\Sym}{Sym}
\DeclareMathOperator{\Skew}{Skew}
\DeclareMathOperator{\Alt}{Alt}
\DeclareMathOperator{\Quad}{Quad}
\DeclareMathOperator{\Sin}{Sin}
\DeclareMathOperator{\Supp}{Supp}
\DeclareMathOperator{\Char}{char}
\DeclareMathOperator{\Teich}{Teich}
\DeclareMathOperator{\GL}{GL}
\DeclareMathOperator{\tr}{tr}
\DeclareMathOperator{\codim}{codim}


%Row operations
\newcommand{\elem}[1]{% elementary operations
\xrightarrow{\substack{#1}}%
}

\newcommand{\lelem}[1]{% elementary operations (left alignment)
\xrightarrow{\begin{subarray}{l}#1\end{subarray}}%
}

%SS
\DeclareMathOperator{\supp}{supp}
\DeclareMathOperator{\Var}{Var}

%NT
\DeclareMathOperator{\ord}{ord}

%Alg
\DeclareMathOperator{\Rad}{Rad}
\DeclareMathOperator{\Jac}{Jac}

%Misc
\newcommand{\SL}{{\mathrm{SL}}}
\newcommand{\mobgp}{{\mathrm{PSL}_2(\mathbb{C})}}
\newcommand{\id}{{\mathrm{id}}}
\newcommand{\MCG}{{\mathrm{MCG}}}
\newcommand{\PMCG}{{\mathrm{PMCG}}}
\newcommand{\SMCG}{{\mathrm{SMCG}}}
\newcommand{\Diff}{{\mathrm{Diff}}}
\newcommand{\ud}{{\mathrm{d}}}
\newcommand{\Vol}{{\mathrm{Vol}}}
\newcommand{\Area}{{\mathrm{Area}}}
\newcommand{\diam}{{\mathrm{diam}}}
\newcommand{\End}{{\mathrm{End}}}


\newcommand{\reg}{{\mathtt{reg}}}
\newcommand{\geo}{{\mathtt{geo}}}

\newcommand{\tori}{{\mathcal{T}}}
\newcommand{\cpn}{{\mathtt{c}}}
\newcommand{\pat}{{\mathtt{p}}}

\let\Cap\undefined
\newcommand{\Cap}{{\mathcal{C}}ap}
\newcommand{\Push}{{\mathcal{P}}ush}
\newcommand{\Forget}{{\mathcal{F}}orget}


\title{Assignment 3}
\author{Jonas Trepiakas}
\date{}



\begin{document}
\maketitle
    \begin{problem}[1 (Quotient manifolds)]
        Let $M$ be a smooth manifold, and let 
        $\tau \colon M \to M$ be a fixed-point free
        involution on $M$, i.e., a map
        $\tau \colon M \to M$ such that
        $\tau \circ \tau = \id$ and $\tau (x) \neq x$ 
        for all $x \in M$.
        \begin{enumerate}
            \item Show that $\tau$ is a diffeomorphism, and
                that the quotient
                $M / \tau$ is a topological manifold.
            \item Show that $M / \tau$ admits a unique smooth
                structure so that the
                quotient map $q \colon M \to 
                M / \tau$ is a local diffeomorphism.
            \item Give an example of a fixed-point free involution
                on a smooth manifold, and describe its quotient.
        \end{enumerate}
    \end{problem}

    \begin{proof}
        (1) We assume $\tau$ is meant to be smooth here.
        Then we immediately find that
         $\tau$ is a diffeomorphism since
         $\tau$ has itself as an inverse as
         $\tau \circ \tau = \id$.
         To see that $M / \tau$ is a topological manifold,
         we can simply using Hausdorffness of $M$ find
         $\tau$-invariant neighborhoods
         $U$ and $V$ of
         $\tau^{-1}(\overline{x})$ and
         $\tau^{-1}\left( \overline{y} \right) $ for
         arbitrary distinct points $\overline{x},\overline{y}
         \in M / \tau$. Since these open neighborhoods
         are saturated with respect to $\tau$,
         their images form open disjoint neighborhoods
         of $\overline{x}$ and $\overline{y}$, so
         $M / \tau$ is Hausdorff.
         Second countability follows if we can show that
         $\tau$ is an open map. 
         But for an open subset $U$ in $M$ and
         a point $x \in U$, we can find a neighborhood
         $U_x$ contained in
         $U$ around $x$ such that $\tau (U_x) \cap U_x = \varnothing$ 
         simply by taking two disjoint neighborhoods of
         $x$ and $\tau(x)$ (the one for
         $x$ contained in $U$ ) and intersecting
         the image under $\tau$ of the neighborhood of $x$ with
         the neighborhood of $\tau$ and then transferring it
         back to $x$ also. Thus the image
         of the neighborhood of $x$ is open in
         $M / \tau$. But this shows that an arbitrary open
         set is the union of sets whose image is
         open under $\tau$, so
         $\tau$ is an open map. 
         Lastly, $M / \tau$ can be seen to be
         locally Euclidean as follows. 
         Restricting $\tau$ to an open set
         on which it is injective gives an embedding,
         so choosing a point $\overline{x} \in M / \tau$ and
         a point $x \in \tau^{-1}\left( \overline{x} \right) $ 
         and a neighborhood as constructed above
         such that $\tau$ is indeed injective
         on this neighborhoods, we can intersect this
         neighborhood with a chart around $x$ and compose
         $\tau^{-1}$ with the chart to get a chart around
         $\overline{x}$.\\
         \linebreak
         (2) We take the same charts as we constructed above.
         We need to check smoothness. 
         Suppose
         $\left( U, \varphi \circ \tau^{-1} \right) $ 
         and $\left( V, \psi \circ \tau^{-1} \right) $ 
         are charts in $M / \tau$ with
         $U \cap V \neq \varnothing$.
         Then we get
         the transition map to be
          \[
         \psi \circ \tau^{-1} \circ \left( \varphi \circ
         \tau^{-1} \right)^{-1}
         = \psi  \circ \tau^{-1} \circ \tau \circ
         \varphi^{-1}
         = \psi \circ \varphi^{-1}
         \] 
         on some open subset of $\mathbb{R}^{n}$ which
         is assumed to be smooth by assumption. Hence
         this atlas indeed induces a smooth structure. 
         In this structure, choose a chart
          $\left( U, \varphi  \right) $ 
          about $x \in M$ such that
          $q|_{U}$ is a homeomorphism onto its image
          where $q \colon M \to M / \tau$ is the quotient map.
          Then $q$ on this open set has a coordinate
          representation given by
          \[
          \left( \varphi 
          \circ q^{-1} \right)  \circ q \circ \varphi^{-1}
          = \id
          \] 
          which is a diffeomorphism on any subset of
          $\mathbb{R}^{n}$, so
          $q$ is a diffeomorphism on $U$, and as
          $x$ was arbitrary, $q$ is a local
          diffeomorphism.\\
          \linebreak
          (3) Here is a nice one: let
          $\iota \colon S_{g,1} \to S_{g,1}$ be the
          hyperelliptic involution of the surface with
          genus $g$ and one boundary component. Then
          $S_{g,1} / \iota \cong
          D^2$.\\
          \linebreak
          But for the sake of the problem, 
          we could take $S^{n}$ with the
          antipodal action $\mathbb{Z}/2$ on
          $S^{n}$ giving $\mathbb{R}\mathbb{P}^{n}$ as
          the quotient. It is a fixed-point free
          involution, and we have checked that
          $S^{n}$ is a smooth manifold and that
          its quotient $\mathbb{R}\mathbb{P}^{n}$ is
          also smooth.

     \end{proof}


    \begin{problem}[2 (Covering spaces and smooth liftings)]
        Let $M$ and $N$ be smooth manifolds, and
        $f, g \colon M \to N$ two smooth maps.
        We say $f$ and $g$ are smoothly homotopic if there exists
        a smooth map
        $H \colon M \times \mathbb{R} \to N$ such that
        $H(x,t) = f(x)$ for all
        $t \in (-\infty, 0]$ and
        $H (x,t) = g(x)$ for all
        $t \in [1, \infty)$.
        \begin{enumerate}
            \item If $M$ is a smooth, compact connected
                manifold, and $p \colon
                M' \to M$ is a covering map of
                topological spaces, show that
                $M'$ is a topological manifold which
                admits a unique smooth structure so that
                $p \colon M' \to M$ is a local
                diffeomorphism.
            \item Fix $M$ a smooth, compact connected
                manifold and $p \colon M' \to M$ a
                covering. Let $f \colon N \to M$ be
                a smooth map, where we assume
                $N$ is a connected smooth manifold. 
                Suppose we have points
                $x \in N$ and
                $y \in p^{-1}\left( f(x) \right) $ such that
                the induced maps
                \[
                f_* \colon \pi_1(N,x) \to \pi_1(M,f(x))
                \] 
                and
                \[
                p_* \colon \pi_1 (M',y)
                \to \pi_1(M, f(x))
                \] 
                is such that
                $\im f_* \subset \im p_*$. Show that there
                exists a unique smooth map
                $f' \colon N \to M'$ such that
                $f'(x) = y$ and
                $p \circ f' = f$.
            \item Show that any smooth map
                $S^{2} \to S^{1} \times S^{1}$ admits
                a smooth nullhomotopy.
        \end{enumerate}
    \end{problem}

    \begin{proof}
        (1) \textbf{I will add the assumption that $M'$ is
        connected as otherwise second-countability can fail}\\

        Suppose $\dim M = m$. First, we show that
        $M'$ is a topological manifold.\\
        \textit{Locally Euclidean:} 
        Let $x \in M'$. Then there exists a chart
        $\left( U, \varphi  \right) $ around
        $p(x)$ in $M$ which we
        can assume to be connected. Now, since  $p$ is a covering map,
        $p(x)$ has an evenly covered neighborhood
        $V$, so in particular, letting
        $W$ be the component of
        $p^{-1}(V)$ which contains $p$, 
        $p|_{W \cap p^{-1}(U)} \colon
        W \cap p^{-1}(U) \to U \cap V$ is a homeomorphism of
        open sets. Composing this with the chart homeomorphism,
        we get a homeomorphism of
        $W \cap p^{-1}(U)$ to an open subset of
        $\mathbb{R}^{m}$.\\
        \linebreak
        \textit{Hausdorffness:} Let
        $x,y \in M'$. There are two cases to consider:
        $p(x) = p(y)$ and $p(x) \neq p(y)$.
        Suppose first that $p(x) \neq p(y)$.
        Then take two disjoint neighborhoods of
        $p(x)$ and $p(y)$.
        Then taking preimages, we get two disjoint open sets
        in $M'$ containing $x$ and $y$, respectively.\\
        \linebreak
        \textit{Second-countability:} Since
        $M'$ is connected and locally-Euclidean hence
        locally path-connected, $M'$ is path-connected.
        By theorem 2.3.9 in AlgTop1, the monodromy action
        $\pi_1(M)$ on any fiber is transitive, so
        the fiber of the covering has cardinality
        $\le \left| \pi_1(M) \right| $, and since
        the fundamental group of a manifold is countable, we
        obtain that the covering space has countably
        many sheets. Since
        $M$ is second-countable, it has a countable
        basis for its topology. Now
        pulling back each of these open sets and taking components,
        we obtain a countable union of a countable collection
        of open sets which is thus countable. Lastly,
        this collection is a basis for $M'$ since for
        any point $x \in M'$ and any open neighborhood
        $U$ of $x$, we can find some
        local neighborhood $V$ of $x$ homeomorphic to an open
        neighborhood of $p(x)$ downstairs. 
        Taking a basis element $W$ contained in
        $p\left( U \cap V \right) $ containing $p(x)$, 
        $p^{-1}(W)$ will be an open neighborhood of
        $x$ contained in $U \cap V \subset U$.\\
        \linebreak
        \textit{Smooth structure}
        Now we must show that $M'$ admits a unique smooth
        structure such that
        $p \colon M' \to M$ is a local diffeomorphism.
        For this, it suffices to show that
        $p \colon M' \to M$ forces $M'$ to have
        a smooth atlas compatible with its smooth structure.
        In particular, suppose
        $x \in M'$ and choose some local neighborhood
        $U$ of $x$
        using that $p$ is a covering map such that
        $p$ is a homeomorphism of $U$ onto its image
        $p(U) \subset M$. Now take a smooth chart
        $\left( V, \varphi  \right) $ in $M$ around
        $p(x)$. We may assume that $V \subset 
        \varphi (U)$ as otherwise we can just intersect
        $V$ with $\varphi (U)$ and take $\varphi $ to
        be the restriction onto this subspace. Then
        we let
        the composite
        $U \cap p^{-1}(V) \stackrel{p }{\to } V
        \stackrel{\varphi }{\to } \varphi (V)$ be a smooth chart
        for $x$ in $M'$. 
        To see that these are compatible, suppose
        $\left( V, \varphi  \circ p \right) $ and
        $\left( W, \psi \circ p \right) $ are
        two charts with $V \cap W \neq \varnothing$.
        Then recall that
        $p \colon V \cap W \to 
        p\left( V \cap W \right) $ is a homeomorphism, wo
        $\varphi \circ \psi^{-1} =
        \varphi \circ p \circ p^{-1} \circ \psi^{-1} 
        = \left( \varphi  \circ p \right) 
        \circ \left( \psi  \circ p \right)^{-1} \colon
        \psi  \circ p \left( V \cap W \right)
        \to \varphi \circ p \left( V \cap W \right) $
        is smooth by assumption of
        $\varphi $ and $\psi $ being smooth chart maps
        for $M$. Thus
        this collection gives a smooth atlas for
        $M'$ hence a unique smooth structure by taking the
        maximal compatible smooth atlas. Furthermore, smooth
        charts
        are diffeomorphisms onto their images, so
        for a chart
        $\left( V, \varphi  \circ p \right) $, we have
        that since $\varphi  \circ p$ is a diffeomorphism on
        $V$ and $\varphi $ is a diffeomorphism on
        $p(V)$, the composite
        $V \stackrel{\varphi \circ p}{\to }
        \varphi \circ p(V) \stackrel{\varphi^{-1}}{\to }
        p(V)$ is a diffeomorphism on $V$. Hence
        $p$ is a diffeomorphism on $V$. As these charts
        for an atlas, we conclude that $p$ is a local diffeomorphism
        in this smooth structure on $M'$.\\




        To see that this structure is the unique
        one making $p$ a local diffeomorphism, suppose
        $\mathcal{A} = \left\{ 
        \left( W_{\alpha}, \psi_{\alpha} \right) \right\} 
        $ is another smooth structure on
        $M'$ making $p$ a local diffeomorphism.
        Let $\left( V, \varphi  \circ p \right) $ be
        a smooth chart from the smooth structure constructed
        previously. 

        Without loss of generality,
        let $W_{\alpha} $ be a smooth chart
        such that $p \colon W_{\alpha} \to 
        p\left( W_{\alpha} \right)$ is a diffeomorphism.
        Then $\psi_{\alpha} \circ p^{-1} \colon
        p\left( W_{\alpha} \right) \to 
        \mathbb{R}^{m}$ is a diffeomorphism of
        the open set $p\left( W_{\alpha} \right) 
        \subset M$ onto an open subset of $\mathbb{R}^{m}$.
        Suppose $W_{\alpha} \cap V \neq  \varnothing$  where
        $\left( V, \varphi \circ p \right) $ is a smooth chart
        from the smooth structure constructed before.
        Then 
        $\left( \varphi \circ p \right) \circ
        p^{-1} \colon p\left( W_{\alpha} \cap V \right) 
        \to \mathbb{R}^{m}$  is by construction a diffeomorphism
        as $\varphi $ is a smooth chart map. Hence
        these two charts on $M$ are compatible, so
        $\varphi \circ p \circ \psi_{\alpha}^{-1}
        = \left( \varphi \circ p \right) \circ
        p^{-1} \circ p \circ \psi_{\alpha}^{-1} \colon
        \psi_{\alpha} \circ p^{-1} \left( 
        W_{\alpha} \cap V\right) \to 
        \left( \varphi \circ p \right) \circ p^{-1}
        \left( W_{\alpha} \cap V \right) $ is a diffeomorphism,
        thus making the two charts compatible.
        Hence the two structures coincide.\\\qed 
        \linebreak
        (2) Since $N$ is a connected manifold, it is
        locally path-connected hence path-connected. 
        Theorem 2.7.2 now gives that
        because $\im f_* \subset \im p_*$, a unique
        topological
        lift $f' \colon
        N \to M'$ exists making the following diagram commute
        \begin{equation*}
        \begin{tikzcd}
            & (M', y) \ar[d, "p"] \\
            (N,x) \ar[r, "f"'] \ar[ru, " \exists ! f'"] & (M,f(x)).
        \end{tikzcd}
        \end{equation*}
        Hence we must show that this map
        is smooth when we equip $M'$ with
        the unique smooth structure from
        the previous exercise making $p$ a local
        diffeomorphism.\\
        To see this,
        take a chart $\left( V, \varphi \circ p \right) $ 
        such that $p$ is a diffeomorphism
        of $V$ onto its image. Take
        also a chart
        $\left( U, \psi  \right) $ in
        $N$. Then the coordinate representation of
        $f'$ becomes
        $\varphi \circ p \circ f' \circ \psi^{-1}
        = \varphi \circ f \circ \psi^{-1}$ which is
        a coordinate representation of
        $f$. Since $f$ is assumed to be smooth,
        this coordinate representation is
        smooth, hence 
        $p \circ f'$ is smooth since the charts
        chosen in the atlases were arbitrary.\\
        \linebreak
        (3) We have that $S^{1} \times S^{1} 
        \cong T^2$ is the torus.
        Now, this is a manifold being the product
        of two manifolds and it is compact being
        the finite product of two compact spaces. 
        Furthermore, it is connected being the
        product of two connected spaces. We can
        apply the first two subproblems 
        to the usual topological covering
        space
        $p \colon \mathbb{R}^2 \to T^2$ which
        obtains $T^2$ as the quotient space
        $\mathbb{R}^2 / \mathbb{Z}^2$ under the action
        of $\mathbb{Z}^2$ on the plane by
        translations. 
        Since $S^2$ is simply connected, we
        trivially have that for any map
        $f \colon S^2 \to S^{1} \times S^{1}$,
         $\im f_* = \left\{ 0 \right\} \subset 
         \im p_* \subset \pi_1 \left( M, f(x) \right)$.
         Thus a smooth lift
         $f' \colon
         S^{2} \to S^{1} \times S^{1}$ exists such that


        \begin{equation*}
        \begin{tikzcd}
            & \mathbb{R}^2 \ar[d, "p"] \\
            S^{2} \ar[r, "f"'] \ar[ru, "f'"] & S^{1}\times 
            S^{1}
        \end{tikzcd}
        \end{equation*}
        commutes.
        But $\mathbb{R}^2$ is contractible, and this
        contraction can be done smoothly:
        namely, let $\varphi \colon
        \mathbb{R}^2 \times I \to \mathbb{R}^2$ be
        given by
        $\varphi (x,t) = x(1-t)$. This is clearly smooth
        and thus a smooth homotopy from
        $\id$ to $c_{0}$, the constant map at $0$.
        This, in return, gives us a smooth homotopy
        $H(x,t) = p \circ \varphi (f'(x),t)$
        from
        $p \circ f' = f$ at $t = 0$ to
        $c_{p(0)}$ at $t = 1$.

    \end{proof}

    \begin{problem}[3 (Mapping class groups of some manifolds)]
        For $M$ a smooth manifold, we define the 
        \textit{mapping class group} of $M$ by
        \[
        \pi_0 \Diff (M) := \Diff (M) / \sim
        \] 
        as the group of self-diffeomorphisms
        up to isotopy, where an isotopy of two
        diffeomorphisms $f,g$ is a map
        $H \colon M \times I \to M$ such that
        $H(x,0) = f(x)$ and
        $H(x,1) = g(x)$ and
        $H(-,t)$ is a self-diffeomorphism of
        $M$ for all $t \in I$.
        \begin{enumerate}
            \item Show that
                $\pi_0 \Diff (\mathbb{R}) 
                \cong \mathbb{Z}/2$.
            \item Compute
                $\pi_0 \Diff \left( S^{1} \right) $.
        \end{enumerate}
    \end{problem}

    \begin{proof}
        (1) I will start by showing that any
        two orientation-preserving diffeomorphisms
        of $\mathbb{R}$ are isotopic.
        For this, it suffices to show that
        any orientation-preserving diffeomorphism
        is isotopic to the identity.
        Since  $\mathbb{R} \cong (-1,1)$, any 
        diffeomorphism 
        $f$ of $\mathbb{R}$ can be considered as
        a diffeomorphism of $\left( -1,1 \right) $. 
        Now, we claim that
        $\lim_{x \to \infty}f(x) = \infty$ and
        $\lim_{x \to -\infty} f(x) = -\infty$.
        As $f$ is orientation-preserving, we
        have  $\frac{df}{dx}(0) > 0$.
        In particular, there exists a neighborhood 
        around $0$ in which $\frac{df}{dx}$ is positive. 
        By the mean value theorem, on this open neighborhood,
        $f$ is strictly increasing. Since we can successively cover
        $\mathbb{R} $ by overlapping intervals where
        the orientations must agree since
        we choose it to be a continuous section of
        $\Lambda^{1} T\mathbb{R}$, we must have
        that $f$ is strictly increasing everywhere. Hence
        $\lim_{x \to \pm \infty}f(x) = \pm \infty$.

        Now the map
        $\varphi \colon \mathbb{R} \times I \to \mathbb{R}$ 
        given by
        $\varphi (x,t) = x(1-t) + f (x) t$ is in fact
        a diffeomorphism at
        each time $t$ since $f$ is strictly increasing.

        Now to see that
        $\pi_0 \Diff \left( \mathbb{R} \right) 
        \cong \mathbb{Z}/2$, if $f$ and $g$ are orientation 
        reversing, then
        $f^{-1} g$ is orientation-preserving, so
        $f^{-1} g \simeq \id$ implies
        $g \simeq f$.\\
        \linebreak
        (2) 
        
        Let $p \colon \mathbb{R}
        \to S^{1}$ be the usual parametrization
        $t \mapsto e^{2 \pi i t}$. Let
        $f \colon S^{1} \to S^{1}$ be a diffeomorphism.
        By composing $f$ with the isotopy
        $H(x,t) = x e^{- i \arg f((1,0)) t}$, we
        may assume that $f$ fixes $(1,0)$.
        Then by problem 2.2 above, there
        exists a unique smooth path
        $\tilde{f} \colon I \to \mathbb{R}$
        sending $0$ to $0$ such that
        \begin{equation*}
        \begin{tikzcd}
            & & \mathbb{R} \ar[d, "p"]\\
            I \ar[r, "p|_{I}"'] 
            \ar[rru, "\tilde{f}"] & S^{1} \ar[r, "f"'] & S^{1}
        \end{tikzcd}
        \end{equation*}
        commutes.
        But
        since $f$ is a diffeomorphism, it 
        induces an isomorphism on
        fundamental groups,
        so $f$ sends the generating circle
        $1 \in \mathbb{Z} \cong \pi_1(S^{1})$ to
        either $1$ or $-1$ (the two generators for
        $\mathbb{Z}$ ).
        But then by uniqueness of endpoints of lifts,
        $\tilde{f}(1) \in \left\{ 1,-1 \right\} $.
        Now suppose firstly that
        $\tilde{f}(1) = 1$.
        Since $\tilde{f}$ is smooth and
        $f = p \circ \tilde{f}$ on $I$ is a diffeomorphism
        on the interior of $I$, it must be
        a strictly increasing function, and
        since $f$ is smooth, 
        its limits in  $S^{1}$ about $(1,0)$ agree
        from both sides, so we conclude that
        $\lim_{t \to 0+} \frac{d^{n}}{dt^{n}} \tilde{f}(t)
        = \lim_{t \to 1-} \frac{d^{n}}{dt^n}
        \tilde{f}(t)$ for all $n$.
        Now let
        $\varphi \colon
        \mathbb{R} \times I \to \mathbb{R}$ 
        be the isotopy
        which is the identity on
        $\mathbb{R} - (0,1)$ and
        on $\left( 0,1 \right) $, it is given by
        \[
        \varphi (x,t) =
        x t + f(x) (1-t)
        \] 
        Then
        clearly
        $\varphi $ is smooth and
        $\varphi (x,1) = x$ and
        $\varphi (x,0) = f(x)$.
        Furthermore,
        $\varphi $
        is a diffeomorphism of  $\left( 0,1 \right) $ 
        at each time $t$ and of 
        $\mathbb{R} - \left( 0,1 \right) $.
        We must check that the limits of
         the derivatives of $\varphi $ coincide for
         the two endpoints of the interval at each
         time $t$.
         Indeed,
         \[
         \lim_{x \to 0+} \frac{d}{dx}
         \varphi (x,t)
         = \lim_{x \to 0+} t + (1-t) f'(x)
         = \lim_{x \to 1-} t + (1-t) f'(x)
         = \lim_{x \to 1-} \frac{d}{dx}\varphi (x,t)
         \] 
         since the limits for
         $f'(x)$ agree.
         Higher limits can be checked likewise. 
         Hence $\varphi (-,t)$ is a diffeomorphism
         for all $t$. The composite
         $p \circ \varphi (\tilde{f}(x),t) $ gives
         an isotopy from
         $f$ to the identity.
         For the case where
         $\tilde{f}(1) = -1$, simply note that the
         reflection map also induces this isomorphism,
         so if $g$ is the reflection map,
         then $\tilde{f \circ g^{-1}}(1) = 1$, so
         $f \circ g^{-1} $ is isotopic to the identity,
         hence $f$ is isotopic to $g$.
         
         

        


 
    \end{proof}


    \begin{problem}[4 (Local properties of homeomorphisms
        of $\mathbb{R}^{d}$ )]
        Let $d \in \mathbb{N} $ and
        denote by $D^{d}$ the unit disc in
        $\mathbb{R}^{d}$. Show that if
        $f,g \in \Homeo \left( \mathbb{R}^{d} \right) $ are
        two homeomorphisms such that there exists
        an open subset $U \subset \mathbb{R}^{d}$ where
        $f|_{U} = g|_{U}$, then $f$ and $g$ are isotopic.
    \end{problem}

    \begin{proof}
        Since isotopic forms an equivalence relation, it
        suffices to show the situation for
        $g = \id$.
        Now, a homeomorphism  $f$ of $\mathbb{R}^{d}$ extends
        to a homeomorphism of its one-point compactification
        $S^{d}$. Now remove an interior disc
        $D^2 \hookrightarrow 
        S^{d}$ contained in the open subset $U 
        \subset S^{d}$ on which $f$ is the identity. The resulting
        space is a closed disc, and by assumption, $f$ is
        the identity on the boundary. Using the Alexander trick
        for the punctured disc (the puncture being the
        point at infinity),
        we find an isotopy of $f$ on this disc to the identity.
        Now glue back the disc we removed. Since the isotopy
        fixes the boundary pointwise at all times, the isotopy
        extends to an isotopy of $S^{d}$. Now transfer
        this isotopy back to $\mathbb{R}^{d}$ using the
        stereographic projection. This gives an isotopy of
        $f$ with the identity on $\mathbb{R}^{d}$.
    \end{proof}



    %\bibliography{../refs.bib}
\end{document}
