\documentclass[reqno]{amsart}
\usepackage{amscd, amssymb, amsmath, amsthm}
\usepackage{graphicx}
\usepackage[colorlinks=true,linkcolor=blue]{hyperref}
\usepackage[utf8]{inputenc}
\usepackage[T1]{fontenc}
\usepackage{textcomp}
\usepackage{babel}
%% for identity function 1:
\usepackage{bbm}
%%For category theory diagrams:
\usepackage{tikz-cd}

%\usepackage[backend=biber]{biblatex}
%\addbibresource{.bib}


\setlength\parindent{0pt}

\pdfsuppresswarningpagegroup=1

\newtheorem{theorem}{Theorem}[section]
\newtheorem{lemma}[theorem]{Lemma}
\newtheorem{proposition}[theorem]{Proposition}
\newtheorem{corollary}[theorem]{Corollary}
\newtheorem{conjecture}[theorem]{Conjecture}

\theoremstyle{definition}
\newtheorem{definition}[theorem]{Definition}
\newtheorem{example}[theorem]{Example}
\newtheorem{exercise}[theorem]{Exercise}
\newtheorem{problem}[theorem]{Problem}
\newtheorem{question}[theorem]{Question}

\theoremstyle{remark}
\newtheorem*{remark}{Remark}
\newtheorem*{note}{Note}
\newtheorem*{solution}{Solution}



%Inequalities
\newcommand{\cycsum}{\sum_{\mathrm{cyc}}}
\newcommand{\symsum}{\sum_{\mathrm{sym}}}
\newcommand{\cycprod}{\prod_{\mathrm{cyc}}}
\newcommand{\symprod}{\prod_{\mathrm{sym}}}

%Linear Algebra

\DeclareMathOperator{\Span}{span}
\DeclareMathOperator{\im}{im}
\DeclareMathOperator{\diag}{diag}
\DeclareMathOperator{\Ker}{Ker}
\DeclareMathOperator{\ob}{ob}
\DeclareMathOperator{\Hom}{Hom}
\DeclareMathOperator{\Mor}{Mor}
\DeclareMathOperator{\sk}{sk}
\DeclareMathOperator{\Vect}{Vect}
\DeclareMathOperator{\Set}{Set}
\DeclareMathOperator{\Group}{Group}
\DeclareMathOperator{\Ring}{Ring}
\DeclareMathOperator{\Ab}{Ab}
\DeclareMathOperator{\Top}{Top}
\DeclareMathOperator{\hTop}{hTop}
\DeclareMathOperator{\Htpy}{Htpy}
\DeclareMathOperator{\Cat}{Cat}
\DeclareMathOperator{\CAT}{CAT}
\DeclareMathOperator{\Cone}{Cone}
\DeclareMathOperator{\dom}{dom}
\DeclareMathOperator{\cod}{cod}
\DeclareMathOperator{\Aut}{Aut}
\DeclareMathOperator{\Mat}{Mat}
\DeclareMathOperator{\Fin}{Fin}
\DeclareMathOperator{\rel}{rel}
\DeclareMathOperator{\Int}{Int}
\DeclareMathOperator{\sgn}{sgn}
\DeclareMathOperator{\Homeo}{Homeo}
\DeclareMathOperator{\SHomeo}{SHomeo}
\DeclareMathOperator{\PSL}{PSL}
\DeclareMathOperator{\Bil}{Bil}
\DeclareMathOperator{\Sym}{Sym}
\DeclareMathOperator{\Skew}{Skew}
\DeclareMathOperator{\Alt}{Alt}
\DeclareMathOperator{\Quad}{Quad}
\DeclareMathOperator{\Sin}{Sin}
\DeclareMathOperator{\Supp}{Supp}
\DeclareMathOperator{\Char}{char}
\DeclareMathOperator{\Teich}{Teich}
\DeclareMathOperator{\GL}{GL}
\DeclareMathOperator{\tr}{tr}
\DeclareMathOperator{\codim}{codim}
\DeclareMathOperator{\coker}{coker}
\DeclareMathOperator{\Diff}{Diff}
\DeclareMathOperator{\Bun}{Bun}
\DeclareMathOperator{\Sm}{Sm}



%Row operations
\newcommand{\elem}[1]{% elementary operations
\xrightarrow{\substack{#1}}%
}

\newcommand{\lelem}[1]{% elementary operations (left alignment)
\xrightarrow{\begin{subarray}{l}#1\end{subarray}}%
}

%SS
\DeclareMathOperator{\supp}{supp}
\DeclareMathOperator{\Var}{Var}

%NT
\DeclareMathOperator{\ord}{ord}

%Alg
\DeclareMathOperator{\Rad}{Rad}
\DeclareMathOperator{\Jac}{Jac}

%Misc
\newcommand{\SL}{{\mathrm{SL}}}
\newcommand{\mobgp}{{\mathrm{PSL}_2(\mathbb{C})}}
\newcommand{\id}{{\mathrm{id}}}
\newcommand{\MCG}{{\mathrm{MCG}}}
\newcommand{\PMCG}{{\mathrm{PMCG}}}
\newcommand{\SMCG}{{\mathrm{SMCG}}}
\newcommand{\ud}{{\mathrm{d}}}
\newcommand{\Vol}{{\mathrm{Vol}}}
\newcommand{\Area}{{\mathrm{Area}}}
\newcommand{\diam}{{\mathrm{diam}}}
\newcommand{\End}{{\mathrm{End}}}


\newcommand{\reg}{{\mathtt{reg}}}
\newcommand{\geo}{{\mathtt{geo}}}

\newcommand{\tori}{{\mathcal{T}}}
\newcommand{\cpn}{{\mathtt{c}}}
\newcommand{\pat}{{\mathtt{p}}}

\let\Cap\undefined
\newcommand{\Cap}{{\mathcal{C}}ap}
\newcommand{\Push}{{\mathcal{P}}ush}
\newcommand{\Forget}{{\mathcal{F}}orget}




\begin{document}

\section{Problems}

\begin{definition}[]
    Let $M$ be a smooth manifold. A Morse function
    $f \colon M \to \mathbb{R}$ is a smooth map such that
    all its critical points are non-degenerate, with
    pairwise distinct critical values in $\mathbb{R}$.
\end{definition}


\begin{problem}[Reeb's Theorem]\label{Reeb's-Theorem}
        (6 pts) Let $M$ be a smooth, compact manifold of
        dimension $d$. Show that if $M$ admits a Morse
        function with only two critical points, then
        $M$ is homeomorphic to the sphere $S^{d}$. Indicate
        why the above proof fails in showing that $M$ is
        diffeomorphic to the sphere $S^{d}$.
    \end{problem}

    For the proof, we state a theorem that we will need:
    \begin{definition}[]
        For a smooth map $f \colon M \to \mathbb{R}$ on a 
        smooth manifold $M$, let
        $M^{a} = f^{-1} (-\infty, a]$.
    \end{definition}

    \begin{theorem}[]\label{Thm1}
        Let $f \in C^{\infty}(M)$ on a manifold $M$.
        Let $a < b$ and suppose that the set
        $f^{-1}\left[ a,b \right] $ is compact and
        contains no critical points of $f$. Then
        $M^{a}$ is diffeomorphic to $M^{b}$. Furthermore,
        $M^{a}$ is a deformation retract of
        $M^{b}$, so the inclusion $M^{a} \hookrightarrow 
        M^{b}$ is a homotopy equivalence.
    \end{theorem}


    \begin{proof}[Proof of Problem \ref{Reeb's-Theorem}]
        Since $M$ is compact,
        we have that
        $f(M) = \left[ a,b \right] \subset \mathbb{R}$.
        Without loss of generality, assume that
        $f(M) = \left[ 0,1 \right] $.\\

        We shall need the following lemma from
        analysis:
        \begin{lemma}[Fermat's Theorem]
            Let $f \colon \left( a,b \right) \to \mathbb{R}$ 
            be a function on an open interval
            $\left( a,b \right) \subset \mathbb{R}$.
            Suppose $f$ has a local extremum at
            $x_0 \in (a,b)$. If
            $f$ is differentiable at $x_0$, then
            $f'(x_0) = 0$.
        \end{lemma}

        Now, we claim that
        the two critical points are precisely the preimages
        of $0$ and $1$.
        For suppose
        $x \in f^{-1}(0)$.
        Then $x$ is a global minimum for $f$.
        Taking some chart centered around $x$, we have a local
        representation
        of $f$ as a function $\mathbb{R}^{d} \to 
        \left[ 0,1 \right] $ 
        with a global minimum at $0$.
        Taking the partial derivatives of
        $f$ and applying Fermat's theorem to each of them,
        we find that each partial derivative evaluated at $0$ 
        is $0$: $\frac{\partial f}{\partial x^{i}}(0) = 0$.
        Hence we find that
        $Df(0) = 0$, so transfering back to the manifold,
        $Df(x) = 0$, so $x \in M$ is a critical point.
        The same argument applies to show that
        any $y \in f^{-1}(1)$ is a critical point.
        Since there are only two critical points, this
        immediately forces
        $f^{-1}(0)$ and $f^{-1}(1)$ to be singletons
        and thus global maximum and minimum of $M$.
        Suppose without loss of generality that
        $p \in M$ is the minimum and
        $q \in M$ is the maximum.

        By Morse's Lemma, in some coordinate system about
         $p$, let's say in a neighborhood $U$,
         $f$ takes the form
         \[
         f\left( x_1,\ldots,x_n \right) = 
         - x_1^2 - \ldots - x_{\lambda}^2 +
         x_{\lambda+1}^2 +\ldots + x_n^2.
         \] 
         Now $p$ is a global minimum, so in fact, we
         must have that $\lambda = 0$. That is
         \[
         f\left( x_1,\ldots,x_n \right) =
         x_{1}^2 + \ldots + x_n^2
         \] 
         in this neighborhood.
         Since also
         $f(U)$ is open in the subspace topology
         and contains $0$,
         we can find an open disk $\tilde{D}_1$ centered
         at $0$ of radius $\varepsilon_1$ such that
         $ \tilde{D}_1 \cap \left[ 0,1 \right] \subset 
         f(U)$, and
         let $D_1$ be the inverse of $\tilde{D}_1$ under this
         local diffeomorphism.\\
         Similarly, in a neighborhood $V$ of
         $q$, $f$ takes the form
         \[
         f\left( x_1,\ldots,x_n \right)  = 
         1- x_1^2 - x_2^2 - \ldots - x_n^2.
         \] 
         Again take some open disk $\tilde{D}_2$
         centered at $1$ of radius $\varepsilon_2$ 
         such that $ \tilde{D}_2 \cap
         \left[ 0,1 \right] \subset f(V)$.
         Let $D_2$ be the inverse image under $f$ of
         $\tilde{D}_2$.\\
         We wish to show that there
         exists some $\varepsilon > 0$ such that
         $f^{-1}\left[ 0,\varepsilon \right] $ 
         and $f^{-1}\left[ 1-\varepsilon,1 \right] $ are
         homeomorphic to the closed $n$-disk $D^{n}$.
         There
         exist
         $\alpha, \beta \in \left( 0,1 \right) $ such that
         $f\left( M - D_1 \cup D_2 \right) 
         = \left[ \alpha, \beta \right] $ since
         $M - D_1 \cup  D_2$ is still compact.
         Now simply let
         $0 < \varepsilon < \min 
         \left\{ \alpha, 1 - \beta, \varepsilon_1,
         1-\varepsilon_2, 1-\varepsilon_1, \frac{1}{4}\right\} $.
         To see that this works, simply note that
         $f^{-1}\left[ 0, \varepsilon \right] 
         \subset D_1 \cup  D_2$.
         On $D_1$, $f$ takes values in
         $\left[ 0, \varepsilon_1 \right] $ and
         on $D_2$, $f$ takes values in
         $\left[ 1- \varepsilon_2 , 1 \right] $.
         But $\varepsilon < \varepsilon_1$, so
         $\left[ 0, \varepsilon_1 \right] \subset 
         \left[ 0, \varepsilon \right] $, so
         $D_1 \subset f^{-1}\left[ 0,\varepsilon \right] $,
         while
         $\varepsilon < 1- \varepsilon_2$, so
         $\left[ 1-\varepsilon_2,1 \right] \not \subset 
         f^{-1}\left[ 0,\varepsilon \right] $.
         Similarly,
         $1- \varepsilon > \varepsilon_1$, so
         $D_1 \subset \left[ 0,\varepsilon_1 \right] 
         \not \subset f^{-1}\left[ 1-\varepsilon,1 \right] $
         while $1-\varepsilon_2 > 1- \varepsilon$, so
         $D_2 \subset \left[ 1-\varepsilon_2, 1 \right] 
         \subset f^{-1}\left[ 1-\varepsilon,1 \right] $.\\
         \linebreak
         Therefore,
         since $f^{-1}\left[ 0,\varepsilon \right] \subset D_1
         \subset U$ and
         we know that on
         $U$, $f$ takes the form
         \[
         f\left( x_1,\ldots,x_n \right) 
         = x_1^2 + \ldots + x_n^2,
         \] 
         we know that
         $f^{-1}\left[ 0,\varepsilon \right] $ is
         precisely a closed disk about
         $p$. Likewise,
         $f^{-1}\left[ 1-\varepsilon,1 \right] $ can
         be seen to be a closed disk about $q$.\\

         But now by Theorem \ref{Thm1}, since
         there are no critical points in
         $f^{-1}\left[ \varepsilon, 1- \varepsilon \right] $ 
         by assumption,
         $M^{\varepsilon}$ is diffeomorphic to
         $M^{1- \varepsilon}$.
         Hence we find that
         $M^{1-\varepsilon}$ and
         $f^{-1}\left[ 1-\varepsilon,1 \right] $ 
         are both diffeomorphic to closed $d$-disks, and
         furthermore,
         $M$ is obtained by gluing these  $d$-disks along their
         boundary. We claim that this is sufficient
         to conclude that $M$ is \textit{homeomorphic} to
         $S^{d}$.


    \end{proof}




    %\printbibliography
\end{document}
