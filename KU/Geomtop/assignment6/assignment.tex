\documentclass[reqno]{amsart}
\usepackage{amscd, amssymb, amsmath, amsthm}
\usepackage{graphicx}
\usepackage[colorlinks=true,linkcolor=blue]{hyperref}
\usepackage[utf8]{inputenc}
\usepackage[T1]{fontenc}
\usepackage{textcomp}
\usepackage{babel}
%% for identity function 1:
\usepackage{bbm}
%%For category theory diagrams:
\usepackage{tikz-cd}

\usepackage[backend=biber]{biblatex}
\addbibresource{assignment.bib}


\setlength\parindent{0pt}

\pdfsuppresswarningpagegroup=1

\newtheorem{theorem}{Theorem}[section]
\newtheorem{lemma}[theorem]{Lemma}
\newtheorem{proposition}[theorem]{Proposition}
\newtheorem{corollary}[theorem]{Corollary}
\newtheorem{conjecture}[theorem]{Conjecture}

\theoremstyle{definition}
\newtheorem{definition}[theorem]{Definition}
\newtheorem{example}[theorem]{Example}
\newtheorem{exercise}[theorem]{Exercise}
\newtheorem{problem}[theorem]{Problem}
\newtheorem{question}[theorem]{Question}

\theoremstyle{remark}
\newtheorem*{remark}{Remark}
\newtheorem*{note}{Note}
\newtheorem*{solution}{Solution}



%Inequalities
\newcommand{\cycsum}{\sum_{\mathrm{cyc}}}
\newcommand{\symsum}{\sum_{\mathrm{sym}}}
\newcommand{\cycprod}{\prod_{\mathrm{cyc}}}
\newcommand{\symprod}{\prod_{\mathrm{sym}}}

%Linear Algebra

\DeclareMathOperator{\Span}{span}
\DeclareMathOperator{\im}{im}
\DeclareMathOperator{\diag}{diag}
\DeclareMathOperator{\Ker}{Ker}
\DeclareMathOperator{\ob}{ob}
\DeclareMathOperator{\Hom}{Hom}
\DeclareMathOperator{\Mor}{Mor}
\DeclareMathOperator{\sk}{sk}
\DeclareMathOperator{\Vect}{Vect}
\DeclareMathOperator{\Set}{Set}
\DeclareMathOperator{\Group}{Group}
\DeclareMathOperator{\Ring}{Ring}
\DeclareMathOperator{\Ab}{Ab}
\DeclareMathOperator{\Top}{Top}
\DeclareMathOperator{\hTop}{hTop}
\DeclareMathOperator{\Htpy}{Htpy}
\DeclareMathOperator{\Cat}{Cat}
\DeclareMathOperator{\CAT}{CAT}
\DeclareMathOperator{\Cone}{Cone}
\DeclareMathOperator{\dom}{dom}
\DeclareMathOperator{\cod}{cod}
\DeclareMathOperator{\Aut}{Aut}
\DeclareMathOperator{\Mat}{Mat}
\DeclareMathOperator{\Fin}{Fin}
\DeclareMathOperator{\rel}{rel}
\DeclareMathOperator{\Int}{Int}
\DeclareMathOperator{\sgn}{sgn}
\DeclareMathOperator{\Homeo}{Homeo}
\DeclareMathOperator{\SHomeo}{SHomeo}
\DeclareMathOperator{\PSL}{PSL}
\DeclareMathOperator{\Bil}{Bil}
\DeclareMathOperator{\Sym}{Sym}
\DeclareMathOperator{\Skew}{Skew}
\DeclareMathOperator{\Alt}{Alt}
\DeclareMathOperator{\Quad}{Quad}
\DeclareMathOperator{\Sin}{Sin}
\DeclareMathOperator{\Supp}{Supp}
\DeclareMathOperator{\Char}{char}
\DeclareMathOperator{\Teich}{Teich}
\DeclareMathOperator{\GL}{GL}
\DeclareMathOperator{\tr}{tr}
\DeclareMathOperator{\codim}{codim}
\DeclareMathOperator{\coker}{coker}
\DeclareMathOperator{\Diff}{Diff}
\DeclareMathOperator{\Bun}{Bun}
\DeclareMathOperator{\Sm}{Sm}



%Row operations
\newcommand{\elem}[1]{% elementary operations
\xrightarrow{\substack{#1}}%
}

\newcommand{\lelem}[1]{% elementary operations (left alignment)
\xrightarrow{\begin{subarray}{l}#1\end{subarray}}%
}

%SS
\DeclareMathOperator{\supp}{supp}
\DeclareMathOperator{\Var}{Var}

%NT
\DeclareMathOperator{\ord}{ord}

%Alg
\DeclareMathOperator{\Rad}{Rad}
\DeclareMathOperator{\Jac}{Jac}

%Misc
\newcommand{\SL}{{\mathrm{SL}}}
\newcommand{\mobgp}{{\mathrm{PSL}_2(\mathbb{C})}}
\newcommand{\id}{{\mathrm{id}}}
\newcommand{\MCG}{{\mathrm{MCG}}}
\newcommand{\PMCG}{{\mathrm{PMCG}}}
\newcommand{\SMCG}{{\mathrm{SMCG}}}
\newcommand{\ud}{{\mathrm{d}}}
\newcommand{\Vol}{{\mathrm{Vol}}}
\newcommand{\Area}{{\mathrm{Area}}}
\newcommand{\diam}{{\mathrm{diam}}}
\newcommand{\End}{{\mathrm{End}}}


\newcommand{\reg}{{\mathtt{reg}}}
\newcommand{\geo}{{\mathtt{geo}}}

\newcommand{\tori}{{\mathcal{T}}}
\newcommand{\cpn}{{\mathtt{c}}}
\newcommand{\pat}{{\mathtt{p}}}

\let\Cap\undefined
\newcommand{\Cap}{{\mathcal{C}}ap}
\newcommand{\Push}{{\mathcal{P}}ush}
\newcommand{\Forget}{{\mathcal{F}}orget}




\begin{document}

\section{Problems}

\begin{definition}[]
    Let $M$ be a smooth manifold. A Morse function
    $f \colon M \to \mathbb{R}$ is a smooth map such that
    all its critical points are non-degenerate, with
    pairwise distinct critical values in $\mathbb{R}$.
\end{definition}

\subsection{Reeb's Theorem}

\begin{problem}[Reeb's Theorem]\label{Reeb's-Theorem}
        (6 pts) Let $M$ be a smooth, compact manifold of
        dimension $d$. Show that if $M$ admits a Morse
        function with only two critical points, then
        $M$ is homeomorphic to the sphere $S^{d}$. Indicate
        why the above proof fails in showing that $M$ is
        diffeomorphic to the sphere $S^{d}$.
    \end{problem}

    For the proof, we state a theorem that we will need:
    \begin{definition}[]
        For a smooth map $f \colon M \to \mathbb{R}$ on a 
        smooth manifold $M$, let
        $M^{a} = f^{-1} (-\infty, a]$.
    \end{definition}

    \begin{theorem}[]\label{Thm1}
        Let $f \in C^{\infty}(M)$ on a manifold $M$.
        Let $a < b$ and suppose that the set
        $f^{-1}\left[ a,b \right] $ is compact and
        contains no critical points of $f$. Then
        $M^{a}$ is diffeomorphic to $M^{b}$. Furthermore,
        $M^{a}$ is a deformation retract of
        $M^{b}$, so the inclusion $M^{a} \hookrightarrow 
        M^{b}$ is a homotopy equivalence.
    \end{theorem}


    \begin{proof}[Proof of Problem \ref{Reeb's-Theorem}]
        Since $M$ is compact,
        we have that
        $f(M) = \left[ a,b \right] \subset \mathbb{R}$.
        Without loss of generality, assume that
        $f(M) = \left[ 0,1 \right] $.\\

        We shall need the following lemma from
        analysis:
        \begin{lemma}[Fermat's Theorem]
            Let $f \colon \left( a,b \right) \to \mathbb{R}$ 
            be a function on an open interval
            $\left( a,b \right) \subset \mathbb{R}$.
            Suppose $f$ has a local extremum at
            $x_0 \in (a,b)$. If
            $f$ is differentiable at $x_0$, then
            $f'(x_0) = 0$.
        \end{lemma}

        Now, we claim that
        the two critical points are precisely the preimages
        of $0$ and $1$.
        For suppose
        $x \in f^{-1}(0)$.
        Then $x$ is a global minimum for $f$.
        Taking some chart centered around $x$, we have a local
        representation
        of $f$ as a function $\mathbb{R}^{d} \to 
        \left[ 0,1 \right] $ 
        with a global minimum at $0$.
        Taking the partial derivatives of
        $f$ and applying Fermat's theorem to each of them,
        we find that each partial derivative evaluated at $0$ 
        is $0$: $\frac{\partial f}{\partial x^{i}}(0) = 0$.
        Hence we find that
        $Df(0) = 0$, so transfering back to the manifold,
        $Df(x) = 0$, so $x \in M$ is a critical point.
        The same argument applies to show that
        any $y \in f^{-1}(1)$ is a critical point.
        Since there are only two critical points, this
        immediately forces
        $f^{-1}(0)$ and $f^{-1}(1)$ to be singletons
        and thus global maximum and minimum of $M$.
        Suppose without loss of generality that
        $p \in M$ is the minimum and
        $q \in M$ is the maximum.

        By Morse's Lemma, in some coordinate system about
         $p$, let's say in a neighborhood $U$,
         $f$ takes the form
         \[
         f\left( x_1,\ldots,x_n \right) = 
         - x_1^2 - \ldots - x_{\lambda}^2 +
         x_{\lambda+1}^2 +\ldots + x_n^2.
         \] 
         Now $p$ is a global minimum, so in fact, we
         must have that $\lambda = 0$. That is
         \[
         f\left( x_1,\ldots,x_n \right) =
         x_{1}^2 + \ldots + x_n^2
         \] 
         in this neighborhood.
         Since also
         $f(U)$ is open in the subspace topology
         and contains $0$,
         we can find an open disk $\tilde{D}_1$ centered
         at $0$ of radius $\varepsilon_1$ such that
         $ \tilde{D}_1 \cap \left[ 0,1 \right] \subset 
         f(U)$, and
         let $D_1$ be the inverse of $\tilde{D}_1$ under this
         local diffeomorphism.\\
         Similarly, in a neighborhood $V$ of
         $q$, $f$ takes the form
         \[
         f\left( x_1,\ldots,x_n \right)  = 
         1- x_1^2 - x_2^2 - \ldots - x_n^2.
         \] 
         Again take some open disk $\tilde{D}_2$
         centered at $1$ of radius $\varepsilon_2$ 
         such that $ \tilde{D}_2 \cap
         \left[ 0,1 \right] \subset f(V)$.
         Let $D_2$ be the inverse image under $f$ of
         $\tilde{D}_2$.\\
         We wish to show that there
         exists some $\varepsilon > 0$ such that
         $f^{-1}\left[ 0,\varepsilon \right] $ 
         and $f^{-1}\left[ 1-\varepsilon,1 \right] $ are
         homeomorphic to the closed $n$-disk $D^{n}$.
         There
         exist
         $\alpha, \beta \in \left( 0,1 \right) $ such that
         $f\left( M - D_1 \cup D_2 \right) 
         = \left[ \alpha, \beta \right] $ since
         $M - D_1 \cup  D_2$ is still compact.
         Now simply let
         $0 < \varepsilon < \min 
         \left\{ \alpha, 1 - \beta, \varepsilon_1,
         1-\varepsilon_2, 1-\varepsilon_1, \frac{1}{4}\right\} $.
         To see that this works, simply note that
         $f^{-1}\left[ 0, \varepsilon \right] 
         \subset D_1 \cup  D_2$.
         On $D_1$, $f$ takes values in
         $\left[ 0, \varepsilon_1 \right] $ and
         on $D_2$, $f$ takes values in
         $\left[ 1- \varepsilon_2 , 1 \right] $.
         But $\varepsilon < \varepsilon_1$, so
         $\left[ 0, \varepsilon_1 \right] \subset 
         \left[ 0, \varepsilon \right] $, so
         $D_1 \subset f^{-1}\left[ 0,\varepsilon \right] $,
         while
         $\varepsilon < 1- \varepsilon_2$, so
         $\left[ 1-\varepsilon_2,1 \right] \not \subset 
         f^{-1}\left[ 0,\varepsilon \right] $.
         Similarly,
         $1- \varepsilon > \varepsilon_1$, so
         $D_1 \subset \left[ 0,\varepsilon_1 \right] 
         \not \subset f^{-1}\left[ 1-\varepsilon,1 \right] $
         while $1-\varepsilon_2 > 1- \varepsilon$, so
         $D_2 \subset \left[ 1-\varepsilon_2, 1 \right] 
         \subset f^{-1}\left[ 1-\varepsilon,1 \right] $.\\
         \linebreak
         Therefore,
         since $f^{-1}\left[ 0,\varepsilon \right] \subset D_1
         \subset U$ and
         we know that on
         $U$, $f$ takes the form
         \[
         f\left( x_1,\ldots,x_n \right) 
         = x_1^2 + \ldots + x_n^2,
         \] 
         we know that
         $f^{-1}\left[ 0,\varepsilon \right] $ is
         precisely a closed disk about
         $p$. Likewise,
         $f^{-1}\left[ 1-\varepsilon,1 \right] $ can
         be seen to be a closed disk about $q$.\\

         But now by Theorem \ref{Thm1}, since
         there are no critical points in
         $f^{-1}\left[ \varepsilon, 1- \varepsilon \right] $ 
         by assumption,
         $M^{\varepsilon}$ is diffeomorphic to
         $M^{1- \varepsilon}$.
         Hence we find that
         $M^{1-\varepsilon}$ and
         $f^{-1}\left[ 1-\varepsilon,1 \right] $ 
         are both diffeomorphic to closed $d$-disks, and
         furthermore,
         $M$ is obtained by gluing these  $d$-disks along their
         boundary which
         is homeomorphic to
         $S^{d-1}$. We claim that this is sufficient
         to conclude that $M$ is \textit{homeomorphic} to
         $S^{d}$.
         The problem is that while
         we have individual diffeomorphisms
         $M^{1-\varepsilon} \cong
         D^{n}$ and
         $f^{-1}\left[ 1-\varepsilon,1 \right] 
         \cong D^{n}$, the identifications of the boundaries
         might not be preserved under these diffeomorphisms,
         so we might not be able to reglue after.
         Let
         $\varphi_1 \colon M^{1- \varepsilon}
         \cong D^{d}$ and
         $\varphi_2 \colon
         f^{-1}\left[ 1-\varepsilon,1 \right] 
         \cong D^{d}$ be the diffeomorphisms.
         Then 
         $\varphi_1 \circ \varphi_2^{-1}$ is a diffeomorphism
         of $S^{d-1}$, and
         \[
         M \cong D^{d} \sqcup_{\varphi_1 \circ \varphi_2^{-1}}
         D^{d}.
         \] 
         We construct a homeomorphism
         $\psi \colon D_1 \sqcup_{\id} D_2 \to 
         D^{d} \sqcup_{\varphi_1 \circ \varphi_2^{-1}}
         D^{d}$ by
         \[
         \psi (x)
         =
         \begin{cases}
             x&, x \in D_1\\
             0&, x \in D_2 \text{ and } x = 0\\
             \|x\| \varphi_1 \circ
             \varphi_2^{-1} \left( \frac{x}{\|x\|} \right) ,&
             x \in D_2 - \left\{ 0 \right\} 
         \end{cases}
         \] 
         As the sphere is compact and
          the twisted sphere Hausdorff, this
          map is a homeomorphism.\\
          The reason it might fail to be a diffeomorphism, is
          that on $D_2 - \left\{ 0 \right\} $, as
          we  let $x$ approach $0$, we might have
          non-agreeing derivatives from different directions.\\
          \linebreak
          A different way of obtaining a homeomorphism
          is as follows: since
          $\varphi_1$ and $\varphi_2$ can be
          chosen to both be orientation-preserving, for example
          by precomposing with an orientation reversing
          self-homeomorphism of the disk, we find that
          $\varphi_1 \circ \varphi_2^{-1}$ is isotopic through
          topological embeddings to the identity. Now
          applying an isotopy extension theorem,
          \cite[Thm 1.3, p. 180]{Hirsch}, this isotopy
          extends to an ambient isotopy of
          $S^{d}$ with compact support.


    \end{proof}


    \subsection{Existence of Morse functions}

    \begin{problem}[Existence of Morse functions]
        (6pts) Show that any smooth manifold $M$ admits
        a Morse function.
    \end{problem}

    \begin{proof}
        Suppose $M$ is of dimension $k$. By the Whitney embedding
        theorem, we can smoothly
        embed $M$ in $\mathbb{R}^{n}$ for some
        $n\ge k$.
        Let $N \subset M \times \mathbb{R}^{n}$ be
        defined by
        \[
        N = \left\{ \left( q,v \right) \colon
        q \in M, v \in  M_q^{\perp} \right\} 
        \] 
        \begin{lemma}[]
            $N$ is an $n$-dimensional manifold smoothly
            embedded in $\mathbb{R}^{2n}$.
        \end{lemma}

        Define
        $E \colon N \to \mathbb{R}^{n}$ by
        $E(q,v) = q+v$.

        \begin{definition}[]
            A point $e \in \mathbb{R}^{n}$ is called
            a \textit{focal point of $(M,q)$ with
            multiplicity $\mu$ } if
        $e = q+v$ where
    $(q,v) \in N$ and the Jacobian of $E$ at
    $\left( q,v \right) $ has nullity
    $\mu > 0$. The point $e$ will be called a \textit{focal point}
    of $M$ if $e$ is a focal point of $\left( M,q \right) $ 
    for some $q \in M$.
        \end{definition}

        \begin{definition}[Critical point]
            For our purposes, we will define
            a critical point of a smooth map
            $f$ to be a point where the Jacobian is
            singular, i.e., $\det df = 0$.
            In particular, critical points in the usual definition
            where $df$ vanishes at the point are included in this
            definition since if $df$ vanishes at $p$, then
            $\det df(p) = 0$.
        \end{definition}


        Now, since $N$ is an $n$-manifold, note that
        $E \colon N \to \mathbb{R}^{n}$ is a map
        between two $n$-dimensional manifolds. In particular,
        $dE$ is a map between two $n$-dimensional tangent spaces
        at each point. Therefore,
        by definition, a point
        $e \in \mathbb{R}^{n}$ is a focal point
        $e = q+v$ with
        $\left( q,v \in N \right) $ 
        if and only if $dE$ is not injective at
        $(q,v)$ if and only if
        $\det dE_{(q,v)} = 0$ if and only if
        $(q,v)$ is a critical point of $E$.
        But $E$ is clearly smooth, so by
        Sard's theorem, the set of critical \textit{values}
        of $E$ which corresponds to the set of focal points
        has Lebesgue measure $0$.\\
        \linebreak
        Let now $\left( U, \left( u^{i} \right) = \varphi  \right) $ 
        be a chart on $M$ with $i = 1,\ldots,k$, and
        consider the inclusion
        $M \hookrightarrow \mathbb{R}^{n}$.
        Then we obtain natural coordinates
        in $\mathbb{R}^{n}$ given by
        $\mathbb{R}^{k} \stackrel{\varphi^{-1}}{\hookrightarrow} M \hookrightarrow 
        \mathbb{R}^{n}$.
        We let
        $x_1\left( u_1,\ldots,u_k \right) ,\ldots,
        x_n \left( u_1,\ldots,u_k \right) $ be these
        maps and
        $x = \left( x_1,\ldots, x_n \right) \colon
        \mathbb{R}^{k} \to \mathbb{R}^{n}$.
        
        \begin{definition}[First and second fundamental forms]
            Given the above setup, we call
            the following matrix the first fundamental form:
            \[
            g_{ij} = 
            \begin{pmatrix} \frac{\partial x}{\partial u^i}
            \cdot \frac{\partial x}{\partial u^{j}} \end{pmatrix} ,
            \] 
            the dot signaling the usual dot product.

            Similarly, we define a matrix
            $\left( l_{ij} \right) $ called the
            second fundamental form where
            $l_{ij}$ is the summand of the vector
            $\frac{\partial^2 x}{\partial u^{i} \partial
            u^{j}}$ which is normal to $M$.



        \end{definition}


    \end{proof}


    \subsection{On the Transversality Theorem}

    \begin{problem}[On the transversality theorem]
        Let $M$ be a smooth manifold.
        \begin{enumerate}
            \item Let $X \subset M$ be a smooth
                submanifold, and let $f \colon Y \to M$ 
                be a smooth map, where
                $Y$ is a smooth manifold. Show that
                $f$ is smoothly homotopic to a map that
                intersects $X$ transversally at every
                point.
            \item Show that in the above setting, if
                $f \colon Y \to M$ intersects $X$ transversally,
                then $f^{-1}(X)$ is a smooth submanifold
                of $Y$ such that $\dim Y + \dim
                f^{-1}(X) = \dim X$.
        \end{enumerate}
    \end{problem}



\printbibliography
\end{document}
