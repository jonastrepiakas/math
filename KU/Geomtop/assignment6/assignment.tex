\documentclass[reqno]{amsart}
\usepackage{amscd, amssymb, amsmath, amsthm}
\usepackage{graphicx}
\usepackage[colorlinks=true,linkcolor=blue]{hyperref}
\usepackage[utf8]{inputenc}
\usepackage[T1]{fontenc}
\usepackage{textcomp}
\usepackage{babel}
%% for identity function 1:
\usepackage{bbm}
%%For category theory diagrams:
\usepackage{tikz-cd}

\usepackage[backend=biber]{biblatex}
\addbibresource{assignment.bib}


\setlength\parindent{0pt}

\pdfsuppresswarningpagegroup=1

\newtheorem{theorem}{Theorem}[section]
\newtheorem{lemma}[theorem]{Lemma}
\newtheorem{proposition}[theorem]{Proposition}
\newtheorem{corollary}[theorem]{Corollary}
\newtheorem{conjecture}[theorem]{Conjecture}

\theoremstyle{definition}
\newtheorem{definition}[theorem]{Definition}
\newtheorem{example}[theorem]{Example}
\newtheorem{exercise}[theorem]{Exercise}
\newtheorem{problem}[theorem]{Problem}
\newtheorem{question}[theorem]{Question}

\theoremstyle{remark}
\newtheorem*{remark}{Remark}
\newtheorem*{note}{Note}
\newtheorem*{solution}{Solution}



%Inequalities
\newcommand{\cycsum}{\sum_{\mathrm{cyc}}}
\newcommand{\symsum}{\sum_{\mathrm{sym}}}
\newcommand{\cycprod}{\prod_{\mathrm{cyc}}}
\newcommand{\symprod}{\prod_{\mathrm{sym}}}

%Linear Algebra

\DeclareMathOperator{\Span}{span}
\DeclareMathOperator{\im}{im}
\DeclareMathOperator{\diag}{diag}
\DeclareMathOperator{\Ker}{Ker}
\DeclareMathOperator{\ob}{ob}
\DeclareMathOperator{\Hom}{Hom}
\DeclareMathOperator{\Mor}{Mor}
\DeclareMathOperator{\sk}{sk}
\DeclareMathOperator{\Vect}{Vect}
\DeclareMathOperator{\Set}{Set}
\DeclareMathOperator{\Group}{Group}
\DeclareMathOperator{\Ring}{Ring}
\DeclareMathOperator{\Ab}{Ab}
\DeclareMathOperator{\Top}{Top}
\DeclareMathOperator{\hTop}{hTop}
\DeclareMathOperator{\Htpy}{Htpy}
\DeclareMathOperator{\Cat}{Cat}
\DeclareMathOperator{\CAT}{CAT}
\DeclareMathOperator{\Cone}{Cone}
\DeclareMathOperator{\dom}{dom}
\DeclareMathOperator{\cod}{cod}
\DeclareMathOperator{\Aut}{Aut}
\DeclareMathOperator{\Mat}{Mat}
\DeclareMathOperator{\Fin}{Fin}
\DeclareMathOperator{\rel}{rel}
\DeclareMathOperator{\Int}{Int}
\DeclareMathOperator{\sgn}{sgn}
\DeclareMathOperator{\Homeo}{Homeo}
\DeclareMathOperator{\SHomeo}{SHomeo}
\DeclareMathOperator{\PSL}{PSL}
\DeclareMathOperator{\Bil}{Bil}
\DeclareMathOperator{\Sym}{Sym}
\DeclareMathOperator{\Skew}{Skew}
\DeclareMathOperator{\Alt}{Alt}
\DeclareMathOperator{\Quad}{Quad}
\DeclareMathOperator{\Sin}{Sin}
\DeclareMathOperator{\Supp}{Supp}
\DeclareMathOperator{\Char}{char}
\DeclareMathOperator{\Teich}{Teich}
\DeclareMathOperator{\GL}{GL}
\DeclareMathOperator{\tr}{tr}
\DeclareMathOperator{\codim}{codim}
\DeclareMathOperator{\coker}{coker}
\DeclareMathOperator{\corank}{corank}
\DeclareMathOperator{\rank}{rank}
\DeclareMathOperator{\Diff}{Diff}
\DeclareMathOperator{\Bun}{Bun}
\DeclareMathOperator{\Sm}{Sm}



%Row operations
\newcommand{\elem}[1]{% elementary operations
\xrightarrow{\substack{#1}}%
}

\newcommand{\lelem}[1]{% elementary operations (left alignment)
\xrightarrow{\begin{subarray}{l}#1\end{subarray}}%
}

%SS
\DeclareMathOperator{\supp}{supp}
\DeclareMathOperator{\Var}{Var}

%NT
\DeclareMathOperator{\ord}{ord}

%Alg
\DeclareMathOperator{\Rad}{Rad}
\DeclareMathOperator{\Jac}{Jac}

%Misc
\newcommand{\SL}{{\mathrm{SL}}}
\newcommand{\mobgp}{{\mathrm{PSL}_2(\mathbb{C})}}
\newcommand{\id}{{\mathrm{id}}}
\newcommand{\MCG}{{\mathrm{MCG}}}
\newcommand{\PMCG}{{\mathrm{PMCG}}}
\newcommand{\SMCG}{{\mathrm{SMCG}}}
\newcommand{\ud}{{\mathrm{d}}}
\newcommand{\Vol}{{\mathrm{Vol}}}
\newcommand{\Area}{{\mathrm{Area}}}
\newcommand{\diam}{{\mathrm{diam}}}
\newcommand{\End}{{\mathrm{End}}}


\newcommand{\reg}{{\mathtt{reg}}}
\newcommand{\geo}{{\mathtt{geo}}}

\newcommand{\tori}{{\mathcal{T}}}
\newcommand{\cpn}{{\mathtt{c}}}
\newcommand{\pat}{{\mathtt{p}}}

\let\Cap\undefined
\newcommand{\Cap}{{\mathcal{C}}ap}
\newcommand{\Push}{{\mathcal{P}}ush}
\newcommand{\Forget}{{\mathcal{F}}orget}

\title{Assignment 6}
\author{Jonas Trepiakas}
\date{}


\begin{document}
\maketitle
\tableofcontents

\section{Theory and Results}

    \subsection{Jet Bundles}

\begin{definition}[]
   Let $X,Y$ be smooth manifolds and $p \in X$.
   Suppose $f,g \colon X \to Y$ are smooth
   with $f(p) = g(p) = q$.
   \begin{enumerate}
       \item We that $f$ has \textit{first order contact
           with $g$ at $p$ } if $\left( df \right)_p
           = \left( dg \right)_p \colon
           T_p X \to T_q Y$ .
       \item We say that \textit{$f$ has $k$ th order
           contact with $g$ at $p$} if
           $\left( df \right) \colon
           TX \to TY$ has $\left( k-1 \right) $ st
           order contact with $\left( dg \right) $ at
           every point in $T_p X$. This is written
           as $f \sim_k g$ at $p$.
       \item Let $J^{k}(X,Y)_{p,q}$ denote the
           set of equivalence classes under
           " $\sim_k$ at $p$ " of smooth maps
           $f \colon X \to Y$ where
           $f(p) = q$.
       \item Define $J^{k}(X,Y) :=
           \bigcup_{\left( p,q \right) \in X \times Y}
           J^{k}(X,Y)_{p,q}$. An element
           $\sigma \in J^{k}(X,Y)$ is called a
           \textit{$k$-jet of mappings (or just a $k$-jet)
           from $X$ to $Y$}.
       \item Let $\sigma$ be a $k$-jet. Then
           for some $\left( p,q \right) \in X \times Y$,
           $\sigma \in J^{k}(X,Y)_{p,q}$. Then
           $p$ is called the source of $\sigma$ and
           $q$ is called the target of $\sigma$. The
           mapping $\alpha \colon J^{k}(X,Y) \to X$
           given by $\sigma \mapsto \text{source of } \sigma$
           is called the source map and the mapping
           $\beta \colon J^{k}(X,Y) \to Y$ given by
           $\sigma \mapsto \text{target of }\sigma$ is called
           the target map.
   \end{enumerate}
\end{definition}

\begin{definition}[$k$-jet or the $k$-prolongation of a map]
    For a smooth map
    $f \colon X \to Y$, there is a canonically defined map
    $j^{k}f \colon X \to J^{k}(X,Y)$ called the
    $k$-jet of $f$ defined by
    $j^{k}f(p) = \left[ f, p \right] $, the
    equivalence class of $f$ in
    $J^{k}(X,Y)_{p, f(p)}$, for every
    $p \in X$.
\end{definition}

\begin{lemma}[]\label{Jets-Taylor-Series}
    Let $U \subset \mathbb{R}^{n}$ be open and
    $p \in U$. Let $f,g \colon U \to \mathbb{R}^{m}$ be
    smooth. Then
    $f \sim_k g$ at $p$ if and only if
    \[
    \frac{\partial^{\left| \alpha \right| } f_i}{\partial
    x^{\alpha}}(p) = \frac{\partial^{\left| \alpha \right| }
g_i}{\partial x^{\alpha}}(p)
    \]
    for every multi-index $\alpha$ with
    $\left| \alpha \right| \le k$ and
    $1\le y \le m$ where $f_i$ and $g_i$ are the
    coordinate functions determined by
    $f$ and $g$, respectively, and $x_1, \ldots, x_n$
    are coordinates on $U$.
\end{lemma}



\begin{lemma}[]
    Let $U \subset \mathbb{R}^{n}$ and
    $V \subset \mathbb{R}^{m}$ be open.
    Let $f_1,f_2 \colon U \to V$ and
    $g_1,g_2 \colon V \to \mathbb{R}^{l}$ be smooth.
    Let $p \in U$.
    If $f_1 \sim_k f_2$ at $p$ and
    $g_1 \sim_k g_2$ at $q = f_1(p) = f_2(p)$, then
    $g_1 \circ f_1 \sim_k g_2 \circ f_2$ at $p$.
\end{lemma}

\begin{proof}
    We proceed by induction. First, we show the case
    when $k = 1$. In this case, the statement is precisely that
    \[
    d \left( g_1 \circ f_1 \right)_p = d\left( g_2
    \circ f_2\right)_p
    \]
    for all $p \in U$.
    But this is true by the chain rule:
    \[
    d\left( g_1 \circ f_1 \right)_p =
    \left( dg_1 \right)_q \left( df_1 \right)_p
    = \left( dg_2 \right)_q \left( df_2 \right)_p
    = d\left( g_2 \circ f_2 \right)_p.
    \]
    Suppose now the statement is
    true for $k-1$.
    Then since
    $(df_1) \sim_{k-1} \left( df_2 \right) $ at
    $p$ and $\left( dg_1 \right) \sim_{k-1} \left( dg_2 \right) $
    at $q = f_1(p) = f_2(p)$, we have by
    induction that
    \[
        \left( dg_1  \right) \circ \left( df_1 \right)
        \sim_{k-1} \left( dg_2 \right) \circ
        \left( df_2 \right) \quad
        \forall \left( p,v \right) \in \left\{ p \right\} \times
        \mathbb{R}^{n}
    \]
    which by the chain rule is precisely saying that
    \[
    d\left( g_1 \circ f_1 \right) \sim_{k-1}
    d\left( g_2 \circ f_2 \right)
    \]
    for all $(p,v) \in \left\{ p \right\} \times \mathbb{R}^{n}$.
    But this is precisely the definition of
    $g_1 \circ f_1 \sim_k g_2 \circ f_2$ at
    $p$.

\end{proof}

\begin{definition}[]
    Let $X,Y,Z,W$ be smooth manifolds.
    \begin{enumerate}
        \item Let $h \colon Y \to Z$ be smooth.a
            Then $h$ induces a map
            $h_* \colon J^{k}(X,Y) \to
            J^{k}(X,Z)$ as follows:
            if $\left[ f,p \right]  \in J^{k}(X,Y)_{p,q}$, then
            $h_* \left[ f,p \right] =
            \left[ h \circ f, p \right]
            \in J^{k}\left( X,Z \right)_{p,h(q)}$.
        \item If $a \colon Z \to W$ is smooth, then
            $\left( a \circ h \right)_{*}
            = a_* \circ h_*$ and
            $\left( \id_Y \right)_* =
            \id_{J^{k}(X,Y)}$. So if
            $h$ is a diffeomorphism, then
            $h_*$ is a bijection.
        \item Let $g \colon Z \to X$ be a smooth
            diffeomorphism. Then
            $g$ induces a map
            $g^{*} \colon J^{k}(X,Y) \to
            J^{k}(Z,Y)$ by
            $g^{*}\left[ f,p \right] =
            \left[ f \circ g, g^{-1}(p) \right]
            \in J^k (Z,Y)$.
            \item Let $a \colon W \to Z$ be a smooth diffeomorphism.
            Then $\left( g \circ a \right)^{*}=
            a^{*} g^{*}$ and
            $\left( \id_X \right)^{*} =
            \id_{J^{k}(X,Y)}$.
    \end{enumerate}
\end{definition}



Next, let
$A_n^{k}$ be the vector space of polynomials in $n$ variables
of degree $\le k$ which have constant term equal to $0$.
As coordinates for $A_n^{k}$, we can choose the
coefficients of the polynomials.
Let $B_{n,m}^{k} = \oplus_{i=1}^{m} A_n^{k}$.
Both $A_{n}^{k}$ and
$B_{n,m}^{k}$ are smooth manifolds.\\

Let now $U \subset \mathbb{R}^{n}$ be open and
$f \colon U \to \mathbb{R}$ smooth. Define
$T_kf \colon U \to A_n^{k}$ as
$T_kf(x_0)$ being
the $k$th order Taylor polynomial of $f$ at $x_0$ without the
constant term.

Let $V \subset \mathbb{R}^{m}$ be open. There is a canonical
bijection
$T_{U,V} \colon
J^{k}(U,V) \to U \times V \times B_{n,m}^{k}$ given by
\[
T_{U,V} \left( \left[ f,x_0 \right]  \right)
= \left( x_0, f(x_0), T_{k}f_1 (x_0),
\ldots, T_kf_m (x_0) \right) .
\]

This map is well-defined and injective by
Lemma \ref{Jets-Taylor-Series}.

\begin{lemma}[]
    Let $U, U' \subset \mathbb{R}^{n}$ be open and
    $V,V' \subset \mathbb{R}^{m}$ open.
    Suppose $h \colon V \to V'$ is smooth and
    $g \colon U \to U'$ a diffeomorphism. Then
    \[
    T_{U',V'}\left( g^{-1} \right)^{*} h_* T_{U,V}^{-1}\colon
    U \times V \times B_{n,m}^{k} \to
    U' \times V' \times B_{n,m}^{k}
    \]
    is smooth.
\end{lemma}


\begin{definition}[Smooth structure on
    $J^{k}(X,Y)$]
    Let $X,Y$ be smooth manifolds of dimension
    $n$ and $m$, respectively.
    Let $\left( U, \varphi  \right) $ and
    $\left( V,\psi  \right) $ be smooth charts
    in $X$ and $Y$, respectively. Let
    $U' = \varphi (U), V' = \psi (V)$. Then
    let
    $\tau_{U,V} :=
    T_{U', V'} \circ \left( \varphi^{-1} \right)^{*}
    \psi_* \colon
    J^{k}(U,V) \to U' \times V' \times B_{n,m}^{k}$.
    We declare
    $\left( J^{k}(U,V), \tau_{U,V} \right) $ to
    be a chart for
    $J^{k}(X,Y)$. We equip
    $J^{k}(X,Y)$ with the smooth structure induced by
    these smooth charts.\\
    We thus see that
            \[
            \dim J^{k}(X,Y) = m+n+ \dim \left( B_{n,m}^{k} \right)
            \]
\end{definition}


\begin{theorem}[]
    Let $X$ and $Y$ be smooth manifolds with
    $n = \dim X$ and $m = \dim Y$. Then
    \begin{enumerate}
        \item $\alpha \colon J^{k}(X,Y) \to X,
            \beta \colon J^{k}(X,Y) \to Y$ and
            $\alpha \times \beta \colon
            J^{k}(X,Y) \to X \times Y$ are
            submersions.
        \item If $h \colon Y \to Z$ is smooth,
            then $h_* \colon J^{k}(X,Y) \to
            J^{k}(X,Z)$ is smooth.
            If $g \colon X \to Y$ is a diffeomorphism,
            then $g^{*} \colon J^{k}(Y,Z)
            \to J^{k}(X,Z)$ is a diffeomorphism.
        \item If $g \colon X \to Y$ is smooth, then
            $j^{k}g \colon X \to
            J^{k}(X,Y)$ is smooth.
    \end{enumerate}
\end{theorem}


\begin{proof}
    (3) Let $\left( U, \varphi  \right) ,
    \left( V, \psi  \right) $ be charts about
    $x_0$ and $g(x_0)$, respectively.
    Then
    in local coordinates,
    \begin{align*}
    \tau_{U,V} \circ j^{k}g \circ \varphi^{-1}(x)
    &= \tau_{U,V} \left[ g, \varphi^{-1}(x) \right]
    T_{U',V'} \left[ \psi \circ g \circ \varphi^{-1}, x \right] \\
    &=
    \left( x, \psi \circ g \circ \varphi^{-1}(x),
    T_k\left( \psi_1 \circ g \circ \varphi^{-1} \right) (x),
\ldots, T_k \left( \psi_m \circ g \circ \varphi^{-1} \right) (x)
\right)
    \end{align*}
    Now, each
    $T_k \left( \psi_i \circ g \circ \varphi^{-1} \right) $
    is smooth being a sum of partial derivatives of the
    $\psi_i \circ g \circ \varphi^{-1}$
    which are smooth functions between Euclidean spaces.
    Since $j^{k}g$ is locally smooth everywhere, we find
    that it is smooth.
\end{proof}





\subsection{The Whitney $C^{\infty}$ topology (compact-open topology)}

\begin{definition}[Residual, Baire space]
    Let $F$ be a topological space.
    Then
    \begin{enumerate}
        \item A subset $G$ of $F$ is called
            \textit{residual} if it is the countable
            intersection of open dense subsets of $F$.
        \item $F$ is called a \textit{Baire space} if
            every residual set is dense.
    \end{enumerate}
\end{definition}

\begin{proposition}[]\label{Sm-Maps-Mfds-Baire}
    Let $X$ and $Y$ be smooth manifolds. Then
    $C^{\infty}(X,Y)$ is a Baire space in the
    Whitney $C^{\infty}$ topology.
\end{proposition}

\subsection{Transversality}

\begin{definition}[Transversality]
    Let $X$ and $Y$ be smooth manifolds and
    $f \colon X \to Y$ a smooth map. Let
    $W$ be a submanifold of $Y$ and
    $x \in X$. Then $f$ intersects $W$ transversally
    at $x$, denoted by
    $f \pitchfork W $ at $x$, if either
    $f(x) \not\in W$ or
    $f(x) \in W$ and
    $T_{f(x)}Y = T_{f(x)} W \oplus
    \left( df \right)_x \left( T_x X \right) $.
\end{definition}

\begin{proposition}[]\label{Transversality-Dimensions}
    Let $X$ and $Y$ be smooth manifolds,
    $W \subset Y$ a submanifold.
    Suppose $\dim W + \dim X < \dim Y$ (i.e.,
    $\dim X < \codim W$ ). Let
    $f \colon X \to Y$ be smooth and suppose
    $f \pitchfork W$. Then
    $f(X) \cap W = \varnothing$.
\end{proposition}

\begin{proof}
    Exercise.
\end{proof}

\begin{lemma}[]\label{Transversality-Submersion}
    Let $X,Y$ be smooth manifolds and
    $W \subset Y$ a submanifold, and
    $f \colon X \to Y$ smooth. Let
    $p \in X$ and $f(p) \in W$. Suppose there
    exists a neighborhood $U$ of $f(p)$ in
    $Y$ and a submersion $\varphi \colon
    U \to \mathbb{R}^{k}$, where
    $k = \codim W$, such that
    $W \cap U = \varphi^{-1}(0)$. Then
    $f \pitchfork W$ at $p$ if and only if
    $\varphi \circ f$ is a submersion at $p$.
\end{lemma}

\begin{proof}
    We have that since
    $f(p) \in W$,
    $f \pitchfork W$ at $p$ if and only if
    $T_{f(p)} Y = T_{f(p)} W \oplus
    (df)_p \left( T_p X \right) $.
    Since
    $\varphi \left( W \cap U \right)
    = 0$,
    $(d \varphi)_p T_pW = 0$, we have
    $\ker \left( d \varphi  \right)_p
    = T_p W$ for all $ p$.
    Hence $f \pitchfork W$ at $p$ if and only if
    \[
        T_{f(p)} Y =
        \ker \left( d \varphi  \right)_p
        \oplus \left( df \right)_p \left( T_pX \right)
    \]
    but
    $\dim \ker \left( d \varphi  \right)_p
    = \dim T_{f(p)} U - \dim
    \im \left( d \varphi  \right)_p
    = \dim T_{f(p)} Y - k$, so
    $f \pitchfork W$ at $p$ if and only if
    $\dim \left( df \right)_p \left( T_p X \right)
    = k$, so in particular,
    since $\left( d \varphi  \right)_{f(p)}$ is surjective,
    this happens if and only if
    $\dim \left( d \varphi \circ f \right)_p
    \left( T_p X \right) = k$, i.e.,
    $\varphi \circ f$ is a submersion at $p$.




\end{proof}

\begin{theorem}[Thom Transversality Theorem]
    Let $X$ and $Y$ be smooth manifolds and
    $W$ a submanifold of $J^{k}(X,Y)$. Let
    \[
    T_W = \left\{ f \in
    C^{\infty}(X,Y)  \mid j^{k}f \pitchfork W \right\} .
    \]
    Then $T_W$ is a residual subset of
    $C^{\infty}(X,Y)$ in the $C^{\infty}$ topology.
\end{theorem}



\subsubsection{Multijet Spaces}

\begin{definition}[]
Let $X$ and $Y$ be smooth manifolds.
Define
\begin{align*}
    X^{s} &= X \times \ldots \times X\\
    X^{(s)}
    &= \left\{ \left( x_1,\ldots,x_s \right)
    \in X^{s}  \mid x_i \neq x_j, \quad 1 \le i <j \le s\right\} .
\end{align*}
Let
$\alpha \colon J^{k}(X,Y) \to X$ be the source map.
Define
$\alpha^s \colon J^{k}(X,Y)^{s} \to X^{s}$
by $\left( \sigma_1,\ldots, \sigma_s \right)
\mapsto \left( \alpha \sigma_1, \ldots,
\alpha \sigma_s \right) $.
Then
define
$J_s^{k}(X,Y) = \left( \alpha^s \right)^{-1}
\left( X^{(s)} \right) $, called the
$s$-fold $k$-jet bundle. \\
A multijet bundle is some
$s$-fold $k$-jet bundle,
$X^{(s)}$ is a manifold since it is an open
subset of $X^{s}$, so
$J_s^{k}(X,Y)$ is an open subset
of
$J^{k}(X,Y)^{s}$, hence also a smooth manifold.\\
Let
$f \colon X \to Y$ be smooth.
Define
$j_s^{k} f \colon X^{(s)}\to
J_{s}^{k}(X,Y)$ by
\[
j_s^{k}f\left( x_1,\ldots, x_s \right)
= \left( j^{k}f(x_1), \ldots,
j^{k}f(x_s) \right).
\]
\end{definition}


\begin{theorem}[Multijet Transversality Theorem]\label{Multijet-Transversality}
    Let $X$ and $Y$ be smooth manifolds with
    $W$ a submanifold of $J_s^{k}(X,Y)$. Let
    \[
    T_W =
    \left\{ f \in C^{\infty}(X,Y) \mid
    j_s^{k}f \pitchfork W \right\} .
    \]
    Then
    $T_W$ is a residual subset of
    $C^{\infty}(X,Y)$ in the
    $C^{\infty}$ topology.
    Moreover, if $W$ is compact, then
    $T_W$ is open.
\end{theorem}

\subsection{Critical Values and Non-degenerate Critical
Values}

\begin{definition}[]
    Given smooth manifolds $X,Y$, let
    $\sigma = \left[ f,p \right]
    \in J^{1}(X,Y)$. Then define
    $\rank \sigma = \rank \left( df \right)_p$ and
    $\corank \sigma = q - \rank \sigma$ where
    $q = \min \left\{ \dim X, \dim Y \right\} $.\\
    Define
    \[
    S_r = \left\{ \sigma \in J^{1}(X,Y)  \mid
    \corank \sigma = r \right\}
    \]
\end{definition}

Let's use these definitions to
reformulate the definitions of critical points and
degenerate critical points.\\

\begin{lemma}[]
    $p \in X$ is a critical value for
    $f \colon X \to \mathbb{R}$ if and only if
    $\left[ f,p \right] \in S_1$.
\end{lemma}

\begin{proof}
Firstly, for a map $f \colon X \to \mathbb{R}$, a
point $p \in X$ is a critical point if
$(df)_p = 0$. Thus
$\rank j^{1} f = \rank (df)_p = 0$, so
$\corank j^{1}f = 1$. Therefore
 if $p$ is a critical point for  $f$, then
 $\left[ f,p \right] \in S_1$.

 Conversely, if
 $\left[ f,p \right] \in S_1$, then
 $\corank \left[ f,p \right] = 1$, so
 $\rank \left( df \right)_p = 0$, but
 $\left( df \right)_p \colon T_p X \to \mathbb{R}$, so
 having rank $0$ means that it must be the $0$ map, so
 $(df)_p = 0$. Hence $p$ is a critical point. So we find that
 $p \in X$ is a critical point for $f$ if and only if
 $\left[ f,p \right] \in S_1$.
\end{proof}

To relate non-degeneracy, 
we make use of the following proposition:
 \begin{proposition}[]\label{Nondegenerate-Equiv-Jets}
     Let $U \subset \mathbb{R}^{n}$ be open
     and $f \colon U \to \mathbb{R}$ smooth. Then
     a point $p \in U$ is a nondegenerate critical point
     for $f$ if and only if
     $p$ is a critical point and
     $j^{1}f \pitchfork S_1$ at $p$.
 \end{proposition}

 \begin{proof}
     First recall that
     $J^{1}(U,\mathbb{R}) \cong
     U \times \mathbb{R} \times B_{n,1}^{1}$ by
     definition/construction. Now,
     $B_{n,1}^{1} \cong
     \Hom \left( \mathbb{R}^{n},\mathbb{R} \right) $.
     Since
     $T_p J^{1}(U,\mathbb{R}) \cong
     T_{p} \left( U \times \mathbb{R} \times
     \Hom \left( \mathbb{R}^{n},\mathbb{R} \right) \right)
     \cong T_{p_1}U \oplus
     T_{p_2} \mathbb{R} \oplus
     T_{p_3} \Hom\left( \mathbb{R}^{n},\mathbb{R} \right) $,
     we find that the projection
     $\pi \colon J^{1}\left( U,\mathbb{R} \right)
     \to \Hom \left( \mathbb{R}^{n},\mathbb{R} \right) $
     under this identification on tangent spaces
     simply becomes the projection on the
     $T_{p_3}\Hom \left( \mathbb{R}^{n},\mathbb{R} \right) $
     factor, hence $\pi$ is a submersion.
     Furthermore,
     if $\pi (\sigma) = 0$, that means then
     in local coordinates, the first degree
     Taylor expansions without constant term
     of a smooth representative  $f$ for $\pi$ at
     $p$ vanish, so since these determine
     the equivalence class of $\left[ f,p \right] =
     \sigma$, we
     have  $(df)_p = 0$, that is, $\sigma
     \in S_1$.
     Hence $S_1 = \pi^{-1}(0)$. In particular,
     $S_1$ is a submanifold as the preimage of a
     regular value.
     Applying Lemma \ref{Transversality-Submersion},
     $j^{1}f \pitchfork S_1$ at $p$ if and only if
     $\pi \circ j^{1} f$ is a submersion at $p$.
     Now \[
     \pi \circ j^{1} f(x) =
     \left( df \right)_x
     =
     \left( \frac{\partial f}{\partial x_1}(x),
     \ldots, \frac{\partial f}{\partial x_n}(x)\right)
     \]
     so $\pi \circ j^{1}f$ is a submersion at $p$ if
     and only if
     the map
     $\mathbb{R}^{n} \to \mathbb{R}^{n}$
     given by
     \[
     x\mapsto \left( \frac{\partial f}{\partial x_1}(x),
     \ldots, \frac{\partial f}{\partial x_n}(x)\right)
     \] is a submersion at $p$ if and only if
     \[
     \det H (f)_p = \det \left( \frac{\partial^2 f}{
     \partial x_i \partial x_j} (p) \right) \neq 0.
     \]
 \end{proof}


\section{Problems}

\begin{definition}[]
    Let $M$ be a smooth manifold. A Morse function
    $f \colon M \to \mathbb{R}$ is a smooth map such that
    all its critical points are non-degenerate, with
    pairwise distinct critical values in $\mathbb{R}$.
\end{definition}

\subsection{Reeb's Theorem}

\begin{problem}[Reeb's Theorem]\label{Reeb's-Theorem}
        (6 pts) Let $M$ be a smooth, compact manifold of
        dimension $d$. Show that if $M$ admits a Morse
        function with only two critical points, then
        $M$ is homeomorphic to the sphere $S^{d}$. Indicate
        why the above proof fails in showing that $M$ is
        diffeomorphic to the sphere $S^{d}$.
    \end{problem}

    For the proof, we state a theorem that we will need:
    \begin{definition}[]
        For a smooth map $f \colon M \to \mathbb{R}$ on a 
        smooth manifold $M$, let
        $M^{a} = f^{-1} (-\infty, a]$.
    \end{definition}

    \begin{theorem}[]\label{Thm1}
        Let $f \in C^{\infty}(M)$ on a manifold $M$.
        Let $a < b$ and suppose that the set
        $f^{-1}\left[ a,b \right] $ is compact and
        contains no critical points of $f$. Then
        $M^{a}$ is diffeomorphic to $M^{b}$. Furthermore,
        $M^{a}$ is a deformation retract of
        $M^{b}$, so the inclusion $M^{a} \hookrightarrow 
        M^{b}$ is a homotopy equivalence.
    \end{theorem}


    \begin{proof}[Proof of Problem \ref{Reeb's-Theorem}]
        Since $M$ is compact,
        we have that
        $f(M) = \left[ a,b \right] \subset \mathbb{R}$.
        Without loss of generality, assume that
        $f(M) = \left[ 0,1 \right] $.\\

        We shall need the following lemma from
        analysis:
        \begin{lemma}[Fermat's Theorem]
            Let $f \colon \left( a,b \right) \to \mathbb{R}$ 
            be a function on an open interval
            $\left( a,b \right) \subset \mathbb{R}$.
            Suppose $f$ has a local extremum at
            $x_0 \in (a,b)$. If
            $f$ is differentiable at $x_0$, then
            $f'(x_0) = 0$.
        \end{lemma}

        Now, we claim that
        the two critical points are precisely the preimages
        of $0$ and $1$.
        For suppose
        $x \in f^{-1}(0)$.
        Then $x$ is a global minimum for $f$.
        Taking some chart centered around $x$, we have a local
        representation
        of $f$ as a function $\mathbb{R}^{d} \to 
        \left[ 0,1 \right] $ 
        with a global minimum at $0$.
        Taking the partial derivatives of
        $f$ and applying Fermat's theorem to each of them,
        we find that each partial derivative evaluated at $0$ 
        is $0$: $\frac{\partial f}{\partial x^{i}}(0) = 0$.
        Hence we find that
        $Df(0) = 0$, so transfering back to the manifold,
        $Df(x) = 0$, so $x \in M$ is a critical point.
        The same argument applies to show that
        any $y \in f^{-1}(1)$ is a critical point.
        Since there are only two critical points, this
        immediately forces
        $f^{-1}(0)$ and $f^{-1}(1)$ to be singletons
        and thus global maximum and minimum of $M$.
        Suppose without loss of generality that
        $p \in M$ is the minimum and
        $q \in M$ is the maximum.

        By Morse's Lemma, in some coordinate system about
         $p$, let's say in a neighborhood $U$,
         $f$ takes the form
         \[
         f\left( x_1,\ldots,x_n \right) = 
         - x_1^2 - \ldots - x_{\lambda}^2 +
         x_{\lambda+1}^2 +\ldots + x_n^2.
         \] 
         Now $p$ is a global minimum, so in fact, we
         must have that $\lambda = 0$. That is
         \[
         f\left( x_1,\ldots,x_n \right) =
         x_{1}^2 + \ldots + x_n^2
         \] 
         in this neighborhood.
         Since also
         $f(U)$ is open in the subspace topology
         and contains $0$,
         we can find an open disk $\tilde{D}_1$ centered
         at $0$ of radius $\varepsilon_1$ such that
         $ \tilde{D}_1 \cap \left[ 0,1 \right] \subset 
         f(U)$, and
         let $D_1$ be the inverse of $\tilde{D}_1$ under this
         local diffeomorphism.\\
         Similarly, in a neighborhood $V$ of
         $q$, $f$ takes the form
         \[
         f\left( x_1,\ldots,x_n \right)  = 
         1- x_1^2 - x_2^2 - \ldots - x_n^2.
         \] 
         Again take some open disk $\tilde{D}_2$
         centered at $1$ of radius $\varepsilon_2$ 
         such that $ \tilde{D}_2 \cap
         \left[ 0,1 \right] \subset f(V)$.
         Let $D_2$ be the inverse image under $f$ of
         $\tilde{D}_2$.\\
         We wish to show that there
         exists some $\varepsilon > 0$ such that
         $f^{-1}\left[ 0,\varepsilon \right] $ 
         and $f^{-1}\left[ 1-\varepsilon,1 \right] $ are
         homeomorphic to the closed $n$-disk $D^{n}$.
         There
         exist
         $\alpha, \beta \in \left( 0,1 \right) $ such that
         $f\left( M - D_1 \cup D_2 \right) 
         = \left[ \alpha, \beta \right] $ since
         $M - D_1 \cup  D_2$ is still compact.
         Now simply let
         $0 < \varepsilon < \min 
         \left\{ \alpha, 1 - \beta, \varepsilon_1,
         1-\varepsilon_2, 1-\varepsilon_1, \frac{1}{4}\right\} $.
         To see that this works, simply note that
         $f^{-1}\left[ 0, \varepsilon \right] 
         \subset D_1 \cup  D_2$.
         On $D_1$, $f$ takes values in
         $\left[ 0, \varepsilon_1 \right] $ and
         on $D_2$, $f$ takes values in
         $\left[ 1- \varepsilon_2 , 1 \right] $.
         But $\varepsilon < \varepsilon_1$, so
         $\left[ 0, \varepsilon \right] \subset 
         \left[ 0, \varepsilon_1 \right] $, so
         $f^{-1}\left[ 0,\varepsilon \right] \subset 
         D_1$,
         while
         $\varepsilon < 1- \varepsilon_2$, so
         $D_2 \cap
         f^{-1}\left[ 0,\varepsilon \right] = \varnothing $.
         Similarly,
         $1- \varepsilon > \varepsilon_1$, so
         $D_1 \cap
         f^{-1}\left[ 1-\varepsilon,1 \right] = \varnothing$
         while $1-\varepsilon_2 > 1- \varepsilon$, so
         $D_2 
         \subset f^{-1}\left[ 1-\varepsilon,1 \right] $.\\
         \linebreak
         Therefore,
         since $f^{-1}\left[ 0,\varepsilon \right] \subset D_1
         \subset U$ and
         we know that on
         $U$, $f$ takes the form
         \[
         f\left( x_1,\ldots,x_n \right) 
         = x_1^2 + \ldots + x_n^2,
         \] 
         we know that
         $f^{-1}\left[ 0,\varepsilon \right] $ is
         precisely a closed disk about
         $p$. Likewise,
         $f^{-1}\left[ 1-\varepsilon,1 \right] $ can
         be seen to be a closed disk about $q$.\\

         But now by Theorem \ref{Thm1}, since
         there are no critical points in
         $f^{-1}\left[ \varepsilon, 1- \varepsilon \right] $ 
         by assumption,
         $M^{\varepsilon}$ is diffeomorphic to
         $M^{1- \varepsilon}$.
         Hence we find that
         $M^{1-\varepsilon}$ and
         $f^{-1}\left[ 1-\varepsilon,1 \right] $ 
         are both diffeomorphic to closed $d$-disks, and
         furthermore,
         $M$ is obtained by gluing these  $d$-disks along their
         boundary which
         is homeomorphic to
         $S^{d-1}$. We claim that this is sufficient
         to conclude that $M$ is \textit{homeomorphic} to
         $S^{d}$.
         The problem is that while
         we have individual diffeomorphisms
         $M^{1-\varepsilon} \cong
         D^{n}$ and
         $f^{-1}\left[ 1-\varepsilon,1 \right] 
         \cong D^{n}$, the identifications of the boundaries
         might not be preserved under these diffeomorphisms,
         so we might not be able to reglue after.
         Let
         $\varphi_1 \colon M^{1- \varepsilon}
         \cong D^{d}$ and
         $\varphi_2 \colon
         f^{-1}\left[ 1-\varepsilon,1 \right] 
         \cong D^{d}$ be the diffeomorphisms.
         Then 
         $\varphi_1 \circ \varphi_2^{-1}$ is a diffeomorphism
         of $S^{d-1}$, and
         \[
         M \cong D^{d} \sqcup_{\varphi_1 \circ \varphi_2^{-1}}
         D^{d}.
         \] 
         We construct a homeomorphism
         $\psi \colon D_1 \sqcup_{\id} D_2 \to 
         D^{d} \sqcup_{\varphi_1 \circ \varphi_2^{-1}}
         D^{d}$ by
         \[
         \psi (x)
         =
         \begin{cases}
             x&, x \in D_1\\
             0&, x \in D_2 \text{ and } x = 0\\
             \|x\| \varphi_1 \circ
             \varphi_2^{-1} \left( \frac{x}{\|x\|} \right) ,&
             x \in D_2 - \left\{ 0 \right\} 
         \end{cases}
         \] 
         As the sphere is compact and
          the twisted sphere Hausdorff, this
          map is a homeomorphism.\\
          The reason it might fail to be a diffeomorphism, is
          that on $D_2 - \left\{ 0 \right\} $, as
          we  let $x$ approach $0$, we might have
          non-agreeing derivatives from different directions.

    \end{proof}


    \subsection{Existence of Morse functions}



 \begin{problem}[Existence of Morse functions]
     Show that any smooth manifold admits a Morse function.
 \end{problem}

 \begin{proof}
     
 The proof of this problem will
 consist of first showing that
 the set of Morse functions is an open dense
 subset of $C^{\infty}(M,\mathbb{R})$. We will
 thereafter intersect this set with another residual
 set in $C^{\infty}(M,\mathbb{R})$ which will force
 critical values to be distinct. Then
 we will finish the problem by making use of
 $C^{\infty}(M,\mathbb{R})$ being a Baire space in
 the Whitney $C^{\infty}$ topology when
 $M$ is a manifold.



 \begin{theorem}[]\label{Morse-Functions-Open-Dense}
         Let $M$ be a manifold. The set of Morse
         functions is an open dense subset of
         $C^{\infty}(M,\mathbb{R})$.
     \end{theorem}

 \begin{proof}


     Recall that $S_1 $ is a submanifold of
     $J^{1}(M, \mathbb{R})$.
     Hence
     \[
     T_{S_1} =
     \left\{ f \in C^{\infty}(M,\mathbb{R})
      \mid j^{1}f \pitchfork S_1 \right\}
     \]
     is a residual subset of
     $C^{\infty}(X,Y)$ in the $C^{\infty}$ topology.

     By Theorem \ref{Nondegenerate-Equiv-Jets},
     $j^{1}f \pitchfork S_1$ if and only if
     for all points
     $x \in X$, either
     $j_1f (x) \not\in S_1$ or
     $j_1f(x) \in S_1$ and
     $j_1f \pitchfork S_1$ at $x$.
     If $j_1f(x) \not\in S_1$, then
     $x$ is not a critical value of $f$.
     If $j_1f(x) \in S_1$, then
     $x$ is a critical value. Then
     $j_1f \pitchfork S_1$ at $x$ precisely means
     that $x$ is a nondegenerate critical point.
     Hence
     $T_{S_1}$ precisely consists of all
     smooth maps $M \to \mathbb{R}$ which are Morse functions
     (not necessarily distinct critical values).

     But by Proposition \ref{Sm-Maps-Mfds-Baire},
     $C^{\infty}(X,Y)$ is a Baire space in the
     Whitney $C^{\infty}$ topology when
     $X$ and $Y$ are manifolds, so by
     definition, every residual set is dense. Hence
     $T_{S_1}$ is dense in
     $C^{\infty}(M,\mathbb{R})$. Since
     $0$ is an element, it is in particular nonempty.

 \end{proof}

 \begin{theorem}[]
     Let $M$ be a smooth manifold.
     The set of Morse functions all of whose
     crtiical values are distinct form a residual set
     in $C^{\infty}(M, \mathbb{R})$
 \end{theorem}

 \begin{proof}
     Let
     $S = \left( S_1 \times S_1 \right) \cap
     J_2^{1}(M,\mathbb{R}) \cap
     \left( \beta^2 \right)^{-1}
     \left( \Delta \mathbb{R} \right) $.
     We claim that $S$ is a submanifold of the
     multijet bundle
     $J_2^{1}(M,\mathbb{R})$.
     It suffices to check that it is locally
     a submanifold.
     Let $U$ be an open coordinate
     neighborhood in $M$ diffeomorphic to
     $\mathbb{R}^{n}$. Recall that
     $J^{1}(U,\mathbb{R}) \cong
     U \times \mathbb{R} \times B_{n,1}^{1}
     \cong \mathbb{R} \times \mathbb{R} \times
     \Hom \left( \mathbb{R}^{n},1 \right) $, so
     seeing as the coordinates on
     $J_1^2 (X,Y)$ are inherited from the product smooth
     structure and that of an open subset of a smooth
     manifold, we find
     $J_1^2(U, \mathbb{R}) \cong
     \left( \mathbb{R}^{n} \times \mathbb{R}^{n}
     - \Delta\mathbb{R}^{n} \right) \times
     \left( \mathbb{R} \times \mathbb{R} \right)
     \times \Hom\left(\mathbb{R}^{n},\mathbb{R}  \right)^2 $.
     Inserting this in the expression for
     $S$ and noting that
     $\left( \beta^2 \right)^{-1}
     \left( \Delta \mathbb{R} \right) $ means
     that the codomain coordinates must be the same,
     so $\left( \mathbb{R} \times \mathbb{R} \right) $ is
     replaced by $\Delta \mathbb{R}$, and
     intersecting with $\left( S_1 \times S_1 \right) $ means
     that the coordinates for the
     partial derivatives all vanish, so
     $\Hom \left( \mathbb{R}^{n},\mathbb{R} \right)^2$ reduces
     to $\left( 0,0 \right) $. So we get
     \[
     S \cong \left( \mathbb{R}^{n} \times \mathbb{R}^{n}
     - \Delta \mathbb{R}^{n} \right) \times
     \Delta \mathbb{R} \times \left( 0,0 \right)
     \]
     which indeed is a submanifold of
     \[
         J_1^2 \left( U, \mathbb{R} \right)
         \cong
         \left( \mathbb{R}^{n} \times \mathbb{R}^{n}
         - \Delta \mathbb{R}^{n} \right) \times
         \left( \mathbb{R} \times \mathbb{R} \right)
         \times \Hom \left( \mathbb{R}^{n},\mathbb{R} \right)^2.
     \]
     Since $S$ is locally a submanifold of
     $J_2^{1}(M,\mathbb{R})$ at each point, it is a submanifold.
     Moreover,
     $\codim S = 2n+1$ where $n = \dim M$: since indeed
     $\dim J_1^2 \left( U,\mathbb{R} \right)
     = 2n - 1 + 2 + 2n$ and
     $\dim S = 2n-1 + 1$.\\
     Now applying the Multijet Transversality Theorem
 (Theorem \ref{Multijet-Transversality}), we obtain that
  $T_{S} =
  \left\{ f \in C^{\infty}(M, \mathbb{R}) \mid
  j_2^{1} f \pitchfork S \right\} $ is
  residual in
  $C^{\infty}(M, \mathbb{R})$ equipped with the
  $C^{\infty}$ topology.\\


     But by Proposition \ref{Sm-Maps-Mfds-Baire},
     $C^{\infty}(X,Y)$ is a Baire space in the
     Whitney $C^{\infty}$ topology when
     $X$ and $Y$ are manifolds, so by
     definition, every residual set is dense. Hence
     $T_{S}$ is dense in
     $C^{\infty}(M,\mathbb{R})$. Since
     $0$ is an element, it is in particular nonempty.

  Now, if $f \colon M \to \mathbb{R}$ is a smooth map.
  Then $j_2^{1}f \colon
  M^{(s)} \to J_2^{1}(M, \mathbb{R})$.
  In particular,
  suppose that $j_2^{1}f \pitchfork S$, then since
  $\codim S = 2n+1$, while
  $\dim M^{(2)} =
  \dim M \times M - \Delta M= 2n-1 $, we obtain immediately from
  Proposition \ref{Transversality-Dimensions} that
  $j_2^{1}f( M \times M - \Delta M) \cap S = \varnothing$.

  So if $p,q$ are critical points of
  $f$, the fact that
  $j_2^{1}f(p,q) \not\in  S$ means that
  since $\left( j^{1}f(p), j^{1}f(q) \right)
  \in S_1 \times S_1 \cap J_2^{1}(M, \mathbb{R})$, it
  must be the failure of being in
  $\left( \beta^2 \right)^{-1}\left( \Delta \mathbb{R} \right) $
  that prevents $j_2^{1}f
  \left( M \times M- \Delta M \right) $  from intersecting
  $S$. I.e., the targets are not equal:
  $f(p) \neq f(q)$. Since $p,q$ were arbitrary
  critical values,
  the critical values of any
  $f \in T_S$ are thus pairwise distinct.\\

  Now taking the set
  $T_S$ and $T_{S_1}$ from Theorem \ref{Morse-Functions-Open-Dense},
  since $T_{S_1}$ was shown to be an open dense
  subset of $C^{\infty}(M,\mathbb{R})$, and
   $T_S$ was just shown to be
   residual in $C^{\infty}(M,\mathbb{R})$, i.e.,
   the countable intersection of open dense subsets
   of $C^{\infty}(M,\mathbb{R})$, we find that
   $T_S \cap T_{S_1}$ is the countable intersection of
   open dense subsets of $C^{\infty}(M,\mathbb{R})$ also,
   hence residual in $C^{\infty}(M,\mathbb{R})$.
   From Proposition \ref{Sm-Maps-Mfds-Baire}, we now obtain
   that $T_S \cap T_{S_1}$ is dense in
   $C^{\infty}(M,\mathbb{R})$, giving us
   the collection we wanted.






 \end{proof}

 This completes the proof.

 \end{proof}




    \subsection{On the Transversality Theorem}

    \begin{problem}[On the transversality theorem]
        Let $M$ be a smooth manifold.
        \begin{enumerate}
            \item Let $X \subset M$ be a smooth
                submanifold, and let $f \colon Y \to M$ 
                be a smooth map, where
                $Y$ is a smooth manifold. Show that
                $f$ is smoothly homotopic to a map that
                intersects $X$ transversally at every
                point.
            \item Show that in the above setting, if
                $f \colon Y \to M$ intersects $X$ transversally,
                then $f^{-1}(X)$ is a smooth submanifold
                of $Y$ such that $\dim Y + \dim
                f^{-1}(X) = \dim X$.
        \end{enumerate}
    \end{problem}

    \begin{proof}
        (2) We were given the following lemma in class:
        \begin{lemma}[]
            If $f \colon X \to Z$ and
            $g \colon Y \to Z $ are smooth maps between manifolds
            and
            $f \pitchfork g$, then
            the pullback exists:
            \begin{equation*}
            \begin{tikzcd}
                \exists W \ar[r] \ar[d] \ar[dr, phantom,
                "\lrcorner", very near start]
                & Y \ar[d, "g"] \\
                X \ar[r, "f"'] & Z
            \end{tikzcd}
            \end{equation*}
            
        \end{lemma}
            Saying that $f \colon Y \to M$ intersects
            $X$ transversally then amounts to
            $f \colon Y \to M$ and
            $\iota \colon X \hookrightarrow M$ intersecting
            transversally, so the following pullback
            can be completed:

            \begin{equation*}
            \begin{tikzcd}
                \exists W \ar[r] \ar[d] \ar[dr, phantom,
                "\lrcorner", very near start]
                & X \ar[d, "\iota"] \\
                Y \ar[r, "f"'] & M
            \end{tikzcd}
            \end{equation*}

            In particular, $W$ is the fiber
            product
            $X \times_{M} Y = 
            \left\{ 
            (x,y)  \mid 
        \iota(x) = f(y) \right\} 
        \cong f^{-1}(X)$.
        Thus $f^{-1}(X)$ is a
        smooth manifold.\\

        I didn't get to the dimension part in time.




    \end{proof}



\printbibliography
\end{document}
