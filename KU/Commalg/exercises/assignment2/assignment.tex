\documentclass[reqno]{amsart}
\usepackage{amscd, amssymb, amsmath, amsthm}
\usepackage{graphicx}
\usepackage[colorlinks=true,linkcolor=blue]{hyperref}
\usepackage[utf8]{inputenc}
\usepackage[T1]{fontenc}
\usepackage{textcomp}
\usepackage{babel}
%% for identity function 1:
\usepackage{bbm}
%%For category theory diagrams:
\usepackage{tikz-cd}


\setlength\parindent{0pt}

\pdfsuppresswarningpagegroup=1

\newtheorem{theorem}{Theorem}[section]
\newtheorem{lemma}[theorem]{Lemma}
\newtheorem{proposition}[theorem]{Proposition}
\newtheorem{corollary}[theorem]{Corollary}
\newtheorem{conjecture}[theorem]{Conjecture}

\theoremstyle{definition}
\newtheorem{definition}[theorem]{Definition}
\newtheorem{example}[theorem]{Example}
\newtheorem{exercise}[theorem]{Exercise}
\newtheorem{problem}[theorem]{Problem}
\newtheorem{question}[theorem]{Question}

\theoremstyle{remark}
\newtheorem*{remark}{Remark}
\newtheorem*{note}{Note}
\newtheorem*{solution}{Solution}



%Inequalities
\newcommand{\cycsum}{\sum_{\mathrm{cyc}}}
\newcommand{\symsum}{\sum_{\mathrm{sym}}}
\newcommand{\cycprod}{\prod_{\mathrm{cyc}}}
\newcommand{\symprod}{\prod_{\mathrm{sym}}}

%Linear Algebra

\DeclareMathOperator{\Span}{span}
\DeclareMathOperator{\im}{im}
\DeclareMathOperator{\diag}{diag}
\DeclareMathOperator{\Ker}{Ker}
\DeclareMathOperator{\ob}{ob}
\DeclareMathOperator{\Hom}{Hom}
\DeclareMathOperator{\Mor}{Mor}
\DeclareMathOperator{\sk}{sk}
\DeclareMathOperator{\Vect}{Vect}
\DeclareMathOperator{\Set}{Set}
\DeclareMathOperator{\Group}{Group}
\DeclareMathOperator{\Ring}{Ring}
\DeclareMathOperator{\Ab}{Ab}
\DeclareMathOperator{\Top}{Top}
\DeclareMathOperator{\hTop}{hTop}
\DeclareMathOperator{\Htpy}{Htpy}
\DeclareMathOperator{\Cat}{Cat}
\DeclareMathOperator{\CAT}{CAT}
\DeclareMathOperator{\Cone}{Cone}
\DeclareMathOperator{\dom}{dom}
\DeclareMathOperator{\cod}{cod}
\DeclareMathOperator{\Aut}{Aut}
\DeclareMathOperator{\Mat}{Mat}
\DeclareMathOperator{\Fin}{Fin}
\DeclareMathOperator{\rel}{rel}
\DeclareMathOperator{\Int}{Int}
\DeclareMathOperator{\sgn}{sgn}
\DeclareMathOperator{\Homeo}{Homeo}
\DeclareMathOperator{\SHomeo}{SHomeo}
\DeclareMathOperator{\PSL}{PSL}
\DeclareMathOperator{\Bil}{Bil}
\DeclareMathOperator{\Sym}{Sym}
\DeclareMathOperator{\Skew}{Skew}
\DeclareMathOperator{\Alt}{Alt}
\DeclareMathOperator{\Quad}{Quad}
\DeclareMathOperator{\Sin}{Sin}
\DeclareMathOperator{\Supp}{Supp}
\DeclareMathOperator{\Char}{char}
\DeclareMathOperator{\Teich}{Teich}
\DeclareMathOperator{\GL}{GL}
\DeclareMathOperator{\tr}{tr}
\DeclareMathOperator{\codim}{codim}


%Row operations
\newcommand{\elem}[1]{% elementary operations
\xrightarrow{\substack{#1}}%
}

\newcommand{\lelem}[1]{% elementary operations (left alignment)
\xrightarrow{\begin{subarray}{l}#1\end{subarray}}%
}

%SS
\DeclareMathOperator{\supp}{supp}
\DeclareMathOperator{\Var}{Var}

%NT
\DeclareMathOperator{\ord}{ord}

%Alg
\DeclareMathOperator{\Rad}{Rad}
\DeclareMathOperator{\Jac}{Jac}

%Misc
\newcommand{\SL}{{\mathrm{SL}}}
\newcommand{\mobgp}{{\mathrm{PSL}_2(\mathbb{C})}}
\newcommand{\id}{{\mathrm{id}}}
\newcommand{\MCG}{{\mathrm{MCG}}}
\newcommand{\PMCG}{{\mathrm{PMCG}}}
\newcommand{\SMCG}{{\mathrm{SMCG}}}
\newcommand{\ud}{{\mathrm{d}}}
\newcommand{\Vol}{{\mathrm{Vol}}}
\newcommand{\Area}{{\mathrm{Area}}}
\newcommand{\diam}{{\mathrm{diam}}}
\newcommand{\End}{{\mathrm{End}}}


\newcommand{\reg}{{\mathtt{reg}}}
\newcommand{\geo}{{\mathtt{geo}}}

\newcommand{\tori}{{\mathcal{T}}}
\newcommand{\cpn}{{\mathtt{c}}}
\newcommand{\pat}{{\mathtt{p}}}

\let\Cap\undefined
\newcommand{\Cap}{{\mathcal{C}}ap}
\newcommand{\Push}{{\mathcal{P}}ush}
\newcommand{\Forget}{{\mathcal{F}}orget}



\title{Assignment 2}
\author{Jonas Trepiakas - hvn548}
\date{}

\begin{document}
\maketitle
    \begin{exercise}[]
        Given two local rings $\left( R,
        \mathfrak{m} \right) $ and
        $\left( S, \mathfrak{n} \right) $ be local rings.
        A ring homomorphism
        $f \colon R \to S$ is called a local
        homomorphism if the image of
        $\mathfrak{m}$ under $f$ is contained in
        $\mathfrak{n}$.
        \begin{enumerate}
            \item Let $g \colon A \to B$ be a ring
                homomorphism between two arbitrary
                rings and let 
                $\mathfrak{p} \subset B$ and
                $\mathfrak{q} = g^{-1}(\mathfrak{p}) \subset A$ be
                prime ideals. Show that
                $g$ localizes to a ring homomorphism
                $A_{\mathfrak{q}} \to 
                B_{\mathfrak{p}}$ to be more precise, let
                $\pi_{A} \colon A \to A_{\mathfrak{q}}$ 
                and $\pi_{B} \colon B \to 
                B_{\mathfrak{p}}$ be the natural ring
                homomorphisms from the
                rings to their localizations. You need
                to construct a ring homomorphism
                $g' \colon
                A_{\mathfrak{q}} \to 
                B_{\mathfrak{p}}$ such that
                $\pi_{B} \circ g = g' \circ \pi_A$. Show that
                the map $g'$ you construct is a local
                homomorphism.
            \item Find a ring homomorphims between
                local rings which is not a local
                homomorphism.
        \end{enumerate}
    \end{exercise}

\begin{proof}
    (1) Consider the diagram

    \begin{equation*}
    \begin{tikzcd}
        A \ar[d, "\pi_A"'] \ar[r, "g"] & B \ar[d, "\pi_B"] \\
        A_{\mathfrak{q}} \ar[r, "g'"', dashed] & B_{\mathfrak{p}}
    \end{tikzcd}
    \end{equation*}
    Note that
    by the universal property of localizations, 
    the map $g'$ exists if and only if
    $\pi_B \circ g \left( \left( A- \mathfrak{q} \right) 
    \right) \subset 
    B_{\mathfrak{p}}^{\times }
    = \left( R - \mathfrak{p} \right)_{\mathfrak{p}}$.
    Let 
    $a \in A - \mathfrak{q}$. By assumption, 
    $g (a) \not\in \mathfrak{p}$, so
    $\pi_B \left( g (a) \right) = \frac{g(a)}{1}
    \in \left( R - \mathfrak{p} \right)_{\mathfrak{p}} $.
    This gives the existence of $g'$.\\
    Next we show that
    $g'$ is a local homomorphism. 
    Note that $A_{\mathfrak{q}}$ and
    $B_{\mathfrak{p}}$ are local rings
    with unique maximal ideals
    $\mathfrak{q}_{\mathfrak{q}}$ and
    $\mathfrak{p}_{\mathfrak{p}}$, respectively. Thus, to
    show that $g'$ is a local homomorphism, we must show
    that $g' \left( \mathfrak{q}_{\mathfrak{q}} \right) 
    \subset \mathfrak{p}_{\mathfrak{p}}$.
    Explicitly, 
    \begin{align*}
        \mathfrak{q}_{\mathfrak{q}} &=
        \left\{ \frac{a}{b} \in A_{\mathfrak{q}} \colon
        a \in \mathfrak{q} \right\} \\
        \mathfrak{p}_{\mathfrak{p} } &=
        \left\{ \frac{a}{b} \in B_{\mathfrak{p}} \colon
        a \in \mathfrak{q} \right\} .
    \end{align*}
    Let $x \in 
    \mathfrak{q}_{\mathfrak{q}}$, so
    there exist $a \in \mathfrak{q}$ and
    $b \in A - \mathfrak{q}$ such that
    $x = \frac{a}{b}$.
    Then
    $\frac{a}{b} = \pi_A (a) \pi_A(b)^{-1}$, so
    $g' \left( \frac{a}{b} \right) 
    = g' \circ \pi_A (a) 
    \left( g' \circ \pi_A (b) \right)^{-1}
    = \pi_B \circ g(a) \left( \pi_B \circ g(b) \right)^{-1}
    = \frac{g(a)}{g(b)}$ where
    we can invert by $g(b)$ since
    $g(b) \in B - \mathfrak{p}$ as
    $b \in A - \mathfrak{q}$ and
    $\mathfrak{q} = g^{-1}\left( \mathfrak{p} \right) $.
    Since $g(b) \in B - \mathfrak{p}$ and
    $g\left( \mathfrak{q} \right) 
    \subset \mathfrak{p}$, 
    so $g(a) \in \mathfrak{p}$, we have
    $\frac{g(a)}{g(b)} \in 
    \mathfrak{p}_{\mathfrak{p}}$, so
    $g'(x) \in \mathfrak{p}_{\mathfrak{p}}$, hence
    $g\left( \mathfrak{q}_{\mathfrak{q}} \right) 
    \subset \mathfrak{p}_{\mathfrak{p}}$.
    \\
    \linebreak
    (2) 
The ring $\mathbb{Z}_{(p)}$ is local and 
$\mathbb{Q}$ being a field is also local. However,
the inclusion  $\mathbb{Z}_{(p)} \hookrightarrow
\mathbb{Q}$ is not a local homomorphism since
$\frac{p}{1}$ is not mapped to $0$, for example.
\end{proof}












    %\bibliography{../refs.bib}
\end{document}
