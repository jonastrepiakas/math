\documentclass[reqno]{amsart}
\usepackage{amscd, amssymb, amsmath, amsthm}
\usepackage{graphicx}
\usepackage[colorlinks=true,linkcolor=blue]{hyperref}
\usepackage[utf8]{inputenc}
\usepackage[T1]{fontenc}
\usepackage{textcomp}
\usepackage{babel}
%% for identity function 1:
\usepackage{bbm}
%%For category theory diagrams:
\usepackage{tikz-cd}


\setlength\parindent{0pt}

\pdfsuppresswarningpagegroup=1

\newtheorem{theorem}{Theorem}[section]
\newtheorem{lemma}[theorem]{Lemma}
\newtheorem{proposition}[theorem]{Proposition}
\newtheorem{corollary}[theorem]{Corollary}
\newtheorem{conjecture}[theorem]{Conjecture}

\theoremstyle{definition}
\newtheorem{definition}[theorem]{Definition}
\newtheorem{example}[theorem]{Example}
\newtheorem{exercise}[theorem]{Exercise}
\newtheorem{problem}[theorem]{Problem}
\newtheorem{question}[theorem]{Question}

\theoremstyle{remark}
\newtheorem*{remark}{Remark}
\newtheorem*{note}{Note}
\newtheorem*{solution}{Solution}



%Inequalities
\newcommand{\cycsum}{\sum_{\mathrm{cyc}}}
\newcommand{\symsum}{\sum_{\mathrm{sym}}}
\newcommand{\cycprod}{\prod_{\mathrm{cyc}}}
\newcommand{\symprod}{\prod_{\mathrm{sym}}}

%Linear Algebra

\DeclareMathOperator{\Span}{span}
\DeclareMathOperator{\im}{im}
\DeclareMathOperator{\diag}{diag}
\DeclareMathOperator{\Ker}{Ker}
\DeclareMathOperator{\ob}{ob}
\DeclareMathOperator{\Hom}{Hom}
\DeclareMathOperator{\Mor}{Mor}
\DeclareMathOperator{\sk}{sk}
\DeclareMathOperator{\Vect}{Vect}
\DeclareMathOperator{\Set}{Set}
\DeclareMathOperator{\Group}{Group}
\DeclareMathOperator{\Ring}{Ring}
\DeclareMathOperator{\Ab}{Ab}
\DeclareMathOperator{\Top}{Top}
\DeclareMathOperator{\hTop}{hTop}
\DeclareMathOperator{\Htpy}{Htpy}
\DeclareMathOperator{\Cat}{Cat}
\DeclareMathOperator{\CAT}{CAT}
\DeclareMathOperator{\Cone}{Cone}
\DeclareMathOperator{\dom}{dom}
\DeclareMathOperator{\cod}{cod}
\DeclareMathOperator{\Aut}{Aut}
\DeclareMathOperator{\Mat}{Mat}
\DeclareMathOperator{\Fin}{Fin}
\DeclareMathOperator{\rel}{rel}
\DeclareMathOperator{\Int}{Int}
\DeclareMathOperator{\sgn}{sgn}
\DeclareMathOperator{\Homeo}{Homeo}
\DeclareMathOperator{\SHomeo}{SHomeo}
\DeclareMathOperator{\PSL}{PSL}
\DeclareMathOperator{\Bil}{Bil}
\DeclareMathOperator{\Sym}{Sym}
\DeclareMathOperator{\Skew}{Skew}
\DeclareMathOperator{\Alt}{Alt}
\DeclareMathOperator{\Quad}{Quad}
\DeclareMathOperator{\Sin}{Sin}
\DeclareMathOperator{\Supp}{Supp}
\DeclareMathOperator{\Char}{char}
\DeclareMathOperator{\Teich}{Teich}
\DeclareMathOperator{\GL}{GL}
\DeclareMathOperator{\tr}{tr}
\DeclareMathOperator{\codim}{codim}


%Row operations
\newcommand{\elem}[1]{% elementary operations
\xrightarrow{\substack{#1}}%
}

\newcommand{\lelem}[1]{% elementary operations (left alignment)
\xrightarrow{\begin{subarray}{l}#1\end{subarray}}%
}

%SS
\DeclareMathOperator{\supp}{supp}
\DeclareMathOperator{\Var}{Var}

%NT
\DeclareMathOperator{\ord}{ord}

%Alg
\DeclareMathOperator{\Rad}{Rad}
\DeclareMathOperator{\Jac}{Jac}

%Misc
\newcommand{\SL}{{\mathrm{SL}}}
\newcommand{\mobgp}{{\mathrm{PSL}_2(\mathbb{C})}}
\newcommand{\id}{{\mathrm{id}}}
\newcommand{\MCG}{{\mathrm{MCG}}}
\newcommand{\PMCG}{{\mathrm{PMCG}}}
\newcommand{\SMCG}{{\mathrm{SMCG}}}
\newcommand{\ud}{{\mathrm{d}}}
\newcommand{\Vol}{{\mathrm{Vol}}}
\newcommand{\Area}{{\mathrm{Area}}}
\newcommand{\diam}{{\mathrm{diam}}}
\newcommand{\End}{{\mathrm{End}}}


\newcommand{\reg}{{\mathtt{reg}}}
\newcommand{\geo}{{\mathtt{geo}}}

\newcommand{\tori}{{\mathcal{T}}}
\newcommand{\cpn}{{\mathtt{c}}}
\newcommand{\pat}{{\mathtt{p}}}

\let\Cap\undefined
\newcommand{\Cap}{{\mathcal{C}}ap}
\newcommand{\Push}{{\mathcal{P}}ush}
\newcommand{\Forget}{{\mathcal{F}}orget}




\begin{document}
\section*{Sheet 0}
\begin{exercise}[1]
    \begin{proof}
        \begin{enumerate}
            \item 
        Choose
        $\varphi  \colon \mathbb{Z}/2 \to \mathbb{Z}/4$ by
        $1 \mapsto 2$.
    \item Any map $ \varphi \colon 
        \mathbb{Z} /m \to \mathbb{Z} /n$ in
        $\Ring$ must take $1 \mapsto 1$, and
        is uniquely determined thereby
        since $\varphi (k) =
        \varphi \left( 1 + \ldots+ 1 \right) 
        = \varphi (1) + \ldots + \varphi (1)
        = k$.
        Therefore, $0 = \varphi (m) = 
        m$ in $\mathbb{Z} / n$, so $n  \mid m$.
        And it is clear that if $n  \mid m$, then
        $1 \mapsto 1$ is a well-defined ring
        homomorphism. Thus
        \[
        \Hom_{\Ring}\left( \mathbb{Z}/m,
        \mathbb{Z}/n \right) 
        = 
        \begin{cases}
            1 \mapsto 1,& n  \mid m\\
            \varnothing,& \text{otherwise}
        \end{cases}.
        \] 
    \item We claim that the correspondence
        \begin{align*}
            \Hom_{\Ring}\left( \mathbb{Z}[x], R \right) 
            &\to R\\
            \varphi &\mapsto \varphi (x)
        \end{align*}
        is bijective.
        Indeed, $\varphi (1) = 1$ necessarily,
        so $\varphi (k) = k$ for
        all $k \in \mathbb{Z}$, so we
        simply have
        $\varphi \left( \sum \alpha_i x^{i} \right) 
        = \sum \varphi \left( \alpha_i \right) 
        \varphi (x)^{i} 
        = \sum \alpha_i \varphi (x)^{i}$.
    \item Suppose $\left( \mathbb{Q}/ \mathbb{Z},
        +\right) $ admits a ring structure with
        multiplication $\cdot $. Let
        $\frac{a}{b} \in \mathbb{Q} / \mathbb{Z}$ be
        the unit. Then for any $\frac{x}{y} \in 
        \mathbb{Q} / \mathbb{Z}$
        \[
        \frac{x}{y}
        =\frac{x}{y} \cdot \frac{a}{b}
        = \frac{x}{y}\cdot \left( \frac{1}{b} + \ldots
        + \frac{1}{b}\right) 
        = \frac{x}{y}\cdot \frac{1}{b} + \ldots
        + \frac{x}{y}\cdot \frac{1}{b}.
        \] 
        Therefore, we must have
        $\frac{x}{y} \cdot  \frac{1}{b} =
        \frac{x}{ay}$.
        But then
        \[
        \frac{x}{ay} = 
        \frac{x}{y} \cdot \frac{1}{b}
        \] 
        implies
        \[
        \frac{bx}{ay} = \frac{x}{y} \cdot \frac{1}{b}+
        \ldots + \frac{x}{y} \cdot \frac{1}{b}
        = \frac{bx}{y} \cdot \frac{1}{b}
        \] 
        Hence we get
        \[
        \frac{x}{y} = \frac{x}{y} \cdot  \frac{a}{b}
        = \frac{bx}{ay}
        \] 
        and so  $\frac{a}{b} = 1$ as
        $\frac{x}{y}$ was arbitrary. However,
        $1 = 0$ in $\mathbb{Q} / \mathbb{Z}$, giving
        $\mathbb{Q} / \mathbb{Z} = \left\{ 0 \right\} $,
        which is a contradiction.
    \item By (2), 
        $\Hom_{\Ring}\left( \mathbb{Z}/m, \mathbb{Z}/n \right) $ 
        is either a single map or empty, and as
        the empty set is not a ring, this
        Hom set in general does not admit a ring
        structure - when it does, it must be
        the trivial one.
        \end{enumerate}
    \end{proof}
\end{exercise}


\begin{exercise}[2]
    \begin{enumerate}
        \item $A$ admits a unique $\mathbb{Z}$-algebra
            structure since any ring homomorphism
            $\mathbb{Z} \to A$ is uniquely determined
            by $1 \mapsto 1$.
        \item Take $A = \mathbb{Z}[x]$ and
            $R$ to be any ring with more
            than one element. By exercise 1.(c),
            $\Hom_{\Ring}\left( \mathbb{Z}[x],
            R\right) \cong R$, so
            $R$ admits more than one
            $R$-algebra structure.

            However, these structures could be isomorphic
            in the sense that there
            exists maps $\varphi , \psi  \colon
            R \to R$ with $\varphi \psi = \id = \psi  \varphi $ 
            and composing one algebra structure
            $\mathbb{Z}[x] \to R$ with $\varphi $ gives
            $\psi $ and vice versa. 

            So we must find explicit examples which
            are non-isomorphic. 
            Define 
            $f,g \colon \mathbb{Z}[x] \to \mathbb{Z}/6$ by
            $f(x) = 2$ and
            $g(x) = 3$. Now, there
            is no ring homomorphism
            $\varphi \mathbb{Z} / 6 \to \mathbb{Z}/6$
            such that $\varphi \circ f = g$ since
            $0 = \varphi (0) =\varphi \circ f(3x) = g(3x)
            = 3$ gives a contradiction.
    \end{enumerate}
\end{exercise}

\begin{exercise}[3]
    This is just the 4th isomorphism theorem
    for ideals of rings.

    Define a map
    $\pi \colon \mathcal{A} \to \mathcal{B}$ by
    sending $A \to \overline{A} = A + I$.

    Suppose $\pi(A) = \pi(B)$. Then
    for any $a \in A$, there exists $b \in B$ such that
    $a-b \in I \subset A \cap B$, hence
    $a,b \in A \cap B$. Thus $A \subset B \subset A$, so
    $A = B$. 

    Now, suppose $V \in \mathcal{B}$. Let
    $A = \pi^{-1}(V)$. This is an ideal containing $I$.
    If $a,b \in A$ then $\pi(a), \pi(b) \in V$ so
    $\pi(ab) = \pi(a) \pi(b) \in V$, hence
    $ab \in \pi^{-1}(V)$. Similar closure for the rest.
    And if $r \in R$ then
    $\pi\left( ar \right) =\pi(a) \pi(r) \in V$ as
    $\pi(a) \in V$ and $V$ is an ideal, so
    $ar \in A$, hence $A$ is an ideal. This gives surjectivity.
\end{exercise}

\begin{exercise}[4]
    $\left( 2,x \right) \subset \mathbb{Z}[x]$ is not principle
    as an ideal generated over $\mathbb{Z}$.
    If $\left( 2,x \right) = \left( p(x) \right) $, then
    $p(x)  \mid 2$ implies that that the
    degree of $p$ is $0$. Now
    let $q(x)$ be such that
    $p(x)q(x) = x$ and
     $h(x)$ such that $p(x) h(x) = 2$. Then
     the degree of $2$ is $0$ and that of $q$ is $1$.
     Furthermore, as $p \in \left( 2,x \right) $, we must
     have $p(x) = 2 k(x)$. But this implies
     $2 k(x) q(x) = x$, so
     $2  \mid x$ in $\mathbb{Z}[x]$, which is
     impossible.
\end{exercise}

\begin{exercise}[7]
    \begin{enumerate}
        \item Surjectivity amounts to finding an
            $f \in K \left[ x_1, \ldots, x_n \right] $ 
            such that $f(y) = k$ for some arbitrary $k \in K$.
            Consider the map
            $f(x_1,\ldots,x_n) =
            k + \left( x_1-y_1 \right) \ldots \left( x_n
            -y_n\right) $. Or the constant polynomial at $k$ works
            also. Now,
            $\varphi_y \left( f+g \right) 
            = \left( f+g \right) (y)
            = f(y) + g(y) = \varphi_y(f) + 
            \varphi_y(g)$, and
            $\varphi_y(fg) = \varphi_y(f)\varphi_y(g)$ is
            seen likewise. 

            That it is a homomorphism of
            $K$-algebras (with the
            standard $K$-algebra structure)
            amounts to showing that
            $\varphi_y\left( k  \right) 
            = k $ which is clear.
        \item Let $\varphi  \colon K \left[ x_1, \ldots, x_n
            \right]  \to K$ be a ring homomorphism. 
            Let $y_i = \varphi (x_i)$. Then
            $\varphi \left( \sum a_I x_I \right) 
            = \sum \varphi \left( a_I \right) 
             y_I = 
             \varphi_{y_I}\left( \sum a_I x_I \right) $
             (for this we need
             $\varphi $ to be a $K$-algebra
             homomorphism of with
              $K\left[ x_1, \ldots,x_n \right] $ 
              in the standard structure.
              Is there a different way of arguing?)
    \end{enumerate}
\end{exercise}








    %\bibliography{../refs.bib}
\end{document}
