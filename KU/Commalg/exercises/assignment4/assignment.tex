\documentclass[reqno]{amsart}
\usepackage{amscd, amssymb, amsmath, amsthm}
\usepackage{graphicx}
\usepackage[colorlinks=true,linkcolor=blue]{hyperref}
\usepackage[utf8]{inputenc}
\usepackage[T1]{fontenc}
\usepackage{textcomp}
\usepackage{babel}
%% for identity function 1:
\usepackage{bbm}
%%For category theory diagrams:
\usepackage{tikz-cd}


\setlength\parindent{0pt}

\pdfsuppresswarningpagegroup=1

\newtheorem{theorem}{Theorem}[section]
\newtheorem{lemma}[theorem]{Lemma}
\newtheorem{proposition}[theorem]{Proposition}
\newtheorem{corollary}[theorem]{Corollary}
\newtheorem{conjecture}[theorem]{Conjecture}

\theoremstyle{definition}
\newtheorem{definition}[theorem]{Definition}
\newtheorem{example}[theorem]{Example}
\newtheorem{exercise}[theorem]{Exercise}
\newtheorem{problem}[theorem]{Problem}
\newtheorem{question}[theorem]{Question}

\theoremstyle{remark}
\newtheorem*{remark}{Remark}
\newtheorem*{note}{Note}
\newtheorem*{solution}{Solution}



%Inequalities
\newcommand{\cycsum}{\sum_{\mathrm{cyc}}}
\newcommand{\symsum}{\sum_{\mathrm{sym}}}
\newcommand{\cycprod}{\prod_{\mathrm{cyc}}}
\newcommand{\symprod}{\prod_{\mathrm{sym}}}

%Linear Algebra

\DeclareMathOperator{\Span}{span}
\DeclareMathOperator{\im}{im}
\DeclareMathOperator{\diag}{diag}
\DeclareMathOperator{\Ker}{Ker}
\DeclareMathOperator{\ob}{ob}
\DeclareMathOperator{\Hom}{Hom}
\DeclareMathOperator{\Mor}{Mor}
\DeclareMathOperator{\sk}{sk}
\DeclareMathOperator{\Vect}{Vect}
\DeclareMathOperator{\Set}{Set}
\DeclareMathOperator{\Group}{Group}
\DeclareMathOperator{\Ring}{Ring}
\DeclareMathOperator{\Ab}{Ab}
\DeclareMathOperator{\Top}{Top}
\DeclareMathOperator{\hTop}{hTop}
\DeclareMathOperator{\Htpy}{Htpy}
\DeclareMathOperator{\Cat}{Cat}
\DeclareMathOperator{\CAT}{CAT}
\DeclareMathOperator{\Cone}{Cone}
\DeclareMathOperator{\dom}{dom}
\DeclareMathOperator{\cod}{cod}
\DeclareMathOperator{\Aut}{Aut}
\DeclareMathOperator{\Mat}{Mat}
\DeclareMathOperator{\Fin}{Fin}
\DeclareMathOperator{\rel}{rel}
\DeclareMathOperator{\Int}{Int}
\DeclareMathOperator{\sgn}{sgn}
\DeclareMathOperator{\Homeo}{Homeo}
\DeclareMathOperator{\SHomeo}{SHomeo}
\DeclareMathOperator{\PSL}{PSL}
\DeclareMathOperator{\Bil}{Bil}
\DeclareMathOperator{\Sym}{Sym}
\DeclareMathOperator{\Skew}{Skew}
\DeclareMathOperator{\Alt}{Alt}
\DeclareMathOperator{\Quad}{Quad}
\DeclareMathOperator{\Sin}{Sin}
\DeclareMathOperator{\Supp}{Supp}
\DeclareMathOperator{\Char}{char}
\DeclareMathOperator{\Teich}{Teich}
\DeclareMathOperator{\GL}{GL}
\DeclareMathOperator{\tr}{tr}
\DeclareMathOperator{\codim}{codim}
\DeclareMathOperator{\Diff}{Diff}
\DeclareMathOperator{\Bun}{Bun}


%Row operations
\newcommand{\elem}[1]{% elementary operations
\xrightarrow{\substack{#1}}%
}

\newcommand{\lelem}[1]{% elementary operations (left alignment)
\xrightarrow{\begin{subarray}{l}#1\end{subarray}}%
}

%SS
\DeclareMathOperator{\supp}{supp}
\DeclareMathOperator{\Var}{Var}

%NT
\DeclareMathOperator{\ord}{ord}

%Alg
\DeclareMathOperator{\Rad}{Rad}
\DeclareMathOperator{\Jac}{Jac}

%Misc
\newcommand{\SL}{{\mathrm{SL}}}
\newcommand{\mobgp}{{\mathrm{PSL}_2(\mathbb{C})}}
\newcommand{\id}{{\mathrm{id}}}
\newcommand{\MCG}{{\mathrm{MCG}}}
\newcommand{\PMCG}{{\mathrm{PMCG}}}
\newcommand{\SMCG}{{\mathrm{SMCG}}}
\newcommand{\ud}{{\mathrm{d}}}
\newcommand{\Vol}{{\mathrm{Vol}}}
\newcommand{\Area}{{\mathrm{Area}}}
\newcommand{\diam}{{\mathrm{diam}}}
\newcommand{\End}{{\mathrm{End}}}


\newcommand{\reg}{{\mathtt{reg}}}
\newcommand{\geo}{{\mathtt{geo}}}

\newcommand{\tori}{{\mathcal{T}}}
\newcommand{\cpn}{{\mathtt{c}}}
\newcommand{\pat}{{\mathtt{p}}}

\let\Cap\undefined
\newcommand{\Cap}{{\mathcal{C}}ap}
\newcommand{\Push}{{\mathcal{P}}ush}
\newcommand{\Forget}{{\mathcal{F}}orget}



\title{Assignment 4}
\author{Jonas Trepiakas - hvn548}
\date{}

\begin{document}
\maketitle
    \begin{exercise}[2]
        Which of the following modules are flat over the
        corresponding rings?
        Justify your answer
        \begin{enumerate}
            \item $R = \mathbb{C}\left[ x,y \right] $ and
                the module is
                $I = \left( x,y \right) \subset R$.
            \item $R = \mathbb{C}\left[ x \right] /
                (x^2)$ and the module is
                $I = (x) \subset R$.
            \item $R = \mathbb{C}\left[ x \right] $ and
                the module is the ring
                $\mathbb{C} \left[ y \right] $ considered
                as an $R$-module by ring
                homomorphism $\mathbb{C} \left[ x \right] 
                \to \mathbb{C} \left[ y \right] \colon
                x \mapsto y^2$.
            \item $R = \mathbb{C}\left[ x \right] $ and the
                module is the ring
                $\mathbb{C} \left[ x,y \right] 
                /(xy)$ considered as an
                $R$-module by ring
                homomorphism
                $\mathbb{C} \left[ x \right] \to 
                \mathbb{C} \left[ x,y \right] / (xy) \colon
                x \mapsto x$.
        \end{enumerate}
    \end{exercise}

    \begin{solution}
        (1) We claim that
        $I = \left( x,y \right) $ is not
        a flat $R = \mathbb{C}\left[ x,y \right] $ module.
        Firstly,
        $\mathbb{C} \left[ x,y \right] $ is Noetherian by
        Hilbert's basis theorem since
        $\mathbb{C}$ is, and
        it is also local: we claim that
        $\left( x,y \right) = I$ is precisely the
        maximal ideal. Firstly,
        it is maximal because
        $\mathbb{C} \left[ x,y \right] / \left( x,y \right) 
        \cong \mathbb{C}$ is a field. Now if 
        $M \subset \mathbb{C} \left[ x,y \right] $ is a
        maximal ideal, then $1 \not\in M$, so
        for any $f \in M$, we have that
        $f(x,y) = 
        \sum_{i+j\ge 1} \alpha_{ij} x^{i} y^{j}
        \in \left( x,y \right) $.
        Thus $M \subset (x,y)$, so 
        $M$ is not maximal unless $M = (x,y)$.
        Therefore $(x,y)$ is the only maximal ideal.
        Now, $I$ is finitely generated as
        a $\mathbb{C}\left[ x,y \right] $-module
        with generators
        $x$ and  $y$, hence
        proposition 9.15 applies. Since
        $(x,y)$ is not free since it in particular is a
        proper submodule of $R$, we have that
        it is not flat.\\
        \linebreak
        (2) We claim that
         $I = \left( x \right) \subset 
         \mathbb{C} \left[ x \right] / (x^2)$ is indeed
         flat. This can be seen since
         $\mathbb{C} \left[ x \right] / (x^2) = 
         \mathbb{C} \oplus (x) \cong
         \mathbb{C} \oplus \mathbb{C}$, so
         since $(x)$ is a direct summand of
         $R = \mathbb{C} \left[ x \right] /(x^2)$,
         it is flat by proposition 9.13 and the fact that
         $R$ is itself flat by example 9.2.\\
         \linebreak
         (3) 
         \textbf{I will give two solutions since I'm
         not sure whether I may use that over a PID,
     a module is flat iff it is torsion-free}
     Suppose there
     is a relation
     $\sum a_i y^{i} = 0$ in $\mathbb{C}[y]$ where
     $a_i \in \mathbb{C}[x]$. However,
     then taking the maximal degree of $x^{j}$ in
     $a_j$ for $y^{j}$ the maximal degree
     of $y $ in the relation, we find that
     $a_j = 0$. But this contradicts $y^{j}$ being the
     maximal degree. Hence
     $a_i = 0$ for all $i$. But this
     shows that the relation is trivial. Now
     remark 9.21 tells us that
     $\mathbb{C}[y]$ is flat cosidered as a
     $\mathbb{C}[x]$ module by the homomorphism
     $\mathbb{C}[x] \to \mathbb{C}[y]$ by
     $x \mapsto y^2$.\\
     \linebreak
     The other solution is the following:
         Since $R = \mathbb{C} \left[ x \right] $ is a 
         PID, we immediately find that
         $\mathbb{C} \left[ y \right] $ is
         flat if and only if it is torsion-free considered
         as a  $\mathbb{C}\left[ x \right] $-module by
         restriction of scalars along
         $x \mapsto y^2$. Suppose
         $f(y) \in \mathbb{C}[y]$ is such that
         for  $g(x) \in R$,
         $g(x) f(y) = 0$, i.e.,
         $g(y^2) f(y) = 0$ in $\mathbb{C}[y]$. However, this
         forces $f$ or $g$ to be $0$, so
         we find that
         $\mathbb{C} [y]$ is torsion-free as a
         $\mathbb{C} [x]$ module under the ring-homomorphism
         $x \mapsto y^2$. Thus
         $\mathbb{C}[y]$ is a flat
         $\mathbb{C}[x]$-module by the ring homomorphism
         $\mathbb{C}[x] \to \mathbb{C}[y]$ sending
         $x \mapsto y^2$.\\
         \linebreak
         (4) 
         We note that if a module has torsion, this
         gives an non-trivial relation since
         $am = 0$ with $a \neq 0$  and
         $m \neq 0$
         admitting a genuinely trivial reparametrization
         implies $a = 0$, contradiction. Hence
         proposition 9.20 gives that if a module
         has torsion, then it cannot be flat.
         $\mathbb{C} \left[ x,y \right] /(xy)$ is clearly
         not a flat
         $\mathbb{C} \left[ x \right] $-module under the
         homomorphism
         $\mathbb{C} \left[ x \right] \to 
         \mathbb{C} \left[ x,y \right] /(xy)$ sending
         $x\mapsto x$ since
         $y $ is nonzero in
         $\mathbb{C} \left[ x,y \right] /(xy)$ however,
         $x \cdot y := xy = 0$, hence
         $\mathbb{C} \left[ x,y \right] /(xy)$ is not
         torsion-free over $\mathbb{C}\left[ x \right] $.
    \end{solution}










    %\bibliography{../refs.bib}
\end{document}
