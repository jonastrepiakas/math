\documentclass[reqno]{amsart}
\usepackage{amscd, amssymb, amsmath, amsthm}
\usepackage{graphicx}
\usepackage[colorlinks=true,linkcolor=blue]{hyperref}
\usepackage[utf8]{inputenc}
\usepackage[T1]{fontenc}
\usepackage{textcomp}
\usepackage{babel}
%% for identity function 1:
\usepackage{bbm}
%%For category theory diagrams:
\usepackage{tikz-cd}


\setlength\parindent{0pt}

\pdfsuppresswarningpagegroup=1

\newtheorem{theorem}{Theorem}[section]
\newtheorem{lemma}[theorem]{Lemma}
\newtheorem{proposition}[theorem]{Proposition}
\newtheorem{corollary}[theorem]{Corollary}
\newtheorem{conjecture}[theorem]{Conjecture}

\theoremstyle{definition}
\newtheorem{definition}[theorem]{Definition}
\newtheorem{example}[theorem]{Example}
\newtheorem{exercise}[theorem]{Exercise}
\newtheorem{problem}[theorem]{Problem}
\newtheorem{question}[theorem]{Question}

\theoremstyle{remark}
\newtheorem*{remark}{Remark}
\newtheorem*{note}{Note}
\newtheorem*{solution}{Solution}



%Inequalities
\newcommand{\cycsum}{\sum_{\mathrm{cyc}}}
\newcommand{\symsum}{\sum_{\mathrm{sym}}}
\newcommand{\cycprod}{\prod_{\mathrm{cyc}}}
\newcommand{\symprod}{\prod_{\mathrm{sym}}}

%Linear Algebra

\DeclareMathOperator{\Span}{span}
\DeclareMathOperator{\im}{im}
\DeclareMathOperator{\diag}{diag}
\DeclareMathOperator{\Ker}{Ker}
\DeclareMathOperator{\ob}{ob}
\DeclareMathOperator{\Hom}{Hom}
\DeclareMathOperator{\Mor}{Mor}
\DeclareMathOperator{\sk}{sk}
\DeclareMathOperator{\Vect}{Vect}
\DeclareMathOperator{\Set}{Set}
\DeclareMathOperator{\Group}{Group}
\DeclareMathOperator{\Ring}{Ring}
\DeclareMathOperator{\Ab}{Ab}
\DeclareMathOperator{\Top}{Top}
\DeclareMathOperator{\hTop}{hTop}
\DeclareMathOperator{\Htpy}{Htpy}
\DeclareMathOperator{\Cat}{Cat}
\DeclareMathOperator{\CAT}{CAT}
\DeclareMathOperator{\Cone}{Cone}
\DeclareMathOperator{\dom}{dom}
\DeclareMathOperator{\cod}{cod}
\DeclareMathOperator{\Aut}{Aut}
\DeclareMathOperator{\Mat}{Mat}
\DeclareMathOperator{\Fin}{Fin}
\DeclareMathOperator{\rel}{rel}
\DeclareMathOperator{\Int}{Int}
\DeclareMathOperator{\sgn}{sgn}
\DeclareMathOperator{\Homeo}{Homeo}
\DeclareMathOperator{\SHomeo}{SHomeo}
\DeclareMathOperator{\PSL}{PSL}
\DeclareMathOperator{\Bil}{Bil}
\DeclareMathOperator{\Sym}{Sym}
\DeclareMathOperator{\Skew}{Skew}
\DeclareMathOperator{\Alt}{Alt}
\DeclareMathOperator{\Quad}{Quad}
\DeclareMathOperator{\Sin}{Sin}
\DeclareMathOperator{\Supp}{Supp}
\DeclareMathOperator{\Char}{char}
\DeclareMathOperator{\Teich}{Teich}
\DeclareMathOperator{\GL}{GL}
\DeclareMathOperator{\tr}{tr}
\DeclareMathOperator{\codim}{codim}


%Row operations
\newcommand{\elem}[1]{% elementary operations
\xrightarrow{\substack{#1}}%
}

\newcommand{\lelem}[1]{% elementary operations (left alignment)
\xrightarrow{\begin{subarray}{l}#1\end{subarray}}%
}

%SS
\DeclareMathOperator{\supp}{supp}
\DeclareMathOperator{\Var}{Var}

%NT
\DeclareMathOperator{\ord}{ord}

%Alg
\DeclareMathOperator{\Rad}{Rad}
\DeclareMathOperator{\Jac}{Jac}

%Misc
\newcommand{\SL}{{\mathrm{SL}}}
\newcommand{\mobgp}{{\mathrm{PSL}_2(\mathbb{C})}}
\newcommand{\id}{{\mathrm{id}}}
\newcommand{\MCG}{{\mathrm{MCG}}}
\newcommand{\PMCG}{{\mathrm{PMCG}}}
\newcommand{\SMCG}{{\mathrm{SMCG}}}
\newcommand{\ud}{{\mathrm{d}}}
\newcommand{\Vol}{{\mathrm{Vol}}}
\newcommand{\Area}{{\mathrm{Area}}}
\newcommand{\diam}{{\mathrm{diam}}}
\newcommand{\End}{{\mathrm{End}}}


\newcommand{\reg}{{\mathtt{reg}}}
\newcommand{\geo}{{\mathtt{geo}}}

\newcommand{\tori}{{\mathcal{T}}}
\newcommand{\cpn}{{\mathtt{c}}}
\newcommand{\pat}{{\mathtt{p}}}

\let\Cap\undefined
\newcommand{\Cap}{{\mathcal{C}}ap}
\newcommand{\Push}{{\mathcal{P}}ush}
\newcommand{\Forget}{{\mathcal{F}}orget}




\begin{document}
\begin{exercise}[2]
    Let $R$ be a Noetherian ring. Show that every ideal of
    $R$ is a finite intersection of
    irreducible ideals.
\end{exercise}

\begin{proof}
    Let
    $\mathcal{A}$ be the set of ideals
    of $R$ which are not a finite intersection of
    irreducible ideals. 
    Suppose
    $\left\{ X_i \right\} 
    = \mathcal{X} \subset \mathcal{A}$ is a chain with
    respect to inclusion.
    Then since $R$ is Noetherian, this chain stabilizes, hence
    it has an upper bound. Thus
    $\mathcal{A}$ has a maximal element, call it 
    $M$. Now, $M$ is not a finite intersection
    of irreducible ideals, so in particular,
    $M$ is not irreducible, so write
    $M = I \cap J$ where
    $I$ and $J$ contain $M$ properly. But then
    $I,J \not\in \mathcal{A}$, so
    they are finite intersections of irreducible ideals.
    However, then
    their intersection is also a finite intersection of
    irreducible ideals.
\end{proof}

\begin{exercise}[3]
    Let $R$ be a ring. Show that an ideal
    $\mathfrak{B}$ of $R$ is a prime ideal if and only
    if for all ideals $I$ and $J$ of $R$,
    $IJ \subset \mathfrak{B}$ implies
    $I \subset \mathfrak{B}$ or
    $J \subset \mathfrak{B}$.
\end{exercise}

\begin{proof}
    If $\mathcal{B}$ is a prime ideal, then
    suppose $IJ \subset \mathcal{B}$, and suppose
    $I \not \subset  \mathcal{B}$.
    Then there exists some $i \in I$ such that 
    $i \not\in \mathcal{B}$ but
    $i J \subset \mathcal{B}$.
    Since $\mathcal{B}$ is prime,
    we must have that
    $J \subset \mathcal{B}$.\\
    Conversely, suppose that for all $I,J$,
    $IJ \subset \mathcal{B}$ implies $I \subset \mathcal{B}$ 
    or $J \subset \mathcal{B}$. Let
    $a,b \in R$ such that
    $ab \in \mathcal{B}$. Then

    $\left( a \right)  \left( b \right) 
    \subset \mathcal{B}$, so either
    $\left( a \right)  \subset \mathcal{B}$ or
    $\left( b \right)  \subset \mathcal{B}$, so
    either $a \in \mathcal{B}$ or $b \in \mathcal{B}$.
    So $\mathcal{B}$ is prime.
\end{proof}



\begin{exercise}[4]
    Let $R_S$ be the localization of the integral domain
    $R$ by a multiplicative subset
    $S$ which does not contain $0$. Let
    $\mathfrak{U}$ be a primary ideal of
    $R$. Show that the extension
    $\mathfrak{U}R_S$ is a primary ideal of
    $R_S$ and that $R \cap \mathfrak{U}R_S = 
    \mathfrak{U}$.
\end{exercise}

\begin{proof}
    Let $\tau \colon R \to R_S = \left( R-S \right)^{-1}R$.
    Recall that
    $\mathfrak{U}R_S = \left( \tau \left( 
    \mathfrak{U}\right)  \right) $.
    Suppose
    $ab \in \mathfrak{U}R_S$. Then
    there exists $\sum \alpha_i u_i
    \in \mathfrak{U}$ such that
    $ab = \sum \alpha_i \tau(u_i)
    = \sum \alpha_i \frac{u_i}{1}$.
    We can write
    $a = \frac{x}{y}$ and
    $b = \frac{v}{w}$.
    Then
    $xv \in \mathfrak{U}$, so
    either $x \in \mathfrak{U}$, in which case
    $\frac{x}{y} \in \mathfrak{U}R_S$, or
    $v^{n} \in \mathfrak{U}$, in which case
    $\left( \frac{v}{w} \right)^{n} \in 
    \mathfrak{U}R_S$ since $S$ is multiplicative, so
    $w^{n} \in S$. Thus
    $a \in \mathfrak{U}R_S$ or
    $b^{n} \in \mathfrak{U}R_S$. Hence
    $\mathfrak{U}R_S$ is primary in $R_S$.\\
    Now recall that
    $R \cap \mathfrak{U}R_S$ denote the contraction
    of $\mathfrak{U}R_S$ along $\tau$, i.e.,
    $R \cap \mathfrak{U}R_S = 
    \tau^{-1} \left( \mathfrak{U}R_S \right) $. 
    Clearly,
    $\mathfrak{U} \subset 
    R \cap \mathfrak{U}R_S$.
    Conversely, suppose
    $a \in \tau^{-1}\left( \mathfrak{U}R_S \right) $.
    Then
    $\frac{a}{1} \in \mathfrak{U}R_S$, so
    $\frac{a}{1} = 
    \sum \alpha_i \frac{u_i}{1}$. So
    there exists some
    $r \in R-S$ such that
    $ra = r \sum \alpha_i u_i$. So
    since $r \neq 0$ as $0 \not\in S$, we have
    $a = \sum \alpha_i u_i$ since $R$ is an integral domain.
    Thus $a \in \mathfrak{U}$.
\end{proof}




\begin{exercise}[5]
    Let $R$ be a ring and $\mathfrak{U}$ a 
    $\mathfrak{B}$-primary ideal. Show the following.
    \begin{enumerate}
        \item For all $x \in \mathfrak{U}$, we have
            $\left( \mathfrak{U} \colon x
            \right) = R$.
        \item For all $x \in R - \mathfrak{U}$,
            we have that 
            $\left( \mathfrak{U} \colon x \right) $ is
            $\mathfrak{B}$-primary.
        \item For all $x \in 
            R - \mathfrak{B}$, we have
            that $\left( \mathfrak{U}\colon x \right) 
            = \mathfrak{U}$.
        \item If $R$ is Noetherian, then there is some
            $x \in R - \mathfrak{U}$ such that
            $\left( \mathfrak{U} \colon x \right) 
            = \mathfrak{B}$.
    \end{enumerate}
\end{exercise}

\begin{proof}
    So $\sqrt{\mathfrak{U}} = \mathfrak{B}$ by assumption.\\
    \linebreak
    (1) Since $\mathfrak{U}$ is an ideal,
    $x \mathfrak{U} \subset  \mathfrak{U}$ for
    all $x \in R$, so
     for all $x \in \mathfrak{U}$,
     $\left( \mathfrak{U} \colon x \right) = R$.\\
     (2) Suppose $a \in \sqrt{\left( \mathfrak{U}\colon
     x\right) } $, so
     $a^{n} x \in \mathfrak{U}$. Since $\mathfrak{U}$ is primary,
     either $x \in \mathfrak{U}$, which we have assumed it is
     not, or $a^{nm} \in \mathfrak{U}$ for some $m$. Hence
     the latter must be true. So
     $a \in \sqrt{\mathfrak{U}} = \mathfrak{B}$.\\
     Hence
     $ \sqrt{\left( \mathfrak{U} : x \right) } \subset 
     \mathfrak{B}$.
     Conversely, suppose
     $a \in \mathfrak{B}$, so
     $a^{n} \in \mathfrak{U}$. So for
     any $x \in R - \mathfrak{U}$, we have
     $a^{n} x \in \mathfrak{U}$, so
     $a \in \sqrt{\left( \mathfrak{U} : x \right) } $.\\
     (3) Suppose
     $x \in R - \mathfrak{B}$.
     Now if $y \in \mathfrak{U}$ then
     $xy \in \mathfrak{U}$, so
     $y \in \left( \mathfrak{U} : x \right) $.
     Conversely, suppose
     $y \in \left( \mathfrak{U} : x \right) $.
     So $xy \in \mathfrak{U}$, so
     since  $x \not\in \mathfrak{B} = 
     \sqrt{\mathfrak{U}} $, we must have that
     $y \in \mathfrak{U}$.\\
     (4) Let $n$ be minimal such that
     $\mathfrak{B}^{n} \subset \mathfrak{U}$.
     Let 
     $x \in \mathfrak{B}^{n-1} - \mathfrak{U}$.
     We claim
     $\left( \mathfrak{U} : x \right) = \mathfrak{B}$.
     For $b \in \mathfrak{B}$, we have
     $bx \in \mathfrak{B}^{n} \subset 
     \mathfrak{U}$, so
     $b \in \left( \mathfrak{U} : x \right) $.
     Conversely, if
     $bx \in \mathfrak{U}$, then
     since  $x \not\in U$, we have
     $b^{m} \in \mathfrak{U}$, so
     $b \in \sqrt{U} = \mathfrak{B}$.
\end{proof}





    %\bibliography{../refs.bib}
\end{document}
