\documentclass[reqno]{amsart}
\usepackage{amscd, amssymb, amsmath, amsthm}
\usepackage{graphicx}
\usepackage[colorlinks=true,linkcolor=blue]{hyperref}
\usepackage[utf8]{inputenc}
\usepackage[T1]{fontenc}
\usepackage{textcomp}
\usepackage{babel}
%% for identity function 1:
\usepackage{bbm}
%%For category theory diagrams:
\usepackage{tikz-cd}

%\usepackage[backend=biber]{biblatex}
%\addbibresource{.bib}


\setlength\parindent{0pt}

\pdfsuppresswarningpagegroup=1

\newtheorem{theorem}{Theorem}[section]
\newtheorem{lemma}[theorem]{Lemma}
\newtheorem{proposition}[theorem]{Proposition}
\newtheorem{corollary}[theorem]{Corollary}
\newtheorem{conjecture}[theorem]{Conjecture}

\theoremstyle{definition}
\newtheorem{definition}[theorem]{Definition}
\newtheorem{example}[theorem]{Example}
\newtheorem{exercise}[theorem]{Exercise}
\newtheorem{problem}[theorem]{Problem}
\newtheorem{question}[theorem]{Question}

\theoremstyle{remark}
\newtheorem*{remark}{Remark}
\newtheorem*{note}{Note}
\newtheorem*{solution}{Solution}



%Inequalities
\newcommand{\cycsum}{\sum_{\mathrm{cyc}}}
\newcommand{\symsum}{\sum_{\mathrm{sym}}}
\newcommand{\cycprod}{\prod_{\mathrm{cyc}}}
\newcommand{\symprod}{\prod_{\mathrm{sym}}}

%Linear Algebra

\DeclareMathOperator{\Span}{span}
\DeclareMathOperator{\im}{im}
\DeclareMathOperator{\diag}{diag}
\DeclareMathOperator{\Ker}{Ker}
\DeclareMathOperator{\ob}{ob}
\DeclareMathOperator{\Hom}{Hom}
\DeclareMathOperator{\Mor}{Mor}
\DeclareMathOperator{\sk}{sk}
\DeclareMathOperator{\Vect}{Vect}
\DeclareMathOperator{\Set}{Set}
\DeclareMathOperator{\Group}{Group}
\DeclareMathOperator{\Ring}{Ring}
\DeclareMathOperator{\Ab}{Ab}
\DeclareMathOperator{\Top}{Top}
\DeclareMathOperator{\hTop}{hTop}
\DeclareMathOperator{\Htpy}{Htpy}
\DeclareMathOperator{\Cat}{Cat}
\DeclareMathOperator{\CAT}{CAT}
\DeclareMathOperator{\Cone}{Cone}
\DeclareMathOperator{\dom}{dom}
\DeclareMathOperator{\cod}{cod}
\DeclareMathOperator{\Aut}{Aut}
\DeclareMathOperator{\Mat}{Mat}
\DeclareMathOperator{\Fin}{Fin}
\DeclareMathOperator{\rel}{rel}
\DeclareMathOperator{\Int}{Int}
\DeclareMathOperator{\sgn}{sgn}
\DeclareMathOperator{\Homeo}{Homeo}
\DeclareMathOperator{\SHomeo}{SHomeo}
\DeclareMathOperator{\PSL}{PSL}
\DeclareMathOperator{\Bil}{Bil}
\DeclareMathOperator{\Sym}{Sym}
\DeclareMathOperator{\Skew}{Skew}
\DeclareMathOperator{\Alt}{Alt}
\DeclareMathOperator{\Quad}{Quad}
\DeclareMathOperator{\Sin}{Sin}
\DeclareMathOperator{\Supp}{Supp}
\DeclareMathOperator{\Char}{char}
\DeclareMathOperator{\Teich}{Teich}
\DeclareMathOperator{\GL}{GL}
\DeclareMathOperator{\tr}{tr}
\DeclareMathOperator{\codim}{codim}
\DeclareMathOperator{\coker}{coker}
\DeclareMathOperator{\Diff}{Diff}
\DeclareMathOperator{\Bun}{Bun}
\DeclareMathOperator{\Sm}{Sm}



%Row operations
\newcommand{\elem}[1]{% elementary operations
\xrightarrow{\substack{#1}}%
}

\newcommand{\lelem}[1]{% elementary operations (left alignment)
\xrightarrow{\begin{subarray}{l}#1\end{subarray}}%
}

%SS
\DeclareMathOperator{\supp}{supp}
\DeclareMathOperator{\Var}{Var}

%NT
\DeclareMathOperator{\ord}{ord}

%Alg
\DeclareMathOperator{\Rad}{Rad}
\DeclareMathOperator{\Jac}{Jac}

%Misc
\newcommand{\SL}{{\mathrm{SL}}}
\newcommand{\mobgp}{{\mathrm{PSL}_2(\mathbb{C})}}
\newcommand{\id}{{\mathrm{id}}}
\newcommand{\MCG}{{\mathrm{MCG}}}
\newcommand{\PMCG}{{\mathrm{PMCG}}}
\newcommand{\SMCG}{{\mathrm{SMCG}}}
\newcommand{\ud}{{\mathrm{d}}}
\newcommand{\Vol}{{\mathrm{Vol}}}
\newcommand{\Area}{{\mathrm{Area}}}
\newcommand{\diam}{{\mathrm{diam}}}
\newcommand{\End}{{\mathrm{End}}}


\newcommand{\reg}{{\mathtt{reg}}}
\newcommand{\geo}{{\mathtt{geo}}}

\newcommand{\tori}{{\mathcal{T}}}
\newcommand{\cpn}{{\mathtt{c}}}
\newcommand{\pat}{{\mathtt{p}}}

\let\Cap\undefined
\newcommand{\Cap}{{\mathcal{C}}ap}
\newcommand{\Push}{{\mathcal{P}}ush}
\newcommand{\Forget}{{\mathcal{F}}orget}




\begin{document}
    \begin{exercise}[1]
        Let $R$ be a Noetherian ring and $A$ be a finitely

        generated $R$-algebra. Show that if
        $B \subset A$ is a subalgebra such that $A$ is
        a finitely generated $B$-module, then
        $B$ is also a finitely generated
        $R$-algebra.
    \end{exercise}


    \begin{proof}
        We want to show that
        $B$ is finitely generated as an $R$-algebra.\\

        Suppose $\left\{ y_1, \ldots, y_n \right\} \subset A$ 
        generate $A$ as an $R$-algebra, so
        $A = R \left[ y_1, \ldots y_n \right] $.
        Since $A$ is also finitely generated
        as a $R$-module, there
        exist $a_1, \ldots, a_m \in A$ such that
        $A = Ba_1 + \ldots + Ba_m$.
        Now using the module expression for $A$, write
        \[
        y_i = \sum_j b_{ij} a_j
        \] 
        and similarly, since 
        $a_i a_j \in A$,
        \[
        a_i a_j = \sum_k b_{ijk} a_k.
        \] 
        Then given arbitrary
        $u,v \in A$, 
        we can write
        \[
        u = \sum_{i,j} \alpha_i b_{ij} a_{j}
        \] 
        and 
         \[
        v = \sum_{i,j} \beta_i b_{ij} a_{j}
        \] 
        We have then seen that
        \[
        uv = \sum_{i,j,k,l} \alpha_i b_{ij} a_j \beta_k b_{kl}a_l
        = \sum_{i,j,k,l} \left( \alpha_i \beta_k \right) 
        \left( b_{ij} b_{kl} \right) \sum_{r} b_{jlr} a_r
        = \sum_{i,,k,l,r} \left( \alpha_i b_{k} \right) 
        \left( b_{ij} b_{kl} b_{jlr} \right) a_r.
        \] 
        This shows that
        $A$ is generated by
        $a_1, \ldots, a_n$  as an
        $D = R \left[ b_{ij}, b_{jlr} \mid 
        j,l = 1,\ldots, n \quad
    i,r = 1,\ldots,m\right] $ algebra.
        In particular, $D$ is Noetherian by corollary
        6.15, so $A$ is a Noetherian
        $D$-module by applying by
        theorem 6.11. Hence
        since
        $B$ is a natural $D$-submodule of $A$ it is 
        finitely generated as a $D$-module, so
        $B = D b_1 + \ldots + D b_n$.
        However, this in particular expresses
        $B$ as the $R$-algebra
        $R \left[ b_{ij},b_{ijk}, b_1,\ldots, b_n
         \mid  i,j,k = 1, \ldots, n\right] $.

    \end{proof}

    \begin{exercise}[2]
        Let $K$ be a field and let $A$ be a finitely
        generated $K$-algebra. Show that if
        $A$ is a field, then $A$ is finite-dimensional
        as a $K$-vector space. In particular, note that
        for every maximal ideal 
        $\mathfrak{R} \subset A$, 
        $A / \mathfrak{R}$ is a finite dimensional
        $K$-vector space.
    \end{exercise}


    \begin{proof}
        If $A$ is a field extension of $K$ such that
        $A$ is finite type over $K$, then
        by Zariski's lemma, we directly find that
        $A$ is finite over $K$ - i.e. that it
        is finitely generated as a $K$-module, and since
        $K$ is a field, this is saying that
        $A$ is finitely generated as a $K$-vector space.
        The latter part is corollary 11.7.\\
        \linebreak
    \end{proof}








    %\printbibliography
\end{document}
