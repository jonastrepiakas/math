\documentclass[reqno]{amsart}
\usepackage{amscd, amssymb, amsmath, amsthm}
\usepackage{graphicx}
\usepackage[colorlinks=true,linkcolor=blue]{hyperref}
\usepackage[utf8]{inputenc}
\usepackage[T1]{fontenc}
\usepackage{textcomp}
\usepackage{babel}
%% for identity function 1:
\usepackage{bbm}
%%For category theory diagrams:
\usepackage{tikz-cd}


\setlength\parindent{0pt}

\pdfsuppresswarningpagegroup=1

\newtheorem{theorem}{Theorem}[section]
\newtheorem{lemma}[theorem]{Lemma}
\newtheorem{proposition}[theorem]{Proposition}
\newtheorem{corollary}[theorem]{Corollary}
\newtheorem{conjecture}[theorem]{Conjecture}

\theoremstyle{definition}
\newtheorem{definition}[theorem]{Definition}
\newtheorem{example}[theorem]{Example}
\newtheorem{exercise}[theorem]{Exercise}
\newtheorem{problem}[theorem]{Problem}
\newtheorem{question}[theorem]{Question}

\theoremstyle{remark}
\newtheorem*{remark}{Remark}
\newtheorem*{note}{Note}
\newtheorem*{solution}{Solution}



%Inequalities
\newcommand{\cycsum}{\sum_{\mathrm{cyc}}}
\newcommand{\symsum}{\sum_{\mathrm{sym}}}
\newcommand{\cycprod}{\prod_{\mathrm{cyc}}}
\newcommand{\symprod}{\prod_{\mathrm{sym}}}

%Linear Algebra

\DeclareMathOperator{\Span}{span}
\DeclareMathOperator{\im}{im}
\DeclareMathOperator{\diag}{diag}
\DeclareMathOperator{\Ker}{Ker}
\DeclareMathOperator{\ob}{ob}
\DeclareMathOperator{\Hom}{Hom}
\DeclareMathOperator{\Mor}{Mor}
\DeclareMathOperator{\sk}{sk}
\DeclareMathOperator{\Vect}{Vect}
\DeclareMathOperator{\Set}{Set}
\DeclareMathOperator{\Group}{Group}
\DeclareMathOperator{\Ring}{Ring}
\DeclareMathOperator{\Ab}{Ab}
\DeclareMathOperator{\Top}{Top}
\DeclareMathOperator{\hTop}{hTop}
\DeclareMathOperator{\Htpy}{Htpy}
\DeclareMathOperator{\Cat}{Cat}
\DeclareMathOperator{\CAT}{CAT}
\DeclareMathOperator{\Cone}{Cone}
\DeclareMathOperator{\dom}{dom}
\DeclareMathOperator{\cod}{cod}
\DeclareMathOperator{\Aut}{Aut}
\DeclareMathOperator{\Mat}{Mat}
\DeclareMathOperator{\Fin}{Fin}
\DeclareMathOperator{\rel}{rel}
\DeclareMathOperator{\Int}{Int}
\DeclareMathOperator{\sgn}{sgn}
\DeclareMathOperator{\Homeo}{Homeo}
\DeclareMathOperator{\SHomeo}{SHomeo}
\DeclareMathOperator{\PSL}{PSL}
\DeclareMathOperator{\Bil}{Bil}
\DeclareMathOperator{\Sym}{Sym}
\DeclareMathOperator{\Skew}{Skew}
\DeclareMathOperator{\Alt}{Alt}
\DeclareMathOperator{\Quad}{Quad}
\DeclareMathOperator{\Sin}{Sin}
\DeclareMathOperator{\Supp}{Supp}
\DeclareMathOperator{\Char}{char}
\DeclareMathOperator{\Teich}{Teich}
\DeclareMathOperator{\GL}{GL}
\DeclareMathOperator{\tr}{tr}
\DeclareMathOperator{\codim}{codim}
\DeclareMathOperator{\coker}{coker}
\DeclareMathOperator{\Diff}{Diff}
\DeclareMathOperator{\Bun}{Bun}


%Row operations
\newcommand{\elem}[1]{% elementary operations
\xrightarrow{\substack{#1}}%
}

\newcommand{\lelem}[1]{% elementary operations (left alignment)
\xrightarrow{\begin{subarray}{l}#1\end{subarray}}%
}

%SS
\DeclareMathOperator{\supp}{supp}
\DeclareMathOperator{\Var}{Var}

%NT
\DeclareMathOperator{\ord}{ord}

%Alg
\DeclareMathOperator{\Rad}{Rad}
\DeclareMathOperator{\Jac}{Jac}

%Misc
\newcommand{\SL}{{\mathrm{SL}}}
\newcommand{\mobgp}{{\mathrm{PSL}_2(\mathbb{C})}}
\newcommand{\id}{{\mathrm{id}}}
\newcommand{\MCG}{{\mathrm{MCG}}}
\newcommand{\PMCG}{{\mathrm{PMCG}}}
\newcommand{\SMCG}{{\mathrm{SMCG}}}
\newcommand{\ud}{{\mathrm{d}}}
\newcommand{\Vol}{{\mathrm{Vol}}}
\newcommand{\Area}{{\mathrm{Area}}}
\newcommand{\diam}{{\mathrm{diam}}}
\newcommand{\End}{{\mathrm{End}}}


\newcommand{\reg}{{\mathtt{reg}}}
\newcommand{\geo}{{\mathtt{geo}}}

\newcommand{\tori}{{\mathcal{T}}}
\newcommand{\cpn}{{\mathtt{c}}}
\newcommand{\pat}{{\mathtt{p}}}

\let\Cap\undefined
\newcommand{\Cap}{{\mathcal{C}}ap}
\newcommand{\Push}{{\mathcal{P}}ush}
\newcommand{\Forget}{{\mathcal{F}}orget}



\title{Assignment 5}
\author{Jonas Trepiakas}
\date{}

\begin{document}
\maketitle
    \begin{exercise}[1]
        (2) 
        
    We claim that
    $\mathbb{Z} \left[ x_1, x_2, x_3 \right] $ is a
    finite extension. 
    Now, clearly, we can express
    $1, x_1, x_1^{2}, x_1^3$ as linear combinations over
    $1, x_1, x_1^{2}, x_1^3$.
    Suppose we can express $x_1^{n}$ as
    a linear combination of
    $1, x_1, x_1^2, x_1^{3}$ for 
    $n = 1, \ldots, N-1$ for
    some $N \ge 4$. Since
    \begin{align*}
        x^{n} 
        &= x_1^{n-1} \left( x_1+ x_2 + x_3  \right) 
        - x_1^{n-1} x_2 - x_1^{n-1} x_3\\
        &= x_1^{n-1} \sigma_1 -
        x_1^{n-2} \sigma_2 + x_1^{n-2} x_2 x_3\\
        &= x_1^{n-1} \sigma_1 - x_1^{n-2} \sigma_2
        + x_1^{n-3} \sigma_3
    \end{align*}
    we find that for
    $N \ge 4$, 
    $x_1^{N}$ can be written as a linear combination
    over 
    $x_1^{N-1}, x_1^{N-2}$ and
    $x_1^{N-3}$ which by the inductive assumption
    can be written as linear combinations
    of $1, x_1, x_1^2, x_1^3$.
    Hence
    $\mathbb{Z}\left[ x_1, x_2, x_3 \right] $ is
    a finite extension over
    $\mathbb{Z}\left[ \sigma_1, \sigma_2, \sigma_3 \right] $ with
    \[
    \left\{ 1, x_1, x_1^2, x_1^3, x_2,x_2^2,
    x_2^3, x_3, x_3^2, x_3^3 \right\} 
    \] 
    as a finite generating set.
    By proposition 10.11, this also implies that
    the extension is integral.\\
    \linebreak
    (3) We claim that
    $\mathbb{Z}\left[ x,y \right] $ is
    not a finite extension of
    $\mathbb{Z}\left[ x,xy \right] $. 
     Suppose
     $\left\{ g_1, \ldots, g_n \right\} 
     \in \mathbb{Z}\left[ x,y \right] $ is a
     generating set as a module.
     Let $N$ be the maximal degree of $y$ over
     $g_1, \ldots, g_n$.
     Then
     $y^{N+1} \in 
     \Span \left( g_1, \ldots, g_n \right) $, so
     let
     $y^{N+1} = f_1(x,xy) g_1(x,y) + \ldots +
     f_n (x,xy) g_n(x,y)$.

     Writing each 
     $f_i (x,xy) = 
     \sum \alpha_{i, k,l}
     x^{k} \left( xy \right)^{l}$, we see that
     \begin{align*}
         y^{N+1} &
         = 
     \sum \alpha_{1,k,l} x^{k} (xy)^{l} g_1(x,y)
     + \ldots +
     \sum \alpha_{n,k,l} x^{k}(xy)^{l} g_n(x,y)\\
     &= \sum \left( \alpha_{1,k,l}g_1(x,y) +
     \ldots + \alpha_{n,k,l}g_n(x,y) \right) 
     x^{k} (xy)^{l}.
     \end{align*}
     So in particular, we must have
     that
     for $\left( k,l \right) \neq (0,0)$,
     \[
     \alpha_{1,k,l}g_1(x,y) + \ldots+
     \alpha_{n,k,l}g_n(x,y) = 0.
     \] 
     But then we get
     \[
     y^{N+1} = 
     \alpha_{1,0,0} g_1(x,y) +\ldots
     + \alpha_{n,0,0}g_n(x,y)
     \] 
     which has  maximal $y$ degree $N$, giving a contradiction. 
     Thus the extension is not finite. 
     However,
     clearly, it is a finite-type extension, since
     $y$ together with
     $\mathbb{Z}\left[ x,xy \right] $ precisely generate all
     of $\mathbb{Z}\left[ x,y \right] $.
     By proposition 10.11, we then
     find that $\mathbb{Z}\left[ x,y \right] $ is not an
     integral extension.\\
     \linebreak
     (7) If the map
     $\mathbb{C}\left[ x \right] 
     \hookrightarrow 
     \mathbb{C}\left[ x,y,z \right] / \left( z^2
     -xy\right) $ were integral, proposition 10.6 gives that
     $\mathbb{C}\left[ x,y \right]
     \subset \mathbb{C}\left[ x,y,z \right] 
     / \left( z^2 - xy \right)$ would be
     finitely generated as a
      $\mathbb{C}\left[ x \right] $-module. 
      Suppose
      $f_1, \ldots, f_n$ generated
      $\mathbb{C}\left[ x,y \right] $ as
      a $\mathbb{C}\left[ x \right] $ module in
      $\mathbb{C}\left[ x,y,z \right] /
      \left( z^2 - xy \right) $.
      If $y^{N}$ is the maximal degree of $y$ among
      $f_1, \ldots, f_n$, then
      $y^{N+1} = \sum g_i f_i$ for
      $g_i \in \mathbb{C}\left[ x \right] $.
      However, there is clearly no way to obtain
      $y^{N+1}$ in such a way. So
      the extension is not integral. Since it is
      clearly finite type, it is also not finite by proposition
      10.11.\\
      \linebreak
      (9) If
      $\mathbb{C}[x]
      \hookrightarrow
      \mathbb{C}\left[ x,y,z \right] / 
      \left( z^2 - xy, x^3 - yz \right) $ were
      integral, $z$ would be integral over
      $\mathbb{C}\left[ x \right] $, so there
      would be some
      linear combination
      \[
      f_n(x) z^{n} + \ldots + f_0 (x) = 0
      \] 
      However, there is not relation
      converting $xz$ to something different, so if
      $x^{k_n}$ is the highest term of $x$ in
      $f_n(x)$, then
      $x^{k_n} z^{n}$ is a term that cannot
      cancel in the above linear combination. 
      So the extension cannot be integral. Hence it
       can also not be finite.\\
       \linebreak
       (10) The extension
       $\mathbb{C} \hookrightarrow
       \mathbb{C}\left[ x_1, x_2, x_3, \ldots \right] 
       / \left( x_1^2, x_2^2, x_3^2, \ldots \right) $ 
       is not finite: suppose
       it were generated by
       $f_1, \ldots, f_n \in 
       \mathbb{C}\left[ x_1, x_2, \ldots \right] 
       / \left( x_1^2, x_2^2, \ldots \right) $, and let
        $m$ be maximal such that
        one of the $f_i$ has a term with
        $x_m$. Then
        $x_{m+1} = c_1 f_1 + \ldots+ c_n f_n$ with
        $c_i \in \mathbb{C}$. However, then multiplying 
        both sides by $x_{m+1}$, we see that
        each non-zero term in
        $c_1 f_1+ \ldots + c_n f_n$ must have a $x_{m+1}$, 
        contradicting maximality of $m$.\\
        \linebreak
        The extension is integral, however, since
        for any $b \in 
        \mathbb{C}\left[ x_1, x_2, \ldots \right] 
        / \left( x_1^2, x_2^2, \ldots \right) $, 
        let $x_{i_1}, \ldots, x_{i_k}$ be the
        $x_i$ which appear in $b$. Then
        $b \left( x_{i_1} \cdots x_{i_k} \right) 
        = 0 \subset 
        \left( x_{i_1} \cdots x_{i_k} \right) $, and
        $\left( x_{i_1} \cdots
        x_{i_k} \right) $ is clearly finitely generated,
        so by proposition 10.6, 
        $b$ is integral over $\mathbb{C}$.
      



    \end{exercise}

    %\bibliography{../refs.bib}
\end{document}
