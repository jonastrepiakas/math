\documentclass[reqno]{amsart}
\usepackage{amscd, amssymb, amsmath, amsthm}
\usepackage{graphicx}
\usepackage[colorlinks=true,linkcolor=blue]{hyperref}
\usepackage[utf8]{inputenc}
\usepackage[T1]{fontenc}
\usepackage{textcomp}
\usepackage{babel}
%% for identity function 1:
\usepackage{bbm}
%%For category theory diagrams:
\usepackage{tikz-cd}


\setlength\parindent{0pt}

\pdfsuppresswarningpagegroup=1

\newtheorem{theorem}{Theorem}[section]
\newtheorem{lemma}[theorem]{Lemma}
\newtheorem{proposition}[theorem]{Proposition}
\newtheorem{corollary}[theorem]{Corollary}
\newtheorem{conjecture}[theorem]{Conjecture}

\theoremstyle{definition}
\newtheorem{definition}[theorem]{Definition}
\newtheorem{example}[theorem]{Example}
\newtheorem{exercise}[theorem]{Exercise}
\newtheorem{problem}[theorem]{Problem}
\newtheorem{question}[theorem]{Question}

\theoremstyle{remark}
\newtheorem*{remark}{Remark}
\newtheorem*{note}{Note}
\newtheorem*{solution}{Solution}



%Inequalities
\newcommand{\cycsum}{\sum_{\mathrm{cyc}}}
\newcommand{\symsum}{\sum_{\mathrm{sym}}}
\newcommand{\cycprod}{\prod_{\mathrm{cyc}}}
\newcommand{\symprod}{\prod_{\mathrm{sym}}}

%Linear Algebra

\DeclareMathOperator{\Span}{span}
\DeclareMathOperator{\im}{im}
\DeclareMathOperator{\diag}{diag}
\DeclareMathOperator{\Ker}{Ker}
\DeclareMathOperator{\ob}{ob}
\DeclareMathOperator{\Hom}{Hom}
\DeclareMathOperator{\Mor}{Mor}
\DeclareMathOperator{\sk}{sk}
\DeclareMathOperator{\Vect}{Vect}
\DeclareMathOperator{\Set}{Set}
\DeclareMathOperator{\Group}{Group}
\DeclareMathOperator{\Ring}{Ring}
\DeclareMathOperator{\Ab}{Ab}
\DeclareMathOperator{\Top}{Top}
\DeclareMathOperator{\hTop}{hTop}
\DeclareMathOperator{\Htpy}{Htpy}
\DeclareMathOperator{\Cat}{Cat}
\DeclareMathOperator{\CAT}{CAT}
\DeclareMathOperator{\Cone}{Cone}
\DeclareMathOperator{\dom}{dom}
\DeclareMathOperator{\cod}{cod}
\DeclareMathOperator{\Aut}{Aut}
\DeclareMathOperator{\Mat}{Mat}
\DeclareMathOperator{\Fin}{Fin}
\DeclareMathOperator{\rel}{rel}
\DeclareMathOperator{\Int}{Int}
\DeclareMathOperator{\sgn}{sgn}
\DeclareMathOperator{\Homeo}{Homeo}
\DeclareMathOperator{\SHomeo}{SHomeo}
\DeclareMathOperator{\PSL}{PSL}
\DeclareMathOperator{\Bil}{Bil}
\DeclareMathOperator{\Sym}{Sym}
\DeclareMathOperator{\Skew}{Skew}
\DeclareMathOperator{\Alt}{Alt}
\DeclareMathOperator{\Quad}{Quad}
\DeclareMathOperator{\Sin}{Sin}
\DeclareMathOperator{\Supp}{Supp}
\DeclareMathOperator{\Char}{char}
\DeclareMathOperator{\Teich}{Teich}
\DeclareMathOperator{\GL}{GL}
\DeclareMathOperator{\tr}{tr}
\DeclareMathOperator{\codim}{codim}


%Row operations
\newcommand{\elem}[1]{% elementary operations
\xrightarrow{\substack{#1}}%
}

\newcommand{\lelem}[1]{% elementary operations (left alignment)
\xrightarrow{\begin{subarray}{l}#1\end{subarray}}%
}

%SS
\DeclareMathOperator{\supp}{supp}
\DeclareMathOperator{\Var}{Var}

%NT
\DeclareMathOperator{\ord}{ord}

%Alg
\DeclareMathOperator{\Rad}{Rad}
\DeclareMathOperator{\Jac}{Jac}

%Misc
\newcommand{\SL}{{\mathrm{SL}}}
\newcommand{\mobgp}{{\mathrm{PSL}_2(\mathbb{C})}}
\newcommand{\id}{{\mathrm{id}}}
\newcommand{\MCG}{{\mathrm{MCG}}}
\newcommand{\PMCG}{{\mathrm{PMCG}}}
\newcommand{\SMCG}{{\mathrm{SMCG}}}
\newcommand{\ud}{{\mathrm{d}}}
\newcommand{\Vol}{{\mathrm{Vol}}}
\newcommand{\Area}{{\mathrm{Area}}}
\newcommand{\diam}{{\mathrm{diam}}}
\newcommand{\End}{{\mathrm{End}}}


\newcommand{\reg}{{\mathtt{reg}}}
\newcommand{\geo}{{\mathtt{geo}}}

\newcommand{\tori}{{\mathcal{T}}}
\newcommand{\cpn}{{\mathtt{c}}}
\newcommand{\pat}{{\mathtt{p}}}

\let\Cap\undefined
\newcommand{\Cap}{{\mathcal{C}}ap}
\newcommand{\Push}{{\mathcal{P}}ush}
\newcommand{\Forget}{{\mathcal{F}}orget}


\title{Assignment 1}
\author{Jonas Trepiakas}
\date{}


\begin{document}
\maketitle
    \begin{exercise}[1]
        \begin{proof}
            (i)
                We claim that 
                    $\left( x^2+1 \right) $ is a radical,
                    prime and maximal
                    ideal in $\mathbb{R}\left[ x \right] $.
                    This can be seen
                    by noting that
                    $\mathbb{R}\left[ x \right] /
                    \left( x^2 +1 \right) \cong \mathbb{C}$ which
                    is a field. Hence
                    $\left( x^2+1 \right) $ is
                    maximal. Suppose
                    $f^{n} \in \left( x^2+1 \right) $.
                    Since $x^2 +1$ is irreducible
                    and $x^2 +1  \mid f^{n}$, we must
                    have $x^2 + 1  \mid f$, hence
                    $f \in \left( x^2 +1 \right) $,
                    so $\sqrt{\left( x^2+1 \right) } 
                    =\left( x^2 +1 \right) $.
                Over $\mathbb{C}\left[ x \right] $,
                    we claim that
                    $\left( x^2+1 \right) $ is neither
                    prime nor maximal, but still radical.
                    It is not prime as
                    $x^2+1 = \left( x+i \right) \left( x-i \right) $
                    and hence also not maximal
                    since $\left( x^2+1 \right) 
                    \subset \left( x+i \right) \neq 
                    \mathbb{C}[x]$, where
                    inequality follows from 
                    $\left( x+i \right) $ only
                    having polynomials of degree $\ge 1$.

                    Now suppose
                    $f^{n} \in \left( x^2+1 \right) $.
                    Then $x+i, x-i  \mid f^{n}$, hence
                    both must divide $f$ as they are
                    irreducible, so
                    $x^2 +1  \mid f$. Thus
                    $f \in \left( x^2+1 \right) $, so
                    $\sqrt{\left( x^2+1 \right) } 
                    = \left( x^2+1 \right)  $ over
                    $\mathbb{C}\left[ x \right] $
                    as well.\\
                    \linebreak
                    (ii) Since $\left( x^2 +1 \right) $  is
                    a prime ideal in $
                    \mathbb{R}[x]$ by the previous exercise,
                    we find by Eisenstein's criterion that
                    $y^2 + x^2 + 1$ is irreducible
                    in $\mathbb{R}[x][y] =:
                    \mathbb{R}[x,y]$.\\
                    \linebreak
                    

                    (iii)
                 
                    Let $C =
                    \left\{ f \in 
                    C\left( \mathbb{R}^2,\mathbb{R} \right)
                 \mid \forall x \in \mathbb{R}\colon
             f\left( x,0 \right) =0 \right\}
             \subset C\left( \mathbb{R}^2,\mathbb{R} \right)$.
             We claim that $C$ is radical, but neither
             prime nor maximal.

             To see that it is radical, suppose
             $g^{n} \in C$, so
             $g(x,0) \cdot \ldots \cdot 
             g(x,0) = g^{n}(x,0) = 0$. Since
             $\mathbb{R}$ is an integral domain,
             this forces $g(x,0) = 0$, so
             $g \in C$. Thus $\sqrt{C}  = C$.

             Now let
             $h(x) = \mathbbm{1}_{\ge 0}(x) x $
             and
             $k(x) = \mathbbm{1}_{\le 0}(x) x$. Then
             $h,k \not\in C$, but
             $hk \in C$. Therefore,
             $C$ is not prime.
             Since $C\left( \mathbb{R}^2,\mathbb{R} \right) $ is
             a commutative ring and maximal ideals are
             prime over a commutative ring, we
             thus also conclude that $C$ is not maximal.\\
             \linebreak
             (iv) The ideal $\left( 5 \right) $ in
             $\mathbb{Z}[i]$ is not prime, hence
             not maximal as $\mathbb{Z}[i]$ is commutative.
             It is not prime because $5 = 
             (2+i) (2-i)$.

             For the radical part, if
             $a+bi \in \sqrt{(5)} $, then
             $(2+i)(2-i)  \mid \left( a+bi \right)^{n}$,
             so $2+i  \mid a+bi$ and
             $2-i  \mid a+bi$ since
             each is irreducible, hence
             $5  \mid a+bi$, so
             $\sqrt{\left( 5 \right) } =
             \left( 5 \right) $.\\
             \linebreak
             
             (v) We claim that
             $\left( n \right) \subset \mathbb{Z}$ 
             is prime and maximal whenever
             $n$ is a prime and not otherwise.
             If $n$ is not prime, then
             writing $n = ab$ for $a,b >1$, we
             have $\left( n \right) 
             = \left( a \right) (b)$, so $\left( n \right) $ is
             not prime, hence not maximal as
             $\mathbb{Z}$ is commutative so
             all maximal ideals are prime ideals.
             If instead $n$ is a prime, $n=p$, then
             $\left( p \right) $ is both
             maximal and prime since
             $\mathbb{Z} / p$ is a field.


             Suppose now
             $m \in \sqrt{\left( n \right) } $.
             Then $m^{k} \in \left( n \right) $, so
             $m^{k} = nq$ for some $q \in \mathbb{Z}$.
             Suppose $n$ is squarefree. Let
             $p  \mid n$. Then $p  \mid m^{k}$ and thus
             $p  \mid m$, so $n  \mid m$, hence
             $m \in \left( n \right) $. Conversely,
             if $n$ is not squarefree, then
             letting $n = p^k q$ for some $k>1$, we have
             $p^{k-1}q \in \sqrt{\left( n \right) } $ while
             $p^{k-1} q \not\in \left( n \right) $, so
             $\left( n \right) $ is not radical.
        \end{proof}
    \end{exercise}



    \begin{exercise}[2]
        Let $n \in \mathbb{N}$. We denote the set
        of orthogonal matrices on
        $\mathbb{R}^{n}$ by
        \[
        O \left( \mathbb{R}^{n} \right) 
        = \left\{ A \in \mathbb{R}^{n\times n}
         \mid A^{T} A = I_n \right\} 
        \] 
        Write $R = 
        \mathbb{R} \left[ x_{ij} \mid 
        i,j = 1,\ldots, n\right] $ for the polynomial
        ring in variables $X = \left( x_{ij} \right)_{i,j
        = 1,\ldots, n}$. 
        \begin{enumerate}
            \item Show that $O \left( \mathbb{R}^{n} \right) $ 
                is the zero set 
                $\mathbb{V} (I)$ of the ideal
                $I = 
                \left( \left\{ f_{ij}  \mid 
                i,j = 1,\ldots,n\right\}  \right) $  of
                $R$ where
                \[
                f_{i,j} = \sum_{k=1}^{n} x_{ki}x_{kj}
                \quad \text{for } i\neq j
                \quad \text{and} \quad
                f_{ii} = \sum_{k=1}^{n} x_{ki}^2 -1.
                \] 
            \item Show that $O \left( \mathbb{R}^{n} \right) $ 
                is also the real zero set of the ideal
                $J = \left( g,h \right) $ generated by
                the polynomials
                $g = \det (X)^2 -1$ and
                $h = \sum_{i,j=1}^{n} x_{ij}^2 -n$.
        \end{enumerate}
    \end{exercise}
    \begin{proof}
        (a) We know that $A = 
        \left( \alpha_{ij} \right)
        \in O \left( \mathbb{R}^{n} \right) $ if and only
        if $A^{T} A = I$.
        Taking entries of either side of this equality, we
        get that $A$ is orthogonal if and only if
        both of the following conditions hold:
        \begin{align*}
            0 &= \left( A^{T}A \right)_{ij}
            = \sum_{k=1}^{n} \left( A^{T} \right)_{ik}
            A_{kj}
            = \sum_{k=1}^{n} \alpha_{ki} \alpha_{kj}\\
            1 &= \left( A^{T}A \right)_{ii}
            = \sum_{k=1}^{n} \alpha_{ki}^2.
        \end{align*}
        That is,
        $A \in O\left( \mathbb{R}^{n} \right) $ if and
        only if
        $A \in \mathbb{V} (I)$ where we identify
        $R \cong M_{n} \left( \mathbb{R} \right) $ by
        $\sum \alpha_{ij} x_{ij}
        \mapsto \left( \alpha_{ij} \right) $.\\
        \linebreak
        (b) Firstly, suppose
        $A \in O \left( \mathbb{R}^{n} \right) $.
        Then $A^{T}A = I$. Therefore,
        $\det (A)^2 -1 = 
        \det (A^{T}) \det (A) -1
        = \det (A^{T}A) -1 
        = \det (I) - 1
        = 0$, so
        $g(A) = 0$. And now
        \[
        n = \tr (I) = \tr (A^{T}A)
        = \sum_{k=1}^{n} \left( A^{T}A \right)_{kk}
        = \sum_{k,r=1}^{n} \alpha_{rk}^2
        \] 
        hence also $h(A) = 0$.
        Therefore
        $O \left( \mathbb{R}^{n} \right) 
        \subset \mathbb{V} (J)$.\\
        Conversely,
        suppose
        $A = \left( \alpha_{ij} \right)  \in \mathbb{V}(J)$.
        Firstly, note that since
        \[
        x^{T} A^{T}A x
        = \left( Ax \right)^{T} Ax
        = \|Ax\|^2 \ge 0,
        \] 
        the matrix $A^{T}A$ is positive semi-definite,
        hence all its eigenvalues are non-negative real numbers.
        In particular,
        $\sqrt[n]{\Pi_{i=1}^{n} \lambda_i} $ is well-defined
        where
        $\lambda_1, \ldots, \lambda_m$ are the eigenvalues
        of $A^{T}A$ listed
        with repetition, and
        we can apply the AMGM inequality.
        Recall that the AMGM inequality
        tells us that
        \[
        \frac{\sum_i \lambda_i}{n} \ge 
        \sqrt[n]{\prod_i \lambda_i} 
        \] 
        with equality if and only if all the
        $\lambda_i$ are equal. But 
        by Jordan normal form,
        $\tr \left( A^{T}A \right) =
        \sum_{i} \lambda_i$, and
        $\det \left( A^{T}A \right) =
        \prod_i \lambda_i$, so we obtain
        \[
        \tr(A^{T}A) \ge n \det \left( A^{T}A \right) 
        \] 
        with equality if and only if all eigenvalues are equal.
        But since $A \in \mathbb{V} (J)$, we have
        $g (A) = 0$, so
        $\det \left( A^{T}A \right) 
        = \det (A)^2 = 1$. Likewise,
        \[
        \tr \left( A^{T}A \right) 
        = \sum_{k=1}^{n} \left( A^{T}A \right)_{kk} 
        = \sum_{k,r=1}^{n} \alpha_{rk}^2
        = h(A)+n = n
        \] 
        Thus we get
        \[
        n = \tr\left( A^{T}A \right) \ge 
        n \det \left( A^{T}A \right) = n
        \] 
        so we conclude that, since we have equality between
        the two sides, all eigenvalues of
        $A^{T}A$ must be equal.
        Now since
        $A^{T}A$ is self-adjoint, it is in particular
        diagonable, hence it has
        precisely $n$ eigenvalues counted with multiplicity.
        Therefore, it has one eigenvalue with multiplicity $n$.
        Letting wlog $\lambda$ denote this eigenvalue, we
        get
        \[
        n \lambda = \tr \left( A^{T}A \right) 
        = n
        \] 
        so $\lambda = 1$ since $\mathbb{R}$ is an integral
        domain.
        To see that this forces $A^{T}A$ to be $I$, 
        we note that since $A^{T}A$ is diagonable, we can
        find some invertible linear map
        $P \in \GL_n \left( \mathbb{R} \right) $ such that
        $P A^{T}A P^{-1} = I$, implying
        $A^{T}A = I$. Thus $A$ is orthogonal, so
        $A \in O \left( \mathbb{R}^{n} \right) $.
        This gives the inclusion
        $\mathbb{V} (J) \subset O \left( \mathbb{R}^{n} \right) $.
    \end{proof}









    %\bibliography{../refs.bib}
\end{document}
