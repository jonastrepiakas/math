\documentclass[reqno]{amsart}
\usepackage{amscd, amssymb, amsmath, amsthm}
\usepackage{graphicx}
\usepackage[colorlinks=true,linkcolor=blue]{hyperref}
\usepackage[utf8]{inputenc}
\usepackage[T1]{fontenc}
\usepackage{textcomp}
\usepackage{babel}
%% for identity function 1:
\usepackage{bbm}
%%For category theory diagrams:
\usepackage{tikz-cd}


\setlength\parindent{0pt}

\pdfsuppresswarningpagegroup=1

\newtheorem{theorem}{Theorem}[section]
\newtheorem{lemma}[theorem]{Lemma}
\newtheorem{proposition}[theorem]{Proposition}
\newtheorem{corollary}[theorem]{Corollary}
\newtheorem{conjecture}[theorem]{Conjecture}

\theoremstyle{definition}
\newtheorem{definition}[theorem]{Definition}
\newtheorem{example}[theorem]{Example}
\newtheorem{exercise}[theorem]{Exercise}
\newtheorem{problem}[theorem]{Problem}
\newtheorem{question}[theorem]{Question}

\theoremstyle{remark}
\newtheorem*{remark}{Remark}
\newtheorem*{note}{Note}
\newtheorem*{solution}{Solution}



%Inequalities
\newcommand{\cycsum}{\sum_{\mathrm{cyc}}}
\newcommand{\symsum}{\sum_{\mathrm{sym}}}
\newcommand{\cycprod}{\prod_{\mathrm{cyc}}}
\newcommand{\symprod}{\prod_{\mathrm{sym}}}

%Linear Algebra

\DeclareMathOperator{\Span}{span}
\DeclareMathOperator{\im}{im}
\DeclareMathOperator{\diag}{diag}
\DeclareMathOperator{\Ker}{Ker}
\DeclareMathOperator{\ob}{ob}
\DeclareMathOperator{\Hom}{Hom}
\DeclareMathOperator{\Mor}{Mor}
\DeclareMathOperator{\sk}{sk}
\DeclareMathOperator{\Vect}{Vect}
\DeclareMathOperator{\Set}{Set}
\DeclareMathOperator{\Group}{Group}
\DeclareMathOperator{\Ring}{Ring}
\DeclareMathOperator{\Ab}{Ab}
\DeclareMathOperator{\Top}{Top}
\DeclareMathOperator{\hTop}{hTop}
\DeclareMathOperator{\Htpy}{Htpy}
\DeclareMathOperator{\Cat}{Cat}
\DeclareMathOperator{\CAT}{CAT}
\DeclareMathOperator{\Cone}{Cone}
\DeclareMathOperator{\dom}{dom}
\DeclareMathOperator{\cod}{cod}
\DeclareMathOperator{\Aut}{Aut}
\DeclareMathOperator{\Mat}{Mat}
\DeclareMathOperator{\Fin}{Fin}
\DeclareMathOperator{\rel}{rel}
\DeclareMathOperator{\Int}{Int}
\DeclareMathOperator{\sgn}{sgn}
\DeclareMathOperator{\Homeo}{Homeo}
\DeclareMathOperator{\SHomeo}{SHomeo}
\DeclareMathOperator{\PSL}{PSL}
\DeclareMathOperator{\Bil}{Bil}
\DeclareMathOperator{\Sym}{Sym}
\DeclareMathOperator{\Skew}{Skew}
\DeclareMathOperator{\Alt}{Alt}
\DeclareMathOperator{\Quad}{Quad}
\DeclareMathOperator{\Sin}{Sin}
\DeclareMathOperator{\Supp}{Supp}
\DeclareMathOperator{\Char}{char}
\DeclareMathOperator{\Teich}{Teich}
\DeclareMathOperator{\GL}{GL}
\DeclareMathOperator{\tr}{tr}
\DeclareMathOperator{\codim}{codim}


%Row operations
\newcommand{\elem}[1]{% elementary operations
\xrightarrow{\substack{#1}}%
}

\newcommand{\lelem}[1]{% elementary operations (left alignment)
\xrightarrow{\begin{subarray}{l}#1\end{subarray}}%
}

%SS
\DeclareMathOperator{\supp}{supp}
\DeclareMathOperator{\Var}{Var}

%NT
\DeclareMathOperator{\ord}{ord}

%Alg
\DeclareMathOperator{\Rad}{Rad}
\DeclareMathOperator{\Jac}{Jac}

%Misc
\newcommand{\SL}{{\mathrm{SL}}}
\newcommand{\mobgp}{{\mathrm{PSL}_2(\mathbb{C})}}
\newcommand{\id}{{\mathrm{id}}}
\newcommand{\MCG}{{\mathrm{MCG}}}
\newcommand{\PMCG}{{\mathrm{PMCG}}}
\newcommand{\SMCG}{{\mathrm{SMCG}}}
\newcommand{\ud}{{\mathrm{d}}}
\newcommand{\Vol}{{\mathrm{Vol}}}
\newcommand{\Area}{{\mathrm{Area}}}
\newcommand{\diam}{{\mathrm{diam}}}
\newcommand{\End}{{\mathrm{End}}}


\newcommand{\reg}{{\mathtt{reg}}}
\newcommand{\geo}{{\mathtt{geo}}}

\newcommand{\tori}{{\mathcal{T}}}
\newcommand{\cpn}{{\mathtt{c}}}
\newcommand{\pat}{{\mathtt{p}}}

\let\Cap\undefined
\newcommand{\Cap}{{\mathcal{C}}ap}
\newcommand{\Push}{{\mathcal{P}}ush}
\newcommand{\Forget}{{\mathcal{F}}orget}



\title{Assignment 3}
\author{Jonas Trepiakas}
\date{}

\begin{document}
\maketitle
    \begin{exercise}[]
        Let $R$ be a Noetherian ring. Show the following
        \begin{enumerate}
            \item For every ideal
                $I \subset R$, there exists $n \in \mathbb{N} $ 
                such that $\left( \sqrt{I}  \right)^{n}
                \subset I$.
            \item Every radical ideal of $R$ is a finite
                intersection of prime ideals.
            \item If a radical ideal of $R$ is irreducible,
                then it is a prime ideal.
        \end{enumerate}
    \end{exercise}

    \begin{proof}
        (1) Since $I \subset \sqrt{I} $ are
        sub-$R$-modules of $R$ considered as a module
        over itself, we find that
        $\sqrt{I} $ must be finitely generated, so let
        $\sqrt{I}  = \left< x_1, \ldots, x_n \right>$, and
        by assumption,
        there exist $\alpha_1, \ldots, \alpha_n$ such that
        $x_i^{\alpha_i} \in I$.
        Let $\alpha = \alpha_1 + \ldots
        + \alpha_n$.
        Now let
        $x \in \sqrt{I} $ and write
        $x = \sum_{i} c_i x_i$. Then
        any term in
        $x^{\alpha}$ will contain some
        $x_i$ to the power of at
        least $\alpha_i$ by
        the pigeonhole principle. Since
        $I$ is an ideal, the whole term is in $I$, so
        again, ideals are closed under sums, so
         $x^{\alpha} \in I$.
         Since $x$ was arbitrary, we find
         that
         $\left( \sqrt{I}  \right)^{\alpha}
         \subset I$.\\
         \linebreak
         (2) Let
         $I$ be a radical ideal of $R$, so
         $\sqrt{I}  = I$.
         By theorem 7.19 (Primary decomposition),
         $I$ is the finite intersection
         of primary ideals, so
         \[
         I = \mathfrak{p}_1 \cap \ldots
         \cap \mathfrak{p_n}
         \] 
         where each $\mathfrak{p}_i$ is primary.

         \begin{lemma}[]
             For an ideal $J =
             J_1 \cap \ldots \cap J_n$, we have
             \[
             \sqrt{J}  = \sqrt{J_1}  \cap
             \ldots \cap \sqrt{J_n} 
             \] 
         \end{lemma}
         \begin{proof}
             Suppose
             $x \in \sqrt{J} $ so
             $x^{i} \in J = 
             J_1 \cap \ldots \cap J_n$, then
             $x^{i} \in J_j$ for all $j$ so
             $x \in \sqrt{J_j} $ for all $J$, so
             $x \in \sqrt{J_1}  \cap \ldots
             \cap \sqrt{J_n} $.
             Conversely, if
             $x \in \sqrt{J_1} \cap \ldots
             \cap \sqrt{J_n} $ then
             there exist  $\alpha_1, \ldots, \alpha_n$ 
             such that
             $x^{\alpha_i} \in J_i$.
             Let $\alpha = \max_{i} \left\{ \alpha_i \right\} $.
             Then
             $x^{\alpha} \in 
             J_1 \cap \ldots \cap J_n = J$, so
             $x \in \sqrt{J} $.
         \end{proof}
         Hence we obtain
         \[
         I = \sqrt{I}  =
         \sqrt{\mathfrak{p}_1}  \cap
         \ldots \cap
         \sqrt{\mathfrak{p}_n} .
         \] 
         To finish it off, we note that
         by Lemma 7.11,
         each $\sqrt{\mathfrak{p}_i} $ is prime.\\
         \linebreak
         (3) Suppose
         $I \subset R$ is a radical ideal which is
         irreducible.
         By Lemma 7.16, $I$ is primary, and now by
         Lemma 7.11, 
         $I = \sqrt{I} $ is prime.
    \end{proof}


    \begin{exercise}[]
        Let $V \subset K^{n}$ be an affine
        algebraic set. Show the following.
        \begin{enumerate}
            \item $V$ is irreducible if and only if
                $\mathbb{I}(V)$ is a prime ideal.
            \item $V$ can be written as a finite
                union of irreducible affine algebraic
                sets.
            \item There is a minimal decomposition
                $V = V_1 \cup \ldots \cup 
                V_m$ of $V$ into irreducible affine
                algebraic sets $V_i$, where
                $m \in \mathbb{N}_0$. This is meant in the
                sense that no $V_i$ is contained in
                $\bigcup_{j \neq i} V_j$.
            \item The minimal decomposition $V = 
                V_1 \cup \ldots \cup V_m$ is unique,
                up to reordering of 
                $V_1,\ldots, V_m$. We
                call $V_1, \ldots, V_m$ the irreducible
                components of $V$.
        \end{enumerate}
    \end{exercise}


    \begin{proof}
        (1) Since $V \subset K^{n}$ is an affine
        algebraic set, there exists
        some ideal $I \subset k\left[ x_1, \ldots,
        x_n\right] $ such that
        $V = \mathbb{V}(I)$.
        Suppose
        $V = V_1 \cap V_2$ with
        both $V_1$ and $V_2$ being affine
        algebraic sets properly containing
        $V$.
        Then
        $\mathbb{I}(V) \subset 
        \mathbb{I}(V_1) \cap \mathbb{I}(V_2)$ since
        any polynomial vanishing on $V$ must vanish on
        both $V_1$ and on $V_2$. But now any
        prime ideal is irreducible, so
        $\mathbb{I}(V_1) = \mathbb{I}(V)$  or
        $\mathbb{I}(V_2) = \mathbb{I}(V)$.
        Suppose without loss
        of generality that
        $\mathbb{I}(V_2) = \mathbb{I}(V)$. Then
        $V_2 = \mathbb{V} (\mathbb{I}(V_2)) = 
        \mathbb{V} (\mathbb{I}(V)) = V$.
        For this, we need to show that
        $\mathbb{V} \left( \mathbb{I} (W) \right) 
        = W$ when $W$ is an affine algebraic set.
        But $\mathbb{I} \left( \mathbb{V}(U) \right) 
        \subset U$ always, so since
        $\mathbb{V}$ is containment-reversing, we get
        $W = \mathbb{V}(U) \subset 
        \mathbb{V} \left( \mathbb{I} (W) \right) $.
        For the opposite direction, we simply have
        that if
        $x \in \mathbb{V} \left( \mathbb{I}(W) \right) $, then
        any $f \in \mathbb{I}(W)
        = \mathbb{I}\left( \mathbb{V}(U) \right)
        \subset U$ vanishes on
        $x$. Suppose
        $x \not\in W = \mathbb{V}(U)$.
        Then there exists some $g \in U$ such that
        $g(x) \neq 0$. But 
        $g \in \mathbb{I} \left( \mathbb{V}
        (U) \right) = \mathbb{I}(W)$ by definition
        which gives a contradiction.
        Hence
        $\mathbb{V}\left( \mathbb{I}(W) \right) 
        \subset W$.
        Having concluded that
        $V = V_1$ or $V = V_2$, this shows that
        $V$ is irreducible.\\
        \linebreak

        \textbf{A faster way to see this, I suppose
        would be the following:}
        If $V = V_1 \cup  V_2$, then
        $\mathbb{I}(V_1) \cap
        \mathbb{I}(V_2) \subset 
        \mathbb{I}(V)$, showing that 
        $\mathbb{I}(V)$ is not irreducible, contradicting
        lemma 7.3.\\
        \linebreak
        
        Conversely, if
        $\mathbb{I}(V)$ is not prime, let
        $fg \in \mathbb{I}(V)$ such that
        $f,g \not\in \mathbb{I}(V)$.
        Then
        $V = 
        \mathbb{V} \left( \mathbb{I}(V) \right) 
        \subset \mathbb{V}
        \left( (f) (g) \right) 
        \subset \mathbb{V}(f) \cap
        \mathbb{V}(g)$ using that
        $\mathbb{V}$ is inclusion-reversing. 
        Now by assumption,
        if $V = \mathbb{V}(f)$, then
        $f$ would vanish on all of $V$, contradicting
        $f \not\in \mathbb{I}(V)$. Similarly for
        $g$. Hence
        $V$ is shown to not be irreducible.





        (2) Since $V$ is an affine algebraic set,
        there exists an ideal $I$ such that
        $V = \mathbb{V}(I)$. Now,
        $I \subset k\left[ x_1, \ldots, x_n \right] $ which
        is Noetherian by applying
        Hilbert's basis theorem
        iteratively since a field is Noetherian (having
        only $\left( 0 \right) $ and itself as ideals) considered
        as $k$-modules. This in particular
        gives us a decomposition
        \[
        I = \mathfrak{p}_1 \cap \ldots \cap
        \mathfrak{p}_n
        \] 
        where each $\mathfrak{p}_i$ is primary. Hence
        \[
        \mathbb{V}(I)
        = \mathbb{V}(\mathfrak{p}_1) \cup 
        \ldots \cup 
        \mathbb{V}\left( \mathfrak{p}_n \right) .
        \] 
        To show that
        $\mathbb{V}\left( \mathfrak{p}_i \right) $ 
        is irreducible, we can
        show that
        $\mathbb{I} \left( \mathbb{V}
        \left( \mathfrak{p}_i \right) \right) $ is a
        prime ideal.
        This can be easily achieved if we may use
        Nullstellensatz since then
        $\mathbb{I} \left( \mathbb{V}\left( 
        \mathfrak{p}_i\right)  \right) 
        = \sqrt{\mathfrak{p}_i} 
        $ which is prime by Lemma 7.11.\\
        \linebreak
        (3) By part (2), 
        $V$ can be decomposed as
        $V = V_1 \cup \ldots \cup  V_n$ where
        each $V_i$ is an irreducible affine
        algebraic set. Suppose now that
        $V_1 \subset \bigcup_{i=2}^{n}V_i $.
        But then
        \[
        V_1 = \bigcup_{i=2}^{n} \left( V_1 \cap
        V_i \right).
        \] 
        Now, the intersection of affine algebraic sets
        is still an affine algebraic set since
        $\mathbb{V}(I_1) \cap
        \mathbb{V}(I_2) = 
        \mathbb{V}\left( I_1 \cup I_2 \right) $.
       Similarly, a union of finitely many affine
       algebraic sets is also an affine algebraic set since
       $\mathbb{V}(I_1 \cdots I_n)
       = \mathbb{V}(I_1) \cap \ldots \cap
       \mathbb{V}(I_n)$.
       So by irreducibility of $V_1$, either
       $V_1 = V_1 \cap V_2$ or
       $V_1 = \bigcup_{i=3}^{n} V_1 \cap V_i$.
       Inductively, we obtain that for some
       $i\ge 2$, 
       $V_1 = V_1 \cap V_2$, i.e., 
       $V_1 \subset  V_2$. Hence we may
       discard $ V_1$ from the collection, so
       $V = V_2 \cup  \ldots \cup  V_n$.
       Thus if we have a collection 
       $V = V_1 \cup \ldots \cup  V_n$ such that
       $V_i \subset \bigcup_{j\neq i} V_j$, then we
       can simply discard $V_i$. We can continue to do so and
       after at most $n-1$ steps, we will obtain a minimal
       decomposition.\\
       \linebreak
       (4) Suppose
       \[
       V = V_1 \cup \ldots \cup  V_m = 
       W_1 \cup  \ldots \cup W_n
       \] 
       are two minimal decompositions.
       Then
       $W_i \subset 
       V_1 \cup  \ldots \cup V_m$, so
       \[
       W_i = \bigcup_{j=1}^{m} V_j \cap W_i
       \] 
       By part (3), this is a union of affine algebraic sets,
       so we completely equivalently obtain that
       $W_i = V_j \cap W_i$ for some $j$.
       Hence $W_i \subset V_j$.
       For each  $i$, let
       $j_i$ be such that
       $W_i \subset V_{j_i}$.
       Repeating this the other way around, we obtain
       $i_k$ such that
       $V_k \subset W_{i_k}$.
       Now
       $V_k \subset W_{i_k} \subset 
       V_{j_{i_k}}$. So since the decomposition is minimal,
       we must have
       $k = j_{i_{k}}$, so
       $V_k = W_{i_k}$ for all $k$.
       This in particular gives an injective map
       $\left\{ 1, \ldots,m \right\} \to 
       \left\{ 1, \ldots, n \right\} $, so
       $m\le n$.
       And similarly, $W_i = V_{j_i}$ for all
       $i$, so we similarly get $n\le n$.
       This implies that
       $m = n$ and that indeed the decompositions
       are the same up to reordering, namely by the reordering
       $\sigma \colon k \mapsto i_k$ giving
       $V_{k} = 
       W_{\sigma (k)}$.
    \end{proof}



    %\bibliography{../refs.bib}
\end{document}
