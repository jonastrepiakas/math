\documentclass[reqno]{amsart}
\usepackage{amscd, amssymb, amsmath, amsthm}
\usepackage{graphicx}
\usepackage[colorlinks=true,linkcolor=blue]{hyperref}
\usepackage[utf8]{inputenc}
\usepackage[T1]{fontenc}
\usepackage{textcomp}
\usepackage{babel}
%% for identity function 1:
\usepackage{bbm}
%%For category theory diagrams:
\usepackage{tikz-cd}

%\usepackage[backend=biber]{biblatex}
%\addbibresource{.bib}


\setlength\parindent{0pt}

\pdfsuppresswarningpagegroup=1

\newtheorem{theorem}{Theorem}[section]
\newtheorem{lemma}[theorem]{Lemma}
\newtheorem{proposition}[theorem]{Proposition}
\newtheorem{corollary}[theorem]{Corollary}
\newtheorem{conjecture}[theorem]{Conjecture}

\theoremstyle{definition}
\newtheorem{definition}[theorem]{Definition}
\newtheorem{example}[theorem]{Example}
\newtheorem{exercise}[theorem]{Exercise}
\newtheorem{problem}[theorem]{Problem}
\newtheorem{question}[theorem]{Question}

\theoremstyle{remark}
\newtheorem*{remark}{Remark}
\newtheorem*{note}{Note}
\newtheorem*{solution}{Solution}



%Inequalities
\newcommand{\cycsum}{\sum_{\mathrm{cyc}}}
\newcommand{\symsum}{\sum_{\mathrm{sym}}}
\newcommand{\cycprod}{\prod_{\mathrm{cyc}}}
\newcommand{\symprod}{\prod_{\mathrm{sym}}}

%Linear Algebra

\DeclareMathOperator{\Span}{span}
\DeclareMathOperator{\im}{im}
\DeclareMathOperator{\diag}{diag}
\DeclareMathOperator{\Ker}{Ker}
\DeclareMathOperator{\ob}{ob}
\DeclareMathOperator{\Hom}{Hom}
\DeclareMathOperator{\Mor}{Mor}
\DeclareMathOperator{\sk}{sk}
\DeclareMathOperator{\Vect}{Vect}
\DeclareMathOperator{\Set}{Set}
\DeclareMathOperator{\Group}{Group}
\DeclareMathOperator{\Ring}{Ring}
\DeclareMathOperator{\URing}{URing}
\DeclareMathOperator{\Ab}{Ab}
\DeclareMathOperator{\Top}{Top}
\DeclareMathOperator{\hTop}{hTop}
\DeclareMathOperator{\Htpy}{Htpy}
\DeclareMathOperator{\Cat}{Cat}
\DeclareMathOperator{\CAT}{CAT}
\DeclareMathOperator{\Cone}{Cone}
\DeclareMathOperator{\dom}{dom}
\DeclareMathOperator{\cod}{cod}
\DeclareMathOperator{\Aut}{Aut}
\DeclareMathOperator{\Mat}{Mat}
\DeclareMathOperator{\Fin}{Fin}
\DeclareMathOperator{\rel}{rel}
\DeclareMathOperator{\Int}{Int}
\DeclareMathOperator{\sgn}{sgn}
\DeclareMathOperator{\Homeo}{Homeo}
\DeclareMathOperator{\SHomeo}{SHomeo}
\DeclareMathOperator{\PSL}{PSL}
\DeclareMathOperator{\Bil}{Bil}
\DeclareMathOperator{\Sym}{Sym}
\DeclareMathOperator{\Skew}{Skew}
\DeclareMathOperator{\Alt}{Alt}
\DeclareMathOperator{\Quad}{Quad}
\DeclareMathOperator{\Sin}{Sin}
\DeclareMathOperator{\Supp}{Supp}
\DeclareMathOperator{\Char}{char}
\DeclareMathOperator{\Teich}{Teich}
\DeclareMathOperator{\GL}{GL}
\DeclareMathOperator{\tr}{tr}
\DeclareMathOperator{\codim}{codim}
\DeclareMathOperator{\coker}{coker}
\DeclareMathOperator{\corank}{corank}
\DeclareMathOperator{\rank}{rank}
\DeclareMathOperator{\Diff}{Diff}
\DeclareMathOperator{\Bun}{Bun}
\DeclareMathOperator{\Sm}{Sm}
\DeclareMathOperator{\Fr}{Fr}
\DeclareMathOperator{\Cob}{Cob}
\DeclareMathOperator{\Ext}{Ext}
\DeclareMathOperator{\Tor}{Tor}



%Row operations
\newcommand{\elem}[1]{% elementary operations
\xrightarrow{\substack{#1}}%
}

\newcommand{\lelem}[1]{% elementary operations (left alignment)
\xrightarrow{\begin{subarray}{l}#1\end{subarray}}%
}

%SS
\DeclareMathOperator{\supp}{supp}
\DeclareMathOperator{\Var}{Var}

%NT
\DeclareMathOperator{\ord}{ord}

%Alg
\DeclareMathOperator{\Rad}{Rad}
\DeclareMathOperator{\Jac}{Jac}

%Misc
\newcommand{\SL}{{\mathrm{SL}}}
\newcommand{\mobgp}{{\mathrm{PSL}_2(\mathbb{C})}}
\newcommand{\id}{{\mathrm{id}}}
\newcommand{\MCG}{{\mathrm{MCG}}}
\newcommand{\PMCG}{{\mathrm{PMCG}}}
\newcommand{\SMCG}{{\mathrm{SMCG}}}
\newcommand{\ud}{{\mathrm{d}}}
\newcommand{\Vol}{{\mathrm{Vol}}}
\newcommand{\Area}{{\mathrm{Area}}}
\newcommand{\diam}{{\mathrm{diam}}}
\newcommand{\End}{{\mathrm{End}}}


\newcommand{\reg}{{\mathtt{reg}}}
\newcommand{\geo}{{\mathtt{geo}}}

\newcommand{\tori}{{\mathcal{T}}}
\newcommand{\cpn}{{\mathtt{c}}}
\newcommand{\pat}{{\mathtt{p}}}

\let\Cap\undefined
\newcommand{\Cap}{{\mathcal{C}}ap}
\newcommand{\Push}{{\mathcal{P}}ush}
\newcommand{\Forget}{{\mathcal{F}}orget}




\begin{document}
\section{Sheet 0}
\begin{exercise}[1]

    \begin{enumerate}
        \item Find rings $R$ and $S$ and a nonzero
            map  $\varphi \colon R \to S$ such that
            \[
            \varphi (a+b) = \varphi (a)+\varphi (b)
            \quad \text{and} \quad
            \varphi (ab) = \varphi (a) \varphi (b)
            \] 
            for all $a,b \in R$, but
            for which $\varphi \left( 1_R \right) \neq 
            1_S$.
        \item Describe the set
            $\Hom_{\URing}
            \left( \mathbb{Z} /m, 
            \mathbb{Z}/n \right) $ for all
            possible choices of
            $n,m \in \mathbb{Z}_{\ge 0}$.
        \item Let $R$ be a ring. Prove that there
            is a bijective correspondence
            \[
            \Hom_{\URing} \left( \mathbb{Z}
            \left[ x \right] ,R\right) \cong R.
            \] 
        \item Prove that the abelian group
            $\left( \mathbb{Q} /\mathbb{Z}, + \right) $ 
            does not admit a ring structure.
            Precisely, this means that for any
            operation
            $\circ \colon \mathbb{Q} / \mathbb{Z}
            \times \mathbb{Q}/\mathbb{Z} \to 
            \mathbb{Q} / \mathbb{Z}$, the triplet
            $\left( \mathbb{Q} /\mathbb{Z},
            +, \circ\right) $ does not satisfy the ring
            axioms.
        \item Show that the set $\Hom_{\URing}
            \left( \mathbb{Z} /m, \mathbb{Z}/n \right) $ 
            does in general not admit a ring structure.
    \end{enumerate}


\end{exercise}

    \begin{proof}
        \begin{enumerate}
            \item 
        Choose
        $\varphi  \colon \mathbb{Z}/2 \to \mathbb{Z}/4$ by
        $1 \mapsto 2$.
    \item Any map $ \varphi \colon 
        \mathbb{Z} /m \to \mathbb{Z} /n$ in
        $\URing$ must take $1 \mapsto 1$, and
        is uniquely determined thereby
        since $\varphi (k) =
        \varphi \left( 1 + \ldots+ 1 \right) 
        = \varphi (1) + \ldots + \varphi (1)
        = k$.
        Therefore, $0 = \varphi (m) = 
        m$ in $\mathbb{Z} / n$, so $n  \mid m$.
        And it is clear that if $n  \mid m$, then
        $1 \mapsto 1$ is a well-defined ring
        homomorphism. Thus
        \[
        \Hom_{\URing}\left( \mathbb{Z}/m,
        \mathbb{Z}/n \right) 
        = 
        \begin{cases}
            \left\{ 1 \mapsto 1 \right\},& n  \mid m\\
            \varnothing,& \text{otherwise}
        \end{cases}.
        \] 
    \item We claim that the correspondence
        \begin{align*}
            \Hom_{\URing}\left( \mathbb{Z}[x], R \right) 
            &\to R\\
            \varphi &\mapsto \varphi (x)
        \end{align*}
        is bijective.
        Indeed, $\varphi (1) = 1$ necessarily,
        so $\varphi (k) = k$ for
        all $k \in \mathbb{Z}$, so we
        simply have
        $\varphi \left( \sum \alpha_i x^{i} \right) 
        = \sum \varphi \left( \alpha_i \right) 
        \varphi (x)^{i} 
        = \sum \alpha_i \varphi (x)^{i}$, so
        $\varphi $ is uniquely determined by
        where it sends $x$. Furthermore,
        for any $r \in R$, define 
        $\varphi_r \colon
        \mathbb{Z}\left[ x \right] \to R$ by
        $x \mapsto r$. This extends uniquely to a
        ring homomorphism
        $\mathbb{Z}\left[ x \right] \to R$.
        








    \item Suppose $\left( \mathbb{Q}/ \mathbb{Z},
        +\right) $ admits a ring structure with
        multiplication $\cdot $. Let
        $\frac{a}{b} \in \mathbb{Q} / \mathbb{Z}$ be
        the unit. 
        Then, in particular,
        $\frac{a}{b} \cdot  \frac{1}{x} = \frac{1}{x}$ for all
        $x$, so
        $\left( \frac{a}{b}-1 \right) \frac{1}{x}
        \in \mathbb{Z}$ for all $x$, meaning that
        any $x$ divides $\frac{a}{b}-1$. Thus is only
        possible if $\frac{a}{b} = 1$. But
        $1 \in \mathbb{Z}$, so
        the unit is the zero element, so
        for any $\frac{x}{y} \in \mathbb{Q} /\mathbb{Z}$, we
        have $\frac{x}{y}= \frac{x}{y}\cdot 1
        = 0$. Hence
        $\mathbb{Q} / \mathbb{Z}$ becomes the zero ring,
        but then $1$ does not act by the identity
        on any rational element in $\mathbb{Q} / \mathbb{Z}$.
    \item By (2), 
        $\Hom_{\URing}\left( \mathbb{Z}/m, \mathbb{Z}/n \right) $ 
        is either a single map or empty, and as
        the empty set is not a ring, this
        Hom set in general does not admit a ring
        structure - when it does, it must be
        the trivial one.
        \end{enumerate}
    \end{proof}


\begin{exercise}[2]
    \begin{enumerate}
        \item Let $A$ be a ring. How many
            $\mathbb{Z}$-algebra structures does $A$ admit?
        \item Find rings $R$ and $A$, such that
            $A$ admits more than one $R$-algebra structure.
    \end{enumerate}
\end{exercise}


    \begin{enumerate}
        \item $A$ admits a unique $\mathbb{Z}$-algebra
            structure since any ring homomorphism
            $\mathbb{Z} \to A$ is uniquely determined
            by $1 \mapsto 1$.
        \item Take $A = \mathbb{Z}[x]$ and
            $R$ to be any ring with more
            than one element. By exercise 1.(c),
            $\Hom_{\Ring}\left( \mathbb{Z}[x],
            R\right) \cong R$, so
            $R$ admits more than one
            $R$-algebra structure.

            However, these structures could be isomorphic
            in the sense that there
            exists maps $\varphi , \psi  \colon
            R \to R$ with $\varphi \psi = \id = \psi  \varphi $ 
            and composing one algebra structure
            $\mathbb{Z}[x] \to R$ with $\varphi $ gives
            $\psi $ and vice versa. 

            So we must find explicit examples which
            are non-isomorphic. 
            Define 
            $f,g \colon \mathbb{Z}[x] \to \mathbb{Z}/6$ by
            $f(x) = 2$ and
            $g(x) = 3$. Now, there
            is no ring homomorphism
            $\varphi \mathbb{Z} / 6 \to \mathbb{Z}/6$
            such that $\varphi \circ f = g$ since
            $0 = \varphi (0) =\varphi \circ f(3x) = g(3x)
            = 3$ gives a contradiction.
    \end{enumerate}


\begin{exercise}[3]
    This is just the 4th isomorphism theorem
    for ideals of rings.

    Define a map
    $\pi \colon \mathcal{A} \to \mathcal{B}$ by
    sending $A \to \overline{A} = A + I$.

    Suppose $\pi(A) = \pi(B)$. Then
    for any $a \in A$, there exists $b \in B$ such that
    $a-b \in I \subset A \cap B$, hence
    $a,b \in A \cap B$. Thus $A \subset B \subset A$, so
    $A = B$. 

    Now, suppose $V \in \mathcal{B}$. Let
    $A = \pi^{-1}(V)$. This is an ideal containing $I$.
    If $a,b \in A$ then $\pi(a), \pi(b) \in V$ so
    $\pi(ab) = \pi(a) \pi(b) \in V$, hence
    $ab \in \pi^{-1}(V)$. Similar closure for the rest.
    And if $r \in R$ then
    $\pi\left( ar \right) =\pi(a) \pi(r) \in V$ as
    $\pi(a) \in V$ and $V$ is an ideal, so
    $ar \in A$, hence $A$ is an ideal. This gives surjectivity.
\end{exercise}

\begin{exercise}[4]
    $\left( 2,x \right) \subset \mathbb{Z}[x]$ is not principle
    as an ideal generated over $\mathbb{Z}$.
    If $\left( 2,x \right) = \left( p(x) \right) $, then
    $p(x)  \mid 2$ implies that that the
    degree of $p$ is $0$. Now
    let $q(x)$ be such that
    $p(x)q(x) = x$ and
     $h(x)$ such that $p(x) h(x) = 2$. Then
     the degree of $h$ is $0$ and that of $q$ is $1$.
     Furthermore, as $p \in \left( 2,x \right) $, we must
     have either $p(x) = 2 k(x)$ or
     $p(x) = x l(x)$. But this implies that either
     $2 k(x) q(x) = x$, so
     $2  \mid x$ in $\mathbb{Z}[x]$, which is
     impossible, or
     $x l(x)h(x) = 2$, so
     $x  \mid 2$ in $\mathbb{Z}\left[ x \right] $ which is
     impossible by degree as $\mathbb{Z}$ is an integral
     domain. Hence
    we obtain a contradiction.
\end{exercise}

\begin{exercise}[5]
    \begin{enumerate}
        \item Let $I,J$ be ideals in a ring
            $R$. Show that we always have the inclusion
            of ideals
            \[
                \left( I+ J \right) 
                \left( I \cap J \right) 
                \subset IJ \subset I \cap J\subset 
                I+J.
            \] 
        \item For each of the inclusions above, find an example
            where it is a proper inclusion.
        \item Show that if $I + J = R$, then all
            the inclusions above are in fact equalities.
        \item Prove the following generalization of the
            Chinese Remainder Theorem. Given a ring
            $R$ and finitely many ideals
            $\left\{ I_1, \ldots, I_n \right\} $ in
            $R$, such that for each pair
            $1 \le i \neq j \le n$, one has
            $I_i + I_j = R$.
            Let
            \[
            I = \bigcap_{i} I_i .
            \] 
            Construct an isomorphism of rings
            \[
            \prod R/ I_i \cong R / I.
            \] 
    \end{enumerate}
\end{exercise}

\begin{proof}
    (1) Recall that
    $\left( I+J \right) = 
    \left( I \cup J \right) $ and that
    for ideals $N,M$, 
    $NM = \left( nm  \mid n \in N, m \in M \right) $.

    If $x \in \left( I+J \right) \left( I \cap J \right) $, then
    $x = \sum \alpha_i \beta_i$ where
    $\alpha_i \in \left( I+J \right) $ and
    $\beta_i \in \left( I \cap J \right) $.
    So let $\alpha_i = \sum c_{ij}$ where
    $c_{ij}$ is in $I$ or in $J$. Writing
    $\beta_i = \sum e_i$ where
    $e_i \in I \cap J$, we have
    $x = \sum c_{ij} e_k$. If
    $c_{ij}$ is in $I$, then
    $c_{ij} e_k \in IJ$ since
    $e_k \in J$, and similarly if  $c_{ij} \in J$.\\
    \linebreak
    Next assume $x \in IJ$, so
    $x = \sum c_i d_i, c_i \in I, d_i \in J$.
    Since $I$ and $J$ are ideals, 
    each $c_i d_i \in I \cap J$, hence
    $IJ \subset I \cap J$.\\
    \linebreak
    Lastly, 
    $x \in I \cap J$ then
    $x \in I,J$ so also in
    $I \cup  J$, hence
    $x \in  I + J$.\\
    \linebreak
    (2) Let $R = \mathbb{Z}[x]$. Then
    Let $I = (2x), J = (3x^2)$. Then
    $I + J = (2x, 3x^2)$ while
    $I \cap J = \left( 6x^2 \right) $. 
    Then 
    $\left( I+J \right) 
    \left( I \cap J \right) 
    = \left( 12 x^3,18x^{4} \right) $ while
    $IJ = \left( 6x^3 \right) $, so the containment
    is clearly proper as
    $6x^3$ is not in
    $\left( I+J \right) \left( I \cap J \right) $.\\
    Next, if $I = (x) = J$, then
    $IJ = (x^2)$ is properly contained in
    $(x) = I \cap J$.\\
    Lastly, if $I = (2x)$ and
    $J = (3x)$, then
    $I \cap J = (6x)$ while
    $I + J = (x)$ since
    $3x - 2x = x \in I + J$.\\
    \linebreak
    (3) Suppose $I + J = R$. Since
    $1 \in I + J$ then, it suffices to show
    that $I + J \subset I \cap J$.
    Let
    $M$ be the set of elements of $I+J$ which
    are not in $I \cap J$. Suppose
    $x \in M$
\end{proof}




\begin{exercise}[7]
    \begin{enumerate}
        \item Surjectivity amounts to finding an
            $f \in K \left[ x_1, \ldots, x_n \right] $ 
            such that $f(y) = k$ for some arbitrary $k \in K$.
            Consider the map
            $f(x_1,\ldots,x_n) =
            k + \left( x_1-y_1 \right) \ldots \left( x_n
            -y_n\right) $. Or the constant polynomial at $k$ works
            also. Now,
            $\varphi_y \left( f+g \right) 
            = \left( f+g \right) (y)
            = f(y) + g(y) = \varphi_y(f) + 
            \varphi_y(g)$, and
            $\varphi_y(fg) = \varphi_y(f)\varphi_y(g)$ is
            seen likewise. 

            That it is a homomorphism of
            $K$-algebras (with the
            standard $K$-algebra structure)
            amounts to showing that
            $\varphi_y\left( k  \right) 
            = k $ which is clear.
        \item Let $\varphi  \colon K \left[ x_1, \ldots, x_n
            \right]  \to K$ be a ring homomorphism. 
            Let $y_i = \varphi (x_i)$. Then
            $\varphi \left( \sum a_I x_I \right) 
            = \sum \varphi \left( a_I \right) 
             y_I = 
             \varphi_{y_I}\left( \sum a_I x_I \right) $
             (for this we need
             $\varphi $ to be a $K$-algebra
             homomorphism of with
              $K\left[ x_1, \ldots,x_n \right] $ 
              in the standard structure.
              Is there a different way of arguing?)
    \end{enumerate}
\end{exercise}

\section{Sheet 1}

    \begin{exercise}[1]
        \begin{proof}
            (i)
                We claim that 
                    $\left( x^2+1 \right) $ is a radical,
                    prime and maximal
                    ideal in $\mathbb{R}\left[ x \right] $.
                    This can be seen
                    by noting that
                    $\mathbb{R}\left[ x \right] /
                    \left( x^2 +1 \right) \cong \mathbb{C}$ which
                    is a field. Hence
                    $\left( x^2+1 \right) $ is
                    maximal. Suppose
                    $f^{n} \in \left( x^2+1 \right) $.
                    Since $x^2 +1$ is irreducible
                    and $x^2 +1  \mid f^{n}$, we must
                    have $x^2 + 1  \mid f$, hence
                    $f \in \left( x^2 +1 \right) $,
                    so $\sqrt{\left( x^2+1 \right) } 
                    =\left( x^2 +1 \right) $.
                Over $\mathbb{C}\left[ x \right] $,
                    we claim that
                    $\left( x^2+1 \right) $ is neither
                    prime nor maximal, but still radical.
                    It is not prime as
                    $x^2+1 = \left( x+i \right) \left( x-i \right) $
                    and hence also not maximal
                    since $\left( x^2+1 \right) 
                    \subset \left( x+i \right) \neq 
                    \mathbb{C}[x]$, where
                    inequality follows from 
                    $\left( x+i \right) $ only
                    having polynomials of degree $\ge 1$.

                    Now suppose
                    $f^{n} \in \left( x^2+1 \right) $.
                    Then $x+i, x-i  \mid f^{n}$, hence
                    both must divide $f$ as they are
                    irreducible, so
                    $x^2 +1  \mid f$. Thus
                    $f \in \left( x^2+1 \right) $, so
                    $\sqrt{\left( x^2+1 \right) } 
                    = \left( x^2+1 \right)  $ over
                    $\mathbb{C}\left[ x \right] $
                    as well.\\
                    \linebreak
                    (ii) Since $\left( x^2 +1 \right) $  is
                    a prime ideal in $
                    \mathbb{R}[x]$ by the previous exercise,
                    we find by Eisenstein's criterion that
                    $y^2 + x^2 + 1$ is irreducible
                    in $\mathbb{R}[x][y] =:
                    \mathbb{R}[x,y]$.\\
                    \linebreak
                    

                    (iii)
                 
                    Let $C =
                    \left\{ f \in 
                    C\left( \mathbb{R}^2,\mathbb{R} \right)
                 \mid \forall x \in \mathbb{R}\colon
             f\left( x,0 \right) =0 \right\}
             \subset C\left( \mathbb{R}^2,\mathbb{R} \right)$.
             We claim that $C$ is radical, but neither
             prime nor maximal.

             To see that it is radical, suppose
             $g^{n} \in C$, so
             $g(x,0) \cdot \ldots \cdot 
             g(x,0) = g^{n}(x,0) = 0$. Since
             $\mathbb{R}$ is an integral domain,
             this forces $g(x,0) = 0$, so
             $g \in C$. Thus $\sqrt{C}  = C$.

             Now let
             $h(x) = \mathbbm{1}_{\ge 0}(x) x $
             and
             $k(x) = \mathbbm{1}_{\le 0}(x) x$. Then
             $h,k \not\in C$, but
             $hk \in C$. Therefore,
             $C$ is not prime.
             Since $C\left( \mathbb{R}^2,\mathbb{R} \right) $ is
             a commutative ring and maximal ideals are
             prime over a commutative ring, we
             thus also conclude that $C$ is not maximal.\\
             \linebreak
             (iv) The ideal $\left( 5 \right) $ in
             $\mathbb{Z}[i]$ is not prime, hence
             not maximal as $\mathbb{Z}[i]$ is commutative.
             It is not prime because $5 = 
             (2+i) (2-i)$.

             For the radical part, if
             $a+bi \in \sqrt{(5)} $, then
             $(2+i)(2-i)  \mid \left( a+bi \right)^{n}$,
             so $2+i  \mid a+bi$ and
             $2-i  \mid a+bi$ since
             each is irreducible, hence
             $5  \mid a+bi$, so
             $\sqrt{\left( 5 \right) } =
             \left( 5 \right) $.\\
             \linebreak
             
             (v) We claim that
             $\left( n \right) \subset \mathbb{Z}$ 
             is prime and maximal whenever
             $n$ is a prime and not otherwise.
             If $n$ is not prime, then
             writing $n = ab$ for $a,b >1$, we
             have $\left( n \right) 
             = \left( a \right) (b)$, so $\left( n \right) $ is
             not prime, hence not maximal as
             $\mathbb{Z}$ is commutative so
             all maximal ideals are prime ideals.
             If instead $n$ is a prime, $n=p$, then
             $\left( p \right) $ is both
             maximal and prime since
             $\mathbb{Z} / p$ is a field.


             Suppose now
             $m \in \sqrt{\left( n \right) } $.
             Then $m^{k} \in \left( n \right) $, so
             $m^{k} = nq$ for some $q \in \mathbb{Z}$.
             Suppose $n$ is squarefree. Let
             $p  \mid n$. Then $p  \mid m^{k}$ and thus
             $p  \mid m$, so $n  \mid m$, hence
             $m \in \left( n \right) $. Conversely,
             if $n$ is not squarefree, then
             letting $n = p^k q$ for some $k>1$, we have
             $p^{k-1}q \in \sqrt{\left( n \right) } $ while
             $p^{k-1} q \not\in \left( n \right) $, so
             $\left( n \right) $ is not radical.
        \end{proof}
    \end{exercise}



    \begin{exercise}[2]
        Let $n \in \mathbb{N}$. We denote the set
        of orthogonal matrices on
        $\mathbb{R}^{n}$ by
        \[
        O \left( \mathbb{R}^{n} \right) 
        = \left\{ A \in \mathbb{R}^{n\times n}
         \mid A^{T} A = I_n \right\} 
        \] 
        Write $R = 
        \mathbb{R} \left[ x_{ij} \mid 
        i,j = 1,\ldots, n\right] $ for the polynomial
        ring in variables $X = \left( x_{ij} \right)_{i,j
        = 1,\ldots, n}$. 
        \begin{enumerate}
            \item Show that $O \left( \mathbb{R}^{n} \right) $ 
                is the zero set 
                $\mathbb{V} (I)$ of the ideal
                $I = 
                \left( \left\{ f_{ij}  \mid 
                i,j = 1,\ldots,n\right\}  \right) $  of
                $R$ where
                \[
                f_{i,j} = \sum_{k=1}^{n} x_{ki}x_{kj}
                \quad \text{for } i\neq j
                \quad \text{and} \quad
                f_{ii} = \sum_{k=1}^{n} x_{ki}^2 -1.
                \] 
            \item Show that $O \left( \mathbb{R}^{n} \right) $ 
                is also the real zero set of the ideal
                $J = \left( g,h \right) $ generated by
                the polynomials
                $g = \det (X)^2 -1$ and
                $h = \sum_{i,j=1}^{n} x_{ij}^2 -n$.
        \end{enumerate}
    \end{exercise}
    \begin{proof}
        (a) We know that $A = 
        \left( \alpha_{ij} \right)
        \in O \left( \mathbb{R}^{n} \right) $ if and only
        if $A^{T} A = I$.
        Taking entries of either side of this equality, we
        get that $A$ is orthogonal if and only if
        both of the following conditions hold:
        \begin{align*}
            0 &= \left( A^{T}A \right)_{ij}
            = \sum_{k=1}^{n} \left( A^{T} \right)_{ik}
            A_{kj}
            = \sum_{k=1}^{n} \alpha_{ki} \alpha_{kj}\\
            1 &= \left( A^{T}A \right)_{ii}
            = \sum_{k=1}^{n} \alpha_{ki}^2.
        \end{align*}
        That is,
        $A \in O\left( \mathbb{R}^{n} \right) $ if and
        only if
        $A \in \mathbb{V} (I)$ where we identify
        $R \cong M_{n} \left( \mathbb{R} \right) $ by
        $\sum \alpha_{ij} x_{ij}
        \mapsto \left( \alpha_{ij} \right) $.\\
        \linebreak
        (b) Firstly, suppose
        $A \in O \left( \mathbb{R}^{n} \right) $.
        Then $A^{T}A = I$. Therefore,
        $\det (A)^2 -1 = 
        \det (A^{T}) \det (A) -1
        = \det (A^{T}A) -1 
        = \det (I) - 1
        = 0$, so
        $g(A) = 0$. And now
        \[
        n = \tr (I) = \tr (A^{T}A)
        = \sum_{k=1}^{n} \left( A^{T}A \right)_{kk}
        = \sum_{k,r=1}^{n} \alpha_{rk}^2
        \] 
        hence also $h(A) = 0$.
        Therefore
        $O \left( \mathbb{R}^{n} \right) 
        \subset \mathbb{V} (J)$.\\
        Conversely,
        suppose
        $A = \left( \alpha_{ij} \right)  \in \mathbb{V}(J)$.
        Firstly, note that since
        \[
        x^{T} A^{T}A x
        = \left( Ax \right)^{T} Ax
        = \|Ax\|^2 \ge 0,
        \] 
        the matrix $A^{T}A$ is positive semi-definite,
        hence all its eigenvalues are non-negative real numbers.
        In particular,
        $\sqrt[n]{\Pi_{i=1}^{n} \lambda_i} $ is well-defined
        where
        $\lambda_1, \ldots, \lambda_m$ are the eigenvalues
        of $A^{T}A$ listed
        with repetition, and
        we can apply the AMGM inequality.
        Recall that the AMGM inequality
        tells us that
        \[
        \frac{\sum_i \lambda_i}{n} \ge 
        \sqrt[n]{\prod_i \lambda_i} 
        \] 
        with equality if and only if all the
        $\lambda_i$ are equal. But 
        by Jordan normal form,
        $\tr \left( A^{T}A \right) =
        \sum_{i} \lambda_i$, and
        $\det \left( A^{T}A \right) =
        \prod_i \lambda_i$, so we obtain
        \[
        \tr(A^{T}A) \ge n \det \left( A^{T}A \right) 
        \] 
        with equality if and only if all eigenvalues are equal.
        But since $A \in \mathbb{V} (J)$, we have
        $g (A) = 0$, so
        $\det \left( A^{T}A \right) 
        = \det (A)^2 = 1$. Likewise,
        \[
        \tr \left( A^{T}A \right) 
        = \sum_{k=1}^{n} \left( A^{T}A \right)_{kk} 
        = \sum_{k,r=1}^{n} \alpha_{rk}^2
        = h(A)+n = n
        \] 
        Thus we get
        \[
        n = \tr\left( A^{T}A \right) \ge 
        n \det \left( A^{T}A \right) = n
        \] 
        so we conclude that, since we have equality between
        the two sides, all eigenvalues of
        $A^{T}A$ must be equal.
        Now since
        $A^{T}A$ is self-adjoint, it is in particular
        diagonable, hence it has
        precisely $n$ eigenvalues counted with multiplicity.
        Therefore, it has one eigenvalue with multiplicity $n$.
        Letting wlog $\lambda$ denote this eigenvalue, we
        get
        \[
        n \lambda = \tr \left( A^{T}A \right) 
        = n
        \] 
        so $\lambda = 1$ since $\mathbb{R}$ is an integral
        domain.
        To see that this forces $A^{T}A$ to be $I$, 
        we note that since $A^{T}A$ is diagonable, we can
        find some invertible linear map
        $P \in \GL_n \left( \mathbb{R} \right) $ such that
        $P A^{T}A P^{-1} = I$, implying
        $A^{T}A = I$. Thus $A$ is orthogonal, so
        $A \in O \left( \mathbb{R}^{n} \right) $.
        This gives the inclusion
        $\mathbb{V} (J) \subset O \left( \mathbb{R}^{n} \right) $.
    \end{proof}


\section{Sheet 3}

\begin{exercise}[2]
    Let $R$ be a Noetherian ring. Show that every ideal of
    $R$ is a finite intersection of
    irreducible ideals.
\end{exercise}

\begin{proof}
    Let
    $\mathcal{A}$ be the set of ideals
    of $R$ which are not a finite intersection of
    irreducible ideals.
    Suppose
    $\left\{ X_i \right\}
    = \mathcal{X} \subset \mathcal{A}$ is a chain with
    respect to inclusion.
    Then since $R$ is Noetherian, this chain stabilizes, hence
    it has an upper bound. Thus
    $\mathcal{A}$ has a maximal element, call it
    $M$. Now, $M$ is not a finite intersection
    of irreducible ideals, so in particular,
    $M$ is not irreducible, so write
    $M = I \cap J$ where
    $I$ and $J$ contain $M$ properly. But then
    $I,J \not\in \mathcal{A}$, so
    they are finite intersections of irreducible ideals.
    However, then
    their intersection is also a finite intersection of
    irreducible ideals.
\end{proof}

\begin{exercise}[3]
    Let $R$ be a ring. Show that an ideal
    $\mathfrak{B}$ of $R$ is a prime ideal if and only
    if for all ideals $I$ and $J$ of $R$,
    $IJ \subset \mathfrak{B}$ implies
    $I \subset \mathfrak{B}$ or
    $J \subset \mathfrak{B}$.
\end{exercise}

\begin{proof}
    If $\mathcal{B}$ is a prime ideal, then
    suppose $IJ \subset \mathcal{B}$, and suppose
    $I \not \subset  \mathcal{B}$.
    Then there exists some $i \in I$ such that
    $i \not\in \mathcal{B}$ but
    $i J \subset \mathcal{B}$.
    Since $\mathcal{B}$ is prime,
    we must have that
    $J \subset \mathcal{B}$.\\
    Conversely, suppose that for all $I,J$,
    $IJ \subset \mathcal{B}$ implies $I \subset \mathcal{B}$
    or $J \subset \mathcal{B}$. Let
    $a,b \in R$ such that
    $ab \in \mathcal{B}$. Then

    $\left( a \right)  \left( b \right)
    \subset \mathcal{B}$, so either
    $\left( a \right)  \subset \mathcal{B}$ or
    $\left( b \right)  \subset \mathcal{B}$, so
    either $a \in \mathcal{B}$ or $b \in \mathcal{B}$.
    So $\mathcal{B}$ is prime.
\end{proof}



\begin{exercise}[4]
    Let $R_S$ be the localization of the integral domain
    $R$ by a multiplicative subset
    $S$ which does not contain $0$. Let
    $\mathfrak{U}$ be a primary ideal of
    $R$. Show that the extension
    $\mathfrak{U}R_S$ is a primary ideal of
    $R_S$ and that $R \cap \mathfrak{U}R_S =
    \mathfrak{U}$.
\end{exercise}

\begin{proof}
    Let $\tau \colon R \to R_S = \left( R-S \right)^{-1}R$.
    Recall that
    $\mathfrak{U}R_S = \left( \tau \left(
    \mathfrak{U}\right)  \right) $.
    Suppose
    $ab \in \mathfrak{U}R_S$. Then
    there exists $\sum \alpha_i u_i
    \in \mathfrak{U}$ such that
    $ab = \sum \alpha_i \tau(u_i)
    = \sum \alpha_i \frac{u_i}{1}$.
    We can write
    $a = \frac{x}{y}$ and
    $b = \frac{v}{w}$.
    Then
    $xv \in \mathfrak{U}$, so
    either $x \in \mathfrak{U}$, in which case
    $\frac{x}{y} \in \mathfrak{U}R_S$, or
    $v^{n} \in \mathfrak{U}$, in which case
    $\left( \frac{v}{w} \right)^{n} \in
    \mathfrak{U}R_S$ since $S$ is multiplicative, so
    $w^{n} \in S$. Thus
    $a \in \mathfrak{U}R_S$ or
    $b^{n} \in \mathfrak{U}R_S$. Hence
    $\mathfrak{U}R_S$ is primary in $R_S$.\\
    Now recall that
    $R \cap \mathfrak{U}R_S$ denote the contraction
    of $\mathfrak{U}R_S$ along $\tau$, i.e.,
    $R \cap \mathfrak{U}R_S =
    \tau^{-1} \left( \mathfrak{U}R_S \right) $.
    Clearly,
    $\mathfrak{U} \subset
    R \cap \mathfrak{U}R_S$.
    Conversely, suppose
    $a \in \tau^{-1}\left( \mathfrak{U}R_S \right) $.
    Then
    $\frac{a}{1} \in \mathfrak{U}R_S$, so
    $\frac{a}{1} =
    \sum \alpha_i \frac{u_i}{1}$. So
    there exists some
    $r \in R-S$ such that
    $ra = r \sum \alpha_i u_i$. So
    since $r \neq 0$ as $0 \not\in S$, we have
    $a = \sum \alpha_i u_i$ since $R$ is an integral domain.
    Thus $a \in \mathfrak{U}$.
\end{proof}




\begin{exercise}[5]
    Let $R$ be a ring and $\mathfrak{U}$ a
    $\mathfrak{B}$-primary ideal. Show the following.
    \begin{enumerate}
        \item For all $x \in \mathfrak{U}$, we have
            $\left( \mathfrak{U} \colon x
            \right) = R$.
        \item For all $x \in R - \mathfrak{U}$,
            we have that
            $\left( \mathfrak{U} \colon x \right) $ is
            $\mathfrak{B}$-primary.
        \item For all $x \in
            R - \mathfrak{B}$, we have
            that $\left( \mathfrak{U}\colon x \right)
            = \mathfrak{U}$.
        \item If $R$ is Noetherian, then there is some
            $x \in R - \mathfrak{U}$ such that
            $\left( \mathfrak{U} \colon x \right)
            = \mathfrak{B}$.
    \end{enumerate}
\end{exercise}

\begin{proof}
    So $\sqrt{\mathfrak{U}} = \mathfrak{B}$ by assumption.\\
    \linebreak
    (1) Since $\mathfrak{U}$ is an ideal,
    $x \mathfrak{U} \subset  \mathfrak{U}$ for
    all $x \in R$, so
     for all $x \in \mathfrak{U}$,
     $\left( \mathfrak{U} \colon x \right) = R$.\\
     (2) Suppose $a \in \sqrt{\left( \mathfrak{U}\colon
     x\right) } $, so
     $a^{n} x \in \mathfrak{U}$. Since $\mathfrak{U}$ is primary,
     either $x \in \mathfrak{U}$, which we have assumed it is
     not, or $a^{nm} \in \mathfrak{U}$ for some $m$. Hence
     the latter must be true. So
     $a \in \sqrt{\mathfrak{U}} = \mathfrak{B}$.\\
     Hence
     $ \sqrt{\left( \mathfrak{U} : x \right) } \subset
     \mathfrak{B}$.
     Conversely, suppose
     $a \in \mathfrak{B}$, so
     $a^{n} \in \mathfrak{U}$. So for
     any $x \in R - \mathfrak{U}$, we have
     $a^{n} x \in \mathfrak{U}$, so
     $a \in \sqrt{\left( \mathfrak{U} : x \right) } $.\\
     (3) Suppose
     $x \in R - \mathfrak{B}$.
     Now if $y \in \mathfrak{U}$ then
     $xy \in \mathfrak{U}$, so
     $y \in \left( \mathfrak{U} : x \right) $.
     Conversely, suppose
     $y \in \left( \mathfrak{U} : x \right) $.
     So $xy \in \mathfrak{U}$, so
     since  $x \not\in \mathfrak{B} =
     \sqrt{\mathfrak{U}} $, we must have that
     $y \in \mathfrak{U}$.\\
     (4) Let $n$ be minimal such that
     $\mathfrak{B}^{n} \subset \mathfrak{U}$.
     Let
     $x \in \mathfrak{B}^{n-1} - \mathfrak{U}$.
     We claim
     $\left( \mathfrak{U} : x \right) = \mathfrak{B}$.
     For $b \in \mathfrak{B}$, we have
     $bx \in \mathfrak{B}^{n} \subset
     \mathfrak{U}$, so
     $b \in \left( \mathfrak{U} : x \right) $.
     Conversely, if
     $bx \in \mathfrak{U}$, then
     since  $x \not\in U$, we have
     $b^{m} \in \mathfrak{U}$, so
     $b \in \sqrt{U} = \mathfrak{B}$.
\end{proof}


\section{Assignment 2}

    \begin{exercise}[]
        Given two local rings $\left( R,
        \mathfrak{m} \right) $ and
        $\left( S, \mathfrak{n} \right) $ be local rings.
        A ring homomorphism
        $f \colon R \to S$ is called a local
        homomorphism if the image of
        $\mathfrak{m}$ under $f$ is contained in
        $\mathfrak{n}$.
        \begin{enumerate}
            \item Let $g \colon A \to B$ be a ring
                homomorphism between two arbitrary
                rings and let 
                $\mathfrak{p} \subset B$ and
                $\mathfrak{q} = g^{-1}(\mathfrak{p}) \subset A$ be
                prime ideals. Show that
                $g$ localizes to a ring homomorphism
                $A_{\mathfrak{q}} \to 
                B_{\mathfrak{p}}$ to be more precise, let
                $\pi_{A} \colon A \to A_{\mathfrak{q}}$ 
                and $\pi_{B} \colon B \to 
                B_{\mathfrak{p}}$ be the natural ring
                homomorphisms from the
                rings to their localizations. You need
                to construct a ring homomorphism
                $g' \colon
                A_{\mathfrak{q}} \to 
                B_{\mathfrak{p}}$ such that
                $\pi_{B} \circ g = g' \circ \pi_A$. Show that
                the map $g'$ you construct is a local
                homomorphism.
            \item Find a ring homomorphims between
                local rings which is not a local
                homomorphism.
        \end{enumerate}
    \end{exercise}

\begin{proof}
    (1) Consider the diagram

    \begin{equation*}
    \begin{tikzcd}
        A \ar[d, "\pi_A"'] \ar[r, "g"] & B \ar[d, "\pi_B"] \\
        A_{\mathfrak{q}} \ar[r, "g'"', dashed] & B_{\mathfrak{p}}
    \end{tikzcd}
    \end{equation*}
    Note that
    by the universal property of localizations, 
    the map $g'$ exists if and only if
    $\pi_B \circ g \left( \left( A- \mathfrak{q} \right) 
    \right) \subset 
    B_{\mathfrak{p}}^{\times }
    = \left( R - \mathfrak{p} \right)_{\mathfrak{p}}$.
    Let 
    $a \in A - \mathfrak{q}$. By assumption, 
    $g (a) \not\in \mathfrak{p}$, so
    $\pi_B \left( g (a) \right) = \frac{g(a)}{1}
    \in \left( R - \mathfrak{p} \right)_{\mathfrak{p}} $.
    This gives the existence of $g'$.\\
    Next we show that
    $g'$ is a local homomorphism. 
    Note that $A_{\mathfrak{q}}$ and
    $B_{\mathfrak{p}}$ are local rings
    with unique maximal ideals
    $\mathfrak{q}_{\mathfrak{q}}$ and
    $\mathfrak{p}_{\mathfrak{p}}$, respectively. Thus, to
    show that $g'$ is a local homomorphism, we must show
    that $g' \left( \mathfrak{q}_{\mathfrak{q}} \right) 
    \subset \mathfrak{p}_{\mathfrak{p}}$.
    Explicitly, 
    \begin{align*}
        \mathfrak{q}_{\mathfrak{q}} &=
        \left\{ \frac{a}{b} \in A_{\mathfrak{q}} \colon
        a \in \mathfrak{q} \right\} \\
        \mathfrak{p}_{\mathfrak{p} } &=
        \left\{ \frac{a}{b} \in B_{\mathfrak{p}} \colon
        a \in \mathfrak{q} \right\} .
    \end{align*}
    Let $x \in 
    \mathfrak{q}_{\mathfrak{q}}$, so
    there exist $a \in \mathfrak{q}$ and
    $b \in A - \mathfrak{q}$ such that
    $x = \frac{a}{b}$.
    Then
    $\frac{a}{b} = \pi_A (a) \pi_A(b)^{-1}$, so
    $g' \left( \frac{a}{b} \right) 
    = g' \circ \pi_A (a) 
    \left( g' \circ \pi_A (b) \right)^{-1}
    = \pi_B \circ g(a) \left( \pi_B \circ g(b) \right)^{-1}
    = \frac{g(a)}{g(b)}$ where
    we can invert by $g(b)$ since
    $g(b) \in B - \mathfrak{p}$ as
    $b \in A - \mathfrak{q}$ and
    $\mathfrak{q} = g^{-1}\left( \mathfrak{p} \right) $.
    Since $g(b) \in B - \mathfrak{p}$ and
    $g\left( \mathfrak{q} \right) 
    \subset \mathfrak{p}$, 
    so $g(a) \in \mathfrak{p}$, we have
    $\frac{g(a)}{g(b)} \in 
    \mathfrak{p}_{\mathfrak{p}}$, so
    $g'(x) \in \mathfrak{p}_{\mathfrak{p}}$, hence
    $g\left( \mathfrak{q}_{\mathfrak{q}} \right) 
    \subset \mathfrak{p}_{\mathfrak{p}}$.
    \\
    \linebreak
    (2) 
The ring $\mathbb{Z}_{(p)}$ is local and 
$\mathbb{Q}$ being a field is also local. However,
the inclusion  $\mathbb{Z}_{(p)} \hookrightarrow
\mathbb{Q}$ is not a local homomorphism since
$\frac{p}{1}$ is not mapped to $0$, for example.
\end{proof}



\section{Assignment 3}
    \begin{exercise}[]
        Let $R$ be a Noetherian ring. Show the following
        \begin{enumerate}
            \item For every ideal
                $I \subset R$, there exists $n \in \mathbb{N} $
                such that $\left( \sqrt{I}  \right)^{n}
                \subset I$.
            \item Every radical ideal of $R$ is a finite
                intersection of prime ideals.
            \item If a radical ideal of $R$ is irreducible,
                then it is a prime ideal.
        \end{enumerate}
    \end{exercise}

    \begin{proof}
        (1) Since $I \subset \sqrt{I} $ are
        sub-$R$-modules of $R$ considered as a module
        over itself, we find that
        $\sqrt{I} $ must be finitely generated, so let
        $\sqrt{I}  = \left< x_1, \ldots, x_n \right>$, and
        by assumption,
        there exist $\alpha_1, \ldots, \alpha_n$ such that
        $x_i^{\alpha_i} \in I$.
        Let $\alpha = \alpha_1 + \ldots
        + \alpha_n$.
        Now let
        $x \in \sqrt{I} $ and write
        $x = \sum_{i} c_i x_i$. Then
        any term in
        $x^{\alpha}$ will contain some
        $x_i$ to the power of at
        least $\alpha_i$ by
        the pigeonhole principle. Since
        $I$ is an ideal, the whole term is in $I$, so
        again, ideals are closed under sums, so
         $x^{\alpha} \in I$.
         Since $x$ was arbitrary, we find
         that
         $\left( \sqrt{I}  \right)^{\alpha}
         \subset I$.\\
         \linebreak
         (2) Let
         $I$ be a radical ideal of $R$, so
         $\sqrt{I}  = I$.
         By theorem 7.19 (Primary decomposition),
         $I$ is the finite intersection
         of primary ideals, so
         \[
         I = \mathfrak{p}_1 \cap \ldots
         \cap \mathfrak{p_n}
         \]
         where each $\mathfrak{p}_i$ is primary.

         \begin{lemma}[]
             For an ideal $J =
             J_1 \cap \ldots \cap J_n$, we have
             \[
             \sqrt{J}  = \sqrt{J_1}  \cap
             \ldots \cap \sqrt{J_n}
             \]
         \end{lemma}
         \begin{proof}
             Suppose
             $x \in \sqrt{J} $ so
             $x^{i} \in J =
             J_1 \cap \ldots \cap J_n$, then
             $x^{i} \in J_j$ for all $j$ so
             $x \in \sqrt{J_j} $ for all $J$, so
             $x \in \sqrt{J_1}  \cap \ldots
             \cap \sqrt{J_n} $.
             Conversely, if
             $x \in \sqrt{J_1} \cap \ldots
             \cap \sqrt{J_n} $ then
             there exist  $\alpha_1, \ldots, \alpha_n$
             such that
             $x^{\alpha_i} \in J_i$.
             Let $\alpha = \max_{i} \left\{ \alpha_i \right\} $.
             Then
             $x^{\alpha} \in
             J_1 \cap \ldots \cap J_n = J$, so
             $x \in \sqrt{J} $.
         \end{proof}
         Hence we obtain
         \[
         I = \sqrt{I}  =
         \sqrt{\mathfrak{p}_1}  \cap
         \ldots \cap
         \sqrt{\mathfrak{p}_n} .
         \]
         To finish it off, we note that
         by Lemma 7.11,
         each $\sqrt{\mathfrak{p}_i} $ is prime.\\
         \linebreak
         (3) Suppose
         $I \subset R$ is a radical ideal which is
         irreducible.
         By Lemma 7.16, $I$ is primary, and now by
         Lemma 7.11,
         $I = \sqrt{I} $ is prime.
    \end{proof}


    \begin{exercise}[]
        Let $V \subset K^{n}$ be an affine
        algebraic set. Show the following.
        \begin{enumerate}
            \item $V$ is irreducible if and only if
                $\mathbb{I}(V)$ is a prime ideal.
            \item $V$ can be written as a finite
                union of irreducible affine algebraic
                sets.
            \item There is a minimal decomposition
                $V = V_1 \cup \ldots \cup
                V_m$ of $V$ into irreducible affine
                algebraic sets $V_i$, where
                $m \in \mathbb{N}_0$. This is meant in the
                sense that no $V_i$ is contained in
                $\bigcup_{j \neq i} V_j$.
            \item The minimal decomposition $V =
                V_1 \cup \ldots \cup V_m$ is unique,
                up to reordering of
                $V_1,\ldots, V_m$. We
                call $V_1, \ldots, V_m$ the irreducible
                components of $V$.
        \end{enumerate}
    \end{exercise}


    \begin{proof}
        (1) Since $V \subset K^{n}$ is an affine
        algebraic set, there exists
        some ideal $I \subset k\left[ x_1, \ldots,
        x_n\right] $ such that
        $V = \mathbb{V}(I)$.
        Suppose
        $V = V_1 \cap V_2$ with
        both $V_1$ and $V_2$ being affine
        algebraic sets properly containing
        $V$.
        Then
        $\mathbb{I}(V) \subset
        \mathbb{I}(V_1) \cap \mathbb{I}(V_2)$ since
        any polynomial vanishing on $V$ must vanish on
        both $V_1$ and on $V_2$. But now any
        prime ideal is irreducible, so
        $\mathbb{I}(V_1) = \mathbb{I}(V)$  or
        $\mathbb{I}(V_2) = \mathbb{I}(V)$.
        Suppose without loss
        of generality that
        $\mathbb{I}(V_2) = \mathbb{I}(V)$. Then
        $V_2 = \mathbb{V} (\mathbb{I}(V_2)) =
        \mathbb{V} (\mathbb{I}(V)) = V$.
        For this, we need to show that
        $\mathbb{V} \left( \mathbb{I} (W) \right)
        = W$ when $W$ is an affine algebraic set.
        But $\mathbb{I} \left( \mathbb{V}(U) \right)
        \subset U$ always, so since
        $\mathbb{V}$ is containment-reversing, we get
        $W = \mathbb{V}(U) \subset
        \mathbb{V} \left( \mathbb{I} (W) \right) $.
        For the opposite direction, we simply have
        that if
        $x \in \mathbb{V} \left( \mathbb{I}(W) \right) $, then
        any $f \in \mathbb{I}(W)
        = \mathbb{I}\left( \mathbb{V}(U) \right)
        \subset U$ vanishes on
        $x$. Suppose
        $x \not\in W = \mathbb{V}(U)$.
        Then there exists some $g \in U$ such that
        $g(x) \neq 0$. But
        $g \in \mathbb{I} \left( \mathbb{V}
        (U) \right) = \mathbb{I}(W)$ by definition
        which gives a contradiction.
        Hence
        $\mathbb{V}\left( \mathbb{I}(W) \right)
        \subset W$.
        Having concluded that
        $V = V_1$ or $V = V_2$, this shows that
        $V$ is irreducible.\\
        \linebreak

        \textbf{A faster way to see this, I suppose
        would be the following:}
        If $V = V_1 \cup  V_2$, then
        $\mathbb{I}(V_1) \cap
        \mathbb{I}(V_2) \subset
        \mathbb{I}(V)$, showing that
        $\mathbb{I}(V)$ is not irreducible, contradicting
        lemma 7.3.\\
        \linebreak

        Conversely, if
        $\mathbb{I}(V)$ is not prime, let
        $fg \in \mathbb{I}(V)$ such that
        $f,g \not\in \mathbb{I}(V)$.
        Then
        $V =
        \mathbb{V} \left( \mathbb{I}(V) \right)
        \subset \mathbb{V}
        \left( (f) (g) \right)
        \subset \mathbb{V}(f) \cap
        \mathbb{V}(g)$ using that
        $\mathbb{V}$ is inclusion-reversing.
        Now by assumption,
        if $V = \mathbb{V}(f)$, then
        $f$ would vanish on all of $V$, contradicting
        $f \not\in \mathbb{I}(V)$. Similarly for
        $g$. Hence
        $V$ is shown to not be irreducible.





        (2) Since $V$ is an affine algebraic set,
        there exists an ideal $I$ such that
        $V = \mathbb{V}(I)$. Now,
        $I \subset k\left[ x_1, \ldots, x_n \right] $ which
        is Noetherian by applying
        Hilbert's basis theorem
        iteratively since a field is Noetherian (having
        only $\left( 0 \right) $ and itself as ideals) considered
        as $k$-modules. This in particular
        gives us a decomposition
        \[
        I = \mathfrak{p}_1 \cap \ldots \cap
        \mathfrak{p}_n
        \]
        where each $\mathfrak{p}_i$ is primary. Hence
        \[
        \mathbb{V}(I)
        = \mathbb{V}(\mathfrak{p}_1) \cup
        \ldots \cup
        \mathbb{V}\left( \mathfrak{p}_n \right) .
        \]
        To show that
        $\mathbb{V}\left( \mathfrak{p}_i \right) $
        is irreducible, we can
        show that
        $\mathbb{I} \left( \mathbb{V}
        \left( \mathfrak{p}_i \right) \right) $ is a
        prime ideal.
        This can be easily achieved if we may use
        Nullstellensatz since then
        $\mathbb{I} \left( \mathbb{V}\left(
        \mathfrak{p}_i\right)  \right)
        = \sqrt{\mathfrak{p}_i}
        $ which is prime by Lemma 7.11.\\
        \linebreak
        (3) By part (2),
        $V$ can be decomposed as
        $V = V_1 \cup \ldots \cup  V_n$ where
        each $V_i$ is an irreducible affine
        algebraic set. Suppose now that
        $V_1 \subset \bigcup_{i=2}^{n}V_i $.
        But then
        \[
        V_1 = \bigcup_{i=2}^{n} \left( V_1 \cap
        V_i \right).
        \]
        Now, the intersection of affine algebraic sets
        is still an affine algebraic set since
        $\mathbb{V}(I_1) \cap
        \mathbb{V}(I_2) =
        \mathbb{V}\left( I_1 \cup I_2 \right) $.
       Similarly, a union of finitely many affine
       algebraic sets is also an affine algebraic set since
       $\mathbb{V}(I_1 \cdots I_n)
       = \mathbb{V}(I_1) \cap \ldots \cap
       \mathbb{V}(I_n)$.
       So by irreducibility of $V_1$, either
       $V_1 = V_1 \cap V_2$ or
       $V_1 = \bigcup_{i=3}^{n} V_1 \cap V_i$.
       Inductively, we obtain that for some
       $i\ge 2$,
       $V_1 = V_1 \cap V_2$, i.e.,
       $V_1 \subset  V_2$. Hence we may
       discard $ V_1$ from the collection, so
       $V = V_2 \cup  \ldots \cup  V_n$.
       Thus if we have a collection
       $V = V_1 \cup \ldots \cup  V_n$ such that
       $V_i \subset \bigcup_{j\neq i} V_j$, then we
       can simply discard $V_i$. We can continue to do so and
       after at most $n-1$ steps, we will obtain a minimal
       decomposition.\\
       \linebreak
       (4) Suppose
       \[
       V = V_1 \cup \ldots \cup  V_m =
       W_1 \cup  \ldots \cup W_n
       \]
       are two minimal decompositions.
       Then
       $W_i \subset
       V_1 \cup  \ldots \cup V_m$, so
       \[
       W_i = \bigcup_{j=1}^{m} V_j \cap W_i
       \]
       By part (3), this is a union of affine algebraic sets,
       so we completely equivalently obtain that
       $W_i = V_j \cap W_i$ for some $j$.
       Hence $W_i \subset V_j$.
       For each  $i$, let
       $j_i$ be such that
       $W_i \subset V_{j_i}$.
       Repeating this the other way around, we obtain
       $i_k$ such that
       $V_k \subset W_{i_k}$.
       Now
       $V_k \subset W_{i_k} \subset
       V_{j_{i_k}}$. So since the decomposition is minimal,
       we must have
       $k = j_{i_{k}}$, so
       $V_k = W_{i_k}$ for all $k$.
       This in particular gives an injective map
       $\left\{ 1, \ldots,m \right\} \to
       \left\{ 1, \ldots, n \right\} $, so
       $m\le n$.
       And similarly, $W_i = V_{j_i}$ for all
       $i$, so we similarly get $n\le n$.
       This implies that
       $m = n$ and that indeed the decompositions
       are the same up to reordering, namely by the reordering
       $\sigma \colon k \mapsto i_k$ giving
       $V_{k} =
       W_{\sigma (k)}$.
    \end{proof}


    \section{Assignment 4}
       \begin{exercise}[2]
        Which of the following modules are flat over the
        corresponding rings?
        Justify your answer
        \begin{enumerate}
            \item $R = \mathbb{C}\left[ x,y \right] $ and
                the module is
                $I = \left( x,y \right) \subset R$.
            \item $R = \mathbb{C}\left[ x \right] /
                (x^2)$ and the module is
                $I = (x) \subset R$.
            \item $R = \mathbb{C}\left[ x \right] $ and
                the module is the ring
                $\mathbb{C} \left[ y \right] $ considered
                as an $R$-module by ring
                homomorphism $\mathbb{C} \left[ x \right]
                \to \mathbb{C} \left[ y \right] \colon
                x \mapsto y^2$.
            \item $R = \mathbb{C}\left[ x \right] $ and the
                module is the ring
                $\mathbb{C} \left[ x,y \right]
                /(xy)$ considered as an
                $R$-module by ring
                homomorphism
                $\mathbb{C} \left[ x \right] \to
                \mathbb{C} \left[ x,y \right] / (xy) \colon
                x \mapsto x$.
        \end{enumerate}
    \end{exercise}

    \begin{solution}
        (1) We claim that
        $I = \left( x,y \right) $ is not
        a flat $R = \mathbb{C}\left[ x,y \right] $ module.
        Firstly,
        $\mathbb{C} \left[ x,y \right] $ is Noetherian by
        Hilbert's basis theorem since
        $\mathbb{C}$ is, and
        it is also local: we claim that
        $\left( x,y \right) = I$ is precisely the
        maximal ideal. Firstly,
        it is maximal because
        $\mathbb{C} \left[ x,y \right] / \left( x,y \right)
        \cong \mathbb{C}$ is a field. Now if
        $M \subset \mathbb{C} \left[ x,y \right] $ is a
        maximal ideal, then $1 \not\in M$, so
        for any $f \in M$, we have that
        $f(x,y) =
        \sum_{i+j\ge 1} \alpha_{ij} x^{i} y^{j}
        \in \left( x,y \right) $.
        Thus $M \subset (x,y)$, so
        $M$ is not maximal unless $M = (x,y)$.
        Therefore $(x,y)$ is the only maximal ideal.
        Now, $I$ is finitely generated as
        a $\mathbb{C}\left[ x,y \right] $-module
        with generators
        $x$ and  $y$, hence
        proposition 9.15 applies. Since
        $(x,y)$ is not free since it in particular is a
        proper submodule of $R$, we have that
        it is not flat.\\
        \linebreak
        (2) We claim that
         $I = \left( x \right) \subset
         \mathbb{C} \left[ x \right] / (x^2)$ is indeed
         flat. This can be seen since
         $\mathbb{C} \left[ x \right] / (x^2) =
         \mathbb{C} \oplus (x) \cong
         \mathbb{C} \oplus \mathbb{C}$, so
         since $(x)$ is a direct summand of
         $R = \mathbb{C} \left[ x \right] /(x^2)$,
         it is flat by proposition 9.13 and the fact that
         $R$ is itself flat by example 9.2.\\
         \linebreak
         (3)
         \textbf{I will give two solutions since I'm
         not sure whether I may use that over a PID,
     a module is flat iff it is torsion-free}
     Suppose there
     is a relation
     $\sum a_i y^{i} = 0$ in $\mathbb{C}[y]$ where
     $a_i \in \mathbb{C}[x]$. However,
     then taking the maximal degree of $x^{j}$ in
     $a_j$ for $y^{j}$ the maximal degree
     of $y $ in the relation, we find that
     $a_j = 0$. But this contradicts $y^{j}$ being the
     maximal degree. Hence
     $a_i = 0$ for all $i$. But this
     shows that the relation is trivial. Now
     remark 9.21 tells us that
     $\mathbb{C}[y]$ is flat cosidered as a
     $\mathbb{C}[x]$ module by the homomorphism
     $\mathbb{C}[x] \to \mathbb{C}[y]$ by
     $x \mapsto y^2$.\\
     \linebreak
     The other solution is the following:
         Since $R = \mathbb{C} \left[ x \right] $ is a
         PID, we immediately find that
         $\mathbb{C} \left[ y \right] $ is
         flat if and only if it is torsion-free considered
         as a  $\mathbb{C}\left[ x \right] $-module by
         restriction of scalars along
         $x \mapsto y^2$. Suppose
         $f(y) \in \mathbb{C}[y]$ is such that
         for  $g(x) \in R$,
         $g(x) f(y) = 0$, i.e.,
         $g(y^2) f(y) = 0$ in $\mathbb{C}[y]$. However, this
         forces $f$ or $g$ to be $0$, so
         we find that
         $\mathbb{C} [y]$ is torsion-free as a
         $\mathbb{C} [x]$ module under the ring-homomorphism
         $x \mapsto y^2$. Thus
         $\mathbb{C}[y]$ is a flat
         $\mathbb{C}[x]$-module by the ring homomorphism
         $\mathbb{C}[x] \to \mathbb{C}[y]$ sending
         $x \mapsto y^2$.\\
         \linebreak
         (4)
         We note that if a module has torsion, this
         gives an non-trivial relation since
         $am = 0$ with $a \neq 0$  and
         $m \neq 0$
         admitting a genuinely trivial reparametrization
         implies $a = 0$, contradiction. Hence
         proposition 9.20 gives that if a module
         has torsion, then it cannot be flat.
         $\mathbb{C} \left[ x,y \right] /(xy)$ is clearly
         not a flat
         $\mathbb{C} \left[ x \right] $-module under the
         homomorphism
         $\mathbb{C} \left[ x \right] \to
         \mathbb{C} \left[ x,y \right] /(xy)$ sending
         $x\mapsto x$ since
         $y $ is nonzero in
         $\mathbb{C} \left[ x,y \right] /(xy)$ however,
         $x \cdot y := xy = 0$, hence
         $\mathbb{C} \left[ x,y \right] /(xy)$ is not
         torsion-free over $\mathbb{C}\left[ x \right] $.
    \end{solution}




    \section{Assignment 5}
        \begin{exercise}[1]
        (2)

    We claim that
    $\mathbb{Z} \left[ x_1, x_2, x_3 \right] $ is a
    finite extension.
    Now, clearly, we can express
    $1, x_1, x_1^{2}, x_1^3$ as linear combinations over
    $1, x_1, x_1^{2}, x_1^3$.
    Suppose we can express $x_1^{n}$ as
    a linear combination of
    $1, x_1, x_1^2, x_1^{3}$ for
    $n = 1, \ldots, N-1$ for
    some $N \ge 4$. Since
    \begin{align*}
        x^{n}
        &= x_1^{n-1} \left( x_1+ x_2 + x_3  \right)
        - x_1^{n-1} x_2 - x_1^{n-1} x_3\\
        &= x_1^{n-1} \sigma_1 -
        x_1^{n-2} \sigma_2 + x_1^{n-2} x_2 x_3\\
        &= x_1^{n-1} \sigma_1 - x_1^{n-2} \sigma_2
        + x_1^{n-3} \sigma_3
    \end{align*}
    we find that for
    $N \ge 4$,
    $x_1^{N}$ can be written as a linear combination
    over
    $x_1^{N-1}, x_1^{N-2}$ and
    $x_1^{N-3}$ which by the inductive assumption
    can be written as linear combinations
    of $1, x_1, x_1^2, x_1^3$.
    Hence
    $\mathbb{Z}\left[ x_1, x_2, x_3 \right] $ is
    a finite extension over
    $\mathbb{Z}\left[ \sigma_1, \sigma_2, \sigma_3 \right] $ with
    \[
    \left\{ 1, x_1, x_1^2, x_1^3, x_2,x_2^2,
    x_2^3, x_3, x_3^2, x_3^3 \right\}
    \]
    as a finite generating set.
    By proposition 10.11, this also implies that
    the extension is integral.\\
    \linebreak
    (3) We claim that
    $\mathbb{Z}\left[ x,y \right] $ is
    not a finite extension of
    $\mathbb{Z}\left[ x,xy \right] $.
     Suppose
     $\left\{ g_1, \ldots, g_n \right\}
     \in \mathbb{Z}\left[ x,y \right] $ is a
     generating set as a module.
     Let $N$ be the maximal degree of $y$ over
     $g_1, \ldots, g_n$.
     Then
     $y^{N+1} \in
     \Span \left( g_1, \ldots, g_n \right) $, so
     let
     $y^{N+1} = f_1(x,xy) g_1(x,y) + \ldots +
     f_n (x,xy) g_n(x,y)$.

     Writing each
     $f_i (x,xy) =
     \sum \alpha_{i, k,l}
     x^{k} \left( xy \right)^{l}$, we see that
     \begin{align*}
         y^{N+1} &
         =
     \sum \alpha_{1,k,l} x^{k} (xy)^{l} g_1(x,y)
     + \ldots +
     \sum \alpha_{n,k,l} x^{k}(xy)^{l} g_n(x,y)\\
     &= \sum \left( \alpha_{1,k,l}g_1(x,y) +
     \ldots + \alpha_{n,k,l}g_n(x,y) \right)
     x^{k} (xy)^{l}.
     \end{align*}
     So in particular, we must have
     that
     for $\left( k,l \right) \neq (0,0)$,
     \[
     \alpha_{1,k,l}g_1(x,y) + \ldots+
     \alpha_{n,k,l}g_n(x,y) = 0.
     \]
     But then we get
     \[
     y^{N+1} =
     \alpha_{1,0,0} g_1(x,y) +\ldots
     + \alpha_{n,0,0}g_n(x,y)
     \]
     which has  maximal $y$ degree $N$, giving a contradiction.
     Thus the extension is not finite.
     However,
     clearly, it is a finite-type extension, since
     $y$ together with
     $\mathbb{Z}\left[ x,xy \right] $ precisely generate all
     of $\mathbb{Z}\left[ x,y \right] $.
     By proposition 10.11, we then
     find that $\mathbb{Z}\left[ x,y \right] $ is not an
     integral extension.\\
     \linebreak
     (7) If the map
     $\mathbb{C}\left[ x \right]
     \hookrightarrow
     \mathbb{C}\left[ x,y,z \right] / \left( z^2
     -xy\right) $ were integral, proposition 10.6 gives that
     $\mathbb{C}\left[ x,y \right]
     \subset \mathbb{C}\left[ x,y,z \right]
     / \left( z^2 - xy \right)$ would be
     finitely generated as a
      $\mathbb{C}\left[ x \right] $-module.
      Suppose
      $f_1, \ldots, f_n$ generated
      $\mathbb{C}\left[ x,y \right] $ as
      a $\mathbb{C}\left[ x \right] $ module in
      $\mathbb{C}\left[ x,y,z \right] /
      \left( z^2 - xy \right) $.
      If $y^{N}$ is the maximal degree of $y$ among
      $f_1, \ldots, f_n$, then
      $y^{N+1} = \sum g_i f_i$ for
      $g_i \in \mathbb{C}\left[ x \right] $.
      However, there is clearly no way to obtain
      $y^{N+1}$ in such a way. So
      the extension is not integral. Since it is
      clearly finite type, it is also not finite by proposition
      10.11.\\
      \linebreak
      (9) If
      $\mathbb{C}[x]
      \hookrightarrow
      \mathbb{C}\left[ x,y,z \right] /
      \left( z^2 - xy, x^3 - yz \right) $ were
      integral, $z$ would be integral over
      $\mathbb{C}\left[ x \right] $, so there
      would be some
      linear combination
      \[
      f_n(x) z^{n} + \ldots + f_0 (x) = 0
      \]
      However, there is not relation
      converting $xz$ to something different, so if
      $x^{k_n}$ is the highest term of $x$ in
      $f_n(x)$, then
      $x^{k_n} z^{n}$ is a term that cannot
      cancel in the above linear combination.
      So the extension cannot be integral. Hence it
       can also not be finite.\\
       \linebreak
       (10) The extension
       $\mathbb{C} \hookrightarrow
       \mathbb{C}\left[ x_1, x_2, x_3, \ldots \right]
       / \left( x_1^2, x_2^2, x_3^2, \ldots \right) $
       is not finite: suppose
       it were generated by
       $f_1, \ldots, f_n \in
       \mathbb{C}\left[ x_1, x_2, \ldots \right]
       / \left( x_1^2, x_2^2, \ldots \right) $, and let
        $m$ be maximal such that
        one of the $f_i$ has a term with
        $x_m$. Then
        $x_{m+1} = c_1 f_1 + \ldots+ c_n f_n$ with
        $c_i \in \mathbb{C}$. However, then multiplying
        both sides by $x_{m+1}$, we see that
        each non-zero term in
        $c_1 f_1+ \ldots + c_n f_n$ must have a $x_{m+1}$,
        contradicting maximality of $m$.\\
        \linebreak
        The extension is integral, however, since
        for any $b \in
        \mathbb{C}\left[ x_1, x_2, \ldots \right]
        / \left( x_1^2, x_2^2, \ldots \right) $,
        let $x_{i_1}, \ldots, x_{i_k}$ be the
        $x_i$ which appear in $b$. Then
        $b \left( x_{i_1} \cdots x_{i_k} \right)
        = 0 \subset
        \left( x_{i_1} \cdots x_{i_k} \right) $, and
        $\left( x_{i_1} \cdots
        x_{i_k} \right) $ is clearly finitely generated,
        so by proposition 10.6,
        $b$ is integral over $\mathbb{C}$.




    \end{exercise}


    \section{Assignment 6}
        \begin{exercise}[1]
        Let $R$ be a Noetherian ring and $A$ be a finitely

        generated $R$-algebra. Show that if
        $B \subset A$ is a subalgebra such that $A$ is
        a finitely generated $B$-module, then
        $B$ is also a finitely generated
        $R$-algebra.
    \end{exercise}


    \begin{proof}
        We want to show that
        $B$ is finitely generated as an $R$-algebra.\\

        Suppose $\left\{ y_1, \ldots, y_n \right\} \subset A$
        generate $A$ as an $R$-algebra, so
        $A = R \left[ y_1, \ldots y_n \right] $.
        Since $A$ is also finitely generated
        as a $R$-module, there
        exist $a_1, \ldots, a_m \in A$ such that
        $A = Ba_1 + \ldots + Ba_m$.
        Now using the module expression for $A$, write
        \[
        y_i = \sum_j b_{ij} a_j
        \]
        and similarly, since
        $a_i a_j \in A$,
        \[
        a_i a_j = \sum_k b_{ijk} a_k.
        \]
        Then given arbitrary
        $u,v \in A$,
        we can write
        \[
        u = \sum_{i,j} \alpha_i b_{ij} a_{j}
        \]
        and
         \[
        v = \sum_{i,j} \beta_i b_{ij} a_{j}
        \]
        We have then seen that
        \[
        uv = \sum_{i,j,k,l} \alpha_i b_{ij} a_j \beta_k b_{kl}a_l
        = \sum_{i,j,k,l} \left( \alpha_i \beta_k \right)
        \left( b_{ij} b_{kl} \right) \sum_{r} b_{jlr} a_r
        = \sum_{i,,k,l,r} \left( \alpha_i b_{k} \right)
        \left( b_{ij} b_{kl} b_{jlr} \right) a_r.
        \]
        This shows that
        $A$ is generated by
        $a_1, \ldots, a_n$  as an
        $D = R \left[ b_{ij}, b_{jlr} \mid
        j,l = 1,\ldots, n \quad
    i,r = 1,\ldots,m\right] $ algebra.
        In particular, $D$ is Noetherian by corollary
        6.15, so $A$ is a Noetherian
        $D$-module by applying by
        theorem 6.11. Hence
        since
        $B$ is a natural $D$-submodule of $A$ it is
        finitely generated as a $D$-module, so
        $B = D b_1 + \ldots + D b_n$.
        However, this in particular expresses
        $B$ as the $R$-algebra
        $R \left[ b_{ij},b_{ijk}, b_1,\ldots, b_n
         \mid  i,j,k = 1, \ldots, n\right] $.

    \end{proof}

    \begin{exercise}[2]
        Let $K$ be a field and let $A$ be a finitely
        generated $K$-algebra. Show that if
        $A$ is a field, then $A$ is finite-dimensional
        as a $K$-vector space. In particular, note that
        for every maximal ideal
        $\mathfrak{R} \subset A$,
        $A / \mathfrak{R}$ is a finite dimensional
        $K$-vector space.
    \end{exercise}


    \begin{proof}
        If $A$ is a field extension of $K$ such that
        $A$ is finite type over $K$, then
        by Zariski's lemma, we directly find that
        $A$ is finite over $K$ - i.e. that it
        is finitely generated as a $K$-module, and since
        $K$ is a field, this is saying that
        $A$ is finitely generated as a $K$-vector space.
        The latter part is corollary 11.7.\\
        \linebreak
    \end{proof}





    %\printbibliography
\end{document}
