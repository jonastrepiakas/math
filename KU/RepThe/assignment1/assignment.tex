\documentclass[reqno]{amsart}
\usepackage{amscd, amssymb, amsmath, amsthm}
\usepackage{graphicx}
\usepackage[colorlinks=true,linkcolor=blue]{hyperref}
\usepackage[utf8]{inputenc}
\usepackage[T1]{fontenc}
\usepackage{textcomp}
\usepackage{babel}
%% for identity function 1:
\usepackage{bbm}
%%For category theory diagrams:
\usepackage{tikz-cd}


\setlength\parindent{0pt}

\pdfsuppresswarningpagegroup=1

\newtheorem{theorem}{Theorem}[section]
\newtheorem{lemma}[theorem]{Lemma}
\newtheorem{proposition}[theorem]{Proposition}
\newtheorem{corollary}[theorem]{Corollary}
\newtheorem{conjecture}[theorem]{Conjecture}

\theoremstyle{definition}
\newtheorem{definition}[theorem]{Definition}
\newtheorem{example}[theorem]{Example}
\newtheorem{exercise}[theorem]{Exercise}
\newtheorem{problem}[theorem]{Problem}
\newtheorem{question}[theorem]{Question}

\theoremstyle{remark}
\newtheorem*{remark}{Remark}
\newtheorem*{note}{Note}
\newtheorem*{solution}{Solution}



%Inequalities
\newcommand{\cycsum}{\sum_{\mathrm{cyc}}}
\newcommand{\symsum}{\sum_{\mathrm{sym}}}
\newcommand{\cycprod}{\prod_{\mathrm{cyc}}}
\newcommand{\symprod}{\prod_{\mathrm{sym}}}

%Linear Algebra

\DeclareMathOperator{\Span}{span}
\DeclareMathOperator{\im}{im}
\DeclareMathOperator{\diag}{diag}
\DeclareMathOperator{\Ker}{Ker}
\DeclareMathOperator{\ob}{ob}
\DeclareMathOperator{\Hom}{Hom}
\DeclareMathOperator{\Mor}{Mor}
\DeclareMathOperator{\sk}{sk}
\DeclareMathOperator{\Vect}{Vect}
\DeclareMathOperator{\Set}{Set}
\DeclareMathOperator{\Group}{Group}
\DeclareMathOperator{\Ring}{Ring}
\DeclareMathOperator{\Ab}{Ab}
\DeclareMathOperator{\Top}{Top}
\DeclareMathOperator{\hTop}{hTop}
\DeclareMathOperator{\Htpy}{Htpy}
\DeclareMathOperator{\Cat}{Cat}
\DeclareMathOperator{\CAT}{CAT}
\DeclareMathOperator{\Cone}{Cone}
\DeclareMathOperator{\dom}{dom}
\DeclareMathOperator{\cod}{cod}
\DeclareMathOperator{\Aut}{Aut}
\DeclareMathOperator{\Mat}{Mat}
\DeclareMathOperator{\Fin}{Fin}
\DeclareMathOperator{\rel}{rel}
\DeclareMathOperator{\Int}{Int}
\DeclareMathOperator{\sgn}{sgn}
\DeclareMathOperator{\Homeo}{Homeo}
\DeclareMathOperator{\SHomeo}{SHomeo}
\DeclareMathOperator{\PSL}{PSL}
\DeclareMathOperator{\Bil}{Bil}
\DeclareMathOperator{\Sym}{Sym}
\DeclareMathOperator{\Skew}{Skew}
\DeclareMathOperator{\Alt}{Alt}
\DeclareMathOperator{\Quad}{Quad}
\DeclareMathOperator{\Sin}{Sin}
\DeclareMathOperator{\Supp}{Supp}
\DeclareMathOperator{\Char}{char}
\DeclareMathOperator{\Teich}{Teich}
\DeclareMathOperator{\GL}{GL}
\DeclareMathOperator{\tr}{tr}
\DeclareMathOperator{\codim}{codim}
\DeclareMathOperator{\Diff}{Diff}
\DeclareMathOperator{\Bun}{Bun}


%Row operations
\newcommand{\elem}[1]{% elementary operations
\xrightarrow{\substack{#1}}%
}

\newcommand{\lelem}[1]{% elementary operations (left alignment)
\xrightarrow{\begin{subarray}{l}#1\end{subarray}}%
}

%SS
\DeclareMathOperator{\supp}{supp}
\DeclareMathOperator{\Var}{Var}

%NT
\DeclareMathOperator{\ord}{ord}

%Alg
\DeclareMathOperator{\Rad}{Rad}
\DeclareMathOperator{\Jac}{Jac}

%Misc
\newcommand{\SL}{{\mathrm{SL}}}
\newcommand{\mobgp}{{\mathrm{PSL}_2(\mathbb{C})}}
\newcommand{\id}{{\mathrm{id}}}
\newcommand{\MCG}{{\mathrm{MCG}}}
\newcommand{\PMCG}{{\mathrm{PMCG}}}
\newcommand{\SMCG}{{\mathrm{SMCG}}}
\newcommand{\ud}{{\mathrm{d}}}
\newcommand{\Vol}{{\mathrm{Vol}}}
\newcommand{\Area}{{\mathrm{Area}}}
\newcommand{\diam}{{\mathrm{diam}}}
\newcommand{\End}{{\mathrm{End}}}


\newcommand{\reg}{{\mathtt{reg}}}
\newcommand{\geo}{{\mathtt{geo}}}

\newcommand{\tori}{{\mathcal{T}}}
\newcommand{\cpn}{{\mathtt{c}}}
\newcommand{\pat}{{\mathtt{p}}}

\let\Cap\undefined
\newcommand{\Cap}{{\mathcal{C}}ap}
\newcommand{\Push}{{\mathcal{P}}ush}
\newcommand{\Forget}{{\mathcal{F}}orget}



\title{Assignment 1}
\author{Jonas Trepiakas - hvn548}
\date{}

\begin{document}
\maketitle
    \begin{problem}[1]
        Let $\pi \colon \mathbb{R} \to 
        \GL \left( \mathbb{R}^2 \right) $ by
        $\pi(t) = 
        \begin{pmatrix} \cos t & \sin t\\
        - \sin t & \cos t \end{pmatrix} $ with respect to the
        standard basis $\left\{ e_1,e_2 \right\} $ of
        $\mathbb{R}^2$. Consider the tensor product
        \[
        \pi \otimes \pi \colon
        \mathbb{R} \to \GL \left( \mathbb{R}^2 
        \otimes \mathbb{R}^2\right) .
        \] 
        Identify $\mathbb{R}^2 \otimes \mathbb{R}^2\to 
        M_{2\times 2}(\mathbb{R})$ via
        $e_i \otimes e_j \mapsto E_{ij}$.
        \begin{enumerate}
            \item Fix the following basis
                $\mathcal{B} = \left\{ A_1,A_2,A_3,A_4 \right\} $ 
                of $M_{2\times 2}\left( \mathbb{R} \right) 
                \cong \mathbb{R}^2 \otimes \mathbb{R}^2$ :
                \[
                A_1 =I_2, \quad
                A_2 =
                \begin{pmatrix} 1 & 0 \\ 0
                & -1 \end{pmatrix} , \quad
                    A_3 = \begin{pmatrix} 0 & 1\\
                    1 & 0\end{pmatrix} , \quad
                        A_4 = \begin{pmatrix} 0 & 1\\
                        -1 & 0 \end{pmatrix} .
                \] 
                Compute 
                $\left( \pi \otimes \pi \right)_{\mathcal{B}}
                \colon \mathbb{R} \to \GL_{4}\left( \mathbb{R}
                \right) $.
            \item Compute the decomposition of $M_{2\times 2}
                (\mathbb{R})$ into minimal invariant subspaces
                (with respect to the representation
                $\pi \otimes \pi$ ) and prove that
                the stated subspaces are indeed minimal.
        \end{enumerate}
    \end{problem}

    \begin{proof}
        (1) 
        Let $\varphi_{\mathcal{B}}
        \colon \mathbb{R}^4 \to \mathbb{R}^{4}$ be the
        change of basis given by
        $\varphi_{\mathcal{B}}
        (A_i) = e_i$.\\
        By definition,
        \[
            \left( \pi \otimes \pi \right)_{\mathcal{B}}
            (t) e_j =
            \varphi_{\mathcal{B}}
            \left( \pi \otimes \pi(t)  A_j \right) 
        \] 
        Now,
        \begin{align*}
            A_1 &= e_1 \otimes e_1 + e_2 \otimes e_2\\
            A_2 &= e_1 \otimes e_1 - e_2 \otimes e_2\\
            A_3 &= e_2 \otimes e_1 + e_1 \otimes e_2\\
            A_4 &= e_1 \otimes e_2 - e_2 \otimes e_1
        \end{align*}
        so
        \begin{align*}
            \pi \otimes \pi(t) A_1
            &= \pi(t) e_1 \otimes \pi(t) e_1 +
            \pi(t) e_2 \otimes \pi(t) e_2\\
            &= 
            \left( \cos t e_1 - \sin
            t e_2 \right) \otimes
            \left( \cos t e_1- \sin t e_2 \right)
            + \left( \sin t e_1 + \cos t e_2 \right) 
            \otimes \left( \sin t e_1 + \cos t e_2 \right) \\
            &= \cos^2 t e_1 \otimes e_1 + \sin^2 t e_2 \otimes
            e_2 - \sin t \cos t
            \left( e_1 \otimes e_2 + e_2 \otimes e_1 \right) \\
            &+ \sin^2 t e_1 \otimes e_1
            + \cos^2 t e_2 \otimes e_2
            + \cos t \sin t 
            \left( e_1 \otimes e_2 + e_2 \otimes e_1 \right) \\
            &= e_1 \otimes e_1
            + e_2 \otimes e_2\\
            &= A_1\\
            \pi \otimes \pi(t) A_2
            &= \pi(t) e_1 \otimes \pi(t) e_1
            - \pi(t) e_2 \otimes \pi(t) e_2\\
            &= \left( 
            \cos t e_1 - \sin t e_2\right) 
            \otimes \left( \cos t e_1 - \sin t e_2 \right) 
            - \left( \sin t e_1 + \cos t e_2 \right) 
            \otimes \left( \sin t e_1 + \cos t e_2 \right) \\
            &= \cos^2 t e_1 \otimes e_1
            + \sin^2 t e_2 \otimes e_2
            - \cos t \sin t \left( e_1 \otimes e_2 +
            e_2 \otimes e_1 \right) \\
            &- \sin^2 t e_1 \otimes e_1
            - \cos^2 t e_2 \otimes e_2
            - \sin t \cos t \left( e_1 \otimes e_2
            + e_2 \otimes e_1 \right) \\
            &=\cos(2t) e_1 \otimes e_1
            - \cos(2t) e_2 \otimes e_2
            - \sin (2t) \left( e_1 \otimes e_2
            + e_2 \otimes e_1 \right) \\
            &= \cos(2t) A_2 - \sin (2t) A_3 \\
            \pi \otimes \pi (t) A_3
            &= \left( \sin t e_1 + \cos t e_2 \right) 
            \otimes \left( \cos t e_1 - \sin t e_2 \right) 
            + \left( \cos t e_1 - \sin t e_2 \right) 
            \otimes \left( \sin t e_1 + \cos t e_2 \right) \\
            &= \cos t \sin t e_1 \otimes e_1 -
            \cos t \sin t e_2 \otimes e_2
            + \cos^2 t e_2 \otimes e_1
            - \sin^2 t e_1 \otimes e_2\\
            &+ \cos t \sin t e_1 \otimes e_1
            - \sin t \cos t e_2 \otimes e_2
            + \cos^2 t e_1 \otimes e_2
            - \sin^2 t e_2 \otimes e_1\\
            &=  \sin(2t) e_1 \otimes e_1
            -  \sin(2t) e_2 \otimes e_2
            + \cos(2t) (e_2 \otimes e_1 + e_1 \otimes e_2)\\
            &=  \sin(2t) A_2
            + \cos(2t) A_3\\
            \pi \otimes \pi(t) A_4
            &= 
            \left( \cos t e_1 - \sin t e_2 \right) 
            \otimes \left( \sin t e_1 + \cos t e_2 \right) 
            - \left( \sin t e_1 + \cos t e_2 \right) 
            \otimes \left( \cos t e_1 - \sin t e_2 \right) \\
            &= \cos t \sin t e_1 \otimes e_1
            - \cos t \sin t e_2 \otimes e_2
            + \cos^2 t e_1 \otimes e_2
            - \sin^2 t e_2 \otimes e_1\\
            &- \sin t \cos t e_1 \otimes e_1
            + \cos t \sin t e_2 \otimes e_2
            + \sin^2 t e_1 \otimes e_2 
            - \cos^2 t e_2 \otimes e_1\\
            &= e_1 \otimes e_2 - e_2 \otimes e_1\\
            &= A_4
        \end{align*}

        Thus the matrix for $\left( \pi \otimes \pi \right)_{\mathcal{B}}$
        with respect to the standard basis
        becomes
        \[
        \begin{pmatrix} 
            1 & 0 & 0 & 0\\
            0 & \cos (2t) & \sin(2t) & 0\\
            0 & - \sin(2t) & \cos(2t) & 0\\
            0 & 0 & 0 & 1
        \end{pmatrix} 
        \] 


        (2) 
        It suffices to decompose
        $\left( \pi \otimes \pi \right)_{\mathcal{B}}$ into
        irreducible subspaces. Now
        since  $\left( \pi \otimes \pi \right)_{\mathcal{B}}
        (e_i) = e_i$ for $i = 1,4$, we have that
        $\Span (e_1)$ and $\Span (e_4)$ are
        $\left( \pi \otimes \pi \right)_{\mathcal{B}}$ 
        invariant subspaces. Furthermore, since
        they are $1$-dimensional, they are irreducible.
        Let $W =
        \Span (e_2, e_3)$. On
        $W$, $\left( \pi \otimes \pi \right)_{\mathcal{B}}$ 
        restricts to the representation
        \[
            \begin{pmatrix} \cos (2t) & \sin (2t)\\
            - \sin (2t) & \cos (2t) \end{pmatrix} 
        \] 
        which is a rotation matrix. By example 1.20,
        this representation is irreducible since it is
        irreducible for all $s$ when
        we insert $s = \frac{1}{2}t$, hence also for all $t$.
        Hence
        $\Span (e_1), 
        \Span (e_2, e_3)$ and
        $\Span(e_4)$ are the minimal invariant
        subspaces of 
        $\mathbb{R}^{4}$ with respect to
        $\left( \pi \otimes \pi \right)_{\mathcal{B}}$.
        By changing back in bases,
        we find that
        $\Span(A_1), \Span(A_2,A_3)$ and
        $\Span(A_4)$ are the minimal invariant
        subspaces of $\pi \otimes \pi$.
    \end{proof}


    \begin{problem}[2]
        Let $\pi \colon \mathbb{R} \to \GL_2 (\mathbb{R})$ be
        given by
        $\pi(t) = \begin{pmatrix} 1 & t \\ 0 & 1 \end{pmatrix} $ and
        let $\pi_0 \colon \mathbb{R} \to 
        \GL_1 (\mathbb{R}) = \mathbb{R}^{\times }$ be
        given by $\pi_0(t) = 1$ be the one-dimensional
        trivial representation of $\mathbb{R}$. Find non-zero
        intertwining operators from
        $\pi_0$ to $\pi$ and from $\pi$ to $\pi_0$.
    \end{problem}

    \begin{solution}
        An intertwining operator from
        $\pi_0$ to $\pi$ is a linear map
        $\varphi \colon
        \mathbb{R} \to \mathbb{R}^2$ such that
        $\pi (g) \circ \varphi = \varphi \circ \pi_0 (g)$ for all
        $g$.
        Thus we want to find $\varphi $ such that for
        all $t$
        \[
            \begin{pmatrix} 1 & t \\
            0 & 1\end{pmatrix}  \circ \varphi 
            = \varphi \circ I
        \] 
        For this, we can
        simply choose
        $\varphi (e_1) = e_1$, giving
        $\left[ \varphi  \right] 
        = \begin{pmatrix} 1 \\ 0 \end{pmatrix} $.\\
        Now, an intertwining operator
        from $\pi$ to $\pi_0$ is a linear map
        $\psi \colon \mathbb{R}^2 \to \mathbb{R}$ such that
        \[
        \pi_0(t) \circ \psi = \psi \circ \pi(t)
        = \psi \circ \begin{pmatrix} 1 & t\\
        0 & 1\end{pmatrix} 
        \] 
        If we define $\psi (e_1) = 0$ and $\psi (e_2) = e_1$ so
        $\left[ \psi  \right] =
        \begin{pmatrix} 0 & 1 \end{pmatrix} $, we get
        $\pi_0(t) \circ \psi (e_1) = 0 = 
        \psi \circ \pi(t) (e_1)$, while
        $\pi_0(t) \circ \psi (e_2) = e_1 = 
        \psi \circ \pi(t) e_2$.
        So $\psi $ is an intertwining
        operator from
        $\pi$ to $\pi_0$.\qed
    \end{solution}

    \begin{problem}[3]
        Suppose that $\pi, \sigma_1, \sigma_2$ are finite
        dimensional completely reducible representations
        of a group $G$. Show that if
        $\pi \oplus \sigma_1$ is equivalent to
        $\pi \oplus \sigma_2$, then
        $\sigma_1$ is equivalent to $\sigma_2$.
    \end{problem}

    \begin{solution}

        Suppose
        $\pi \colon G \to 
        \GL(V)$,
        $\sigma_1 \colon G \to \GL(U_1)$ and
        $\sigma_2 \colon G \to \GL(U_2)$ are
        the representations.
    
        Now, by assumption, there exists
        an isomorphism
        $\varphi \colon
        V \oplus U_1 \to 
        V \oplus U_2$ such that
        $\pi \oplus \sigma_2 =
        \varphi \circ (\pi \oplus \sigma_1) \circ
        \varphi^{-1}$.
        Note that
        since $\pi, \sigma_1, \sigma_2$ are completely
        reducible, we can write
        $\pi \simeq \pi_1 \oplus \ldots \oplus \pi_n,
        \sigma_1 \simeq \sigma_{1,1} \oplus \ldots
        \oplus \sigma_{1,m}$ and
        $\sigma_2 \simeq \sigma_{2,1} \oplus \ldots
        \oplus \sigma_{2,k}$. Then
         \[
        \pi \oplus \sigma_1
        \simeq \bigoplus_{i=1}^{n} \pi_i
        \oplus \bigoplus_{j=1}^{m} \sigma_{1,j}
        \] 
        and
         \[
        \pi \oplus \sigma_2
        \simeq \bigoplus_{i=1}^{n} \pi_i
        \oplus \bigoplus_{j=1}^{k} \sigma_{2,j}
        \] 
        are also completely reducible by theorem 3.7 (as
        well as the direct sum of two irreducible representations
        again being irreducible). Furthermore, by
        corollary 3.10, $m = k$.\\

        Now, $\varphi $ must take an irreducible
        subspace of
        $V \oplus U_1$ to an irreducible subspace of
        $V \oplus U_2$.
        So $\varphi $ induces a bijection 
        using Corollary 3.10:

        $\tilde{\varphi } \colon 
        \left\{ V_1, \ldots, V_n, U_{1,1}, \ldots,
        U_{1,m} \right\} \to 
        \left\{ V_1, \ldots, V_n, 
        U_{2,1}, \ldots, U_{2,m}\right\} $ where
        $V_i$ is the irreducible subspace of
        $\pi_i$, $U_{1,i}$ of $\sigma_{1,i}$ and
        $U_{2,i}$ of $\sigma_{2,i}$.
        Thus, for each $i$,
        $\varphi \circ \sigma_{1,i} \circ \varphi^{-1}$ must
        be one of the 
        $\pi_j$ or $\sigma_{2,j}$. Now, note that
        if it is  $\pi_j$, then in the decomposition of
        $\pi \oplus \sigma_1$, we have
        $2$ irreducible representations equivalent to
        $\sigma_{1,i}$. Hence in the decomposition of
        $\pi \oplus \sigma_2$, we must also have two, so
        there must be another irreducible factor also
        equivalent to $\sigma_{1,i}$. The same argument applies as
        above, and after a finite number of steps, we find
        that there must be some
        $\sigma_{2,j}$ equivalent to
        $\sigma_{1,i}$, since otherwise we would obtain
        at least $n+1$ irreducible factors equivalent to
        $\sigma_{1,i}$ in $\pi \oplus \sigma_{1}$, but only
        $n$ in $\pi \oplus \sigma_2$, giving a contradiction.
        Let $\varphi_{i,j_{i}}$ be the isomorphism
        $U_{1,i} \to U_{2,j_{i}}$ such that
        $\varphi_{i,j_{i}} \circ \sigma_{1,i} \circ
        \varphi_{i,j_{i}}^{-1} = \sigma_{2,j_{i}}$.
        Applying this argument to each
        $\sigma_{1,i}$, we obtain isomorphisms
        $ \varphi_{1,j_{1}},
        \ldots, \varphi_{m,j_{m}}$. But then
        pasting these together
        by letting 
        $ \varphi (x) = \varphi_{i,j_{i}}(x)$ when
        $x \in U_{1,i}$, we get an isomorphism
        $\varphi \colon U_1 \to U_2$ such that
        $\varphi \circ \sigma_{1,i} \circ \varphi^{-1}
        = \sigma_{2, j_{i}}$ for all $i$, so
        $\varphi \circ \sigma_1 \circ \varphi^{-1}
        = \sigma_2$. Thus
        $\sigma_1 \simeq \sigma_2$.\qed



        





    \end{solution}


    \begin{problem}[4]
        Let $G$ be a finite group and $\pi_1, \pi_2$ two
        one-dimensional representations of $G$. Let
        $\pi = \pi_1 \oplus \pi_2$. Compute
        $\pi \otimes \pi$ and $\Sym^2 \pi$, that is,
        find a decomposition of those representations
        into irreducible ones.
    \end{problem}

    \begin{solution}
        Suppose
        $\pi_1,\pi_2 \colon G
        \to \GL (V)$ where
        $V$ is a one-dimensional vector space
        over a field $k$.
        Fix an isomorphism $V \cong k$, so that we
        can from now assume $V = k$.
        Then since
        $\GL (k) \cong k^{\times }
        = k - \left\{ 0 \right\} $,
        we have that
        $\pi_1(g) = c_g$ and
        $\pi_2(g) = d_g$ where
        $c_g$ and $d_g$ are non-zero constants.
        Since $\pi_1,\pi_2$ are one-dimensional,
        $\pi$ is two-dimensional, hence
        $\pi \otimes \pi$ is four-dimensional.

        \[
            \pi \otimes \pi(g) 
            = \pi(g) \otimes \pi(g)
            = \left( \pi_1(g), \pi_2(g) \right) 
            \otimes \left( \pi_1 (g) , \pi_2(g) \right) 
            = \left( c_g, d_g \right) 
            \otimes \left( c_g, d_g \right) 
        \] 
        Thus
        \begin{align*}
            \pi \otimes \pi(g) \left( (x,y) \otimes
            (v,w) \right) 
            = \pi(g) (x,y) \otimes
            \pi(g) (v,w)
            = \left( c_g x, d_g y \right) 
            \otimes \left( c_g v, d_g w \right) 
        \end{align*}
        Hence, if we let
        $v_1 = e_1 \otimes e_1,
        v_2 = e_1 \otimes e_2,
        e_3 = e_2 \otimes e_1$ and
        $e_4 = e_2 \otimes e_2$, a matrix representation for
        $\pi \otimes \pi(g)$ in the basis
        $\mathcal{B} = \left\{ v_1,v_2,v_3,v_4 \right\} $ becomes
        \[
            \begin{pmatrix} c_g^2 & 0 & 0 & 0\\
            0 & c_g d_g & 0 & 0\\
        0 & 0 & c_g d_g & 0\\
     0 & 0 & 0 & d_g^2 \end{pmatrix} 
        \] 

        Another way to see this is as follows: By example 3.13, 
        $\pi \otimes \pi$ is equivalent to
        $\prod \colon G \to 
        \GL_{2 \times 2}(V)$ given by
        $\prod (g) A = 
        \left( \pi \right)_{\mathcal{B}}(g)
        A \left( \pi \right)_{\mathcal{B}}(g)^{t}.
        $ 
        Choosing any basis $\mathcal{B}$, we find that
        $\pi(g) = \left( \pi_1(g), \pi_2(g) \right) $ and
        each $\pi_i(g)$ is constant nonzero, so its matrix 
        representation becomes
        \[
        \pi(g) = 
        \begin{pmatrix} c & 0\\
        0 & d \end{pmatrix} 
        \] 
        Hence if
        $A = \begin{pmatrix} \alpha_{11} & \alpha_{12}\\
        \alpha_{21} & \alpha_{22}\end{pmatrix} $, we get

        \[
        \pi(g) A \pi(g)^{t}
        = 
        \begin{pmatrix} c & 0 \\
        0 & d \end{pmatrix} 
            \begin{pmatrix} \alpha_{11} & 
            \alpha_{12} \\
        \alpha_{21} & \alpha_{22} \end{pmatrix} 
                \begin{pmatrix} c & 0 \\
                0 & d \end{pmatrix} 
                =
                \begin{pmatrix}
                    c & 0\\0 & d
                \end{pmatrix} 
                    \begin{pmatrix} \alpha_{11} c &
                        \alpha_{12} d\\
                        \alpha_{21} c & 
                        \alpha_{22} d
                    \end{pmatrix} 
                    = \begin{pmatrix} 
                        \alpha_{11} c^2 & \alpha_{12} cd\\
                        \alpha_{21} cd & \alpha_{22} d^2
                    \end{pmatrix} 
        \] 
        Hence each subspace
        $V_i \subset 
        M_{2\times 2}(F)$ where
        $V_i$ is spanned by the matrix
        which has entry $1$ in the $i$ th entry and $0$ elsewhere,
        where we just enumerate the entries using a lexicographic
        ordering, for example.
        So
        \begin{align*}
            V_1 
            &= \Span  \begin{pmatrix} 1 & 0\\
            0 & 0\end{pmatrix} \\
            V_2 
            &= \Span \begin{pmatrix} 0 & 1 \\0 & 0 \end{pmatrix} \\
            V_3 &= \Span \begin{pmatrix} 0 & 0 \\ 1 & 0 \end{pmatrix} \\
            V_4 &= \Span \begin{pmatrix} 0 & 0 \\ 0 & 1 \end{pmatrix} 
        \end{align*}
            These are all $1$-dimensional hence
            irreducible. As shown above,
            they are also invariant under $\prod$.
            So
            \[
            \pi \otimes \pi
            = 
            (\pi \otimes \pi)_{V_1} \oplus
            \ldots \oplus
            (\pi \otimes \pi)_{V_4}
            \] 
            is a decomposition of
            $\pi \otimes \pi$ into irreducible components.\\
            \linebreak
            Now we wish to find
            $\Sym^2 \pi$.
            Recall that
            \[
            \Sym^2 k^2
            = k^2 \otimes k^2 / I
            \] 
            where $I$ is generated by
            elements of the form
            \[
            v \otimes w - w \otimes v.
            \] 
            Thus we get the following identification in
            $\Sym^2 k^2$:
            \begin{align*}
                (0,1) \otimes (1,0) 
                &= \left( 1,0 \right) \otimes (0,1)\\
            \end{align*}
            So in fact,
            \[
            \Sym^2 k^2
            \cong
            \Span \left( \left( 1,0 \right) \otimes 
            (1,0) \right) \oplus
            \Span \left( \left( 0,1 \right) \otimes
            (1,0)\right) 
            \oplus
            \Span\left( (0,1) \otimes (0,1) \right) 
            \] 
            Now, since
            $\Sym^2 \pi$ is simply the descent of
            $\pi \otimes \pi$ to the quotient
            $k^2 \times k^2 / I$, 
            $V_1$ and $V_4$ remain invariant. Furthermore,
            $\Sym^2 \pi 
            \left( \left( 1,0 \right) \otimes
            \left( 0,1) \right) \right) 
            = \left( 1,0 \right) \otimes
            (0,1)$, so
            $W_2:= \Span
            \left( \left( 0,1 \right) \otimes
            (1,0)\right) $ is also invariant.
            Let $W_1 := V_1$ and
            $W_3 := V_4$. Then
            we have shown that
            $W_1,W_2$ and $W_3$ are invariant under
            $\Sym^2 \pi$.
            Furthermore, each is $1$-dimensional, and 
            $\Sym^2 k^2$ decomposes to a
            direct sum
            $\Sym^2 k^2 \cong
            W_1 \oplus W_2 \oplus W_3$.
            Since each
            $\Sym^2 \pi|_{W_i}$ is invariant, this gives
            a decomposition of
            $\Sym^2 \pi$ into three irreducible representations.







    \end{solution}

    



    %\bibliography{../refs.bib}
\end{document}
