\documentclass[a4paper]{article}

\usepackage[margin=2.5cm]{geometry}
\usepackage[pdftex]{graphicx}
\usepackage[utf8]{inputenc}
\usepackage[T1]{fontenc}
\usepackage{textcomp}
\usepackage{babel}
\usepackage{amsmath, amssymb}
\usepackage[colorlinks=true,linkcolor=blue]{hyperref}
\usepackage{float}
\usepackage{mathrsfs}
%\usepackage{enumitem}

\newcommand{\incfig}[2][1]{%
\def\svgwidth{#1\columnwidth}
\import{./figures/}{#2.pdf_tex}
}


% figure support
\usepackage{import}
\usepackage{xifthen}
\pdfminorversion=7
\usepackage{pdfpages}
\usepackage{transparent}

\pdfsuppresswarningpagegroup=1

\setlength\parindent{0pt}

\newcommand{\qed}{\tag*{$\blacksquare$}}
\newcommand{\qedwhite}{\hfill \ensuremath{\Box}}

%Inequalities
\newcommand{\cycsum}{\sum_{\mathrm{cyc}}}
\newcommand{\symsum}{\sum_{\mathrm{sym}}}
\newcommand{\cycprod}{\prod_{\mathrm{cyc}}}
\newcommand{\symprod}{\prod_{\mathrm{sym}}}

%Linear Algebra

%Redeclaring Span and image
\DeclareMathOperator{\Span}{span}
\DeclareMathOperator{\Ima}{Im}
\DeclareMathOperator{\diag}{diag}
\DeclareMathOperator{\Ker}{Ker}

%Row operations
\newcommand{\elem}[1]{% elementary operations
\xrightarrow{\substack{#1}}%
}

\newcommand{\lelem}[1]{% elementary operations (left alignment)
\xrightarrow{\begin{subarray}{l}#1\end{subarray}}%
}

%SS
\DeclareMathOperator{\supp}{supp}
\DeclareMathOperator{\Var}{Var}

%NT
\DeclareMathOperator{\ord}{ord}

%Alg
\DeclareMathOperator{\Rad}{Rad}
\DeclareMathOperator{\Jac}{Jac}

\DeclareMathAlphabet{\pazocal}{OMS}{zplm}{m}{n}
\newcommand{\unif}{\pazocal{U}}

\begin{document}
    
\textbf{Opgave 1:} Let $p$ be prime.\\
(a) Show that any ring with unit and order $|R| = p^2$ is commutative.\\
\linebreak
\textit{Solution:} Assume $R$ is not commutative. Let $a,b \in R$ such that
$ab \neq ba$. By Lagrange, the order of $a,b$ must divide $p^2$. If 
the order of $a$ is $p^2$ then $a $ generates $R$ so in particular
$ab = a a^{k} = a^{k} a = ba$. Hence both $a$ and $b$ must have order $p$.




\subsection{Eksamen 2021}

\textbf{Opgave 4:} $f(x) = x^{4} + 4x^2 + 6.$ \\
(a) Vis, at $f(x)$ er irreducibelt i $\mathbb{Q} [x]$.\\
\textit{Solution:} Brøklegemet for $\mathbb{Z}$ er $\mathbb{Q}$, så per Gauss' 
lemma er $f(x)$ irreducibelt i $\mathbb{Q}[x]$ hvis og kun hvis det er
irreducibelt i $\mathbb{Z}[x]$ idet det er monisk. Men i $\mathbb{Z}$ er $(2)$
et primideal, og $2  \mid 4,6$, men $2^2 = 4 \nmid 6$, så per Eisensteins
kriterium, er $f(x)$ irreducibelt i $\mathbb{Z}[x]$ og $\mathbb{Q}[x]$.\\
\linebreak
(b) Bestem en irreducibel opløsning af $f(x)$ i $\left( \mathbb{Z}/11\mathbb{Z}
\right) [x]$.\\
\linebreak
\textit{Solution:} Vi har, at $1$ og $-1$ er rødder modulo $11$, så
\[
    x^{4} + 4x^2 + 6 \equiv (x-1)(x+1)(x^2 +5)
\] 
Vi har $\binom{-5}{11} = -1$, så højresiden er faktoriseringen i irreducible
faktorer.\\
\linebreak
(c) Vis, at $f(x)$ har en faktorisering $f(x) = g(x) h(x)$ i
$\left( \mathbb{Z}\left[ \sqrt{-2}  \right]  \right) [x]$, hvor $g(x),h(x)$ har
grad $2$.\\
\linebreak
\textit{Solution:} Vi har $x^{4} + 4x^2 + 6 = 0$ i $\mathbb{C}$ hvis og kun
hvis
$x^2 = \frac{-4 \pm \sqrt{-8} }{2} = -2 \pm \sqrt{-2} $, så
\[
x^4 + 4x^2 + 6 = \left( x^2 - \left( -2 + \sqrt{-2}  \right)  \right) 
\left( x^2 - \left( -2 - \sqrt{-2}  \right)  \right).
\] 
(d) Vis, at $f(x)$ ikke har nogen rod i $\mathbb{Z}\left[ \sqrt{-2}  \right]
$.\\
\linebreak
\textit{Solution:} $\mathbb{Z}\left[ \sqrt{-2}  \right] $ er UFD, dermed også
et integritetsområde, så hvis $f$ har rod i $x_0$, må højresiden i (c) være
nul og dermed en af dets faktorer være nul og dermed reducibel med en lineær
faktor.\\
Da $N \left( -2 + \sqrt{-2}  \right) = 4 +2 = 6$, ville
$N(x_0^2) = 6$, så hvis $x_0 = a+ b\sqrt{-2} $, er
$N(x_0^2) = N(x_0)^2 = (a^2 + 2 b^2)^2 = 6$. Hvis $b=1$, fås en modstrid ved at
tjekke efter, og lignende hvis $b=0$.


\subsection{Reeksamen 2021}

\textbf{Opgave 4:} $f(x) = x^{4} - 6x^2 +12$.\\
\linebreak
(a) Vis, at $f(x)$ er irreducibelt i $\mathbb{Q}[x]$.\\
\linebreak
\textit{Solution:} Da $f(x)$ er monisk, er $f$ irreducibelt i $\mathbb{Q}[x]$
hvis og kun hvis $f$ er irreducibel i $\mathbb{Z}[x]$. I $\mathbb{Z}$ er $(3)$
et primideal, og $3 \mid -6, 12$, men $3^2 \nmid 12$, så per Eisensteins
kriterium, er $f$ irreducibel i $\mathbb{Z}[x]$ og dermed også
i $\mathbb{Q}[x]$.\\
\linebreak
(b) Det er klart, at $\pm 1$ er rødder i $f(x)$, så
\[
    x^{4} - 6x^2 + 12 \equiv (x-1)(x+1) (x^2 -5) \pmod{7}
\] 
De lineære faktorer er irreducible, da $\mathbb{Z}/7\mathbb{Z}$ er et
integritetsområde, og da $\binom{5}{7} = \binom{2}{5} = (-1)^{\frac{24}{8}}
= -1$, er $(x^2 -5)$ også irreducibel i $\mathbb{Z}/7\mathbb{Z}[x]$.\\
\linebreak
$g(x) = x^3 + x + 4$.\\
(c) Vis, at $g(x)$ er irreducibelt i $\mathbb{Z}/5\mathbb{Z}[x]$.\\
\linebreak
\textit{Solution:} $\mathbb{Z}/5\mathbb{Z}$ er et legeme, så hvis $g(x)$ var
reducibelt, ville den have rod i $\mathbb{Z}/5\mathbb{Z}$ - dvs. en lineær
faktor -, men ved indsættelse ses
$g(0)=4, g(1) = 6, g(2) = 4, g(3) = 4, g(4) = 2$ alle modulo $5$, så  da $g$
ikke har nogen rod, er $g$ irreducibel i $\mathbb{Z}/5\mathbb{Z} [x]$.\\
\linebreak
(d) Vis, at $g(x)$ er irreducibelt i $\mathbb{Q}[x]$.\\
\linebreak
\textit{Solution:} Per rationel rod testen, er de eneste mulige rødder
i $\mathbb{Q}$ netop $\pm 1, \pm 2$ og $\pm 4$.\\
Men det er klart, at for alle positive versioner, er $g(x)>0$, og $g(-1) = 2,
g(-2)= -6, g(-3) = -26$, så ingen af disse er rødder. Dermed har $g$ ingen
rødder i $\mathbb{Q}[x]$. Da $\mathbb{Q}$ er et legeme, følger derfor, at $g$
ikke har nogen lineære faktorer og er dermed ikke reducibel.


\subsection{Prøveeksamen}

\textbf{1:}\\
(a) Vi har, at $0\neq \overline{a} \in \mathbb{Z}/15\mathbb{Z}$ er en nuldivisor, hvis og kun
hvis $ab \in 15\mathbb{Z}$ for et $\overline{b}\neq 0$. Da $\mathbb{Z}$ er UFD
og $15 = 3 \cdot 5$, må $3,5  \mid ab$. Vi kan ikke have, at $3,5  \mid b$, da
 $\overline{b}\neq 0$, så enten må $3  \mid a$ eller $5 \mid a$. Vi kan heller
 ikke have $3,5 \mid a$ af samme grund. Så vi får
 $\overline{a} = 3$ eller $\overline{a}=5$.\\
 For alle andre $\overline{a} \in \mathbb{Z}/15\mathbb{Z}$, må $(a,15)=1$, så
 per Bezout eksisterer $s,t \in \mathbb{Z}$ så $as+15t = 1$, så
 $\overline{a} \overline{s}=1$, hvormed $a$ er en enhed.\\
 \linebreak
 (b) Vi har, at $N ( 2+ \sqrt{6} ) = 4 - 6 = -2$, som er $\pm$ et primtal,
 hvormed den er irreducibel per Generel bemærkning 1.\\
 \linebreak
 (c) Det er det ikke. $\mathbb{C}$ er et legeme og dermed et integritetsområde.
 Vi har nu $x^2 \cdot x = x^3 \in (x^3)$. Hvis $x^2$ eller $x \in (x^3)$, ville
 $x^2 = x^3 q(x)$ eller $x = x^3 q(x)$, men da det er et integritetsområde, fås
 $2 = deg x^2 = deg x^3 + deg q = 3 + deg q \ge 3$, som er en modstrid.\\
 \linebreak
 (d) $\frac{x}{1}$ er ikke en enhed i $\mathbb{C}[x]_{(x)}$. Antag for
 modstrid, at
 $\frac{x}{1} \frac{r}{s} = \frac{1}{1}$, så må $xr u(x) = s u(x) \in
 \mathbb{C}[x] \backslash (x)$, men $x r u(x) = s u(x) \in (x)$; dermed
 modstrid, da $s, u $.\\
 \linebreak
 (e) $7$ og $x$ er ikke enheder og ikke nul i $\mathbb{Z}[x]$. De er desuden
 irreducible, da $7$ er irreducibel i $\mathbb{Z}$ og dermed i $\mathbb{Z}[x]$ 
 og
 da $x$ er primitiv og irreducibel i $\mathbb{Q}[x]$ og dermed
 i $\mathbb{Z}[x]$. Dermed er $7x$ reducibel i $\mathbb{Z}[x]$.\\
 $7x$ er dog irreducibel i $\mathbb{Q}[x]$, da den er en lineær faktor og

 \textbf{Opgave 2:}

 (a) 
 \begin{align*}
     x^{7} + x^{5} - 3x^2 -3
     &= (x^{6} + x^{5} - 3x -3)(x -1) + 2 x^{5}  - 6\\
     (x^{6} + x^{5} - 3x -3) 
     &= (2x^{5} - 6) (\frac{1}{2}x  + \frac{1}{2})
 \end{align*}
 Så gcd er derfor  $2x^{5} -6$, som er associeret til $x^{5} - 3$.\\
 (b) Vi har, at $x^{5}-3$ er monisk, så den er reducibel i $\mathbb{Q}[x]$ hvis
 og kun hvis den er reducibel i $\mathbb{Z}[x]$, men i $\mathbb{Z}[x]$ er den
 Eisenstein i  $3$, så dermed irreducibel.\\
 \linebreak
 (c) Vi har, at $\mathbb{Q}[x]$ er Euklidisk så specielt PID, da $\mathbb{Q}$ 
 er et legeme. Dermed for $\left( f(x),g(x) \right) =(d(x))$, og da $d(x)$ er
 irreducibel, er $\mathbb{Q}[x] /\left( d(x) \right) $ et legeme per en
 opgave.\\
 \textbf{Opgaven:} Hvis $\mathbb{F}$ er et legeme, og $f(x) \in \mathbb{F}[x]$ 
 er irreducibel, da er $\mathbb{F}[x]/\left( f(x) \right) $ et legeme.\\
 \textbf{Bevis:} I et PID er et ikke-nul primideal et maksimalideal og
 irreducible elementer er primelementer, så $(f(x))$ er et maksimalideal.\\
 \linebreak
 (d) Vi har 
 \[
 x^{5} -3 = x \cdot x^{4} - 3,
 \] 
 så i legemet, er $\overline{x} \cdot \frac{1}{3} \overline{x^{4}}
 = \overline{1}$.\\
 \linebreak
 \textbf{Opgave 3:}\\
 (a) Vi har $x^2 - (x+1)x = -x$ og $x+1 + -x = 1$, så
 for $a_1= x^2$ og $a_2 = -(x+1) x + (x+1)$, fås $a_1 + a_2 =1$. Dermed er
 $(x^2)$ og $(x+1)$ komaksimale.\\
 \linebreak
 (b) Vi har, $(x^2)(x+1) = \left( (x^2)(x+1) \right) = \left( x^3 + x^2 \right)
 $, så per den kinesiske restklassesætning, er
 \[
     \mathbb{R}[x]/\left( x^3 + x^2 \right) \cong \mathbb{R}[x]/(x^2) \times 
     \mathbb{R}[x]/(x+1)
 \] 
 ved $[f(x)] \to \left( [f]_{(x^2)}, [f]_{(x+1)} \right) $.\\
 \linebreak
 (c) Den inverse afbildning er givet ved
 \[
     ([a],[b]) \to \left[ a a_2 + b a_1 \right],
 \] 
 så $\left( x,1 \right) \to \left[ x (x+1)(1-x) + x^2 \right] = \left[ 
 -x^3 + x^2 + x\right] $.\\
 \linebreak
 (d) Lad $\varphi   \colon R[x] \to \mathbb{R}$ ved evaluering i $-1$.\\
 Denne er klart surjektiv og har kerne $(x+1)$.\\
 Da $\varphi (f+g) = (f+g)(-1) = f(-1) + g(-1) = \varphi(f) + \varphi(g)$ 
 og $\varphi(fg) = (fg)(-1) = f(-1) g(-1) = \varphi(f) \varphi(g)$ er $\varphi$ en
 ringhomomorfi, som etablerer den givne isomorfi.\\
 \linebreak
 \textbf{Opgave 4:} $f(x) = x^{5} + 21x +63$.\\
 \linebreak
 (a) Da $f$ er primitiv, er $f$ irreducibel i $\mathbb{Q}[x]$ hvis og kun hvis
 den er irreducibel i $\mathbb{Z}[x]$. Men $f$ er Eisenstein i $7$
 i $\mathbb{Z}[x]$, og dermed irreducibel.\\
\linebreak
(b) Da $\lim_{x \to \infty} f(x) = \infty$ og $\lim_{x \to -\infty} f(x)
= -\infty$ og $f$ er kontinuert, har $f$ en rod i $\mathbb{R}$. Dermed har den
en lineær faktor $(x-\alpha)$ for et $\alpha \in \mathbb{R}$.\\
\linebreak
(c) I $\left( \mathbb{Z}/2\mathbb{Z} \right) [x]$ er $f(x) = x^{5} + x + 1$.
$f$ har ingen rod i $\mathbb{Z}/2\mathbb{Z}$, så en faktorisering ville skulle
have form
\[
x^{5} + x + 1 = (x^3 + ax^2 + bx +c)(x^2 + d x + e)
= x^{5} + ax^{4} + (e + ad + b) x^3 + (a + bd + c) x^2 + (be + cd)x + c e
\] 
så $a = 0$, og $e+b = 0, bd+c = 0, be+cd = 1, ce = 1$.\\
Da må $cd - b^2 = 1$. Vi har $e = 1$, så $b=1$, så $cd = 0 \implies d=0$. Dette
giver
\[
x^{5} + x + 1 = (x^3 + x +1) (x^2 + 1).
\] 
(d) Da $N(7) = 49 = 7^2$, ville en faktorisering have faktornorm $7$, men
$a^2 + 2 b^2 = 7$ er umuligt, så $7$ er irreducibel i $\mathbb{Z}\left[
\sqrt{-2}  \right] $ og dermed også i $\mathbb{Z}\left[ \sqrt{-2}  \right]
[x]$. Da $f$ er Eisenstein i $7$, som dermed også er et primelementer, da
$\mathbb{Z}\left[ \sqrt{-2}  \right] $ er PID, er $f$ irreducibelt.




\end{document}
