\documentclass[a4paper]{article}

\usepackage[margin=2.5cm]{geometry}
\usepackage[pdftex]{graphicx}
\usepackage[utf8]{inputenc}
\usepackage[T1]{fontenc}
\usepackage{textcomp}
\usepackage{babel}
\usepackage{amsmath, amssymb}
\usepackage[colorlinks=true,linkcolor=blue]{hyperref}
\usepackage{float}
\usepackage{mathrsfs}
%\usepackage{enumitem}

\newcommand{\incfig}[2][1]{%
\def\svgwidth{#1\columnwidth}
\import{./figures/}{#2.pdf_tex}
}


% figure support
\usepackage{import}
\usepackage{xifthen}
\pdfminorversion=7
\usepackage{pdfpages}
\usepackage{transparent}

\pdfsuppresswarningpagegroup=1

\setlength\parindent{0pt}

\newcommand{\qed}{\tag*{$\blacksquare$}}
\newcommand{\qedwhite}{\hfill \ensuremath{\Box}}

%Inequalities
\newcommand{\cycsum}{\sum_{\mathrm{cyc}}}
\newcommand{\symsum}{\sum_{\mathrm{sym}}}
\newcommand{\cycprod}{\prod_{\mathrm{cyc}}}
\newcommand{\symprod}{\prod_{\mathrm{sym}}}

%Linear Algebra

%Redeclaring Span and image
\DeclareMathOperator{\Span}{span}
\DeclareMathOperator{\Ima}{Im}
\DeclareMathOperator{\diag}{diag}

%Row operations
\newcommand{\elem}[1]{% elementary operations
\xrightarrow{\substack{#1}}%
}

\newcommand{\lelem}[1]{% elementary operations (left alignment)
\xrightarrow{\begin{subarray}{l}#1\end{subarray}}%
}

%SS
\DeclareMathOperator{\supp}{supp}
\DeclareMathOperator{\Var}{Var}

%NT
\DeclareMathOperator{\ord}{ord}

\DeclareMathAlphabet{\pazocal}{OMS}{zplm}{m}{n}
\newcommand{\unif}{\pazocal{U}}

\begin{document}
    \textbf{5.3:} Is $2^{10} \cdot \left( 2^{11}-1 \right) $ perfect?\\
    \linebreak
    \textit{Solution:} If it were perfect, then by theorem 5.1.2, 
    $2^{11}-1$ would have to be prime. Now, since $\varphi(23)=22=2 \cdot 11$ 
    and
    $(2,23)=1$, we have $2^{22}-1 \equiv 0 \pmod{23}$, so
    if $2^{11}-1$ is prime, $23$ must divide $2^{11}+1$, i.e. $2$ must be
    a primitive root modulo 23. However, we find
    $2^{5} = 32 \equiv 9 \to -5 \to -10 \to 3 \to 6 \to 12 \to 1 \equiv 2^{11} 
    \pmod {23}$, so $\ord(2) = 11$, and hence $2$ is not a primitive root.\\
    \linebreak
    \textbf{Exercise 5.4:} Show that $\sum_{d \mid n} \left| \mu (d) \right| 
    = \Pi_{p  \mid n}2$.\\
    \linebreak
    \textit{Solution:} Assume $n= p_1^{\alpha_1} p_2^{\alpha_2} \ldots
    p_k^{\alpha_k}$. Then
    \[
    \sum_{d \mid n} \left| \mu (d) \right| 
    = \sum_{d  \mid p_1 p_2 \ldots p_k} \left| \mu(d) \right| 
    = \sum_{d  \mid p_1 \ldots p_k} 1
    = \tau (p_1 \ldots p_k) 
    \stackrel{prop. 5.1.1}{=} \Pi_{i=1}^{k} 2 = \Pi_{p  \mid n} 2
    .\] 
    \textbf{Exercise 5.5:}\\
    Assume that there exists a complex sequence $s_k = \sigma_k + it_k$ with
    $\sigma_k \to \infty$ when $k\to \infty$ satisfying that $D_f (s_k)=0$ for
    all $k$ sufficiently large. We will show that this implies that $f(n)=0$ 
    for all $n$.\\
    \linebreak
    (i) Assume that $f$ is not identically zero, and choose $N \in \mathbb{N}$
    minimal such that $f(N) \neq 0$. Then for all $k$ sufficiently large. we
    have
    \begin{align*}
        0 &= \sum_{n=N}^{\infty} f(n) n^{-s_k}\\
        \iff f(N) &= - N^{s_k} \sum_{n=N+1}^{\infty} f(n)n^{-s_k}.
    .\end{align*}
    (ii) Let $c > \sigma_a$. Then for $k$ sufficiently large, we have
    \begin{align*}
        \left| f(N) \right| 
        &\leq -N^{\sigma_k}  \sum_{n=N+1}^{\infty} \left| f(n)\right| \left|
        n^{-(\sigma_k - c)} \right| \left| n^{-c} \right| \\
        &\le -N^{\sigma_k} \left( N+1 \right)^{-(\sigma_k -c)}
        \sum_{n=N+1}^{\infty} \left| f(n) \right|  n^{-c} 
    .\end{align*}
    (iii) Since $\left( \frac{N}{N+1} \right)^{\sigma_k} \to 0$ for
    $\sigma_k \to \infty$ which is equivalent to letting $k \to \infty$ by
    assumption. Since this is true for all $k$ sufficiently large, we let $k\to
    \infty$ and find $\left| f(N) \right| \le 0$ since
    $\sum_{n=N+1}^{\infty} |f(n)| n^{-c}<\infty$ as $c> \sigma_a$, hence $f(n) = 0$ for all $n
    \in \mathbb{N}$.\\
    \linebreak
    \textbf{Exercise 5.6:} We will show that the sum of reciprocal of primes,
    $\sum_{p\text{ prime}} p^{-1}$, diverges.\\
    Let $p_n$ denote the n'th prime. Assume that
    $\sum_{n=1}^{\infty} p_n^{-1}$ is convergent with sum $l$.\\
    \linebreak
    (i) By assumption, since $\sum_{n=1}^{\infty} p_n^{-1} = l$, there must
    exist $N \in \mathbb{N}$ such that
    \[
    \left| \sum_{n>N} p_n^{-1} \right| = \left| l - \sum_{n=1}^{N} p_n^{-1} \right| 
    \le \frac{1}{2}.
    \] 
    (ii) Now
    \[
    \sum_{k=1}^{\infty} \left| \sum_{n>N} p_n^{-1} \right|^{k}
    \le \sum_{k=1}^{\infty} \left( \frac{1}{2} \right)^{k} = 1,
    \] 
    so $\sum_{k=1}^{\infty} \left( \sum_{n>N} p_n^{-1} \right)^{k}$ is
    absolutely convergent.\\
    \linebreak
    (iii) Let $W = p_1 \ldots p_N$. For $r \in \mathbb{N}$ consider $Wr +1$.
    Since all $p_i  \mid W$, we have $p_i \nmid Wr+1$.\\
    \linebreak
    (iv) Now, by Cauchy multiplication, we have
    \begin{align*}
        \left( \sum_{n>N} p_n^{-1} \right)^{k}
        &= \sum_{n_1, \ldots, n_k > N} \left( p_{n_1} \ldots p_{n_k}
        \right)^{-1}\\
        &= \sum_{\substack{n = q_1 \ldots q_k \\ q_i \text{ prime}\\
        q_i > p_N}} \frac{1}{n}.
    .\end{align*}
    (v) Now, for the sum
    \[
    \sum_{r=1}^{\infty} \frac{1}{Wr+1},
    \] 
    we find that it is a sum of reciprocals of numbers whose prime factors are
    all greater than $p_N$, so each term is contained in the series
    \[
    \sum_{\substack{n=q_1 \ldots q_k\\ q_i \text{ prime} \\ q_i > p_N}}
    \frac{1}{n}
\]
for some $k$ (by the fundamental theorem of arithmetic), so
\begin{align*}
    \sum_{r=1}^{\infty} \frac{1}{Wr+1}
    &\le \sum_{k=1}^{\infty} 
    \sum_{\substack{n=q_1 \ldots q_k\\ q_i \text{ prime} \\ q_i > p_N}}
    \frac{1}{n}\\
    &= \sum_{k=1}^{\infty} \left( \sum_{n>N} p_n^{-1} \right)^{k}
.\end{align*}
(vi)   
Now 
\begin{align*}
    \infty 
    &= \frac{1}{W} \sum_{r=1}^{\infty} \frac{1}{r+1}\\
    &= \sum_{r=1}^{\infty} \frac{1}{Wr + W}\\
    &\le \sum_{r=1}^{\infty} \frac{1}{Wr+1}\\
    &\le \sum_{k=1}^{\infty} \left( \sum_{n>N} p_n^{-1} \right)^{k}
.\end{align*}
This contradicts (ii), so we are done.\\
\linebreak
\textbf{Exercise 5.7:} Let $g(n)$ be the sum of primitive nth roots of $1$,
i.e.
\[
    g(n) = \sum_{\substack{\zeta^{n}=1 \\ \zeta^{m}\neq 1\\ 
    \text{for } 0<m<n}} \zeta.
\] 
Claim: $\mu(n) = g(n)$.\\
\textbf{Proof:} We have
\[
    g(n) = \sum_{\substack{\zeta^{n}=1\\ \zeta^{m}\neq 1\\
    0 < m < n}} \zeta 
    = \sum_{\substack{\gcd(k,n)=1\\ 0<k<n}} \zeta_{n}^{k}.
\] 

We prove that $g$ is multiplicative first.\\
Assume $n = st$ where $(s,t)=1$. Let $a,b \in \mathbb{Z}$ such that
$(a,s)=1=(b,t)$. Then $e^{\frac{2\pi i}{s}a} e^{\frac{2 \pi i}{t}b}
= e^{\frac{2\pi i \left( at + bs  \right) }{n}}$. We claim
$\gcd(at+bs ,n)=1$. If $p  \mid at+bs  \mid n$, then $p  \mid s$ or $p \mid t$.
Assume wlog $p  \mid s$. Then $p  \mid at$, but since $(s,t)=1$, $p  \mid a$,
however $(a,s)=1$. Contradiction. So $\gcd(at+bs,n)=1$.\\
Conversely, if $(k, st)=1$ then since $(s,t)=1$, write $us+vt = 1$, then
$s(uk)+ t(vk)=k$, so
$e^{\frac{2\pi i \left( suk + tvk \right) }{st}} = e^{\frac{2\pi i}{t}uk}
e^{\frac{2 \pi i}{s}vk}$. Now, $(k,st)=1$ so $(k,t)=1$ and $(u,t)=1$ since 
$(u,t)  \mid 1$. Similarly $(s,vk)=1$.\\
Thus we have
\[
    g(st) = \sum_{\substack{\gcd(k,st)=1\\ 0 < k < st}} \zeta_{st}^{k}
    = \sum_{\substack{\gcd(k,s)=1\\ 0 < k < s}} \zeta_{s}^{k} \cdot 
    \sum_{\substack{\gcd(k,t)=1\\ 0 < k <t}} \zeta_{t}^{k} 
    = g(s) g(t).
\] 
Now, firstly we have for any prime  $p$ that
$g(p) = \sum_{\substack{\gcd(k,p)=1 \\ 0 < k <p}} \zeta_p^{k}
= \zeta_p + \zeta_p^2 + \zeta_p^3 + \ldots + \zeta_{p}^{p-1}
= \frac{\zeta_{p}^{p} - \zeta_p}{\zeta_p - 1} = \frac{1- \zeta_p}{\zeta_p - 1} 
= -1$.\\
It thus just remains to show that for any $\alpha \ge 2$, 
$g\left( p^{\alpha} \right) =0$. Now
\begin{align*}
    g\left( p^{\alpha} \right) 
    &= \sum_{\substack{\gcd(k,p^{\alpha})\\ 0<k<p^{\alpha}}}
    \zeta_{p^{\alpha}}^{k}\\
    &= \sum_{k=0}^{p^{\alpha}-1} \zeta_{p^{\alpha}}^{k} 
    - \sum_{n=0}^{\alpha - 1} \zeta_{p^{\alpha}}^{k}\\
    &= \frac{\zeta_{p^{\alpha}}^{p^{\alpha}}-1}{\zeta_{p^{\alpha}-1}}
    - \frac{\zeta_{p^{\alpha}}^{p^{\alpha}}-1}{\zeta_{p^{\alpha}}-1}\\
    &= 0.
\end{align*}
Combining these 3 results we find $g(n) = \mu(n)$.


\textbf{Exercise 5.9:} Prove Gottschalck's theorem: Let $n$ be a $k$-perfect
number such that $2$ divides $n$ precisely $m$ times. Then
$2^{m} \left( 2^{m+1}-1 \right) $ divides $kn$.\\
\linebreak
\textit{Solution:} By proposition 5.1.1, we have have for $n = 2^{m}r$,
\[
    nk = \sigma (2^{m}r) \stackrel{\text{multiplicative}}{=} \sigma(2^{m})
    \sigma(r) \stackrel{5.1.1}{=} (2^{m+1}-1) \sigma(r).
\]
Now, since $2^{m} \nmid 2^{m+1}-1$ but $2^{m} \mid n$, we must have
$2^{m} \mid \sigma(r)$, thus we get
$nk = (2^{m+1}-1) 2^{m}\sigma(r)'$ from which the result follows.


\textbf{7.7:} Prove theorem 7.1.5 by showing - using propositions 7.1.1 and
7.1.2, combined with (7.6) - that
\[
\left| c_n - c_m \right| \le \sqrt{2} \sum_{k=m}^{n} \frac{1}{2^{k}},
\] 
and conclude that $c_n$ is a Cauchy sequence.\\
\linebreak
\textit{Solution:}  
We proceed by induction. We have
\[
\left| c_{n+1}-c_n \right| = \left| \frac{p_{n+1}q_n - p_n q_{n+1}}{q_n q_{n+1}} \right| 
\le \frac{\sqrt{2} }{2^{n}} = \sqrt{2} \sum_{k=n}^{n} \frac{1}{2^{k}}.
\] 
Now we claim that for $m>n$, $\left| c_m - c_n \right| \le \sqrt{2} \sum_{k=n}^{m-1}
\frac{1}{2^{k}}$ which is stronger than what we wanted for $m >n$.\\
It is true when $m=n+1$. Assume it is true for $m=N > n$. Then for $m=N+1$,
\[
|c_{m}-c_n| \le \left| c_{N+1} - c_N \right| + \left| c_N - c_n \right| 
\le \frac{\sqrt{2} }{2^{N}} + \sqrt{2}  \sum_{k=n}^{N-1} \frac{1}{2^{k}}
= \sqrt{2}  \sum_{k=n}^{N} \frac{1}{2^{k}}.
\] 
Hence the result follows when $n\neq m$. If $n=m$, it is trivial.


\textbf{Exercise 6.6} Show that the average value of $r_2 (n)$ equals $\pi$,
i.e. that
\[
\frac{1}{n} \sum_{m=1}^{n} r_2 (m) \to \pi \quad \text{as }n \to \infty.
\] 
\textit{Solution:} Let $N(n)$ denote the number of lattice points inside the
circle of radius $n$. Then
 \[
N (r) = \sum_{n=0}^{r^2} r_2(n)
\] 
And geometrically, we can say that on average, if we place a random unit square
or disc of area 1, we would expect it to cover $1$ lattice point, so since the
number of unit discs that can fit inside a circle of radius $r$ approaches its
area as $r$ grows, we get
that $N(r) \approx \pi r^2$ for $r$ large, so
\[
\pi = \lim_{r \to \infty} \frac{N(r)}{r^2} = \frac{1}{r^2} \sum_{n=0}^{r^2} r_2
(n).
\] 
Since $\frac{1}{r^2} \sum_{n=0}^{r^2} r_2(n)$ is a monotone (positive)
subsequence of $\frac{1}{r} \sum_{n=0}^{r} r_2(n)$, we also have
\[
\lim_{n \to \infty} \frac{1}{n} \sum_{m=0}^{n} r_2(m) = \pi.
\] 















\newpage
\textbf{Exercise from class:} Write $5+i$ as a product of irreducibles in
$\mathbb{Z}[i]$.\\
\linebreak
\textit{Solution:} $N(5+i) = 25^2 + 1 = 26 = 2 \cdot 13 = (1+i)(1-i)
(2+3i)(2-3i)$. Now we simply guess and find
$(1-i)(2+3i) = 5+i $.\\
\linebreak









    

$r_2 (n) = 4 \cdot u * \chi_4 (n)$ where
$\chi_4(n) = \begin{cases}
    0 & (n,2)>0\\
    \left( -1 \right)^{\frac{n-1}{2}} & \text{otherwise}
\end{cases}$.



































\end{document}
