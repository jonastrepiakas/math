\documentclass[a4paper]{article}

\usepackage[margin=2.5cm]{geometry}
\usepackage[pdftex]{graphicx}
\usepackage[utf8]{inputenc}
\usepackage[T1]{fontenc}
\usepackage{textcomp}
\usepackage{babel}
\usepackage{amsmath, amssymb}
\usepackage[colorlinks=true,linkcolor=blue]{hyperref}
\usepackage{float}
\usepackage{mathrsfs}


\newcommand{\incfig}[2][1]{%
\def\svgwidth{#1\columnwidth}
\import{./figures/}{#2.pdf_tex}
}


% figure support
\usepackage{import}
\usepackage{xifthen}
\pdfminorversion=7
\usepackage{pdfpages}
\usepackage{transparent}

\pdfsuppresswarningpagegroup=1

\setlength\parindent{0pt}

\newcommand{\qed}{\tag*{$\blacksquare$}}
\newcommand{\qedwhite}{\hfill \ensuremath{\Box}}

%Inequalities
\newcommand{\cycsum}{\sum_{\mathrm{cyc}}}
\newcommand{\symsum}{\sum_{\mathrm{sym}}}
\newcommand{\cycprod}{\prod_{\mathrm{cyc}}}
\newcommand{\symprod}{\prod_{\mathrm{sym}}}

%Linear Algebra

%Redeclaring Span and image
\DeclareMathOperator{\Span}{span}
\DeclareMathOperator{\Ima}{Im}
\DeclareMathOperator{\diag}{diag}

%Row operations
\newcommand{\elem}[1]{% elementary operations
\xrightarrow{\substack{#1}}%
}

\newcommand{\lelem}[1]{% elementary operations (left alignment)
\xrightarrow{\begin{subarray}{l}#1\end{subarray}}%
}

%SS
\DeclareMathOperator{\supp}{supp}
\DeclareMathOperator{\Var}{Var}

\DeclareMathAlphabet{\pazocal}{OMS}{zplm}{m}{n}
\newcommand{\unif}{\pazocal{U}}


\title{Aflevering 1}
\author{Jonas Trepiakas - hvn548@alumni.ku.dk}
\date{}


\begin{document}
\maketitle
\newpage
    \textbf{Opgave 1:} Vi har
    \begin{align*}
        (1216,551) &= (1216 - 551 \cdot 2, 551)\\
                   &= (114, 551)\\
                   &= (114, 551 - 4\cdot 114)\\
                   &= (114,95)\\
                   &= (114-95, 95)\\
                   &= (19,95)\\
                   &= (19,0)\\
                   &= 19,
    \end{align*}
ved gentagen brug af proposition 1.2.3. Da $19 \nmid  56$ findes ingen
heltallige løsninger ifølge sætning 1.3.1. til
$1216 x+ 551y = 56$. Da $19 \cdot 4 = 76$, har vi $19  \mid 76$, og ifølge
sætning 1.3.1 er samtlige løsninger givet ved:
\[
    (x,y) = 4 \left( x_0, y_0 \right) + n \left(
    \frac{551}{19},-\frac{1216}{19} \right),
\] 
hvor $(x_0,y_0)$ er en vilkårlig løsning til systemet. En sådan løsning finder
vi ved:
\begin{align*}
    19 &= 114 - 95\\
       &= (1216-551 \cdot 2) - (551 - 4\cdot 114)\\
       &= (1216 - 551\cdot 2) - (551 - 4\cdot (1216- 551\cdot 2))\\
       &= 5 \cdot 1216 - 11\cdot 551
.\end{align*}
Hvormed $76 = 19\cdot 4 = 20 \cdot 1216 - 44 \cdot 551$, så $(x_0,
y_0)=(20,-44)$ er en løsning.


Samtlige reelle løsninger findes ved
\[
    1216 x + 551 y = 76
    \iff y = \frac{76-1216x}{551}
\]
og
\[
1216 x + 551 y = 56 \iff y = \frac{56-1216x}{551}
.\] 
Hvormed $(x,y) = \left( x, \frac{76-1216x}{551} \right) , (x,y)=(x,
\frac{56-1216x}{551})$ for $x \in \mathbb{R}$ er samtlige løsninger til
ligningssystemerne
$1216x+551y=76$ og $1216x+551y=56$ henholdsvis.\\
\linebreak
\textbf{Opgave 2:} 
Vi har
\[
51 \cdot 41 = 2091 = 110\cdot 19 + 1,
\] 
så $51 \cdot 41 \equiv 1 \pmod{110}$, dvs. $51^{-1} \equiv 41 \pmod{110}$,
hvormed
\[
    51 x \equiv 4 \iff x \equiv 4 \cdot 41 \equiv 164 \equiv 54 \pmod{110}
.\] 
Dermed er samtlige løsninger til kongruensligningen netop $x=54 +110n, n \in
\mathbb{Z}$.\\
\linebreak
\textbf{Opgave 3:} Vi har $3 \cdot 5 = 15 \equiv 1 \pmod{7}$, så
ligningssystemet er ækvivalent med
\begin{align*}
    x &\equiv 2 \pmod{13}\\
    x &\equiv 5\cdot 5 \equiv 4 \pmod{7}
.\end{align*}
Da $(13,7)=1$, eksisterer en løsning til systemet, der er unikt modulo $13\cdot
7$
ifølge den kinesiske restklassesætning.
Vi har $2 + 13 t \equiv 4 \pmod{7} \iff 13 t \equiv 2 \pmod{7} \iff t \equiv
6\cdot 2 \equiv 5 \pmod{7}$, så
$x = 2+13\cdot 5 = 67$ løser systemet. Samtlige løsninger er dermed givet ved
\[
x = 67 + 13\cdot 7 n = 67 + 91 n, \quad n \in \mathbb{Z}
.\] 
\textbf{Opgave 4:} Vi har $4^2 + 7^2 = 16 + 49 = 65$. Nu har vi fra sætning
1.4.6, at de primitive heltallige løsninger til
\[
X^2 + Y^2 = 65^2
\] 
med $X$ ulige og $Y$ lige, er netop
\[
    (X,Y,65) = \left( p^2 - q^2 , -2pq, p^2 + q^2 \right) \] 
hvor $(p,q)=1$ med  $p\ge 0$ og $p-q$ er ulige. Vi har, at $p=7, q=4$ opfylder
betingelserne, så $(X,Y) = \left( 7^2 - 4^2, -2\cdot 4\cdot 7 \right)
=(33,-56)$ løser ligningen med $\gcd(X,Y)=1$.\\
\linebreak
\textbf{Opgave 5:} Antag, at en Pythagoræisk tripel $(x,y,z)$ findes med
$z \ge 0$ og $ z\equiv 2 \pmod{4}$ eller $z\equiv 3 \pmod{4}$. Da er
\[
    x^2 + y^2 = z^2 \equiv 2 \text{ eller } 3 \pmod{4}
.\] 
Men da $2$ og $3$ ikke er kvadratiske rester modulo 4, findes ingen heltalige
løsninger til systemet og dermed ingen Pythagoræisk tripel med ovenstående
betingelser.




































\end{document}
