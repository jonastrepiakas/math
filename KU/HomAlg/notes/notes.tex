\documentclass[reqno]{amsart}
\usepackage{amscd, amssymb, amsmath, amsthm}
\usepackage{graphicx}
\usepackage[colorlinks=true,linkcolor=blue]{hyperref}
\usepackage[utf8]{inputenc}
\usepackage[T1]{fontenc}
\usepackage{textcomp}
\usepackage{babel}
%% for identity function 1:
\usepackage{bbm}
%%For category theory diagrams:
\usepackage{tikz-cd}


\setlength\parindent{0pt}

\pdfsuppresswarningpagegroup=1

\newtheorem{theorem}{Theorem}[section]
\newtheorem{lemma}[theorem]{Lemma}
\newtheorem{proposition}[theorem]{Proposition}
\newtheorem{corollary}[theorem]{Corollary}
\newtheorem{conjecture}[theorem]{Conjecture}

\theoremstyle{definition}
\newtheorem{definition}[theorem]{Definition}
\newtheorem{example}[theorem]{Example}
\newtheorem{exercise}[theorem]{Exercise}
\newtheorem{problem}[theorem]{Problem}
\newtheorem{question}[theorem]{Question}

\theoremstyle{remark}
\newtheorem*{remark}{Remark}
\newtheorem*{note}{Note}
\newtheorem*{solution}{Solution}



%Inequalities
\newcommand{\cycsum}{\sum_{\mathrm{cyc}}}
\newcommand{\symsum}{\sum_{\mathrm{sym}}}
\newcommand{\cycprod}{\prod_{\mathrm{cyc}}}
\newcommand{\symprod}{\prod_{\mathrm{sym}}}

%Linear Algebra

\DeclareMathOperator{\Span}{span}
\DeclareMathOperator{\im}{im}
\DeclareMathOperator{\diag}{diag}
\DeclareMathOperator{\Ker}{Ker}
\DeclareMathOperator{\ob}{ob}
\DeclareMathOperator{\Hom}{Hom}
\DeclareMathOperator{\Mor}{Mor}
\DeclareMathOperator{\sk}{sk}
\DeclareMathOperator{\Vect}{Vect}
\DeclareMathOperator{\Set}{Set}
\DeclareMathOperator{\Group}{Group}
\DeclareMathOperator{\Ring}{Ring}
\DeclareMathOperator{\Ab}{Ab}
\DeclareMathOperator{\Top}{Top}
\DeclareMathOperator{\hTop}{hTop}
\DeclareMathOperator{\Htpy}{Htpy}
\DeclareMathOperator{\Cat}{Cat}
\DeclareMathOperator{\CAT}{CAT}
\DeclareMathOperator{\Cone}{Cone}
\DeclareMathOperator{\dom}{dom}
\DeclareMathOperator{\cod}{cod}
\DeclareMathOperator{\Aut}{Aut}
\DeclareMathOperator{\Mat}{Mat}
\DeclareMathOperator{\Fin}{Fin}
\DeclareMathOperator{\rel}{rel}
\DeclareMathOperator{\Int}{Int}
\DeclareMathOperator{\sgn}{sgn}
\DeclareMathOperator{\Homeo}{Homeo}
\DeclareMathOperator{\SHomeo}{SHomeo}
\DeclareMathOperator{\PSL}{PSL}
\DeclareMathOperator{\Bil}{Bil}
\DeclareMathOperator{\Sym}{Sym}
\DeclareMathOperator{\Skew}{Skew}
\DeclareMathOperator{\Alt}{Alt}
\DeclareMathOperator{\Quad}{Quad}
\DeclareMathOperator{\Sin}{Sin}
\DeclareMathOperator{\Supp}{Supp}
\DeclareMathOperator{\Char}{char}
\DeclareMathOperator{\colim}{colim}


%Row operations
\newcommand{\elem}[1]{% elementary operations
\xrightarrow{\substack{#1}}%
}

\newcommand{\lelem}[1]{% elementary operations (left alignment)
\xrightarrow{\begin{subarray}{l}#1\end{subarray}}%
}

%SS
\DeclareMathOperator{\supp}{supp}
\DeclareMathOperator{\Var}{Var}

%NT
\DeclareMathOperator{\ord}{ord}

%Alg
\DeclareMathOperator{\Rad}{Rad}
\DeclareMathOperator{\Jac}{Jac}

%Misc
\newcommand{\SL}{{\mathrm{SL}}}
\newcommand{\mobgp}{{\mathrm{PSL}_2(\mathbb{C})}}
\newcommand{\id}{{\mathrm{id}}}
\newcommand{\Mod}{{\mathrm{Mod}}}
\newcommand{\PMod}{{\mathrm{PMod}}}
\newcommand{\SMod}{{\mathrm{SMod}}}
\newcommand{\ud}{{\mathrm{d}}}
\newcommand{\Vol}{{\mathrm{Vol}}}
\newcommand{\Area}{{\mathrm{Area}}}
\newcommand{\diam}{{\mathrm{diam}}}
\newcommand{\End}{{\mathrm{End}}}


\newcommand{\reg}{{\mathtt{reg}}}
\newcommand{\geo}{{\mathtt{geo}}}

\newcommand{\tori}{{\mathcal{T}}}
\newcommand{\cpn}{{\mathtt{c}}}
\newcommand{\pat}{{\mathtt{p}}}

\let\Cap\undefined
\newcommand{\Cap}{{\mathcal{C}}ap}
\newcommand{\Push}{{\mathcal{P}}ush}
\newcommand{\Forget}{{\mathcal{F}}orget}




\begin{document}
\section{Rotman}

\subsection{Modules}

\begin{definition}[Representations]
    If $M$ is an abelian group, then
    \[
    \End_{\mathbb{Z}} (M) = \left\{ \text{homomorphisms}
     f\colon M \to M\right\} 
    \] 
    is a ring under pointwise addition and composition as
    multiplication.
    A representation of a ring $R$ is a ring homomorphism
    $\varphi  \colon R \to \End_{\mathbb{Z}}(M)$ for some abelian
    group $M$.
\end{definition}

\begin{definition}[Group ring]
    Let $G$ be a finite group and $k$ be a commutative ring.
    The group ring is the set of all
    functions $\alpha \colon G \to k$ made into a ring
    with pointwise operations: for all $x \in G$,
    \[
        \left( \alpha+\beta \right) (x)
        = \alpha(x) + \beta(x) \quad \text{and} \quad
        \left( \alpha \beta \right) (x)
        = \alpha(x) \beta(x).
    \] 
\end{definition}

\begin{definition}[$k$-representation]
    If $G$ is a group and $k$ is a commutative ring, then
    a $k$-representation of $G$ is a function
    $\sigma \colon G \to \Mat_n (k)$ with
    \begin{align*}
        \sigma (xy) 
        &= \sigma (x) \sigma (y)\\
        \sigma(1)
        &= I
    \end{align*}.
\end{definition}

\begin{lemma}[$\Hom_R (A,B)$ is an abelian group]
    For left (resp. right) $R$-modules,
    $\Hom_R (A,B)$ is an abelian group and
    if $p \colon A' \to A$ and $q \colon B \to B'$ are
    $R$-maps, then
    \[
        \left( f+g \right) p = fp + gp \quad
        \text{and} \quad
        q \left( f+g \right) = qf + qg.
    \] 
\end{lemma}

\begin{proposition}[]
    Let $R$ be a ring, $A,B,B'$ be left $R$-modules. Then
    \begin{enumerate}
        \item $\Hom_R (A,-)$ is an additive functor
             $_R\Mod \to \Ab$
             sending $B \to \Hom_R (A,B)$ and
             a morphism  $q \colon B \to B'$ to
             $q_* \colon \Hom_R(A,B) \to \Hom_R(A,B')$ by postcomposition.
         \item If $A$ is a left $R$-module, then
             $\Hom_R(A,B)$ is a $Z(R)$-module, where
             $Z(R)$ is the center of $R$, if
             we define
             \[
                 \left( rf \right) (a) = f(ra) = r f(a)
             \] 
             for all $r \in Z(R)$ and
             $f \colon A \to B$. Then
             $\Hom_R(A,-)$ is a functor\\
             $_R\Mod \to _{Z(R)}{\Mod}$
    \end{enumerate}
\end{proposition}

\begin{proof}
    (1) Since
    $q \left( f+g \right) = qf+qg$, we have
    $q_*\left( f+g \right) = q_* (f) + q_*(g)$, so 
    $\Hom_R \left( A,- \right) (q) = q_* \in 
    \Mor \left( \Hom_R (A,B) , \Hom_R (A,B') \right) $ in
    $\Ab$.

    Furthermore, for  $q \colon A \to B$ and
    $p \colon B \to C$, we have
    \[
        \left( pq \right)_* (a) =
        pqa = p_* \left( qa \right) = 
        (p_* q_*) (a)
    \] 
    so composition is preserved. And for
    any $a \colon A \to B$, we have
    \[
        \left( \mathbbm{1}_B \right)_* (a)
        = \mathbbm{1}_B \circ a = a
    \] 
    so $\left( \mathbbm{1}_B \right)_* = 
    \mathbbm{1}_{\Hom_R (A,B)}$.
\end{proof}


\begin{exercise}[Example of a quotient group which is
    not a quotient module]
    We have that
    $\mathbb{Q}$ is a module over
    itself and $\mathbb{Q} / \mathbb{Z}$ is a quotient group, but
    since $\mathbb{Z}$ is not a submodule of $\mathbb{Q}$ 
    - it is not closed under scalar multiplication
    from $\mathbb{Q}$ -, we
    are not guaranteed that $\mathbb{Q} / \mathbb{Z}$ is
    a quotient module. And in fact, it is not:
    $2 \left( \frac{1}{2}+ \mathbb{Z} \right) = \mathbb{Z}$ 
    in $\mathbb{Q} / \mathbb{Z}$ but neither
    factor is zero, but $\mathbb{Q}$ is a field,
    so if  $\mathbb{Q} / \mathbb{Z}$ were
    a quotient module (over $\mathbb{Q}$ ), it would
    have to be a vector space, but
    in a vector space, we have
    $av = 0$ iff $a = 0$ or $v = 0$.
\end{exercise}


\subsection{Isomorphism theorems}

\begin{theorem}[First isomorphism theorem]
    If $f \colon M \to N$ is an $R$-map of left
    $R$-modules, then there is an $R$-isomorphism
    \[
    \varphi  \colon M / \ker f \to \im f
    \] 
    given by
    \[
    \varphi  \colon m + \ker f \mapsto f(m).
    \] 
\end{theorem}

\begin{theorem}[Second isomorphism]
    If $S$ and $T$ are submodules of a left $R$-module
    $M$, then there is an $R$-isomorphism
    \[
    S / \left( S \cap T \right) \to 
    \left( S+T \right) /T
    \] 
\end{theorem}

\begin{theorem}[Third isomorphism theorem]
    If $T \subset S \subset M$ is a tower
    of submodules of a left  $R$-module $M$, then the
    enlargement of cosets 
    $e \colon M /T \to M /S$ induces an $R$-isomorphism
    \[
        \left( M /T \right) / \left( S/T  \right) 
        \to M /S
    \] 
\end{theorem}

\begin{theorem}[Fourth (Correspondence) isomorphism theorem]
    If $T$ is a submodule of a left
    $R$-module $M$, then $\varphi  \colon S \to  S/T $
    is a bijection:
    \[
    \varphi  \colon
    \left\{ \text{intermediate submodules }
    T \subset S \subset M \right\} 
    \to \left\{ \text{submodules of }
    M/T \right\} .
    \] 
    Moreover, $T \subset S \subset S'$ in
    $M$ if and only if $S /T \subset S' / T$ in
    $M /T$.
\end{theorem}

\begin{definition}[Simple/irreducible modules]
    A left $R$-module $M$ is simple
    (or irreducible) if $M \neq \left\{ 0 \right\} $ and
    $M$ has no proper nonzero submodules; i.e.,
    $\left\{ 0 \right\} $ and
    $M$ are the only submodules of $M$.
\end{definition}


\begin{lemma}[]
    A left $R$-module $M$ is simple if and only if
    $M \approx R /I$, where $I$ is a maximal left ideal.
\end{lemma}

\begin{theorem}[1st and 3rd isomorphism theorem rephrased]
    \begin{enumerate}
        \item If $0 \to A \stackrel{f}{\to }
            B \stackrel{g}{\to } C \to 0$ is a short
            exact sequence, then
            \[
            A \approx \im f \quad \text{and } \quad
            B / \im f \approx C.
            \] 
        \item If $T \subset S \subset M$ is a tower
            of submodules, then there is an
            exact sequence
            \[
            0 \to S /T \to M /S \to M /T \to 0.
            \] 
    \end{enumerate}
\end{theorem}
    

\begin{lemma}[]
    Suppose $M$ is an $R$-module. Then
    \[
        M = \bigcup_{\substack{M' \subset M\\ M' \text{ fin. gen.}}} 
        M'.
    \] 
    We claim that
    \begin{enumerate}
        \item $M \otimes_R A = 
            \bigcup_{\substack{M' \subset M \\
            M' \text{ fin. gen.}}} M' \otimes_R A $.
        \item If $\id_{M'} \otimes f\colon 
            M' \otimes_R A \to M' \otimes_R B$ is injective
            for all finitely generated
            $M' \subset M$, then so is
            \[
            \id_M \otimes f \colon
\bigcup_{\substack{M' \subset M \\
            M' \text{ fin. gen.}}} M' \otimes_R A
            \to \bigcup_{\substack{M' \subset M \\
            M' \text{ fin. gen.}}} M' \otimes_R B
            \] 
    \end{enumerate}
\end{lemma}

\begin{proof}
    \begin{enumerate}
        \item Define a diagram $F \colon I \to \, _{R}\Mod$
            which has maps to all 
            finitely generated submodules of
            $M$ and all
            inclusions between them.
            Then by the universal property, the
            $\colim F = \bigcup_{\substack{M' \subset M \\
            M' \text{ fin. gen.}}} M' = M 
            $ with maps
            $i_{M'} \colon M' \to \colim F$ satisfying the
            commutativity of the inclusions.
            Since $- \otimes A$ is a left adjoint, we have
            \[
                M \otimes_R A = 
                L \left( \colim F \right) 
                = \colim \left( L \circ F \right) 
                =  \bigcup_{\substack{M' \subset M \\
            M' \text{ fin. gen.}}} M' \otimes_R A
            \] 
           the last equality again following from the
           universal property.
       \item 
    \end{enumerate}
\end{proof}




    %\bibliography{../refs.bib}
\end{document}
