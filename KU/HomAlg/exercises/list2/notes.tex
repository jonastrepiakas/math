\documentclass[reqno]{amsart}
\usepackage{amscd, amssymb, amsmath, amsthm}
\usepackage{graphicx}
\usepackage[colorlinks=true,linkcolor=blue]{hyperref}
\usepackage[utf8]{inputenc}
\usepackage[T1]{fontenc}
\usepackage{textcomp}
\usepackage{babel}
%% for identity function 1:
\usepackage{bbm}
%%For category theory diagrams:
\usepackage{tikz-cd}


\setlength\parindent{0pt}

\pdfsuppresswarningpagegroup=1

\newtheorem{theorem}{Theorem}[section]
\newtheorem{lemma}[theorem]{Lemma}
\newtheorem{proposition}[theorem]{Proposition}
\newtheorem{corollary}[theorem]{Corollary}
\newtheorem{conjecture}[theorem]{Conjecture}

\theoremstyle{definition}
\newtheorem{definition}[theorem]{Definition}
\newtheorem{example}[theorem]{Example}
\newtheorem{exercise}[theorem]{Exercise}
\newtheorem{problem}[theorem]{Problem}
\newtheorem{question}[theorem]{Question}

\theoremstyle{remark}
\newtheorem*{remark}{Remark}
\newtheorem*{note}{Note}
\newtheorem*{solution}{Solution}



%Inequalities
\newcommand{\cycsum}{\sum_{\mathrm{cyc}}}
\newcommand{\symsum}{\sum_{\mathrm{sym}}}
\newcommand{\cycprod}{\prod_{\mathrm{cyc}}}
\newcommand{\symprod}{\prod_{\mathrm{sym}}}

%Linear Algebra

\DeclareMathOperator{\Span}{span}
\DeclareMathOperator{\im}{im}
\DeclareMathOperator{\diag}{diag}
\DeclareMathOperator{\Ker}{Ker}
\DeclareMathOperator{\ob}{ob}
\DeclareMathOperator{\Hom}{Hom}
\DeclareMathOperator{\Mor}{Mor}
\DeclareMathOperator{\sk}{sk}
\DeclareMathOperator{\Vect}{Vect}
\DeclareMathOperator{\Set}{Set}
\DeclareMathOperator{\Group}{Group}
\DeclareMathOperator{\Ring}{Ring}
\DeclareMathOperator{\Ab}{Ab}
\DeclareMathOperator{\Top}{Top}
\DeclareMathOperator{\hTop}{hTop}
\DeclareMathOperator{\Htpy}{Htpy}
\DeclareMathOperator{\Cat}{Cat}
\DeclareMathOperator{\CAT}{CAT}
\DeclareMathOperator{\Cone}{Cone}
\DeclareMathOperator{\dom}{dom}
\DeclareMathOperator{\cod}{cod}
\DeclareMathOperator{\Aut}{Aut}
\DeclareMathOperator{\Mat}{Mat}
\DeclareMathOperator{\Fin}{Fin}
\DeclareMathOperator{\rel}{rel}
\DeclareMathOperator{\Int}{Int}
\DeclareMathOperator{\sgn}{sgn}
\DeclareMathOperator{\Homeo}{Homeo}
\DeclareMathOperator{\SHomeo}{SHomeo}
\DeclareMathOperator{\PSL}{PSL}
\DeclareMathOperator{\Bil}{Bil}
\DeclareMathOperator{\Sym}{Sym}
\DeclareMathOperator{\Skew}{Skew}
\DeclareMathOperator{\Alt}{Alt}
\DeclareMathOperator{\Quad}{Quad}
\DeclareMathOperator{\Sin}{Sin}
\DeclareMathOperator{\Supp}{Supp}
\DeclareMathOperator{\Char}{char}


%Row operations
\newcommand{\elem}[1]{% elementary operations
\xrightarrow{\substack{#1}}%
}

\newcommand{\lelem}[1]{% elementary operations (left alignment)
\xrightarrow{\begin{subarray}{l}#1\end{subarray}}%
}

%SS
\DeclareMathOperator{\supp}{supp}
\DeclareMathOperator{\Var}{Var}

%NT
\DeclareMathOperator{\ord}{ord}

%Alg
\DeclareMathOperator{\Rad}{Rad}
\DeclareMathOperator{\Jac}{Jac}

%Misc
\newcommand{\SL}{{\mathrm{SL}}}
\newcommand{\mobgp}{{\mathrm{PSL}_2(\mathbb{C})}}
\newcommand{\id}{{\mathrm{id}}}
\newcommand{\Mod}{{\mathrm{Mod}}}
\newcommand{\PMod}{{\mathrm{PMod}}}
\newcommand{\SMod}{{\mathrm{SMod}}}
\newcommand{\ud}{{\mathrm{d}}}
\newcommand{\Vol}{{\mathrm{Vol}}}
\newcommand{\Area}{{\mathrm{Area}}}
\newcommand{\diam}{{\mathrm{diam}}}
\newcommand{\End}{{\mathrm{End}}}


\newcommand{\reg}{{\mathtt{reg}}}
\newcommand{\geo}{{\mathtt{geo}}}

\newcommand{\tori}{{\mathcal{T}}}
\newcommand{\cpn}{{\mathtt{c}}}
\newcommand{\pat}{{\mathtt{p}}}

\let\Cap\undefined
\newcommand{\Cap}{{\mathcal{C}}ap}
\newcommand{\Push}{{\mathcal{P}}ush}
\newcommand{\Forget}{{\mathcal{F}}orget}




\begin{document}
    \begin{exercise}[18]
        Show that every free module is projective.
    \end{exercise}

    \begin{proof}
        Suppose $F$ is a free $R$-module on a set $X$.
        Let $f \colon F \to M$ be a map
        and $q \colon N \to M$ a surjective map. We have
        the diagram
        \begin{equation*}
        \begin{tikzcd}
            & P \ar[d, "f"] \ar[dl, "g"] & X \ar[l] \ar[ld, dashed, " 
            \exists ! \tilde{f}"] \\
            N \ar[r, "q"]  & M & 
        \end{tikzcd}
        \end{equation*}
        We define a map
        $X \to N$ by sending $x \in X$ to
        some $n \in N$ such that
        $q(n) = \tilde{f}(x)$ - this exists by
        surjectivity of $q$. Then this
        extends to a unique map $g \colon P \to N$ such that
        $q \circ g \circ \iota (x) = \tilde{f}(x)$ for all $x \in X$.
        But this extends to a unique map
        $\tilde{g} \colon P \to M$ such that
        $\tilde{g} \circ \iota = q \circ g \circ \iota
        = \tilde{f}$. Since
        $q \circ g$ is the unique map for
        the first equality, we obtain
        $\tilde{g} = q \circ g$. But for
        $\tilde{g} \circ \iota = \tilde{f}$, we obtain again
        by uniqueness that $\tilde{g} = f$. Hence
        $f = q \circ g$.
        
    \end{proof}

    \begin{exercise}[19]
        Suppose $R$ is a PID and let $F$ be a projective
        $R$-module. Show that $F$ is free.
    \end{exercise}

    \begin{proof}
        Since $F$ is projective, $F$ is a direct
        summand in a free $R$-module, hence in particular,
        it is finitely generated. Hence
        $\bigoplus_{i \in I} R \approx F \oplus K$ and
        by the structure theorem, also
        $F \approx R^{n} \oplus \bigoplus_{i=1}^{r} 
        \bigoplus_{j=1}^{s_i} R /\left( p_i^{m_{ij}} \right) $.
        But $\bigoplus_{i\in I} R$ has no torsion, so
        $F$ must also be torsion-free, hence
        $F \approx R^{n}$, i.e, $F$ is free.
    \end{proof}

    \begin{exercise}[20]
        Let $M$ be an $R$-module. We say that
        $M$ is torsion-free if whenver
        $rm = 0$ for $r \in R$ and $m \in M$, then
        either $r = 0$ or $m=0$.
        \begin{enumerate}
            \item Show that $R$ is torsion-free
                when considered as a module over itself
                if and only if $R$ is an integral domain.
            \item Let $R$ be an integral domain.
                Show that any projective $R$-module is
                torsion-free.
        \end{enumerate}

    \end{exercise}

    \begin{proof}
        \begin{enumerate}
            \item Saying that $R$ is torsion-free as a 
                module over itself is the same as saying
                $r r' = 0$ for $r \in R$ and $r' \in R$ 
                if and only if $r = 0$ or $r' =0$ which
                is precisely the same as saying that
                $R$ is an integral domain.

            \item If $P$ is a projective $R$-module,
                then it is a direct summand in a free
                $R$-module. But any free $R$-module is
                torsion-free, so $P$ must be as well.

            \item $\mathbb{Q}$ as a $\mathbb{Z}$-module
                is torsion-free, however
                $\mathbb{Q}$ is not projective over
                $\mathbb{Z}$ since
                the sequence
                $0 \to \mathbb{Z} \to \mathbb{Q} \to 
                \mathbb{Q} / \mathbb{Z} \to 0$ is exact, but
                \[0 \to \underbrace{\Hom_{\mathbb{Z}}\left(
                        \mathbb{Q},
                \mathbb{Z}\right)}_{=0} \to 
                \underbrace{\Hom_{\mathbb{Z}}
                \left( \mathbb{Q},\mathbb{Q} \right)}_{\approx 
            \mathbb{Q}}
                \to \underbrace{\Hom_{\mathbb{Z}} 
                \left( \mathbb{Q}, \mathbb{Q} / \mathbb{Z} \right)}_{
            = 0}
            \to  0\]
            is not exact.





        \end{enumerate}
    \end{proof}

    \begin{exercise}[21]
        Let $P$ be a projective $R$-module. Prove
        that there exists a free $R$-module $F$ such that
        $P \oplus F$ is free.
    \end{exercise}

    \begin{proof}
        $P$ is a quotient of a 
        free $R$-module, $T$, say
        $P = T /A$. Also, $P$ is a direct summand of
        a free $R$-module $\overline{F}$, say
        $\overline{F} = P \oplus S$.
        Then
        \[
        P \oplus \left( \oplus_{i \in \mathbb{N} }
        \overline{F}_i \right)
        \approx
        P \oplus \left( \oplus_{i \in \mathbb{N} }
        S_i \oplus P_i\right) 
        \approx
        \oplus_{i \in \mathbb{N} }
        P \oplus S
        \approx \oplus_{i \in \mathbb{N} }\overline{F}.
        \] 
    \end{proof}

















    %\bibliography{../refs.bib}
\end{document}
