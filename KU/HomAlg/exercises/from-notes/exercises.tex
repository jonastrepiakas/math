\documentclass[reqno]{amsart}
\usepackage{amscd, amssymb, amsmath, amsthm}
\usepackage{graphicx}
\usepackage[colorlinks=true,linkcolor=blue]{hyperref}
\usepackage[utf8]{inputenc}
\usepackage[T1]{fontenc}
\usepackage{textcomp}
\usepackage{babel}
%% for identity function 1:
\usepackage{bbm}
%%For category theory diagrams:
\usepackage{tikz-cd}


\setlength\parindent{0pt}

\pdfsuppresswarningpagegroup=1

\newtheorem{theorem}{Theorem}[section]
\newtheorem{lemma}[theorem]{Lemma}
\newtheorem{proposition}[theorem]{Proposition}
\newtheorem{corollary}[theorem]{Corollary}
\newtheorem{conjecture}[theorem]{Conjecture}

\theoremstyle{definition}
\newtheorem{definition}[theorem]{Definition}
\newtheorem{example}[theorem]{Example}
\newtheorem{exercise}[theorem]{Exercise}
\newtheorem{problem}[theorem]{Problem}
\newtheorem{question}[theorem]{Question}

\theoremstyle{remark}
\newtheorem*{remark}{Remark}
\newtheorem*{note}{Note}
\newtheorem*{solution}{Solution}



%Inequalities
\newcommand{\cycsum}{\sum_{\mathrm{cyc}}}
\newcommand{\symsum}{\sum_{\mathrm{sym}}}
\newcommand{\cycprod}{\prod_{\mathrm{cyc}}}
\newcommand{\symprod}{\prod_{\mathrm{sym}}}

%Linear Algebra

\DeclareMathOperator{\Span}{span}
\DeclareMathOperator{\im}{im}
\DeclareMathOperator{\diag}{diag}
\DeclareMathOperator{\ob}{ob}
\DeclareMathOperator{\coker}{coker}
\DeclareMathOperator{\Hom}{Hom}
\DeclareMathOperator{\Mor}{Mor}
\DeclareMathOperator{\sk}{sk}
\DeclareMathOperator{\Vect}{Vect}
\DeclareMathOperator{\Set}{Set}
\DeclareMathOperator{\Group}{Group}
\DeclareMathOperator{\Ring}{Ring}
\DeclareMathOperator{\Ab}{Ab}
\DeclareMathOperator{\Top}{Top}
\DeclareMathOperator{\hTop}{hTop}
\DeclareMathOperator{\Htpy}{Htpy}
\DeclareMathOperator{\Cat}{Cat}
\DeclareMathOperator{\CAT}{CAT}
\DeclareMathOperator{\Cone}{Cone}
\DeclareMathOperator{\dom}{dom}
\DeclareMathOperator{\cod}{cod}
\DeclareMathOperator{\Aut}{Aut}
\DeclareMathOperator{\Mat}{Mat}
\DeclareMathOperator{\Fin}{Fin}
\DeclareMathOperator{\rel}{rel}
\DeclareMathOperator{\Int}{Int}
\DeclareMathOperator{\sgn}{sgn}
\DeclareMathOperator{\Homeo}{Homeo}
\DeclareMathOperator{\SHomeo}{SHomeo}
\DeclareMathOperator{\PSL}{PSL}
\DeclareMathOperator{\Bil}{Bil}
\DeclareMathOperator{\Sym}{Sym}
\DeclareMathOperator{\Skew}{Skew}
\DeclareMathOperator{\Alt}{Alt}
\DeclareMathOperator{\Quad}{Quad}
\DeclareMathOperator{\Sin}{Sin}
\DeclareMathOperator{\Supp}{Supp}
\DeclareMathOperator{\Char}{char}
\DeclareMathOperator{\Ext}{Ext}


%Row operations
\newcommand{\elem}[1]{% elementary operations
\xrightarrow{\substack{#1}}%
}

\newcommand{\lelem}[1]{% elementary operations (left alignment)
\xrightarrow{\begin{subarray}{l}#1\end{subarray}}%
}

%SS
\DeclareMathOperator{\supp}{supp}
\DeclareMathOperator{\Var}{Var}

%NT
\DeclareMathOperator{\ord}{ord}

%Alg
\DeclareMathOperator{\Rad}{Rad}
\DeclareMathOperator{\Jac}{Jac}

%Misc
\newcommand{\SL}{{\mathrm{SL}}}
\newcommand{\mobgp}{{\mathrm{PSL}_2(\mathbb{C})}}
\newcommand{\id}{{\mathrm{id}}}
\newcommand{\Mod}{{\mathrm{Mod}}}
\newcommand{\PMod}{{\mathrm{PMod}}}
\newcommand{\SMod}{{\mathrm{SMod}}}
\newcommand{\ud}{{\mathrm{d}}}
\newcommand{\Vol}{{\mathrm{Vol}}}
\newcommand{\Area}{{\mathrm{Area}}}
\newcommand{\diam}{{\mathrm{diam}}}
\newcommand{\End}{{\mathrm{End}}}


\newcommand{\reg}{{\mathtt{reg}}}
\newcommand{\geo}{{\mathtt{geo}}}

\newcommand{\tori}{{\mathcal{T}}}
\newcommand{\cpn}{{\mathtt{c}}}
\newcommand{\pat}{{\mathtt{p}}}

\let\Cap\undefined
\newcommand{\Cap}{{\mathcal{C}}ap}
\newcommand{\Push}{{\mathcal{P}}ush}
\newcommand{\Forget}{{\mathcal{F}}orget}




\begin{document}


\section{Modules}
    \begin{exercise}[]
        Show that an $R$-module homomorphism
        $f \colon M \to N$ is an isomorphism if and
        only if it is injective and surjective.
    \end{exercise}

    \begin{proof}
        Suppose $f$ is an isomorphism, so
        there exists a homomorphism
        $f^{-1} \in \Hom_{R}(N,M)$ such that
        $f \circ f^{-1} = \mathbbm{1}_{N}$ and
        $f^{-1} \circ f = \mathbbm{1}_{M}$.
        Suppose $f(v) = f(w)$. Then
        $v = f^{-1} \circ f(v) = 
        f^{-1} \circ f(w) = w$, so $f$ is injective.
        Now for $ n\in N$, we have
        $n = f \left( f^{-1}(n) \right) $, so
        $f$ is surjective.

        Conversely, if $f$ is  injective, it has
        a left inverse as a function, and if it surjective, it
        has a right inverse, and these are unique and equal (by
        the same general property for functions). 
        Denote this function by $f^{-1} \colon
        N \to M$. We claim this is an
        $R$-module homomorphism. Indeed, as
        $f$ is a group homomorphism, we know the
        inverse is as well. 
        Now, since 
        $f \circ f^{-1}(rn) = rn$ and
        $rn = r f \circ f^{-1}(n) =
        f \left( r f^{-1}(n) \right) $, we have
        by injectivity of $f$ that
        $f^{-1}(rn) = r f^{-1}(n)$, so
        indeed $f^{-1}$ is an $R$-module homomorphism.
    \end{proof}

    \begin{exercise}[]
        $\Hom_{R} (M,N)$ has the structure of an abelian group
        with
        \begin{align*}
            \left( f+g \right) (x) 
            &:= f(x) + g(x)\\
            (-f) (x) 
            &:= - f(x)
        \end{align*}
    \end{exercise}

    \begin{proof}
        We must check associativity, identity, inverse and
        commutativity when
        the set  $\Hom_R (M,N)$ is equipped with the
        binary operator and inverse defined above.


        Associativity is inherited from associativity of
        functions. 
        Now, let
        $0_N$ denote the identity of
        the abelian group $N$. Define a map
        $0$ by
        $0 (m)=
         0_N$ for all $m \in M$.
        Then
        for any $f \in \Hom_R(M,N)$, we have
        \[
            (f+ 0)(x)
            = f(x) + 0(x)
            = f(x) + 0_N
            = f(x)
        \] 
        and similarly, $0 + f = f$. Thus
        $0$ is an identity for
        $\Hom_R (M,N)$. 
        Now
        \[
            \left( f + (-f) \right) (x)
            = f(x) + (-f)(x)
            = f(x) - f(x)
            = 0_N
        \] 
        so $f+ (-f) = 0$, and similarly,
        $(-f) + f = 0$.

        Lastly, commutativity follows from commutativity
        of $N$.
    \end{proof}

    \begin{exercise}[]
        For a homomorphism $f \in \Hom_R(M,N)$, the kernel
        $\ker (f) = \left\{ x  \mid 
        f(x) = 0 \right\} $ and the
        image $\im (f)$ are submodules.
    \end{exercise}

    \begin{proof}
        For $x,y \in \ker f$, we have
        $f(x-y) = f(x) - f(y) = \mathbbm{1}_R$ by
        $f$ being a group homomorphism, and
        $f(rx) = r f(x) = r \cdot \mathbbm{1}_R = \mathbbm{1}_R$
        by the definition of an $R$-module and an
        $R$-linear map, hence
        $\ker f$ is also closed under multiplication by
        elements of $R$. The inclusion
        $\ker f \to M$ is $R$-linear since
        if $x,y \in \ker f$, then
        $\iota (rx+y) = rx+y = r \iota(x) + \iota (y)$.

        Likewise, if $x,y \in \im f$, then let
        $u,v \in M$ such that $f(u) =x$ and
        $f(v) = y$. Then
        $f(u-v) = f(u) - f(v) = x-y \in \im f$ and
        $f(ru) = r f(u) = rx \in \im f$ for all $r \in R$ and
        for all $x,y \in \im f$.
        Furthermore,
        for $x,y \in \im f$,
        $\iota (rx+y) = rx+y = r \iota (x) + \iota (y)$
        so the inclusion $\im f \to N$ is $R$-linear.
    \end{proof}

    
    \begin{exercise}[]
        $R /I$ is a ring if $I$ is a two-sided ideal of
        $R$.
    \end{exercise}

    \begin{proof}
        If $I$ is a two-sided ideal, then firstly,
        $R /I$ is an abelian group under $+$ since
        everything is abelian hence normal. 
        All other requirements for a ring are inherited from
        $R$. We must only check that multiplication is well-defined.
        If $r + I = r'+I$ and $s + I = s' + I$ then
        $s^{-1} \underbrace{r^{-1} r'}_{\in I} s'
        \in I$, so indeed
        $rs+ I = r's' + I$.
    \end{proof}


    \begin{exercise}[]
        Show that a cokernel exists and is unique up
        to isomorphism.
    \end{exercise}

    \begin{proof}
        For an $R$-linear map $f \colon M \to N$, define
        $\coker f := N / \im f$. 
        Since $\im f$ is a submodule of $N$,
        $N / \im f$ is an $R$-module. Furthermore, it
        satisfies the diagram
        \begin{equation*}
        \begin{tikzcd}
            M \ar[r, "f"] \ar[dr,"0"] & N \ar[r, "q"] 
            \ar[d, "g"] & \coker f \ar[dl, "\exists ! \overline{g}"]\\
                          & L &
        \end{tikzcd}
        \end{equation*}
        since $\overline{g}$ must be defined by
        $\overline{g}(q(n)) = g(n)$.
        We must check that this is well-defined.

        Suppose $\overline{a} = \overline{b} \in 
        \coker f$, so $a-b \in \im f$, so
        $f(m) = a-b$. Then
        $0 = g f (m) = \overline{g} (q(a)- q(b))
        = \overline{g}\left( \overline{a}-
        \overline{b} \right) 
        = \overline{g}(a) - \overline{g}(b)$.

        It is also $R$-linear because
        $\overline{g}\left( \overline{x}+ \overline{y} \right) 
        = \overline{g}\left( \overline{x+y} \right) 
        = g(x+y) = g(x) + g(y) = 
        \overline{g}\left( \overline{x} \right) +
        \overline{g}\left( \overline{y} \right) $, and
        $\overline{g}\left( r \overline{x} \right) 
        = \overline{g}\left( \overline{rx} \right) 
        = g\left( rx \right) 
        = r g(x) = r \overline{g}(\overline{x})$.

        To check uniqueness, suppose
        $(K,q')$ also satisfies the above diagram. Then
        letting $g = q$ and $L = \coker f$, we get
        \begin{equation*}
        \begin{tikzcd}
            M \ar[r, "f"] \ar[dr, "0"] & N \ar[r, "q'"]
            \ar[d, "q"]& K \ar[dl, 
            "\exists ! \overline{q}"] \\
                          & \coker f &
        \end{tikzcd}
        \end{equation*}

        \begin{equation*}
        \begin{tikzcd}
            M \ar[r, "f"] \ar[dr, "0"] & N \ar[r, "q"]
            \ar[d, "q'"]& \coker f \ar[dl, 
            "\exists ! \overline{q'}"] \\
                          & K &
        \end{tikzcd}
        \end{equation*}


        Interchanging $\coker f$ and 
        $K$, we also get a unique map
        $\overline{q'} \colon \coker f \to K$. These
        have the property that
        $ \overline{q'} \circ \overline{q} \circ q'
        = \overline{q'} \circ q = q'$ so
        by uniqueness of $\mathbbm{1}_K$ in the diagram
        
        \begin{equation*}
        \begin{tikzcd}
            M \ar[r, "f"] \ar[dr, "0"] & N \ar[r, "q'"]
            \ar[d, "q'"]& K \ar[dl, 
            "\mathbbm{1}_K"] \\
                          & K &
        \end{tikzcd}
        \end{equation*}
        we get 
        $\overline{q'} \circ \overline{q} = \mathbbm{1}_K$, and
        likewise, $\overline{q} \circ \overline{q'} = 
        \mathbbm{1}_{\coker f}$. So
        $\overline{q} \colon \coker f \to K$ is
        an isomorphism.

        
    \end{proof}


\subsection{Direct sum and direct product}

    \begin{exercise}[]
        Prove uniqueness of the direct product.
    \end{exercise}

    \begin{proof}
        Suppose
        $A$ and $B$ are both direct products
        with maps $\pi_{A,j} \colon A \to M_j$ and
        $\pi_{B,j} \colon B \to M_j$ for all $j$ such
        that the universal diagram is fulfilled.
        Then, since we have
        maps $\pi_{B,j} \colon B \to M_j$ for all $j$, we
        have a unique map
        $u \colon B \to A$ such that
        $\pi_{A,j} \circ u = \pi_{B,j}$ for all $j$. And
        similarly, we have a map
        $v \colon A \to B$ such that
        $\pi_{B,j} \circ v = \pi_{A,j}$ for all $j$.
        But then
        $\pi_{A,j} \circ u \circ v = \pi_{A,j}$ for all
        $j$, so since $\pi_{A,j} \circ \mathbbm{1}_{A}
        = \pi_{A,j}$ for all $j$, we get by uniqueness
        that $u \circ v = \mathbbm{1}_A$, and similarly,
        interchanging  $A$ for $B$ above, we get
        $v \circ u = \mathbbm{1}_B$. Hence
        $u \colon B \to A$ is an isomorphism.


        


    \end{proof}



    \begin{remark}[Direct product]
        Note that the direct product
        of a family $\left( M_{i} \right)_{ i \in I}$
        is simply the universal cone over the
        diagram $I \to \Mod_R$ given by
        sending $i \mapsto M_i$ where $I$ is 
        a discrete category.
    \end{remark}

    \begin{remark}[Direct sum]
        The direct sum is the dual of the direct
        product.\\
        Given a diagram
        $I \to \Mod_R$ where $I$ is discrete, we
        define the direct sum as the cone
        under $I \to \Mod_R$ and let its nadir be
        denoted $\bigoplus_{i \in I} M_i$ together with
        $R$-linear maps 
        $\iota_j \colon M_j \to \bigoplus_{i \in I}M_i$.

    \end{remark}


    \begin{remark}[]
        For each $i \in I$, there is a unique map
        $f_j \colon M_j \to \prod_{j \in I} M_j$ with
        $\pi_i f_j = 0$ if $i \neq j$ and
        $\pi_j f_j = \mathbbm{1}_{M_j}$.
        By the universal property, we thus get a unique
        map
        $u \colon \bigoplus_{i \in I} M_i \to \prod_{i \in I}M_i$ 
        given by  $u \left( x \right) 
        = \sum_{i \in I} f_i\left( x(i) \right) $.
        This map is an isomorphism when $I$ is finite, but
        not necessarily when $I$ is infinite.
    \end{remark}

    \begin{corollary}
        Summarizing, we have bijections


        \begin{align*}
            \Hom_R \left( N, \prod_{i \in I} M_i \right) 
            &\stackrel{\approx}{\to} \prod_{i \in I} \Hom_R (N,M_i)\\
            u 
            &\mapsto \left( \pi_i u \right)_{i \in I}
        \end{align*}
        where we send
        $\left( \pi_i u \right)_{i \in I} (j)
        = \pi_j u \colon N \stackrel{u}{\to} \prod_{i \in I}M_i 
        \stackrel{\pi_j}{\to} 
        M_j$
        and
         \begin{align*}
            \Hom_R\left( \bigoplus_{i \in I}M_i,
            N\right) 
            &\stackrel{\approx}{\to }
            \prod_{i \in I}\Hom_R \left( M_i, N \right) \\
            u
            &\mapsto \left( u \iota_i \right)_{i \in I}
        \end{align*}
        which are in fact isomorphisms of abelian groups.
    \end{corollary}

    \begin{exercise}[]
        Show that the above bijections are isomorphisms of
        abelian groups.
    \end{exercise}

    \begin{proof}
        Since they are bijections between abelian groups, we
        must simply show that they are homomorphisms.

        Now,
        \[
        u+v \mapsto \left( \pi_i \left( u+v \right)  \right)_{i \in I}
        = \left( \pi_i u + \pi_i v \right)_{i \in I}
        \] 
        and
        $\left( \pi_i u + \pi_i v \right)(j)
        = \left( \pi_iu \right) (j) + 
        \left( \pi_i v \right) (j)$
        so $u+v \mapsto \left( \pi_i u \right) +
        \left( \pi_i v \right) $ by the definition
        of addition in the direct product, hence this
        is indeed a homomorphism.

        For the direct sum, we similarly have
        \[
        u+v \mapsto \left( \left( u+v \right) \iota_i \right)_{i \in I}
        = \left( u \iota_i + v \iota_i \right)_{i \in I}
        \] 
        and
        $\left( u \iota_i + v \iota_i \right)_{i \in I}(j)
        = \left( u \iota_i \right)_{i \in I} (j)
        + \left( v \iota_i \right)_{i \in I}(j)$, so
        $u+v \mapsto \left( u \iota_i \right)_{i \in I}
        + \left( v \iota_i \right)_{i \in I}$
        which is indeed a homomorphism by
        the additive structure on the direct product which
        the direct sum inherits.
        
    \end{proof}
        
        \begin{exercise}[]
            Show that cyclic left $R$-modules (of $R$?)
            are precisely those
            of the form $R /I$ for some left ideal
            $I \subset R$.
        \end{exercise}

        \begin{proof}
            Suppose $M$ is a cyclic left $R$-module.
            Then $M = R x$ for some $x \in M$.

        \end{proof}


        \subsection{Generation and free modules}

        \begin{exercise}[]
            Show that an $R$-module is free if and only if
            it has a basis.
        \end{exercise}

        \begin{proof}
            Suppose $M$ is a free $R$-module.
            Let $X$ be a generating set
            and $\mu \colon X \to M$ the map satisfying the
            universal property that for
            any $R$-module $N$ and any map
            $f \colon X \to N$, there exists a unique
            $R$-linear map $\tilde{f} \colon M \to N$ 
            such that
            $\tilde{f}\mu = f$.

            \begin{equation*}
            \begin{tikzcd}
                X \ar[r, "\mu"] \ar[dr, "f"] & M \ar[d, "\exists!
                \tilde{f}",
                dashed]\\
                                             & N
            \end{tikzcd}
            \end{equation*}

            We claim $M$ has $\mu (X)$ as a basis.
            Suppose there exists
            a finite linear combination
            $\sum_{i=1}^{n} r_i x_i = 0$ with
            $x_i \in \mu(X)$ and $r_i \in R$ such that
            not all $r_i$ are $0$.
            Then let $N = \bigoplus_{X} R$ and
            $f \colon X \to \bigoplus_X R$ be the
            inclusion
            $x \mapsto \iota_x \left( \mathbbm{1}_R \right) $.
            By uniqueness, we must have
            that 
            $0 = \tilde{f}\left( \mu (x_i) \right) 
            = \iota_x \left( \mathbbm{1}_R \right) $, so
            $\tilde{f}\left( \sum r_i x_i \right) 
            = \sum r_i \tilde{f} \mu (x_i')
            = \sum r_i f\left( x_i' \right)
            = \sum r_i \iota_{x_i'}\left( \mathbbm{1}_R \right) $,
            but by
            definition then 
            $0 = \left( \sum r_i f\left( x_i' \right)  \right) (j)
            = \left( \sum r_i \iota_{x_i'} \left( \mathbbm{1}_R \right) 
            \right) (j) =
            \sum r_i \iota_{x_i'}\left( \mathbbm{1}_R \right) (j)
            \sum r_i \mathbbm{1}_R \delta_{i,j}
            = r_j
            $ for all $j$, hence
            we obtain linear indepence.

            Conversely, suppose
            $M$ has $X \subset M$ as a basis.
            Let $\mu \colon X \to M$ be the inclusion.
            We claim this is a free $R$-module.

            Suppose $N$ is any other $R$-module
            and we have any (not necessarily an
            $R$-module homomorphism) map $f \colon X \to N$.
            If $\tilde{f} \mu = f$, then we
            must have $\tilde{f} (x) = f(x)$ for
            all $x \in X$. Now, for any linear combination
            $\sum r_i x_i \in M$, we have
            by linearity, 
            $\tilde{f} \left( \sum r_i x_i \right) 
            = \sum r_i \tilde{f}(x_i)
            = \sum r_i f(x_i)$, so
            $\tilde{f}$ is indeed uniquely determined
            by $f$.

        \end{proof}


        \begin{exercise}[]
            Complete the proof that every commutative
            ring $R$ has invariant basis number.
        \end{exercise}

        \begin{proof}
            By Zorn's lemma, every commutative ring
            $R$ has a maximal ideal $I \le R$. Then
            $R /I$ is a field. Let
            now $M$ be a free  $R$-module with basis
            $\left\{ x_i \right\}_{i \in J}$. 

            We claim $M / IM$ is an $R /I$ module.
            Define
            $\overline{r} \cdot  \overline{x} = \overline{rx}$ where
            multiplication of $rx$ is done in $M$ over
            $R$, and define $\overline{v} + \overline{w}
            = \overline{v+w}$.\\
            Suppose $\overline{r} = \overline{r'}$ and
            $\overline{x} = \overline{x'}$.
            So $r' -r \in I$ and
            $x' -x \in IM$.
            Then $r'x' - r x' \in IM$ and
             $rx' - rx \in IM$, so
             $r'x' - rx = r'x' - rx' + rx' - rx \in IM$,
             hence $\overline{r'x'} = \overline{rx}$ is well
             defined.
             Similarly,  if 
             $\overline{v} = \overline{v'}$ and
             $\overline{w} = \overline{w'}$ then
             $v-v', w-w' \in IM$, so
             $v+w - \left( v'+w' \right) 
             \in IM$, hence
             $\overline{v+w} = \overline{v'+w'}$.\\

             The properties for
             $M /IM$ being an $R /I$-module are then inherited
             from $M$ as an $R$-module.\\
             \linebreak
             But as $R /I$ is a field, $M /IM$ is a vector space
             over $R /I$, hence its dimension is well-defined.
             Now, suppose
             $\sum \overline{r_i} \cdot  \overline{x_i} = 0$ in
             $M /IM$. Then
             $\overline{\sum r_i x_i} = 0$, hence
             $\sum r_i x_i \in IM$. But  then
             there exists a linear combination
             $\sum s_i x_i$ such that $s_i \in I$ for all $i$,
             and such that $\sum r_ix_i = \sum s_i x_i$.
             Then $\sum (r_i - s_i) x_i = 0$ and linear
             independence of $\left\{ x_i \right\}_{i \in J}$ 
             gives $r_i = s_i$ for all $i$, hence
             $\overline{r_i} = \overline{0}$.
             So $\left\{ \overline{x_i} \right\}_{i \in J}$ 
             is linearly independent over $R /I$ as well.
             Hence as it clearly also spans
             $M /IM$, we have that
             $\dim_{R /I} M/IM = \left| J \right| $. 
             Thus an bases for
             $M $ over $R$ have the same cardinality, so
             $R$ has invariant basis number.
        \end{proof}

        \begin{exercise}[]
            Find a free non commutative ring $R$ with
            bases of different cardinalities.
        \end{exercise}

        \subsection{Free modules over PID}

        \subsection{Structure theorem for modules over
        PID}
        \subsection{Exact sequences}

        \begin{exercise}[]
            In the 2-out-of-3 lemma, what can
            you say about a third map if
            two of the $h_1,h_2,h_3$ are just
            injective (or just surjective)?

            \begin{equation*}
            \begin{tikzcd}
                0 \ar[r] & M' \ar[r, "f"]
                \ar[d, "h_1"] & M \ar[r, "g"] 
                \ar[d, "h_2"] &
                M'' \ar[r] \ar[d, "h_3"] & 0\\
                0 \ar[r] & N' \ar[r, "f'"] & N \ar[r, "g'"] &
                N'' \ar[r] & 0
            \end{tikzcd}
            \end{equation*}
        \end{exercise}


        \begin{solution}
            Suppose $h_1$ and $h_3$ are injective.
            By the same argument as the one in
            the notes, $h_2$ is also injective.

            Similarly, if $h_1$ and $h_3$ are surjective, the
            same argument as in the notes shows that
            $h_2$ is surjective.

            Suppose $h_1,h_2$ are injective. Let
            $h_3 (m) = 0$. Then
            by surjectivity of $g$,
            let $m' \in M$ such that
            $g(m') = m$. Then $g' h_2 (m') = 0$ so
            there exists  $m'' \in M'$ such that
            $f' h_1 (m'') = h_2 (m')$, so
            $h_2 f (m'') = h_2 (m')$. But
            $h_2$ is injective, so
            $m' = f(m'')$. Thus
            $m = g(m') = gf (m'') = 0$ by exactness.
            So $h_3$ is injective.
            If $h_2$ is surjective, then
            letting $n \in N''$, there exists
            $n' \in N$ with $g' (n') = n$ so by
            surjectivity of
            $h_2$, there exists $m \in M$ such that
            $h_2 (m) = n'$. Then
            $h_3 \left( g(m) \right) = n$, so
            $h_3$ is surjective.
        \end{solution}


        \newpage




        \begin{exercise}[]
            Show that if $0 \to N' \stackrel{f}{\to} N 
            \stackrel{g}{\to} N''$ is an exact sequence
            of $R$-modules and $g$ is surjective, then
            $g_*$ need not be surjective.
        \end{exercise}


        \subsection{Projective modules}

        \begin{definition}[]
            An $R$-module $P$ is projective if for
            every $R$-linear map $f \colon P \to M$ and
            every $R$-linear surjective map $q \colon N \to M$ 
            of $R$-modules, there exists
            an $R$-linear map $h \colon P \to N$ such that
            \begin{equation*}\label{projective}
            \begin{tikzcd}
                & P \ar[dl, "\exists h"] \ar[d, "f"] \\
                N \ar[r, "q", twoheadrightarrow] & M
            \end{tikzcd} \tag{$\Omega$}
            \end{equation*}
            commutes.
        \end{definition}

        \begin{exercise}[]
            Show that $h$ in~\eqref{projective}
            is not necessarily unique.
        \end{exercise}

        \begin{proof}
            a
        \end{proof}
        



        \begin{exercise}[]
            Let $R$ be a commutative ring and
            $A$ an $R$-algebra. Then a left
            $A$-module is the same thing as an
            $R$-module $M$ together with a homomorphism
            of $R$-algebras
            \[
            A \to \End_{_R Mod}(M)
            \] 
        \end{exercise}



    %\bibliography{../refs.bib}
\end{document}
