\documentclass[reqno]{amsart}
\usepackage{amscd, amssymb, amsmath, amsthm}
\usepackage{graphicx}
\usepackage[colorlinks=true,linkcolor=blue]{hyperref}
\usepackage[utf8]{inputenc}
\usepackage[T1]{fontenc}
\usepackage{textcomp}
\usepackage{babel}
%% for identity function 1:
\usepackage{bbm}
%%For category theory diagrams:
\usepackage{tikz-cd}
\usepackage{todonotes}


\setlength\parindent{0pt}

\pdfsuppresswarningpagegroup=1

\newtheorem{theorem}{Theorem}[section]
\newtheorem{lemma}[theorem]{Lemma}
\newtheorem{proposition}[theorem]{Proposition}
\newtheorem{corollary}[theorem]{Corollary}
\newtheorem{conjecture}[theorem]{Conjecture}

\theoremstyle{definition}
\newtheorem{definition}[theorem]{Definition}
\newtheorem{example}[theorem]{Example}
\newtheorem{exercise}[theorem]{Exercise}
\newtheorem{problem}[theorem]{Problem}
\newtheorem{question}[theorem]{Question}

\theoremstyle{remark}
\newtheorem*{remark}{Remark}
\newtheorem*{note}{Note}
\newtheorem*{solution}{Solution}



%Inequalities
\newcommand{\cycsum}{\sum_{\mathrm{cyc}}}
\newcommand{\symsum}{\sum_{\mathrm{sym}}}
\newcommand{\cycprod}{\prod_{\mathrm{cyc}}}
\newcommand{\symprod}{\prod_{\mathrm{sym}}}

%Linear Algebra

\DeclareMathOperator{\Span}{span}
\DeclareMathOperator{\im}{im}
\DeclareMathOperator{\diag}{diag}
\DeclareMathOperator{\Ker}{Ker}
\DeclareMathOperator{\ob}{ob}
\DeclareMathOperator{\Hom}{Hom}
\DeclareMathOperator{\Mor}{Mor}
\DeclareMathOperator{\sk}{sk}
\DeclareMathOperator{\Vect}{Vect}
\DeclareMathOperator{\Set}{Set}
\DeclareMathOperator{\Group}{Group}
\DeclareMathOperator{\Ring}{Ring}
\DeclareMathOperator{\Ab}{Ab}
\DeclareMathOperator{\Top}{Top}
\DeclareMathOperator{\hTop}{hTop}
\DeclareMathOperator{\Htpy}{Htpy}
\DeclareMathOperator{\Cat}{Cat}
\DeclareMathOperator{\CAT}{CAT}
\DeclareMathOperator{\Cone}{Cone}
\DeclareMathOperator{\dom}{dom}
\DeclareMathOperator{\cod}{cod}
\DeclareMathOperator{\Aut}{Aut}
\DeclareMathOperator{\Mat}{Mat}
\DeclareMathOperator{\Fin}{Fin}
\DeclareMathOperator{\rel}{rel}
\DeclareMathOperator{\Int}{Int}
\DeclareMathOperator{\sgn}{sgn}
\DeclareMathOperator{\Homeo}{Homeo}
\DeclareMathOperator{\SHomeo}{SHomeo}
\DeclareMathOperator{\PSL}{PSL}
\DeclareMathOperator{\Bil}{Bil}
\DeclareMathOperator{\Sym}{Sym}
\DeclareMathOperator{\Skew}{Skew}
\DeclareMathOperator{\Alt}{Alt}
\DeclareMathOperator{\Quad}{Quad}
\DeclareMathOperator{\Sin}{Sin}
\DeclareMathOperator{\Supp}{Supp}
\DeclareMathOperator{\Char}{char}


%Row operations
\newcommand{\elem}[1]{% elementary operations
\xrightarrow{\substack{#1}}%
}

\newcommand{\lelem}[1]{% elementary operations (left alignment)
\xrightarrow{\begin{subarray}{l}#1\end{subarray}}%
}

%SS
\DeclareMathOperator{\supp}{supp}
\DeclareMathOperator{\Var}{Var}

%NT
\DeclareMathOperator{\ord}{ord}

%Alg
\DeclareMathOperator{\Rad}{Rad}
\DeclareMathOperator{\Jac}{Jac}

%Misc
\newcommand{\SL}{{\mathrm{SL}}}
\newcommand{\mobgp}{{\mathrm{PSL}_2(\mathbb{C})}}
\newcommand{\id}{{\mathrm{id}}}
\newcommand{\Mod}{{\mathrm{Mod}}}
\newcommand{\PMod}{{\mathrm{PMod}}}
\newcommand{\SMod}{{\mathrm{SMod}}}
\newcommand{\ud}{{\mathrm{d}}}
\newcommand{\Vol}{{\mathrm{Vol}}}
\newcommand{\Area}{{\mathrm{Area}}}
\newcommand{\diam}{{\mathrm{diam}}}
\newcommand{\End}{{\mathrm{End}}}


\newcommand{\reg}{{\mathtt{reg}}}
\newcommand{\geo}{{\mathtt{geo}}}

\newcommand{\tori}{{\mathcal{T}}}
\newcommand{\cpn}{{\mathtt{c}}}
\newcommand{\pat}{{\mathtt{p}}}

\let\Cap\undefined
\newcommand{\Cap}{{\mathcal{C}}ap}
\newcommand{\Push}{{\mathcal{P}}ush}
\newcommand{\Forget}{{\mathcal{F}}orget}




\begin{document}
    \begin{exercise}[1]
        Let
        \[
            N' \stackrel{f}{\to} N \stackrel{g}{\to} N'' \to 0
        \] 
        be exact and consider
        \[
            0 \to \Hom_R \left( N'', M \right) \stackrel{g^{*}}{\to}
            \Hom_R \left( N,M \right) \stackrel{f^{*}}{\to }
        \Hom_R \left( N', M \right) .
        \] 

        \begin{enumerate}
            \item Show that the induced sequence
                is exact.
            \item Show that even though
                $N' \to N$ is injective, the map
                $\Hom_R \left( N,M \right) \to 
                \Hom_R \left( N', M \right) $ may fail to
                be surjective.
        \end{enumerate}
    \end{exercise}

    \begin{proof}
        \begin{enumerate}
            \item Suppose
                $g^{*} \left( k \right) =
                g^{*} (h)$, so
                $kg = hg$ as maps
                $N \to M$. For any $n'' \in N''$, there
                exists $n \in N$ such that $g(n) = n''$, and
                so $k(n'') = kg(n) = hg(n) = g(n'')$ for all
                $n'' \in N''$, hence $k = h$. Suppose
                $f^{*}(h) = 0$, so $hf = 0$. By injectivity of
                $f$,  $\ker g = \im f \subset \ker h$.
                This means that $h$ factors through
                $g$ in the sense that there
                exists a map $l \colon N'' \to M$ such that
                $h = lg$. Thus
                $h \in \im g^{*}$.
            \item We have $\Hom_{\mathbb{Z}}
                \left( \mathbb{Q}, \mathbb{Q} / \mathbb{Z} \right) 
                = 0$ and
                $\Hom_{\mathbb{Z}}\left( \mathbb{Q},
                \mathbb{Z}\right) = 0$, so
                while
                $0 \to \mathbb{Z}\to \mathbb{Q} \to 
                \mathbb{Q} / \mathbb{Z} \to 0$ is exact,
                \[
                0 \to 0 \to \Hom_{\mathbb{Z}}\left( \mathbb{Q},
                \mathbb{Q}\right) \to 
                0 \to 0
                \] 
                cannot be, since
                $\Hom_{\mathbb{Z}}\left( \mathbb{Q},
                \mathbb{Q}\right) $ contains
                $\id \neq 0$, and is thus not isomorphic
                to the trivial ring.
        \end{enumerate}
    \end{proof}

    \begin{exercise}[3]
        Let $R$ be a commutative ring.
        \begin{enumerate}
            \item Show that if
                $\left\{ e_i  \mid i \in I \right\} $ and
                $\left\{ e_{j}'  \mid j \in J \right\} $ 
                are bases for
                $E$ and $E'$, respectively, where
                $E$ and $E'$ are both free $R$-modules,
                then $E \otimes_{R} E'$ is a free
                $R$-module with
                the simple tensors $e_i \otimes e_{j}'$  as a
                basis.
            \item Show that if $E$ and $E'$ are 
                projective $R$-modules, then
                $E \otimes_R E'$ is a projective $R$-module.
        \end{enumerate}
    \end{exercise}

    \begin{proof}
        \begin{enumerate}
            \item 
                The simple tensors
                span $E \otimes_R E'$ since 
                $\left( e_i, e_j' \right) $ are
                a basis for $E \times E'$ of which
                $E \otimes_R E'$ is a quotient.

                Suppose
                 \[
                \sum c_{i,j} e_i \otimes e_j' = 0,
                \] 
                with $c_{i_0,j_0}\neq 0$. Then
                since $E \times E'$ is free
                on $\left\{ \left( e_i, e_{j}' \right)  \right\} $,
                we define a map
                $\tilde{\varphi } 
                \colon E \times E' \to 
                E \otimes_R E'$ by
                $\left( e_i, e_j' \right) =
                e_i \otimes e_j'$ when
                $\left( i,j \right) = \left( i_0,j_0 \right) $ 
                and $0$ otherwise. Then
                the induced map
                 $\varphi \colon E \otimes_R E' \to E \otimes_R E'$ is not
                 identically $0$. Then
                 \[
                      0 = \sum c_{i,j} \varphi \left( e_i
                      \otimes e_j' \right) 
                      = c_{i_0,j_0} e_{i_0} \otimes
                      e_{j_0}'.
                 \] 
                 Now, suppose
                 $e_{i_0} \otimes
                 e_{j_0}' = 0$. But
                 then let 
                 $f \colon E \times E' \to E \times E'$ 
                 be the identity.
                 We then get a map
                 $\overline{f} \colon
                 E \otimes_R E' \to 
                 E \times E'$ sending
                 $e_{i_0} \otimes e_{j_0}'$ to
                 $\left( e_{i_0},e_{j_0}' \right) $, but
                 also
                 $\overline{f}$ is a homomorphism of
                 abelian groups, so it sends
                 $0$ to $0$, giving
                 $\left( e_{i_0}, e_{j_0}' \right) = \left( 0,0
                 \right) $, contradicting it being
                 a basis element. Hence
                 the set
                 $\left\{ e_i \otimes e_{j}'  \mid 
                 i \in I, j \in J\right\} $ is a basis
                 for $E \otimes_R E'$.
             \item 

                 Suppose we have maps $f,g$ as follows
                 \begin{equation*}
                 \begin{tikzcd}
                     E \times E' \ar[r, "h"]
                     \ar[dashed, dr, "\overline{h}"] 
                     & E \otimes_R E' \ar[d, "f"] \\
                     X \ar[r, two heads, "g"] & Y
                 \end{tikzcd}
                 \end{equation*}
                 Since $E$ and $E'$ are projective,
                 restricting $\overline{h}$ to either
                 coordinate induces a map
                 into $X$. So for
                 $e \in E$ fixed, there exists
                 $k_e \colon E' \to X$ such that
                 $g \circ k_e = \overline{h}(e,-)$.
                 Define a map
                 $k \colon E \times E' \to X$ by
                 $k\left( e,e' \right) =
                 k_{e}(e')$. Then
                 $g \circ k \left( e,e' \right) 
                 = g \circ k_e (e')
                 = \overline{h}\left( e,e' \right) 
                 = f \circ h \left( e,e' \right) $, hence
                 the diagram
                 
                 \begin{equation*}
                 \begin{tikzcd}
                     E \times E' \ar[r, "h"]
                     \ar[dashed, dr, "\overline{h}"] 
                     \ar[d, "k"]
                     & E \otimes_R E' \ar[d, "f"] \\
                     X \ar[r, two heads, "g"] & Y
                 \end{tikzcd}
                 \end{equation*}
                 commutes.
                 By the universal property of tensor products,
                 there exists a unique map
                 $\overline{k} \colon
                 E \otimes_R E' \to X$ such that

                 \begin{equation*}
                 \begin{tikzcd}
                     E \times E' \ar[r, "h"]
                     \ar[d, "k"]
                     & E \otimes_R E' \ar[d, "f"]
                     \ar[dl, "\exists !\overline{k}"']\\
                     X \ar[r, two heads, "g"] & Y
                 \end{tikzcd}
                 \end{equation*}
                 commutes.
                 Thus $E \otimes_R E'$ is projective.

        \end{enumerate}
    \end{proof}


    \begin{exercise}[4]
        Let $M$ be an $R$-module.
        \begin{enumerate}
            \item Show that for any two-sided ideal
                $\mathfrak{a} \subset R$, the set
                \[
                M \left[ \mathfrak{a} \right] 
                = \left\{ x \in M
                 \mid ax = 0 \text{ for each }
             a \in \mathfrak{a} \right\} \subset 
             M.
                \] 
              is a submodule of $M$.  
        \end{enumerate}
    \end{exercise}


    \begin{proof}
        \begin{enumerate}
            \item A submodule is a subgroup which is
                closed under multiplication by
                elements of the ring.

                Suppose $x,y \in 
                M\left[ \mathfrak{a} \right] $. Then
                for any $a \in \mathfrak{a}$,
                $a \left( x-y \right) 
                = ax + a (-1)y
                = 0$ whenever $\mathfrak{a}$ is a right
                ideal.

                Now, if $x \in M \left[ \mathfrak{a} \right] $,
                then
                for any $r \in R$ and any
                $a \in \mathfrak{a}$, we have
                $ar \in \mathfrak{a}$, so
                $a \cdot \left( rx \right) 
                = \left( ar \right) \cdot x
                = 0$ by assumption, so
                $rx \in M \left[ \mathfrak{a} \right] $.
            \item If $\mathfrak{b} \subset \mathfrak{a}$, then
                for any
                $x \in M\left[ \mathfrak{a} \right] $, we have
                that for all $b \in \mathfrak{b} \subset 
                \mathfrak{a}$,
                $bx = 0$, so
                $x \in M \left[ \mathfrak{b} \right] $.
                In particular, since
                $\mathfrak{a}^{s} \subset 
                \mathfrak{a}^{r}$ for $r \le s$,
                we have
                $M \left[ \mathfrak{a}^{r} \right] 
                \subset M\left[ \mathfrak{a}^{s} \right] $.
            \item We define the
                scalar multiplication by
                $\left[ r \right]_{R / \mathfrak{a}}
                \cdot \left[ x \right]_{M \left[ \mathfrak{a}^{r} \right] 
                / M\left[ \mathfrak{a}^{r-1} \right] }
                = \left[ rx \right]_{
                M\left[ \mathfrak{a}^{r} \right] 
            / M\left[ \mathfrak{a}^{r-1} \right] }$.
            We check that this is well-defined.
            Suppose
            $\overline{r} = \overline{r'}$ and
            $\overline{x } = \overline{x'}$. Then
            $r^{-1} r' \in \mathfrak{a}$ and
            $x' - x  \in M\left[ \mathfrak{a}^{r-1} \right] $.
            Then
            $r'x' - rx 
            = r' \left( x'-x \right) +
            \left( r'-r \right) x
            \in M \left[ \mathfrak{a}^{r-1} \right] $.
            The remaining module properties are inherited
            from $M$.

            \item
                Suppose $t \ge n$. Then
                for any $\overline{r} \in 
                R / \mathfrak{m}^{n}$ and
                any $m \in \mathfrak{m}^{t}$, we have
                $rm \in \mathfrak{m}^{n}$, so
                $\left( R / \mathfrak{m}^{n} \right) 
                \left[ \mathfrak{m}^{t} \right] 
                = R / \mathfrak{m}^{n}$.
                Suppose $t < n$. Then
                we claim
                $\left( R / \mathfrak{m}^{n} \right) 
                \left[ \mathfrak{m}^{t} \right] 
                = \left( \mathfrak{m}^{n-t}R \right) /
                \mathfrak{m}^{n}$. 
                $\mathfrak{m}^{n-t}R$ is generated
                by elements of the form
                $m_1 \cdots m_{n-t}r$, and
                $\mathfrak{m}^{t}$ is generated by
                elements of the form
                $m_{n-t+1} \cdots m_{n}$, so
                $r m_{n-t+1} \cdots m_n
                (r'm_1) m_2 \cdots m_{n-t}$ and
                $r' m_1 \in \mathfrak{m}$ as it is an ideal.
                Hence this is an element in
                $\mathfrak{m}^{n}$. So
                we have $\left( \mathfrak{m}^{n-t}R \right) /
                \mathfrak{m}^{n} \subset 
                \left( R / \mathfrak{m}^{n} 
                \right) \left[ \mathfrak{m}^{t} \right] $.
                If
                $n \in \left( R / \mathfrak{m}^{n} \right) 
                \left[ \mathfrak{m}^{t} \right] $, then
                for all  $m \in \mathfrak{m}^{t}$, we have
                $m n = 0$, so
                $mn \in \mathfrak{m}^{n}$.
                Let's postpone this for later.

                \todo{}

        \end{enumerate}
    \end{proof}

    \begin{exercise}[5]
        Show that $\mathbb{Z} / 3 $ is a projective
        $\mathbb{Z} /6$-module.
    \end{exercise}

    \begin{proof}
        $\mathbb{Z} /3$ is a projective
        $\mathbb{Z} / 6$-module if and only if
        $\mathbb{Z} / 3$ is a direct summand in a 
        free $\mathbb{Z} / 6$-module which it indeed is:
        $\mathbb{Z} / 6 \approx \mathbb{Z} / 3 \oplus
        \mathbb{Z} / 2$.
    \end{proof}

    











    %\bibliography{../refs.bib}
\end{document}
