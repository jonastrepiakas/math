\documentclass[reqno]{amsart}
\usepackage{amscd, amssymb, amsmath, amsthm}
\usepackage{graphicx}
\usepackage[colorlinks=true,linkcolor=blue]{hyperref}
\usepackage[utf8]{inputenc}
\usepackage[T1]{fontenc}
\usepackage{textcomp}
\usepackage{babel}
%% for identity function 1:
\usepackage{bbm}
%%For category theory diagrams:
\usepackage{tikz-cd}


\setlength\parindent{0pt}

\pdfsuppresswarningpagegroup=1

\newtheorem{theorem}{Theorem}[section]
\newtheorem{lemma}[theorem]{Lemma}
\newtheorem{proposition}[theorem]{Proposition}
\newtheorem{corollary}[theorem]{Corollary}
\newtheorem{conjecture}[theorem]{Conjecture}

\theoremstyle{definition}
\newtheorem{definition}[theorem]{Definition}
\newtheorem{example}[theorem]{Example}
\newtheorem{exercise}[theorem]{Exercise}
\newtheorem{problem}[theorem]{Problem}
\newtheorem{question}[theorem]{Question}

\theoremstyle{remark}
\newtheorem*{remark}{Remark}
\newtheorem*{note}{Note}
\newtheorem*{solution}{Solution}



%Inequalities
\newcommand{\cycsum}{\sum_{\mathrm{cyc}}}
\newcommand{\symsum}{\sum_{\mathrm{sym}}}
\newcommand{\cycprod}{\prod_{\mathrm{cyc}}}
\newcommand{\symprod}{\prod_{\mathrm{sym}}}

%Linear Algebra

\DeclareMathOperator{\Span}{span}
\DeclareMathOperator{\Ima}{Im}
\DeclareMathOperator{\diag}{diag}
\DeclareMathOperator{\Ker}{Ker}
\DeclareMathOperator{\ob}{ob}
\DeclareMathOperator{\Hom}{Hom}
\DeclareMathOperator{\Mor}{Mor}
\DeclareMathOperator{\sk}{sk}
\DeclareMathOperator{\Vect}{Vect}
\DeclareMathOperator{\Set}{Set}
\DeclareMathOperator{\Group}{Group}
\DeclareMathOperator{\Ring}{Ring}
\DeclareMathOperator{\Ab}{Ab}
\DeclareMathOperator{\Top}{Top}
\DeclareMathOperator{\hTop}{hTop}
\DeclareMathOperator{\Htpy}{Htpy}
\DeclareMathOperator{\Cat}{Cat}
\DeclareMathOperator{\CAT}{CAT}
\DeclareMathOperator{\Cone}{Cone}
\DeclareMathOperator{\dom}{dom}
\DeclareMathOperator{\cod}{cod}
\DeclareMathOperator{\Aut}{Aut}
\DeclareMathOperator{\Mat}{Mat}
\DeclareMathOperator{\Fin}{Fin}
\DeclareMathOperator{\rel}{rel}
\DeclareMathOperator{\Int}{Int}
\DeclareMathOperator{\sgn}{sgn}
\DeclareMathOperator{\Homeo}{Homeo}
\DeclareMathOperator{\SHomeo}{SHomeo}
\DeclareMathOperator{\PSL}{PSL}
\DeclareMathOperator{\Bil}{Bil}
\DeclareMathOperator{\Sym}{Sym}
\DeclareMathOperator{\Skew}{Skew}
\DeclareMathOperator{\Alt}{Alt}
\DeclareMathOperator{\Quad}{Quad}
\DeclareMathOperator{\Sin}{Sin}
\DeclareMathOperator{\Supp}{Supp}
\DeclareMathOperator{\Char}{char}


%Row operations
\newcommand{\elem}[1]{% elementary operations
\xrightarrow{\substack{#1}}%
}

\newcommand{\lelem}[1]{% elementary operations (left alignment)
\xrightarrow{\begin{subarray}{l}#1\end{subarray}}%
}

%SS
\DeclareMathOperator{\supp}{supp}
\DeclareMathOperator{\Var}{Var}

%NT
\DeclareMathOperator{\ord}{ord}

%Alg
\DeclareMathOperator{\Rad}{Rad}
\DeclareMathOperator{\Jac}{Jac}

%Misc
\newcommand{\SL}{{\mathrm{SL}}}
\newcommand{\mobgp}{{\mathrm{PSL}_2(\mathbb{C})}}
\newcommand{\id}{{\mathrm{id}}}
\newcommand{\Mod}{{\mathrm{Mod}}}
\newcommand{\PMod}{{\mathrm{PMod}}}
\newcommand{\SMod}{{\mathrm{SMod}}}
\newcommand{\ud}{{\mathrm{d}}}
\newcommand{\Vol}{{\mathrm{Vol}}}
\newcommand{\Area}{{\mathrm{Area}}}
\newcommand{\diam}{{\mathrm{diam}}}
\newcommand{\End}{{\mathrm{End}}}


\newcommand{\reg}{{\mathtt{reg}}}
\newcommand{\geo}{{\mathtt{geo}}}

\newcommand{\tori}{{\mathcal{T}}}
\newcommand{\cpn}{{\mathtt{c}}}
\newcommand{\pat}{{\mathtt{p}}}

\let\Cap\undefined
\newcommand{\Cap}{{\mathcal{C}}ap}
\newcommand{\Push}{{\mathcal{P}}ush}
\newcommand{\Forget}{{\mathcal{F}}orget}




\begin{document}
    Given a $\Delta$-complex ($X$, $\Sigma = (\Sigma_p)_{p \in \mathbb{N}}$), we can consider the evaluation map $ev \colon \Sigma_n \times \Delta^n \to X^n$ by
$(\sigma,t) \mapsto \sigma(t)$ where $X^n$ is the $n$-skeleton
$$ X^n = \cup_{p \leq n} \cup_{\sigma \in \Sigma_p} \sigma(\Delta^p) $$.

We give each $\Sigma_p$ the discrete topolgoy so that the evaluation map becomes continuous.

We then are supposed to get a homeomorphism of quotients
$$ Y:= \frac{\Sigma_n \times \Delta^n}{\Sigma_n \times \partial \Delta^n} 
\stackrel{ev}{\to} \frac{X^n}{X^{n-1}}$$.

I wanted to do this by showing that $ev$ is closed, but I'm not sure if the details work out. Here it is:

Let $A \subset Y$ be closed and let $Z = \Sigma_n \times \partial \Delta^n$ and
$\overline{Z} = \pi (Z)$ in the quotient. If $\overline{Z} \in A$ then $\pi^{-1}(Y-A)
\subset \Sigma_n \times (\Delta^n - \partial \Delta^n)$ is open and $ev$ is injective on here. I claim $ev$ is a local homeomorphism on $\Sigma_n \times (\Delta^n - \partial \Delta^n)$: let $(\sigma, t) \in \Sigma_n \times (\Delta^n - \partial \Delta^n)$. Then
we can find some open neighborhood $U$ of $t$ contained in $\Delta^n - \partial \Delta^n$ and since $\sigma$ is injective on $U$ and the subsets $\tau (\Delta^n - \partial \Delta^n)$ are disjoint for $\tau$ ranging over $\Sigma_n$, we have that $U = \sigma^{-1} (\sigma(U))$, hence $\sigma(U)$ is open in $X^n - X^{n-1}$, thus also saturated with respect to the quotient, hence also open in the quotient space. Bijective local homeomorphisms are homeomorphisms, so $ev$ restricted to
$\Sigma_n \times (\Delta^n - \partial \Delta^n)$ is an embedding, so $\frac{X^n}{X^{n-1}} - ev(A) = ev (\pi^{-1}(Y-A))$ is open, hence $A$ is closed.

In the case where $\overline{Z} \not \subset A$, we have $\pi^{-1}(A) \subset \Sigma_n \times (\Delta^n - \partial \Delta^n)$ is closed, so by the above, its image under $ev$ is closed.
    %\bibliography{../refs.bib}
\end{document}
