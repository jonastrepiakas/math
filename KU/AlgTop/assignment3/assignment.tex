\documentclass[reqno]{amsart}
\usepackage{amscd, amssymb, amsmath, amsthm}
\usepackage{graphicx}
\usepackage[colorlinks=true,linkcolor=blue]{hyperref}
\usepackage[utf8]{inputenc}
\usepackage[T1]{fontenc}
\usepackage{textcomp}
\usepackage{babel}
%% for identity function 1:
\usepackage{bbm}
%%For category theory diagrams:
\usepackage{tikz-cd}


\setlength\parindent{0pt}

\pdfsuppresswarningpagegroup=1

\newtheorem{theorem}{Theorem}[section]
\newtheorem{lemma}[theorem]{Lemma}
\newtheorem{proposition}[theorem]{Proposition}
\newtheorem{corollary}[theorem]{Corollary}
\newtheorem{conjecture}[theorem]{Conjecture}

\theoremstyle{definition}
\newtheorem{definition}[theorem]{Definition}
\newtheorem{example}[theorem]{Example}
\newtheorem{exercise}[theorem]{Exercise}
\newtheorem{problem}[theorem]{Problem}
\newtheorem{question}[theorem]{Question}

\theoremstyle{remark}
\newtheorem*{remark}{Remark}
\newtheorem*{note}{Note}
\newtheorem*{solution}{Solution}



%Inequalities
\newcommand{\cycsum}{\sum_{\mathrm{cyc}}}
\newcommand{\symsum}{\sum_{\mathrm{sym}}}
\newcommand{\cycprod}{\prod_{\mathrm{cyc}}}
\newcommand{\symprod}{\prod_{\mathrm{sym}}}

%Linear Algebra

\DeclareMathOperator{\Span}{span}
\DeclareMathOperator{\Ima}{Im}
\DeclareMathOperator{\diag}{diag}
\DeclareMathOperator{\Ker}{Ker}
\DeclareMathOperator{\ob}{ob}
\DeclareMathOperator{\Hom}{Hom}
\DeclareMathOperator{\sk}{sk}
\DeclareMathOperator{\Vect}{Vect}
\DeclareMathOperator{\Set}{Set}
\DeclareMathOperator{\Group}{Group}
\DeclareMathOperator{\Ring}{Ring}
\DeclareMathOperator{\Ab}{Ab}
\DeclareMathOperator{\Top}{Top}
\DeclareMathOperator{\hTop}{hTop}
\DeclareMathOperator{\Htpy}{Htpy}
\DeclareMathOperator{\Cat}{Cat}
\DeclareMathOperator{\CAT}{CAT}
\DeclareMathOperator{\Cone}{Cone}
\DeclareMathOperator{\dom}{dom}
\DeclareMathOperator{\cod}{cod}
\DeclareMathOperator{\Aut}{Aut}
\DeclareMathOperator{\Mat}{Mat}
\DeclareMathOperator{\Fin}{Fin}
\DeclareMathOperator{\rel}{rel}
\DeclareMathOperator{\Int}{Int}
\DeclareMathOperator{\sgn}{sgn}
\DeclareMathOperator{\Homeo}{Homeo}
\DeclareMathOperator{\PSL}{PSL}
\DeclareMathOperator{\Bil}{Bil}
\DeclareMathOperator{\Sym}{Sym}
\DeclareMathOperator{\Skew}{Skew}
\DeclareMathOperator{\Alt}{Alt}
\DeclareMathOperator{\Quad}{Quad}


%Row operations
\newcommand{\elem}[1]{% elementary operations
\xrightarrow{\substack{#1}}%
}

\newcommand{\lelem}[1]{% elementary operations (left alignment)
\xrightarrow{\begin{subarray}{l}#1\end{subarray}}%
}

%SS
\DeclareMathOperator{\supp}{supp}
\DeclareMathOperator{\Var}{Var}

%NT
\DeclareMathOperator{\ord}{ord}

%Alg
\DeclareMathOperator{\Rad}{Rad}
\DeclareMathOperator{\Jac}{Jac}

%Misc
\newcommand{\SL}{{\mathrm{SL}}}
\newcommand{\mobgp}{{\mathrm{PSL}_2(\mathbb{C})}}
\newcommand{\id}{{\mathrm{id}}}
\newcommand{\Mod}{{\mathrm{Mod}}}
\newcommand{\ud}{{\mathrm{d}}}
\newcommand{\Vol}{{\mathrm{Vol}}}
\newcommand{\Area}{{\mathrm{Area}}}
\newcommand{\diam}{{\mathrm{diam}}}
\newcommand{\End}{{\mathrm{End}}}


\newcommand{\reg}{{\mathtt{reg}}}
\newcommand{\geo}{{\mathtt{geo}}}

\newcommand{\tori}{{\mathcal{T}}}
\newcommand{\cpn}{{\mathtt{c}}}
\newcommand{\pat}{{\mathtt{p}}}




\begin{document}
    \begin{exercise}[]
        \begin{enumerate}
            \item Let $G \times X \to X$ be a continuous left
                properly discontinuous action. Prove
                that the action is free.
            \item Let $r \in \mathbb{R}$ be an irrational number,
                and let $G = \left( \mathbb{Z}^2, + \right) $ 
                act on $X = \mathbb{R}$ as 
                $\left( n,m \right) . t = t+ n + rm$. Prove
                that this action is free, but not
                properly discontinuous. Compare with the
                statement of exercise 3.
        \end{enumerate}
    \end{exercise}

    \begin{solution}
        (i) 
        Suppose it is not free, so there exists $g \in G -
        \left\{ e \right\} $ such
        that $g.x = x$ for some $x \in X$. If
        $G \times X \to X$ is properly discontinuous, then
        there exists $V \subset X$ with $x \in V$ such that
        $V \cap gV = \varnothing$ for all 
        $g \in G - \left\{ e \right\} $. But then
        $x = g.x \in gV$ and $x \in V$, so
        $x \in V \cap gV = \varnothing$, contradiction. Hence
        $G \times X \to X$ can not be properly discontinuous. Taking
        the contraposition gives the desired result.

        (ii)
        
        Suppose $t = (n,m).t = t + n + rm $, so
        $n + rm = 0$. But if $m\neq 0$, then
        $r = -\frac{n}{m} \in \mathbb{Q}$, contradicting irrationality
        of $r$, so $m = 0$ and hence $n = 0$, giving that
        the action is free. 

        Now, suppose there exists an open neighborhood
        $V$ of $0$ such that
        $\left( n,m \right) V \cap V \neq \varnothing$ implies
        $\left( n,m \right)  = (0,0)$. Since $V$ is open,
        we can find some basis open ball $B\left( 0,
        \delta \right) \subset V$. But by Dirichlet's
        approximation theorem, we can find integers
        $n,m$ with $m\ge 1$ such that
        $\left| n + rm \right| < \delta$, which implies that
        $n+rm \in (n,m) V \cap B\left( 0, \delta \right) 
        \subset (n,m) V \cap V$, giving
        $ (0,0) = \left( n,m \right) $, but
        $m \ge 1$, contradiction.\\
        \linebreak
        Since this action is not properly discontinuous, it
        means that the quotient map
        $\mathbb{R} \to  \mathbb{R} /G$ cannot be a covering
        map since $0$, for example, has no evenly covered neighborhood(
        if it had, then this neighborhood would split
        into homeomorphic disjoint copies in $\mathbb{R}$ 
        satisfying which would satisfy the requirement of
        the action being properly discontinuous at  $0$ ).


        \begin{exercise}[]
            Let $X$ and $B$ be Hausdorff path-connected
            spaces, and let $p \colon X \to B$ be a
            local homeomorphism, i.e., for all $x \in X$,
            there is a neighborhood $U \subset X$ of
            $x$ such that $p(U)$ is open in $B$, and
            the restriction
            \[
            p|_{U} \colon U \to p(U)
            \] 
            is a homeomorphism. Also, suppose that for all
            $b \in B$, $p^{-1}\left( \left\{ b \right\}  \right) $ 
            is finite, of the same cardinality for all
            points. Show that $p$ is a covering map.
        \end{exercise}

        \begin{solution}
            Let $F = \left\{ 1, \ldots, n:= \left| p^{-1}\left( 
            \left\{ b \right\} \right)  \right|  \right\}\subset 
            \mathbb{N} $. In particular,
            $ p^{-1}\left( \left\{ b \right\}  \right) 
            $ is non-empty for all $b \in B$ since
            it has the same cardinality for all $b \in B$ and
            so if it were empty, then
            $p^{-1}(B) = \varnothing$, contradicting $p$ being
            a function. Now let $b \in B$ and let
            $ p^{-1}(\left\{ b \right\} ) = 
            \left\{ x_1, \ldots, x_n \right\}$. Then
            by Hausdorffness, we can choose
            open neighborhoods  $U_{1,i}$ and
            $U_i$ for $x_1$ and $x_i$ for all $i =2, \ldots, n$ 
            such that $U_{1,i} \cap U_i = \varnothing$. Let
            $V_1 = \bigcap_{i =2}^{n} U_{1,i}$ which is a 
            neighborhood of $x_1$. Now
            for each $i = 2,\ldots,n$, define
            $V_i = p^{-1}\left( p(V_1) \right) 
            \cap U_i$.
            Suppose $y \in V_i \cap V_j \neq \varnothing$. Then
            let $\gamma \colon I \to V_i$



            So in particular, $p\left( U_i \right) $ is an 
            open neighborhood of $b$. Let
            $V = \bigcap_{i=1}^{n} p\left( U_i \right) $.
            Then let $U_i' = U_i \cap p^{-1}(V)
            \subset U_i$ which is
            open for each $i$, and since
            a restriction of a homeomorphism to a smaller
            open set is still a homeomorphism onto its
            image, we get that
            $p |_{U_i'} \colon U_{i}' \to 
            p\left( U_i' \right) = V$ is a homeomorphism
            for each $i$.
            Suppose
            $x \in U_i' \cap U_j'$ for distinct $i$ and $j$.
            Then since the intersection of finitely many 
            open path-connected sets with a common point 
            is still path connected, we can choose
            a path $\gamma \colon I \to U_i' \cup U_j'$
            from $x_i$ to $x_j$. Thus, under

        \end{solution}


        \newpage

        To show that the action is not properly discontinuous,
        we show something stronger:
        \begin{lemma}[]
            If $G \neq \left\{ e \right\} $ acts
            properly discontinuously on $\mathbb{R}$, then
            $G \approx \mathbb{Z}$.
        \end{lemma}

        \begin{proof}
            We have $\pi_1 \left( \mathbb{R} / G \right) 
            \approx G$. Let $\alpha \in 
            \pi_1 \left( \mathbb{R} / G, x_0 \right) -
            \left\{ e \right\} $. Then
            $\alpha$ lifts to a path $\tilde{\alpha}
            \colon I \to \mathbb{R}$ starting
            at some $\tilde{x_0}\in p^{-1}(x_0)$ 
            where $p \colon \mathbb{R} \to \mathbb{R} /G$ is
            the covering map.\\
            \linebreak
            

            Firstly, we show that, for any
            two $y,z \in p^{-1}(x_0)$,
            $\left( y,z \right) \cap p^{-1}(x_0)$ is finite.
            For the map $\mathbb{R} \to \mathbb{R}$ by
            $x\mapsto g.x$ for some $g \in G$ is a homeomorphism,
            so letting $V$ be an open
            neighborhood around $\tilde{x_0}$ for
            which $g V \cap V \neq \varnothing$ implies
            $g = e$, we would get
            that there exist infinitely many elements
            $g \in G$ for which $g V \subset 
            \left( y,z \right) $ and for
            two distinct such $g,g'$, 
            $gV \cap g'V = \varnothing$ since
            otherwise $g^{-1}g' V \cap V \neq \varnothing$ and
            hence $g^{-1} g' = e$ so
            $g = g'$. Thus this would imply that
            there is an infinite disjoint union of
            open sets $gV$ of the same measure (since $x \mapsto 
            gx$ is a homeomorphism) contained in
            $\left( y,z \right) $. Since
            $\left( y,z \right) $ has finite measure, 
            this implies that
            $g V$ has measure $0$, contradicting it being open.\\
            \linebreak
            This gives a total ordering
            on $p^{-1}(x_0)$ inherited from
            $\mathbb{R}$ with finitely many elements
            in $p^{-1}(x_0)$ between any two elements
            in $p^{-1}(x_0)$.
            
            Let $\tilde{x_0} \in p^{-1}(x_0)$ and
            $\tilde{x_0}'$ be its successor in
            $p^{-1}(x_0)$. We claim that the image, $\gamma$,
            under $p$ of the
            path $\tilde{\gamma} \colon I \to \mathbb{R}$ connecting
            $\tilde{x_0}$ to $\tilde{x_0}'$ linearly
            generates $\pi_1 \left( \mathbb{R} /G \right) $ 
            as a cyclic group. Suppose
            $\alpha \in \pi_1 \left( \mathbb{R} /G \right) $ and
            lift it to a path $\tilde{\alpha}$ starting
            at $\tilde{x_0}$. Let
            $\tilde{x_n} = \tilde{\alpha}(1)$. By
            uniqueness of lifts, we may assume by homotopy
            that this is the straight line
            path between the points $\tilde{x_0}$ and
            $\tilde{x_n}$, and
            suppose
            $\tilde{\alpha}^{-1} \left( p^{-1}(x_0) \right)
            = \left\{ 0, t_1, \ldots, t_{n-1}, 1 \right\} $.
            
            Now, since any loop at $x_0$ lifts
            to a path between two points in
            $p^{-1}(x_0)$, we thus see that
            there are precisely two simple loops
            in $\pi_1 \left( \mathbb{R} /G \right) $ corresponding
            to the projections of the path going from
            $\tilde{x_i}$ to $\tilde{x_{i+1}}$ and
            the path going from $\tilde{x_{i}}$ to 
            $\tilde{x_{i-1}}$. In particular, since
            the inverse of one of them is again simple, these
            loops are each other's inverse. 


            Thus, letting $\gamma \in 
            \pi_1 \left( \mathbb{R} /G, x_0 \right) $ denote
            one of these paths, we see that
            $p \circ \tilde{\alpha}
            = \gamma^{k_1} * \ldots * \gamma^{k_n}$ where
            $k_i = \pm 1 $. So
            $\alpha \in  \left<\gamma  \right>$.
            Thus
            $\mathbb{Z} \approx \left<\gamma \right> =
            \pi_1 \left( \mathbb{R} / G \right) \approx
            G$.


        \end{proof}
        







    \end{solution}









    %\bibliography{../refs.bib}
\end{document}
