\documentclass[reqno]{amsart}
\usepackage{amscd, amssymb, amsmath, amsthm}
\usepackage{graphicx}
\usepackage[colorlinks=true,linkcolor=blue]{hyperref}
\usepackage[utf8]{inputenc}
\usepackage[T1]{fontenc}
\usepackage{textcomp}
\usepackage{babel}
%% for identity function 1:
\usepackage{bbm}
%%For category theory diagrams:
\usepackage{tikz-cd}


\setlength\parindent{0pt}

\pdfsuppresswarningpagegroup=1

\newtheorem{theorem}{Theorem}[section]
\newtheorem{lemma}[theorem]{Lemma}
\newtheorem{proposition}[theorem]{Proposition}
\newtheorem{corollary}[theorem]{Corollary}
\newtheorem{conjecture}[theorem]{Conjecture}

\theoremstyle{definition}
\newtheorem{definition}[theorem]{Definition}
\newtheorem{example}[theorem]{Example}
\newtheorem{exercise}[theorem]{Exercise}
\newtheorem{problem}[theorem]{Problem}
\newtheorem{question}[theorem]{Question}

\theoremstyle{remark}
\newtheorem*{remark}{Remark}
\newtheorem*{note}{Note}
\newtheorem*{solution}{Solution}



%Inequalities
\newcommand{\cycsum}{\sum_{\mathrm{cyc}}}
\newcommand{\symsum}{\sum_{\mathrm{sym}}}
\newcommand{\cycprod}{\prod_{\mathrm{cyc}}}
\newcommand{\symprod}{\prod_{\mathrm{sym}}}

%Linear Algebra

\DeclareMathOperator{\Span}{span}
\DeclareMathOperator{\Ima}{Im}
\DeclareMathOperator{\diag}{diag}
\DeclareMathOperator{\Ker}{Ker}
\DeclareMathOperator{\ob}{ob}
\DeclareMathOperator{\Hom}{Hom}
\DeclareMathOperator{\sk}{sk}
\DeclareMathOperator{\Vect}{Vect}
\DeclareMathOperator{\Set}{Set}
\DeclareMathOperator{\Group}{Group}
\DeclareMathOperator{\Ring}{Ring}
\DeclareMathOperator{\Ab}{Ab}
\DeclareMathOperator{\Top}{Top}
\DeclareMathOperator{\hTop}{hTop}
\DeclareMathOperator{\Htpy}{Htpy}
\DeclareMathOperator{\Cat}{Cat}
\DeclareMathOperator{\CAT}{CAT}
\DeclareMathOperator{\Cone}{Cone}
\DeclareMathOperator{\dom}{dom}
\DeclareMathOperator{\cod}{cod}
\DeclareMathOperator{\Aut}{Aut}
\DeclareMathOperator{\Mat}{Mat}
\DeclareMathOperator{\Fin}{Fin}
\DeclareMathOperator{\rel}{rel}
\DeclareMathOperator{\Int}{Int}
\DeclareMathOperator{\sgn}{sgn}
\DeclareMathOperator{\Homeo}{Homeo}
\DeclareMathOperator{\PSL}{PSL}
\DeclareMathOperator{\Bil}{Bil}
\DeclareMathOperator{\Sym}{Sym}
\DeclareMathOperator{\Skew}{Skew}
\DeclareMathOperator{\Alt}{Alt}
\DeclareMathOperator{\Quad}{Quad}


%Row operations
\newcommand{\elem}[1]{% elementary operations
\xrightarrow{\substack{#1}}%
}

\newcommand{\lelem}[1]{% elementary operations (left alignment)
\xrightarrow{\begin{subarray}{l}#1\end{subarray}}%
}

%SS
\DeclareMathOperator{\supp}{supp}
\DeclareMathOperator{\Var}{Var}

%NT
\DeclareMathOperator{\ord}{ord}

%Alg
\DeclareMathOperator{\Rad}{Rad}
\DeclareMathOperator{\Jac}{Jac}

%Misc
\newcommand{\SL}{{\mathrm{SL}}}
\newcommand{\mobgp}{{\mathrm{PSL}_2(\mathbb{C})}}
\newcommand{\id}{{\mathrm{id}}}
\newcommand{\Mod}{{\mathrm{Mod}}}
\newcommand{\ud}{{\mathrm{d}}}
\newcommand{\Vol}{{\mathrm{Vol}}}
\newcommand{\Area}{{\mathrm{Area}}}
\newcommand{\diam}{{\mathrm{diam}}}
\newcommand{\End}{{\mathrm{End}}}


\newcommand{\reg}{{\mathtt{reg}}}
\newcommand{\geo}{{\mathtt{geo}}}

\newcommand{\tori}{{\mathcal{T}}}
\newcommand{\cpn}{{\mathtt{c}}}
\newcommand{\pat}{{\mathtt{p}}}




\begin{document}
    \begin{exercise}[]
        Let $U \subset V = \ell^2 $ consist of all sequences with finite
        support (i.e., $x_k \neq 0$ only for finitely many $k \in \mathbb{N}
        $ ).
        Show that $U$ has no orthocomplement and that the inclusion
        map $U \hookrightarrow V$ has no adjoint.
    \end{exercise}


    \begin{solution}
        Suppose $W$ is an orthocomplement to $U$.
        So $U \oplus W = \ell^2$. Then $W$ in particular must
        be the subspace containing all sequences
        for which there are infinitely many non-zero entries.
        Now, choose some element 
        $v \in U$, and let
        $v_N$ be the highest index entry in $v$ which is
        non-zero. Now let $w$ be given by
        $\left( 0, \ldots, 0, v_N, \frac{1}{N+1},
        \frac{1}{N+2}, \ldots \right) $ where
        the entry at $M > N$ is $\frac{1}{M}$ and
        the entry at $m < N$ is $0$. Then
        $\sum \left| w_k \right|^2 =
        \left| v_N \right|^2 + \sum_{k > n} \frac{1}{k^2} < 
        \infty $, so
        $w \in \ell^2$ and since it does not have finite support,
        $w \in W$. In particular, then, if
        $U \oplus W$ is an orthogonal direct sum, we must have
        \[
        \left| w_N \right|^2 = 
        \sum_{k \in \mathbb{N} } v_k \overline{w_k}
        = \left<v, w \right> = 0,
        \] 
        contradicting that $w_N$ was a nonzero entry from
        $F$. Thus no orthocomplement can exist
        to $U$.

        Let $A \in \Hom (U , \ell^2)$ be the inclusion map.
        Suppose $A$ has an adjoint $A^{*}$. Suppose
        $y \perp \Ima A$. Then assume $y_N$ is some
        non-zero entry, and let
        $v \in U$ be the element for which
        $v_N = y_N$ and $v_n = 0$ when $n \neq N$. Then
        $\left< Av, y \right> = 
        \left<v , y \right> = 
        \left| y_N \right|^2 \neq 0$ which
        contradicts $y \perp \Ima A$. Hence
        $y = 0$, so
        $\left( \Ima A \right)^{\perp} = \left\{ 0 \right\} $
        and thus $N \left( A^{*} \right) = 
        \left\{ 0 \right\} $ by lemma 7.16, so
        $A^{*}$ is injective.
        Now define the functional
        $z \in U'$ by $z(x) = \|x\|=
        \left< x,x \right> = \left<A x , x \right>
        = \left<x, A^{*}x \right>$. By uniqueness of
        theorem 7.13.(a), $A^{*}x = x$ for all
        $x \in U$, so in particular,
        $A^{*} A = \mathbbm{1}_U$. Now suppose
        $y \in \ell^2 - U$ which is not empty since
        for example $\left( \frac{1}{n} \right) \in 
        \ell^2 - U$. Then $A^{*} y =
        A^{*} A A^{*}(y)$, so
        $y = A A^{*} (y) \in \Ima A$ by injectivity of
        $A^{*}$, so
        $y$ has finite support, contradiction.
    \end{solution}














    %\bibliography{../refs.bib}
\end{document}
