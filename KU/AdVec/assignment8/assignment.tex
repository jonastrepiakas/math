\documentclass[reqno]{amsart}
\usepackage{amscd, amssymb, amsmath, amsthm}
\usepackage{graphicx}
\usepackage[colorlinks=true,linkcolor=blue]{hyperref}
\usepackage[utf8]{inputenc}
\usepackage[T1]{fontenc}
\usepackage{textcomp}
\usepackage{babel}
%% for identity function 1:
\usepackage{bbm}
%%For category theory diagrams:
\usepackage{tikz-cd}


\setlength\parindent{0pt}

\pdfsuppresswarningpagegroup=1

\newtheorem{theorem}{Theorem}[section]
\newtheorem{lemma}[theorem]{Lemma}
\newtheorem{proposition}[theorem]{Proposition}
\newtheorem{corollary}[theorem]{Corollary}
\newtheorem{conjecture}[theorem]{Conjecture}

\theoremstyle{definition}
\newtheorem{definition}[theorem]{Definition}
\newtheorem{example}[theorem]{Example}
\newtheorem{exercise}[theorem]{Exercise}
\newtheorem{problem}[theorem]{Problem}
\newtheorem{question}[theorem]{Question}

\theoremstyle{remark}
\newtheorem*{remark}{Remark}
\newtheorem*{note}{Note}
\newtheorem*{solution}{Solution}



%Inequalities
\newcommand{\cycsum}{\sum_{\mathrm{cyc}}}
\newcommand{\symsum}{\sum_{\mathrm{sym}}}
\newcommand{\cycprod}{\prod_{\mathrm{cyc}}}
\newcommand{\symprod}{\prod_{\mathrm{sym}}}

%Linear Algebra

\DeclareMathOperator{\Span}{span}
\DeclareMathOperator{\Ima}{Im}
\DeclareMathOperator{\diag}{diag}
\DeclareMathOperator{\Ker}{Ker}
\DeclareMathOperator{\ob}{ob}
\DeclareMathOperator{\Hom}{Hom}
\DeclareMathOperator{\Mor}{Mor}
\DeclareMathOperator{\sk}{sk}
\DeclareMathOperator{\Vect}{Vect}
\DeclareMathOperator{\Set}{Set}
\DeclareMathOperator{\Group}{Group}
\DeclareMathOperator{\Ring}{Ring}
\DeclareMathOperator{\Ab}{Ab}
\DeclareMathOperator{\Top}{Top}
\DeclareMathOperator{\hTop}{hTop}
\DeclareMathOperator{\Htpy}{Htpy}
\DeclareMathOperator{\Cat}{Cat}
\DeclareMathOperator{\CAT}{CAT}
\DeclareMathOperator{\Cone}{Cone}
\DeclareMathOperator{\dom}{dom}
\DeclareMathOperator{\cod}{cod}
\DeclareMathOperator{\Aut}{Aut}
\DeclareMathOperator{\Mat}{Mat}
\DeclareMathOperator{\Fin}{Fin}
\DeclareMathOperator{\rel}{rel}
\DeclareMathOperator{\Int}{Int}
\DeclareMathOperator{\sgn}{sgn}
\DeclareMathOperator{\Homeo}{Homeo}
\DeclareMathOperator{\PSL}{PSL}
\DeclareMathOperator{\Bil}{Bil}
\DeclareMathOperator{\Sym}{Sym}
\DeclareMathOperator{\Skew}{Skew}
\DeclareMathOperator{\Alt}{Alt}
\DeclareMathOperator{\Quad}{Quad}
\DeclareMathOperator{\Sin}{Sin}


%Row operations
\newcommand{\elem}[1]{% elementary operations
\xrightarrow{\substack{#1}}%
}

\newcommand{\lelem}[1]{% elementary operations (left alignment)
\xrightarrow{\begin{subarray}{l}#1\end{subarray}}%
}

%SS
\DeclareMathOperator{\supp}{supp}
\DeclareMathOperator{\Var}{Var}

%NT
\DeclareMathOperator{\ord}{ord}

%Alg
\DeclareMathOperator{\Rad}{Rad}
\DeclareMathOperator{\Jac}{Jac}

%Misc
\newcommand{\SL}{{\mathrm{SL}}}
\newcommand{\mobgp}{{\mathrm{PSL}_2(\mathbb{C})}}
\newcommand{\id}{{\mathrm{id}}}
\newcommand{\Mod}{{\mathrm{Mod}}}
\newcommand{\ud}{{\mathrm{d}}}
\newcommand{\Vol}{{\mathrm{Vol}}}
\newcommand{\Area}{{\mathrm{Area}}}
\newcommand{\diam}{{\mathrm{diam}}}
\newcommand{\End}{{\mathrm{End}}}


\newcommand{\reg}{{\mathtt{reg}}}
\newcommand{\geo}{{\mathtt{geo}}}

\newcommand{\tori}{{\mathcal{T}}}
\newcommand{\cpn}{{\mathtt{c}}}
\newcommand{\pat}{{\mathtt{p}}}

\let\Cap\undefined
\newcommand{\Cap}{{\mathcal{C}}ap}




\begin{document}
    \begin{exercise}[]
        Let $\left[ A \right] = 
        \begin{pmatrix} a & b \\ c & d \end{pmatrix} $ with
        $ad - bc \neq 0$. Determine an explicit Bruhat decomposition
         $\left[ A \right] = \left[ S \right] 
         \left[ T \right]  \left[ U \right] $.
         Show that $T$ is unique, but
         not $\left[ S \right] $ and $\left[ U \right]$.
    \end{exercise}


    \begin{solution}
        We have $\left[ A \right] \begin{pmatrix} 
        1 \\ 0 \end{pmatrix} = 
        \begin{pmatrix} a \\ c \end{pmatrix} $ and
        $\left[ A \right] \begin{pmatrix} 0 \\ 1 \end{pmatrix} 
        = \begin{pmatrix} b \\ d \end{pmatrix} $.
        The inverse of $\left[ A \right] $ is
         \[
        \left[ A \right]^{-1}
        = \frac{1}{ad-bc} \begin{pmatrix} d & -b \\
        -c & a \end{pmatrix} .
        \] 
        Let $\left( x_1,x_2 \right) $ 
        be the basis $x_1 = e_1$ and
        $x_2$ whichever one of
        $\begin{pmatrix} a \\ c \end{pmatrix} $ and
        $\begin{pmatrix} b \\ d \end{pmatrix} $ it is
        linearly independent with. Suppose
        it is $\begin{pmatrix} a \\ c \end{pmatrix} $.
        In this case
        $\sigma = \left( 1 \, 2 \right) $.
        Thus $S = 
        \begin{pmatrix} 1 & a \\
        0 & c \end{pmatrix} $,
            $T = \begin{pmatrix} 0 & 1 \\
            1 & 0 \end{pmatrix} $ and
            \[
                R = 
                \begin{pmatrix} 
                    1 & \frac{d}{ad-bc}\\
                    0 & \frac{-c}{ad-bc}
                \end{pmatrix}.
            \] 
            Suppose
            \[
            A = STU = S'T'U'.
            \] 
            Then
            $T = S^{-1}S' T' U' U^{-1}$.
            So this reduces to showing that if
            $T = K T' L$ with
            $K$ and $L$ upper triangular, then
            $T = T'$.
            Now, $T,T$ are one of
                $I , \begin{pmatrix} 0 & 1 \\ 1 & 0 \end{pmatrix}  $
                Suppose
                $T = I$ and
                $T' = \begin{pmatrix} 0 & 1 \\ 1 & 0 \end{pmatrix}  $.
                Let
                $K = \begin{pmatrix} a & b \\
                0 & c\end{pmatrix} $ and
                $L =\begin{pmatrix} 
                d & e \\
            0 & f\end{pmatrix} $.
            Then
            \[
            I = KT'L = 
            \begin{pmatrix} bd & af+be \\
            cd & ce \end{pmatrix} 
            \] 
            so $cd = 0$ hence either
            $bd = 0$ or $ce = 0$, contradiction.
            So $T = T'$.
            If $T' = I$ and
                    $T = \begin{pmatrix} 0 & 1\\
                    1 & 0\end{pmatrix} $, then
                    \[
                        \begin{pmatrix} 0 & 1 \\
                        1 & 0\end{pmatrix} 
                            = \begin{pmatrix} ad & 
                            ae+ bf \\
                        0 & cf \end{pmatrix},
                    \] 
                    contradiction. So
                    indeed $T = T'$.\\
                    \linebreak
                    Now, the only requirements
                    in the proof of theorem 8.20 for
                    $x_1$ and $x_2$ are that
                    $x_1 \in \Span (e_1)$ and
                    $x_{\sigma(1)} \in 
                    A \left( \Span (e_1) \right) 
                    = \Span \left( \begin{pmatrix} a \\ c
                        \end{pmatrix} 
                    \right) $ while
                    $\Span \left( x_1, x_2 \right) = \mathbb{R}^2$.
                    So indeed any
                    $x_1 = \left( \lambda,0 \right) $ 
                    and $x_2 = 
                    \mu \begin{pmatrix} a \\ c \end{pmatrix} $ 
                    give a valid flag. In
                    this case, $S$ will for example
                    be
                    $S = \begin{pmatrix} \lambda
                    & \mu a \\ 0
                    & \mu c \end{pmatrix} $.
                    Thus we see that
                    $S$ and $U$ are not unique.

                    



        \end{solution}










    %\bibliography{../refs.bib}
\end{document}
