\documentclass[reqno]{amsart}
\usepackage{amscd, amssymb, amsmath, amsthm}
\usepackage{graphicx}
\usepackage[colorlinks=true,linkcolor=blue]{hyperref}
\usepackage[utf8]{inputenc}
\usepackage[T1]{fontenc}
\usepackage{textcomp}
\usepackage{babel}
%% for identity function 1:
\usepackage{bbm}
%%For category theory diagrams:
\usepackage{tikz-cd}


\setlength\parindent{0pt}

\pdfsuppresswarningpagegroup=1

\newtheorem{theorem}{Theorem}[section]
\newtheorem{lemma}[theorem]{Lemma}
\newtheorem{proposition}[theorem]{Proposition}
\newtheorem{corollary}[theorem]{Corollary}
\newtheorem{conjecture}[theorem]{Conjecture}

\theoremstyle{definition}
\newtheorem{definition}[theorem]{Definition}
\newtheorem{example}[theorem]{Example}
\newtheorem{exercise}[theorem]{Exercise}
\newtheorem{problem}[theorem]{Problem}
\newtheorem{question}[theorem]{Question}

\theoremstyle{remark}
\newtheorem*{remark}{Remark}
\newtheorem*{note}{Note}
\newtheorem*{solution}{Solution}



%Inequalities
\newcommand{\cycsum}{\sum_{\mathrm{cyc}}}
\newcommand{\symsum}{\sum_{\mathrm{sym}}}
\newcommand{\cycprod}{\prod_{\mathrm{cyc}}}
\newcommand{\symprod}{\prod_{\mathrm{sym}}}

%Linear Algebra

\DeclareMathOperator{\Span}{span}
\DeclareMathOperator{\Ima}{Im}
\DeclareMathOperator{\diag}{diag}
\DeclareMathOperator{\Ker}{Ker}
\DeclareMathOperator{\ob}{ob}
\DeclareMathOperator{\Hom}{Hom}
\DeclareMathOperator{\sk}{sk}
\DeclareMathOperator{\Vect}{Vect}
\DeclareMathOperator{\Set}{Set}
\DeclareMathOperator{\Group}{Group}
\DeclareMathOperator{\Ring}{Ring}
\DeclareMathOperator{\Ab}{Ab}
\DeclareMathOperator{\Top}{Top}
\DeclareMathOperator{\hTop}{hTop}
\DeclareMathOperator{\Htpy}{Htpy}
\DeclareMathOperator{\Cat}{Cat}
\DeclareMathOperator{\CAT}{CAT}
\DeclareMathOperator{\Cone}{Cone}
\DeclareMathOperator{\dom}{dom}
\DeclareMathOperator{\cod}{cod}
\DeclareMathOperator{\Aut}{Aut}
\DeclareMathOperator{\Mat}{Mat}
\DeclareMathOperator{\Fin}{Fin}
\DeclareMathOperator{\rel}{rel}
\DeclareMathOperator{\Int}{Int}
\DeclareMathOperator{\sgn}{sgn}
\DeclareMathOperator{\Homeo}{Homeo}
\DeclareMathOperator{\PSL}{PSL}
\DeclareMathOperator{\Char}{char}
\DeclareMathOperator{\Bil}{Bil}
\DeclareMathOperator{\Sym}{Sym}
\DeclareMathOperator{\Skew}{Skew}
\DeclareMathOperator{\Alt}{Alt}
\DeclareMathOperator{\Quad}{Quad}

%Row operations
\newcommand{\elem}[1]{% elementary operations
\xrightarrow{\substack{#1}}%
}

\newcommand{\lelem}[1]{% elementary operations (left alignment)
\xrightarrow{\begin{subarray}{l}#1\end{subarray}}%
}

%SS
\DeclareMathOperator{\supp}{supp}
\DeclareMathOperator{\Var}{Var}

%NT
\DeclareMathOperator{\ord}{ord}

%Alg
\DeclareMathOperator{\Rad}{Rad}
\DeclareMathOperator{\Jac}{Jac}

%Misc
\newcommand{\SL}{{\mathrm{SL}}}
\newcommand{\mobgp}{{\mathrm{PSL}_2(\mathbb{C})}}
\newcommand{\id}{{\mathrm{id}}}
\newcommand{\Mod}{{\mathrm{Mod}}}
\newcommand{\ud}{{\mathrm{d}}}
\newcommand{\Vol}{{\mathrm{Vol}}}
\newcommand{\Area}{{\mathrm{Area}}}
\newcommand{\diam}{{\mathrm{diam}}}
\newcommand{\End}{{\mathrm{End}}}


\newcommand{\reg}{{\mathtt{reg}}}
\newcommand{\geo}{{\mathtt{geo}}}

\newcommand{\tori}{{\mathcal{T}}}
\newcommand{\cpn}{{\mathtt{c}}}
\newcommand{\pat}{{\mathtt{p}}}


\title{Assignment 4}
\author{Jonas Trepiakas}


\begin{document}

\maketitle
    \begin{exercise}[]
        Let $\dim V = n < \infty$ and assume $\Char F \neq 2$.
        Let $B \in \Bil (V)$ be represented by
        a matrix $\left[ B \right] $ with respect to some
        basis. Prove that $B$ is symmetric, respectively skew-symmetric
        if and only if $\left[ B \right] $ has this
        property. Then determine the dimensions of $\Sym(V),
        \Skew (V), \Alt(V)$ and $\Quad(V)$.
    \end{exercise}

    \begin{solution}
        Let $\left\{ x_1, \ldots, x_n \right\} $ be a basis
        for $V$. If $B$ is symmetric then
        $B(u,v) = B(v,u)$ for all $u,v \in V$, so
        in particular, $B(x_i, x_j) = B(x_j, x_i)$.
        Hence
        \[
        \left[ B \right]_{ij}=
        B\left( x_i, x_j \right) =
        B\left( x_j, x_i \right) =
        \left[ B \right]_{ji}
        = \left[ B \right]_{ij}^{t}
        \] 
        showing that the matrix $\left[ B \right] $ is symmetric.
        If $\left[ B \right] $ is symmetric, then
        \[
        B(x_i, x_j) = \left[ B \right]_{ij}
        = \left[ B \right]_{ji}
        = B\left( x_j, x_j \right) 
        \] 
        for all $i,j$,
        so $B$ is symmetric.

        Similarly, if $B$ is
        skew-symmetric, then
        \[
        \left[ B \right]_{ij}
        = B\left( x_i, x_j \right) 
        = - B(x_j , x_i)
        = - \left[ B \right]_{ji}
        = - \left[ B \right]_{ij}^{t}
        \] 
        so $\left[ B \right] $ is skew-symmetric, and
        conversely,
        if $\left[ B \right] $ is skew-symmetric, then
        \[
        B(x_i, x_j) =\left[ B \right]_{ij} =
        - \left[ B \right]_{ji}
        = - B(x_j, x_i)
        \] 
        for all $i,j$.\\
        \linebreak
        Let
        $\left\{ y_1, \ldots, y_n \right\} $ be the dual
        basis for $\left\{ x_1,\ldots,x_n \right\} $.
        Let $w_{ij} = y_i y_j \in \Bil(V)$.
        Then by theorem 4.3, any
        symmetric bilinear form $B$ can be written
        \[
        B = \sum_{i< j} B\left( x_i, x_j \right) \left( w_{ij}+
        w_{ji} \right) + \sum_{i} B\left( x_i, x_i \right) 
        w_{ii}
        \] 
        and similarly, skew symmetric matrices $B$ can be written
        \[
        \sum_{i<j} B\left( x_i, x_j \right) \left( w_{ij}
        - w_{ji} \right) 
        \] 
        (no $w_{ii}$ since  skew-symmetric matrices
        are alternating for $\Char F \neq 2$).
        Hence
        $\mathcal{B} = \left\{ w_{ij}+w_{ji} \right\}_{i<j} \cup 
        \left\{ w_{ii} \right\}_{i}$ spans
        $\Sym (V)$ and 
        $\mathcal{B}' = 
        \left\{ w_{ij} - w_{ji} \right\}_{i<j}$ spans
        $\Skew(V)$.
        Since $\left\{ w_{ij} \right\}$ forms
        a basis for $\Bil(V)$, they are linearly independent,
        so in particular $\mathcal{B}$ and
        $\mathcal{B}'$ are linearly independent since
        each $w_{ij}$ only occurs once in each set, and
        hence form bases for
        $\Sym(V)$ and $\Skew(V)$, respectively.
        Therefore,
        $\dim \Sym(V) =
        1 + 2 + \ldots + (n-1) + n = \frac{(n+1)n}{2}$ while
        $\dim \Skew(V) = 1 + 2 + \ldots + (n-1)
        =  \frac{(n-1)n}{2}$.\\
        \linebreak
        Similarly, we see that
        $\left\{ w_{ij} \right\}_{i \neq j}$ is
        a basis for $\Alt(V)$, so
        $\dim \Alt(V) = n^2 - n$.\\
        \linebreak
        Lastly, since $\Char F \neq 2$, lemma 4.19
        gives an isomorphism $\Sym(V) \cong 
        \Quad (V)$, so
        in particular, $\dim \Quad (V) = \dim \Sym (V)
        = \frac{(n+1)n}{2}$.





    \end{solution}







    %\bibliography{../refs.bib}
\end{document}
