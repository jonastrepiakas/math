\documentclass[reqno]{amsart}
\usepackage{amscd, amssymb, amsmath, amsthm}
\usepackage{graphicx}
\usepackage[colorlinks=true,linkcolor=blue]{hyperref}
\usepackage[utf8]{inputenc}
\usepackage[T1]{fontenc}
\usepackage{textcomp}
\usepackage{babel}
%% for identity function 1:
\usepackage{bbm}
%%For category theory diagrams:
\usepackage{tikz-cd}


\setlength\parindent{0pt}

\pdfsuppresswarningpagegroup=1

\newtheorem{theorem}{Theorem}[section]
\newtheorem{lemma}[theorem]{Lemma}
\newtheorem{proposition}[theorem]{Proposition}
\newtheorem{corollary}[theorem]{Corollary}
\newtheorem{conjecture}[theorem]{Conjecture}

\theoremstyle{definition}
\newtheorem{definition}[theorem]{Definition}
\newtheorem{example}[theorem]{Example}
\newtheorem{exercise}[theorem]{Exercise}
\newtheorem{problem}[theorem]{Problem}
\newtheorem{question}[theorem]{Question}

\theoremstyle{remark}
\newtheorem*{remark}{Remark}
\newtheorem*{note}{Note}
\newtheorem*{solution}{Solution}



%Inequalities
\newcommand{\cycsum}{\sum_{\mathrm{cyc}}}
\newcommand{\symsum}{\sum_{\mathrm{sym}}}
\newcommand{\cycprod}{\prod_{\mathrm{cyc}}}
\newcommand{\symprod}{\prod_{\mathrm{sym}}}

%Linear Algebra

\DeclareMathOperator{\Span}{span}
\DeclareMathOperator{\Ima}{Im}
\DeclareMathOperator{\diag}{diag}
\DeclareMathOperator{\Ker}{Ker}
\DeclareMathOperator{\ob}{ob}
\DeclareMathOperator{\Hom}{Hom}
\DeclareMathOperator{\sk}{sk}
\DeclareMathOperator{\Vect}{Vect}
\DeclareMathOperator{\Set}{Set}
\DeclareMathOperator{\Group}{Group}
\DeclareMathOperator{\Ring}{Ring}
\DeclareMathOperator{\Ab}{Ab}
\DeclareMathOperator{\Top}{Top}
\DeclareMathOperator{\hTop}{hTop}
\DeclareMathOperator{\Htpy}{Htpy}
\DeclareMathOperator{\Cat}{Cat}
\DeclareMathOperator{\CAT}{CAT}
\DeclareMathOperator{\Cone}{Cone}
\DeclareMathOperator{\dom}{dom}
\DeclareMathOperator{\cod}{cod}
\DeclareMathOperator{\Aut}{Aut}
\DeclareMathOperator{\Mat}{Mat}
\DeclareMathOperator{\Fin}{Fin}
\DeclareMathOperator{\rel}{rel}
\DeclareMathOperator{\Int}{Int}
\DeclareMathOperator{\sgn}{sgn}
\DeclareMathOperator{\Homeo}{Homeo}
\DeclareMathOperator{\PSL}{PSL}
\DeclareMathOperator{\Bil}{Bil}


%Row operations
\newcommand{\elem}[1]{% elementary operations
\xrightarrow{\substack{#1}}%
}

\newcommand{\lelem}[1]{% elementary operations (left alignment)
\xrightarrow{\begin{subarray}{l}#1\end{subarray}}%
}

%SS
\DeclareMathOperator{\supp}{supp}
\DeclareMathOperator{\Var}{Var}

%NT
\DeclareMathOperator{\ord}{ord}

%Alg
\DeclareMathOperator{\Rad}{Rad}
\DeclareMathOperator{\Jac}{Jac}

%Misc
\newcommand{\SL}{{\mathrm{SL}}}
\newcommand{\mobgp}{{\mathrm{PSL}_2(\mathbb{C})}}
\newcommand{\id}{{\mathrm{id}}}
\newcommand{\Mod}{{\mathrm{Mod}}}
\newcommand{\ud}{{\mathrm{d}}}
\newcommand{\Vol}{{\mathrm{Vol}}}
\newcommand{\Area}{{\mathrm{Area}}}
\newcommand{\diam}{{\mathrm{diam}}}
\newcommand{\End}{{\mathrm{End}}}


\newcommand{\reg}{{\mathtt{reg}}}
\newcommand{\geo}{{\mathtt{geo}}}

\newcommand{\tori}{{\mathcal{T}}}
\newcommand{\cpn}{{\mathtt{c}}}
\newcommand{\pat}{{\mathtt{p}}}


\title{Assignment 5}

\author{Jonas Trepiakas}


\begin{document}

\maketitle


    \begin{exercise}[]
        Show that $\Hom \left( X \otimes Y, Z \right) 
        \simeq \Bil \left( X, Y; Z \right) $ for all
        vector spaces. In particular, $\left( X \otimes
        Y\right)' \simeq \Bil(X,Y)$.
    \end{exercise}

    \begin{solution}
        Suppose
        $B \in \Bil (X,Y; Z)$. By universality,
        there exists a unique linear map
        $L \colon X \otimes Y \to Z$ such that

        \begin{equation*}
        \begin{tikzcd}
            X \times Y \ar[r, "\otimes"] \ar[dr, "M"']
            & X \otimes Y \ar[d, "L"] \\
                              & Z
        \end{tikzcd}
        \end{equation*}
        commutes, i.e., we can define a map
        $\Bil \left( X, Y ; Z \right) \to 
        \Hom \left( X \otimes Y , Z \right) $ by
        sending $M \mapsto L$.

        Similarly, given some linear map
        $L \in \Hom \left( X \otimes Y, Z \right) $, we have
        \begin{align*}
            L \circ \otimes \left( a_1 x_1 + a_2 x_2, y \right) 
            &= L \left( \left( a_1 x_1 + a_2 x_2  \right) 
            \otimes y\right) 
            = L \left( a_1 x_1 \otimes y +a_2 x_2 \otimes y \right) \\
            &=a_1  L \left(  x_1 \otimes y \right) +
            a_2 L \left( x_1 \otimes y \right) 
        \end{align*}
        so $L \circ \otimes$ is linear in $x$,
        and
        by doing the same for $y$, $L \circ \otimes $ is
        seen to be linear in $y$ as well. Hence
        $L \circ \otimes  $ is bilinear, so
        each $L$ induces a map $M$ as well such that
        the diagram above commutes.

        Furthermore, it is clear that if
        $L \mapsto M$, then $M \mapsto L$ and vice versa,
        so we, in fact, have a bijective correspondence
        between the classes
        $\Hom \left( X \otimes Y, Z \right) $ and
        $\Bil \left( X, Y ; Z \right) $.\\
        \linebreak
        


        It remains to check that the maps
        $M \mapsto L$ and $L \mapsto M$ are linear.

        Suppose we have a bilinear map
        $a_1 M_1 + a_2 M_2$. Let $L_1$ and $L_2$ be
        the linear maps corresponding to $M_1 $ and $M_2$. Then
        \begin{align*}
            \left( a_1 M_1+ a_2 M_2 \right) (x,y)
            &=
            a_1 M_1(x,y) + a_2 M_2(x,y)
            = a_1 L_1 (x \otimes y) + a_2 L_2 \left( x \otimes y
            \right)\\
            &= \left( a_1 L_1 + a_2 L_2
             \right) \circ \otimes (x,y)
        \end{align*}
        so by uniqueness of the universal property,
        $a_1 M_1 + a_2 M_2$ maps to
        $a_1 L_1 + a_2 L_2$ which is indeed linear.
        Now, since the inverse map of a bijective linear map
        is also linear, we get that
        the other map is also linear. Since, as
        we remarked above, the composition of them
        is the identity, we find that
        $\Hom \left( X \otimes Y , Z \right) \simeq
        \Bil \left( X, Y ; Z \right) $.\\

        Choosing $Z = F$, we get
        $\left( X \otimes Y \right) ' \simeq
        \Bil (X,Y)$.
        


    \end{solution}









    %\bibliography{../refs.bib}
\end{document}
