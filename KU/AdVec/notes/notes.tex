\documentclass[reqno]{amsart}
\usepackage{amscd, amssymb, amsmath, amsthm}
\usepackage{graphicx}
\usepackage[colorlinks=true,linkcolor=blue]{hyperref}
\usepackage[utf8]{inputenc}
\usepackage[T1]{fontenc}
\usepackage{textcomp}
\usepackage{babel}
%% for identity function 1:
\usepackage{bbm}
%%For category theory diagrams:
\usepackage{tikz-cd}


\setlength\parindent{0pt}

\pdfsuppresswarningpagegroup=1

\newtheorem{theorem}{Theorem}[section]
\newtheorem{lemma}[theorem]{Lemma}
\newtheorem{proposition}[theorem]{Proposition}
\newtheorem{corollary}[theorem]{Corollary}
\newtheorem{conjecture}[theorem]{Conjecture}

\theoremstyle{definition}
\newtheorem{definition}[theorem]{Definition}
\newtheorem{example}[theorem]{Example}
\newtheorem{exercise}[theorem]{Exercise}
\newtheorem{problem}[theorem]{Problem}
\newtheorem{question}[theorem]{Question}

\theoremstyle{remark}
\newtheorem*{remark}{Remark}
\newtheorem*{note}{Note}
\newtheorem*{solution}{Solution}



%Inequalities
\newcommand{\cycsum}{\sum_{\mathrm{cyc}}}
\newcommand{\symsum}{\sum_{\mathrm{sym}}}
\newcommand{\cycprod}{\prod_{\mathrm{cyc}}}
\newcommand{\symprod}{\prod_{\mathrm{sym}}}

%Linear Algebra

\DeclareMathOperator{\Span}{span}
\DeclareMathOperator{\Ima}{Im}
\DeclareMathOperator{\diag}{diag}
\DeclareMathOperator{\Ker}{Ker}
\DeclareMathOperator{\ob}{ob}
\DeclareMathOperator{\Hom}{Hom}
\DeclareMathOperator{\Mor}{Mor}
\DeclareMathOperator{\sk}{sk}
\DeclareMathOperator{\Vect}{Vect}
\DeclareMathOperator{\Set}{Set}
\DeclareMathOperator{\Group}{Group}
\DeclareMathOperator{\Ring}{Ring}
\DeclareMathOperator{\Ab}{Ab}
\DeclareMathOperator{\Top}{Top}
\DeclareMathOperator{\hTop}{hTop}
\DeclareMathOperator{\Htpy}{Htpy}
\DeclareMathOperator{\Cat}{Cat}
\DeclareMathOperator{\CAT}{CAT}
\DeclareMathOperator{\Cone}{Cone}
\DeclareMathOperator{\dom}{dom}
\DeclareMathOperator{\cod}{cod}
\DeclareMathOperator{\Aut}{Aut}
\DeclareMathOperator{\Mat}{Mat}
\DeclareMathOperator{\Fin}{Fin}
\DeclareMathOperator{\rel}{rel}
\DeclareMathOperator{\Int}{Int}
\DeclareMathOperator{\sgn}{sgn}
\DeclareMathOperator{\Homeo}{Homeo}
\DeclareMathOperator{\SHomeo}{SHomeo}
\DeclareMathOperator{\PSL}{PSL}
\DeclareMathOperator{\Bil}{Bil}
\DeclareMathOperator{\Sym}{Sym}
\DeclareMathOperator{\Skew}{Skew}
\DeclareMathOperator{\Alt}{Alt}
\DeclareMathOperator{\Quad}{Quad}
\DeclareMathOperator{\Sin}{Sin}
\DeclareMathOperator{\Supp}{Supp}
\DeclareMathOperator{\Char}{char}
\DeclareMathOperator{\GL}{GL}


%Row operations
\newcommand{\elem}[1]{% elementary operations
\xrightarrow{\substack{#1}}%
}

\newcommand{\lelem}[1]{% elementary operations (left alignment)
\xrightarrow{\begin{subarray}{l}#1\end{subarray}}%
}

%SS
\DeclareMathOperator{\supp}{supp}
\DeclareMathOperator{\Var}{Var}

%NT
\DeclareMathOperator{\ord}{ord}

%Alg
\DeclareMathOperator{\Rad}{Rad}
\DeclareMathOperator{\Jac}{Jac}

%Misc
\newcommand{\SL}{{\mathrm{SL}}}
\newcommand{\mobgp}{{\mathrm{PSL}_2(\mathbb{C})}}
\newcommand{\id}{{\mathrm{id}}}
\newcommand{\Mod}{{\mathrm{Mod}}}
\newcommand{\PMod}{{\mathrm{PMod}}}
\newcommand{\SMod}{{\mathrm{SMod}}}
\newcommand{\ud}{{\mathrm{d}}}
\newcommand{\Vol}{{\mathrm{Vol}}}
\newcommand{\Area}{{\mathrm{Area}}}
\newcommand{\diam}{{\mathrm{diam}}}
\newcommand{\End}{{\mathrm{End}}}


\newcommand{\reg}{{\mathtt{reg}}}
\newcommand{\geo}{{\mathtt{geo}}}

\newcommand{\tori}{{\mathcal{T}}}
\newcommand{\cpn}{{\mathtt{c}}}
\newcommand{\pat}{{\mathtt{p}}}

\let\Cap\undefined
\newcommand{\Cap}{{\mathcal{C}}ap}
\newcommand{\Push}{{\mathcal{P}}ush}
\newcommand{\Forget}{{\mathcal{F}}orget}





\begin{document}

\section{Glossary for exam}

\begin{itemize}
    \item All subspaces have a complement (thm 2.14)
    \item $A,\tilde{A}\in \Hom (U,V)$ are equivalent if there
        exist $S \in \GL (U)$ and $T \in \GL (V)$ such that
         \begin{equation*}
        \begin{tikzcd}
            U  \ar[r, "A"] \ar[d, "S"] & V \ar[d, "T"] \\
            U \ar[r, "\tilde{A}"] & V
        \end{tikzcd}
        \end{equation*}
    \item $A,\tilde{A} \in \End(V)$ are called similar if
        there exists $T \in \GL (V)$ such that
        $TA = \tilde{A}T$.
    \item \textbf{Theorem 2.22:} Assume $U,V$ fin.dim., then
        $A, \tilde{A}$ are equivalent iff they have the same rank.
    \item For $y \in V'$, the null-space of $y$ has codimension
        one in $V$ and only the non-zero scalar multiples
        of $y$ have the same null-space as $y$. (lemma 3.15 + see 
        thm 3.14)
    \item The dual map is the pullback: if $A \in \Hom (U,V)$, then
        $A' \in \Hom(V',U')$ is defined by
        $A'(y) = y \circ A$.
       \begin{equation*}
        \begin{tikzcd}
            U \ar[r, "A"] \ar[dr, "A'(y)"']& V \ar[d, "y"] \\
              & k
        \end{tikzcd}
        \end{equation*}
        
\end{itemize}


\section*{Chapter 1}

    

\begin{exercise}[12.2]
    Let $U$ and $V$ be normed spaces, and assume that $V$ is complete.
    Show that then $B(U,V)$ is also complete with the operator
    norm.
\end{exercise}

\begin{solution}
    Suppose $B_n$ is a Cauchy sequence
    in $B \left( U,V \right) $, so
    \[
    \forall \varepsilon > 0 \exists N \in \mathbb{N}  \mid 
    \|B_n - B_m\|< \varepsilon, \quad \forall m,n \ge N.
    \] 

    That is, for all $m, n \ge N$,
    \[
    \sup_{\|x\|= 1} \|B_n x - B_m x \| < \varepsilon
    \] 
    For a fixed $n$, this becomes a Cauchy sequence
    in $V$ which thus converges, so we can define
    $Bx = \lim_{n \to \infty} B_n x$.
    We claim that $B$ is a bounded operator too.
    It is clear that it is linear since each $B_n$ is a 
    continuous map. What remains is to show that
    $B$ is bounded. It suffices to show that
    it is bounded on $ S = \left\{ x  \mid \|x \|= 1 \right\} $.
    Suppose it were not bounded and choose
    a sequence $\left( x_n \right) \subset S$ such that
    $\|Bx_n\|> n$.
    Choose $\varepsilon = \frac{1}{2}$ and let
    $N$ be such that for $n, m \ge N$, we have
    \[
    \|B_n - B_m\|< \varepsilon
    \] 
    Then for all $k$
    \[
    \|B_n x_k -  B_m x_k \|< \varepsilon
    \] 
    for all $n \ge N$, so in particular
    \[
    \|Bx_k - B_m x_k\|
    = \lim_{n \to \infty} \|B_n x_k -  B_m x_k \|< \varepsilon
    \] 
    But $B_m$ is bounded, so let $\|B_m\| = R$.
    Choose $M$ such that that for all $k \ge M$, we have
    $\|Bx_k \| \ge \|B_m x_k\|$, then
    \[
    \|B x_k\| - \|B_m x_k\| < \varepsilon
    \] 
    giving
    \[
    \|B x_k\| < \varepsilon + R
    \] 
    contradicting $\|Bx_k\| \to \infty$.

\end{solution}

    \begin{exercise}[12.3]
        Let $A \in B(V)$ for a complete normed space
        $V$. Prove
        \begin{enumerate}
            \item If $\|A\| < 1$ then
                $\sum_{k=0}^{\infty} A^{k}$ converges
                in $B(V)$ to an inverse of $I - A$.
            \item If $B \in B(V)$ is invertible and
                $\|A\| < \frac{1}{\|B^{-1}\|}$ then
                $B - A$ is invertible.
            \item The set of invertible bounded operators is an
                open subset of $B(V)$.
        \end{enumerate}
        (Here invertible means there is a bounded inverse)
    \end{exercise}

    \begin{solution}
        (1) geometric series.

        (2) $B - A = B \left( 1 - \frac{A}{B} \right) $.
        Now $\|\frac{A}{B}\| \le 
        \|A\| \|B^{-1}\| < 1$, so
        by (1),  $1 - \frac{A}{B}$ has inverse
        $\sum_{k=0}^{\infty} \left( \frac{A}{B} \right)^{k}$.
        But then $B - A$ is a composition of invertible
        maps hence invertible since $\text{GL} (V)$ is a group.

        (3) Suppose 
        $A \in B\left( B, \frac{1}{\|B^{-1}\|} \right) $, so
        $\| B - A\| < \frac{1}{\|B^{-1}\|}$. By (2), 
        $B - (B-A) = A$ is then invertible. Hence
        $B \left( B, \frac{1}{\|B^{-1}\|} \right) $ is an
        open neighborhood of $B$ in $B(V)$ consisting
        of invertible maps. Thus the
        set of invertible maps is open in $B(V)$.
    \end{solution}

    \begin{exercise}[12.6]
        Let $S \in \End \left( \ell^2 \right) $ denote the
        right shift taking the sequence
        $\left( x_1, x_2, \ldots \right) $  to
        $\left( 0, x_1, x_2, \ldots \right) $. Show it
        is bounded and determine the operator
        norm $\|S\|$. Find also the adjoint $S^{*}$, and
        verify that $S^{*}S = I$ but
        $S S^{*} \neq I$.
    \end{exercise}

    \begin{solution}
        Recall that we are dealing with the norm
        $\| \left( x_1, x_2, \ldots \right) \|^2
        = \sum_{k=1}^{\infty} \left| x_k \right|^2 $.
        But indeed then if
        $\|\left( x_1, \ldots \right) \| = 1$, then
        \[
        \|S \left( x_1, \ldots \right) 
        \|^2 = 
        \| \left( 0, x_1, x_2, \ldots \right) \|^2
        = \sum_{k=1}^{\infty} \left| x_k \right|^2
        = 1
        \] 
        so, in fact, $S$ preserves the norm.
        But then
        since $\|S x\| = \|x\|$ for all $x$ by linearity, we have
        $\|S\| = 1$.
        Now, the inner product is
        $\left<x,y \right> = 
        \sum_k x_k \overline{y_k}$. Then
        \[
        \left< Sx,y \right>
        = \sum_{k=2}^{\infty} x_{k} y_{k-1} =
        \left< x, S^{*}y \right>
        \] 
        if we define $S^{*}\left( y_1, y_2, \ldots \right) 
        = \left( y_2, y_3 , \ldots \right) $.
        We then indeed get
        $S^{*}S = I$ clearly, but
        $S S^{*} \left( x_1,x_2,  \ldots \right) 
        = \left( 0, x_2, x_3, \ldots \right) $.
    \end{solution}

    \begin{exercise}[12.8]
        Show $\|Ax \pm ix\|^2 = 
        \|Ax\|^2 + \|x\|^2 $ for $A$ Hermitian
        and $\dim V < \infty$. Then show
        $A \pm i I$ is invertible and
        $\left( A - i I \right) \left( A+iI \right)^{-1}$ 
        unitary.
    \end{exercise}

    \begin{solution}

        \begin{align*}
        \left<Ax \pm ix, Ax \pm ix \right> 
        &=
        \|Ax\|^2 + \left< Ax , \pm ix \right>
        + \left< \pm ix, Ax \right>
        + \left<\pm ix, \pm ix \right>
        \\
        &= \|Ax\|^2 + \|x\|^2
        \mp i \left<Ax , x \right>
        \pm i \left< x, Ax \right>\\
        &= \|Ax\|^2 + \|x\|^2
        \mp i \left<Ax,x \right>
        \pm i \left<Ax,x \right>\\
        &= \|Ax\|^2 + \|x\|^2.
        \end{align*}

        Now, if $A \pm i I$ were not invertible, it
        would not be injective, so for 
        $x \neq 0$, we would get
        \[
        0 = \|Ax \pm i x\|^2
        = \|Ax\|^2 + \|x\|^2
        \] 
        but $\|x\|^2 > 0$ and
        $\|Ax\|^2 \ge 0$, so this gives a contradiction.

        Lastly, what is the adjoint of
        $\left( A-iI \right) \left( A+iI \right)^{-1}$?
        Well, $\left( A-iI \right)^{*}
        = A + iI$ by the rules on page 70. Hence
        the expression is of the form
        $X^{*} X^{-1}$ which has adjoint
        $\left( X^{-1} \right)^{*} X$. Then
        $\left( X^{-1} \right)^{*} X X^{*} X^{-1}$.


        Now, since $A$ is self-adjoint, it is in particular
        normal, so $A + iI$ is normal and
        hence orthogonally diagonable. Writing
        $A+ iI = \sum \lambda E_{\lambda}$, we get
        $(A+iI)^{*} = \sum \overline{\lambda} E_{\lambda}$, so
        $A+iI$ and $A-iI$ commute. Hence we get
         $X X^{*} = X^{*} X$, and the expression above becomes
         the identity.
    \end{solution}



    \begin{exercise}[12.4]
        Give a simple proof of the Hahn-Banach theorem for
        a continuous linear form on a closed subspace of
        a Hilbert space.
    \end{exercise}

    \begin{solution}
        Let $V$ be a Hilbert space and let
        $U \subset V$ be a closed subspace. Then
        $V = U \oplus U^{\perp}$. Let
        $\mathcal{B}$ be a basis for
        $U$ and extend it to a basis 
        $\mathcal{A}$ for $V$. Take the duals
        $\mathcal{B}^{'}$ and $\mathcal{A}^{'}$.
        For $z \in U^{*}$ we can write
        $z = \sum_{y_i' \in \mathcal{B}'}
        a_i y_i$. Then $z$ can also be considered a linear
        form on $V$ by letting the coefficient for
        $y_i \in \mathcal{A}'$ be $0$ if
        $y_i \not\in \mathcal{B}'$ and
        $a_i$ otherwise. The restrictions are clearly
        the same. 
        By the Riesz-Fréchet representation theorem,
        since $U$ is a closed subspace of a Hilbert space, it
        is also a Hilbert space, so by continuity of $z$, there
        exists $u \in U$ such that
        $z(x) = \left< x,u \right>$ for all $x \in U$ and
        such that $\|z\| = \|u\|$.
        But since  $z|_{U^{\perp}} = 0$, we also
        have $z(x) = \left< x,u \right>$ for all $x \in V$, so
        by the Riesz-Fréchet theorem, $\|z\| = \|u\|$ over $V$ as well.


    \end{solution}


    \begin{exercise}[13.1]
        Prove $\rho (A+B) \le \rho (A) + \rho(B)$ if
        $A$ and $B$ are normal. Prove it for general
        $A,B \in \End (V)$, now assuming they commute.
        Show the inequality can fail in general.
    \end{exercise}

    \begin{solution}
        
        If $A$ and $B$ are normal, then
        they are orthogonally diagonable with respect
        to the associated inner product, hence
        $\rho (A) = \|A\|$ and
        $\rho(B) = \|B\|$.
        Now, in general, we have
        $\rho(X) \le \|X\|$, we we get
        \[
        \rho(A+B) \le \|A+B\|
        \le \|A\| + \|B\|
        = \rho(A) + \rho(B).
        \] 


        If $A$ and $B$ commute, then
        \begin{align*}
        \rho(A+B) = 
        \lim_{k\to \infty}
        \|\left( A+B \right)^{k}\|^{\frac{1}{k}}
        &=
        \lim_{k\to \infty}
        \| \sum_{i=0}^{k} \begin{pmatrix} k\\i \end{pmatrix} 
        A^{i} B^{k-i} \|^{\frac{1}{k}}\\
        &\le \lim_{k\to \infty}
        \left| \sum_{i=0}^{k} 
        \begin{pmatrix} k\\i \end{pmatrix} 
        \|A\|^{i} \|B\|^{k-i} \right|^{\frac{1}{k}}\\
        &= \lim_{k\to \infty} 
         \|A\|+\|B\|\\
        &= \|A\| + \|B\|\\
        &= \lim_{k\to \infty} \|A^{k}\|^{\frac{1}{k}} +
        \lim_{k\to \infty} \|B^{k}\|^{\frac{1}{k}}
        \end{align*}
        where the last equality follows from
        $\|X^{k}\| = \|X\|^{k}$ when $X$ is diagonable (by
        the proof of lemma 13.4).




        To show that it can fail in general, note that
        for $\left[ A \right] =
        \begin{pmatrix} 1 & 0 \\ 1 & 1\end{pmatrix} $ and
        $\left[ B \right] =
        \begin{pmatrix} 1 & 1\\
        0 & 1\end{pmatrix} $, we have
        that the eigenvalues of both are precisely
        $1$, hence $\rho(A) + \rho(B) = 2$, while
            $A+B = \begin{pmatrix} 2 & 1\\
            1 & 2\end{pmatrix} $ which has characteristic
            polynomial
            $(x-3)(x-1)$ and thus
             $3$ as an eigenvalue.
    \end{solution}





    \begin{exercise}[13.2]
        Let $F = \mathbb{C}$. Find a counterexample to the
        statement:
        $\rho \left( p \left( A \right)  \right) 
        = p \left( \rho(A) \right) $ for all polynomials
        $p$, where $\rho$ is the spectral radius. 
    \end{exercise}

    \begin{solution}

        Consider $p(x) = ix - i$ and
        $A = -I$. So
        $p(A) = \begin{pmatrix} -2i &0 \\0 & -2i \end{pmatrix} $ 
        which has spectral radius
        $2$. However,
        $-I$ has spectral radius $1$ and
        $p(1) = 0$.
    \end{solution}

    \begin{exercise}[13.3]
        Show $\rho \left( A^{*}A \right) 
        = \|A^{*}A\| = \|A\|^2$ for the
        operator
        norm of an inner product.
    \end{exercise}


    \begin{solution}
        The first equality holds when
        the matrix is orthogonally diagonable.
        But  $A^{*}A$ is self-adjoint, hence
        normal hence orthogonally diagonable.

        The latter equality holds since
        \[
        \|A^{*}A\| =
        \sup_{\|x\|=1} \left<A^{*}Ax, x \right>
        = \sup_{\|x\|=1} \|Ax\|^2
        = \|A\|^2
        \] 
    \end{solution}



























    %\bibliography{../refs.bib}
\end{document}
