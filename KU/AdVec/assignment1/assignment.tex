\documentclass[reqno]{amsart}
\usepackage{amscd, amssymb, amsmath, amsthm}
\usepackage{graphicx}
\usepackage[colorlinks=true,linkcolor=blue]{hyperref}
\usepackage[utf8]{inputenc}
\usepackage[T1]{fontenc}
\usepackage{textcomp}
\usepackage{babel}
%% for identity function 1:
\usepackage{bbm}
%%For category theory diagrams:
\usepackage{tikz-cd}


\setlength\parindent{0pt}

\pdfsuppresswarningpagegroup=1

\newtheorem{theorem}{Theorem}[section]
\newtheorem{lemma}[theorem]{Lemma}
\newtheorem{proposition}[theorem]{Proposition}
\newtheorem{corollary}[theorem]{Corollary}
\newtheorem{conjecture}[theorem]{Conjecture}

\theoremstyle{definition}
\newtheorem{definition}[theorem]{Definition}
\newtheorem{example}[theorem]{Example}
\newtheorem{exercise}[theorem]{Exercise}
\newtheorem{problem}[theorem]{Problem}
\newtheorem{question}[theorem]{Question}

\theoremstyle{remark}
\newtheorem*{remark}{Remark}
\newtheorem*{note}{Note}
\newtheorem*{solution}{Solution}



%Inequalities
\newcommand{\cycsum}{\sum_{\mathrm{cyc}}}
\newcommand{\symsum}{\sum_{\mathrm{sym}}}
\newcommand{\cycprod}{\prod_{\mathrm{cyc}}}
\newcommand{\symprod}{\prod_{\mathrm{sym}}}

%Linear Algebra

\DeclareMathOperator{\Span}{span}
\DeclareMathOperator{\Ima}{Im}
\DeclareMathOperator{\diag}{diag}
\DeclareMathOperator{\Ker}{Ker}
\DeclareMathOperator{\ob}{ob}
\DeclareMathOperator{\Hom}{Hom}
\DeclareMathOperator{\sk}{sk}
\DeclareMathOperator{\Vect}{Vect}
\DeclareMathOperator{\Set}{Set}
\DeclareMathOperator{\Group}{Group}
\DeclareMathOperator{\Ring}{Ring}
\DeclareMathOperator{\Ab}{Ab}
\DeclareMathOperator{\Top}{Top}
\DeclareMathOperator{\hTop}{hTop}
\DeclareMathOperator{\Htpy}{Htpy}
\DeclareMathOperator{\Cat}{Cat}
\DeclareMathOperator{\CAT}{CAT}
\DeclareMathOperator{\Cone}{Cone}
\DeclareMathOperator{\dom}{dom}
\DeclareMathOperator{\cod}{cod}
\DeclareMathOperator{\Aut}{Aut}
\DeclareMathOperator{\Mat}{Mat}
\DeclareMathOperator{\Fin}{Fin}
\DeclareMathOperator{\rel}{rel}
\DeclareMathOperator{\Int}{Int}
\DeclareMathOperator{\sgn}{sgn}
\DeclareMathOperator{\Homeo}{Homeo}

%Row operations
\newcommand{\elem}[1]{% elementary operations
\xrightarrow{\substack{#1}}%
}

\newcommand{\lelem}[1]{% elementary operations (left alignment)
\xrightarrow{\begin{subarray}{l}#1\end{subarray}}%
}

%SS
\DeclareMathOperator{\supp}{supp}
\DeclareMathOperator{\Var}{Var}

%NT
\DeclareMathOperator{\ord}{ord}

%Alg
\DeclareMathOperator{\Rad}{Rad}
\DeclareMathOperator{\Jac}{Jac}

%Misc
\newcommand{\SL}{{\mathrm{SL}}}
\newcommand{\mobgp}{{\mathrm{PSL}_2(\mathbb{C})}}
\newcommand{\id}{{\mathrm{id}}}
\newcommand{\Mod}{{\mathrm{Mod}}}
\newcommand{\ud}{{\mathrm{d}}}
\newcommand{\Vol}{{\mathrm{Vol}}}
\newcommand{\Area}{{\mathrm{Area}}}
\newcommand{\diam}{{\mathrm{diam}}}
\newcommand{\End}{{\mathrm{End}}}


\newcommand{\reg}{{\mathtt{reg}}}
\newcommand{\geo}{{\mathtt{geo}}}

\newcommand{\tori}{{\mathcal{T}}}
\newcommand{\cpn}{{\mathtt{c}}}
\newcommand{\pat}{{\mathtt{p}}}




\begin{document}
    \begin{exercise}[]
        Find $\dim \left\{ A \in \End (V)  \mid A (W) \subset W
        \right\} $ for a subspace $W \subset V$.
    \end{exercise}

    \begin{solution}
        $ $ \newline
        Suppose $V$ is finite-dimensional.
        Then $W$ is also finite dimensional, so
        choose a basis $\left\{ x_1, \ldots, x_k \right\} $ for
        $W$ and extend it to a basis $\left\{ x_1,
        \ldots, x_n \right\} $ for $V$.
        Then
        any $A \in U = \left\{ A \in \End(V)  \mid A(W) \subset W
        \right\} $,
        can by
        theorem 2.28 be written as
        \[
        A = \sum_{i,j} \alpha_{ij} E_{ij}
        \] 
        where $i,j$ run over $ \left\{ 1, \ldots, n := \dim V 
        \right\} $ 
        and $E_{ij}(x_k) = \delta_{jk} x_i$.
        Since $A (W) \subset W$, we get by uniqueness of
        linear combinations (lemma 1.10), that
        $E_{ij} (x_i) \in \Span \left( x_1, \ldots, x_k \right) $ 
        for $i \in \left\{ 1,\ldots, k \right\} $ which is
        equivalent to $\alpha_{ij} = 0$ whenever
        $j \in \left\{ 1, \ldots, k \right\} $ and
        $i \in \left\{ k+1, \ldots, n \right\} $. This is the
        only requirement for $A(W)$ to be contained in $W$, so any
        $A$ of this form is also in
        $U$. Hence a basis for $U$ is
        all $E_{ij}$ such that $(i,j) \not\in 
        \left\{ k+1, \ldots,n \right\} \times 
        \left\{ 1, \ldots, k \right\} $ which has
        $n^2 - (n-k) k = n^2 - nk + k^2 $ elements, so
        $\dim U = n^2 - (n-k) k$.\\
        \linebreak
        For $V$ infinite dimensional, $\dim U = \infty$:\\
        \linebreak
        Let $\left\{ v_{\alpha} \right\}_{\alpha \in I} $ be some
        basis for $W$ and extend it to a basis 
        $\left\{ v_{\alpha} \right\}_{\alpha \in I \cup J}$ for
        $V$. If $\left| I \right| = \infty$, then
        define a map $\varphi_{\alpha, \beta}  \colon V \to V$ by
        $\varphi_{\alpha, \beta}  (v_{\alpha}) = v_{\beta}$ for
        $\alpha, \beta \in I$ distinct, and
        the zero map on the remaining basis elements.
        There are infinitely many 
        such maps and clearly, each one is in $U$. Suppose
        $0 = \sum_{i,j} c_{ij} \varphi_{\alpha_i, \beta_j}$ is some
        linear combination. Then
        applying this on $v_{\beta_j}$, we get
        $0 = \sum_{i} c_{ij} v_{\alpha_i} $, so by
        linear independence, we get $c_{ij} = 0$ for all $i$.
        Since $j$ was arbitrarily chosen in the linear combination,
        we get $c_{ij}$ for all $i$ and $j$ occurring in the sum.
        Hence $\left\{ \varphi_{\alpha, \beta} \right\}_{\alpha,
        \beta \in I}$ is a linearly independent subset of
        $U$, so $\dim U = \infty$.\\
        \linebreak
        If $\left| I \right| = n < \infty$ is finite,
        then $J$ is infinite. We can do the above construction
        again to get maps $\varphi_{\alpha,\beta} \colon
        V \to V$ but now for $\alpha,\beta \in J$ distinct.
        In this case, they all map $W$ to $0$ which is indeed
        in $W$, so they are in $U$. Completely equivalently to the
        above, we see again that they are linearly independent, so
        $\dim U = \infty$.
    \end{solution}
    

    %\bibliography{../refs.bib}
\end{document}
