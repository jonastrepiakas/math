\documentclass[reqno]{amsart}
\usepackage{amscd, amssymb, amsmath, amsthm}
\usepackage{graphicx}
\usepackage[colorlinks=true,linkcolor=blue]{hyperref}
\usepackage[utf8]{inputenc}
\usepackage[T1]{fontenc}
\usepackage{textcomp}
\usepackage{babel}
%% for identity function 1:
\usepackage{bbm}
%%For category theory diagrams:
\usepackage{tikz-cd}


\setlength\parindent{0pt}

\pdfsuppresswarningpagegroup=1

\newtheorem{theorem}{Theorem}[section]
\newtheorem{lemma}[theorem]{Lemma}
\newtheorem{proposition}[theorem]{Proposition}
\newtheorem{corollary}[theorem]{Corollary}
\newtheorem{conjecture}[theorem]{Conjecture}

\theoremstyle{definition}
\newtheorem{definition}[theorem]{Definition}
\newtheorem{example}[theorem]{Example}
\newtheorem{exercise}[theorem]{Exercise}
\newtheorem{problem}[theorem]{Problem}
\newtheorem{question}[theorem]{Question}

\theoremstyle{remark}
\newtheorem*{remark}{Remark}
\newtheorem*{note}{Note}
\newtheorem*{solution}{Solution}



%Inequalities
\newcommand{\cycsum}{\sum_{\mathrm{cyc}}}
\newcommand{\symsum}{\sum_{\mathrm{sym}}}
\newcommand{\cycprod}{\prod_{\mathrm{cyc}}}
\newcommand{\symprod}{\prod_{\mathrm{sym}}}

%Linear Algebra

\DeclareMathOperator{\Span}{span}
\DeclareMathOperator{\Ima}{Im}
\DeclareMathOperator{\diag}{diag}
\DeclareMathOperator{\Ker}{Ker}
\DeclareMathOperator{\ob}{ob}
\DeclareMathOperator{\Hom}{Hom}
\DeclareMathOperator{\sk}{sk}
\DeclareMathOperator{\Vect}{Vect}
\DeclareMathOperator{\Set}{Set}
\DeclareMathOperator{\Group}{Group}
\DeclareMathOperator{\Ring}{Ring}
\DeclareMathOperator{\Ab}{Ab}
\DeclareMathOperator{\Top}{Top}
\DeclareMathOperator{\hTop}{hTop}
\DeclareMathOperator{\Htpy}{Htpy}
\DeclareMathOperator{\Cat}{Cat}
\DeclareMathOperator{\CAT}{CAT}
\DeclareMathOperator{\Cone}{Cone}
\DeclareMathOperator{\dom}{dom}
\DeclareMathOperator{\cod}{cod}
\DeclareMathOperator{\Aut}{Aut}
\DeclareMathOperator{\Mat}{Mat}
\DeclareMathOperator{\Fin}{Fin}
\DeclareMathOperator{\rel}{rel}
\DeclareMathOperator{\Int}{Int}
\DeclareMathOperator{\sgn}{sgn}
\DeclareMathOperator{\Homeo}{Homeo}
\DeclareMathOperator{\PSL}{PSL}
\DeclareMathOperator{\Bil}{Bil}
\DeclareMathOperator{\Sym}{Sym}
\DeclareMathOperator{\Skew}{Skew}
\DeclareMathOperator{\Alt}{Alt}
\DeclareMathOperator{\Quad}{Quad}


%Row operations
\newcommand{\elem}[1]{% elementary operations
\xrightarrow{\substack{#1}}%
}

\newcommand{\lelem}[1]{% elementary operations (left alignment)
\xrightarrow{\begin{subarray}{l}#1\end{subarray}}%
}

%SS
\DeclareMathOperator{\supp}{supp}
\DeclareMathOperator{\Var}{Var}

%NT
\DeclareMathOperator{\ord}{ord}

%Alg
\DeclareMathOperator{\Rad}{Rad}
\DeclareMathOperator{\Jac}{Jac}

%Misc
\newcommand{\SL}{{\mathrm{SL}}}
\newcommand{\mobgp}{{\mathrm{PSL}_2(\mathbb{C})}}
\newcommand{\id}{{\mathrm{id}}}
\newcommand{\Mod}{{\mathrm{Mod}}}
\newcommand{\ud}{{\mathrm{d}}}
\newcommand{\Vol}{{\mathrm{Vol}}}
\newcommand{\Area}{{\mathrm{Area}}}
\newcommand{\diam}{{\mathrm{diam}}}
\newcommand{\End}{{\mathrm{End}}}


\newcommand{\reg}{{\mathtt{reg}}}
\newcommand{\geo}{{\mathtt{geo}}}

\newcommand{\tori}{{\mathcal{T}}}
\newcommand{\cpn}{{\mathtt{c}}}
\newcommand{\pat}{{\mathtt{p}}}


\title{Assignment 6}
\author{Jonas Trepiakas}


\begin{document}

\maketitle

    \begin{exercise}[]
        Assume $A \in \End (F^2)$ is diagonalizable with
        distinct non-zero eigenvalues $\lambda $ and
        $\mu$. Express $A^{-1}$ as a polynomial of $A$.
    \end{exercise}

    \begin{solution}
        Firstly, by corollary 6.18, since
        $\dim F^2 = 2$, $\sigma (A) = \left\{ \lambda, \mu \right\} $.
        Now, since $A$ is diagonalizable, we have
        $F^2 = V_{\lambda} \oplus V_{\mu} 
        $.
        So
        choosing some non-zero $x_{\lambda} \in V_{\lambda}$ 
        and $x_{\mu} \in V_{\mu}$, we have that with respect
        to the basis $\left\{ x_{\lambda}, x_{\mu} \right\} $,
        the matrix for $A$ is
        \[
        \left[ A \right] =
        \begin{pmatrix} \lambda & 0 \\ 0 & \mu \end{pmatrix} 
        \] 
        The inverse of this matrix is
        \[
        \left[ A \right]^{-1} = 
        \begin{pmatrix} \lambda^{-1} & 0 \\ 0 & \mu^{-1} \end{pmatrix}. 
        \] 
        That is
        $A^{-1} = \lambda^{-1} E_{\lambda} +
        \mu^{-1} E_{\mu}$.
        But $E_{\lambda} = p_{\lambda}(A)
        = \frac{A -\mu}{\lambda - \mu}$ 
        and
        $E_{\mu} = p_{\mu}(A)
        = \frac{A - \lambda}{\mu - \lambda}$, so
        \begin{align*}
            A^{-1} 
            &=\lambda^{-1} \frac{A - \mu}{\lambda - \mu} +
            \mu^{-1} \frac{A - \lambda}{\mu - \lambda} \\
            &= \frac{\lambda^{-1} - \mu^{-1}}{\lambda - \mu}A
            - \frac{\lambda^{-1} \mu - \mu^{-1} \lambda }{\lambda
            - \mu}.
        \end{align*}
    \end{solution}



















    %\bibliography{../refs.bib}
\end{document}
