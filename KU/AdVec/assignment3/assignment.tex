\documentclass[reqno]{amsart}
\usepackage{amscd, amssymb, amsmath, amsthm}
\usepackage{graphicx}
\usepackage[colorlinks=true,linkcolor=blue]{hyperref}
\usepackage[utf8]{inputenc}
\usepackage[T1]{fontenc}
\usepackage{textcomp}
\usepackage{babel}
%% for identity function 1:
\usepackage{bbm}
%%For category theory diagrams:
\usepackage{tikz-cd}


\setlength\parindent{0pt}

\pdfsuppresswarningpagegroup=1

\newtheorem{theorem}{Theorem}[section]
\newtheorem{lemma}[theorem]{Lemma}
\newtheorem{proposition}[theorem]{Proposition}
\newtheorem{corollary}[theorem]{Corollary}
\newtheorem{conjecture}[theorem]{Conjecture}

\theoremstyle{definition}
\newtheorem{definition}[theorem]{Definition}
\newtheorem{example}[theorem]{Example}
\newtheorem{exercise}[theorem]{Exercise}
\newtheorem{problem}[theorem]{Problem}
\newtheorem{question}[theorem]{Question}

\theoremstyle{remark}
\newtheorem*{remark}{Remark}
\newtheorem*{note}{Note}
\newtheorem*{solution}{Solution}



%Inequalities
\newcommand{\cycsum}{\sum_{\mathrm{cyc}}}
\newcommand{\symsum}{\sum_{\mathrm{sym}}}
\newcommand{\cycprod}{\prod_{\mathrm{cyc}}}
\newcommand{\symprod}{\prod_{\mathrm{sym}}}

%Linear Algebra

\DeclareMathOperator{\Span}{span}
\DeclareMathOperator{\Ima}{Im}
\DeclareMathOperator{\diag}{diag}
\DeclareMathOperator{\Ker}{Ker}
\DeclareMathOperator{\ob}{ob}
\DeclareMathOperator{\Hom}{Hom}
\DeclareMathOperator{\sk}{sk}
\DeclareMathOperator{\Vect}{Vect}
\DeclareMathOperator{\Set}{Set}
\DeclareMathOperator{\Group}{Group}
\DeclareMathOperator{\Ring}{Ring}
\DeclareMathOperator{\Ab}{Ab}
\DeclareMathOperator{\Top}{Top}
\DeclareMathOperator{\hTop}{hTop}
\DeclareMathOperator{\Htpy}{Htpy}
\DeclareMathOperator{\Cat}{Cat}
\DeclareMathOperator{\CAT}{CAT}
\DeclareMathOperator{\Cone}{Cone}
\DeclareMathOperator{\dom}{dom}
\DeclareMathOperator{\cod}{cod}
\DeclareMathOperator{\Aut}{Aut}
\DeclareMathOperator{\Mat}{Mat}
\DeclareMathOperator{\Fin}{Fin}
\DeclareMathOperator{\rel}{rel}
\DeclareMathOperator{\Int}{Int}
\DeclareMathOperator{\sgn}{sgn}
\DeclareMathOperator{\Homeo}{Homeo}

%Row operations
\newcommand{\elem}[1]{% elementary operations
\xrightarrow{\substack{#1}}%
}

\newcommand{\lelem}[1]{% elementary operations (left alignment)
\xrightarrow{\begin{subarray}{l}#1\end{subarray}}%
}

%SS
\DeclareMathOperator{\supp}{supp}
\DeclareMathOperator{\Var}{Var}

%NT
\DeclareMathOperator{\ord}{ord}

%Alg
\DeclareMathOperator{\Rad}{Rad}
\DeclareMathOperator{\Jac}{Jac}

%Misc
\newcommand{\SL}{{\mathrm{SL}}}
\newcommand{\mobgp}{{\mathrm{PSL}_2(\mathbb{C})}}
\newcommand{\id}{{\mathrm{id}}}
\newcommand{\Mod}{{\mathrm{Mod}}}
\newcommand{\ud}{{\mathrm{d}}}
\newcommand{\Vol}{{\mathrm{Vol}}}
\newcommand{\Area}{{\mathrm{Area}}}
\newcommand{\diam}{{\mathrm{diam}}}
\newcommand{\End}{{\mathrm{End}}}


\newcommand{\reg}{{\mathtt{reg}}}
\newcommand{\geo}{{\mathtt{geo}}}

\newcommand{\tori}{{\mathcal{T}}}
\newcommand{\cpn}{{\mathtt{c}}}
\newcommand{\pat}{{\mathtt{p}}}

\title{Assignment 3}
\author{Jonas Trepiakas}


\begin{document}

\maketitle

Suppose firstly that $y \in W^{\circ} \cap U^{\circ}$.
Since $V = U \oplus W$, we can write an arbitrary
$x \in V$ as $u + w$ uniquely, and then
$y (x) = y(u+w) = y(u) + y(w) = 0 + 0 = 0$ by assumption on
$y$ annihilating $W$ and $U$. But then
taking the contraposition of lemma 3.4.(1), we get
$y = 0$, so $U^{\circ} \cap W^{\circ} = \left\{ 0 \right\} $,
hence $U^{\circ} \oplus W^{\circ}$ is indeed a direct sum.

The inclusion $U^{\circ} \oplus W^{\circ} \subset V'$ 
follows by definition, so
suppose conversely that $y \in V'$. Then we can
define linear functionals $y_{U}\in U^{\circ}$ and $ y_{W}
\in W^{\circ}$ as follows:
$y_U (u+w) = y(w)$ and  $y_W (u+w) = y(u)$ for
$u \in U$ and $w \in W$.

There is no problem of being well defined since
if $u + w = u' + w'$ then
$u-u' = w-w' \in U \cap W = \left\{ 0 \right\} $.

Since
$y_U \left( \lambda \left( u+w \right) + \left( 
u'+w' \right)  \right) 
= y_U \left( \lambda u + u' + \lambda w + w' \right) 
= y \left( \lambda w + w' \right) 
= \lambda y(w) + y(w') = 
\lambda y_U (u+w) + y_U (u'+w')$, we
find that $y_U$ is linear, and repeating for $y_W$, we
find that $y_W$ is also linear. So $y_U, y_W
\in V'$.

Furthermore, if $u \in U$, then
the unique way of writing it as a direct sum of an element from
$U$ and an element from $W$ is $u + 0$, so
$y_U (u) = y_U(u+0) = y(0) = 0$ since $y$ is linear.
Hence $y_U \in U^{\circ}$, and similarly, $y_W \in W^{\circ}$.

Lastly, for all $x \in V$, write
$x = u+w$ for $u \in U$ and $w \in W$. Then
 $y (x) = y(u+w) =y(u) + y(w) = y_W(u+w) + y_U (u+w)
 = \left( y_U + y_W \right) (x)$, hence
 $y = y_U + y_W$, so
 $y \in U^{\circ} \oplus W^{\circ}$ giving us
 the other inclusion.



    %\bibliography{../refs.bib}
\end{document}
