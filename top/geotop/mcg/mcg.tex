\documentclass[reqno]{amsart}
\usepackage{amscd, amssymb, amsmath, amsthm}
\usepackage{graphicx}
\usepackage[colorlinks=true,linkcolor=blue]{hyperref}
\usepackage[utf8]{inputenc}
\usepackage[T1]{fontenc}
\usepackage{textcomp}
\usepackage{babel}
%% for identity function 1:
\usepackage{bbm}
%%For category theory diagrams:
\usepackage{tikz-cd}
\usepackage{enumitem}


\setlength\parindent{0pt}

\pdfsuppresswarningpagegroup=1

\newtheorem{theorem}{Theorem}[section]
\newtheorem{lemma}[theorem]{Lemma}
\newtheorem{proposition}[theorem]{Proposition}
\newtheorem{corollary}[theorem]{Corollary}
\newtheorem{conjecture}[theorem]{Conjecture}

\theoremstyle{definition}
\newtheorem{definition}[theorem]{Definition}
\newtheorem{example}[theorem]{Example}
\newtheorem{exercise}[theorem]{Exercise}
\newtheorem{problem}[theorem]{Problem}
\newtheorem{question}[theorem]{Question}

\theoremstyle{remark}
\newtheorem*{remark}{Remark}
\newtheorem*{note}{Note}
\newtheorem*{solution}{Solution}



%Inequalities
\newcommand{\cycsum}{\sum_{\mathrm{cyc}}}
\newcommand{\symsum}{\sum_{\mathrm{sym}}}
\newcommand{\cycprod}{\prod_{\mathrm{cyc}}}
\newcommand{\symprod}{\prod_{\mathrm{sym}}}

%Linear Algebra

\DeclareMathOperator{\Span}{span}
\DeclareMathOperator{\Ima}{Im}
\DeclareMathOperator{\diag}{diag}
\DeclareMathOperator{\Ker}{Ker}
\DeclareMathOperator{\ob}{ob}
\DeclareMathOperator{\Hom}{Hom}
\DeclareMathOperator{\sk}{sk}
\DeclareMathOperator{\Vect}{Vect}
\DeclareMathOperator{\Set}{Set}
\DeclareMathOperator{\Group}{Group}
\DeclareMathOperator{\Ring}{Ring}
\DeclareMathOperator{\Ab}{Ab}
\DeclareMathOperator{\Top}{Top}
\DeclareMathOperator{\hTop}{hTop}
\DeclareMathOperator{\Htpy}{Htpy}
\DeclareMathOperator{\Cat}{Cat}
\DeclareMathOperator{\CAT}{CAT}
\DeclareMathOperator{\Cone}{Cone}
\DeclareMathOperator{\dom}{dom}
\DeclareMathOperator{\cod}{cod}
\DeclareMathOperator{\Aut}{Aut}
\DeclareMathOperator{\Mat}{Mat}
\DeclareMathOperator{\Fin}{Fin}
\DeclareMathOperator{\rel}{rel}
\DeclareMathOperator{\Int}{Int}
\DeclareMathOperator{\sgn}{sgn}
\DeclareMathOperator{\PSL}{PSL}

%Row operations
\newcommand{\elem}[1]{% elementary operations
\xrightarrow{\substack{#1}}%
}

\newcommand{\lelem}[1]{% elementary operations (left alignment)
\xrightarrow{\begin{subarray}{l}#1\end{subarray}}%
}

%SS
\DeclareMathOperator{\supp}{supp}
\DeclareMathOperator{\Var}{Var}

%NT
\DeclareMathOperator{\ord}{ord}

%Alg
\DeclareMathOperator{\Rad}{Rad}
\DeclareMathOperator{\Jac}{Jac}

%Misc
\newcommand{\SL}{{\mathrm{SL}}}
\newcommand{\mobgp}{{\mathrm{PSL}_2(\mathbb{C})}}
\newcommand{\id}{{\mathrm{id}}}
\newcommand{\Mod}{{\mathrm{Mod}}}
\newcommand{\ud}{{\mathrm{d}}}
\newcommand{\Vol}{{\mathrm{Vol}}}
\newcommand{\Area}{{\mathrm{Area}}}
\newcommand{\diam}{{\mathrm{diam}}}
\newcommand{\Homeo}{{\mathrm{Homeo}}}


\newcommand{\reg}{{\mathtt{reg}}}
\newcommand{\geo}{{\mathtt{geo}}}

\newcommand{\tori}{{\mathcal{T}}}
\newcommand{\cpn}{{\mathtt{c}}}
\newcommand{\pat}{{\mathtt{p}}}




\begin{document}

\section{Objectives}

\begin{itemize}
    \item Read up on transversality in Lee - potentially supplied with Hirsch
        and Guillemin and Pollack.
    \item Read up on classification of surfaces. Potentially
        through Munkres, or through the two papers in the folder
        "Classification of surfaces" under the Topology folder.
        One of them deals with surfaces with boundary.
    \item Read about oriented intersection theory in Guillemin and Pollack.
    \item Work on section 2 in Farb and Margalit.
    \item Read section on $K (G,1)$-spaces.
\end{itemize}

\section{Questions}

\begin{itemize}
    \item Grad school for algtop, geotop, alg?
    \item How does one check that $\gamma$ and $\beta$ fill
        the genus 2 surface in figure 1.7?
    \item How to find (or show existence of) orientation-preserving
        or orientation-reversing maps?
\end{itemize}


\newpage

\section{Curves, Surfaces and Hyperbolic Geometry}
\subsection{Simple closed curves}
There is a bijective correspondence
\[
\left\{ 
    \begin{tabular}{c}
    Nontrivial\\
    conjugacy classes\\
    in $\pi_1 (S)$
\end{tabular}
\right\} 
\longleftrightarrow
\left\{ 
    \begin{tabular}{c}
        Nontrivial free\\
        homotopy classes of oriented\\
        closed curves in $S$
\end{tabular}
\right\} 
\] 

\begin{definition}[Primitive and multiple elements]
    An element $g$ of a group $G$ is \textit{primitive} if there
    does not exist any $h \in G$ so that $g = h^{k}$ for
    $\left| k \right| >1$. The property of being a primitive
    is a conjugacy class invariant. In particular, it makes
    sense to say that a closed curve in a surface is primitive.\\
    A closed curve in $S$ is a multiple if it is a map
    $S^{1} \to S$ that factors through the map
    $S^{1} \stackrel{\times n}{\to } S^{1}$ for
    $n >1$, i.e., there exists a map $\tilde{\alpha} \colon
    S^{1} \to S$ such that the following diagram commutes:
    \begin{equation*}
    \begin{tikzcd}
        S^{1} \ar[r, "\times n"] \ar[rr, bend left = 45, dashed,
        "\tilde{\alpha}"] & S^{1} \ar[r,
        "\alpha"] & S
    \end{tikzcd}
    \end{equation*}
\end{definition}

\begin{definition}[Lifts]
    We make a distinction between lifts: let $p \colon
    \tilde{S} \to S$ be a covering space. By a \textit{lift} of a closed
    curve $\alpha$ to $\tilde{S}$ we will always mean the image of
    a lift $\mathbb{R} \to \tilde{S}$ of the map
    $\alpha \circ \pi$ where $\pi \colon \mathbb{R} \to S^{1}$ is
    the usual covering map. I.e., a lift of $\alpha \colon
    S^{1} \to S$ is a map  $\tilde{\alpha} \colon \mathbb{R} \to 
    \tilde{S}$ such that the following diagram commutes
    \begin{equation*}
    \begin{tikzcd}
        & & \tilde{S} \ar[d, "p"]\\
        \mathbb{R} \ar[r, "\pi"'] \ar[rru, "\tilde{\alpha}"] & S^{1} \ar[r, 
        "\alpha"'] & S
    \end{tikzcd}
    \end{equation*}
    A lift is different from a \textit{path lift} which is
    a proper subset of a lift. Namely, it would be
    the restriction of $\tilde{\alpha}$ to some interval of
    $\mathbb{R}$ of length $2 \pi$ if the covering map
    $\pi$ is of the form $t \mapsto e^{it}$.
\end{definition}

Now suppose $p \colon \tilde{S} \to S$ is the universal cover
and $\alpha$ is a simple closed curve in $S$ that is
not a multiple of another closed curve. In this case, there
is a bijective correspondence
between cosets in $ \pi_1 (S)$
 of the infinite cyclic subgroup $\left<\alpha \right>$ and
 the lifts of $\alpha$. This can be seen as follows: first choose
 a basepoint $\alpha(1) =  x_0 \in S$
and
 some $\tilde{x_0} \in p^{-1}(x_0)$. There exists a unique lift
 $\tilde{\alpha}$ of $\alpha$ such that
 \begin{equation*}
 \begin{tikzcd}
     & & \tilde{S}\ar[d, "p"] \\
     \mathbb{R} \ar[r] \ar[rru, "\tilde{\alpha}"] & S^{1} \ar[r, "\alpha"] & S
 \end{tikzcd}
 \end{equation*}
 commutes and such that
 $\tilde{\alpha}(0) = \tilde{x} \in p^{-1}(\alpha \circ \pi (0))$
 for some specific $\tilde{x}$  \cite[Cor. 4.2]{Bredon}.
 But the set
 $p^{-1} \left( \alpha \circ \pi (0) \right) $ is in bijective
 correspondence with the loops in $\pi_1 (S)$ by the path lifting lemma. 
 Now, under which path lifts are the lifts the same? The lifts of
 $\alpha$ to two points $\tilde{x}, \tilde{y} \in 
 p^{-1}\left( \alpha \circ \pi (0) \right) $ will be the same if
 $\alpha^{k} \cdot \tilde{x} = \tilde{y}$ where
 $\cdot $ denotes the monodromy action of $\pi_1 (S)$ on
 the fiber. Now, there exist $\gamma_x$ and
 $\gamma_y$ in $\pi_1 (S)$ such that
 $\gamma_x \cdot \tilde{x_0} = \tilde{x}$ and
 $\gamma_y \cdot \tilde{x_0} = \tilde{y}$, so
 $\alpha^k \gamma_x = \gamma_y$. Hence the lifts
 corresponding to $\gamma_x$ and $\gamma_y$ are the same if and only
 if $\alpha^k \gamma_x = \gamma_y$ for some $k$, i.e. if and only if
 $\gamma_x = \gamma_y$ in $\pi_1(S) / \left<\alpha \right>$.\\
 \linebreak
 As usual, the group $\pi_1 (S)$ acts on the set of lifts
 of $\alpha$ by deck transformations, and this action agrees
 with the usual left action of $\pi_1 (S)$ on the
 cosets of $\left<\alpha \right>$. The stabilizer of the lift
 corresponding to the coset $\gamma \left<\alpha \right>$ is
 the cyclic group $\left<\gamma \alpha \gamma^{-1} \right>$. See
 figure~\ref{fig:lifts-of-paths}.

 \begin{figure}[http]
     \centering
     \includegraphics[width=0.8\textwidth]{lifts-of-paths.jpg}
     \caption{}
     \label{fig:lifts-of-paths}
 \end{figure}

 \begin{theorem}[]
When $S$ admits a hyperbolic metric and
$\alpha$ is a primitive element of $\pi_1 (S)$, we have
a bijective correspondence
\[
\left\{ 
    \begin{tabular}{c}
    Elements of the conjugacy\\
    class of $\alpha$ in $\pi_1 (S)$
\end{tabular}
\right\} 
\longleftrightarrow
\left\{ 
    \begin{tabular}{c}
    Lifts to $\tilde{S}$ of the\\
    closed curve $\alpha$
\end{tabular}
\right\} 
\] 


More precisely, we claim that the map which sends the lift
given by the coset $\gamma \left<\alpha \right>$ to
$\gamma \alpha \gamma^{-1}$ is bijective and well-defined.\\
 \end{theorem}

 \begin{proof}
To show that it is well-defined, suppose $\gamma \left<\alpha \right>$ and
$\beta \left<\alpha \right>$ give the same lift. Then
$\gamma = \beta \alpha^k$. So in particular,
\[
    \gamma \alpha \gamma^{-1} = \beta \alpha^k \alpha \alpha^{-k} \beta^{-1}
    = \beta \alpha \beta^{-1}
\] 
so they do correspond to the same element of the conjugacy class
$\left[ \alpha \right] $. It is clear that this is a surjective map.
Now suppose that $\gamma \alpha \gamma^{-1} = \beta \alpha \beta^{-1}$. 
Then $\beta^{-1} \gamma \alpha \left( \beta^{-1} \gamma \right)^{-1} =
\alpha$, so in particular, $\beta^{-1} \gamma \in C_{\pi_1(S)}(\alpha)$
which is a cyclic group generated by, say, $\theta$. But then
$\theta^l = \alpha$ since $\alpha$ is trivially in the centralizer of
$\alpha$; however, $\alpha$ is primitive, so
$l$ must be $\pm 1$, but then  $\alpha$ generates the centralizer of
$\alpha$, $C_{\pi_1 (S)}(\alpha) = \left<\alpha \right>$, and hence
$\gamma = \beta \alpha^l$, so $\gamma \left<\alpha \right>
= \beta \left<\alpha \right>$.
 \end{proof}

 \begin{remark}[]
     If $\alpha$ is any multiple, then we still have a bijective correspondence
     between elements of the conjugacy class of $\alpha$ and the
     lifts of $\alpha$. However, if $\alpha$ is not primitive and not
     a multiple, then there are more lifts of $\alpha$ than there
     are conjugates. Indeed, if $\alpha = \beta^{k}$, where $k > 1$, then
     $\beta \left<\alpha \right> \neq \left<\alpha \right>$ while
     $\beta \alpha \beta^{-1} = \alpha$.
 \end{remark}

 \begin{example}[]
     The above correspondence does not hold for the torus $T^2$ because
     each closed curve has infinitely many lifts, while
     each element of $\pi_1 \left( T^2 \right) \approx \mathbb{Z}^2$ 
     is its own conjugacy class because $\pi_1 \left( T^2 \right) $ is
     abelian. 
 \end{example}

 \subsubsection*{Geodesic representatives}
 \begin{proposition}[]\label{unique-geodesic-representative}
     Let $S$ be a hyperbolic surface. If $\alpha$ is a closed curve
     in $S$ that is not homotopic into a neighborhood of a puncture, then
     $\alpha$ is homotopic to a unique geodesic closed curve $\gamma$.
 \end{proposition}
 
 \begin{corollary}
     For compact hyperbolic surfaces, 
     there is a bijective correspondence:
\[
\left\{ 
    \begin{tabular}{c}
        Conjugacy classes\\
        in $\pi_1 (S)$
\end{tabular}
\right\} 
\longleftrightarrow
\left\{ 
    \begin{tabular}{c}
        Oriented geodesic\\
        closed curves in $S$
\end{tabular}
\right\} 
\] 
 \end{corollary}

 \subsection*{Simple closed curves}
 
\begin{definition}[Simple curves]
    A closed curve in $S$ is \textit{simple} if it is topologically embedded, i.e.,
    if the map $S^{1} \to S$ is injective.  
\end{definition}

By \cite[Thm~11.8]{Bredon}, any closed curve $\alpha$ can be approximated 
(arbitrarily close) by
a smooth closed curve which is homotopic to $\alpha$. Moreover,
if $\alpha$ is simple, then the smooth approximation can be chosen to be
simple. Smooth curves are advantageous because we can make use of notions
such as transversality.

Simple closed curves are also natural to study because they represent
primitive elements of $\pi_1 (S)$.

\begin{proposition}[]
    Let $\alpha$ be a simple closed curve in a surface $S$. If $\alpha$ 
    is not null homotopic, then each element of the corresponding conjugacy
    class in $\pi_1(S)$ is primitive.
\end{proposition}

\subsection*{Example: simple closed curves on the torus}

\begin{proposition}[]
    The nontrivial homotopy classes of oriented simple closed
    curves in $T^2$ are in bijective correspondence with the set of primitive
    elements of $\pi_1\left( T^2 \right) \approx \mathbb{Z}^2$ which is
    the set of elements $\left( p,q \right)  \in \mathbb{Z}^2$ such that
    either $(p,q) = (0,\pm 1)$ or  $(p,q) = (\pm 1,0)$ or
     $\gcd(p,q) = 1$.
\end{proposition}

\subsection*{Closed geodesics}

\begin{proposition}[]
    Let $S$ be a hyperbolic surface. Let $\alpha$ be a closed curve
    in $S$ not homotopic into a neighborhood of a puncture. Let
    $\gamma$ be the unique geodesic in the free homotopy class of
    $\alpha$ guaranteed by proposition \ref{unique-geodesic-representative}.
    If $\alpha$ is simple, then $\gamma$ is simple.
\end{proposition}

\begin{proof}
    Follows from the following lemma:
    \begin{lemma}[]
        Let $X$ be a topological space with a universal covering space
         $\tilde{X}$. A closed curve $\beta$ in $X$ is simple if and
        only if the following properties hold:
        \begin{enumerate}
            \item Each lift of $\beta$ to $\tilde{X}$ is simple.
            \item No two lifts of $\beta$ intersect.
            \item  $\beta$ is not a nontrivial multiple of another closed
                curve.
         \end{enumerate}
    \end{lemma}
\end{proof}

\subsection*{Intersection numbers}

It is often useful to put an inner product on a vector space to check
if two vectors are linearly independent. We can pursue something similar for
surfaces.

\begin{definition}[Geometric intersection number]
    Let $\alpha, \beta$ be closed curves on a surface $S$.
    Their \textit{geometric intersection number} is
    \[
    i \left( \alpha, \beta \right) =
    \min_{\alpha' \simeq \alpha, \beta' \simeq \beta}
    \# \left( \alpha' \cap \beta' \right) 
    \] 
\end{definition}

\begin{definition}[Preliminary definition for transversality]
If $\alpha \cap \beta$ is finite and, at every intersection, each
curve locally separates the other curve, then we say that
$\alpha$ and $\beta$ are \textit{transverse}.
\end{definition}

\begin{definition}[Minimal position]
    Two curves $\alpha$ and $\beta$ are in \textit{minimal
    position} if $\# \left( \alpha \cap \beta \right) =
    i\left( \alpha, \beta \right) $.
\end{definition}

\subsection*{Bigons}

We want a procedure to put curves into minimal position so
we can compute intersection numbers.

For this, we need the notion of a \textit{bigon}:

\begin{definition}[Bigon]
    Two transverse simple closed curves $\alpha$ and $\beta$ 
    in a surface $S$ form a \textit{bigon} if there is a
    topologically embedded disk in $S$ (the bigon) whose
    boundary is the union of an arc of $\alpha$ and an
    arc of  $\beta$ intersecting in exactly two points.
\end{definition}

\begin{figure}[http]
    \centering
    \includegraphics[width=0.4\textwidth]{bigon.png}
    \caption{Local picture of a bigon}
    \label{fig:bigon}
\end{figure}

\begin{lemma}[]\label{lemma-intersections-of-lifts}
    If transverse simple closed curves $\alpha$ and $\beta$ in
    a surface $S$ do not form any bigons, then in the
    universal cover of $S$, and pair of lifts
    $\tilde{\alpha}$ and $\tilde{\beta}$ of $\alpha$ and $\beta$ 
    intersect in at most one point.
\end{lemma}

\begin{figure}[h]
    \centering
    \includegraphics[width=0.8\textwidth]{lemma-1-8.jpg}
    \caption{Lemma \ref{lemma-intersections-of-lifts} illustrated}
    \label{fig:lemma-1-8-jpg}
\end{figure}


\begin{proposition}[The bigon criterion]
    Two transverse simple closed curves in a surface $S$ are in
    minimal position if and only if they do not form
    a bigon.
\end{proposition}

\begin{corollary}
    Any two transverse simple closed curves that intersect exactly once are in
    minimal position.
\end{corollary}

\subsection*{Homotopy versus isotopy for simple closed curves}

\begin{definition}[Isotopy]
    Two simple closed curves $\alpha$ and $\beta$ are \textit{isotopic}
    if there is a homotopy 
    \[
    H \colon S^{1} \times \left[ 0,1 \right] \to 
    S
    \] 
    from $\alpha$ to $\beta$ with the property that the closed
    curve $H \left( S^{1} \times \left\{ t \right\}  \right) $ 
    is simple for each $t \in \left[ 0,1 \right] $.
\end{definition}

\begin{proposition}[Baer]
    Let $\alpha$ and $\beta$ be two essential simple closed curves
    in a surface $S$. Then $\alpha$ is isotopic to $\beta$ if and
    only if $\alpha$ is homotopoic to $\beta$.
\end{proposition}

\begin{proof}
    If $\alpha$ is isotopic to $\beta$ then they are clearly also
    homotopic.\\
    \linebreak
    Suppose $\alpha$ and $\beta$ are homotopic. Taking a
    tubular neighborhood around $\alpha$, we can find
    a disjoint simple loop $\tilde{\alpha}$ which is homotopic to
    $\alpha$ but disjoint from it. Then $\beta$ is homotopic
    to $\tilde{\alpha}$, and hence
    $i \left( \alpha, \beta \right) 
    = i \left( \alpha, \tilde{\alpha} \right) = 0$.
    Performing an isotopy of $\alpha$, we may assume that
    $\alpha$ is transverse to $\beta$ (why?).
    If $\alpha$ and $\beta$ are not disjoint, then by
    the bigon criterion, they form a bigon. A bigon
    prescribes an isotopy that reduces intersection, so we
    may remove bigons by isotopy until $\alpha$ and $\beta$ are
    disjoint.\\
    \linebreak
    Suppose $\chi (S) <0$. Lift $\alpha$ and $\beta$ to
    $\tilde{a}$ and $\tilde{\beta}$ with the
    same endpoints in $\partial \mathbb{H}^2$. There
    is a hyperbolic isometry $\varphi$ that leaves
    $\tilde{\alpha}$ and $\tilde{\beta}$ invariant and
    acts by translation on the lifts. As $\tilde{\alpha}$ and
    $\tilde{\beta}$ are disjoint, let $R$ denote the region
    between them. We claim that the quotient surface
    $R' = R / \left<\varphi \right>$ is an annulus.
    The fundamental group of $R'$ is isomorphic to the group of
    deck transformations $\left<\varphi \right>$ and is hence
    infinite cyclic. Furthermore, $R'$ has two boundary components.
\end{proof}


\subsubsection*{Extension of isotopies}

\begin{definition}[]
    An isotopy of a surface $S$ is a homotopy $H \colon 
    S \times I \to S$ such that for each
    $t \in \left[ 0,1 \right] $, the map
    $H \left( S, t \right) \colon S \times \left\{ t \right\} \to 
    S$ is a homeomorphism. Given an isotopy between
    two simple closed curves in $S$, it will often be useful
    to promote this to an isotopy of $S$ which we call an
    ambient isotopy of $S$.
\end{definition}

\begin{proposition}[]
    Let $S$ be any surface. If $F \colon S^{1} \times I \to 
    S$ is a smooth isotopy of simple closed curves, then
    there is an isotopy $H \colon S \times I \to S$ so that
    $H |_{S \times 0}$ is the identity and 
    $H|_{F\left( S^{1} \times 0 \right) \times I} = F$.
\end{proposition}

\begin{proof}
    \cite[Ch~8, Thm 1.3]{Hirsch}
\end{proof}



\subsection*{Arcs}
$ $ \newline \newline
Assume $S$ is a compact surface, possibly with boundary
and possibly with finitely many marked points in the interior.
Denote the set of marked points by $\mathcal{P}$.

\begin{definition}[]
    A \textit{proper arc} in $S$ is a map $\alpha \colon
    \left[ 0,1 \right]  \to S$ such that
    $\alpha^{-1} \left( \mathcal{P} \cup \partial S \right) 
    = \left\{ 0,1 \right\} $.
\end{definition}
\begin{definition}[]
    The arc $\alpha$ is \textit{simple} if it is an embedding
on its interior.
\end{definition}

\begin{remark}[]
    The homotopy class of a proper arc is taken to be
    the homotopy class within the class of proper arcs.
    Thus points on $\partial S$ cannot move off the boundary
    during the homotopy.\\
    A homotopy (or isotopy) of an arc is said to
    be \textit{relative to the boundary} if its endpoints
    stay fixed throughout the homotopy. 
    An arc in a surface $S$ is \textit{essential} if it
    is neither homotopic into a boundary component of $S$ nor a
    marked point of $S$.
\end{remark}

\begin{figure}[htpb]
    \centering
    \includegraphics[width=0.5\textwidth]{bigon-of-arcs.png}
    \caption{bigon-of-arcs.png}
    \label{fig:bigon-of-arcs-png}
\end{figure}
Note in this picture how if isotopies are considered relative
to the boundary, then the two arcs are in minimal position, while
if we consider general isotopies, then the half-bigon shows that
they are not in minimal position as we can pull the top strand down
under the bottom one along the boundary.

\begin{itemize}
    \item The bigon criterion holds for arcs.
    \item Corollary 1.9 (geodesics are in minimal position)
        and prop 1.3 (existence and uniqueness of geodesic 
        representatives) work for arcs in surfaces with
        punctures and/or boundary.
    \item Prop 1.10 (homotopy versus isotopy for curves) and
        theorem 1.13 (extension of isotopies) also
        work for arcs.
\end{itemize}

\subsection*{Change of coordinates principle}

\subsubsection*{Classification of simple closed curves}

\begin{definition}[]
    Given a simple closed curve or
    a simple proper arc $\alpha$ in a surface $S$,
    the surface obtain by cutting $S$ along $\alpha$ is
    a compact surface $S_{\alpha}$ equipped with an attaching
    map $h$ (i.e.
    \begin{enumerate}
        \item $S_{\alpha} / \left( x \sim h(x) \right) 
            \approx S$ 
        \item the image of the distinguished boundary components
            under this quotient map is $\alpha$.
    \end{enumerate}
\end{definition}

\begin{definition}[]
    We say that a simple closed curve $\alpha$ in the surface
    $S$ is \textit{nonseparating} if the cut surface
    $S_{\alpha}$ is connected, and \textit{separating} if
    $S_{\alpha}$ is not connected.
\end{definition}


\begin{theorem}[]
    If $\alpha$ and $\beta$ are any two nonseparating simple
    closed curves in a surface $S$, then there
    is a homeomorphism $\varphi \colon S \to S$ with
    $\varphi \left( \alpha \right) = \beta$.
\end{theorem}

\begin{proof}
    The cut surface $S_{\alpha}$ and $S_{\beta}$ have
    two boundary component corresponding to $\alpha$ and 
    $\beta$, respectively. 
    Now, suppose $S_{\alpha}$ has
    $n_{\alpha}$ vertices, $m_{\alpha}$ edges and
    $t_{\alpha}$ triangles in a triangulation. Then
    in obtaining $S$ from $S_{\alpha}$, we identify the vertices
    and edges, but no triangles are identified, so we get
    $n_{S} = n_{\alpha}-3$ and $m_{S} = m_{\alpha}-3$, but
    $t_{S} = t_{\alpha}$. Thus $\chi (S_{\alpha}) = 
    \chi (S)$.

    Since both $S_{\alpha}$ and
    $S_{\beta}$ have the same Euler characteristic, number of
    boundary components and number of punctures, it follows
    that $S_{\alpha} \approx S_{\beta}$. Choose
    a homeomorphism $\varphi \colon S_{\alpha} \to S_{\beta}$ such 
    that if $h_{\alpha}$ is the attaching map for $S_{\alpha}$ 
    and $h_{\beta}$ is the attaching map for $S_{\beta}$,
    then $\varphi$ takes $\left\{ x, h_{\alpha}(x) \right\} $ 
    to $\left\{ y, h_{\beta}(y) \right\} $ - i.e., the
    identification are respected under the map.
    This homeomorphism gives the desired
    homeomorphism of $S$ taking $\alpha$ to $\beta$.
    If we want an orientation preserving homeomorphism, we
    can postcompose by an orientation-reversing homeomorphism
    fixing $\beta$ if necessary.
\end{proof}

\begin{theorem}[]
    When $S$ is closed, $\beta$ is separating if and only
    if it is the boundary of some subsurface of $S$. Which
    is equivalent to the vanishing of the
    homology class of $\beta$ in $H_1 \left( S, \mathbb{Z} \right) $.
\end{theorem}

\begin{remark}[]
    By the "classification of disconnected surfaces", there
    are finitely many separating simple closed curves in $S$ 
    up to homeomorphism.
\end{remark}


\begin{corollary}\label{cut-surface-homeo-1}
    There is an orientation-preserving homeomorphism of a surface
    taking one simple closed curve to another if and only
    if the corresponding cut surfaces (which may be disconnected)
    are homeomorphic.
\end{corollary}

\begin{definition}[Topological type]
    The existence of a homeomorphism as in \ref{cut-surface-homeo-1}
    is an equivalence relation. The equivalence class
    of a simple closed curve or a collection of simple closed
    curves is called its \textit{topological type}.
\end{definition}

A seprarating simple closed curve in the closed surface
$S_g$ divides $S_g$ into two disjoint subsurfaces of genus
$k$ and $g-k$. The minimum of $\left\{ k, g-k \right\} $ is
called the genus of the separating simple closed curve. 
There are $\left\lfloor \frac{g}{2} \right\rfloor$ topological
types of essential separating simple closed curves in
a closed surface.

\begin{question}
    Suppose $\alpha$ is any nonseparating simple closed curve
    on a surface $S$.
    \begin{enumerate}
        \item Is there a simple closed curve
            $\gamma$ in $S$ so that $\alpha$ and $\gamma$ 
            \textit{fill} $S$, i.e., such that
            $\alpha$ and $\gamma$ are in minimal position
            and the complement of $\alpha \cup  \gamma$ is
            a union of topological disks.
        \item Is there a simple closed curve $\delta$ in
            $S$ with $i(\alpha,\beta) = 0$? $i(\alpha,\beta)=1$?
            $i(\alpha,\beta)=k$?
    \end{enumerate}
\end{question}

\begin{figure}[htpb]
    \centering
    \includegraphics[width=0.7\textwidth]{filling-genus-2-surface.png}
    \caption{a}
    \label{fig:filling-genus-2-surface-png}
\end{figure}

Figure \ref{fig:filling-genus-2-surface.png} shows
two filling simple closed curves on the genus $2$ surface.
By the classification of simple closed curves on a surface,
there is a homeomorphism $\varphi \colon S_2 \to S_2$ such
that $\varphi\left( \beta \right) =\alpha$. Then
the image of $\gamma$ under $\varphi$ fills $S_2$ with
$\alpha$ since filling is a topological property (show this).

\subsubsection*{Examples of change of coordinate principle}

\begin{enumerate}
    \item \textit{Pairs of simple closed curves that intersect
        once are all of the same topological type}.
        Suppose $\alpha_1$ and $\beta_1$ form
        such a pair on a surface $S$. Then
        $\beta_1$ must be an arc connecting the two
        boundary components in $S_{\alpha_1}$. But the boundary
        component is homeomorphic to $S^{1}$, so removing a 
        point leaves it connected. Thus removing
        $\beta_1$ leaves $\left( S_{\alpha_1} \right)_{\beta_1}$
        path-connected. Similarly,
        $\left( S_{\alpha_2} \right)_{\beta_2}$ is path-connected
        for any other pair $\alpha_2$ and $\beta_2$ that
        constitute a pair of simple closed curves
        that intersect once in $S$. By the classification
        of surfaces with boundary, $\left( S_{\alpha_1} \right)_{\beta_1}$ is
        homeomorphic to $\left( S_{\alpha_2} \right)_{\beta_2}$ 
        which preserves equivalence classes on the boundary,
        and as we can construct this homeomorphism first
        for the $\beta$'s and then for the $\alpha$'s, 
        this homeomorphism descends to a self-homeomorphism
        of $S$ taking the pair $\left\{ \alpha_1, \beta_1 \right\} $ 
        to $\left\{ \alpha_2, \beta_2 \right\} $.
\end{enumerate}


\subsection*{Three facts about homeomorphisms}

Suppose $f \colon D \to D$ is an orientation-reversing map.
Then $f$ restricts to a map on $S^{1} \to S^{1}$, and
if $f$ is smooth considered as such a map, then the
reversal of orientation implies that since the fiber
of any point is a single point, the degree of $f$ 
must be $-1$. But thus $f$ is not isotopic to the identity
as the identity has degree $1$ and the isotopy would
have to restrict to a homotopy on the boundary, but
degree is a homotopy invariant for maps $S^{n} \to S^{n}$.\\
However, the straight-line homotopy does give a homotopy
between $f$ and the identity.\\
\linebreak
On $A = S^{1} \times I$, the orientation-reversing map
that fixes the $S^{1}$ factor and reflects the $I$ factor
is homotopic but not isotopic to the identity.


\begin{theorem}[]
    Let $S$ be any compact surface and let $f$ and $g$ be
    homotopic homeomorphisms of $S$. Then $f$ and $g$ are
    isotopic unless they are one of the two examples described
    above (on $S = D^2$ and $S = A$ ). In particular,
    if $f$ and $g$ are orientation-preserving, then they
    are isotopic.
\end{theorem}



\begin{theorem}[]
    Let $S$ be a compact surface. Then every homeomorphism of
    $S$ is isotopic to a diffeomorphism of $S$.
\end{theorem}

\begin{theorem}[Hamstrom]
    Let $S$ be a compact surface, possibly minus a finite
    number of points from the interior. Assume that
    $S$ is not homeomorphic to
    $S^2, \mathbb{R}^2, D^2, T^2$, the closed annulus,
    the once-punctured disk, or the once-punctured plane. Then
    the space $\Homeo_0 (S)$ is contractible.
\end{theorem}




\newpage

\section{Mapping class group basics}


\subsection*{The compact-open topology}

\begin{definition}[]
    The \textit{weak} or \textit{compact-open $C^{r}$ }
    topology on $C^{r} \left( M,N \right) $, where
    $M$ and $N$ are $C^{r}$ manifolds, is generated
    by sets defined as follows:
    let $f \in C^{r}(M,N)$. Let $\left( U, \varphi \right),
    \left( V, \psi  \right) $ be charts on
    $M$ and $N$ ; let $K \subset U$ be compact such that
    $f(K) \subset V$ and let $0 < \varepsilon \le 
    \infty $. Then a \textit{weak subbasic neighborhood}
    \[
    \mathcal{N}^{r} \left( f ; \left( U, \varphi  \right) ,
    \left( V, \psi  \right) , K, \varepsilon \right) 
    \tag{$\zeta$}\label{eq:compact-open-subbasis-nbhds}
    \] 
    is the set of $C^{r}$ maps $g \colon M \to N$ such that
    $g(K) \subset V$ and
    \[
    \|D^{k}\left( \psi f \varphi^{-1} \right) (x)
    -D^{k} \left( \psi g \varphi^{-1} \right) (x)\|< \varepsilon
    \] 
    for all $x \in \varphi(K)$, for $k = 0, \ldots, r$.
    The \textit{compact-open $C^{r}$ topology} on
    $C^{r}(M,N)$ is generated by the set of weak subbasic
    neighborhoods, and defines the topological space
    $C_W^{r} (M,N)$.
    A neighborhood of $f$ is then
    any set containing the intersection
    of a finite number of sets of the
    type \eqref{eq:compact-open-subbasis-nbhds}.
\end{definition}

We are interested in the subspace
$\Homeo (S) \subset C_W^{0} (S,S)$, inheriting the
subspace topology.\\
\linebreak
The compact-open topology might seem a bit confusing, but
we have the following lemma \cite[Prop A.14]{Hatcher}:

\begin{lemma}[]\label{compact-open-isotopy}
    Let $X,Y,Z$ be Hausdorff topological spaces.
    Suppose $Y$ is locally compact. Then a 
    map $f \colon X \to C_W^{0}(Y,Z)$ is
    continuous if and only if the associated map
    $F \colon X \times Y \to Z$ defined by
    \[
    F(x,y) := f(x)(y)
    \] 
    is continuous.
\end{lemma}





\subsection{Definitions and first examples}

\begin{definition}[]
    Let $S$ be a surface which is the connected sum of $g\ge 0$ 
    tori with $b \ge 0$ disjoint open disks removed and
    $n \ge 0$ points removed from the interior. Let
    $\Homeo^{+} \left( S, \partial S \right)$ denote the group
    of orientation-preserving homeomorphisms of $S$ that
    restrict to the identity on $\partial S$. We endow this
    group with the compact-open topology.
    The \textit{mapping class group} of  $S$, denoted
    $\Mod (S)$, is the group
    \[
    \Mod(S) = \pi_0 \left( \Homeo^{+} \left( S, \partial S
    \right) \right) 
    \] 


\begin{remark}[]
    From Lemma \ref{compact-open-isotopy}, we
    see that a path $\gamma \colon I \to 
    \Homeo^{+}\left( S, \partial S \right) $
    is precisely equivalent to an isotopy 
    $F \colon I \times S \to S$ from
    $\gamma(0)$ to $\gamma(1)$ (isotopy because at each
    time $t$, $\gamma(t) \colon S \to S$ is indeed a topological
    embedding as it is a homeomorphism). In fact, it's an isotopy
    of $S$. Here isotopies are required to fix
    boundaries.
\end{remark}





    If
    $\Homeo_0 (S, \partial S)$ denotes the connected component of
    the identity in $\Homeo^{+}\left( S, \partial S \right) $, then
    we can equivalently write
    \[
    \Mod (S) = \Homeo^{+} \left( S, \partial S \right) /
    \Homeo_0 \left( S, \partial S \right) .
    \] 
\end{definition}

\begin{proposition}[]
    \begin{align*}
        \Mod (S) 
        &= \pi_0 \left( \Homeo^{+} \left( S, \partial S \right) 
        \right) \\
        &\approx \Homeo^{+}\left( S, \partial S \right) /
        \text{homotopy}\\
        &\approx \pi_0 \left( \mathrm{Diff}^{+} \left( 
        S, \partial S \right)  \right) \\
        &\approx \textrm{Diff}^{+}\left( S, \partial S \right) /
        \sim
    \end{align*}
    where $\mathrm{Diff}^{+}\left( S, \partial S \right) $ is the group of
    orientation preserving diffeomorphisms of $S$ that are the
    identity on the boundary and $\sim$ can be taken to be either
    smooth homotopy relative to the boundary or smooth isotopy relative to the
    boundary.
\end{proposition}

\subsubsection*{The Alexander Lemma}

\begin{lemma}[Alexander lemma]
   The group $\Mod\left( D^2 \right) $ is trivial. 
\end{lemma}

\begin{remark}
    Also $0 \approx \Mod\left( D - \left\{ 0 \right\}  \right) 
    \approx \Mod \left( S_{0,1} \right) \approx
    \Mod\left( S^2 \right) $.
\end{remark}

\subsubsection*{The mapping class group of the thrice-punctured sphere,
$\Mod \left( S_{0,3} \right) $}

\begin{proposition}[]
    Any two essential simple proper arcs in $S_{0,3}$ with the
    same endpoints are isotopic. Any two essential arcs that both
    start and end at the same marked point of $S_{0,3}$ are isotopic.
\end{proposition}

\begin{proof}
    Let $\alpha$ and $\beta$ be two simple proper arcs in
    $S_{0,3}$ connecting the same two distinct marked points.
    By isotopy, we may modify $\alpha$ so that
    it intersects transversally with $\beta$.
    Letting the last marked point become the point at infinity,
    we can consider $\alpha$ and $\beta$ as being
    arcs in $\mathbb{R}^2 - \left\{ p,q \right\} $ for
    the two marked points $p,q$.
    An example is illustrated below.
    Now, suppose the arcs are disjoint.
    Then, choosing an intersection point, we can follow
    the path to the other intersection point and obtain
    either a bigon, in which case we can remove it by isotopy,
    or a bigon with path segments inside.
    Now, suppose
    the there is some point of $\alpha$ inside the bigon.
    Then since this is part of the arc $\alpha$, we can find
    a simple path connecting this point to two points
    of $\beta$. There could, however, be infinitely many
    such paths inside the bigon, preventing us choosing the
    innermost (think concentric semicircles).
    However, by transversality, the preimages of the 
    intersection points
    form a $0$-dimensional submanifold of $I$ which is closed
    (as the preimage of a closed path segment of $ \beta$)
    and discrete.
    But discrete subsets of compact spaces are finite. Hence
    we can choose the innermost such path of $ \alpha$.
    By isotopy, we can remove the bigon formed by  this alpha.
    Continuing a finite amount of times, we remove the original
    bigon. After a finite amount of reiterations, we can
    therefore remove all bigons, and we get
    disjoint $ \alpha$ and $\beta$.

    Now suppose we remove $\alpha \cup  \beta$. Then
    we get a disjoint union of a disk and a punctured
    disk (by the classification of surfaces - expound on this).
    Thus the embedded disk in $S_{0,3}$ gives an isotopy
    of $\alpha$ to $\beta$.
\end{proof}
 



\begin{proposition}[]
    The natural map
    \[
    \Mod\left( S_{0,3} \right) \to \Sigma_{3}
    \] 
    given by the action of $\Mod\left( S_{0,3} \right) $ on the
    set of marked points of $S_{0,3}$ is an isomorphism.
\end{proposition}

\begin{proof}
    This is just additional notes to the proof in the book.
    The reason it is surjective is that the previous proposition
    gives an isotopy between arcs which we can extend to
    an ambient isotopy relative to the boundary which
    is an element of the mapping class group. (can we
    be sure the last marked point stays fixed?)
\end{proof}




\begin{exercise}[]
    Show similarly that $\Mod \left( S_{0,2} \right) \approx
    \mathbb{Z}/ 2\mathbb{Z}$.
\end{exercise}


\begin{solution}
    Let $\alpha, \beta$ be arcs with the same distinct
    marked endpoints. Equivalently to before, we can reduce
    bigons by isotopy until $\alpha$ and $\beta$ are disjoint.
    Then removing $\alpha \cup  \beta$ we would get
    two disjoint disks (firstly, $\alpha \cup \beta$ 
    make up a closed simple curve which is trivial since
    $H_1 \left( S^{1} \right) = \left\{ 0 \right\} $ and
    thus separating. Therefore we get a disconnected
    space with as many vertices as edges whose
    Euler characteristic must add to
    $2 = \chi \left( S^{2} \right) $, so
    it must precisely have $1$ face each, i.e., they are disks)
    which will descend to give the desired
    isotopy in $S_{0,2}$.
    
    So assume no intersection. Let $\varphi$ be an orientation
    preserving homeomorphism fixing the marked points. Then
     $\varphi \left( \alpha \right) $ is isotopic
     to $\alpha$, so $\varphi$ is isotopic to a homeomorphism which
     fixes $\alpha$ pointwise, call it $\psi$.
     This induces a homeomorphism on $S^{2} - \alpha$ which
     is a disk that is the identity on the boundary, and hence
     isotopic to the identity homeomorphism on the disk
     since $\Mod \left( D^2 \right) \approx \left\{ 0 \right\} $.
     This isotopy gives an isotopy of $ \psi$ to the identity.
     The composition of all these
     isotopies gives an isotopy of 
     $\varphi$ with the identity.
     Hence the map is injective.
\end{solution}



\begin{theorem}[]\label{mcg-of-torus}
    The homomorphism
    \[
    \sigma \colon \Mod \left( T^2 \right) \to 
    \SL \left( 2, \mathbb{Z} \right) 
    \] 
    given by the action on $H_1 \left( T ;\mathbb{Z} \right) 
    \approx \mathbb{Z}^2$ is an isomorphism.
\end{theorem}

\begin{proof}
    Additional notes on the proof:
    why can we for any element $f \in \Mod \left( T^2 \right) $ 
    choose a representative $\varphi $ that fixes a basepoint
    for $T^2$?
\end{proof}


\begin{corollary}
    Since $H_1 \left( S_{1,1}; \mathbb{Z} \right) \approx
    \mathbb{Z}^2$, there is a homomorphism
    $\sigma \colon \Mod \left( S_{1,1} \right) \to 
    SL(2, \mathbb{Z})$ which is determined
    which isomorphism the homomorphism induces in
    homology. This map is an isomorphism.
\end{corollary}

\begin{exercise}[]
    Prove this explicitly.
\end{exercise}

\subsubsection*{The mapping class group of
$S_{0,4}$}

Consider the torus $T^2$ as $I^2 / \sim$ under the usual
identification. Then consider the
linear map $\iota \colon \mathbb{R} \to \mathbb{R}$ 
by
$\iota = \begin{pmatrix} -1 & 0\\ 0 & -1 \end{pmatrix} 
= - I
\in SL(2, \mathbb{Z})$ which rotates about the origin
by $\pi$ radians.

The map is equivarient with respect to the quotient map
so it induces a map $I^2 / \sim \to I^2 / \sim$
and we wish to take the quotient space that identifies
fibers of this map. This is equivalent to taking
the quotient space of $\mathbb{R}^2$ induced by the
following actions: for $\left( a,b \right) \in \mathbb{R}^2$,
\begin{enumerate}
    \item sending $(a,b)$ to $(a+2k,b)$ for $k \in \mathbb{Z}$,
    \item sending $(a,b)$ to $(a, b+ 2t)$ for $t \in \mathbb{Z}$,
    \item or sending $(a,b)$ to $(-a, -b)$.
\end{enumerate}
We claim the quotient of $\left[ 0,2 \right] \times I $ 
under this action
is a fundamental domain for the action.
Clearly, the action is transitive. Now
if $(a,b),(c,d) \in (0,2) \times I $ are in the same orbit,
then
\begin{align*}
    a &= (-1)^{\alpha} c + 2k \\
    b &= (-1)^{\alpha} d + 2t
\end{align*}
for some $k,t \in \mathbb{Z}$. But then
if $b+d = 2t$, we  get $b+d \in 2 \mathbb{Z} \cap \left( 0,2 \right) 
= \varnothing$, so $\alpha$ must be even, and
$b = d$. But then
$a-c \in  2\mathbb{Z} \cap \left( -2, 2 \right) =
\left\{ 0 \right\} $, so $a=c$ and $b=d$.
The identifications on the boundary become as in figure
\ref{fig:involution-on-torus-quotient} which becomes
$S^2$.


\begin{figure}[htpb]
    \centering
    \includegraphics[width=0.6\textwidth]{involution-on-torus-quotient.jpg}
    \caption{}
    \label{fig:involution-on-torus-quotient}
\end{figure}

We identify the quotient
by $S_{0,4}$ where the $4$ marked points are
the $4$ fixed points under the involution, namely, the images
of the
center of $I^2$, the midpoints of the edges and corner vertices.
This is clearly also a $2$-fold cover of the sphere.

Now, since for any $A \in \SL (2,\mathbb{Z})$,
$A (-I) = (-I) A$,
each element of $\Mod \left( T^2 \right) $ induces
an element of $\Mod \left( S_{0,4} \right) $ by
descending to the quotient.


\begin{proposition}[]
    The hyperelliptic involution induces a bijection
    between the set of homotopy classes of essential
    simple closed curves in $T^2$ and the
    set of homotopy classes of essential simple closed 
    curves in $S_{0,4}$.
\end{proposition}

\begin{proof}
    Notes:

    \begin{figure}[htpb]
        \centering
        \includegraphics[width=0.9\textwidth]{twists-along-curves-on-torus.jpg}
        \caption{Twists along the meridian circle on the torus}
        \label{fig:twists-along-curves-on-torus-jpg}
    \end{figure}


    Why is the preimage of a $\left( p,q \right) $-curve in
    $S_{0,4}$ in $T^2$ a $\left( 2p,2q \right) $-curve?

\end{proof}

\begin{proposition}[]\label{mcg-of-4-punctured-sphere}
    $\Mod \left( S_{0,4} \right) \approx
    \PSL \left( 2, \mathbb{Z} \right) \ltimes
    \left( \mathbb{Z}/ 2 \mathbb{Z} \times 
    \mathbb{Z} / 2 \mathbb{Z} \right) $.
\end{proposition}


\begin{lemma}[]
    If the short exact sequence of groups
    \[
    1 \to N \to G \to H \to 1
    \] 
    has a right inverse for $G \to H$, then
    $G$ is naturally isomorphic to
    $N \ltimes H$.
\end{lemma}

\begin{proof}
    Let $f \colon N \to G, g \colon G \to H$ and
    $h \colon H \to G$ be the inverse. Then
    $f$ and $g$ are injective. Suppose
    $z \in f(N) \cap h(H)$. Then there exists a
    $v \in N$ and $u \in H$ such that
    $f(v) = z = h(u)$, so 
    $u = g\left( h(u) \right) = g(z)
    = g(f(v)) = 0$, so $z = 0$.
    Since $f(N)$ is the kernel of $g$, it is
    normal in $G$, so $f(N) h(H)$ forms
    a subgroup of $G$. Now suppose
    $p \in G - f(N)h(H)$. Then since
    $g\left( p - h(g(p)) \right) = 0$, there
    exists $n \in N$ such that
    $p = f(n) + h(g(p)) \in f(N) h(H)$, contradiction.
    So $G = f(N) h(H)$, giving $G = f(N) \ltimes h(H)
    \approx N \ltimes H$.
\end{proof}

\begin{proof}
    To show \ref{mcg-of-4-punctured-sphere}, it
    thus suffices to find a homomorphism
    $\Mod \left( S_{0,4} \right) \to 
    \PSL \left( 2, \mathbb{Z} \right) $ with a right
    inverse, and show that the kernel
    is $\mathbb{Z} /2 \mathbb{Z} \times \mathbb{Z} / 2 \mathbb{Z}$.

    Notes on the proof: for the involutions
    $\iota_1, \iota_2$, we can lift them
    to homeomorphisms of $T^2$ by the lifting theorem
    \cite[Thm~4.1]{Bredon}. But why would these
    necessarily have to be homeomorphisms that rotate
    one of the factors of $T^2 \approx S^{1} \times S^{1}$ by
    $\pi$?
\end{proof}

\section{Dehn Twists}

Let $S$ be an oriented surface and let $\alpha$ be
a simple closed curve in $S$. Let $N$ be a tubular
neighborhood of $\alpha$ and choose
an orientation preserving homeomorphsim
$\varphi \colon A \to N$. We then obtain
a homeomorphism $T_{\alpha} \colon
S \to S$, called a \textit{Dehn twist about $\alpha$}, as
follows:
\[
T_{\alpha}(x) = 
\begin{cases}
    \varphi \circ T \circ \varphi^{-1}& \text{if } 
    x\in N \\
    x& \text{if } x\in S - N
\end{cases}.
\] 

"By the uniqueness of regular neighborhoods, the isotopy
class of $T_{\alpha}$ does not depend on the choice of
$N$ or the choice of homeomorphism $\varphi$. Nor
does $T_{\alpha}$ depend on the choice
of simple closed curve $\alpha$ within its isotopy class." 
Huh, why???


\subsubsection*{Dehn twists on the torus}
Via the isomorphism
$\Mod \left( T^2 \right)  \to \SL\left( 2, \mathbb{Z} \right) $ 
from \ref{mcg-of-torus}, the Dehn twists
about the $(1,0)$-curve and the $(0,1)$-curve
in $\Mod\left( T^2 \right) $ correspond to the
matrices
\[
    \begin{pmatrix} 1 & -1 \\ 0 & 1 \end{pmatrix} 
    \quad \text{and} \quad 
    \begin{pmatrix} 1 & 0 \\ 1 & 1 \end{pmatrix} 
\] 
\subsubsection{Dehn twist facts}

\begin{proposition}[]
    Let $a$ be the isotopy class of a simple closed curve
    $\alpha$ in a surface $S$. If $\alpha$ is not
    homotopic to a point or a puncture of $S$, then the
    Dehn twist $T_a$ is a nontrivial element of $\Mod(S)$.
\end{proposition}

\begin{proposition}[]
    Let $a$ and $b$ be arbitrary isotopy classes of essential
    simple closed curves in a surface and let $k$ be
    an arbitrary integer. We have
    \[
    i \left( T_{a}^{k}(b), b \right) =
    \left| k \right| i\left( a,b \right)^2.
    \] 

\end{proposition}


\newpage

\section{Exercises}

\begin{problem}[]
    Give an example of a surface $S$ of finite type and
    self-diffeomorphism $\varphi $ of $S$ which is
    homotopic to $\id_S$ but not isotopic to $\id_S$.
\end{problem}





\newpage

\section{Glossary}

\begin{definition}[Equivariant maps]
    Suppose a group $G$ acts on spaces $X$ and $Y$, and let $f \colon X
    \to Y$ be a map. Then  $f$ is said to be equivariant if
    $f (g \cdot x) = g \cdot  f(x)$ for all $x \in X$ and all $g \in G$.
\end{definition}

\begin{definition}[Closed surface]
    A \textit{closed surface} is a surface that is compact
    and without boundary.
\end{definition}

\begin{definition}[Isotopy]
    A topological isotopy is a homotopy
    $F \colon X \times I \to Y$ such that for each $t_0 \in I$,
    $F(x,t_0) \colon X \to Y$ is a topological embedding (homeomorphism onto
    some subspace of $Y$ ).

    Two embeddings $f,g \colon X \to Y$ are said to be
    isotopic if there exists an isotopy $F \colon X \times I
    \to Y$ such that $F(x,0) = f(x)$ and $F(x,1) = g(x)$.
\end{definition}


\begin{definition}[Orientation]
    A closed $n$-manifold $M$ is called orientable
    if $H_n \left( M ; \mathbb{Z} \right) = \mathbb{Z}$.
    The choice of generator $\left[ M \right] $ in
    $\mathbb{Z}$ is called an orientation, and the generator
    is called the fundamental class of $M$. A manifold
    together with a choice of orientation is
    called oriented. A compact $n$-manifold $M$ with
    boundary is called orientable if
    $H_n \left( M, \partial M ; \mathbb{Z} \right) = \mathbb{Z}$.
    The choice of generator $\left[ M, \partial M \right] $ in
    $\mathbb{Z}$ is called an orientation, and
    $\left[ M, \partial M \right] $ is referred to as
    the fundamental class of $M$.

    A smooth manifold $M$ is orientable if and only if
    the restriction of its tangent bundle to every smooth
    curve is trivial.

    \begin{remark}[]
        This makes sense since T. Radó showed that every surface
        is triangulable and it is clear then that the $2$-cycles
        form a cyclic group. A choice of generator corresponds
        to choosing an orientation of each $2$-simplex in the
        triangulation (compatibly).
    \end{remark}
\end{definition}







\newpage

\bibliography{mcg}
\end{document}
