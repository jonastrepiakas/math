\documentclass[reqno]{amsart}
\usepackage{amscd, amssymb, amsmath, amsthm}
\usepackage{graphicx}
\usepackage[colorlinks=true,linkcolor=blue]{hyperref}
\usepackage[utf8]{inputenc}
\usepackage[T1]{fontenc}
\usepackage{textcomp}
\usepackage{babel}
%% for identity function 1:
\usepackage{bbm}
%%For category theory diagrams:
\usepackage{tikz-cd}


\setlength\parindent{0pt}

\pdfsuppresswarningpagegroup=1

\newtheorem{theorem}{Theorem}[section]
\newtheorem{lemma}[theorem]{Lemma}
\newtheorem{proposition}[theorem]{Proposition}
\newtheorem{corollary}[theorem]{Corollary}
\newtheorem{conjecture}[theorem]{Conjecture}

\theoremstyle{definition}
\newtheorem{definition}[theorem]{Definition}
\newtheorem{example}[theorem]{Example}
\newtheorem{exercise}[theorem]{Exercise}
\newtheorem{problem}[theorem]{Problem}
\newtheorem{question}[theorem]{Question}

\theoremstyle{remark}
\newtheorem*{remark}{Remark}
\newtheorem*{note}{Note}
\newtheorem*{solution}{Solution}



%Inequalities
\newcommand{\cycsum}{\sum_{\mathrm{cyc}}}
\newcommand{\symsum}{\sum_{\mathrm{sym}}}
\newcommand{\cycprod}{\prod_{\mathrm{cyc}}}
\newcommand{\symprod}{\prod_{\mathrm{sym}}}

%Linear Algebra

\DeclareMathOperator{\Span}{span}
\DeclareMathOperator{\im}{im}
\DeclareMathOperator{\diag}{diag}
\DeclareMathOperator{\Ker}{Ker}
\DeclareMathOperator{\ob}{ob}
\DeclareMathOperator{\Hom}{Hom}
\DeclareMathOperator{\Mor}{Mor}
\DeclareMathOperator{\sk}{sk}
\DeclareMathOperator{\Vect}{Vect}
\DeclareMathOperator{\Set}{Set}
\DeclareMathOperator{\Group}{Group}
\DeclareMathOperator{\Ring}{Ring}
\DeclareMathOperator{\Ab}{Ab}
\DeclareMathOperator{\Top}{Top}
\DeclareMathOperator{\hTop}{hTop}
\DeclareMathOperator{\Htpy}{Htpy}
\DeclareMathOperator{\Cat}{Cat}
\DeclareMathOperator{\CAT}{CAT}
\DeclareMathOperator{\Cone}{Cone}
\DeclareMathOperator{\dom}{dom}
\DeclareMathOperator{\cod}{cod}
\DeclareMathOperator{\Aut}{Aut}
\DeclareMathOperator{\Mat}{Mat}
\DeclareMathOperator{\Fin}{Fin}
\DeclareMathOperator{\rel}{rel}
\DeclareMathOperator{\Int}{Int}
\DeclareMathOperator{\sgn}{sgn}
\DeclareMathOperator{\Homeo}{Homeo}
\DeclareMathOperator{\SHomeo}{SHomeo}
\DeclareMathOperator{\PSL}{PSL}
\DeclareMathOperator{\Bil}{Bil}
\DeclareMathOperator{\Sym}{Sym}
\DeclareMathOperator{\Skew}{Skew}
\DeclareMathOperator{\Alt}{Alt}
\DeclareMathOperator{\Quad}{Quad}
\DeclareMathOperator{\Sin}{Sin}
\DeclareMathOperator{\Supp}{Supp}
\DeclareMathOperator{\Char}{char}
\DeclareMathOperator{\Teich}{Teich}
\DeclareMathOperator{\GL}{GL}
\DeclareMathOperator{\tr}{tr}
\DeclareMathOperator{\codim}{codim}


%Row operations
\newcommand{\elem}[1]{% elementary operations
\xrightarrow{\substack{#1}}%
}

\newcommand{\lelem}[1]{% elementary operations (left alignment)
\xrightarrow{\begin{subarray}{l}#1\end{subarray}}%
}

%SS
\DeclareMathOperator{\supp}{supp}
\DeclareMathOperator{\Var}{Var}

%NT
\DeclareMathOperator{\ord}{ord}

%Alg
\DeclareMathOperator{\Rad}{Rad}
\DeclareMathOperator{\Jac}{Jac}

%Misc
\newcommand{\SL}{{\mathrm{SL}}}
\newcommand{\mobgp}{{\mathrm{PSL}_2(\mathbb{C})}}
\newcommand{\id}{{\mathrm{id}}}
\newcommand{\MCG}{{\mathrm{MCG}}}
\newcommand{\PMCG}{{\mathrm{PMCG}}}
\newcommand{\SMCG}{{\mathrm{SMCG}}}
\newcommand{\ud}{{\mathrm{d}}}
\newcommand{\Vol}{{\mathrm{Vol}}}
\newcommand{\Area}{{\mathrm{Area}}}
\newcommand{\diam}{{\mathrm{diam}}}
\newcommand{\End}{{\mathrm{End}}}


\newcommand{\reg}{{\mathtt{reg}}}
\newcommand{\geo}{{\mathtt{geo}}}

\newcommand{\tori}{{\mathcal{T}}}
\newcommand{\cpn}{{\mathtt{c}}}
\newcommand{\pat}{{\mathtt{p}}}

\let\Cap\undefined
\newcommand{\Cap}{{\mathcal{C}}ap}
\newcommand{\Push}{{\mathcal{P}}ush}
\newcommand{\Forget}{{\mathcal{F}}orget}


\title{Problem Sheet}

\date{}

\begin{document}

\maketitle

\section{Geometric Representations}


\begin{problem}[Easy]
    True or false: the crosscap transposition representation
    is induced by a Yang-Baxter operator on
    some category of surfaces.
\end{problem}

\begin{problem}[Easy]
    True or false: the standard twist representations
    is induced by a Yang-Baxter operator on
    some category of surfaces.
\end{problem}

\begin{problem}[There is an easy and a harder way to
    do it]
    Show that for $g$ odd, the standard
    twist representation $\rho_C \colon
    \mathcal{B}_g \to \MCG \left( N_{g,b} \right) $ is
    the same as the Birman-Hilden embedding
    $\mathcal{B}_g \hookrightarrow 
    S_{\frac{g-1}{2}, b-1} \# M$ into the
    orientable factor.
\end{problem}

\begin{problem}[]
    Can you extend the previous problem to the
    case when $g$ is even?
\end{problem}

\begin{problem}[]
    Experiment yourself by constructing your own
    geometric representations from Yang-Baxter
    operators on the category of decorated surfaces.
\end{problem}

\begin{problem}[Medium]
    Read up on the category of bidecorated surfaces.
    See if you can recover the Birman-Hilden
    embedding in this category from
    a Yang-Baxter operator.
\end{problem}

\section{Mapping Class Group Examples}

\begin{problem}[Easy (might require algtop)]
    Give an example of a surface $S$ of finite type and
    a self-diffeomorphism $\varphi $ of $S$ such that
    $\varphi $ is homotopic to $\id_S$ but not isotopic
    to $\id_S$.
\end{problem}

\begin{proposition}[]
    Any two essential simple proper arcs
    in $S_{0,3}$ with the same endpoints are
    isotopic. Any two essential arcs that both
    start and end at the same marked point
    of $S_{0,3}$ are also isotopic.
\end{proposition}

\begin{problem}[]
    Given the previous proposition, find an
    isomorphism
    $\MCG\left( S_{0,3} \right) \cong \Sigma_{3}$
\end{problem}

\begin{problem}[]
    Show similarly that 
    $\MCG \left( S_{0,3} \right) \cong
    \mathbb{Z} / 2$.
\end{problem}

\section{Birman-Hilden}

\begin{problem}[]
    Prove the Birman exact sequence
    \[
    1 \to \pi_1\left( S,x \right) 
    \stackrel{\Push}{\to } \MCG (S,x) 
    \stackrel{\Forget}{\to }
    \MCG (S) \to 1
    \] 
\end{problem}

This generalizes to an exact sequence (you don't have to 
prove this)
\[
1 \to \pi_1 \left( C \left( S,n \right)  \right) 
\stackrel{\Push}{\to }
\MCG \left( S, \left\{ x_1, \ldots, x_n \right\}  \right) 
\stackrel{\Forget}{\to }
\MCG (S) \to 1
\] 
What do you obtain when we let $S = D^2$?

Let
$\iota$ be the hyperelliptic involution
of $S_{g,1}$. Let
 \[
\SHomeo^{+} \left( S_{g,1} \right) 
= C_{\Homeo^{+} \left( S_{g,1} , \partial S_{g,1} \right) }
\left( \iota \right) 
\] 
Define
\[
\SMCG \left( S_{g,1} \right) =
\SHomeo^{+} \left( S_{g,1} \right) /
\text{isotopy}
\] 
Suppose we are given that two symmetric homeomorphisms
are isotopic if and only if they are symmetrically isotopy,
i.e., isotopic in
$\SHomeo^{+} \left( S_{g,1} \right) $.
Derive the Birman-Hilden theorem:
\[
\SMCG \left( S_{g,1} \right) 
\cong B_{2g+1}.
\] 

    %\bibliography{../refs.bib}
\end{document}
