\documentclass[reqno]{amsart}
\usepackage{amscd, amssymb, amsmath, amsthm}
\usepackage{graphicx}
\usepackage[colorlinks=true,linkcolor=blue]{hyperref}
\usepackage[utf8]{inputenc}
\usepackage[T1]{fontenc}
\usepackage{textcomp}
\usepackage{babel}
\usepackage{quiver}
%% for identity function 1:
\usepackage{bbm}
%%For category theory diagrams:
\usepackage{tikz-cd}
\usepackage{todonotes}

\usepackage[backend=biber]{biblatex}
\addbibresource{string-topology.bib}


\setlength\parindent{0pt}

\pdfsuppresswarningpagegroup=1

\newtheorem{theorem}{Theorem}[section]
\newtheorem{lemma}[theorem]{Lemma}
\newtheorem{proposition}[theorem]{Proposition}
\newtheorem{corollary}[theorem]{Corollary}
\newtheorem{conjecture}[theorem]{Conjecture}

\theoremstyle{definition}
\newtheorem{definition}[theorem]{Definition}
\newtheorem{example}[theorem]{Example}
\newtheorem{exercise}[theorem]{Exercise}
\newtheorem{problem}[theorem]{Problem}
\newtheorem{question}[theorem]{Question}

\theoremstyle{remark}
\newtheorem*{remark}{Remark}
\newtheorem*{note}{Note}
\newtheorem*{solution}{Solution}



%Inequalities
\newcommand{\cycsum}{\sum_{\mathrm{cyc}}}
\newcommand{\symsum}{\sum_{\mathrm{sym}}}
\newcommand{\cycprod}{\prod_{\mathrm{cyc}}}
\newcommand{\symprod}{\prod_{\mathrm{sym}}}

%Linear Algebra

\DeclareMathOperator{\Span}{span}
\DeclareMathOperator{\im}{im}
\DeclareMathOperator{\diag}{diag}
\DeclareMathOperator{\Ker}{Ker}
\DeclareMathOperator{\ob}{ob}
\DeclareMathOperator{\Hom}{Hom}
\DeclareMathOperator{\Mor}{Mor}
\DeclareMathOperator{\sk}{sk}
\DeclareMathOperator{\Vect}{Vect}
\DeclareMathOperator{\Set}{Set}
\DeclareMathOperator{\Group}{Group}
\DeclareMathOperator{\Ring}{Ring}
\DeclareMathOperator{\Ab}{Ab}
\DeclareMathOperator{\Top}{Top}
\DeclareMathOperator{\hTop}{hTop}
\DeclareMathOperator{\Htpy}{Htpy}
\DeclareMathOperator{\Cat}{Cat}
\DeclareMathOperator{\CAT}{CAT}
\DeclareMathOperator{\Cone}{Cone}
\DeclareMathOperator{\dom}{dom}
\DeclareMathOperator{\cod}{cod}
\DeclareMathOperator{\Aut}{Aut}
\DeclareMathOperator{\Mat}{Mat}
\DeclareMathOperator{\Fin}{Fin}
\DeclareMathOperator{\rel}{rel}
\DeclareMathOperator{\Int}{int}
\DeclareMathOperator{\sgn}{sgn}
\DeclareMathOperator{\Homeo}{Homeo}
\DeclareMathOperator{\SHomeo}{SHomeo}
\DeclareMathOperator{\PSL}{PSL}
\DeclareMathOperator{\Bil}{Bil}
\DeclareMathOperator{\Sym}{Sym}
\DeclareMathOperator{\Skew}{Skew}
\DeclareMathOperator{\Alt}{Alt}
\DeclareMathOperator{\Quad}{Quad}
\DeclareMathOperator{\Sin}{Sin}
\DeclareMathOperator{\Supp}{Supp}
\DeclareMathOperator{\Char}{char}
\DeclareMathOperator{\Teich}{Teich}
\DeclareMathOperator{\GL}{GL}
\DeclareMathOperator{\tr}{tr}
\DeclareMathOperator{\codim}{codim}
\DeclareMathOperator{\coker}{coker}
\DeclareMathOperator{\corank}{corank}
\DeclareMathOperator{\rank}{rank}
\DeclareMathOperator{\Diff}{Diff}
\DeclareMathOperator{\Bun}{Bun}
\DeclareMathOperator{\Sm}{Sm}
\DeclareMathOperator{\Fr}{Fr}
\DeclareMathOperator{\Cob}{Cob}
\DeclareMathOperator{\Ext}{Ext}
\DeclareMathOperator{\Tor}{Tor}
\DeclareMathOperator{\Conf}{Conf}
\DeclareMathOperator{\UConf}{UConf}
\DeclareMathOperator{\Map}{Map}
\DeclareMathOperator{\Ori}{Ori}



%Row operations
\newcommand{\elem}[1]{% elementary operations
\xrightarrow{\substack{#1}}%
}

\newcommand{\lelem}[1]{% elementary operations (left alignment)
\xrightarrow{\begin{subarray}{l}#1\end{subarray}}%
}

%SS
\DeclareMathOperator{\supp}{supp}
\DeclareMathOperator{\Var}{Var}

%NT
\DeclareMathOperator{\ord}{ord}

%Alg
\DeclareMathOperator{\Rad}{Rad}
\DeclareMathOperator{\Jac}{Jac}

%Misc
\newcommand{\SL}{{\mathrm{SL}}}
\newcommand{\mobgp}{{\mathrm{PSL}_2(\mathbb{C})}}
\newcommand{\id}{{\mathrm{id}}}
\newcommand{\MCG}{{\mathrm{MCG}}}
\newcommand{\PMCG}{{\mathrm{PMCG}}}
\newcommand{\SMCG}{{\mathrm{SMCG}}}
\newcommand{\ud}{{\mathrm{d}}}
\newcommand{\Vol}{{\mathrm{Vol}}}
\newcommand{\Area}{{\mathrm{Area}}}
\newcommand{\diam}{{\mathrm{diam}}}
\newcommand{\End}{{\mathrm{End}}}


\newcommand{\reg}{{\mathtt{reg}}}
\newcommand{\geo}{{\mathtt{geo}}}

\newcommand{\tori}{{\mathcal{T}}}
\newcommand{\cpn}{{\mathtt{c}}}
\newcommand{\pat}{{\mathtt{p}}}

\let\Cap\undefined
\newcommand{\Cap}{{\mathcal{C}}ap}
\newcommand{\Push}{{\mathcal{P}}ush}
\newcommand{\Forget}{{\mathcal{F}}orget}




\begin{document}

\section{Orientations}

We begin by attempting to give complete rigour and detail
to the definitions of orientation and the many connected
theorems.\\
\linebreak
For this section, we will follow 
\cite{Bredon} and \cite{Dieck}

\begin{definition}[Local Homology Group]
    For $h_*(-)$ a homology theory
    and an $n$-manifold $M$, groups of the form
    $h_k(M , M-\left\{ x \right\} )$ are called
    \textit{local homology groups}.
\end{definition}

For a chart $\varphi  \colon U \to \mathbb{R}^{n}$ 
on $M$ centered at $x$, we get by excision that
\[
h_k(M, M-\left\{ x \right\} ) 
\cong h_k\left( U, U- \left\{ x \right\}  \right) 
\stackrel{\varphi_*}{\to} h_k\left( \mathbb{R}^{n},
\mathbb{R}^{n} - \left\{ 0 \right\} \right) .
\] 
Hence for singular homology, we obtain
$H_n\left( M, M - \left\{ x \right\} ; G \right) 
\cong G$.


\begin{definition}[Local $R$-orientation]
    Let $R$ be a commutative ring.
    A generator of
    $H_n\left( M, M - \left\{ x \right\} ; R \right) 
    \cong R$ is called a 
    \textit{local $R$-orientation} of $M$ about $x$.
\end{definition}

Let $K \subset L \subset M$. The homomorphism
$r_{K}^{L} \colon h_k (M, M-L) \to 
h_k(M, M-K)$ induced by inclusion is
called restriction. We write
$r_{x}^{L}$ when $K = \left\{ x \right\} $.

\begin{proposition}[]
    When $A$ is a compact, convex set contained
    in some chart $\mathbb{R}^{n} \subset M$, then
    $r_{x}^{A}$ is an isomorphism for each
    $x \in A$ and the
    groups are isomorphic to the coefficient group
    $G$.
\end{proposition}

\begin{proof}
    $A$ is contained in the interior of some
    closed $n$-disk $D \subset \mathbb{R}^{n} \subset M$.
    Thus there is a commutative diagram

    \begin{equation*}
    \begin{tikzcd}
        h_n(M, M-A) \ar[r] & h_n(M, M- \left\{ x \right\} )\\
        h_n(\mathbb{R}^{n}, \mathbb{R}^{n} - A) 
        \ar[u, "\cong"] \ar[r] & 
        h_n(\mathbb{R}^{n}, \mathbb{R}^{n} - \left\{ x \right\} )
        \ar[u, "\cong"] \\
        h_n(D, \partial D) 
        \ar[u, "\cong"] \ar[r, equal] &
        h_n(D, \partial D) \ar[u, "\cong"]
    \end{tikzcd}
    \end{equation*}
\end{proof}

\begin{definition}[Orientation bundle]
    We construct a covering
    $\omega \colon h_k(M, M - \bullet) \to  M$.
    Define
    \[
    h_k(M, M - \bullet) =
    \bigsqcup_{x \in M} 
    h_k(M, M- \left\{ x \right\} )
    \] 
    where $h_k(M, M - \left\{ x \right\} )$ is
    the fiber over $x$ and is given
    the discrete topology.

    Let $U$ be an open neighborhood of
    $x$ such that $r_{y}^{U}$ is an isomorphism
    for each $y \in U$.
    Define bundle charts
    \[
    \varphi_{x,U} \colon U \times G
    \to \omega^{-1} (U), \quad
    (y,a) \mapsto r_y^{U}\left( r_x^{U} \right)^{-1} (a).
    \] 
    We then give $h_k(M, M - \bullet)$ the
    topology that makes
    $\varphi_{x,U}$ in a homeomorphism onto
    an open subset. In particular, since
    $h_k(M,M-x)$ is given the
    discrete topology, this is equivalent to
    the map $\varphi_{x,U}(-,\alpha)$ being a homeomorphism
    onto an open subset for each
    $\alpha \in 
    h_k(M, M- x)$.
    It then remains to show that the transition maps
    \[
    \varphi_{y,V}^{-1} \varphi_{x,U} \colon
    (U \cap V) \times 
    h_k(M, M - \left\{ x \right\} )
    \to \left( U \cap V \right) \times 
    h_k(M, M- \left\{ y \right\} )
    \] 
    are continuous.

    Let $z \in U \cap V$, and choose $W$ such that
    $z \in W \subset U \cap V$ and
    $r_{w}^{W}$ is an isomorphism for each $w \in W$.

    Consider the diagram

    \begin{equation*}
    \begin{tikzcd}
        h_k(M, M - x) & \ar[l, "r_x^{U}"] 
        h_k(M, M -U) \ar[r, "r_w^{U}"] 
        \ar[d, "r_{W}^{U}"] & 
        h_k(M, M - w) \\
        & h_k(M, M -W) \ar[ru, "r_w^{W}"] 
        & h_k(M, M - V) \ar[l, "r_W^{V}"] 
        \ar[u, "r_w^{V}"] \ar[d, "r_y^{V}"] \\
        && h_k(M,M-y)
    \end{tikzcd}
    \end{equation*}

    Let
    $\varphi_{x,U,p} \colon
    h_k(M, M- x) \to 
    \omega^{-1}(p)$ be defined by
    \[
    \varphi_{x,U,p}(y) = 
    \varphi_{x,U}(p,y).
    \] 
    Then for $w \in U \cap V$, we have
    \[
    \varphi_{x,U,w}^{-1}
    \varphi_{y,V,w} = r_y^{V} (r_W^{V})^{-1} (r_w^{W})^{-1}
    r_w^{W} r_{W}^{U} (r_x^{U})^{-1} = 
    r_y^{V} (r_W^{V})^{-1} r_{W}^{U} r_x^{U}
    \] 
    Firstly, this coincides with the operation of
    an element of the coefficient group
    $G$ since it is an isomorphism
    $G \to G$, and secondly, 
    note that this does not depend on $w$,
    so the map
    \[
    g_{x,U,y,V} \colon U \cap V \to 
    G
    \]  
    defined by
    $g_{x,U,y,V}(p) = \varphi_{x,U,p}^{-1}
    \varphi_{y,V,p}$ is constant, hence continuous.\\
    \linebreak
    
    Thus $\omega$ is indeed a covering map.

    But even moreso, the fibers are groups, so
    for $A \subset M$, denote by
    $\Gamma(A)$ the set of continuous
    sections over $A$ of
    $\omega$. If
    $s$ and $t$ are section, we can define
    $(s+t)(a) = s(a) + t(a)$. Then
    $s+t$ is again continuous, hence $\Gamma(A)$ is an
    abelian group.

    Denote by $\Gamma_c(A) \subset 
    \Gamma(A)$ the subgroup of sections
    with compact support, i.e., the sections
    which have values $0$ in the fiber away from
    a compact set.
\end{definition}

\begin{proposition}[]\label{Prop:DJIXOZ}
    Let $z \in h_k(M, M- U)$. Then
    $y \mapsto r_y^{U} z
    \in h_k(M, M-y) \subset 
    h_k(M, M - \bullet)$ is a continuous section of $\omega$
    over $U$.
\end{proposition}


\begin{proof}
    The map
    $U \to U \times G$ by
    $y \mapsto (y, r_x^{U}z)$ is constant in the second coordinate,
    hence clearly continuous. Now composing with
    $\varphi_{x,U}$ gives us the section in question.
\end{proof}


\subsection{Homological Orientation}

If we specify to singular homology with coefficient
group $R$, and again let $M$ be an $n$-manifold and $A \subset M$,
then we can define an orientation along
$A$ as follows

\begin{definition}[$R$-orientation of $M$ along $A$]
    An \textit{$R$-orientation of $M$ along $A$} is a section
    $s \in \Gamma(A; R)$ of 
    $\omega \colon H_n \left( M , M - \bullet ; R \right) 
    \to M$ such that
    $s(a) \in 
    H_n(M, M-a; R) \cong R$ is a generator
    for each $a \in A$.\\


    Thus $s$ glues together the local orientations in a 
    continuous manner. 

    When $A = M$, we call $s$ an \textit{$R$-orientation of $M$ }.
\end{definition}

\begin{definition}[Orientation covering]
    Let $\Ori(M) \subset H_n(M, M - \bullet; \mathbb{Z})$ be the
    subset of all generators of all fibers. Then
    the restriction
    $\Ori (M) \to M$ of $\omega$ gives a $2$-fold
    covering of $M$, called the
    \textit{orientation covering} of $M$.
\end{definition}

\begin{proposition}[]
    The following are equivalent:


    \begin{enumerate}
        \item $M$ is orientable
        \item $M$ is orientable along compact subsets.
        \item The orientation covering is a trivial 
            $2$-fold covering map.
        \item The covering $\omega \colon
            H_n(M, M - \bullet ; \mathbb{Z}) \to M$ is
            a trivial covering map.
    \end{enumerate}

\end{proposition}


\begin{proof}
    $(1) \implies (2)$ is a subcase.\\
    $(2) \implies (3)$. The orientation covering is
    trivial if and only if the covering over
    each component is trivial, so we may
    assume that $M$ is connected. 
    Now, if a $2$-fold covering
    $ \tilde{M} \to M$ is trivial, then $\tilde{M}$ splits
    as $M \times \left\{ p,q \right\} $, and
    so $\tilde{M}$ cannot be connected.
    Conversely, if $\tilde{M}$ is not connected, then
    the covering restricted to each component must be
    a covering map, so the covering splits as a
    trivial covering.\\
    Suppose then that
    $\Ori(M) \to M$ is non-trivial. Since
    $\Ori(M)$ is then connected, we can choose
    a path $\gamma$ 
    in $\Ori(M)$ between two points of a given
    fiber. The image $S$ of such a path is compact and connected,
    and the covering is non-trivial over $S$, so
    by assumption $(2)$, the orientation covering has
    a section $s$ over $S$, but then
    $\gamma(0) = 
    s \left( \omega (\gamma(0)) \right) = 
    s \left( \omega \left( \gamma(1) \right)  \right) 
    = \gamma(1)$, which gives a contradiction.\\
    $(3) \implies (4)$. 

    Let $s \colon
    M \to \Ori (M) \cong M \times \left\{ -1,1 \right\} $ be
    the section $m \mapsto (m,1)$.

Now define a map $\varphi \colon M \times \mathbb{Z} \to 
H_n(M, M-\bullet; \mathbb{Z})$ by
$\varphi (m,k) = ks(m)$. This is a bijective map by assumption on $s$ being a section. 
It is furthermore continuous since $s$ is continuous and
since fiber-wise operations in $H_n(M,M-\bullet; \mathbb{Z})$
is continuous.
Furthermore, it is also a morphism between coverings since it commutes with the
projections: $\pi_M = \omega \circ \varphi$.

Lastly, one must show that it also has a continuous inverse. For this, we may take an open basis set in $M \times \mathbb{Z}$ - say $U\times \{k\}$, where $\bar{U}$ is a convex subset of $\mathbb{R}^n \subset M$.

Since $\varphi$ is bijective, we obtain that $\varphi(U \times \{k\}) = k s(U)
= U_\alpha$ if we choose $\alpha$ to be the element in $H_n(M, M-U) \cong \mathbb{Z}$
which maps to $k$ under $r_{x,U}$ for $x \in U$.
And by assumption, $U_\alpha$ is a basis open set for
the topology on $H_n(M, M - \bullet ; \mathbb{Z})$.

Hence $\varphi$ is a homeomorphism, and even an isomorphism of covering spaces
in the sense that $\pi_M = \omega \circ \varphi$.

\begin{note}
    We could also say that it is trivial since
    every point is in the image of some section.
\end{note}

$(4) \implies (1)$ : If $\omega$ is trivial, then
it has a section with constant value in the set
of generators.

\end{proof}



\subsection{Homology in the Dimension of the Manifold}

Let $M$ be an $n$-manifold and $A \subset M$ a closed subset.
We will in this section use singular homology with coefficients
in an abelian group $G$.


\begin{proposition}[]
    For each $\alpha \in H_n(M, M - A ; G)$, the
    section
    \[
    J^{A}(\alpha) \colon
    A \to H_n(M, M - \bullet; G),
    \quad x \mapsto r_x^{A} (\alpha)
    \] 
    of $\omega$ over $A$ is continuous and has
    compact support.
\end{proposition}

\begin{proof}
    Choose a representative
    $c \in \Delta_n(M;G)$ representing $\alpha$.
    There exists a compact set $K$ such that
    $c$ is contained in $K$.
    Suppose $A - K$ is nonempty, and let $x \in A - K$.
    Then the image of $c$ under
    \[
    \Delta_n (K ; G) \to 
    \Delta_n(M;G) \to \Delta_n (M,K;G) 
    \to \Delta_n (M, M - x ; G)
    \] 
    is zero since
    $K \subset M - x$. Since this image
    represents
    $r_x^{A}$, the support of $J^{A}(\alpha)$ is
    contained in $A \cap K$ which is compact.\\
    If $A - K$ is empty, $K$ contains $A$, and then
    the support of $J ^{A}(\alpha)$ is a closed
    subset of a compact space, hence compact.\\
    \linebreak
    The continuity follows from the more general case of
    Proposition \ref{Prop:DJIXOZ}.
\end{proof}

Thus we obtain a homomorphism
\[
J^{A} \colon H_n(M, M - A; G) \to 
\Gamma_c (A;G), \quad
\alpha \mapsto \left( x \mapsto r_x^{A}(\alpha) \right) .
\] 



\subsubsection{Direct Limits}

\begin{definition}[]
    Let $D$ be a directed set and $G_{\alpha}$ an abelian group
    defined for each $\alpha \in D$. Suppose
    we are given homomorphisms
    $f_{\beta, \alpha} \colon G_{\alpha} \to G_{\beta}$ 
    for each $\beta > \alpha$ in $D$. Assume that for all
    $\gamma > \beta > \alpha$ in $D$, we have
    $f_{\gamma, \beta} f_{\beta, \alpha} = f_{\gamma, \alpha}$.
    Such a system is called a \textit{direct system} of abelian
    groups. Then $G = \lim_{\rightarrow} G_{\alpha}$ is defined
    to be the quotient group of the direct sum
    $G = \bigoplus G_{\alpha}$ modulo the relations
    $f_{\beta, \alpha}(g) \sim g$ for all $g \in G_\alpha$ and
    all $\beta > \alpha$.

    \begin{note}
        Hence the direct limit is just the colimit of the direct
        system.
    \end{note}
\end{definition}

\begin{proposition}[]\label{Prop:OXKXKCL}
    Suppose we are given an abelian group
    $A$ with homomorphisms
    $h_{\alpha} \colon G_{\alpha} \to A$ 
    such that the cocone commutes.
    Since $\lim_{\rightarrow} G_{\alpha}$ is the colimit, we
    have a unique induced homomorphism
    $h \colon \lim_{\rightarrow} G_{\alpha} \to A$.
    Then
    \begin{enumerate}
        \item $\im h = 
            \left\{ a \in A \mid a = h_{\alpha}(g) \text{
            for some } g \text{ and } \alpha \right\} 
            = \bigcup \im h_{\alpha} $.
        \item $\ker h = 
            \left\{ g \in \lim_{\rightarrow} G_{\alpha}  \mid 
            \exists \alpha \text{ and } 
        g_{\alpha} \in G_{\alpha} \colon g = 
    i_{\alpha}(g_{\alpha}) \text{ and }
h_{\alpha}(g_{\alpha}) = 0\right\} = 
\bigcup i_{\alpha}(\ker h_{\alpha}) $.
    \end{enumerate}
\end{proposition}

\begin{proof}
    Define $h (g_{\alpha}) = 
    h_{\alpha}(g_{\alpha})$. Then if
    $f_{\beta,\alpha}(g_{\alpha}) \sim g_{\alpha}$, we have
    $h\left( g_{\alpha} \right) 
    = h_{\alpha}\left( g_{\alpha} \right) 
    = h_{\beta} \circ f_{\beta,\alpha}(g_{\alpha})
    = h \left( f_{\beta,\alpha}(g_{\alpha}) \right) $, so
    $h$ respects the equivalence relations, thus it
    is well-defined.\\
    Now property (1) is clear by the way we defined $h$.

    As for (2), note that 
    if $g$ represents the equivalence class
    of $g_{\alpha}$ and
    $h(g) = 0$, then
    $h_{\alpha}\left( g_{\alpha} \right) = 0$ which
    is what (2) is saying.
\end{proof}

\begin{corollary}
    In the situation of Proposition
    \ref{Prop:OXKXKCL},
    $h \colon \lim_{\rightarrow} G_{\alpha} \to A$ is an
    isomorphism if and only if the following two statements
    hold true:
    \begin{enumerate}
        \item $\forall a \in A, \exists \alpha \in D$ and
            $g_{\alpha} \in G_{\alpha} \colon
            h_{\alpha}(g_{\alpha}) = a$, and
        \item if $h_{\alpha}(g_{\alpha}) = 0$ then
            $\exists \beta > \alpha \colon f_{\beta,\alpha}
            (g_{\alpha}) = 0$.
    \end{enumerate}
\end{corollary}

\begin{theorem}[]
    The direct limit is an exact functor.
    So if we have direct systems
    $\left\{ A_{\alpha}' \right\} ,
    \left\{ A_{\alpha} \right\} $ and
    $\left\{ A_{\alpha}'' \right\} $ based on the same
    directed set, and if we have
    an exact sequence
    $A_{\alpha}' \to A_{\alpha} \to A_{\alpha}''$ for each
    $\alpha$, where the maps commute with the ones
    defining the direct systems, then the induced sequence
    \[
    \lim_{\rightarrow} A_{\alpha}' \to 
    \lim_{\rightarrow} A_{\alpha} \to 
    \lim_{\rightarrow} A_{\alpha}''
    \] 
    is exact.
\end{theorem}

\begin{proof}
    We have the following diagram, where all
    maps commute.

\[\begin{tikzcd}
	{A_{\beta}'} & {A_{\beta}} & {A_{\beta}''} \\
	{\lim_{\rightarrow}A_{\alpha}'} & {\lim_{\rightarrow}A_{\alpha}} & {\lim_{\rightarrow}A_{\alpha}''}
	\arrow[from=1-1, to=1-2]
	\arrow[from=1-1, to=2-1]
	\arrow[from=1-2, to=1-3]
	\arrow[from=1-2, to=2-2]
	\arrow[from=1-3, to=2-3]
	\arrow[from=2-1, to=2-2]
	\arrow[from=2-2, to=2-3]
\end{tikzcd}\]

Suppose
$a \in \lim_{\rightarrow} A_*$ is mapped to zero
in $\lim_{\rightarrow} A_*''$.
Then there exists 
$g \in \lim_{\rightarrow}A_{\alpha}$ such that
there exists $\beta$ and
$g_{\beta} \in A_{\beta}$ such that
$g = i_{\beta}(g_{\beta})$ and
$h_{\beta}(g_{\beta}) = 0$.

Recall here that
$h_{\beta}$ is a homomorphism
$A_{\beta} \to \lim_{\rightarrow}A_*^{''}$ and
$i_{\beta}$ is the inclusion
$G_{\beta} \to \lim_{\rightarrow}G_{\alpha}$.

By commutativity of the diagram, there
then exists $k_{\beta} \in 
A_{\beta}'$ such that

$i_{\beta} \left( d_{\beta} (k_{\beta}) \right) 
= d_{\lim_{\rightarrow}} i_{\beta}^{'} (k_{\beta})$.
Hence the kernel is contained in the image.

Now suppose let $\tilde{k} = 
d_{\lim_{\rightarrow}} (k) \in 
\lim_{\rightarrow}A_*$.

Then $\tilde{k} = 
i_{\beta}\left( d (\overline{k}) \right) 
= d_{\lim_{\rightarrow}}
i_{\beta}' \left( \overline{k} \right) $ for
some $\overline{k} \in A_{\beta}'$.


But now
\[
d_{\lim_{\rightarrow}} (\tilde{k})
= d_{\lim_{\rightarrow}}
i_{\beta} \left( d \left( \overline{k} \right)  \right) 
= i_{\beta}'' 
d\left( d \left( \overline{k} \right)  \right) 
= i_{\beta}'' (0) = 0.
\] 
\end{proof}

\begin{theorem}[]
    Suppose we are given two directed sets
    $D$ and $E$. Define an order on
    $D \times E$ by 
    $\left( \alpha, \beta \right) \ge 
    \left( \alpha', \beta' \right) $ if and only if
    $\alpha \ge \alpha'$ and
    $\beta \ge  \beta'$. Suppose
    $G_{\alpha, \beta}$ is a direct system
    based on $D \times E$. Then the maps
    $G_{\alpha,\beta} \to 
    \lim_{\rightarrow, \beta} G_{\alpha, \beta}
    \to \lim_{\rightarrow, \alpha}
    \left( \lim_{\rightarrow, \beta} G_{\alpha,\beta} \right) $
    induce an isomorphism
    \[
    \lim_{\rightarrow, \alpha,\beta} G_{\alpha, \beta}
    \stackrel{\cong}{\to} 
    \lim_{\rightarrow, \alpha} \left( 
    \lim_{\rightarrow, \beta} 
G_{\alpha, \beta}\right) .
    \]  
\end{theorem}

\begin{proof}
    \todo{}
\end{proof}

\begin{proposition}[]\label{Prop:SIDJOXOLLWQU}
    \begin{enumerate}
        \item 
    For $A \supset B$ both closed, the following
    diagram commutes:
    \begin{equation*}
    \begin{tikzcd}
        H_n(M, M - A;G) \ar[r] \ar[d, "J^{A}"] &
        H_n(M, M - B ; G) \ar[d, "J^{B}"] \\
        \Gamma_c (A,
        H_n(M, M - \bullet ; G)) \ar[r] & 
        \Gamma_c \left( B,
        H_n(M, M - \bullet; G) \right) 
    \end{tikzcd}
    \end{equation*}
\item  For $A , B \subset M$ both closed, the sequence
        \begin{align*}
    0 \to \Gamma_c (A \cup B, 
    H_n(M, M - \bullet;G)) 
    &\stackrel{h}{\to} 
    \Gamma_c \left( A, H_n(M, M - \bullet;G) \right) 
    \oplus \Gamma_c \left( 
    B, H_n(M, M - \bullet; G) \right)\\
    &\stackrel{k}{\to} 
    \Gamma_c \left( A \cap B,
    H_n\left( M, M-\bullet; G \right) \right) 
        \end{align*}
    is exact, where $h$ is the sum of restrictions and
    $k$ is the difference of restrictions.
\item If $A_1 \supset A_2 \supset A_3 \supset \ldots$ are
    all compact and $A \bigcap A_i  $, then the restriction
    homomorphisms
    $\Gamma \left( A_i, H_n \left( M, M - \bullet; G \right)  \right) 
    \to \Gamma \left( A, H_n \left( M, M- \bullet; G \right) 
    \right) $ induce an isomorphism
    \[
    \lim_{\rightarrow} \Gamma
    \left( A_i, H_n\left( M, M - \bullet;G \right)  \right) 
    \stackrel{\cong}{\to} 
    \Gamma \left( A, H_n (M, M- \bullet;G) \right) 
    \] 
    \end{enumerate}
\end{proposition}

\begin{proof}
    (1) Let $\alpha \in 
    H_n(M, M - A;G)$, and denote by
    $\iota$ the inclusion $(M, M-A) \hookrightarrow (M, M-B)$.
    Then $\iota_* = r_{B}^{A}$, so
    $J^{B} \left( r_{B}^{A} (\alpha) \right) (x)
    = r_{x}^{B}\left( r_{B}^{A} (\alpha) \right) $.
    On the other hand, 
    $J^{A}(\alpha)|_{B}(x) =
    J^{A}(\alpha)(x) = 
    r_x^{A}(\alpha)$.
    Now, from the composition
    \[
        (M, M - A) \hookrightarrow 
        (M, M - B) \hookrightarrow 
        (M, M - x)
    \] 
    we obtain by taking homology, that
    $r_x^{A} = r_x^{B} r_{B}^{A}$, which gives the
    result.\\
    \linebreak
    
    (2) Firstly, a section that is zero on both
    $A$ and $B$ is then also zero on
    $A \cup  B$, which gives the injective part
    of $h$. Now, 
    suppose $s-t$ is the zero section over $A \cap B$ 
    for $s$ a section over $A$ and $t$ a section over $B$.
    Then $s$ and $t$ agree on $A \cap B$, meaning that
    $s \cup t$ is well-defined and continuous, where
    $s \cup t$ is $s$ on $A$ and $t$ on $B$, and
    $h(s \cup t) = (s,t)$. Likewise, if
    $g$ is a section over $A \cup  B$, then
    $k \circ h(g) = \left( g|_{A} \right)|_{A \cap B}
    - \left( g|_{B} \right)|_{A \cap B}
    = g|_{A \cap B} - g|_{A \cap B}$ is the
    zero section.\\
    \linebreak
    (3) 

\end{proof}


\begin{theorem}[]\label{Thm:OGPAL}
    Let $A \subset M$ be closed. Then
    \begin{enumerate}
        \item $H_i \left( M , M - A; G \right) = 0$ for $i>n$.
        \item $J^{A} \colon H_n (M, M - A, G) \to 
            \Gamma_c (A, 
            H_n\left( M, M - \bullet;G \right) )$ is an isomorphism.
    \end{enumerate}
\end{theorem}



\begin{lemma}[The Bootstrap Lemma]
    Let $P_M (A)$ be a statement about compact sets
    $A$ in a given $n$-manifold $M^{n}$. If
    $(i), (ii), (iii)$ hold, then $P_M(A)$ is true
    for all compact $A$ in $M^{n}$.\\
    If $M^{n}$ is separable metric, and $P_M(A)$ is defined
    for all closed sets $A$, and if 
    $(i), (ii), (iii), (iv)$ hold, then $P_M(A)$ is true
    for all closed sets $A$ in $M$.\\
    For general $M^{n}$, if $P_M(A)$ is defined for all closed
    sets $A$ in $M$, for all $M^{n}$, and if all
    five statement $(i) - (v)$ hold for all $M^{n}$, then
    $P_M(A)$ is true for all closed $A \subset M$ and
    all $M^{n}$.
\end{lemma}


Now note that for a given abelian group $G$ and
$g \in G$, the following maps are natural in
$A \subset M$ (closed):
\[
H_n(M, M- A) \cong H_n \left( M, M- A \right) \otimes
\mathbb{Z} \to H_n(M, M-A) \otimes G
\to H_n(M, M-A;G)
\] 
where the middle map is induced by the homomorphism
$\mathbb{Z} \to G$ taking $1$ to $g$.

In particular, this induces a map
\[
H_n(M, M - \bullet) \to 
H_n(M, M - \bullet; G)
\] 

\begin{lemma}[]\label{Lemma:X2948JJD}
    The sections $\Gamma (A;G)$ of $\omega$ over
    $A$ correspond bijectively to continuous maps
    $\lambda \colon \Ori \left( M \right)|_{A} \to G$ with
    the property $\lambda \circ t = - \lambda$, where
    $t$ acts on $G$ as multiplication by $-1$.
\end{lemma}

\begin{proof}
    We may assume $A$ is connected.\\
    Let $s \in \Gamma \left( A;G \right) $ be a section
    of $\omega$ over $A$. That is, 
    $w \circ s = \id_A$, and
    $s$ is a map $A \to H_n (M, M - \bullet;G)$. We
    can define an associated map
    $\lambda_s \colon \Ori (M)|_{A} \to G$ by
    sending a generator in the fiber $x \in A$ to
    $s(x) \in 
    H_n \left( M, M - \left\{ x \right\} ;G \right) 
    \cong G$. If one chose the other generator, one
    would get the negative of the above map, so
    we have the relation
    $\lambda_s \circ t = - \lambda_s$. Subject to this relation,
    we obtain a well-defined map
    $\Gamma \left( A;G \right) \to 
    S \subset \Hom \left( \Ori \left( M \right) |_A , G \right) $,
    where $S$ is the subset for which
    $\lambda \circ t = - \lambda$ holds.
    This map is certainly injective, since
    the image tells us precisely the value of
    $s$ at any point in $A$.\\
    It is furthermore surjective, since if
    $\Ori(M)|_A$ is connected, then 
    $S$ can only consist of the zero section, and
    if it is not connected, it consists of a map on two
    components on which it is constant, and the
    relation $\lambda \circ t = - \lambda$ then
    determines that is must the required values
    to constitute the induced map of a section.
    \todo{check}
\end{proof}

\begin{theorem}[]\label{Thm:UDWOQJNX}
    Suppose $A \subset M$ is a closed connected subset.
    Then
    \begin{enumerate}
        \item $H_n(M,M-A; G) = 0$ if $A$ is not compact.
        \item $H_n(M, M-A;G) \cong G$ if 
            $M$ is $R$-orientable along $A$ and $A$ is 
            compact. Moreover, $H_n(M,M-A;G) \to 
            H_n(M,M- x ; G)$ is an isomorphism
            for each $x \in A$.
        \item $H_n(M, M-A; G) \cong 
            {}_{2}G = 
            \left\{ g \in G  \mid 2 g = 0 \right\} $ if
            $M$ is not orientable along $A$ and $A$ is compact.
    \end{enumerate}
\end{theorem}

\begin{proof}
    (1) By Lemma \ref{Lemma:XIOOQLSJ}, a section
    in $\Gamma(A;G)$ is determined by its value at a single
    point. By the existence of the zero section, if
    a section is non-zero at any point, then it is non-zero at
    every point. Therefore, there
    do not exist non-zero sections with compact support
    over a non-compact $A$, so by Theorem \ref{Thm:OGPAL},
    $H_n(M, M-A; G) \cong \Gamma_c (A; G) \cong 0$.\\
    \linebreak
    (2) Since $A$ is compact, 
    $H_n\left( M, M-A; G \right) \cong
    \Gamma_c \left( A;G \right) =
    \Gamma \left( A;G \right) $. A section is again
    determined by a single point. Recall now the commutative
    diagram
    \begin{equation*}
    \begin{tikzcd}
        H_n \left( M, M-A;G \right) \ar[r, "\cong"] 
        \ar[d, "r_x^{A}"] & \Gamma(A;G) \ar[d, "b"] \\
        H_n(M, M-x;G) \ar[r, "\cong"] & 
        \Gamma \left( \left\{ x \right\} ;G \right) 
    \end{tikzcd}
    \end{equation*}
    from Proposition \ref{Prop:SIDJOXOLLWQU}, the horizontal
    isomorphisms following from Theorem \ref{Thm:OGPAL}.
    If $M$ is orientable along $A$, there by definition
    exists in $\Gamma(A;G)$ an element such that
    its value at $x$ is a generator.
    Hence $b$ is an isomorphism, and therefore
    also $r_x^{A}$ is an isomorphism.\\
    \linebreak
    (3) 
    By Lemma \ref{Lemma:X2948JJD}, 
    a section in $\Gamma(A;G)$ corresponds to a continuous
    map $\lambda \colon \Ori (M) |_{A} \to G$ with
    $\lambda t = - \lambda$.
    If $M$ is not orientable along $A$, then
    $\Ori(M)|_{A}$ is connected and
    therefore $\lambda$ is constant as
    $G$ has the discrete topology. The relation
    $\lambda t = - \lambda$ shows that
    $\lambda$ is in ${}_{2}G$. 
    Now by the commutative diagram from part (2), note that
    since $\lambda$ must be constant, firstly
    $\Gamma (A;G) \cong {}_{2}G$, and
    secondly,$b$ becomes
    injective, so
    $r_{x}^{A} \colon H_n(M,M-A;G) \to 
    H_n \left( M, M- x ; G \right) \cong G$ is injective and
    has image ${}_{2}G$, so the
    $\Hom$ term vanishes.
\end{proof}

\begin{proposition}[]
    Let $M$ be an $n$-manifold and $A \subset M$ be a closed
    connected subset. Then the torsion subgroup
    of $H_{n-1}(M, M-A; \mathbb{Z})$ is of order $2$ if
    $A$ is compact and $M$ non-orientable along
    $A$, and is $0$ otherwise.
\end{proposition}

\begin{proof}
    By UCT for homology,
    \begin{align*}
       \mathbb{Z}/2
       \cong {}_{2} \mathbb{Z}/2 \cong H_n(M, M-A; \mathbb{Z}/2) 
       &\cong
        H_n (M, M-A) \otimes \mathbb{Z}/2 \oplus
        \Tor_1 \left( H_{n-1}(M,M-A), \mathbb{Z}/2 \right) \\
        &\cong \Tor_1 \left( H_{n-1}(M,M-A) , \mathbb{Z}/2 \right)\\
        &\cong \left\{ 
        g \in H_{n-1}(M, M-A)  \mid 2g = 0 \right\} .
    \end{align*}
    where $H_n \left( M, M-A \right) \cong
    {}_{2}\mathbb{Z} = 0$, and
    $H_n\left( M, M-A; \mathbb{Z}/2 \right) \cong
    {}_{2}\mathbb{Z}/2 \cong \mathbb{Z}/2$ both follow
    from Theorem \ref{Thm:UDWOQJNX}.

    To see that this is the whole torsions subgroup, note
    that for odd $k$,
    \[
    \Tor_1 \left( H_{n-1}(M,M-A) , \mathbb{Z}/k \right) 
    \cong H_n (M, M-A; \mathbb{Z}/k) \cong
    {}_{2}\mathbb{Z}/k \cong 0
    \] 
    When $M$ is orientable along $A$ and $A$ is compact, 
    we simply obtain
    \[
    0 \to H_n (M, M-A) \otimes \mathbb{Z}/n
    \to H_{n}(M, M-A; \mathbb{Z}/n) \to 
    \Tor_1 \left( H_{n-1}(M, M-A), \mathbb{Z}/n \right) \to 0
    \] 
    and since $H_n(M,M-A) \cong \mathbb{Z}$ and
    $H_{n}\left( M, M-A; \mathbb{Z}/n \right) \cong \mathbb{Z}/n$ 
    by Theorem \ref{Thm:UDWOQJNX}, we find that
    $\Tor_1$ vanishes for all $n$.\\
    If  $A$ is non-compact, then Theorem \ref{Thm:UDWOQJNX}
    gives that $\Tor_1$ trivially vanishes
    for all terms.
    
\end{proof}

\subsection{Fundamental Class}

\begin{theorem}[]\label{Thm:IXKEK}
    Let $M$ be a compact connected $n$-manifold. Then
    one of the following assertions holds:
    \begin{enumerate}
        \item $M$ is orientable, $H_n (M) \cong \mathbb{Z}$, and
            for each $x \in M$, the restriction
            $H_n(M) \to H_n(M, M-x)$ is an isomorphism.
        \item $M$ is non-orientable and
            $H_n(M) = 0$.
    \end{enumerate}
\end{theorem}

\begin{proof}
    Special case of Theorem \ref{Thm:UDWOQJNX}.
\end{proof}

Under the hypothesis of Theorem \ref{Thm:IXKEK}, 
the orientations of $M$ correspond to the generators
of $H_n(M)$. Such a generator will be called
a \textit{fundamental class} or \textit{homological class/orientation}
of the orientable manifold.

\begin{definition}[Degree]
    Let $M$ and $N$ be compact oriented $n$-manifolds.
    Let $N$ be connected and suppose
    $M$ has components $M_1, \ldots, M_r$. Then
    we have fundamental classes $z(M_j)$ for each
    $M_j$ and $z(M) \in H_n(M) \cong
    \bigoplus_{j} H_n(M_j)$ is the sum of
    the $z(M_j)$.
    Now, since $H_n(N) \cong \left< z(N) \right> \cong
    \mathbb{Z}$, we
    obtain that there exists a \textit{degree}
    $d(f) \in \mathbb{Z}$ such that
    $f_* z(M) = d(f) z(N)$.
\end{definition}

\begin{lemma}[Properties]
    \begin{enumerate}
        \item The degree is a homotopy invariant.
        \item $d(f \circ g) = d(f) d(g)$.
        \item A homotopy equivalence has degree
            $\pm 1$.
        \item If $M = M_1 \bigsqcup M_2$, then
            $d(f) = d\left( f|_{M_1} \right) +
            d\left( f|_{M_2} \right) $.
        \item If we pass in $M$ or $N$ to the opposite orientation,
            then the degree changes the sign.
    \end{enumerate}
\end{lemma}

\subsubsection{Computations of degrees}

As usual, we can compute degrees in terms of local data of a map.

Let $M$ and $N$ be connected and set
$K = f^{-1}(p)$. Let $U$ be an open neighborhood of
$K$ in $M$. Then in particular
$M - U = \overline{M - U} \subset \Int (M - A) = 
M - A$, so excision gives the bottom left isomorphism in
the following diagram, and the top right isomorphism
follows from Theorem \ref{Thm:IXKEK}:

\begin{equation*}
\begin{tikzcd}
    z(M) \ar[dd, mapsto] 
    \in & H_n(M) \ar[r, "f_*"] \ar[d] & H_n(M) \ar[d, "\cong"]
        & \ni z(N) \ar[dd, mapsto] \\
             & H_n(M, M-K) \ar[r, "f_*"] & H_n(N, N-p) &\\
    z(U,K) \in & H_n(U, U-K) \ar[u, "\cong", "i_*"']
    \ar[r, "f_*^{U}"] &
    H_n(N, N-p) \ar[u, "="'] & \ni z(N,p)
\end{tikzcd}
\end{equation*}

From the outer rectangle, we get $f_*^{U} z(U,K) = d(f) z(N,p)$, where
$z(N,p)$ and $z(U,K)$ are the images of
$z(N)$ and $z(M)$ under the indicated maps.

We want to show additivity of degree as in the case for
spheres.

So suppose $K$ if finite, and choose
$U = \bigcup_{x \in K}  U_x$ where the $U_x$ are pair-wise
disjoint open neighborhoods of $x$. Then
\[
\bigoplus_{x \in K} H_n(U_x, U_x - x) \cong
H_n (U, U-K), \quad H_n (U_x, U_x - x) \cong \mathbb{Z}.
\] 

The image $z \left( U_x, x \right) $ of $z(M)$ is a generator:
it is the image under the following isomorphisms
\[
H_n (M) \stackrel{\cong}{\to} H_n (M, M- x) \stackrel{\cong}{\to}  
H_n (U_x, U_x -x)
\] 
where the first follows from Theorem 
\ref{Thm:IXKEK} and the second from
excision. The local degree 
$d(f,x)$ is determined by
$f_* z\left( U_x, x \right) = d(f,x) z(N,p)$, and
and by additivity above, we have
$d(f) = \sum_{x \in K} d(f,x)$.



\section{Intersection Theory}



\begin{definition}[$k$-disk bundle]
    A $k$-disk bundle is a vector bundle whose
    coordinate transformations are contained in
    $O(k) \subset \GL (\mathbb{R}^{k})$ and such that
    the local trivializations have the form
    $\pi^{-1}(U) \cong U \times D^{k}$.
\end{definition}

Let $N^{n}$ be a connected, oriented, closed $n$-manifold, and
$W^{k+n}$ an $(n+k)$-manifold with boundary
$\partial W$ a $(k-1)$-sphere bundle over $N^{n}$, and let
$\pi \colon W^{n+k} \to N^{n}$ be a $k$-disk bundle over $N$.

Let us assume also that $W$ is also oriented.

\begin{definition}[]
    In the above situation, the \textit{Thom class} of the
    disk bundle $\pi$ is the class $\tau \in 
    H^{k}\left( W, \partial W \right) $ given by
    \[
    \tau = D_W \left( i_* \left[ N \right]  \right) 
    \] 
    where $D_W \colon H_{n-k} (W) \to H^{k}(W, \partial W)$ is
    the inverse of the Poincaré duality isomorphism.\\
    That is,
    \[
    D(a) \cap \left[ M \right]  = a.
    \] 
    Thus
    \[
    \tau \cap \left[ W \right] =
    i_* \left[ N \right] .
    \] 
\end{definition}

We can deformation retract the punctured disk to its boundary, giving
$H^{k}(W, W-N) \cong H^{k}(W, \partial W)$, so we will
sometimes regard
$\tau$ as being in $H^{k}(W, W-N)$.

\begin{lemma}[]
    In the above setup, suppose
    $A \subset N$ is closed. Let
    $\tilde{A} = \pi^{-1}(A) \subset W$ and
    $\partial \tilde{A} = \tilde{A} \cap \partial W$.
    Then
    $\check{H}^{i}\left( \tilde{A},
    \partial \tilde{A} \right) = 0$ for 
    $0<i < k$.
\end{lemma}

\begin{proof}
    Suppose first that
    $A$ is compact convex subset
    of a Euclidean neighborhood in $N$. It also
    suffices consider the case where $A$ is connected,
    so $A \cong D^{n}$.
    Consider the pullback bundle of $A$ :
    \begin{equation*}
    \begin{tikzcd}
        i^{*}(A) \ar[r] \ar[d] & W \ar[d, "\pi"] \\
        A \ar[r, hookrightarrow, "i"] & N
    \end{tikzcd}
    \end{equation*}
    Then
    $i^{*}(A) =
    A \times_{N} W \cong
    \pi^{-1}(A)$, so
    since any vector bundle over a contractible paracompact
    base space is trivial, we conclude that 
    the bundle
    $\tilde{A} \to A$ is trivializable as
    $\tilde{A} \cong A \times D^{k}$ 
    and $\partial \tilde{A} \cong
    A \times S^{k-1}$.
    Now the steps are as follows: calculate
    the homology of 
    $A \times D^{k} $ and
    $A \times S^{k-1} $, then use UCT to obtain
    the cohomology, and then use the LES to find the cohomology
    of
    $\left( A \times D^{k}, A \times S^{k-1} \right) $.\\
    Now...
    But
    by the Künneth theorem,
    \[
    H_m(A \times D^{k}) \cong
    H_m(A)
    \] 
    and
    \[
    H_m(A \times S^{k-1}) \cong
    H_m(A) \oplus H_{m-k+1}(A).
    \] 
\end{proof}


\begin{lemma}[]
    The restriction $\tau_x \in 
    \check{H}^{k}( \tilde{A}, \partial \tilde{A})$ of
    $\tau$, when $A = \left\{ x \right\} $, is a 
    generator.
\end{lemma}

\begin{proof}
    Note that 
    $\left( \tilde{A}, \partial \tilde{A} \right) 
    \cong \left( D^{k}, S^{k-1} \right) $.

    Suppose first that $\tau_x = 0$ for some $x$.

    Now, recall that
    \[
    \tau_{A} = 
    D_W \left( i_* \left[ A \right]  \right) .
    \] 

    Then
    $\tau_x = 0$ if and only if
    $i_* \left[ x \right] = 0$.
    But $i_* \colon 
    H_* \left( N, N - x \right) 
    \to H_* ($



\end{proof}


\section{Thom-Pontryagin Theory}

We start with an element
$\left[ f \right]  \in \pi_{n+k}\left( S^{n} \right) $, so
$f$ is a pointed map
$S^{n+k} \to S^{n}$.

Now insert a disk in place of the
base point, and extend $f$ to a map
$\overline{f}$ which is constant on the
next disk, taking the disk to the basepoint of
$S^{n}$, and is $f$ elsewhere.
There is a deformation retract of the sphere, collapsing
this disk to a point, and composing with this retract
gives $f$. Hence we may replace $f$ by a pointed-homotopic map
which is constant in a small neighborhood of
the basepoint.

Next, we can remove the base point of
$S^{n+k}$ and instead consider $f$ as a map
$\mathbb{R}^{n+k} \to S^{n}$ which is now constant
to the base point outside some compact
subset of $\mathbb{R}^{n+k}$.\\
\linebreak
By the Smooth Approximation Theorem, we can
also restrict attention to smooth maps
$\mathbb{R}^{n+k} \to S^{n}$ and smooth
homotopies. \todo{Insert theorem}

We regard also $S^{n}$ as the one-point compactification
of $\mathbb{R}^{n}$, denoted
$\mathbb{R}_+^{n} = \mathbb{R}^{n} \cup  \left\{ \infty \right\} $.

So suppose now we have a smooth map $f \colon
\mathbb{R}^{n+k} \to \mathbb{R}_+^{n}$ as above.

If $f$ is not null-homotopic, then it must be surjective, hence
in particular the image does not have measure $0$, so there
exists a regular value
$p \in \mathbb{R}^{n} \subset \mathbb{R}_+^{n}$.
By following $f$ by a translation in $\mathbb{R}^{n}$,
we can assume that $p$ is the origin $0 \in \mathbb{R}^{n}$ without
changing the homotopy class of $f$.

\begin{theorem}[\cite{Bredon}, Thm 11.6]\label{Thm:XLOOQKK}
    Let $f \colon \mathbb{R}^{n} \to M^{m}$ be a smooth
    map. Assume that $p \in M^{m}$ is a regular
    value, let $K = f^{-1} \left\{ p \right\} $, and
    assume that $K$ is compact. Then there
    is an open neighborhood $N$ of $K$ inside a tubular neighborhood
    of $K$, with normal retraction $r \colon N \to K$ ,and
    an open neighborhood $E \cong \mathbb{R}^{m}$ of
    $p$ in $M^{m}$ such that the map
    $r \times f \colon N \to K \times E$ is a diffeomorphism.
\end{theorem}

Using Theorem \ref{Thm:XLOOQKK}, we find that
there is a disk $E^{n}$ about $0$ in $\mathbb{R}^{n}$ and
an embedding $M^{k} \times E^{n} \hookrightarrow 
N \subset \mathbb{R}^{n k} $ onto an open neighborhood
$N$ of $M^{k}$ whose inverse $N \to 
M^{k} \times E^{n}$ is $r \times f$, where
$r \colon N \to  M^{k}$ is the normal retraction.\\
\linebreak
Through another homotopy of $f$, we can assume that
$E^{n}$ is the open unit disk
$D^{n}$.\\

We will refer to an embedding
$g \colon M^{k} \times E^{n} \to \mathbb{R}^{n+k}$, with
$M^{k}$ compact, as a
"fattened $k$-manifold".




\section{Terminology}

\begin{definition}[Neighborhood retract]
    If $A \subset X$ and $A$ has a neighborhood in $X$ of
    which it is a retract, then $A$ is called
    a \textit{neighborhood retract} (in $X$ ).
\end{definition}

\begin{note}
    Saying that $A \hookrightarrow X$ is a cofibration
    is stronger than saying that $A$ is a neighborhood retract.
\end{note}

\section{Lemmas}

\begin{lemma}[]\label{Lemma:XIOOQLSJ}
    Let $\pi \colon W \to N$ be a covering map and
    $M$ a connected space. Suppose
    $f,g \colon M \to W$ are maps such that
    $\pi \circ f = \pi \circ g$ and that
    $f(x) = g(x)$ for some $x \in M$. Then
    $f = g$.
\end{lemma}

\begin{proof}
    Show that the set
    \[
    Z = \left\{ z \in M  \mid f(z) = g(z)\right\} 
    \] 
    is closed and open.
\end{proof}





\newpage
\printbibliography
\end{document}
