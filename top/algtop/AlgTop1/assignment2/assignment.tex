\documentclass[reqno]{amsart}
\usepackage{amscd, amssymb, amsmath, amsthm}
\usepackage{graphicx}
\usepackage[colorlinks=true,linkcolor=blue]{hyperref}
\usepackage[utf8]{inputenc}
\usepackage[T1]{fontenc}
\usepackage{textcomp}
\usepackage{babel}
%% for identity function 1:
\usepackage{bbm}
%%For category theory diagrams:
\usepackage{tikz-cd}


\setlength\parindent{0pt}

\pdfsuppresswarningpagegroup=1

\newtheorem{theorem}{Theorem}[section]
\newtheorem{lemma}[theorem]{Lemma}
\newtheorem{proposition}[theorem]{Proposition}
\newtheorem{corollary}[theorem]{Corollary}
\newtheorem{conjecture}[theorem]{Conjecture}

\theoremstyle{definition}
\newtheorem{definition}[theorem]{Definition}
\newtheorem{example}[theorem]{Example}
\newtheorem{exercise}[theorem]{Exercise}
\newtheorem{problem}[theorem]{Problem}
\newtheorem{question}[theorem]{Question}

\theoremstyle{remark}
\newtheorem*{remark}{Remark}
\newtheorem*{note}{Note}
\newtheorem*{solution}{Solution}



%Inequalities
\newcommand{\cycsum}{\sum_{\mathrm{cyc}}}
\newcommand{\symsum}{\sum_{\mathrm{sym}}}
\newcommand{\cycprod}{\prod_{\mathrm{cyc}}}
\newcommand{\symprod}{\prod_{\mathrm{sym}}}

%Linear Algebra

\DeclareMathOperator{\Span}{span}
\DeclareMathOperator{\Ima}{Im}
\DeclareMathOperator{\diag}{diag}
\DeclareMathOperator{\Ker}{Ker}
\DeclareMathOperator{\ob}{ob}
\DeclareMathOperator{\Hom}{Hom}
\DeclareMathOperator{\sk}{sk}
\DeclareMathOperator{\Vect}{Vect}
\DeclareMathOperator{\Set}{Set}
\DeclareMathOperator{\Group}{Group}
\DeclareMathOperator{\Ring}{Ring}
\DeclareMathOperator{\Ab}{Ab}
\DeclareMathOperator{\Top}{Top}
\DeclareMathOperator{\hTop}{hTop}
\DeclareMathOperator{\Htpy}{Htpy}
\DeclareMathOperator{\Cat}{Cat}
\DeclareMathOperator{\CAT}{CAT}
\DeclareMathOperator{\Cone}{Cone}
\DeclareMathOperator{\dom}{dom}
\DeclareMathOperator{\cod}{cod}
\DeclareMathOperator{\Aut}{Aut}
\DeclareMathOperator{\Mat}{Mat}
\DeclareMathOperator{\Fin}{Fin}
\DeclareMathOperator{\rel}{rel}
\DeclareMathOperator{\Int}{Int}
\DeclareMathOperator{\sgn}{sgn}
\DeclareMathOperator{\Homeo}{Homeo}
\DeclareMathOperator{\PSL}{PSL}

%Row operations
\newcommand{\elem}[1]{% elementary operations
\xrightarrow{\substack{#1}}%
}

\newcommand{\lelem}[1]{% elementary operations (left alignment)
\xrightarrow{\begin{subarray}{l}#1\end{subarray}}%
}

%SS
\DeclareMathOperator{\supp}{supp}
\DeclareMathOperator{\Var}{Var}

%NT
\DeclareMathOperator{\ord}{ord}

%Alg
\DeclareMathOperator{\Rad}{Rad}
\DeclareMathOperator{\Jac}{Jac}

%Misc
\newcommand{\SL}{{\mathrm{SL}}}
\newcommand{\mobgp}{{\mathrm{PSL}_2(\mathbb{C})}}
\newcommand{\id}{{\mathrm{id}}}
\newcommand{\Mod}{{\mathrm{Mod}}}
\newcommand{\ud}{{\mathrm{d}}}
\newcommand{\Vol}{{\mathrm{Vol}}}
\newcommand{\Area}{{\mathrm{Area}}}
\newcommand{\diam}{{\mathrm{diam}}}
\newcommand{\End}{{\mathrm{End}}}


\newcommand{\reg}{{\mathtt{reg}}}
\newcommand{\geo}{{\mathtt{geo}}}

\newcommand{\tori}{{\mathcal{T}}}
\newcommand{\cpn}{{\mathtt{c}}}
\newcommand{\pat}{{\mathtt{p}}}




\begin{document}
I didn't finish the problem as I started quite late, and I'm not sure
about the solutions. I'll work on these when I get the time, but
comments would be helpful. Thank you.


    \begin{problem}[]
        \begin{enumerate}
            \item 
        Let $p \colon X \to Y$ and $f \colon Z \to Y$ 
        be maps. Let
        \[
        E = \left\{ \left( z,x \right) \in Z \times X
         \mid f(z) = p(x) \right\} 
        \] 
        equipped with the subspace topology from
        $Z \times X$, and define  $p' \colon E \to Z$ by
        $p' (z,x) = z$. Prove that if $p$ is a covering map,
        then $p'$ is also a covering map.
    \item Give a covering $p$, a map $f$ such that $p'$ is trivial
        and $p$ is not trivial.
    \item If $p$ is a trivial covering, can $p'$ be non-trivial.
        \end{enumerate}
    \end{problem}

    \begin{solution}
        (i)
        Let $z \in Z$. Since $p$ is a covering map,
        we can find some evenly covered neighborhood $U$ of
        $f(z)$. Since
        $p|_{p^{-1}(U)} \colon p^{-1} \left(U \right) \to U$ 
        is a trivial covering,
        the fiber of $f(z)$ over $p$ splits into
        some discrete set of elements all mapped to
        $f(z)$ under $p$.
        Suppose $x \in p^{-1}(f(z))$. Then
        $\left( z,x \right) \in E$, so
        in particular,
        $V := E \cap \left( f^{-1}(U) \times p^{-1}(U) \right) $ 
        contains $p^{-1}(z) \times \left\{ z \right\}  $ and 
        is an open set in $E$. The set
        $W := f^{-1}(U) $ 
        is open in $Z$.\\
        \linebreak
        We claim that $W$ is an evenly
        covered neighborhood of $z$ under $p'$.
        We first show that
        $p'^{-1}(W) = V$.
        If $(z,x) \in p'^{-1}(W)$ then
        $z \in W = f^{-1}(U)$, so $p(x) = f(z) \in U$, so
        $(z,x) \in E \cap \left( f^{-1}(U) \times p^{-1}(U) \right) 
        = V$. Conversely, $V \subset W$ is clear.
        Letting $\varphi \colon p^{-1}(U) \to U \times F$ be
        the trivial covering for $p$, we
        define
        $\psi \colon p'^{-1}(W) \to W \times F$ by
        $\psi (z,x) = (z, \pi_F (\varphi(x)) )$. Since
        $\varphi $ is bijective, clearly, $\psi $ is too.
        Continuity of $\psi$ follows from continuity
        of each of the coordinate functions. Now,
        suppose $\left( z,x \right) \in 
        p'^{-1}(W)$ and
        $A$ is a neighborhood of $\varphi (z,x) =
        \left( z, \pi_F \left( \varphi (x) \right)  \right) $ in
        $W \times F$. Let $B$ be a basis open set in $W$ and
        hence in $Z$ (since $W$ is open in $Z$ ) containing
        $z$ such that we have $B \times \left\{ \pi_F
        \left( \varphi (x) \right) \right\} $ open and
        contained in $A$. If necessary, we can shrink $B$ such
        that $B \times \left\{ \pi_F \left( \varphi (x)
        \right) \right\} $ is evenly covered under $\varphi $.
        In particular, $\varphi$ is a homeomorphism
        of $B' = \varphi^{-1} \left( B \times \left\{ 
        \pi_F \left( \varphi (x) \right) \right\}  \right) $ onto
        $B \times \left\{ \pi_F \left( \varphi (x) \right)  \right\} $,
        so $B'$ is thus open in $X$.
        Now
        $E \cap \left( B \times B' \right) $ is an open
        set in $E$ containing $\left( z,x \right) $ and
        $\varphi(z,x) \in 
        \psi \left( E \cap \left( B \times B' \right)  \right) 
        = B \times \left\{ \pi_F \left( \varphi(x) \right)  \right\} 
        \subset A$. So
        $\psi^{-1}$ is also continuous, and hence
        $\psi $ is a homeomorphism.\\
        \linebreak
        


        (ii) Let $p \colon \mathbb{R} \to S^{1}$ be the usual
        covering map of $S^{1}$ by
        $x \mapsto e^{2 \pi i x}$. This is nontrivial. Let
        $f \colon V := (0, \frac{1}{2}) \to S^{1}$ 
        be the restriction of $p$ to $V$. Then
        $E = \left\{ 
        \left( z,x \right) \in V \times \mathbb{R}  \mid 
    f(z) = p(x) \right\} = \left\{ 
    \left( z,x \right) \in V \times \mathbb{R}  \mid 
x \in z + \mathbb{Z} \right\} $. Then 
$p' \colon E \to V$ becomes the projection
$E \to V$ onto the first coordinate. The map
$\varphi \colon E \to V \times  \mathbb{Z}$ by
$(z,x) \mapsto \left( z, z-x \right) $ is a
bijective map. For any basis open set $U \subset V$, we have
that $\varphi^{-1} \left( U \times \left\{ n \right\}  \right) 
= T_{-n}(U)$, where $T_{n} \colon \mathbb{R} \to \mathbb{R}$ is
the map $T_n(x) = x + n$, which is open in $E$ since it is
open in $V \times \mathbb{R}$. Thus $\varphi$ is continuous.

Let $(z,x) \in V \times \mathbb{Z}  $ and
let $B$ be an open neighborhood of $\varphi^{-1} (z,x) =
\left( z,z+x \right) $. We can find an
open neighborhood $W$ in $V$ of $z$ such that
$\left( z, z+x \right) \in E \cap \left( W \times T_x(W) \right) $.
But $(z,x) \in W \times \left\{ x \right\} $ and
$\varphi^{-1}\left( W \times \left\{ x \right\}  \right) 
\subset E \cap \left( W \times T_x (W) \right) \subset B$, so
$\varphi^{-1}$ is also continuous. Hence $\varphi$ is a homeomorphism.\\
\linebreak
(iii) I didn't get to this one in time.







    \end{solution}







    %\bibliography{../refs.bib}
\end{document}
