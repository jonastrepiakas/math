\documentclass[reqno]{amsart}
\usepackage{amscd, amssymb, amsmath, amsthm}
\usepackage{graphicx}
\usepackage[colorlinks=true,linkcolor=blue]{hyperref}
\usepackage[utf8]{inputenc}
\usepackage[T1]{fontenc}
\usepackage{textcomp}
\usepackage{babel}
%% for identity function 1:
\usepackage{bbm}
%%For category theory diagrams:
\usepackage{tikz-cd}
\usepackage{enumerate}


\setlength\parindent{0pt}

\pdfsuppresswarningpagegroup=1

\newtheorem{theorem}{Theorem}[section]
\newtheorem{lemma}[theorem]{Lemma}
\newtheorem{proposition}[theorem]{Proposition}
\newtheorem{corollary}[theorem]{Corollary}
\newtheorem{conjecture}[theorem]{Conjecture}

\theoremstyle{definition}
\newtheorem{definition}[theorem]{Definition}
\newtheorem{example}[theorem]{Example}
\newtheorem{exercise}[theorem]{Exercise}
\newtheorem{problem}[theorem]{Problem}
\newtheorem{question}[theorem]{Question}

\theoremstyle{remark}
\newtheorem*{remark}{Remark}
\newtheorem*{note}{Note}
\newtheorem*{solution}{Solution}
\newtheorem*{question*}{Question}



%Inequalities
\newcommand{\cycsum}{\sum_{\mathrm{cyc}}}
\newcommand{\symsum}{\sum_{\mathrm{sym}}}
\newcommand{\cycprod}{\prod_{\mathrm{cyc}}}
\newcommand{\symprod}{\prod_{\mathrm{sym}}}

%Linear Algebra

\DeclareMathOperator{\Span}{span}
\DeclareMathOperator{\Ima}{Im}
\DeclareMathOperator{\diag}{diag}
\DeclareMathOperator{\Ker}{Ker}
\DeclareMathOperator{\ob}{ob}
\DeclareMathOperator{\Hom}{Hom}
\DeclareMathOperator{\Mor}{Mor}
\DeclareMathOperator{\sk}{sk}
\DeclareMathOperator{\Vect}{Vect}
\DeclareMathOperator{\Set}{Set}
\DeclareMathOperator{\Group}{Group}
\DeclareMathOperator{\Ring}{Ring}
\DeclareMathOperator{\Ab}{Ab}
\DeclareMathOperator{\Top}{Top}
\DeclareMathOperator{\hTop}{hTop}
\DeclareMathOperator{\Htpy}{Htpy}
\DeclareMathOperator{\Cat}{Cat}
\DeclareMathOperator{\CAT}{CAT}
\DeclareMathOperator{\Cone}{Cone}
\DeclareMathOperator{\dom}{dom}
\DeclareMathOperator{\cod}{cod}
\DeclareMathOperator{\Aut}{Aut}
\DeclareMathOperator{\Mat}{Mat}
\DeclareMathOperator{\Fin}{Fin}
\DeclareMathOperator{\rel}{rel}
\DeclareMathOperator{\Int}{Int}
\DeclareMathOperator{\sgn}{sgn}
\DeclareMathOperator{\Homeo}{Homeo}
\DeclareMathOperator{\PSL}{PSL}
\DeclareMathOperator{\Bil}{Bil}
\DeclareMathOperator{\Sym}{Sym}
\DeclareMathOperator{\Skew}{Skew}
\DeclareMathOperator{\Alt}{Alt}
\DeclareMathOperator{\Quad}{Quad}
\DeclareMathOperator{\Sin}{Sin}


%Row operations
\newcommand{\elem}[1]{% elementary operations
\xrightarrow{\substack{#1}}%
}

\newcommand{\lelem}[1]{% elementary operations (left alignment)
\xrightarrow{\begin{subarray}{l}#1\end{subarray}}%
}

%SS
\DeclareMathOperator{\supp}{supp}
\DeclareMathOperator{\Var}{Var}

%NT
\DeclareMathOperator{\ord}{ord}

%Alg
\DeclareMathOperator{\Rad}{Rad}
\DeclareMathOperator{\Jac}{Jac}

%Misc
\newcommand{\SL}{{\mathrm{SL}}}
\newcommand{\mobgp}{{\mathrm{PSL}_2(\mathbb{C})}}
\newcommand{\id}{{\mathrm{id}}}
\newcommand{\Mod}{{\mathrm{Mod}}}
\newcommand{\ud}{{\mathrm{d}}}
\newcommand{\Vol}{{\mathrm{Vol}}}
\newcommand{\Area}{{\mathrm{Area}}}
\newcommand{\diam}{{\mathrm{diam}}}
\newcommand{\End}{{\mathrm{End}}}


\newcommand{\reg}{{\mathtt{reg}}}
\newcommand{\geo}{{\mathtt{geo}}}

\newcommand{\tori}{{\mathcal{T}}}
\newcommand{\cpn}{{\mathtt{c}}}
\newcommand{\pat}{{\mathtt{p}}}

\let\Cap\undefined
\newcommand{\Cap}{{\mathcal{C}}ap}



\title{Assignment 4 and 5}
\author{Jonas Trepiakas - hvn548}

\begin{document}


\maketitle


    \begin{exercise}[]
        If $\left( X,x \right) $ and $\left( Y,y \right) $ 
        are based spaces, let $C\left( (X,x),
        (Y,y) \right) $ be the set of
        pairs $\left( f, \left[ \lambda \right]  \right) $,
        where $f \colon X \to Y$ is a continuous map,
        $\lambda \colon I \to Y$ is a path from
        $y$ to $f(x)$ and $\left[ \lambda \right] $ 
        denotes its homotopy class relative to $\partial I$.

        If $\left( Z,z \right) $ is a third based space
        and $\left( f , \left[ \lambda \right]  \right) 
        \in  C\left( (X,x), (Y,y) \right) $ and
        $\left( g , \left[ \mu \right]  \right) \in 
        C\left( \left( Y,y \right) , \left( Z,z \right)  \right) $,
        define $\left( g, \left[ \mu \right]  \right) 
        \circ_C \left( f , \left[ \lambda \right]  \right) 
        = \left( g \circ f, \left[ \mu *
        \left( g \circ \lambda \right) \right]  \right) $.

        \begin{enumerate}[(i)]
            \item Check that
                $\left( g, \left[ \mu \right]  \right) 
                \circ_C \left( f, \left[ \lambda \right]  \right) $ 
                is an element of 
                $C\left( (X,x), (Z,z) \right) $.
            \item Explain why $\circ_C$ is associative.
            \item Define identity elements
                $\mathbbm{1}_{(X,x)}
                \in C \left( (X,x), (X,x) \right) $ and
                verify that $C$ is in fact
                a category.
        \end{enumerate}

    \end{exercise}

    \begin{proof}
        (i): Compositions of continuous maps are continuous,
        so $g \circ f \colon X \to Z$ is
        continuous. We must simply check that
        $\mu * \left( g \circ \lambda \right) $ is a path
        from $z$ to $g \circ f(x)$. Since
        $\left( g, \left[ \mu \right]  \right) 
        \in C \left( (Y,y), (Z,z) \right) $,
        $\mu $ starts at $z$ by assumption, hence
        so does $\mu * \left( g \circ \lambda \right) $.
        The path ends at $\mu * \left( g \circ \lambda \right) (1)
        = \left( g \circ \lambda \right) (1)
        = g(\lambda(1)) = g(f(x)) = (g \circ f) (x)$
        where again $\lambda(1) = f(x)$ follows
        from $\left( f, \left[ \lambda \right]  \right) 
        \in C \left( (X,x), (Y,y) \right) $.

        We do not have problems with well-definedness of
        endpoints since our homotopy classes are
        $\rel \partial I$, so endpoints are unique.\\
        \linebreak
        

        (ii): Associativity of the function
        part follows from general associativity of functions.
        Associativity of the paths up to homotopy follows
        from lemma 1.4.2.(ii) (associativity of paths under
        homotopy when
        they are defined).\\
        \linebreak
        (iii): Let $c_x$ denote the constant path
        at $x$, and let $\mathbbm{1}_{X}$ denote
        the identity map $X \to X$ (which indeed is continuous).
        Then $c_x$ is also a path from $\mathbbm{1}_X(x) = x$ to
        $x$, so
        $\left( \mathbbm{1}_X, c_x \right) \in 
        C \left( (X,x), (X,x) \right) $.
        Now, with $\ob C = \ob \Top_*$ and the collection
        of morphisms between the objects $\left( X,x \right) $ and
        $(Y,y)$ being
        $C \left( (X,x), (Y,y) \right)$, we claim
        this forms a category under the above composition operation
        and designated identity morphisms.

        The only thing left to check is that
        for an arbitrary morphism
        $\left( f, \left[ \lambda \right]  \right) 
        \in  C\left( (X,x), (Y,y) \right) $, we have
        $\left( f, \left[ \lambda \right]  \right) 
        = \left( f, \left[ \lambda \right]  \right) 
        \circ_C \left( \mathbbm{1}_X, c_x \right) 
        = \left( \mathbbm{1}_Y, c_y \right) \circ_C
        \left( f, \left[ \lambda \right]  \right) $.
        Now
        \[
            \left( f, \left[ \lambda \right]  \right) 
            \circ_C \left( \mathbbm{1}_X,
            c_x\right) 
            = \left( f \circ \mathbbm{1}_X ,
            \left[ \lambda *
        f \circ c_x \right] \right) 
        = \left( f, \left[ \lambda \right]  \right) 
        \] 
        since $f \circ c_x = c_{f(x)}$, so
        $\lambda * f \circ c_{x} \simeq
        \lambda \circ c_{f(x)}
        \simeq \lambda$. Similarly,
        \[
            \left( \mathbbm{1}_Y, c_y \right) \circ_C
            \left( f, \left[ \lambda \right]  \right) 
            = \left( \mathbbm{1}_Y \circ f,
            \left[ c_y * \mathbbm{1}_Y \circ
        \lambda \right] \right) 
        = \left( f, \left[ c_y * \lambda \right]  \right) 
        = \left( f, \left[ \lambda \right]  \right).
        \] 
        
        (iv) To check that this is well defined,
        suppose $\left[ \alpha \right] = \left[ \beta \right] 
        \in \pi_1 \left( X, x \right) $. Then
         \[
             \left( f, \left[ \lambda \right]  \right)_*
             \left( \left[ \alpha \right]  \right) 
             = \lambda_* \circ f_* \left[ \alpha \right] 
             = \lambda_* \left[ f \circ \alpha \right] 
             = \left[ \lambda * f \circ \alpha *
             \overline{\lambda} \right] ,
        \] 
        and similarly,
        \[
            \left( f, \left[ \lambda \right]  \right) 
            \left[ \beta \right] 
            = \left[ \lambda * f \circ \beta * \overline{\lambda}
            \right].
        \] 

        Now $\lambda * f \circ \alpha * \overline{\lambda}
        \simeq \lambda * f \circ \beta * \overline{\lambda} 
        \rel I$ follows from 
        lemma 1.4.2.(i) and (v). Hence
        the map
        $\left( f, \left[ \lambda \right]  \right)_*$ is
        well-defined.

        Now we check functoriality of
        $\pi_1$. To each
        pointed space $(X,x) \in \ob C$, we
        associate the group $\pi_1 \left( X, x \right) 
        \in \Ab$.
        To each morphism
        $\left( f, \left[ \lambda \right]  \right) 
        \in  \Mor \left( \left( X,x \right) ,
        \left( Y,y \right) \right) $, we associate
        the homomorphism
        $\left( f, \left[ \lambda \right]  \right)_*
        \colon \pi_1 \left( X,x \right) 
        \to \pi_1 \left( Y,y \right) $ as defined above.
        We must check that $\pi_1
        \left( \mathbbm{1}_X, c_x \right) 
        = \mathbbm{1}_{\pi_1 \left( X,x \right) }$ and
        that for
        $\left( f, \left[ \lambda \right]  \right) 
        \in \Mor \left( (X,x), (Y,y) \right) $ and
        $\left( g, \left[ \mu \right]  \right) 
        \in  \Mor \left( (Y,y), (Z,z) \right) $, we have
        $\pi_1 \left( \left( g, \left[ \mu \right]  \right) 
        \circ_C \left( f, \left[ \lambda \right]  \right)  \right) 
        = \pi_1 \left( \left( g, \left[ \mu \right]  \right)  \right) 
        \circ \pi_1 \left( \left( f, \left[ \lambda \right] 
        \right) \right) $.

        Now, for any
        $\left[ \alpha \right] \in 
        \pi_1 \left( X,x \right) $, we have

        \begin{align*}
            \pi_1 \left( \mathbbm{1}_X, c_x \right) 
            \left[ \alpha \right] 
            &= \left( c_x \right)_* \circ 
            \left( \mathbbm{1}_X \right)_{*}
            \left[ \alpha \right] \\
            &= \left( c_x \right)_* \left[ \mathbbm{1}_X
            \circ \alpha \right] \\
            &= \left[ c_x * \mathbbm{1}_X \circ \alpha
            * \overline{c_x} \right] \\
            &= \left[ \alpha \right] \\
            &= \mathbbm{1}_{\pi_1 \left( X,x \right) }
            \left[ \alpha \right] 
        \end{align*}
        giving
        $\pi_1 \left( \mathbbm{1}_X, c_x \right) 
        = \mathbbm{1}_{\pi_1 \left( X, x \right) }$, and
        also

        \begin{align*}
        \pi_1 \left( \left( g, \left[ \mu \right]  \right)  \circ_C
    \left( f, \left[ \lambda \right]  \right) \right) 
    \left[ \alpha \right] 
    &= \pi_1 \left( g \circ f, \left[ \mu * g \circ \lambda
    \right]  \right) \left[ \alpha \right]  \\
    &= \left( \mu * g \circ \lambda \right)_*
    \circ \left( g \circ f \right)_* \left[ \alpha \right]  \\
    &= \left( \mu * g \circ \lambda \right)_*
    \left[ g \circ f \circ \alpha  \right] \\
    &= \left[ \mu * g \circ \lambda *
    g \circ f \circ \alpha *
\overline{\mu * g \circ \lambda} \right] \\
    &= \left[ \mu * g \circ \lambda *
    g \circ f \circ \alpha *
g \circ \overline{\lambda} * \overline{\mu} \right] \\
    &= \left[ \mu * g \circ \left( 
    \lambda * f \circ \alpha * \overline{\lambda}\right) 
* \overline{\mu}\right] \\
    &= \mu_* \circ g_* \circ \lambda_* \circ f_*
    \left[ \alpha \right] \\
    &= \pi_1 \left( g , \left[ \mu \right]  \right) 
    \circ \pi_1 \left( f, \left[ \lambda \right]  \right) 
    \left[ \alpha \right],
        \end{align*}
        so indeed

        \[
        \pi_1 \left( \left( g, \left[ \mu \right]  \right) 
        \circ_C \left( f, \left[ \lambda \right]  \right)  \right) 
        = \pi_1 \left( \left( g, \left[ \mu \right]  \right)  \right) 
        \circ \pi_1 \left( \left( f, \left[ \lambda \right] 
        \right) \right)
        \] 
    \end{proof}


    \begin{question*}
        I'm having trouble understanding precisely
        what the implication of this exercise is in
        in a categorical framework. Is the
        important part that we have essentially
        constructed a category wherein
        $\Mor \left( (X,x), (Y,y) \right) $ consists
        precisely of those maps $f$ which send
        $x$ to the path component of $y$ instead
        of only the maps which send $x$ to $y$ as in
        $\Top_*$, and thus we have "enlarged" the
        morphisms in our category a bit (which essentially
        corresponds to the result that $\pi_1 $ is independent
        of basepoint in a path-connected space). Put
        in a different way,
        $\Top_*$ is a subcategory of
        $C$ and under the inclusion functor $\iota \colon
        \Top_* \to C$ (where
        a morphism $f \colon
        (X,x) \to (Y,y)$ is mapped to
        $\left( f, \left[ c_y \right]  \right) $ )
        , we have that the following diagram commutes:

        \begin{equation*}
        \begin{tikzcd}
            C \ar[r, "\pi_1"] & \Group\\
            \Top_* \ar[u, "\iota"] \ar[ur, "\pi_1"']
        \end{tikzcd}
        \end{equation*}
        If you could shed some light onto whether there
        is something I'm missing, that would be really great!
        Thank you.
    \end{question*}



    \begin{exercise}[]
        Let $p \colon \mathbb{R} \to S^{1}$ denote
        the usual covering map
        $x \mapsto e^{2 \pi i x}$.
        \begin{enumerate}
            \item Prove that for any
                $\sigma \in \Sin_1 \left( S^{1} \right) $,
                there exists a $\tau \in \Sin_1\left( \mathbb{R}
                \right) $ such that
                $\sigma = p \circ \tau$. In particular, we
                get two numbers $d_0 \tau, d_1 \tau
                \in \Sin_0 \left( \mathbb{R} \right) ,
                =
                \mathbb{R}$. Explain why the difference
                $d_0 \tau - d_1 \tau \in \mathbb{R}$ depends
                only on $\sigma$, and not on the
                choice of $\tau$ (remember that
                $d_1$ is the starting point and $d_0$ is the
                end point). Hence we may
                define a function
                \[
                \varphi  \colon \Sin_1 \left( S^{1} \right) \to 
                \mathbb{R}
                \] 
                by $\varphi  \left( \sigma \right) 
                = d_1 \tau - d_0 \tau$ if $\sigma = p \circ \tau$.
            \item Using addition as a group
                structure on $\mathbb{R}$, extend $\varphi $ to
                a homomorphism $C_1 \left( S^{1} \right) 
                \to \mathbb{R}$. Let us use the same letter
                $\varphi  $ for this homomorphism.
            \item Prove that $\varphi  \circ \partial
                \colon C_2 \left( S^{1}  \right) \to \mathbb{R}$ 
                is trivial and deduce a well-defined homomorphism
                $H_1 \left( S^{1} \right) \to \mathbb{R}$,
                sending $\left[ c \right] \mapsto 
                \left[ \varphi  (c) \right] $.
            \item Let $\sigma_0 \colon \Delta^{1} \to 
                S^{1}$ be given by 
                $\sigma_0 \left( t_0, t_1 \right) 
                = e^{2 \pi i t_1}$. Prove that
                $\sigma_0 \in Z_1 \left( S^{1} \right) $,
                so that it represents a class 
                $\left[ \sigma_0 \right] 
                \in H_1 \left( S^{1} \right) $.
            \item Prove that the homomorphism 
                $\varphi  \colon H_1\left( S^{1} \right) 
                \to \mathbb{R}$ constructed above sends
                $\left[ \sigma_0 \right] $ to
                $1 \in \mathbb{R}$.
        \end{enumerate}
    \end{exercise}


    \begin{proof}
        (i) Let $k \colon \Delta^{1} \to I$
        denote the homeomorphism
        $\left( t_0, t_1 \right) \mapsto t_1$ and
        $k^{-1} \colon I \to \Delta^{1}$ denote
        the inverse homeomorphism $t \mapsto \left( 1-t,t \right) $.
        Now, $\sigma \circ k^{-1}$ is a path
        $I \to S^{1}$ which we can thus lift to a path
        $\tilde{\sigma} \colon I \to \mathbb{R}$ such that
        $\sigma \circ k^{-1} = p \circ \tilde{\sigma}$, hence
        $\tau := \tilde{\sigma} \circ k
        \in \Sin_1 \left( \mathbb{R} \right) $ 
        and satisfies $\sigma = p \circ \tau$.

        Now, we know that given a lift
        $\tilde{\sigma}$ of $\sigma \circ k^{-1}$ and
        a starting point $\tilde{\sigma}(0)$ 
        of $\tilde{\sigma}$, the
        endpoint $\tilde{\sigma}(1)$ is uniquely determined.
        Now,
        $d_0 \tau - d_1 \tau = 
        \tau \circ \delta^{0} - \tau \circ \delta^{1}$, where
        $\delta^{0}, \delta^{1} \colon
        \left\{ 1 \right\} = \Delta^{0} \to 
        \Delta^{1}$ is given by
        $\delta^{0}(1) = (0,1)$ and
        $\delta^{1}(1) = (1,0)$. Thus
        $\tau \circ \delta^{0}(1) = \tilde{\sigma} \circ k
        (0,1) = \tilde{\sigma}(1) $ and
        $\tau \circ \delta^{1}(1) = 
        \tilde{\sigma}(0)$. So
        $d_0 \tau - d_1 \tau = \tilde{\sigma}(1) - 
        \tilde{\sigma}(0)$. We have thus reduced
        the question to showing that
        $\tilde{\sigma}(1) - \tilde{\sigma}(0)$ only
        depends on $\sigma $. One way to see this
        is that the convering space $\mathbb{R}$ inherits
        a metric since
        $p$ is a local isometry, so distances are preserved.
        I.e.,  if $\tilde{U}$ is one of the
        sheets of $p^{-1}(U)$ for some evenly covered
        open set $U \subset S^{1}$ contained
        in one of the standard evenly covered sets
        $U_1, U_2, U_3, U_4$ corresponding to the 'right', 'upper',
        'left' and 'bottom' part of $S^{1}$, respective, then
        $p|_{\tilde{U}} \colon \tilde{U} \to U$ is
        an isometry where $S^{1}$ is equipped with the
        metric defined by $\frac{\text{arc length}}{2\pi}$. 
        Now we can cover $S^{1}$ by
        $U_1, U_2, U_3, U_4$ and use the Lebesgue lemma to
        find $\tilde{\sigma}(0)
        = t_0 < t_1 < \ldots < t_n = \tilde{\sigma}(1)$ such that
        $p \circ \tilde{\sigma}\left( \left[ t_{i-1},t_i \right] 
        \right)
        \subset U_{k_i}$ for all $i = 1,\ldots, n$. Then
        if $d_{S^{1}}$ denotes the metric on $S^{1}$, and
        $d$ the metric on $\mathbb{R}$, we have
        $t_i - t_{i-1} =  d  \left( t_{i-1},t_i \right) 
        = d_{S^{1}} \left( p \circ \tilde{\sigma}(t_i),
        p \circ \tilde{\sigma}\left( t_{i-1} \right) \right) $. Hence
        we have
         \[
        \tilde{\sigma}(1) - \tilde{\sigma(0)}
        = \sum d_{S^{1}} \left( p \circ \tilde{\sigma}(t_i),
        p \circ \tilde{\sigma}\left( t_{i-1} \right) \right)
        = \sum d_{S^{1}} \left( \sigma \circ k^{-1}(t_i),
        \sigma \circ k^{-1} (t_{i-1})\right) 
        \] 
        which in particular only depends on $\sigma$.

        \begin{remark}[]
            I am not
            completely sure whether the details work here, to be
            honest.
        \end{remark}

        (ii)
        Equipped with addition, $\left( \mathbb{R}, + \right) $ 
        becomes an abelian group, so lemma 5.1.2 directly
        gives that $\varphi  $ can be extended
        uniquely to $C_1 \left( S^{1} \right) = 
        \mathbb{Z} \Sin_1 \left( S^{1} \right) $ (extended
        linearly).\\
        \linebreak
        
        (iii) It suffices to show it
        only for $\sigma \in \Sin_2 \left( S^{1} \right) $.
        In this case,
        \begin{align*}
            \varphi  \circ \partial \sigma
            &= \varphi \left( d_0 \sigma - d_1 \sigma
            + d_2 \sigma \right) \\
            &= \varphi  d_0 \sigma - \varphi  d_1 \sigma
            + \varphi  d_2 \sigma\\
            &= d_1 \tau_0 - d_0 \tau_0 - \left( 
            d_1 \tau_1 - d_0 \tau_1 \right) 
            + d_1 \tau_2 - d_0 \tau_2
        \end{align*}
        where $\tau_i \in \Sin_1 \left( \mathbb{R} \right) $ is
        such
        that $d_i \sigma = p \circ \tau_i$. Now,
        since 
        \begin{align*}
            d_0 \sigma (0,1) &= \sigma (0,0,1), \quad
            d_0 \sigma (1,0) = \sigma(0,1,0)\\
            d_1 \sigma (0,1) &= \sigma(0,0,1), \quad
            d_1 \sigma (1,0) = \sigma (1,0,0)\\
            d_2 \sigma(0,1) &= \sigma (0,1,0), \quad
            d_2 \sigma(1,0) = \sigma (1,0,0),
        \end{align*}
        we can
        choose that $\tau_2 k^{-1} (1) = \tau_0 k^{-1} (0)$ and
        $\tau_2 k^{-1}(0) = \tau_1 k^{-1}(0)$; so
        $\tau_i k^{-1}$ gives a path lift of
        $d_i \sigma k^{-1}$. But
        $d_2 \sigma k^{-1} * d_0 \sigma k^{-1}
        \simeq d_1 \sigma k^{-1} \rel I$ (by pulling across
        the singular $2$-simplex), so
        in particular,
        $\tau_1 k^{-1} (1) = \tau_0 k^{-1} (1)$. But
        this gives

        \begin{align*}
            d_1 \tau_0 - d_0 \tau_0 &- \left( d_1 \tau_1
            - d_0 \tau_1\right) + d_1 \tau_2 - d_0 \tau_2
            = \tau_0 (1,0) - \tau_0 (0,1)
            - \tau_1 (1,0) + \tau_1 (0,1) + \tau_2 (1,0) -
            \tau_2 (0,1)\\
            &= \tau_0 k^{-1}(0) - \tau_0 k^{-1}(1)
            - \tau_1 k^{-1}(0) + \tau_1 k^{-1}(1)
            + \tau_2 k^{-1}(0) - \tau_2 k^{-1}(1)\\
            &= 0
        \end{align*}
        Hence $\varphi  \circ \partial = 0$.
        Thus $B_1(S^{1}) \subset \ker \varphi $ and
        hence $\varphi $ factors through
        $H_1 \left( S^{1} \right) $.\\
        \linebreak
        

        (iv) 
        We have $\partial \sigma_0 = 
        d_1 \sigma_0 - d_0 \sigma_0 = 
        \sigma_0 (1,0) - \sigma_0 (0,1)
        = e^{2 \pi i 0} - e^{2 \pi i} = 1-1 = 0$, so
        $\sigma_0 \in Z_1 \left( S^{1} \right) $.
        Hence $\left[ \sigma_0 \right] \in 
        H_1 \left( S^{1} \right) $.\\
        \linebreak
        (v) Define $\tau \in \Sin_1 \left( \mathbb{R} \right) $ 
        by $\tau (t_0, t_1) = t_1$. Then
        $p \circ \tau (t_0, t_1) =
        e^{2 \pi i t_1 }= \sigma_0 (t_0, t_1)$, so
        by definition,
         $\varphi  (\sigma_0) 
         = d_1 \tau - d_0 \tau
         = \tau (1,0) - \tau (0,1)
         = 0 - 1 = -1$.

         \begin{remark}[]
             Weird, I get it to be sent to
             $-1$. The definition of the maps
             $\delta^{i}$ in the notes is ambiguous, but
             I believe they are defined as
             $\delta^{i}
             = \left[ e_1, \ldots, \hat{e_i}, \ldots,
             e^{n} \right] $ where
             $\left[ v_1,\ldots 
             , v_n \right] $ is the map
             $\Delta^{n} \to \mathbb{R}^{n}$ taking
             $\sum \lambda_j e_j \mapsto 
             \sum \lambda_j v_j$.
         \end{remark}


        








    \end{proof}








    %\bibliography{../refs.bib}
\end{document}
