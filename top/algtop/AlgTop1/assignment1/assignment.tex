\documentclass[reqno]{amsart}
\usepackage{amscd, amssymb, amsmath, amsthm}
\usepackage{graphicx}
\usepackage[colorlinks=true,linkcolor=blue]{hyperref}
\usepackage[utf8]{inputenc}
\usepackage[T1]{fontenc}
\usepackage{textcomp}
\usepackage{babel}
%% for identity function 1:
\usepackage{bbm}
%%For category theory diagrams:
\usepackage{tikz-cd}


\setlength\parindent{0pt}

\pdfsuppresswarningpagegroup=1

\newtheorem{theorem}{Theorem}[section]
\newtheorem{lemma}[theorem]{Lemma}
\newtheorem{proposition}[theorem]{Proposition}
\newtheorem{corollary}[theorem]{Corollary}
\newtheorem{conjecture}[theorem]{Conjecture}

\theoremstyle{definition}
\newtheorem{definition}[theorem]{Definition}
\newtheorem{example}[theorem]{Example}
\newtheorem{exercise}[theorem]{Exercise}
\newtheorem{problem}[theorem]{Problem}
\newtheorem{question}[theorem]{Question}

\theoremstyle{remark}
\newtheorem*{remark}{Remark}
\newtheorem*{note}{Note}
\newtheorem*{solution}{Solution}



%Inequalities
\newcommand{\cycsum}{\sum_{\mathrm{cyc}}}
\newcommand{\symsum}{\sum_{\mathrm{sym}}}
\newcommand{\cycprod}{\prod_{\mathrm{cyc}}}
\newcommand{\symprod}{\prod_{\mathrm{sym}}}

%Linear Algebra

\DeclareMathOperator{\Span}{span}
\DeclareMathOperator{\Ima}{Im}
\DeclareMathOperator{\diag}{diag}
\DeclareMathOperator{\Ker}{Ker}
\DeclareMathOperator{\ob}{ob}
\DeclareMathOperator{\Hom}{Hom}
\DeclareMathOperator{\sk}{sk}
\DeclareMathOperator{\Vect}{Vect}
\DeclareMathOperator{\Set}{Set}
\DeclareMathOperator{\Group}{Group}
\DeclareMathOperator{\Ring}{Ring}
\DeclareMathOperator{\Ab}{Ab}
\DeclareMathOperator{\Top}{Top}
\DeclareMathOperator{\hTop}{hTop}
\DeclareMathOperator{\Htpy}{Htpy}
\DeclareMathOperator{\Cat}{Cat}
\DeclareMathOperator{\CAT}{CAT}
\DeclareMathOperator{\Cone}{Cone}
\DeclareMathOperator{\dom}{dom}
\DeclareMathOperator{\cod}{cod}
\DeclareMathOperator{\Aut}{Aut}
\DeclareMathOperator{\Mat}{Mat}
\DeclareMathOperator{\Fin}{Fin}
\DeclareMathOperator{\rel}{rel}
\DeclareMathOperator{\Int}{Int}
\DeclareMathOperator{\sgn}{sgn}
\DeclareMathOperator{\Homeo}{Homeo}

%Row operations
\newcommand{\elem}[1]{% elementary operations
\xrightarrow{\substack{#1}}%
}

\newcommand{\lelem}[1]{% elementary operations (left alignment)
\xrightarrow{\begin{subarray}{l}#1\end{subarray}}%
}

%SS
\DeclareMathOperator{\supp}{supp}
\DeclareMathOperator{\Var}{Var}

%NT
\DeclareMathOperator{\ord}{ord}

%Alg
\DeclareMathOperator{\Rad}{Rad}
\DeclareMathOperator{\Jac}{Jac}

%Misc
\newcommand{\SL}{{\mathrm{SL}}}
\newcommand{\mobgp}{{\mathrm{PSL}_2(\mathbb{C})}}
\newcommand{\id}{{\mathrm{id}}}
\newcommand{\Mod}{{\mathrm{Mod}}}
\newcommand{\ud}{{\mathrm{d}}}
\newcommand{\Vol}{{\mathrm{Vol}}}
\newcommand{\Area}{{\mathrm{Area}}}
\newcommand{\diam}{{\mathrm{diam}}}


\newcommand{\reg}{{\mathtt{reg}}}
\newcommand{\geo}{{\mathtt{geo}}}

\newcommand{\tori}{{\mathcal{T}}}
\newcommand{\cpn}{{\mathtt{c}}}
\newcommand{\pat}{{\mathtt{p}}}




\begin{document}
\textit{Solution to (iv):} Let $f_1 \colon I \times I \to \mathbb{C}$ 
be the "retraction" homotopy $f_1 (x,t) =
\gamma_1 * \gamma_2 * \overline{\gamma_1} \left( x(1-t) \right) $.
Thus $\gamma_1 * \gamma_2 * \overline{\gamma_1}$ is freely homotopic
to the constant path at $0$. Similarly
$\gamma_2$ is also freely homotopic to the constant path at $0$ by, for
example, $f_2 (x,t) = \gamma_2 \left( tx \right) $. Then
the map
$H \colon I \times I \to \mathbb{C}$ by
\[
H (x,t) = \begin{cases}
    f_1(x,2t),& t \in \left[ 0, \frac{1}{2} \right],\\
    f_2 (x, 2t-1),& t\in \left[ \frac{1}{2},1 \right] 
\end{cases}
\] 
clearly has $H(x,0) = \gamma_1 * \gamma_2 * \overline{\gamma_1}(x)$ 
and $H(x,1) = \gamma_2(x)$, and
as $f_1(x,1) = c_{0} = f_2(x, 0)$ where
$c_0$ is the constant path at $0$, we get that $H$ is continuous
by the gluing lemma since $H$ is continuous on
$I \times \left[ 0,\frac{1}{2} \right] $ and
$I \times \left[ \frac{1}{2},1 \right] $ and
agrees on the intersection.\\
\linebreak
They are not homotopic relative to the basepoint since
$\pi_1 \left( \mathbb{C} - \left\{ -1,1 \right\}  \right) 
\approx \pi_1 \left( S^{1} \vee S^{1} \right) 
= \left< a,b \right>$ where the loop
$\gamma_1$ corresponds to $a$, say, and $\gamma_2$ corresponds to
$b$, for example. But then
$\overline{\gamma_1} = a^{-1}$, so
$\gamma_1 * \gamma_2 * \overline{\gamma_1}$ corresponds to
$a b a^{-1}$ which does not correspond to 
$b$ which is what $\gamma_2$ corresponds to since
we have no relations giving $aba^{-1} = b$.\\
\linebreak
\textit{Additional note:} Since $aba^{-1}b^{-1}$ is a commutator,
it is trivial in the abelianization of $\pi_1 \left( \mathbb{C}
- \left\{ -1,1 \right\} \right) $ which is the
first homology group of $\mathbb{C} - \left\{ -1,1 \right\} $ as
$\mathbb{C} - \left\{ -1,1 \right\} $ is path-connected, so
the two loops in question are homologous.
    %\bibliography{../refs.bib}
\end{document}
