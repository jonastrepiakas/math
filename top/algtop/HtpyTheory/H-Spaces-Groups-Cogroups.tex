\section{H-Spaces, H-Groups and H-Cogroups}

An H-space or H-group is a space with a product that satisfies
some of the laws of a group \textit{but only up to homotopy}.
An H-cogroup is a dual notion. The
"H" stands for "Hopf" or for "Homotopy".

\begin{definition}[H-Space, homotopy associativity, homotopy inverse]
    An \textit{H-space} is a pointed space $X \in \Top_*$ with
    base point $e$, together with a map
    \[
    \cdot \colon X \times X \to X
    \] 
    sending $\left( x,y \right) \mapsto x \cdot y$ such that
    $e \cdot  e = e$, and the maps
    $X \to X$ taking $x \mapsto x \cdot e$ and
    $x \mapsto e \cdot x$ are each homotopic 
    $\rel \left\{ e \right\} $ to the identity.\\
    \linebreak
    It is said to be \textit{homotopy associative} if the
    maps $X \times X \times X \to X$ taking
    $\left( x,y,z \right) $ to
    $\left( x \cdot y \right) \cdot z$ and
    to $x \cdot  \left( y \cdot z \right) $ are
    homotopic $\rel \left\{ \left( e,e,e \right)  \right\} $.\\
    \linebreak
    It is said to have a \textit{homotopy inverse}
    $\hat{-} \colon X \to X$ if
    $\hat{e} = e$ and the maps
    $X \to X$ taking $x$ to
    $x \cdot \hat{x}$ and to
    $\hat{x}\cdot x$ are each homotopic $\rel \left\{ e \right\} $ 
    to the constant map to $\left\{ e \right\} $.
\end{definition}

\begin{definition}[H-group]
    An \textit{H-group} is a homotopy associative H-space
    with a given homotopy inverse.
\end{definition}

There are two main examples: the first
is the class of topological groups, the second is the class
of "loop spaces". 

\begin{definition}[Loop space]
    The loop space on a space $X$ is the space
    \[
    \Omega X = \left( X,* \right)^{\left( S^{1},* \right) },
    \] 
    i.e., $X^{S^{1}}$ in the pointed category. 
    The product is concatenation of loops, and
    the homotopy inverse is loop reversal.
    $\Omega X$ is a pointed space with base point being
    the constant loop at $*$.
\end{definition}

\begin{definition}[Operations on maps]
    If $f \colon X \to Z$ and
    $g \colon Y \to W$ are maps, let
    $f \vee g \colon X \vee Y \to Z \vee W$ be the
    induced map on the one-point union.\\
    Let $\nabla \colon Z \vee Z \to Z$ be the
    codiagonal, i.e., the identity on
    both factors.\\
    We also define
    $f \veebar g \colon X \vee Y \to Z$ as
    the composition $f \veebar g =
    \nabla \circ \left( f \vee g \right) $ ; i.e., the
    map which is $f$ on $X$ and $g$ on $Y$.
\end{definition}

\begin{definition}[H-cogroup]
    An \textit{H-cogroup} is a pointed space $Y$ and
    a map $\gamma \colon Y \to Y \vee Y$ such that
    the following three conditions are satisfied:
    \begin{enumerate}
        \item The constant map $* \colon Y \to Y$ to the
            base point is a \textit{homotopy identity}; i.e.,
            the compositions
            $\left( * \veebar \id \right) \circ \gamma$ and
            $\left( \id \veebar * \right) \circ \gamma$ of
            $Y \stackrel{\gamma}{\to} Y \vee Y \to Y$ 
            are both homotopic to the identity rel base
            point.
        \item It is homotopy associative. That is, the compositions
            $\left( \gamma \vee \id \right) \circ \gamma$
            and $\left( \id \vee \gamma \right) \circ \gamma$ 
            of $Y \stackrel{\gamma}{\to} Y \vee Y
            \to Y \vee Y \vee Y$ are homotopic to one another
            rel base point.
        \item There is a homotopy inverse $i \colon Y \to Y$.
            That is, $\left( \id \veebar i \right) \circ \gamma$ 
            and $\left( i \veebar \id \right) \circ \gamma$
            of $Y \stackrel{\gamma}{\to} Y \vee Y \to Y$ 
            are both homotopic to the constant map to the
            base point rel base point.
    \end{enumerate}
\end{definition}

One important class of examples is the
reduced suspensions: the
"coproduct" $\gamma \colon SX \to SX \vee SX$ is given by
\[
\gamma (t,x) = 
\begin{cases}
    \left( 2t,x \right)_{1},& t \le \frac{1}{2}\\
    \left( 2t-1,x \right)_{2},& t\ge \frac{1}{2}
\end{cases}
\] 
where the subscripts indicate in which copy
of $SX$ in the one-point union the indicated point lies.

The homotopy inverse is just reversal of the $t$ parameter.


\begin{theorem}[]\label{Thm:H-group-cogroup}
    In the pointed category:
    \begin{enumerate}
        \item If $Y$ is an H-group then
            $\left[ X ; Y \right] $ is a group with
            multiplication induced by
            $\left( f\cdot g \right) (x) = f(x) \cdot g(x)$.
        \item If $X$ is an H-cogroup then
            $\left[ X;Y \right] $ is a group with multiplication
            induced by $f * g = \left( f \veebar g \right) \circ
            \gamma$.
        \item If $X$ is an  H-cogroup and $Y$ is an
            H-space then the two multiplications above
            on $\left[ X;Y \right] $ coincide and are abelian.
    \end{enumerate}
\end{theorem}

\begin{proof}
    We first show that $f\cdot g$ is well defined
    on $\left[ X;Y \right] \times \left[ X;Y \right] 
    \to \left[ X;Y \right] $.
    Suppose $f \simeq f'$ and $g \simeq g'$.
    Then we must show that
    $f(x) \cdot g(x) \simeq f'(x) \cdot g'(x)$.
    By assumption,
    $\id \simeq \widehat{f(x)}\cdot  f'$ and
    $\id \simeq g'(x) \cdot \widehat{g(x)}$, so
    \begin{align*}
        f'(x) \cdot g'(x)
        &\simeq \left( f(x) \cdot \widehat{f(x)} \right) 
        \cdot \left( \left( f'(x) \cdot g'(x) \right) \cdot 
        \left( \widehat{g(x)} \cdot g(x) \right) \right)\\
        &\simeq f(x) \cdot \left( \widehat{f(x)}\cdot 
        f'(x) \right) \cdot \left( g'(x) \cdot 
        \widehat{g(x)} \right) \cdot g(x)\\
        &\simeq f(x) \cdot g(x).
    \end{align*}
    Now we check the group axioms.
    Associativity follows from
    homotopy associativity of $Y$. 
    The constant map to the basepoint
     of $Y$ is an identity, let us denote it by
     $e$, since
     $\left( f \cdot e \right) (x) 
     = f(x) \cdot e(x) \simeq
     f(x) \simeq e(x) \cdot f(x) 
     = \left( e\cdot f \right) (x) \rel
     \left\{ e(x)=* \right\}$ since $Y$ is an
     H-space. So there
     exists a map $H \colon Y \times I \to Y$ such that
     $H(f(x),0) = f(x) \cdot *$ ad
     $H\left( f(x), 1 \right) = f(x)$.
     Define $G \colon X \times I \to Y$ by
     $G(x,t) = H\left( f(x),t \right) $.
     This defines a homotopy
     of $f\cdot e$ with $f$.
     One can do the same for $e\cdot f \simeq f$.
     Lastly, given $f \in \left[ X;Y \right] $, 
     define $\widehat{f} \colon X \to Y$ by
     $\widehat{f}(x) = \widehat{f(x)}$. This
     is continuous as the composition
     $\widehat{-} \circ f \colon X \to Y$. 
     Now, for each $x \in X$,
     $\left( f\cdot \widehat{f} \right) (x)
     = f(x) \cdot \widehat{f(x)}
     \simeq e \rel{e}$, so
     there exists $H \colon Y \times I \to Y$ such that
     $H\left( f\left( x \right) \cdot \widehat{f}(x),
     0 \right) = f(x) \cdot \widehat{f(x)}$ and
     $H\left( f(x) \cdot \widehat{f(x)},1 \right) 
     = e$. Then again defining
     $G(x,t) = H\left( f(x)\cdot \widehat{f(x)},t \right) $ 
     defines a homotopy from
     $f \cdot \widehat{f}$ to $e$ relative
     $*$. Etc.\\
     \linebreak
     For (2), associativity is shown as follows:
     $\left( f*g \right) *h
     = \left[ \left( \left( f \veebar g \right) \circ \gamma
     \right) \veebar h \right] \circ \gamma$ which is the
     composition
     \[
     X \stackrel{\gamma}{\to} X \vee X
     \stackrel{\gamma \vee \id}{\to} 
     \left( X \vee X \right) \vee X
     \stackrel{\left( f \veebar g \right) \veebar h}{\to} 
     Y.
     \] 
     The first composition 
     $\left( \gamma \vee \id \right) \circ \gamma$ is
     homotopic to $\left( \id \vee \gamma \right) \circ \gamma$ 
     by condition (2) of an H-cogroup, and
     the maps $\left( f \veebar g \right) \veebar h$ 
     and $f \veebar \left( g \veebar h \right) $ are
     equal, which provides the homotopy
     from $\left( f*g \right) *h $ to
     $f* \left( g*h \right) $. The other
     parts are similar.\\
     \linebreak
     For (3), we need the following lemma:
     \begin{lemma}[]
         If $X$ is an H-cogroup and $Y$ an
         H-space, then for
         $f,g,h,k \colon X \to Y$, we have
         \[
             \left( f\cdot g \right) * \left( h\cdot k \right) 
             = \left( f*h \right) \cdot \left( g*k \right) .
         \] 
     \end{lemma}

     \begin{proof}
         Suppose $x \in X$ is an arbitary point
         and that $\gamma(x) = 
         \left( w,* \right) \in X \vee X$ (recall
         that we can view $X \vee X$ as
         $X \times \left\{ * \right\} \cup 
         \left\{ * \right\} \times X$ ), so
         this amounts to $\left( w,* \right) $ just
         representing a point in one copy of
         $X$ and $\left( *,w' \right) $ would then
         represent a point in the other copy.\\
         Now
         \[
             \left( f\cdot g \right) *
             \left( h\cdot k \right) (x)
             = \left( \left( f \cdot g \right) 
             \veebar \left( h\cdot k \right) \right) 
             \left( w,* \right) 
             = \left( f\cdot g \right) (w) 
             = f(w) \cdot g(w)
         \] 
         and
         \[
             \left( f*h \right) \cdot 
             \left( g*k \right) (x)
             = (f*h)(x) \cdot (g*k)(x)
             = f(w) \cdot g(w).
         \] 
         The case $\gamma(x) = (*,w')$ is similar.
     \end{proof}

     Returning to part (3), note first that
     for both product, the identity $\id$ is given
     by the constant map to the base point by
     condition (1) for an H-cogroup.\\
     Operating in $\left[ X;Y \right] $, we have
     \[
         \left( \alpha \cdot \beta  \right) *
         \left( \gamma \cdot \delta \right) 
         = \left( \alpha * \gamma \right) \cdot 
         \left( \beta * \delta \right) .
     \] 
     Thus 
     \[
     \alpha * \beta =
     \left( \id \cdot \alpha \right) *
     \left( \beta \cdot \id \right) 
     = \left( \id * \beta \right) \cdot 
     \left( \alpha * \id \right) = 
     \beta \cdot  \alpha,
     \] 
     and
     \[
     \alpha * \beta =
     \left( \alpha \cdot  \id \right) *
     \left( \id \cdot  \beta  \right) =
     \left( \alpha * \id  \right) \cdot 
     \left( \id * \beta  \right) = \alpha \cdot \beta.
     \] 
     Therefore
     $\alpha \cdot  \beta= \beta \cdot  \alpha
     = \alpha * \beta$.



\end{proof}


