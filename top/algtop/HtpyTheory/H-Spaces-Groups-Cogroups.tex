\documentclass[reqno]{amsart}
\usepackage{amscd, amssymb, amsmath, amsthm}
\usepackage{graphicx}
\usepackage[colorlinks=true,linkcolor=blue]{hyperref}
\usepackage[utf8]{inputenc}
\usepackage[T1]{fontenc}
\usepackage{textcomp}
\usepackage{babel}
%% for identity function 1:
\usepackage{bbm}
%%For category theory diagrams:
\usepackage{tikz-cd}

%\usepackage[backend=biber]{biblatex}
%\addbibresource{.bib}


\setlength\parindent{0pt}

\pdfsuppresswarningpagegroup=1

\newtheorem{theorem}{Theorem}[section]
\newtheorem{lemma}[theorem]{Lemma}
\newtheorem{proposition}[theorem]{Proposition}
\newtheorem{corollary}[theorem]{Corollary}
\newtheorem{conjecture}[theorem]{Conjecture}

\theoremstyle{definition}
\newtheorem{definition}[theorem]{Definition}
\newtheorem{example}[theorem]{Example}
\newtheorem{exercise}[theorem]{Exercise}
\newtheorem{problem}[theorem]{Problem}
\newtheorem{question}[theorem]{Question}

\theoremstyle{remark}
\newtheorem*{remark}{Remark}
\newtheorem*{note}{Note}
\newtheorem*{solution}{Solution}



%Inequalities
\newcommand{\cycsum}{\sum_{\mathrm{cyc}}}
\newcommand{\symsum}{\sum_{\mathrm{sym}}}
\newcommand{\cycprod}{\prod_{\mathrm{cyc}}}
\newcommand{\symprod}{\prod_{\mathrm{sym}}}

%Linear Algebra

\DeclareMathOperator{\Span}{span}
\DeclareMathOperator{\im}{im}
\DeclareMathOperator{\diag}{diag}
\DeclareMathOperator{\Ker}{Ker}
\DeclareMathOperator{\ob}{ob}
\DeclareMathOperator{\Hom}{Hom}
\DeclareMathOperator{\Mor}{Mor}
\DeclareMathOperator{\sk}{sk}
\DeclareMathOperator{\Vect}{Vect}
\DeclareMathOperator{\Set}{Set}
\DeclareMathOperator{\Group}{Group}
\DeclareMathOperator{\Ring}{Ring}
\DeclareMathOperator{\Ab}{Ab}
\DeclareMathOperator{\Top}{Top}
\DeclareMathOperator{\hTop}{hTop}
\DeclareMathOperator{\Htpy}{Htpy}
\DeclareMathOperator{\Cat}{Cat}
\DeclareMathOperator{\CAT}{CAT}
\DeclareMathOperator{\Cone}{Cone}
\DeclareMathOperator{\dom}{dom}
\DeclareMathOperator{\cod}{cod}
\DeclareMathOperator{\Aut}{Aut}
\DeclareMathOperator{\Mat}{Mat}
\DeclareMathOperator{\Fin}{Fin}
\DeclareMathOperator{\rel}{rel}
\DeclareMathOperator{\Int}{Int}
\DeclareMathOperator{\sgn}{sgn}
\DeclareMathOperator{\Homeo}{Homeo}
\DeclareMathOperator{\SHomeo}{SHomeo}
\DeclareMathOperator{\PSL}{PSL}
\DeclareMathOperator{\Bil}{Bil}
\DeclareMathOperator{\Sym}{Sym}
\DeclareMathOperator{\Skew}{Skew}
\DeclareMathOperator{\Alt}{Alt}
\DeclareMathOperator{\Quad}{Quad}
\DeclareMathOperator{\Sin}{Sin}
\DeclareMathOperator{\Supp}{Supp}
\DeclareMathOperator{\Char}{char}
\DeclareMathOperator{\Teich}{Teich}
\DeclareMathOperator{\GL}{GL}
\DeclareMathOperator{\tr}{tr}
\DeclareMathOperator{\codim}{codim}
\DeclareMathOperator{\coker}{coker}
\DeclareMathOperator{\corank}{corank}
\DeclareMathOperator{\rank}{rank}
\DeclareMathOperator{\Diff}{Diff}
\DeclareMathOperator{\Bun}{Bun}
\DeclareMathOperator{\Sm}{Sm}
\DeclareMathOperator{\Fr}{Fr}
\DeclareMathOperator{\Cob}{Cob}
\DeclareMathOperator{\Ext}{Ext}
\DeclareMathOperator{\Tor}{Tor}



%Row operations
\newcommand{\elem}[1]{% elementary operations
\xrightarrow{\substack{#1}}%
}

\newcommand{\lelem}[1]{% elementary operations (left alignment)
\xrightarrow{\begin{subarray}{l}#1\end{subarray}}%
}

%SS
\DeclareMathOperator{\supp}{supp}
\DeclareMathOperator{\Var}{Var}

%NT
\DeclareMathOperator{\ord}{ord}

%Alg
\DeclareMathOperator{\Rad}{Rad}
\DeclareMathOperator{\Jac}{Jac}

%Misc
\newcommand{\SL}{{\mathrm{SL}}}
\newcommand{\mobgp}{{\mathrm{PSL}_2(\mathbb{C})}}
\newcommand{\id}{{\mathrm{id}}}
\newcommand{\MCG}{{\mathrm{MCG}}}
\newcommand{\PMCG}{{\mathrm{PMCG}}}
\newcommand{\SMCG}{{\mathrm{SMCG}}}
\newcommand{\ud}{{\mathrm{d}}}
\newcommand{\Vol}{{\mathrm{Vol}}}
\newcommand{\Area}{{\mathrm{Area}}}
\newcommand{\diam}{{\mathrm{diam}}}
\newcommand{\End}{{\mathrm{End}}}


\newcommand{\reg}{{\mathtt{reg}}}
\newcommand{\geo}{{\mathtt{geo}}}

\newcommand{\tori}{{\mathcal{T}}}
\newcommand{\cpn}{{\mathtt{c}}}
\newcommand{\pat}{{\mathtt{p}}}

\let\Cap\undefined
\newcommand{\Cap}{{\mathcal{C}}ap}
\newcommand{\Push}{{\mathcal{P}}ush}
\newcommand{\Forget}{{\mathcal{F}}orget}




\begin{document}
\section{H-Spaces, H-Groups and H-Cogroups}

An H-space or H-group is a space with a product that satisfies
some of the laws of a group \textit{but only up to homotopy}.
An H-cogroup is a dual notion. The
"H" stands for "Hopf" or for "Homotopy".

\begin{definition}[H-Space, homotopy associativity, homotopy inverse]
    An \textit{H-space} is a pointed space $X \in \Top_*$ with
    base point $e$, together with a map
    \[
    \cdot \colon X \times X \to X
    \] 
    sending $\left( x,y \right) \mapsto x \cdot y$ such that
    $e \cdot  e = e$, and the maps
    $X \to X$ taking $x \mapsto x \cdot e$ and
    $x \mapsto e \cdot x$ are each homotopic 
    $\rel \left\{ e \right\} $ to the identity.\\
    \linebreak
    It is said to be \textit{homotopy associative} if the
    maps $X \times X \times X \to X$ taking
    $\left( x,y,z \right) $ to
    $\left( x \cdot y \right) \cdot z$ and
    to $x \cdot  \left( y \cdot z \right) $ are
    homotopic $\rel \left\{ \left( e,e,e \right)  \right\} $.\\
    \linebreak
    It is said to have a \textit{homotopy inverse}
    $\hat{-} \colon X \to X$ if
    $\hat{e} = e$ and the maps
    $X \to X$ taking $x$ to
    $x \cdot \hat{x}$ and to
    $\hat{x}\cdot x$ are each homotopic $\rel \left\{ e \right\} $ 
    to the constant map to $\left\{ e \right\} $.
\end{definition}

\begin{definition}[H-group]
    An \textit{H-group} is a homotopy associative H-space
    with a given homotopy inverse.
\end{definition}

There are two main examples: the first
is the class of topological groups, the second is the class
of "loop spaces". 

\begin{definition}[Loop space]
    The loop space on a space $X$ is the space
    \[
    \Omega X = \left( X,* \right)^{\left( S^{1},* \right) },
    \] 
    i.e., $X^{S^{1}}$ in the pointed category. 
    The product is concatenation of loops, and
    the homotopy inverse is loop reversal.
    $\Omega X$ is a pointed space with base point being
    the constant loop at $*$.
\end{definition}

\begin{definition}[Operations on maps]
    If $f \colon X \to Z$ and
    $g \colon Y \to W$ are maps, let
    $f \vee g \colon X \vee Y \to Z \vee W$ be the
    induced map on the one-point union.\\
    Let $\nabla \colon Z \vee Z \to Z$ be the
    codiagonal, i.e., the identity on
    both factors.\\
    We also define
    $f \veebar g \colon X \vee Y \to Z$ as
    the composition $f \veebar g =
    \nabla \circ \left( f \vee g \right) $ ; i.e., the
    map which is $f$ on $X$ and $g$ on $Y$.
\end{definition}

\begin{definition}[H-cogroup]
    An \textit{H-cogroup} is a pointed space $Y$ and
    a map $\gamma \colon Y \to Y \vee Y$ such that
    the following three conditions are satisfied:
    \begin{enumerate}
        \item The constant map $* \colon Y \to Y$ to the
            base point is a \textit{homotopy identity}; i.e.,
            the compositions
            $\left( * \veebar \id \right) \circ \gamma$ and
            $\left( \id \veebar * \right) \circ \gamma$ of
            $Y \stackrel{\gamma}{\to} Y \vee Y \to Y$ 
            are both homotopic to the identity rel base
            point.
        \item It is homotopy associative. That is, the compositions
            $\left( \gamma \vee \id \right) \circ \gamma$
            and $\left( \id \vee \gamma \right) \circ \gamma$ 
            of $Y \stackrel{\gamma}{\to} Y \vee Y
            \to Y \vee Y \vee Y$ are homotopic to one another
            rel base point.
        \item There is a homotopy inverse $i \colon Y \to Y$.
            That is, $\left( \id \veebar i \right) \circ \gamma$ 
            and $\left( i \veebar \id \right) \circ \gamma$
            of $Y \stackrel{\gamma}{\to} Y \vee Y \to Y$ 
            are both homotopic to the constant map to the
            base point rel base point.
    \end{enumerate}
\end{definition}














    %\printbibliography
\end{document}
