\section{The Compact-Open Topology}

Recall that $Y^{X}$ denotes the \textit{set} of 
\textit{continuous} functions $X \to Y$.

\begin{definition}[]
    The \textit{compact-open topology} on $Y^{X}$ is the
    topology generated by the sets
    $M \left( K, U \right) 
    = \left\{ f \in Y^{X} \mid 
    f(K) \subset U \right\} $ where
    $K \subset X$ is compact and
    $U \subset Y$ is open.
\end{definition}
Generated here means that these sets form
a \textit{subbasis} for the open sets.\\


\begin{lemma}[]\label{Lemma:Compact-Open-Subbasis}
    Let $\mathcal{K}$ be a collection of compact subsets
    of $X$ containing a neighborhood base
    at each point of $X$. Let
    $\mathcal{B}$ be a subbasis for the open sets
    of $Y$. Then
    the collection
     \[
    \left\{ M\left( K,U \right)  \mid 
    K \in \mathcal{K}, B \in \mathcal{B} \right\} 
    \] 
    forms a subbasis for the compact-open topology
    on $Y^{X}$.
\end{lemma}

\begin{proof}
    Recall first that a subbasis is a collection whose
    union is the whole space and such that the
    collection of finite intersections of elements of the
    subbasis form a basis.\\
    In particular, noting that
    $M \left( K, U \right) \cap M\left( K,V \right) =
    M\left( K, U \cap V \right) $, this implies that it
    suffices to consider the case when $\mathcal{B}$ is
    a basis.\\
    So to show that the collection in question
    is a subbasis, it suffices to show that
    given $f \in M(K,U)$, there exist
    $K_1, \ldots,K_n \in \mathcal{K}$ and
    $U_1, \ldots, U_n \in \mathcal{B}$ such that
    $f \in \bigcap_{i=1}^{n} M\left( K_i,U_i \right) 
    \subset M\left( K,U \right) $.

    For each $x \in K$, there is an open
    set $U_x \in \mathcal{B}$ with
    $f(x) \in U_x \subset U$ (since $\mathcal{B}$ was
    assumed to be a basis), and there
    exists a neighborhood $K_x \in \mathcal{K}$  of
    $x$ such that $f(K_x) \subset U_x$ (since
    $f$ is continuous and $\mathcal{K}$ was assumed
    to contain a neighborhood base at each point of $X$ ).
    Thus $f \in M\left( K_x, U_x \right) $.
    Now, covering $K$ with these sets
    $K \subset \bigcup_{x \in K} K_{x}$. By 
    compactness of $K$, there exists a finite subcover
    $K \subset K_{x_1} \cup \ldots \cup K_{x_n}$. Then
    $f \in \bigcap_{i=1}^{n} M\left( K_{x_i}, U_{x_i} \right) 
    \subset M\left( K,U \right) $.
\end{proof}

\begin{proposition}[]\label{Prop:SJUSIO12}
    For $X$ locally compact Hausdorff, the
    "evaluation map" $e\colon Y^{X} \times X
    \to Y$, defined by $e\left( f,x \right) =
    f(x)$, is continuous.
\end{proposition}

\begin{proof}
    Let $\left( f,x \right) \in 
    Y^{X} \times X$ and
    $U$ a neighborhood of $f(x) \in  Y$.
    Now we make use of the following lemma:
    \begin{lemma}[]\label{Lemma:SIJDLII}
        If $X$ is a locally compact Hausdorff space, then
        each neighborhood of a point $x \in X$ contains
        a compact neighborhood of $X$.
        In particular, $X$ is completely regular.
    \end{lemma}

    \begin{proof}
        Let $C$ be a compact neighborhood of $x$ and
        $U$ an arbitrary neighborhood of $x$.
        Since $X$ is Hausdorff, $C$ is closed, so
        $\left( X - U \right) \cap C$ is a closed
        subspace of $C$, hence compact. Now,
        for each point $z \in (X-U) \cap C$, choose, by Hausdorffness,
        open neighborhoods $U_z',V_z'$ of $z$ and $x$, respectively,
        and consider $W' := \bigcup_{z \in (X-U)\cap C} 
        U_z'$. Since this is open,
        $C - W'$ is closed hence compact. Furthermore,
        it is contained in $U$ and contains $x$.\\
        \linebreak
        \textit{Alternative proof due to Bredon:}
        Let $C$ be a compact neighborhood of $x$ and
        $U$ an arbitrary neighborhood
        of $x$. Let $V \subset C \cap U$ be open with
        $x \in V$. Then $\overline{V} \subset C$ is compact
        Hausdorff, hence regular, so there exists
        a neighborhood $N \subset V$ of $x$ in
        $C$ which is closed in $\overline{V}$ 
        and hence closed in $X$. Since
        $N$ is closed in the compact space $C$, it
        is compact. Since $N$ is a neighborhood of
        $x$ in $\overline{V}$ and since
        $N = N \cap V$, $N$ is a neighborhood of $x$ in
        the open set $V$ and hence in $X$.
    \end{proof}

    By Lemma \ref{Lemma:SIJDLII}, 
    there exists a compact neighborhood $K$ of
    $x$ such that
    $f(K) \subset U$. Hence
    $f \in M\left( K,U \right) $, and
    $e \left( M \left( K, U \right) \times 
    K \right) \subset U$.
    This finishes the proof.
\end{proof}

\begin{theorem}[]\label{Thm:Compact-Open-Top}
    Let $X$ be locally compact Hausdorff and $Y$ and
    $T$ arbitrary Hausdorff spaces. Given a function
    $f \colon X \times T \to Y$, define, for each
    $t \in T$, the function
    $f_t \colon X \to Y$ by
    $f_t (x) = f(x,t)$. Then $f$ is continuous
    if and only if both of the following conditions
    hold:
    \begin{enumerate}
        \item Each $f_t$ is continuous
        \item The function $T \to Y^{X}$ taking
            $t \mapsto f_t$ is continuous.
    \end{enumerate}
\end{theorem}

\begin{proof}
    The "if" implication follows from the fact
    that $f$ is the composition
    \[
    X \times T \stackrel{\left( x,t \right) \mapsto 
    \left( f_t, x \right) }{\longrightarrow} Y^{X} \times X
    \stackrel{e}{\to} Y.
    \] 
    Now the evaluation map is continuous
    by Proposition \ref{Prop:SJUSIO12} since
    $X$ is assumed to be locally compact Hausdorff and
    since $f_t$ is assumed to
    be continuous for all $t$ by condition (1); and
    $\left( x,t \right) \mapsto \left( f_t, x \right) $ 
    is continuous since 
    $t \mapsto f_t$ is assumed to be continuous
    by condition (2).\\
    Conversely, for the "only if" implication,
    (1) follows from the fact that $f_t$ is the
    composition
    \[
    X \stackrel{x \mapsto (x,t)}{\to} X \times T
    \stackrel{f}{\to} Y.
    \] 
    To prove (2), let
    $t \in T$ be given and
    $f_t \in M\left( K, U \right) $. It suffices
    to find a neighborhood $W$ of $t$ in $T$ such that
    $t' \in W$ implies that
    $f_{t'} \in M\left( K,U \right) $ (i.e., it suffices to
    prove conditions for continuity for a subbasis only).
    For $x \in K$, there are open neighborhoods $V_x \subset 
    X$ of $x$ and $W_x \subset T$ of $t$ such that
    $f\left( V_x \times W_x \right) \subset U$.
    By compactness, $K \subset 
    V_{x_1} \cup  \ldots \cup  V_{x_n} =: V$ for
    some $V_{x_i}$. Let
    $W = \bigcap_{i=1}^{n} W_{x_i}$. Then
    $f\left( K \times W \right) \subset 
    f\left( V \times W \right)  \subset U$.
    So $t' \in W$ implies that
    $f_{t'} \in M\left( K,U \right) $ as claimed.
\end{proof}

\begin{note}
    This theorem implies that
    a homotopy $X \times I \to Y$ with $X$ locally
    compact is the same thing as a path
    $I \to Y^{X}$ when we give
    $Y^{X}$ the compact-open topology.
\end{note}

\begin{note}
    This is precisely the reason why, when we define
    $\MCG (X)$, we define it as
    $\pi_0 \Homeo^{+}(X, \partial X)$ where we equip
    $\Homeo^{+}\left( X, \partial X \right) $ with the
    subspace topology inherited from
    $X^{X}$ in the compact-open topology.
    By the above theorem, a path 
    $I \to \Homeo^{+} \left( X, \partial X \right) $ 
    given as $t \mapsto \gamma_t$
    is continuous if and only if
    the associated function
    $\gamma \colon X \times I \to X$ given
    by $\gamma(x,t) = \gamma_t(x)$ is
    continuous. But since each
    $\gamma_t$ is a self-homeomorphism of $X$, this
    just tells us that $\gamma$ is an isotopy of
    $X$. So path components
    of $\Homeo^{+} \left( X , \partial X \right) $ 
    correspond to isotopy classes of orientation-preserving
    self-homeomorphisms of $X$ fixing the boundary point-wise.
\end{note}

\begin{theorem}[The Exponential Law]
    Let $X$ and $T$ be locally compact Hausdorff spaces
    and let $Y$ be an arbitrary Hausdorff space. Then
    there is the homeomorphism
    \[
    Y^{X \times T} \stackrel{\cong}{\to} \left( Y^{X} \right)^{T}
    \] 
    taking $f \mapsto f^{*}$ where
    $f^{*}(t) (x) = f(x,t) = f_t(x)$.
\end{theorem}

\begin{proof}
    By Theorem \ref{Thm:Compact-Open-Top}, the assignment
    $f \mapsto f^{*}$ is a bijection.\\
    We must show it and its inverse to be continuous.
    Let $U \subset Y$ be open and
    $K \subset X, K'\subset T$ be compact. Then
    \begin{align*}
        f \in M \left( K \times K', U \right) 
        &\iff \left( t \in K', x \in K
        \implies f_t(x) = f(x,t) \in U \right) \\
        &\iff \left( t \in K' \implies 
        f_t \in M \left( K, U \right) \right) \\
        &\iff f^{*} \in M \left( K', M\left( K ,U \right)  \right).
    \end{align*}
    Now, the $K \times K'$ are compact
    subsets of $X \times T$, and the collection
    of all these over $X \times T$ contain
    a neighborhood basis at each point since
    $X$ and $T$ are both assumed to be locally compact.
    By Lemma \ref{Lemma:Compact-Open-Subbasis},
    the collection
    \[
        \left\{ M\left( K \times K', U \right) 
     \mid U \subset Y \text{ open}, 
 K \subset X, K' \subset T \text{ both compact}\right\} 
\]
forms a subbasis for the compact-open topology on
$Y^{X \times T}$. Also, the
$M\left( K, U \right)  $ give a subbasis for
$Y^{X}$ and therefore the
$M \left( K', M\left( K,U \right)  \right) $ form
a subbasis for the topology on
$\left( Y^{X} \right)^{T}$.
Since we showed that these subbases correspond to one
another under the exponential correspondence, the
theorem is proved.
\end{proof}

\begin{proposition}[]
    If $X$ is locally compact Hausdorff and $Y$ and $W$ are
    Hausdorff, then there is the homeomorphism
    \[
    Y^{X} \times W^{X}
    \stackrel{\cong}{\to} \left( Y \times W \right)^{X}
    \] 
    given by $\left( f,g \right) \mapsto 
    f \times g$.
\end{proposition}

\begin{proof}
    It is clearly a bijection.
    If $K, K' \subset X$ are compact
    and $U \subset Y$ and
    $V \subset W$ are open, then
    \begin{align*}
        \left( f,g \right) 
        \in M \left( K,U \right) \times 
        M\left( K',V \right) 
        &\iff \left( x \in K \implies f(x) \in U \right) 
        \text{ and } \left( x \in K' \implies
        g(x) \in V \right) \\
        &\iff \left( (x,y) \in K \times K' \implies
        f \times g(x,y) \in U \times V \right) \\
        &\iff f\times g \in 
        M\left( K , U \times W \right) \cap
        M\left( K', U \times W \right) .
    \end{align*}
    so $\left( f, g \right) \mapsto f \times g$ is an
    open map.\\
    Also $\left( f,g \right) \in 
    M(K, U) \times M(K,V) \iff
    f \times g \in M \left( K, U \times V \right) $ which
    implies that the function is continuous.
\end{proof}

\begin{proposition}[]
    If $X$ and $T$ are locally compact Hausdorff spaces
    and $Y$ is an arbitrary Hausdorff space, then there
    is the homeomorphism
    \[
    Y^{X \sqcup T} \stackrel{\cong}{\to} 
    Y^{X} \times Y^{T}
    \] 
    taking $f \mapsto \left( f \circ \iota_X,
    f \circ \iota_T \right) $.
\end{proposition}

\begin{proof}
    The map is clearly well-defined and
    injective. Also, given $\left( f,g \right) 
    \in Y^{X} \times Y^{T}$, we can
    define a function $f \cup  g\colon
    X \sqcup T \to Y$ by
    $f$ on $X$ and $g$ on $T$, and
    clearly,
    $f \cup g \mapsto (f,g)$ under the correspondence, giving
    surjectivity. We must show that
    it is continuous and has continuous inverse.\\
    Let $f \colon X \sqcup T \to Y$ and
    suppose $\left( 
    f \circ \iota_X , f \circ \iota_T \right) 
    \in M\left( K, U \right) \times 
    M\left( K', V \right)$.
    Then
    $f \in 
    M \left( K, U \right) \cap
    M\left( K', V \right)$ which is an open set
    that is mapped precisely to
    $M \left( K , U \right) \times 
    M \left( K', V \right) $. Hence
    $f\mapsto \left( f \circ \iota_X, f \circ \iota_T \right) $ 
    is continuous.\\

    Conversely, note that
     under the correspondence,
     $M\left( C \sqcup C',U \right) $ is mapped
     to
     $M(C, U) \times M(C',U)$, so
     the map is also open.
\end{proof}

\begin{theorem}[]
    For $X$ locally compact and both $X$ and $Y$ Hausdorff,
    $Y^{X}$ is a covariant functor of $Y$ and
    a contravariant functor of $X$
    from $\Top$ to $\Top$.
\end{theorem}



\begin{proof}
    A map $\varphi  \colon Y \to Z$ induces
    $\varphi^{X} \colon Y^{X} \to Z^{X}$ (put differently,
    $\varphi $ induces
    $\varphi_* \colon \Hom (X,Y) \to \Hom(X,Z)$.))
    We must show that $\varphi^{X}$ is continuous.
    By Theorem \ref{Thm:Compact-Open-Top}, it suffices
    to show that the map
    $Y^{X} \times X \to Z$ given by
    $\left( f,x \right) \mapsto \varphi \left( f(x) \right) $ 
    is continuous, but this is
    the composition $\varphi \circ e$ which is 
    thus continuous.\\
    Next, for the contravariant part, we must show that
    for $\psi \colon X \to T$, both spaces locally compact, we
    have that 
    $Y^{\psi } \colon Y^{T} \to Y^{X}$ given
    by $\psi^{*} \colon f \mapsto f \circ \psi $ is continuous.
    By the same theorem as above, it suffices to show that
    $Y^{T} \times X \to Y$ taking
    $\left( f,x \right) \mapsto f \left( \psi (x) \right) $ is
    continuous, but this is
    $e \circ \left( \id \times \psi  \right) $, which is
    continuous.
\end{proof}

\begin{corollary}
    For $A \subset X$ both locally compact and $X,Y$ Hausdorff,
    the restriction $Y^{X} \to Y^{A}$ is continuous.
\end{corollary}

\begin{proof}
    Apply the contravariant functor
    $\Hom \left( -,Y \right) = 
    Y^{-}$ to the inclusion $\iota \colon A \hookrightarrow X$.
\end{proof}

\begin{theorem}[]
    For $X,Y$ locally compact, and $X,Y,Z$ Hausdorff, the
    function
    \[
    Z^{Y} \times Y^{X} \to Z^{X}
    \] 
    taking $\left( f,g \right) \mapsto f \circ g$ is continuous.
\end{theorem}

\begin{proof}
    Again, by Theorem \ref{Thm:Compact-Open-Top}, it
    suffices to show that the map
    $Z^{Y} \times Y^{X} \times X \to Z$ taking
    $\left( f,g,x \right) \mapsto \left( f \circ g \right) (x)$ 
    is continuous, but this is simply 
    $e \circ \left( \id \times e \right) $.
\end{proof}

