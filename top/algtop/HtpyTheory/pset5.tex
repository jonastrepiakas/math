\documentclass[reqno]{amsart}
\usepackage{amscd, amssymb, amsmath, amsthm}
\usepackage{graphicx}
\usepackage[colorlinks=true,linkcolor=blue]{hyperref}
\usepackage[utf8]{inputenc}
\usepackage[T1]{fontenc}
\usepackage{textcomp}
\usepackage{babel}
%% for identity function 1:
\usepackage{bbm}
%%For category theory diagrams:
\usepackage{tikz-cd}

%\usepackage[backend=biber]{biblatex}
%\addbibresource{.bib}


\setlength\parindent{0pt}

\pdfsuppresswarningpagegroup=1

\newtheorem{theorem}{Theorem}[section]
\newtheorem{lemma}[theorem]{Lemma}
\newtheorem{proposition}[theorem]{Proposition}
\newtheorem{corollary}[theorem]{Corollary}
\newtheorem{conjecture}[theorem]{Conjecture}

\theoremstyle{definition}
\newtheorem{definition}[theorem]{Definition}
\newtheorem{example}[theorem]{Example}
\newtheorem{exercise}[theorem]{Exercise}
\newtheorem{problem}[theorem]{Problem}
\newtheorem{question}[theorem]{Question}

\theoremstyle{remark}
\newtheorem*{remark}{Remark}
\newtheorem*{note}{Note}
\newtheorem*{solution}{Solution}



%Inequalities
\newcommand{\cycsum}{\sum_{\mathrm{cyc}}}
\newcommand{\symsum}{\sum_{\mathrm{sym}}}
\newcommand{\cycprod}{\prod_{\mathrm{cyc}}}
\newcommand{\symprod}{\prod_{\mathrm{sym}}}

%Linear Algebra

\DeclareMathOperator{\Span}{span}
\DeclareMathOperator{\im}{im}
\DeclareMathOperator{\diag}{diag}
\DeclareMathOperator{\Ker}{Ker}
\DeclareMathOperator{\ob}{ob}
\DeclareMathOperator{\Hom}{Hom}
\DeclareMathOperator{\Mor}{Mor}
\DeclareMathOperator{\sk}{sk}
\DeclareMathOperator{\Vect}{Vect}
\DeclareMathOperator{\Set}{Set}
\DeclareMathOperator{\Group}{Group}
\DeclareMathOperator{\Ring}{Ring}
\DeclareMathOperator{\Ab}{Ab}
\DeclareMathOperator{\Top}{Top}
\DeclareMathOperator{\hTop}{hTop}
\DeclareMathOperator{\Htpy}{Htpy}
\DeclareMathOperator{\Cat}{Cat}
\DeclareMathOperator{\CAT}{CAT}
\DeclareMathOperator{\Cone}{Cone}
\DeclareMathOperator{\dom}{dom}
\DeclareMathOperator{\cod}{cod}
\DeclareMathOperator{\Aut}{Aut}
\DeclareMathOperator{\Mat}{Mat}
\DeclareMathOperator{\Fin}{Fin}
\DeclareMathOperator{\rel}{rel}
\DeclareMathOperator{\Int}{Int}
\DeclareMathOperator{\sgn}{sgn}
\DeclareMathOperator{\Homeo}{Homeo}
\DeclareMathOperator{\SHomeo}{SHomeo}
\DeclareMathOperator{\PSL}{PSL}
\DeclareMathOperator{\Bil}{Bil}
\DeclareMathOperator{\Sym}{Sym}
\DeclareMathOperator{\Skew}{Skew}
\DeclareMathOperator{\Alt}{Alt}
\DeclareMathOperator{\Quad}{Quad}
\DeclareMathOperator{\Sin}{Sin}
\DeclareMathOperator{\Supp}{Supp}
\DeclareMathOperator{\Char}{char}
\DeclareMathOperator{\Teich}{Teich}
\DeclareMathOperator{\GL}{GL}
\DeclareMathOperator{\tr}{tr}
\DeclareMathOperator{\codim}{codim}
\DeclareMathOperator{\coker}{coker}
\DeclareMathOperator{\corank}{corank}
\DeclareMathOperator{\rank}{rank}
\DeclareMathOperator{\Diff}{Diff}
\DeclareMathOperator{\Bun}{Bun}
\DeclareMathOperator{\Sm}{Sm}
\DeclareMathOperator{\Fr}{Fr}
\DeclareMathOperator{\Cob}{Cob}
\DeclareMathOperator{\Ext}{Ext}
\DeclareMathOperator{\Tor}{Tor}
\DeclareMathOperator{\Conf}{Conf}
\DeclareMathOperator{\UConf}{UConf}
\DeclareMathOperator{\Map}{Map}




%Row operations
\newcommand{\elem}[1]{% elementary operations
\xrightarrow{\substack{#1}}%
}

\newcommand{\lelem}[1]{% elementary operations (left alignment)
\xrightarrow{\begin{subarray}{l}#1\end{subarray}}%
}

%SS
\DeclareMathOperator{\supp}{supp}
\DeclareMathOperator{\Var}{Var}

%NT
\DeclareMathOperator{\ord}{ord}

%Alg
\DeclareMathOperator{\Rad}{Rad}
\DeclareMathOperator{\Jac}{Jac}

%Misc
\newcommand{\SL}{{\mathrm{SL}}}
\newcommand{\mobgp}{{\mathrm{PSL}_2(\mathbb{C})}}
\newcommand{\id}{{\mathrm{id}}}
\newcommand{\MCG}{{\mathrm{MCG}}}
\newcommand{\PMCG}{{\mathrm{PMCG}}}
\newcommand{\SMCG}{{\mathrm{SMCG}}}
\newcommand{\ud}{{\mathrm{d}}}
\newcommand{\Vol}{{\mathrm{Vol}}}
\newcommand{\Area}{{\mathrm{Area}}}
\newcommand{\diam}{{\mathrm{diam}}}
\newcommand{\End}{{\mathrm{End}}}


\newcommand{\reg}{{\mathtt{reg}}}
\newcommand{\geo}{{\mathtt{geo}}}

\newcommand{\tori}{{\mathcal{T}}}
\newcommand{\cpn}{{\mathtt{c}}}
\newcommand{\pat}{{\mathtt{p}}}

\let\Cap\undefined
\newcommand{\Cap}{{\mathcal{C}}ap}
\newcommand{\Push}{{\mathcal{P}}ush}
\newcommand{\Forget}{{\mathcal{F}}orget}



\title{Assignment 5}
\author{Jonas Trepiakas}
\date{}


\begin{document}
\maketitle
    \begin{problem}[]
        Let $F := \text{hofib} \left( i \colon
        \mathbb{C}\mathbb{P}^{n} \to \mathbb{C}\mathbb{P}^{\infty}
    \right) $ be the homotopy fiber of the inclusion.
    \begin{enumerate}
        \item Show that $F$ is $2n$-conected.\\
            There is a LES coming from the Puppe
            sequence
            \[
            \ldots \to 
            \left[ X, \Omega \mathbb{C}\mathbb{P}^{\infty}
            \right]_* \to \left[ X,F \right]_* 
            \to \left[ X,\mathbb{C}\mathbb{P}^{n} \right]_*
            \to \left[ X,\mathbb{C}\mathbb{P}^{\infty} \right]_*
            \] 
            Let $X $ be a CW complex of dimension $\le 2n$.
        \item Show that if $f \colon
            X \to \mathbb{C}\mathbb{P}^{n}$ induces
            the $0$-map on $H^{2}$, then
            $f$ is nullhomotopic.
        \item Show that there is a counter-example when
            $X$ is allowed to be
            $(2n+1)$-dimensional.
    \end{enumerate}
    \end{problem}

\begin{solution}
        (1) The LES for
        the pair
        $\left( \mathbb{C}\mathbb{P}^{\infty},
        \mathbb{C}\mathbb{P}^{n}\right) $ and
        the fibration $E_f \to B$ can be identified, so
        in particular,
        $\pi_k \left( F, \gamma_0 \right) 
        \cong \pi_{k+1} \left( \mathbb{C}\mathbb{P}^{\infty},
        \mathbb{C}\mathbb{P}^{n}\right)$.\\

        Recall now (p. 7, Hatcher) that
        $\mathbb{C}\mathbb{P}^{n}$ has a cell
        structure of a single cell in every even
        degree up to $2n$ :
        $\mathbb{C}\mathbb{P}^{n}
        = e^{0} \cup e^{2} \cup \ldots \cup 
        e^{2n}$.
        By definition, 
        $\mathbb{C}\mathbb{P}^{\infty}$ is defined
        as the sequential colimit over the sequence
        $\mathbb{C}\mathbb{P}^{n} \hookrightarrow 
        \mathbb{C}\mathbb{P}^{n+1}$:
        $\mathbb{C}\mathbb{P}^{\infty} :=
        \lim_{\leftarrow_n} \mathbb{C}\mathbb{P}^{n}$.
        In paritcular (Hatcher, p.7), it also
        has a cell structure with one cell in each even
        dimension with $\mathbb{C}\mathbb{P}^{n}$ 
        constituting its
        $2n$-skeleton.
        From Corollary 4.12 in Hatcher, it now follows
        that $(\mathbb{C}\mathbb{P}^{\infty},
        \mathbb{C}\mathbb{P}^{n})$ is $(2n+1)$-connected,
        so
        $F$ is $2n$-connected.\\
        \linebreak
        





        

    (2) 





      Firstly, $F$ is $2n$-connected, so we
      can replace it using Proposition 4.15 in Hatcher
      by a CW complex with a single
      $0$-cell and cells of dimension $>2n$.
      Then by Cellular Approximation,
      $\left[ X,F \right]_* = 0$, so
      $\left[ X,\mathbb{C}\mathbb{P}^{n} \right]_*$ 
      injects into
      $\left[ X, \mathbb{C}\mathbb{P}^{\infty} \right]_*$.
      What is this injection? 

      Recall that the Puppe sequence
      given is obtained by applying
      $\left[ X,- \right]_* $ to the Puppe sequence
      from p. 409 in Hatcher:
      \[
      \ldots \to \Omega^2 \mathbb{C}\mathbb{P}^{\infty}
      \to \Omega F \to \Omega \mathbb{C}\mathbb{P}^{n}
      \to \Omega \mathbb{C}\mathbb{P}^{\infty}
      \to F \to \mathbb{C}\mathbb{P}^{n} 
      \stackrel{i}{\to} \mathbb{C}\mathbb{P}^{n}
      \] 
      so the map
      $\left[ X,\mathbb{C}\mathbb{P}^{n} \right]_*
      \to \left[ X, \mathbb{C}\mathbb{P}^{\infty} \right]_* $ 
      is given by postcomposing with the
      inclusion $i$, and we are told that this is injective.
      In particular,
      $\left[ f \right] $ is zero if and only if
      $\left[ i \circ f \right] \in 
      \left[ X, \mathbb{C}\mathbb{P}^{\infty} \right]_*$ is
      zero.

      But now since $X$ is a CW complex, we have
    by Theorem 4.57 in Hatcher that
    $H^k(X;\mathbb{Z})
    \cong \left[ X, K(\mathbb{Z},k) \right] $, so
     since $\mathbb{C}\mathbb{P}^{\infty}$ is
     $K\left( \mathbb{Z},2 \right) $, we have
     $H^2 (X;\mathbb{Z})
     \cong \left[ X, \mathbb{C}\mathbb{P}^{\infty} \right] $.
     Now, by theorem 4.57, the correspondence
     $\left[ X, \mathbb{C}\mathbb{P}^{\infty} \right]_* 
     \stackrel{\cong}{\to} H^{2}(X;\mathbb{Z})$ takes
     $\left[ g \right]$ to
     $g^{*}(\alpha)$ for some fundamental class
     $\alpha \in 
     H^{2}\left( \mathbb{C}\mathbb{P}^{\infty};\mathbb{Z} \right) $.
     But
     $\left( i \circ f \right)^{*}(\alpha)
     = f^{*} \left( i^{*}(\alpha) \right) = 0$ by assumption
     on $f$. But the $0$ map
     in $\left[ X, \mathbb{C}\mathbb{P}^{n} \right]_* $ 
     sending all of $X$ to the basepoint clearly also induces
     the $0$-map on $H^2$, so by the same reasoning, 
     it is mapped to $0$ in
     $H^2 \left( X; \mathbb{Z} \right) $ under this correspondence.
     Hence
     $\left[ i \circ f \right] 
     = \left[ i \circ 0 \right] $ by Theorem 4.57,
     and so by the injectivity from the Puppe sequence,
     $\left[ f \right] = \left[ 0 \right] $ in
     $\left[ X, \mathbb{C}\mathbb{P}^{n} \right]_*$,
     so $f$ is based nullhomotopic, which was what we wanted
     to show.\\
     \linebreak
     (3) 
     Consider the Hopf fibration
     $S^{1} \to S^{2n+1} \stackrel{f}{\to}  \mathbb{C}\mathbb{P}^{n}$.
     Since $f$ induces a nontrivial isomorphism
     $\pi_k \left( S^{2n+1} \right) 
     \stackrel{f_*}{\to} \pi_k \left( \mathbb{C}\mathbb{P}^{n}
     \right) $ for $k>2$,
     $f$ cannot be a nullhomotopic in
     $\left[ S^{2n+1},\mathbb{C}\mathbb{P}^{n} \right]_*$,
     however, it does induces the $0$ map on
     $H^2$ since $H^2 \left( S^{2n+1} \right) = 0$.




\end{solution}




\begin{problem}[]
    Let $Y$ be a locally compact CW complex.
    Use $\left[ Y; \Omega Z \right] \cong
    \pi_1 \Map (Y,Z)$ to define a group structure
    on $\left[ Y, \Omega Z \right] $ and show that the part
    \[
    \ldots \to \left[ Y, \Omega F \right]_*
    \to \left[ Y, \Omega E \right]_* \to 
    \left[ Y, \Omega B \right]_*
    \] 
    of the Puppe sequence can be made into a LES of
    groups. Using this and the previous exercise,
    show that  $\left[ X, \mathbb{C}\mathbb{P}^{n} \right] 
    \cong H^2 (X;\mathbb{Z})$ for
    $X = \Sigma Y$ and $Y$ an
    $(2n-1)$-dimensional CW complex.
\end{problem}

\begin{proof}
    I will assume that the formulation of the problem
    was meant for based spaces and based maps.
    In that case, given maps
    $\left[ f \right] ,\left[ g \right] \in 
    \left[ Y, \Omega Z \right] $, we can define
    $\left[ f \right] + \left[ g \right] $ as the
    $\varphi^{-1} \left( \varphi \left[ f \right] 
    \cdot  \varphi \left[ g \right] \right) $ where
    $\varphi $ is the isomorphism
    $\varphi \colon \left[ Y; \Omega Z \right] 
    \cong \pi_1 \Map(Y,Z)$, and
    $\cdot $ denotes concatenation of loops.
    This makes the sets into groups as they inherit
    the group structure from
    $\pi_1 \Map \left( Y,Z \right) $.
    In general,
    $\left[ A,B \right] $ is group with multiplication
    $(f\cdot g)(x) = f(x) \cdot g(x)$ whenever
    $B$ is an H-group which loop spaces are (Bredon, Theorem 3.3,
    VII.(3)).

    Then the sequence
    \[
    \left[ Y, \Omega F \right]_* \to 
    \left[ Y, \Omega E \right]_* \to 
    \left[ Y, \Omega B \right]_*
    \] 
    can be made into a LES of groups if we can show that
    the maps between these groups become group homomorphisms
    (exactness is already given).
    For this, note that the
    Puppe sequence in those degrees was already a LES
    of groups, and then
    taking $\left[ Y, - \right]_*$ of this sequence
    just becomes postcomposition which distributes
    over the loop operation since it is simply concatenation,
    so the postcomposed function just distributes over
    the concatenation on $\left[ 0,\frac{1}{2} \right] $ and
     $\left[ \frac{1}{2},1 \right] $.\\



    Now similarly to the previous problem, we then get
    $\left[ X, \mathbb{C}\mathbb{P}^{n} \right] 
    \cong \left[ \Sigma Y, \mathbb{C}\mathbb{P}^{n} \right] 
    \cong \left[ Y, \Omega \mathbb{C}\mathbb{P}^{n} \right] $,
    and also
    $\left[ X, \mathbb{C}\mathbb{P}^{\infty} \right] 
    \cong H^2 (X;\mathbb{Z})$, so
    if we can show that
    $\left[ Y,F \right]_* = 0$ and
    $\left[ \Sigma Y, F \right]_* = 0$, then
    we are done, as then
    $\left[ X, \mathbb{C}\mathbb{P}^{n} \right] 
    \cong \left[ X, \mathbb{C}\mathbb{P}^{\infty} \right] 
    \cong H^2 (X;\mathbb{Z})$ by the LES from problem 1 (now
    as groups).

    And $\left[ Y,F \right]_* = 0 =
    \left[ \Sigma Y,F \right]_* = 0$ follows since
    $Y$ is $(2n-1)$-dimensional hence
    $\Sigma Y$ is $(2n)$-dimensional, so by Cellular Approximation,
    and replacing $F$ by a CW-complex with cells
    only of dimension greater than $2n$ along with the
    basepoint, we obtain that
    any map in
    $ \left[ Y,F \right]_*$ or
    $\left[ \Sigma Y,F \right]_*$ is based nullhomotopic, giving
    the result.



\end{proof}















    %\printbibliography
\end{document}
