\documentclass[reqno]{amsart}
\usepackage{amscd, amssymb, amsmath, amsthm}
\usepackage{graphicx}
\usepackage[colorlinks=true,linkcolor=blue]{hyperref}
\usepackage[utf8]{inputenc}
\usepackage[T1]{fontenc}
\usepackage{textcomp}
\usepackage{babel}
%% for identity function 1:
\usepackage{bbm}
%%For category theory diagrams:
\usepackage{tikz-cd}

%\usepackage[backend=biber]{biblatex}
%\addbibresource{.bib}


\setlength\parindent{0pt}

\pdfsuppresswarningpagegroup=1

\newtheorem{theorem}{Theorem}[section]
\newtheorem{lemma}[theorem]{Lemma}
\newtheorem{proposition}[theorem]{Proposition}
\newtheorem{corollary}[theorem]{Corollary}
\newtheorem{conjecture}[theorem]{Conjecture}

\theoremstyle{definition}
\newtheorem{definition}[theorem]{Definition}
\newtheorem{example}[theorem]{Example}
\newtheorem{exercise}[theorem]{Exercise}
\newtheorem{problem}[theorem]{Problem}
\newtheorem{question}[theorem]{Question}

\theoremstyle{remark}
\newtheorem*{remark}{Remark}
\newtheorem*{note}{Note}
\newtheorem*{solution}{Solution}



%Inequalities
\newcommand{\cycsum}{\sum_{\mathrm{cyc}}}
\newcommand{\symsum}{\sum_{\mathrm{sym}}}
\newcommand{\cycprod}{\prod_{\mathrm{cyc}}}
\newcommand{\symprod}{\prod_{\mathrm{sym}}}

%Linear Algebra

\DeclareMathOperator{\Span}{span}
\DeclareMathOperator{\im}{im}
\DeclareMathOperator{\diag}{diag}
\DeclareMathOperator{\Ker}{Ker}
\DeclareMathOperator{\ob}{ob}
\DeclareMathOperator{\Hom}{Hom}
\DeclareMathOperator{\Mor}{Mor}
\DeclareMathOperator{\sk}{sk}
\DeclareMathOperator{\Vect}{Vect}
\DeclareMathOperator{\Set}{Set}
\DeclareMathOperator{\Group}{Group}
\DeclareMathOperator{\Ring}{Ring}
\DeclareMathOperator{\Ab}{Ab}
\DeclareMathOperator{\Top}{Top}
\DeclareMathOperator{\hTop}{hTop}
\DeclareMathOperator{\Htpy}{Htpy}
\DeclareMathOperator{\Cat}{Cat}
\DeclareMathOperator{\CAT}{CAT}
\DeclareMathOperator{\Cone}{Cone}
\DeclareMathOperator{\dom}{dom}
\DeclareMathOperator{\cod}{cod}
\DeclareMathOperator{\Aut}{Aut}
\DeclareMathOperator{\Mat}{Mat}
\DeclareMathOperator{\Fin}{Fin}
\DeclareMathOperator{\rel}{rel}
\DeclareMathOperator{\Int}{Int}
\DeclareMathOperator{\sgn}{sgn}
\DeclareMathOperator{\Homeo}{Homeo}
\DeclareMathOperator{\SHomeo}{SHomeo}
\DeclareMathOperator{\PSL}{PSL}
\DeclareMathOperator{\Bil}{Bil}
\DeclareMathOperator{\Sym}{Sym}
\DeclareMathOperator{\Skew}{Skew}
\DeclareMathOperator{\Alt}{Alt}
\DeclareMathOperator{\Quad}{Quad}
\DeclareMathOperator{\Sin}{Sin}
\DeclareMathOperator{\Supp}{Supp}
\DeclareMathOperator{\Char}{char}
\DeclareMathOperator{\Teich}{Teich}
\DeclareMathOperator{\GL}{GL}
\DeclareMathOperator{\tr}{tr}
\DeclareMathOperator{\codim}{codim}
\DeclareMathOperator{\coker}{coker}
\DeclareMathOperator{\corank}{corank}
\DeclareMathOperator{\rank}{rank}
\DeclareMathOperator{\Diff}{Diff}
\DeclareMathOperator{\Bun}{Bun}
\DeclareMathOperator{\Sm}{Sm}
\DeclareMathOperator{\Fr}{Fr}
\DeclareMathOperator{\Cob}{Cob}
\DeclareMathOperator{\Ext}{Ext}
\DeclareMathOperator{\Tor}{Tor}
\DeclareMathOperator{\Conf}{Conf}
\DeclareMathOperator{\UConf}{UConf}
\DeclareMathOperator{\Ch}{Ch}
\DeclareMathOperator{\Tot}{Tot}



%Row operations
\newcommand{\elem}[1]{% elementary operations
\xrightarrow{\substack{#1}}%
}

\newcommand{\lelem}[1]{% elementary operations (left alignment)
\xrightarrow{\begin{subarray}{l}#1\end{subarray}}%
}

%SS
\DeclareMathOperator{\supp}{supp}
\DeclareMathOperator{\Var}{Var}

%NT
\DeclareMathOperator{\ord}{ord}

%Alg
\DeclareMathOperator{\Rad}{Rad}
\DeclareMathOperator{\Jac}{Jac}

%Misc
\newcommand{\SL}{{\mathrm{SL}}}
\newcommand{\mobgp}{{\mathrm{PSL}_2(\mathbb{C})}}
\newcommand{\id}{{\mathrm{id}}}
\newcommand{\MCG}{{\mathrm{MCG}}}
\newcommand{\PMCG}{{\mathrm{PMCG}}}
\newcommand{\SMCG}{{\mathrm{SMCG}}}
\newcommand{\ud}{{\mathrm{d}}}
\newcommand{\Vol}{{\mathrm{Vol}}}
\newcommand{\Area}{{\mathrm{Area}}}
\newcommand{\diam}{{\mathrm{diam}}}
\newcommand{\End}{{\mathrm{End}}}


\newcommand{\reg}{{\mathtt{reg}}}
\newcommand{\geo}{{\mathtt{geo}}}

\newcommand{\tori}{{\mathcal{T}}}
\newcommand{\cpn}{{\mathtt{c}}}
\newcommand{\pat}{{\mathtt{p}}}

\let\Cap\undefined
\newcommand{\Cap}{{\mathcal{C}}ap}
\newcommand{\Push}{{\mathcal{P}}ush}
\newcommand{\Forget}{{\mathcal{F}}orget}




\begin{document}

\section{Hatcher}








\subsection{Exact Couples}

\begin{definition}[Exact Couple]
    An \textit{exact couple} is an exact sequence of
    abelian groups of the form
    \begin{equation*}
    \begin{tikzcd}
        A \ar[rr, "i"] & & A \ar[dl, "j"] \\
                       & B \ar[ul, "k"] &
    \end{tikzcd}
    \end{equation*}
    where $i,j$ and $k$ are group homomorphisms. Define
    $d \colon B \to B$ by $d = j \circ k$.
    Then $d^2 = j (kj)k = 0$, so
    $H(B) := \ker d / \im d$ is defined - in particular,
    since $A$ and $B$ are abelian, the quotient
    $H(B)$ is well-defined and a group.
\end{definition}

\begin{definition}[Derived Couple]
    Out of a given exact couple, we can construct
    a new exact couple, called the \textit{derived couple}:

    \begin{equation*}
    \begin{tikzcd}
        A' \ar[rr, "i'"] & & A' \ar[dl, "j'"] \\
                       & B' \ar[ul, "k'"] &
    \end{tikzcd}
    \end{equation*}
    where we define
    \begin{enumerate}
        \item $A' = i(A)$ and $B' = H(B)$.
        \item $i'$ is the induced
            map $i' := i|_{A'} \colon
            A'\to A'$ by $i' (ia) = i(ia)$ 
        \item We define $j'$ by
            $j' a' = \left[ ja \right] $ where
            $a' = ia$ for some $a$ in $A$.
        \item $k'$ is defined by
            $k' \left[ b \right] = kb \in i(A)$.
    \end{enumerate}
    With these definitions, the derived couple is an exact
    couple.
\end{definition}

\begin{exercise}[]
    Check that the maps are well-defined and that the
    derived sequence is exact.
\end{exercise}

\begin{proof}
    We must check that $j'$ and $k'$ are well-defined maps.

    Suppose $a' = ia = i \tilde{a}$.
    Then $a-\tilde{a} \in \ker i = \im k$ so
    $a - \tilde{a} = k\left[ b \right] $. Hence
    Then
    $j a - j \tilde{a} =
    j k \left[ b \right] 
    = d \left[ b \right] \in \im d$, so
    $\left[ j a \right] =
    \left[ j \tilde{a} \right] $.\\
    Next, suppose
    $\left[ b \right] = \left[ \tilde{b} \right] $, so
    $b - \tilde{b}\in 
    \im d$, i.e., $b - \tilde{b} = 
    jk (\overline{b})$.
    Then
    $k b - k \tilde{b} = 
    kjk (\overline{b}) = 
    0$, so
    $k' \left[ b \right] =
    k'\left[ \tilde{b} \right] $.\\
    \linebreak
    Lastly, exactness at $B'$:
     suppose $k' \left[ b \right] = 0$. Then
     $kb = 0$, so by exactness of the original exact
     couple, there exists some $a \in A$ such that
     $j (a) = b$. Then
     let $a' = i(a)$, so
     $j' (a') = \left[ j (a) \right] =
     \left[ b \right] $, hence
     $\ker k' \subset \im j'$.\\
     Conversely, 
     $k' j' (a') =
     k' \left[ ja \right] =
     kja = 0$, by exactness at
     $B$ of the original couple.
\end{proof}

\begin{definition}[Spectral Sequence]
    A \textit{spectral sequence} $\left( E_{*,*},d \right) $ 
    (in homological Serre grading), starting on page
    $r_0 \ge 1$, consists of:
    \begin{enumerate}
        \item a bigraded group
            $\left( E_{p,q}^{r} \right)_{p,q \in \mathbb{Z}}$ 
            for each $r \ge r_0$, called the
            \textit{$r$ th page} of the spectral sequence.
        \item For all $r \ge r_0$ and $p, q \in \mathbb{Z}$ 
            a map of abelian groups
            \[
            d_{p,q}^{r} \colon E_{p,q}^{r} \to 
            E_{p-r,q+r-1}^{r}
            \] 
            called the\textit{$r$ th differential}
            which squares to zero in the
            sense that
             \[
             d_{p-r,q+r-1}^{r} \circ
             d_{p,q}^{r} = 0
             \] 
             holds for all $p,q,r$.
         \item For all $r \ge r_0$ and
             $p,q \in \mathbb{Z}$, isomorphisms of abelian groups
             \[
             E_{p,q}^{r+1} \cong
             \frac{\ker \left( d_{p,q}^{r} \colon
             E_{p,q}^{r} \to E_{p-r,q+r-1}^{r}\right) }{
         \im \left( d_{p+q,q-r+1}^{r} \colon
     E_{p+r,q-r+1}^{r} \to E_{p,q}^{r}\right) }
             \] 
    \end{enumerate}
\end{definition}

\begin{definition}[]
    We say that a spectral sequence
    $\left( E_{*,*},d \right) $ \textit{converges} to
    a graded abelian group
    $H_*$ and write
    \[
    E_{p,q}^2 \implies H_{p+q}
    \] 
    if there is a filtration
    \[
    0 \subset F_n^{0} \subset F_n^{1} \subset \ldots
    \subset F_{n}^{n-1} \subset F_{n}^{n} = H_n
    \] 
    and isomorphisms $E_{p,q}^{\infty} \cong
    F_{p+q}^{p} / F_{p+q}^{p-q}$.
\end{definition}

Note that if
$\left( E_{*,*},d \right) $ is a first quadrant
spectral sequence, then
$E_{p,q}^{r+1} \cong
E_{p,q}^{r}$ for
$r > \max \left\{ p,q+1 \right\} $ because
$d_{p,q}^{r}$ maps to the $0$ group
and $d_{p+r,q-r+1}$ comes from the $0$ group.

\begin{definition}[$E^{\infty}$-page]
    For a first quadrant spectral sequence, we
    define the \textit{$E^{\infty}$-page} as:
    \[
    E_{p,q}^{\infty} :=
    E_{p,q}^{r}, \quad \text{for } r \gg p,q
    \] 
\end{definition}

\begin{lemma}[]
    If
    \[
    1 \to A \to B \to C \to 1
    \] 
    is a SES of groups, then $B = A \rtimes C$.
\end{lemma}

\begin{theorem}[Leray-Serre spectral sequence]\label{LSSS}
    For every abelian group $G$ and every
    fiber sequence
    \[
    F \to E \to B
    \] 
    such that $\pi_1 (B)$ acts trivially on
    $H_* (F;G)$, there is a natural, convergent
    \textit{Leray-Serre spectral sequence}
    of signature
    \[
    E_{p,q}^2 = H_p \left( B; H_q(F;G) \right) \implies
    H_{p+q}(E;G)
    \] 
    meaning that the $E_{p,q}^2$ page is given
    by $E_{p,q}^2 = H_p \left( B; H_p (F;G) \right) $ and
    there is a natural filtration
    \[
    0 = F_{-1}^{n} \subset F_n^{0} \subset 
    \ldots \subset F_{n}^{n} = H_n(E;G)
    \] 
    and natural SES:
    \[
    0 \to F_{p-1}^{p+q} \hookrightarrow 
    F_{p}^{p+q} \twoheadrightarrow E_{p,q}^{\infty} \to 0
    \] 
\end{theorem}

\begin{note}
    The SES in Theorem \ref{LSSS} splits as
    \begin{align*}
    H_n(E;G) = F_{n}^{n} 
    &\cong F_{n-1}^{n} \rtimes E_{n,0}^{\infty}\\
    &\cong F_{n-2}^{n} \rtimes E_{n-1,1}^{\infty} \rtimes
    E_{n,0}^{\infty}\\
    &\vdots \\
    &\cong F_0^{n} \rtimes E_{1,n-1}^{\infty} \rtimes \ldots
    \rtimes E_{n,0}^{\infty}\\
    &\cong E_{0,n}^{\infty} \rtimes E_{1,n-1}^{\infty} \rtimes \ldots
    \rtimes E_{n,0}^{\infty}
    \end{align*}

\end{note}


\begin{example}[]
    Suppose
    \[
    F \to E \to B
    \] 
    is a fiber sequence and that
    $H_n(E;G) = 0$ for an abelian group $G$.
    Then $E_{p, n -p}^{\infty} = 0$ for all
    $0 \le p \le n$.\\
    This can be seen because
    $F_{n}^{n} = H_n(E;G) = 0$, and
    $0 \subset F_n^{0} \subset \ldots \subset 
    F_n^{n} = 0$, hence
    $E_{p, n-p}^{\infty} \cong
    F_n^{p} / F_n^{p-1} \cong 0$.
\end{example}
    

\begin{example}[]
    Suppose that the $E_{p,q}^{\infty}$ are abelian groups.
    Then the semidirect products reduce to normal direct
    products, so that
    \[
    H_n(E;G) \cong
    \bigoplus_{p=0}^{n} E_{p,n-p}^{\infty}
    \] 
    For example, if $G$ is a field, then 
    $H_n(E;G)$ is a $G$-vector space, hence abelian, so
    each $F_n^{p}$ being subgroups of $H_n(E;G)$ is abelian,
    so each $E_{p,q}^{\infty} \cong
    F_{n}^{p} / F_{n}^{p-q}$ is abelian.
\end{example}


\subsection{Serre Classes}

Let
$\mathcal{C}$ be one of the following classes of abelian groups:
\begin{enumerate}
    \item $\mathcal{F} \mathcal{G}$, finitely generated
        abelian groups.
    \item $\mathcal{T}_p$, torsion abelian groups whose
        elements have orders divisible only by
        primes from a fixed set $P$ of primes.
    \item $\mathcal{F}_p$, the finite groups
        in $\mathcal{T}_p$.
\end{enumerate}


\begin{note}
    $P$ could be all primes and then $\mathcal{T}_p$ would
    be all torsion abelian groups and
    $\mathcal{F}_p$ would be all finite abelian groups.
\end{note}

\begin{theorem}[]\label{Thm:SIAO9}
    If $X$ is simply-connected, then $\pi_n (X) \in 
    \mathcal{C}$ for all $n$ if and only if
    $H_n(X;\mathbb{Z}) \in \mathcal{C}$ for all
    $n > 0$. This holds also if
    $X$ is path-connected and abelian, that is, the
    action of $\pi_1 (X)$ on $\pi_n(X)$ is trivial
    for all $n\ge 1$.
\end{theorem}


\begin{theorem}[Hurewicz modulo $\mathcal{C}$]\label{Thm:Hurewicz-Mod-C}
    If a path-connected abelian space
    $X$ has $\pi_i (X) \in \mathcal{C}$ for $i < n$, then
    the Hurewicz homomorphism $h \colon
    \pi_n (X) \to H_n(X)$ is an isomorphism
    $\mod \mathcal{C}$, meaning
    that the kernel and cokernel of $h$ belong
    to $\mathcal{C}$.
\end{theorem}

In order to prove this, we need a lemma:

\begin{lemma}[]\label{Lemma:SIDOA1}
    Let $F \to X\to B$ be a fibration of path-connected
    spaces, with $\pi_1 (B)$ acting trivially
    on $H_*(F)$. Then if two of $F,X$ and $B$ have
    $H_n \in \mathcal{C}$ for all $n>0$, so does
    the third.
\end{lemma}

\begin{proof}
    We shall show the following:
    \begin{enumerate}
        \item For a SES of abelian groups $0 \to A\to B\to C\to 0$,
            the group $B$ is in $\mathcal{C}$ if and only
            if $A$ and $C$ are in $\mathcal{C}$.
        \item If $A$ and $B$ are in $\mathcal{C}$, then
            $A \otimes B$ and $\Tor (A,B)$ are in
            $\mathcal{C}$.
    \end{enumerate}

    \textit{Case 1:} Suppose
    $H_n(F), H_n(B) \in \mathcal{C}$ for all
    $n>0$. In the Serre spectral sequence, we have
    \[
    E_{p,q}^2 = H_p \left( B; H_q(F) \right) 
    \cong H_p (B) \otimes H_q(F) \bigoplus
    \Tor \left( H_{p-1}(B),H_q(F) \right) \in 
    \mathcal{C}
    \] 
    for $(p,q) \neq (0,0)$.
    Here we use property (2) twice - once
    for
    $H_p (B) \otimes H_q(F) \in \mathcal{C}$ and
    once for
    $\Tor \left( H_{p-1}(B), H_q(F) \right) \in \mathcal{C}$.\\
    
    We proceed by induction now - having
    shown the base case $r = 2$. Suppose
    $E_{p,q}^{r} \in \mathcal{C}$ for $(p,q) \neq (0,0)$.
    Then both $\ker d_r$ and $\im d_r$ are in
    $\mathcal{C}$ as can easily be checked in each case.
    Hence the quotient $E_{p,q}^{r+1}$ is also in $\mathcal{C}$ 
    as can also be checked in each case.
    Thus also $E_{p,q}^{\infty} \in \mathcal{C}$ 
    for $(p,q) \neq (0,0)$.

    Now, the groups $E_{p,n-p}^{\infty}$ are quotients
    in the filtration
    $0 \subset F_0 H_n(X) \subset 
    \ldots \subset F_n H_n(X) = H_n(X)$, so by
    induction on $p$, the subgroups
    $F_p H_n(X)$ are in $\mathcal{C}$ for $n>0$, so
    in particular, $H_n(X) \in \mathcal{C}$.
    Here the induction starts
    with $F_0 H_n (X) \in \mathcal{C}$ since
    $F_0 H_n(X) \cong E_{0,n}^{\infty} \in \mathcal{C}$.\\
    \linebreak
    \textit{Case 2:} Suppose
    $H_n(F), H_n(X) \in \mathcal{C}$ for
    all $n>0$. Since
    $H_n(X) \in \mathcal{C}$, all
    the subgroups filtering $H_n(X)$ also lie in
    $\mathcal{C}$, hence also their
    quotients
    $E_{p,n-p}^{\infty} \in \mathcal{C}$.
    Now,
    $H_1(B) = 
    E_{1,0}^2 \cong E_{1,0}^{\infty} \in \mathcal{C}$ since
    any differentials out of and into $E_{1,0}^2$ must vanish.
    So suppose now inductively that
    $H_p (B) \in \mathcal{C}$ for $0<p<k$.

    But now
    \[
    E_{p,q}^2 = 
    H_{p}\left( B, H_q(F) \right) 
    \cong
    H_p(B) \otimes H_q(F) \bigoplus
    \Tor \left( H_{p-1}(B), H_q(F) \right) 
    \] 
    Note that in this application, we use the Künneth theorem
    \[
        H_n(C_* \otimes D_*) \cong
        \bigoplus_{i+j=n} H_i(C_*) \otimes
        H_j(D_*) \oplus
        \bigoplus_{i+j=n-1} \Tor (H_i(C_*),
        H_j(D_*))
    \] 
    where, in our case,
    $C_*$ is the usual singular chain complex for
    $B$ and $D_*$ in this case is the chain complex
    consisting of a single
    nontrivial element in degree $p$ where it is
    $H_q(F)$.
    Thus
    $E_{p,q}^2 \in \mathcal{C}$ for
    $p< k, (p,q)\neq (0,0)$. Since
    $\ker$ and $\im$ are subgroups, they inherit
    this property also as well as their quotient,
    so
    $E_{p,q}^{r} \in \mathcal{C}$ for
    $p<k, (p,q)\neq (0,0)$.

    Next, since
    $E_{k,0}^{r+1} = \ker d_r \subset E_{k,0}^{r}$, we have
    a SES
    \[
    0 \to E_{k,0}^{r+1} \to E_{k,0}^{r} \to 
    \im d_r \to 0
    \] 
    with
    $\im d_r \subset E_{k-r,r-1}^{r}$, and so
    $\im d_r \in \mathcal{C}$ by induction since
    $E_{k-r,r-1}^{r} \in \mathcal{C}$. Then property
    (1) says that
    $E_{k,0}^{r+1} \in \mathcal{C}$ if and only if
    $E_{k,0}^{r} \in \mathcal{C}$.
    By downward induction on $r$, we obtain
    $E_{k,0}^{r} \in \mathcal{C}$ if and only if
    $E_{k,0}^2 = H_k(B) \in \mathcal{C}$.
    But $E_{k,0}^{\infty} \in \mathcal{C}$, so
    we conclude that
     $H_k(B) \in \mathcal{C}$ for all $k$.\\
     \linebreak
     \textit{Case 3:} Suppose
     $H_n (B) , H_n(X) \in \mathcal{C}$ for all
     $n>0$. This is similar to Case 2, so
     we will omit this.

\end{proof}


\begin{lemma}[]\label{Lemma:SIDOA2}
    If $\pi \in \mathcal{C}$, then $H_k\left( K
    \left( \pi,n \right) \right) \in \mathcal{C}$ for all
    $k,n>0$.
\end{lemma}

\begin{proof}
    Using the path fibration
    $K\left( \pi,n-1 \right) \to P\to 
    K\left( \pi, n \right) $ and the previous lemma, it
    suffices to show the case $n=1$. 
    For the classes $\mathcal{F} \mathcal{G}$ and
    $\mathcal{F}_P$, the group $\pi$ is a product
    of cyclic groups in $\mathcal{C}$, and hence
    $K\left( G_1, 1 \right) \times 
    K\left( G_2,1 \right) $ is
     a $K\left( G_1 \times G_2,1 \right) $, so by the
     Künneth formula, it suffices to show the case
     when $\pi$ is cyclic.

     If $\pi = \mathbb{Z}$, we are in the case
     of $\mathcal{C} = \mathcal{F} \mathcal{G}$, and
     $S^{1}$ is a $K(\mathbb{Z},1)$, and
     obviously $H_k\left( S^{1}  \right) \in \mathcal{C}$.
     If $\pi = \mathbb{Z} / m$, we know that
     $H_k \left( K\left( \mathbb{Z}/m,1 \right)  \right) $ 
     is $\mathbb{Z} / m$ for odd $k$ and $0$ for even
     $k>0$, since we can choose an infinite-dimensional
     lens space for $K\left( \mathbb{Z}/m,1 \right) $.
     Hence $H_k \left( K\left( \mathbb{Z}/m,1 \right)  \right) 
     \in \mathcal{C}$ for $k>0$.\\
     \linebreak
     For the class
     $\mathcal{T}_p$, we can use the construction
     in section 1.B in Hatcher's Algebraic Topology of
     a $K\left( \pi,1 \right) $ CW complex 
     $B \pi$ with the property that for any
     subgroup $G \subset \pi$, $BG$ is a subcomplex
     of $B \pi$ (TODO).
     An element $x \in H_k \left( B \pi \right) $ with
     $k > 0$ is represented by a singular
     chain $\sum_i n_i \sigma_i$ with compact
     image contained in some finite subcomplex
     of  $B \pi$. This finite subcomplex
     can involve generation by only finitely many elements
     of $\pi$, hence is contained in a subcomplex $BG$ 
     for some finitely generated subgroup $G \subset \pi$.
     Since $G \in \mathcal{F}_P$, by the first part
     of the proof, we know that the element
     of $H_k(BG)$ represented by $\Sigma_i n_i \sigma_i$ has
     finite order divisible only by primes in $P$, so
     the same is true for its image $x \in H_k\left( B \pi \right) $.
\end{proof}


\begin{proof}[Proof of Theorems
    \ref{Thm:SIAO9} and \ref{Thm:Hurewicz-Mod-C}]
    Assume first that $X$ is simply-connected. Consider
    the Postnikov tower for $X$ :
    \[
    \ldots \to X_n \to X_{n-1} \to \ldots \to 
    X_2 = K\left( \pi_2 (X),2 \right) 
    \] 
    where $X_n \to X_{n-1}$ is a fibration with
    fiber $F_n = K\left( \pi_n(X), n \right) $.
    If $\pi_i (X) \in \mathcal{C}$ for all $i$, then
    by Lemma \ref{Lemma:SIDOA2}, we have
    $H_k(X_2) \in \mathcal{C}$ for all $k$, and
    $H_k(F_n) = H_k\left( K \left( \pi_n (X),n \right)  \right) 
    \in \mathcal{C}$ for all $k$ and $n$.
    Since $X_n \to X_{n-1}$ is a fibration with fiber
    $F_n$, we obtain by induction and
    Lemma \ref{Lemma:SIDOA1} that
    $H_k\left( X_n \right) \in \mathcal{C}$ for all
    $n$ and all $k$.


    Now, by Cor. 4.12 in Hatcher, we know that
    the inclusion $X^{n} \hookrightarrow X$ induces an isomorphism
    on $\pi_i$ for $i<n$ and a surjection for $i=n$, so
    by attaching cells of dimensions $\ge n+1$, we can
    obtain a space $X'$ such that the composite inclusion
    $X^{n} \hookrightarrow X \hookrightarrow X'$ induces
    an isomorphism on all homotopy groups, hence
    is a homotopy equivalence.

    Thus, up
    to homotopy equivalence, we can build
    $X_n$ from $X$ by attaching cells of dimension
    $\ge n+1$, so $H_i(X) \cong
    H_i(X_n)$ for $n\ge i$, and therefore
    $H_i \left( X \right) \in \mathcal{C}$ for all $i >0$.

    Next, the Hurewicz maps $\pi_n (X) \to H_n(X)$ and
    $\pi_n(X_n) \to H_n(X_n)$ are equivalent, and
    we can deal with the latter via the fibration
    $F_n \to X_n \to X_{n-1}$. Recall
     that $F_n = K\left( \pi_n(X), n \right) $, so
     by the Hurewicz theorem,
     $H_i (F_n) = 0$ for $0<i < n$, hence the associated
     spectral sequence to the fibration has nothing
     between the  $0$ th and $n$ th rows, so
     the first nontrivial differential is
     $d_{n+1} \colon H_{n+1}(X_{n-1}) \to 
     H_n (F_n)$. 
     Recalling that from the spectral sequence, we have

     \[
     0 \to F_{n-1}H_n(X_n) \to 
     H_n(X_n) \to E_{n,0}^{\infty} \to 0
     \] 
     we get 
     that since
     $F_{i}H_n(X_n) / F_{i-1}H_n(X_n) \cong
     E_{i,n-i}^{\infty}
     \cong 0$ for $0<i<n$, this implies that
     $F_{n-1}H_n(X_n) \cong
     F_0 H_n(X_n) \cong
     E_{0,n}^{\infty}$, so we get a SES
     \[
     0 \to E_{0,n}^{\infty} \to 
     H_n(X_n) \to E_{n,0}^{\infty} \to 0.
     \] 
     Now, also
     since the only possible nontrivial differential terminating
     at
     $E_{0,n}^{r}$ for all $r$ is
     $d_{n+1} \colon
     H_{n+1}(X_{n-1}) \to H_{n}(F_n)$, we find that
     $E_{0,n}^{\infty}$ must be the cokernel, so
     we get that
     \[
     H_{n+1}(X_{n-1}) \stackrel{d_{n+1}}{\to} H_n(F_n)
     \to E_{0,n}^{\infty} \to 0
     \] 
     is exact.

     Next, what is the map
     $H_n(F_n) \to E_{0,n}^{\infty} \to 
     H_n(X_n)$?
     One can read off that the
     latter map $E_{0,n}^{\infty} \to H_n(X_n)$ is the
     inclusion, and the former is the quotienting map.
     


\end{proof}











\subsection{Supplements}

\subsubsection{Naturality}

Suppose we are given two fibrations
and a map between them, a commutative diagram as
below:

\begin{equation*}
\begin{tikzcd}
    F \ar[r] \ar[d] & X \ar[r] \ar[d, "\tilde{f}"] &
    B \ar[d, "f"] \\
    F' \ar[r] & X' \ar[r] & B'
\end{tikzcd}
\end{equation*}
Suppose that the hypotheses of the LSSS are satisfied
for both fibrations. Then the naturality properties are:
\begin{enumerate}
    \item There are induced maps
        $f_*^{r} \colon E_{pq}^{r} \to 
        E_{pq}^{'r}$ commuting with differentials, with
        $f_*^{r+1}$ the map on homology induced by
        $f_*^{r}$.
    \item The map $\tilde{f}_* \colon
        H_* (X;G) \to H_* (X';G)$ preserves filtrations,
        inducing a map on successive quotient groups
        which is the map $f_*^{\infty}$.
    \item Under the isomorphisms
        $E_{pq}^2 \cong
        H_p (B; H_q(F;G))$ and
        $E_{pq}^{'2} \cong
        H_p (B'; H_q(F';G))$, the map
        $f_*^2$ corresponds to the map induced by
        the maps $B \to B'$ and $F \to F'$.
\end{enumerate}

\subsection{Spectral Sequence Comparison}

\begin{proposition}[]
    Suppose we have a map of fibrations
    as in the diagram:

\begin{equation*}
\begin{tikzcd}
    F \ar[r] \ar[d] & X \ar[r] \ar[d, "\tilde{f}"] &
    B \ar[d, "f"] \\
    F' \ar[r] & X' \ar[r] & B'
\end{tikzcd}
\end{equation*}
and that both fibrations satisfy the hypothesis of trivial
action for the Serre spectral sequence. Then if
two of the three maps $F \to F', B \to B'$ and
$X \to X'$ induce isomorphisms
on $H_* \left( - ; R \right) $ with $R$ a PID,
so does the third.




\end{proposition}




\subsection{Cohomology}

\begin{theorem}[]
    For a fibration $F \to X \to B$ with
    $B$ path-connected and $\pi_1 (B)$ acting
    trivially on $H^{*}(F;G)$, there is a spectral
    sequence $\left\{ E_r^{p,q}, d_r \right\} $ with:
    \begin{enumerate}
        \item $d_r \colon E_r^{p,q} \to E_{r}^{p+r, q-r+1}$ 
            and $E_{r+1}^{p,q} = \ker d_r / \im d_r$ at
            $E_{r}^{p,q}$.
        \item Stable terms $E_{\infty}^{p,n-p}$ isomorphic
            to the successive quotients
            $F_p^{n} / F_{p+1}^{n}$ in a filtration
            $0 \subset F_{n}^{n} \subset \ldots \subset 
            F_0^{n} = H^{n}(X;G)$ of $H^{n}(X;G)$.
        \item $E_2^{p,q} \cong H^{p}(B;H^{q}(F;G))$.
    \end{enumerate}
\end{theorem}

\subsection{Multiplicative structure}

\begin{definition}[Weibel, multiplicative structure]
    Suppose that for $r = a$ we are given a
    bigraded product
    \[
    E^{p_1 q_1}_{r} \times E^{p_2q_2}_{r} \to 
    E^{p_1+p_2 , q_1+ q_2}_{r}
    \] 
    such that the differential $d_{r}$ satisfies
    the Leibnitz relation
    \[
    d_{r}(x_1 x_2) = d_{r}(x_1) x_2 + 
    (-1)^{p_1} x_1 d_{r}(x_2), \quad x_i \in E^{p_i q_i}_{r}.
    \] 
    Then the product of two cycles (boundaries)
    is again a cycle (boundary), and
    by induction, we have the above product for every
    $r \ge a$. We shall call this a \textit{multiplicative structure}
    on the spectral sequence. 
\end{definition}


When considering cohomology with
coefficients in a ring $R$, we can
construct a multiplicative structure on a spectral sequence
with $r = 1$
with the following properties:
\begin{enumerate}
    \item The product
        $E_2^{p,q} \times E_2^{s,t} \to 
        E_2^{p+s,q+t}$ is
        $(-1)^{qs}$ times the standard cup product
        \[
        H^{p}(B; H^{q}(F;R)) \times H^{s}(B;
        H^{t}(F;R)) \to H^{p+s}(B;H^{q+t}(F;R))
        \] 
        sending a pair of cocycles
        $\left( \varphi , \psi  \right) $ to
        $\varphi \smile \psi $ where coefficients
        are multiplied via the cup product
        $H^{q}(F;R) \times H^{t}(F;R) \to H^{q+t}(F;R)$.
    \item The cup product in $H^{*}(X;R)$ restricts
        to  maps $F_p^{m} \times F_s^{n} \to F_{p+s}^{m+n}$.
        These induce quotient maps 
        $F_p^{m} / F_{p+1}^{m} \times F_s^{n} / F_{s+1}^{n} \to 
        F_{p+s}^{m+n} / F_{p+s+1}^{m+n}$ that coincide
        with the products
        $E_{\infty}^{p,m-p} \times E_{\infty}^{s,n-s} \to 
        E_{\infty}^{p+s, m+n-p-s}$.
\end{enumerate}


We shall obtain these products by thinking of the cup product
as the composition
\[
H^{*}(X;R) \times H^{*}(X;R) \stackrel{\times }{\to} 
H^{*}(X \times X; R) \stackrel{\Delta^*}{\to} 
H^{*}(X;R)
\] 
of cross product with the map induced by the
diagonal map $\Delta \colon X \to X \times X$.
This can be seen because
\begin{align*}
    \Delta^{*}  (- \times -) (a,b) (x)
    &= a \times b \circ \Delta (x)\\
    &= p_1^{*}(a) \smile p_2^{*}(b) \circ \Delta (x)\\
    &= p_1^{*}(a)(x,x) p_2^{*}(b)(x,x)\\
    &= a(x) b(x)\\
    &= (a \smile b)(x)
\end{align*}
so
$\Delta^{*} \circ (- \times -) = (- \smile -)$.





\newpage




\subsection{The Spectral Sequence of a Filtered
Complex}

\begin{definition}[Differential Complex]
    A differential complex $K$ with differential operator
    $D$ is an abelian group $K$ together with
    a group homomorphism $D \colon K \to K$ such that
    $D^2 = 0$.
\end{definition}

Let $K$ be a differential complex with differential operator
$D$.
Usually $K$ comes with a grading
$K = \bigoplus_{k \in \mathbb{Z}}C^{k}$ and
$D \colon C^{k} \to C^{k+1}$ increases the
degree by $1$, but the grading is not
absolutely necessary.

\begin{definition}[Subcomplex]
    A \textit{subcomplex} $K'$ of $K$ is a
    graded subgroup such that $DK' \subset 
    K'$.
\end{definition}

\begin{definition}[Filtration, Associated Graded Complex]
    A sequence of subcomplexes
    \[
    K = K_0 \supset K_1 \supset K_2 \supset K_3 \supset
    \ldots
    \] 
    is called a \textit{filtration} on $K$.
    This makes $K$ into a \textit{filtered complex}, with
    \textit{associated graded complex}
    \[
    GK = \bigoplus_{p=0}^{\infty} K_p / K_{p+1}.
    \] 
    For notational reasons, we usually extend the
    filtration to negative indices
    by defining
    $K_p = K$ for $p < 0$.
\end{definition}

\begin{example}[]
    If $K = \bigoplus K^{p,q}$ is a double complex with
    horizontal operator $\delta$ and vertical operator
    $d$ (which we assume
    to commute), we can form a single complex out of it by
    setting $C^{k} = \bigoplus_{p+q=k} K^{p,q}$ and
    then letting
    $K = \bigoplus C^{k}$ and
    the differential operator
    $D \colon C^{k} \to C^{k+1}$ to be
    $D = \delta + (-1)^{p} d$. 
    Then letting
    \[
    K_p = \bigoplus_{i\ge p} \bigoplus_{q \ge 0}
    K^{i,q}
    \] 
    we obtain a filtration on
    $K$.
\end{example}




Suppose now that we have a general filtered
complex $K
= K_0 \supset K_1 \supset \ldots$, and let $A$ be the group defined
by
\[
A = \bigoplus_{p \in \mathbb{Z}} K_p.
\] 
Then $A$ is again a differential complex with
operator $D$.
Let $i \colon A \to A$ be the inclusion
$K_{p+1} \hookrightarrow K_{p}$ on each $p$.
Let $B$ be the cokernel of $i \colon A \to A$.
Then $B = GK =
\bigoplus_{p=0}^{\infty} K_p / K_{p+1}$, and
we have an exact sequence
\[
0 \to A \stackrel{i}{\to} A
\stackrel{j}{\to} GK \to 0.
\] 






\newpage
\section{Weibel}


\section{Double and Total Complexes}

\begin{definition}[Double complex]
    A \textit{double complex} (or \textit{bicomplex}) in
    an abelian category $\mathcal{A}$ is a family
    $\left\{ C_{p,q} \right\} $ of objects of $\mathcal{A}$,
    together with maps
    \[
    d^{h} \colon C_{p,q} \to C_{p-1,q} \quad
    \text{and} \quad
    d^{v} \colon C_{p,q} \to C_{p,q-1}
    \] 
    such that $d^{h} \circ d^{h} =
    d^{v} \circ d^{v} = d^{v} d^{h} +
    d^{h} d^{v} = 0$.
\end{definition}

It is useful to picture the double complex as a lattice
in which the maps $d^{h}$ go horizontally, the maps
$d^{v}$ go vertically, and each square anticommutes.\\
\linebreak
Each row $C_{*q}$ and each
columns $C_{p*}$ is a chain complex.

We say that the double complex
$C$ is \textit{bounded} if $C$ has only finitely many
nonzero terms along each diagonal line
$p+q = n$.
For example, if $C$ is concentrated in the first quadrant of
the plane (a \textit{first quadrant double complex}).


\subsubsection{Sign Trick}

Are the maps
$d^{v}$ and $d^{h}$ maps in
$\Ch$?

Because of anticommutativity, the chain map conditions fail, but
we can construct chain maps
$f_{*q} $ from $C_{*,q}$ to $C_{*,q-1}$ by introducing signs:
\[
f_{p,q} = (-1)^{p} d_{p,q}^{v} \colon
C_{p,q} \to C_{p,q-1}.
\] 
Using this sign trick, we can identify the category of double
complexes with the category
$\Ch \left( \Ch \right) $.

\subsubsection{Total Complexes}

To see why the anticommutativity condition
$d^{v} d^{h} + d^{h} d^{v} = 0$ is useful,
we define the \textit{total complexes} 
$\Tot (C) = Tot^{\prod} (C)$ and
$\Tot^{\oplus} (C)$ as follows:

 \begin{definition}[Total complexes]
    We define
    \[
        \Tot^{\prod}(C)_n = 
        \prod_{p+q=n}C_{p,q} \quad
        \text{and} \quad
        \Tot^{\oplus} (C)_n = 
        \bigoplus_{p+q=n}C_{p,q}.
    \] 
    The formula $d = d^{h} + d^{v}$ define maps
    \[
        d \colon \Tot^{\prod} (C)_n = 
        \prod_{p+q = n} C_{p,q} \quad \text{and} \quad
        d \colon \Tot^{\oplus}(C)_n \to 
        \Tot^{\oplus} (C)_{n-1}
    \] 
    such that $d \circ d = 0$, making
    $\Tot^{\prod}(C)$ and
    $\Tot^{\oplus}(C)$ into chain complexes.
\end{definition}

\begin{exercise}[]
    Check that $d = d^{h} + d^{v}$ define maps as
    claimed.
\end{exercise}

\begin{solution}
    Let
    $\left( \alpha_{p,q} \right) 
    \in \Tot^{\prod}(C)_n$, so
    $p+q = n$. Then
    $d \left( \left( \alpha_{p,q} \right)  \right) 
    = d^{h} \left( \left( \alpha_{p,q} \right)  \right) 
    + d^{v} \left( \left( \alpha_{p,q} \right)  \right) 
    = \left( \alpha_{p-1,q} \right) 
    + \left( \alpha_{p,q-1} \right) \in 
    \prod_{p+q=n-1} C_{p,q}$.
    Clearly, this also works for
    direct products since the number of non-zero terms 
    under $d$ just multiplies by $2$, hence is still finite.
    We also want to show that
    $d \circ d = 0$. For this, note that
    \begin{align*}
        d \circ d \left( \alpha \right) =
    d \left( d^{h} (\alpha) + d^{v}(\alpha) \right) 
    &= d^{h} \left( d^{h}(\alpha) + 
    d^{v} (\alpha) \right) +
    d^{v} \left( d^{h}(\alpha) + d^{v}(\alpha) \right)\\
    &= d^{h}d^{h} (\alpha) + d^{h} d^{v}(\alpha)
+ d^{v} d^{h}(\alpha) + d^{v} d^{v} (\alpha) \\
    &= 0.
    \end{align*}
\end{solution}



\subsection{Terminology}

\begin{definition}[Homology spectral sequence]
    A \textit{homology spectral sequence (starting
    with $E^{a}$ )} in an abelian category
    $\mathcal{A}$ consists of the following data:
    \begin{enumerate}
        \item A family $\left\{ E_{pq}^{r} \right\} $ of
            objects of $\mathcal{A}$ defined for
            all integers $p,q$ and $r \ge a$.
        \item Maps $d_{pq}^{r} \colon
            E_{pq}^{r} \to E_{p-r,q+r-1}^{r}$ that
            are differentials in the sense that
            $d^{r} d^{r} = 0$, so that the "lines
            of slope $-(r+1) / r$ " in the lattice
            $E_{* *}^{r}$ form chain complexes.
        \item Isomorphisms between $E_{pq}^{r+1}$ and the
            homology of $E_{* *}^{r}$ at the spot
            $E_{pq}^{r}$ :
            \[
            E_{pq}^{r+1} \cong \ker d_{pq}^{r} /
            \im d_{p+r, q-r+1}^{r}.
            \] 
    \end{enumerate}
\end{definition}

Note that $E_{pq}^{r+1}$ is a subquotient of
$E_{pq}^{r}$, and that
each differential $d_{pq}^{r}$ decreases the
total degree by one.

\begin{definition}[Total degree]
    The \textit{total degree} of the term
    $E_{pq}^{r}$ is $n = p+q$.
\end{definition}

\begin{example}[]
    A \textit{first quadrant (homology) spectral sequence}
    is one with $E_{pq}^{r} = 0$ unless
    $p\ge 0$ and $q\ge 0$. 

    If we fix $p$ and $q$, then
    $E_{pq}^{r} = E_{pq}^{r+1}$ for all large
    enough $r$ (for $r> \max \left\{ p,q+1 \right\} $ ), because
    $d^{r}$ landing in the $(p,q)$ spot comes from the fourth
    quadrant, and the $d^{r}$ leaving $E_{pq}^{r}$ lands
    in the second quadrant.

    We write
    $E_{pq}^{\infty}$ for this
    stable value of $E_{pq}^{r}$.
\end{example}

\begin{definition}[Dual Definition, Cohomology spectral sequence]
    A \textit{cohomology spectral sequence (starting with
    $E_a$ )} in $\mathcal{A}$ is a family
    $\left\{ E_r^{pq} \right\} $ of objects $(r\ge a)$, together
    with maps $d_r^{pq}$ going "to the right":
    \[
    d_r^{pq} \colon E_{r}^{pq} \to E_{r}^{p+r,q-r+1}
    \] 
    which are differentials in the sense that
    $d_r d_r = 0$.

    So it is the same thing as a homology
    spectral sequence, reindexed via
    $E_r^{pq} = E_{-p,-q}^{r}$, so
    that $d_r$ \textit{increases} the total degree
    $p+q$ of $E_{pq}^{r}$ by one.
\end{definition}


\begin{definition}[Bounded convergence]
    A homology spectral sequence is said to be
    \textit{bounded} if for each $n$, there
    are only finitely many nonzero terms of total
    degree $n$ in $E_{* *}^{a}$.

    \begin{exercise}[]
        Show that if
        $E_{* *}$ is a bounded homology
        spectral sequence, then
        for each $p$ and $q$, there is an $r_0$ such that
    $E_{pq}^{r} = E_{pq}^{r+1}$ for all $r \ge r_0$.
    \end{exercise}
    \begin{proof}
        If the spectral sequence has at most 
        $N$ non-vanishing terms of degree $n$ on page
        $r$, say, then on the following pages,
        we have at most $N$ non-vanishing terms of degree
        $n$ again, since these are homologies of
        the terms of degree $n$ on the previous pages.

        Hence, for the bounded sequence, for each
        $n$, there exists $L (n) \in \mathbb{Z}$ such that
        $E_{p, n-p}^{r} = 0$ for all $p \le L(n)$ and all
        $r$.
        Similarly, there is a $T(n) \in \mathbb{Z}$ such that
        $E_{n-q,q}^{r} = 0$ for all $q \le T(n)$ and all
        $r$.\\
        Now we claim that
        $E^{r}_{p,q} = E_{p,q}^{\infty}$ for
         \[
        r> \max \left\{ p - L(p+q-1), q+1 - T(p+q+1) \right\} .
        \] 

        This is because
        we have
        \begin{enumerate}
            \item $p-r < L(p+q-1)$, so
                $0 = E_{p-r, p+q-1 - (p-r)}^{r} = 
                E^{r}_{p-r, q+r-1}$, so
                $\ker d_{p,q}^{r} = E_{p,q}^{r}$, and
            \item $q-r+1 < T(p+q+1)$, so
                $0 = 
                E_{(p+q+1)-(q-r+1),q-r+1} = E_{p+r, q-r+1}$,
                and hence $d_{p+r,q-r+1} \colon
                0 = E_{p+r,q-r+1}^{r} \to E_{p,q}^{r}$ is $0$.
        \end{enumerate}
        Thus
        \begin{align*}
            E_{pq}^{r+1} 
            &= \ker d_{pq}^{r} / \im d_{p+r,q-r+1}^{r}\\
            &= E_{pq}^{r} / 0\\
            &= E_{pq}^{r}
        \end{align*}
    \end{proof}
    We write
    $E_{pq}^{\infty}$ for this stable value of
    $E_{pq}^{r}$.\\
    \linebreak
    Next, we say that a bounded spectral sequence
    \textit{converges} to $H_*$ if we are given
    a family of objects $H_n$ of $\mathcal{A}$, each
    having a \textit{finite} filtration
    \[
    0 = F_s H_n \subset \ldots \subset 
    F_{p-1} H_n \subset F_p H_n \subset \ldots
    \subset 
    F_t H_n = H_n,
    \] 
    and we are given isomorphisms
    $E_{pq}^{\infty} \cong
    F_p H_{p+q} / F_{p-1} H_{p+q}$.

    The traditional symbolic way of describing such a bounded
    convergence is like this:
    \[
    E_{pq}^{a} \implies H_{p+q}.
    \] 

    Similarly, a cohomology spectral sequence is
    called \textit{bounded} if there are
    only finitely many nonzero terms in each total 
    degree in $E_a^{* *}$. In a bounded cohomology
    spectral sequence, we write
    $E_{\infty}^{p q}$ for the stable value of the
    terms $E_{r}^{pq}$ and say the (bounded) spectral
    sequence converges to $H^{*}$ if there is a
    \textit{finite} filtration
    \[
    0 = F^{t} H^{n} \subset \ldots \subset 
    F^{p+1} H^{n} \subset F^{p} H^{n} \subset \ldots
    \subset  F^{s} H^{n} = H^{n},
    \] 
    so that
    \[
    E_{\infty}^{pq} \cong F^{p} H^{p+q} / F^{p+1} H^{p+q}.
    \] 
\end{definition}

    \begin{example}[]
        If a first quadrant homology spectral sequence converges
        to $H_*$, then each $H_n$ has a finite filtration of length
        $n+1$ :
        \[
        0 = F_{-1} H_n \subset F_0 H_n \subset 
        \ldots \subset F_{n-1} H_n \subset 
        F_n H_n = H_n.
        \] 

        The bottom piece
        $F_0 H_n = E_{0n}^{\infty}$ of $H_n$ is located
        on the $y$-axis, and the top quotient
        $H_n / F_{n-1} H_n \cong E_{n 0}^{\infty}$ is located
        on the $x$-axis.

        Note also that each arrow landing on the
        $x$-axis is zero, and each arrow leaving the
        $y$-axis is zero, hence
        $E_{0n}^{a}$ is a quotient of
        $E_{0 n}^{\infty}$, and each
        $E_{n 0}^{\infty}$ is a subobject of
        $E_{n 0}^{a}$.

        \begin{definition}[Fiber and base terms, edge homomorphism]
            The terms $E_{0n}^{r}$ on the $y$-axis are
            called the \textit{fiber} terms, and the
            terms $E_{n 0}^{r}$ on the $x$-axis are called the
            \textit{base} terms. The
            resulting maps $E_{0 n}^{a} \to 
            E_{0n}^{\infty} \subset H_n$ and
            $H_n \to E_{n 0}^{\infty} \subset 
            E_{n 0}^{a}$ are known as the \textit{edge homomorphisms}
            of the spectral sequence.
        \end{definition}


        Similarly, if a first quadrant cohomology spectral
        sequence converges to $H^{*}$, then
        $H^{n}$ has a finite filtration:
        \[
        0 = F^{n+1} H^{n} \subset F^{n} H^{n} \subset 
        \ldots \subset 
        F^{1} H^{n} \subset F^{0} H^{n} = H^{n}.
        \] 
        In this case, the bottom piece $F^{n} H^{n} \cong
        E_{\infty}^{n 0}$ is located on the $x$-axis, and the
        top quotient $H^{n} / F^{1} H^{n} \cong
        E_{\infty}^{0 n}$ is located on the $y$-axis.
        In this case, the edge homomorphisms are the
        maps $E_a^{n 0} \to E_{\infty}^{n 0} \subset H^{n}$ 
        and $H^{n} \to E_{\infty}^{0n} \subset E_a^{0n}$.
    \end{example}

    \begin{definition}[Collapsing of spectral sequence]
        A (homology) spectral sequence
        \textit{collapses at $E^{r} (r \ge 2)$} if there
        is exactly one nonzero row or column in the lattice
        $\left\{ E_{pq}^{r} \right\} $. If a collapsing spectral
        sequence converges to $H_*$, we can read the $H_n$ off:
        $H_n$ is the unique nonzero $E_{pq}^{r}$ with
        $p+q=n$. The overwhelming majority of all applications
        of spectral sequences involve spectral sequences
        that collapse at $E^{1}$ or $E^2$.
    \end{definition}

    \begin{exercise}[2 columns]
        Suppose that a spectral sequence
        converging to $H_*$ has $E_{pq}^2 = 0$ unless
        $p = 0,1$. Show that there are exact sequences
        \[
        0 \to E_{0, n}^2 \to H_n \to E_{1,n-1}^2 \to 0
        \] 
    \end{exercise}

    \begin{proof}
        We have $E_{p,n-p}^{\infty} \cong 0$ if
        $p>1$, so
        $F_{p}H_n / F_{p-1} H_n \cong 0$ whenever
        $p>1$, so
        $F_p H_n \cong F_{p-1} H_n$ for $p>1$.
        Hence $H_n = F_n H_n \cong F_1 H_n$.  
        Now, $E_{1,n-1}^{\infty} \cong
        H_n / F_0 H_n$, and
        $E_{0n}^{\infty} \cong
        F_0 H_n / F_{-1} H_n \cong
        F_0 H_n$, so
        we have a SES
        \[
        0 \to F_0 H_n \hookrightarrow  H_n \to H_n / F_0 H_n \to 0
        \] 
        which thus becomes
        \[
        0 \to E_{0n}^{\infty} \to 
        H_n \to E_{1,n-1}^{\infty} \to 0
        \] 
        Furthermore, all differentials on pages
        $E^{r}$ for $r\ge 2$ are $0$, so
        $E_{0n}^{\infty} \cong
        E_{0n}^2$ and
        $E_{1,n-1}^{\infty} \cong
        E_{1,n-1}^{2}$.
        So we get
        a SES
        \[
        0 \to E_{0n}^2 \to H_n \to E_{1,n-1}^2 \to 0.
        \]
    \end{proof}


    \begin{example}[2 rows]
        Suppose that a spectral sequence
        converging to $H_*$ has $E_{pq}^2 = 0$ unless
        $q = 0,1$. Show that there is a LES
        \[
        \ldots \to H_{p+1} \to E_{p+1,0}^2
        \stackrel{d}{\to} E_{p-1,1}^2 \to H_p
        \to E_{p 0}^2 \stackrel{d}{\to} 
        E_{p-2,1}^2 \to H_{p-1} \to \ldots
        \] 
    \end{example}

    \begin{proof}
        The maps
        $H_p \to E_{p,0}^2$ and
        $E_{p-1,1}^2 \to H_p$ are the edge homomorphisms given,
        respectively, by the map
        $H_p \to H_p / F_{p-1}H_p \cong E_{p,0}^{\infty}
        \hookrightarrow E_{p,0}^2$, where the last
        part is the inclusion since
        $E_{p,0}^{\infty}$ is the kernel of a map
        out of $E_{p,0}^2$, and
        the map
        $E_{p-1,1}^2 \to 
        E_{p-1,1}^2 / \im d \cong
        E_{p-1,1}^{\infty} \cong F_{p-1}H_p
        \subset H_p$. Thus the kernel of $d$ is
        indeed the image of the edge map
        $H_p \to E_{p,0}^2$, giving exactness
        at $E_{p,0}^2$, and the
        image of the edge map
        $E_{p-1,1}^2 \to H_p$ is the subgroup
        $F_{p-1}H_p$. Now, the kernel
        of the edge map
        $H_p \to E_{p,0}^2$ is
        the subgroup $F_{p-1}H_p$, giving
        exactness at $H_p$.
        Also the image of the edge
        map $E_{p-1,1}^2 \to H_p$ is
        $\im d$ giving exactness at
        $E_{p-1,1}^2$. This proves the claim.
    \end{proof}


    \subsection{The category of homology spectral
    sequences}

    A morphism $f \colon E' \to E$ in the category
    of homology spectral sequences is a family
    of maps $f_{pq}^{r} \colon
    E_{pq}^{'r} \to E_{pq}^{r}$ in $\mathcal{A}$ 
    (for $r$ suitably large) such that
    \begin{equation*}
    \begin{tikzcd}
        E_{pq}^{'r} \ar[r, "f^{r}"] \ar[d, "d^{r}"] & E_{pq}^{r}
        \ar[d, "d^{r}"] \\
        E_{p-r, q-r+1}^{'r} \ar[r, "f^{r}"] & E_{p-r,q-r+1}^{r}
    \end{tikzcd}
    \end{equation*}
    for all $p,q$, and such that
    $f_{pq}^{r+1}$ is the map induced
    by $f_{pq}^{r}$ on homology. That is, from the 
    commutative diagram
    \begin{equation*}
    \begin{tikzcd}
        E_{p+r,q+r-1}^{'r} \ar[r, "f^{r}"] \ar[d, "d^{r}"]
        & E_{p+r,q+r-1}^{r}
        \ar[d, "d^{r}"]\\
        E_{pq}^{'r} \ar[r, "f^{r}"] \ar[d, "d^{r}"] & E_{pq}^{r}
        \ar[d, "d^{r}"] \\
        E_{p-r, q-r+1}^{'r} \ar[r, "f^{r}"] & E_{p-r,q-r+1}^{r}
    \end{tikzcd}
    \end{equation*}
    we obtain a map on
    homology since if 
    $\left[ a \right] \in 
    E_{pq}^{'r+1} = H E_{pq}^{'r}$, then
    $d^{r}a = 0$, so
    since  $f^{r} d^{r} = d^{r} f^{r}$, we get
    that $d^{r} \left( f^{r} \left[ a \right]  \right) 
    = f^{r} \left( d^{r} \left[ a \right]  \right) 
    = f^{r} \left[ d^{r} a \right] = 0$, hence
    $f^{r}$ takes cycles to cycles, and
    similarly,
    if $\left[ b \right] \in \im
    d^{r}$, say $\left[ b \right]  = 
    d^{r}\left[ \tilde{b} \right] 
    $, then
    $f^{r} \left[ b \right] =
    d^{r} f^{r} \left[ \tilde{b} \right] $, so
    $f^{r}$ also maps boundaries to boundaries, hence
    $f^{r}$ induces a map on homology.

    
    \begin{lemma}[Mapping Lemma]
        Let $f \colon \left\{ E_{pq}^{r} \right\} \to 
        \left\{ E_{pq}^{'r} \right\} $ be a morphism
        of spectral sequences such that for some
        fixed $r$, $f^{r} \colon
        E_{pq}^{r} \cong E_{pq}^{'r}$ is an isomorphism
        for all $p$ and $q$. Then
        $f^{s} \colon E_{pq}^{s} \cong
        E_{pq}^{'s}$ for all $s\ge r$ as well.
    \end{lemma}

    \begin{proof}
        Suppose $f^{r}$ is an isomorphism.
        Then since $f^{r} d^{r} = d^{r} f^{r}$, we must have
        that $f^{r}$ induces an isomorphism on
        cycles and boundaries, so
        let $B_{pq}'^{r}$ and $B_{pq}^{r}$ denote the
        boundaries at $pq$ and
        $Z_{pq}^{'r}$ and $Z_{pq}^{r}$ the cycles, respectively.
        Then we have the SES
        
        \begin{equation*}
        \begin{tikzcd}
            0 \ar[r] & B_{pq}^{'r} \ar[d, "f^{r}"] \ar[r] &
            Z_{pq}^{'r} \ar[r] \ar[d, "f^{r}"] & E_{pq}^{'r+1} \ar[r] 
            \ar[d, "f^{r+1}"] & 0 \\
            0 \ar[r] & B_{pq}^{r} \ar[r] &
            Z_{pq}^{r} \ar[r]  & E_{pq}^{r+1} \ar[r] 
             & 0 
        \end{tikzcd}
        \end{equation*}
        Applying the $5$-lemma, we get that
        $f^{r+1}$ is an isomorphism for
        all $pq$. By induction, 
        we obtain the claim.
    \end{proof}
 
    \subsubsection{$E^{\infty}$ terms}
    
    Given a homology spectral
    sequence, we know that each $E_{pq}^{r+1}$ is a subquotient
    of the previous term $E_{pq}^{r}$. Letting
    $Z_{pq}^{r}$ be the 
    kernel of 
    $E_{pq}^{r-1} \to E_{p-r,q+r-1}$ and
    $B_{pq}^{r}$ the image of
    $E_{p+r,q-r+1}^{r-1} \to E_{pq}^{r-1}$, we get
    a nested family of subobjects of
    $E_{pq}^{a}$ :
    \[
    0 = B_{pq}^{a} \subset \ldots \subset 
    B_{pq}^{r} \subset B_{pq}^{r+1} \subset \ldots
    \subset Z_{pq}^{r+1} \subset Z_{pq}^{r} \subset 
    \ldots \subset Z_{pq}^{a} = E_{pq}^{a}
    \] 
    such that $E_{pq}^{r} \cong
    Z_{pq}^{r} / B_{pq}^{r}$.
    Let
    \[
    B_{pq}^{\infty} = \bigcup_{r=a}^{\infty} B_{pq}^{r}
    \quad \text{and} \quad 
    Z_{pq}^{\infty} =
    \bigcap_{r=a}^{\infty} Z_{pq}^{r}
    \] 
    and define $E_{pq}^{\infty} = 
    Z_{pq}^{\infty} / B_{pq}^{\infty}$.










\newpage 
\section{Bott Tu}
\subsection{Introduction to Spectral Sequences}

Consider the problem of computing the homology
of the total chain complex
$T_* = \Tot(E_{* * })$ where
$E_{* *}$ is a first quadrant double complex.

Firstly, it is convenient to forget the horizontal differentials
and add a superscript zero, retaining only the vertical
differentials $d^{v}$ along the columns
$E_{p*}^{0}$.

Let $E_{pq}^{1}$ be the vertical homology
$H_q \left( E_{p*}^{0} \right) $ at the
$(p,q)$ spot.





\subsection{Filtrations}

\begin{definition}[Filtered $R$-module]
    A \textit{filtered $R$-module} is an $R$-module
    $A$ with an increasing sequence
    of submodules 
    $\left\{ F_p \right\}_{p \in \mathbb{Z}}$ 
    such that $F_p A \subset F_{p+1}A$ for all
    $p$ and such that
    $\bigcup_{p} F_pA = A$ and
    $\bigcap_{p} F_p A = \left\{ 0 \right\} $.

    A filtration is said to be \textit{bounded} if
    $F_p A = \left\{ 0 \right\} $ for
    $p$ sufficiently small and
    $F_p A = A$ for $p$ sufficiently larger.
\end{definition}

\begin{definition}[Associated graded module]
    The \textit{associated graded module} is defined
    by $G_p A = F_p A / F_{p-1} A$.
\end{definition}

\begin{definition}[Filtered chain complex]
    A \textit{filtered chain complex} is a chain
    complex $\left( C_*, \partial \right) $ together
    with a filtration $\left\{ F_p C_i \right\}_{p \in \mathbb{Z}}$ 
    of each $C_i$ such that the differential preserves
    the filtration, i.e., s.t. 
    $\partial \left( F_p C_i \right) \subset F_p C_{i-1}$.
\end{definition}

Note that we, in particular, obtain an
induced differential
$\partial \colon G_p C_i \to G_{p} C_{i-1}$ by
the universal property of cokernels

\begin{equation*}
\begin{tikzcd}
    F_p C_i \ar[d] \ar[r, "\partial"] & F_p C_{i-1} \ar[d] \\
    F_{p-1} C_i \ar[d, "\coker"]
    \ar[r, "\partial"] & F_{p-1} C_{i-1} \ar[d, "\coker"] \\
    G_p C_i \ar[r, "\partial", dashed] & G_p C_{i-1}
\end{tikzcd}
\end{equation*}
so we obtain an associated graded
chain complex $G_p C_*$.\\
\linebreak
The filtration on $C_*$ also induces
a filtration on the homology of $C_*$ by

\[
F_p H_i (C_*) = 
\left\{ \alpha \in 
H_i (C_*)  \mid 
\left( \exists x \in F_p C_i \right) \colon
\alpha = \left[ x \right] \right\} .
\] 

This filtration has associated graded pieces
$G_p H_i (C_*)$ which, in favorable cases, determine
$H_i (C_*)$.\\
\linebreak



\subsection{Example}

Suppose we have a chain complex
$C_*$ and a filtration consisting
of a single $F_0 C_*$, so
$F_n C_* = 0$ if
$n <0$ and
$F_n C_* = F_0 C_*$ if $n\ge 0$.

Then
$G_n C_* = 0$ for
$n \neq  0$ and
$G_0 C_* = F_0 C_*$ and



























    %\printbibliography
\end{document}
