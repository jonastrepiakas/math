\documentclass[reqno]{amsart}
\usepackage{amscd, amssymb, amsmath, amsthm}
\usepackage{graphicx}
\usepackage[colorlinks=true,linkcolor=blue]{hyperref}
\usepackage[utf8]{inputenc}
\usepackage[T1]{fontenc}
\usepackage{textcomp}
\usepackage{babel}
%% for identity function 1:
\usepackage{bbm}
%%For category theory diagrams:
\usepackage{tikz-cd}

%\usepackage[backend=biber]{biblatex}
%\addbibresource{.bib}


\setlength\parindent{0pt}

\pdfsuppresswarningpagegroup=1

\newtheorem{theorem}{Theorem}[section]
\newtheorem{lemma}[theorem]{Lemma}
\newtheorem{proposition}[theorem]{Proposition}
\newtheorem{corollary}[theorem]{Corollary}
\newtheorem{conjecture}[theorem]{Conjecture}

\theoremstyle{definition}
\newtheorem{definition}[theorem]{Definition}
\newtheorem{example}[theorem]{Example}
\newtheorem{exercise}[theorem]{Exercise}
\newtheorem{problem}[theorem]{Problem}
\newtheorem{question}[theorem]{Question}

\theoremstyle{remark}
\newtheorem*{remark}{Remark}
\newtheorem*{note}{Note}
\newtheorem*{solution}{Solution}



%Inequalities
\newcommand{\cycsum}{\sum_{\mathrm{cyc}}}
\newcommand{\symsum}{\sum_{\mathrm{sym}}}
\newcommand{\cycprod}{\prod_{\mathrm{cyc}}}
\newcommand{\symprod}{\prod_{\mathrm{sym}}}

%Linear Algebra

\DeclareMathOperator{\Span}{span}
\DeclareMathOperator{\im}{im}
\DeclareMathOperator{\diag}{diag}
\DeclareMathOperator{\Ker}{Ker}
\DeclareMathOperator{\ob}{ob}
\DeclareMathOperator{\Hom}{Hom}
\DeclareMathOperator{\Mor}{Mor}
\DeclareMathOperator{\sk}{sk}
\DeclareMathOperator{\Vect}{Vect}
\DeclareMathOperator{\Set}{Set}
\DeclareMathOperator{\Group}{Group}
\DeclareMathOperator{\Ring}{Ring}
\DeclareMathOperator{\Ab}{Ab}
\DeclareMathOperator{\Top}{Top}
\DeclareMathOperator{\hTop}{hTop}
\DeclareMathOperator{\Htpy}{Htpy}
\DeclareMathOperator{\Cat}{Cat}
\DeclareMathOperator{\CAT}{CAT}
\DeclareMathOperator{\Cone}{Cone}
\DeclareMathOperator{\dom}{dom}
\DeclareMathOperator{\cod}{cod}
\DeclareMathOperator{\Aut}{Aut}
\DeclareMathOperator{\Mat}{Mat}
\DeclareMathOperator{\Fin}{Fin}
\DeclareMathOperator{\rel}{rel}
\DeclareMathOperator{\Int}{Int}
\DeclareMathOperator{\sgn}{sgn}
\DeclareMathOperator{\Homeo}{Homeo}
\DeclareMathOperator{\SHomeo}{SHomeo}
\DeclareMathOperator{\PSL}{PSL}
\DeclareMathOperator{\Bil}{Bil}
\DeclareMathOperator{\Sym}{Sym}
\DeclareMathOperator{\Skew}{Skew}
\DeclareMathOperator{\Alt}{Alt}
\DeclareMathOperator{\Quad}{Quad}
\DeclareMathOperator{\Sin}{Sin}
\DeclareMathOperator{\Supp}{Supp}
\DeclareMathOperator{\Char}{char}
\DeclareMathOperator{\Teich}{Teich}
\DeclareMathOperator{\GL}{GL}
\DeclareMathOperator{\tr}{tr}
\DeclareMathOperator{\codim}{codim}
\DeclareMathOperator{\coker}{coker}
\DeclareMathOperator{\corank}{corank}
\DeclareMathOperator{\rank}{rank}
\DeclareMathOperator{\Diff}{Diff}
\DeclareMathOperator{\Bun}{Bun}
\DeclareMathOperator{\Sm}{Sm}
\DeclareMathOperator{\Fr}{Fr}
\DeclareMathOperator{\Cob}{Cob}
\DeclareMathOperator{\Ext}{Ext}
\DeclareMathOperator{\Tor}{Tor}
\DeclareMathOperator{\Conf}{Conf}
\DeclareMathOperator{\UConf}{UConf}
\DeclareMathOperator{\Ch}{Ch}
\DeclareMathOperator{\Tot}{Tot}



%Row operations
\newcommand{\elem}[1]{% elementary operations
\xrightarrow{\substack{#1}}%
}

\newcommand{\lelem}[1]{% elementary operations (left alignment)
\xrightarrow{\begin{subarray}{l}#1\end{subarray}}%
}

%SS
\DeclareMathOperator{\supp}{supp}
\DeclareMathOperator{\Var}{Var}

%NT
\DeclareMathOperator{\ord}{ord}

%Alg
\DeclareMathOperator{\Rad}{Rad}
\DeclareMathOperator{\Jac}{Jac}

%Misc
\newcommand{\SL}{{\mathrm{SL}}}
\newcommand{\mobgp}{{\mathrm{PSL}_2(\mathbb{C})}}
\newcommand{\id}{{\mathrm{id}}}
\newcommand{\MCG}{{\mathrm{MCG}}}
\newcommand{\PMCG}{{\mathrm{PMCG}}}
\newcommand{\SMCG}{{\mathrm{SMCG}}}
\newcommand{\ud}{{\mathrm{d}}}
\newcommand{\Vol}{{\mathrm{Vol}}}
\newcommand{\Area}{{\mathrm{Area}}}
\newcommand{\diam}{{\mathrm{diam}}}
\newcommand{\End}{{\mathrm{End}}}


\newcommand{\reg}{{\mathtt{reg}}}
\newcommand{\geo}{{\mathtt{geo}}}

\newcommand{\tori}{{\mathcal{T}}}
\newcommand{\cpn}{{\mathtt{c}}}
\newcommand{\pat}{{\mathtt{p}}}

\let\Cap\undefined
\newcommand{\Cap}{{\mathcal{C}}ap}
\newcommand{\Push}{{\mathcal{P}}ush}
\newcommand{\Forget}{{\mathcal{F}}orget}




\begin{document}

\section{Double and Total Complexes}

\begin{definition}[Double complex]
    A \textit{double complex} (or \textit{bicomplex}) in
    an abelian category $\mathcal{A}$ is a family
    $\left\{ C_{p,q} \right\} $ of objects of $\mathcal{A}$,
    together with maps
    \[
    d^{h} \colon C_{p,q} \to C_{p-1,q} \quad
    \text{and} \quad
    d^{v} \colon C_{p,q} \to C_{p,q-1}
    \] 
    such that $d^{h} \circ d^{h} =
    d^{v} \circ d^{v} = d^{v} d^{h} +
    d^{h} d^{v} = 0$.
\end{definition}

It is useful to picture the double complex as a lattice
in which the maps $d^{h}$ go horizontally, the maps
$d^{v}$ go vertically, and each square anticommutes.\\
\linebreak
Each row $C_{*q}$ and each
columns $C_{p*}$ is a chain complex.

We say that the double complex
$C$ is \textit{bounded} if $C$ has only finitely many
nonzero terms along each diagonal line
$p+q = n$.
For example, if $C$ is concentrated in the first quadrant of
the plane (a \textit{first quadrant double complex}).


\subsubsection{Sign Trick}

Are the maps
$d^{v}$ and $d^{h}$ maps in
$\Ch$?

Because of anticommutativity, the chain map conditions fail, but
we can construct chain maps
$f_{*q} $ from $C_{*,q}$ to $C_{*,q-1}$ by introducing signs:
\[
f_{p,q} = (-1)^{p} d_{p,q}^{v} \colon
C_{p,q} \to C_{p,q-1}.
\] 
Using this sign trick, we can identify the category of double
complexes with the category
$\Ch \left( \Ch \right) $.

\subsubsection{Total Complexes}

To see why the anticommutativity condition
$d^{v} d^{h} + d^{h} d^{v} = 0$ is useful,
we define the \textit{total complexes} 
$\Tot (C) = Tot^{\prod} (C)$ and
$\Tot^{\oplus} (C)$ as follows:

 \begin{definition}[Total complexes]
    We define
    \[
        \Tot^{\prod}(C)_n = 
        \prod_{p+q=n}C_{p,q} \quad
        \text{and} \quad
        \Tot^{\oplus} (C)_n = 
        \bigoplus_{p+q=n}C_{p,q}.
    \] 
    The formula $d = d^{h} + d^{v}$ define maps
    \[
        d \colon \Tot^{\prod} (C)_n = 
        \prod_{p+q = n} C_{p,q} \quad \text{and} \quad
        d \colon \Tot^{\oplus}(C)_n \to 
        \Tot^{\oplus} (C)_{n-1}
    \] 
    such that $d \circ d = 0$, making
    $\Tot^{\prod}(C)$ and
    $\Tot^{\oplus}(C)$ into chain complexes.
\end{definition}

\begin{exercise}[]
    Check that $d = d^{h} + d^{v}$ define maps as
    claimed.
\end{exercise}

\begin{solution}
    Let
    $\left( \alpha_{p,q} \right) 
    \in \Tot^{\prod}(C)_n$, so
    $p+q = n$. Then
    $d \left( \left( \alpha_{p,q} \right)  \right) 
    = d^{h} \left( \left( \alpha_{p,q} \right)  \right) 
    + d^{v} \left( \left( \alpha_{p,q} \right)  \right) 
    = \left( \alpha_{p-1,q} \right) 
    + \left( \alpha_{p,q-1} \right) \in 
    \prod_{p+q=n-1} C_{p,q}$.
    Clearly, this also works for
    direct products since the number of non-zero terms 
    under $d$ just multiplies by $2$, hence is still finite.
    We also want to show that
    $d \circ d = 0$. For this, note that
    \begin{align*}
        d \circ d \left( \alpha \right) =
    d \left( d^{h} (\alpha) + d^{v}(\alpha) \right) 
    &= d^{h} \left( d^{h}(\alpha) + 
    d^{v} (\alpha) \right) +
    d^{v} \left( d^{h}(\alpha) + d^{v}(\alpha) \right)\\
    &= d^{h}d^{h} (\alpha) + d^{h} d^{v}(\alpha)
+ d^{v} d^{h}(\alpha) + d^{v} d^{v} (\alpha) \\
    &= 0.
    \end{align*}
\end{solution}

\subsection{Exact Couples}

\begin{definition}[Exact Couple]
    An \textit{exact couple} is an exact sequence of
    abelian groups of the form
    \begin{equation*}
    \begin{tikzcd}
        A \ar[rr, "i"] & & A \ar[dl, "j"] \\
                       & B \ar[ul, "k"] &
    \end{tikzcd}
    \end{equation*}
    where $i,j$ and $k$ are group homomorphisms. Define
    $d \colon B \to B$ by $d = j \circ k$.
    Then $d^2 = j (kj)k = 0$, so
    $H(B) := \ker d / \im d$ is defined - in particular,
    since $A$ and $B$ are abelian, the quotient
    $H(B)$ is well-defined and a group.
\end{definition}

\begin{definition}[Derived Couple]
    Out of a given exact couple, we can construct
    a new exact couple, called the \textit{derived couple}:

    \begin{equation*}
    \begin{tikzcd}
        A' \ar[rr, "i'"] & & A' \ar[dl, "j'"] \\
                       & B' \ar[ul, "k'"] &
    \end{tikzcd}
    \end{equation*}
    where we define
    \begin{enumerate}
        \item $A' = i(A)$ and $B' = H(B)$.
        \item $i'$ is the induced
            map $i' := i|_{A'} \colon
            A'\to A'$ by $i' (ia) = i(ia)$ 
        \item We define $j'$ by
            $j' a' = \left[ ja \right] $ where
            $a' = ia$ for some $a$ in $A$.
        \item $k'$ is defined by
            $k' \left[ b \right] = kb \in i(A)$.
    \end{enumerate}
    With these definitions, the derived couple is an exact
    couple.
\end{definition}

\begin{exercise}[]
    Check that the maps are well-defined and that the
    derived sequence is exact.
\end{exercise}

\begin{proof}
    We must check that $j'$ and $k'$ are well-defined maps.

    Suppose $a' = ia = i \tilde{a}$.
    Then $a-\tilde{a} \in \ker i = \im k$ so
    $a - \tilde{a} = k\left[ b \right] $. Hence
    Then
    $j a - j \tilde{a} =
    j k \left[ b \right] 
    = d \left[ b \right] \in \im d$, so
    $\left[ j a \right] =
    \left[ j \tilde{a} \right] $.\\
    Next, suppose
    $\left[ b \right] = \left[ \tilde{b} \right] $, so
    $b - \tilde{b}\in 
    \im d$, i.e., $b - \tilde{b} = 
    jk (\overline{b})$.
    Then
    $k b - k \tilde{b} = 
    kjk (\overline{b}) = 
    0$, so
    $k' \left[ b \right] =
    k'\left[ \tilde{b} \right] $.\\
    \linebreak
    Lastly, exactness at $B'$:
     suppose $k' \left[ b \right] = 0$. Then
     $kb = 0$, so by exactness of the original exact
     couple, there exists some $a \in A$ such that
     $j (a) = b$. Then
     let $a' = i(a)$, so
     $j' (a') = \left[ j (a) \right] =
     \left[ b \right] $, hence
     $\ker k' \subset \im j'$.\\
     Conversely, 
     $k' j' (a') =
     k' \left[ ja \right] =
     kja = 0$, by exactness at
     $B$ of the original couple.
\end{proof}

\subsection{The Spectral Sequence of a Filtered
Complex}

\begin{definition}[Differential Complex]
    A differential complex $K$ with differential operator
    $D$ is an abelian group $K$ together with
    a group homomorphism $D \colon K \to K$ such that
    $D^2 = 0$.
\end{definition}

Let $K$ be a differential complex with differential operator
$D$.
Usually $K$ comes with a grading
$K = \bigoplus_{k \in \mathbb{Z}}C^{k}$ and
$D \colon C^{k} \to C^{k+1}$ increases the
degree by $1$, but the grading is not
absolutely necessary.

\begin{definition}[Subcomplex]
    A \textit{subcomplex} $K'$ of $K$ is a
    graded subgroup such that $DK' \subset 
    K'$.
\end{definition}

\begin{definition}[Filtration, Associated Graded Complex]
    A sequence of subcomplexes
    \[
    K = K_0 \supset K_1 \supset K_2 \supset K_3 \supset
    \ldots
    \] 
    is called a \textit{filtration} on $K$.
    This makes $K$ into a \textit{filtered complex}, with
    \textit{associated graded complex}
    \[
    GK = \bigoplus_{p=0}^{\infty} K_p / K_{p+1}.
    \] 
    For notational reasons, we usually extend the
    filtration to negative indices
    by defining
    $K_p = K$ for $p < 0$.
\end{definition}

\begin{example}[]
    If $K = \bigoplus K^{p,q}$ is a double complex with
    horizontal operator $\delta$ and vertical operator
    $d$ (which we assume
    to commute), we can form a single complex out of it by
    setting $C^{k} = \bigoplus_{p+q=k} K^{p,q}$ and
    then letting
    $K = \bigoplus C^{k}$ and
    the differential operator
    $D \colon C^{k} \to C^{k+1}$ to be
    $D = \delta + (-1)^{p} d$. 
    Then letting
    \[
    K_p = \bigoplus_{i\ge p} \bigoplus_{q \ge 0}
    K^{i,q}
    \] 
    we obtain a filtration on
    $K$.
\end{example}




Suppose now that we have a general filtered
complex $K
= K_0 \supset K_1 \supset \ldots$, and let $A$ be the group defined
by
\[
A = \bigoplus_{p \in \mathbb{Z}} K_p.
\] 
Then $A$ is again a differential complex with
operator $D$.
Let $i \colon A \to A$ be the inclusion
$K_{p+1} \hookrightarrow K_{p}$ on each $p$.
Let $B$ be the cokernel of $i \colon A \to A$.
Then $B = GK =
\bigoplus_{p=0}^{\infty} K_p / K_{p+1}$, and
we have an exact sequence
\[
0 \to A \stackrel{i}{\to} A
\stackrel{j}{\to} GK \to 0.
\] 






\newpage
\section{Introduction to Spectral Sequences}

Consider the problem of computing the homology
of the total chain complex
$T_* = \Tot(E_{* * })$ where
$E_{* *}$ is a first quadrant double complex.

Firstly, it is convenient to forget the horizontal differentials
and add a superscript zero, retaining only the vertical
differentials $d^{v}$ along the columns
$E_{p*}^{0}$.

Let $E_{pq}^{1}$ be the vertical homology
$H_q \left( E_{p*}^{0} \right) $ at the
$(p,q)$ spot.





\section{Filtrations}

\begin{definition}[Filtered $R$-module]
    A \textit{filtered $R$-module} is an $R$-module
    $A$ with an increasing sequence
    of submodules 
    $\left\{ F_p \right\}_{p \in \mathbb{Z}}$ 
    such that $F_p A \subset F_{p+1}A$ for all
    $p$ and such that
    $\bigcup_{p} F_pA = A$ and
    $\bigcap_{p} F_p A = \left\{ 0 \right\} $.

    A filtration is said to be \textit{bounded} if
    $F_p A = \left\{ 0 \right\} $ for
    $p$ sufficiently small and
    $F_p A = A$ for $p$ sufficiently larger.
\end{definition}

\begin{definition}[Associated graded module]
    The \textit{associated graded module} is defined
    by $G_p A = F_p A / F_{p-1} A$.
\end{definition}

\begin{definition}[Filtered chain complex]
    A \textit{filtered chain complex} is a chain
    complex $\left( C_*, \partial \right) $ together
    with a filtration $\left\{ F_p C_i \right\}_{p \in \mathbb{Z}}$ 
    of each $C_i$ such that the differential preserves
    the filtration, i.e., s.t. 
    $\partial \left( F_p C_i \right) \subset F_p C_{i-1}$.
\end{definition}

Note that we, in particular, obtain an
induced differential
$\partial \colon G_p C_i \to G_{p} C_{i-1}$ by
the universal property of cokernels

\begin{equation*}
\begin{tikzcd}
    F_p C_i \ar[d] \ar[r, "\partial"] & F_p C_{i-1} \ar[d] \\
    F_{p-1} C_i \ar[d, "\coker"]
    \ar[r, "\partial"] & F_{p-1} C_{i-1} \ar[d, "\coker"] \\
    G_p C_i \ar[r, "\partial", dashed] & G_p C_{i-1}
\end{tikzcd}
\end{equation*}
so we obtain an associated graded
chain complex $G_p C_*$.\\
\linebreak
The filtration on $C_*$ also induces
a filtration on the homology of $C_*$ by

\[
F_p H_i (C_*) = 
\left\{ \alpha \in 
H_i (C_*)  \mid 
\left( \exists x \in F_p C_i \right) \colon
\alpha = \left[ x \right] \right\} .
\] 

This filtration has associated graded pieces
$G_p H_i (C_*)$ which, in favorable cases, determine
$H_i (C_*)$.\\
\linebreak



\subsection{Example}

Suppose we have a chain complex
$C_*$ and a filtration consisting
of a single $F_0 C_*$, so
$F_n C_* = 0$ if
$n <0$ and
$F_n C_* = F_0 C_*$ if $n\ge 0$.

Then
$G_n C_* = 0$ for
$n \neq  0$ and
$G_0 C_* = F_0 C_*$ and



























    %\printbibliography
\end{document}
