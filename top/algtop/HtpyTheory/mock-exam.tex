\documentclass[reqno]{amsart}
\usepackage{amscd, amssymb, amsmath, amsthm}
\usepackage{graphicx}
\usepackage[colorlinks=true,linkcolor=blue]{hyperref}
\usepackage[utf8]{inputenc}
\usepackage[T1]{fontenc}
\usepackage{textcomp}
\usepackage{babel}
%% for identity function 1:
\usepackage{bbm}
%%For category theory diagrams:
\usepackage{tikz-cd}

%\usepackage[backend=biber]{biblatex}
%\addbibresource{.bib}


\setlength\parindent{0pt}

\pdfsuppresswarningpagegroup=1

\newtheorem{theorem}{Theorem}[section]
\newtheorem{lemma}[theorem]{Lemma}
\newtheorem{proposition}[theorem]{Proposition}
\newtheorem{corollary}[theorem]{Corollary}
\newtheorem{conjecture}[theorem]{Conjecture}

\theoremstyle{definition}
\newtheorem{definition}[theorem]{Definition}
\newtheorem{example}[theorem]{Example}
\newtheorem{exercise}[theorem]{Exercise}
\newtheorem{problem}[theorem]{Problem}
\newtheorem{question}[theorem]{Question}

\theoremstyle{remark}
\newtheorem*{remark}{Remark}
\newtheorem*{note}{Note}
\newtheorem*{solution}{Solution}



%Inequalities
\newcommand{\cycsum}{\sum_{\mathrm{cyc}}}
\newcommand{\symsum}{\sum_{\mathrm{sym}}}
\newcommand{\cycprod}{\prod_{\mathrm{cyc}}}
\newcommand{\symprod}{\prod_{\mathrm{sym}}}

%Linear Algebra

\DeclareMathOperator{\Span}{span}
\DeclareMathOperator{\im}{im}
\DeclareMathOperator{\diag}{diag}
\DeclareMathOperator{\Ker}{Ker}
\DeclareMathOperator{\ob}{ob}
\DeclareMathOperator{\Hom}{Hom}
\DeclareMathOperator{\Mor}{Mor}
\DeclareMathOperator{\sk}{sk}
\DeclareMathOperator{\Vect}{Vect}
\DeclareMathOperator{\Set}{Set}
\DeclareMathOperator{\Group}{Group}
\DeclareMathOperator{\Ring}{Ring}
\DeclareMathOperator{\Ab}{Ab}
\DeclareMathOperator{\Top}{Top}
\DeclareMathOperator{\hTop}{hTop}
\DeclareMathOperator{\Htpy}{Htpy}
\DeclareMathOperator{\Cat}{Cat}
\DeclareMathOperator{\CAT}{CAT}
\DeclareMathOperator{\Cone}{Cone}
\DeclareMathOperator{\dom}{dom}
\DeclareMathOperator{\cod}{cod}
\DeclareMathOperator{\Aut}{Aut}
\DeclareMathOperator{\Mat}{Mat}
\DeclareMathOperator{\Fin}{Fin}
\DeclareMathOperator{\rel}{rel}
\DeclareMathOperator{\Int}{Int}
\DeclareMathOperator{\sgn}{sgn}
\DeclareMathOperator{\Homeo}{Homeo}
\DeclareMathOperator{\SHomeo}{SHomeo}
\DeclareMathOperator{\PSL}{PSL}
\DeclareMathOperator{\Bil}{Bil}
\DeclareMathOperator{\Sym}{Sym}
\DeclareMathOperator{\Skew}{Skew}
\DeclareMathOperator{\Alt}{Alt}
\DeclareMathOperator{\Quad}{Quad}
\DeclareMathOperator{\Sin}{Sin}
\DeclareMathOperator{\Supp}{Supp}
\DeclareMathOperator{\Char}{char}
\DeclareMathOperator{\Teich}{Teich}
\DeclareMathOperator{\GL}{GL}
\DeclareMathOperator{\tr}{tr}
\DeclareMathOperator{\codim}{codim}
\DeclareMathOperator{\coker}{coker}
\DeclareMathOperator{\corank}{corank}
\DeclareMathOperator{\rank}{rank}
\DeclareMathOperator{\Diff}{Diff}
\DeclareMathOperator{\Bun}{Bun}
\DeclareMathOperator{\Sm}{Sm}
\DeclareMathOperator{\Fr}{Fr}
\DeclareMathOperator{\Cob}{Cob}
\DeclareMathOperator{\Ext}{Ext}
\DeclareMathOperator{\Tor}{Tor}
\DeclareMathOperator{\Conf}{Conf}
\DeclareMathOperator{\UConf}{UConf}



%Row operations
\newcommand{\elem}[1]{% elementary operations
\xrightarrow{\substack{#1}}%
}

\newcommand{\lelem}[1]{% elementary operations (left alignment)
\xrightarrow{\begin{subarray}{l}#1\end{subarray}}%
}

%SS
\DeclareMathOperator{\supp}{supp}
\DeclareMathOperator{\Var}{Var}

%NT
\DeclareMathOperator{\ord}{ord}

%Alg
\DeclareMathOperator{\Rad}{Rad}
\DeclareMathOperator{\Jac}{Jac}

%Misc
\newcommand{\SL}{{\mathrm{SL}}}
\newcommand{\mobgp}{{\mathrm{PSL}_2(\mathbb{C})}}
\newcommand{\id}{{\mathrm{id}}}
\newcommand{\MCG}{{\mathrm{MCG}}}
\newcommand{\PMCG}{{\mathrm{PMCG}}}
\newcommand{\SMCG}{{\mathrm{SMCG}}}
\newcommand{\ud}{{\mathrm{d}}}
\newcommand{\Vol}{{\mathrm{Vol}}}
\newcommand{\Area}{{\mathrm{Area}}}
\newcommand{\diam}{{\mathrm{diam}}}
\newcommand{\End}{{\mathrm{End}}}


\newcommand{\reg}{{\mathtt{reg}}}
\newcommand{\geo}{{\mathtt{geo}}}

\newcommand{\tori}{{\mathcal{T}}}
\newcommand{\cpn}{{\mathtt{c}}}
\newcommand{\pat}{{\mathtt{p}}}

\let\Cap\undefined
\newcommand{\Cap}{{\mathcal{C}}ap}
\newcommand{\Push}{{\mathcal{P}}ush}
\newcommand{\Forget}{{\mathcal{F}}orget}




\begin{document}
    \begin{problem}[]
        It can be shown that
        $\pi_{13} (S^{6}) = \mathbb{Z} / 60$.
        Let $X = S^{6} \cup  e^{14}$ be obtained
        by attaching a $14$-cell to a $S^{6}$ along a generator
        in $\pi_{13}\left( S^{6} \right) $.
        \begin{enumerate}
            \item Calculate $\pi_6(X)$.
            \item Calculate $\pi_{13}(X)$
        \end{enumerate}
    \end{problem}

    \begin{solution}
        Attaching a $14$-cell along a generator
        in $\pi_{13}(S^{6}) = \mathbb{Z} /60$ trivializes
        this homotopy group, so
        $\pi_{13}\left( X \right) \cong 0$.

        By cellular approximation, 
        $\pi_i (X) = 0$ for $i < 6$, so by Hurewicz,
        $\pi_6(X) \cong H_6(X)$, and it is clear that
        $H_6(X) \cong \mathbb{Z}$ by cellular homology.
    \end{solution}

    \begin{problem}[]
        Compute $H_* \left( \Omega \left( S^3
        \vee S^3 \right)  \right) $.
    \end{problem}

    \begin{solution}
        We have the homotopy fibration
        $\Omega \left( S^3 \vee S^3 \right) 
        \to P \left( S^3 \vee S^3 \right) \to 
        S^3 \vee S^3$, since the base space is simply-connected,
        we obtain the following by LSSS:
\[\begin{tikzcd}
	\vdots \\
	{H_3(\Omega (S^3 \vee S^3))} && 0 \\
	{H_2(\Omega (S^3 \vee S^3))} && {(\mathbb{Z} \oplus \mathbb{Z}) \otimes (\mathbb{Z} \oplus \mathbb{Z}) \cong \mathbb{Z}^4} \\
	{0 \cong H_1(\Omega (S^3 \vee S^3))} && 0 \\
	{\mathbb{Z}} && {\mathbb{Z} \oplus \mathbb{Z}} & \cdots
	\arrow[no head, from=2-1, to=1-1]
	\arrow[no head, from=3-1, to=2-1]
	\arrow["\cong", from=3-3, to=1-1]
	\arrow[no head, from=4-1, to=3-1]
	\arrow["\cong", from=4-3, to=2-1]
	\arrow[no head, from=5-1, to=4-1]
	\arrow[no head, from=5-1, to=5-3]
	\arrow["\cong"{description}, from=5-3, to=3-1]
	\arrow[no head, from=5-3, to=5-4]
\end{tikzcd}\]
from which we can deduce that


\[
H_{n} \left( \Omega \left( S^3 \vee S^3 \right)  \right) 
\cong
\begin{cases}
    \mathbb{Z}^{n},& n \text{ even}\\
    0,& n \text{ odd}
\end{cases}
\] 

    \end{solution}


    \begin{problem}[]
        True or false? Briefly justify your answer:
        \begin{enumerate}
            \item If an $n$-dimensional CW complex is $n$-connected,
                then it is contractible.
            \item If $X$ is a simply connected CW complex
                with finitely many cells, then
                $\Omega X$ is weakly equivalent to a CW complex
                with finitely many cells.
            \item If $X$ is an $n$-truncated CW complex, then
                its suspension $\Sigma X$ is
                $(n+1)$-truncated.
        \end{enumerate}
    \end{problem}

    \begin{solution}
        (a) By the Hurewicz theorem, the first nontrivial
        homology group equals the first nontrivial homotopy
        group, so if $\pi_k$ is nontrivial for some
        $k>n$, then $H_k$ is also nontrivial, but since
        there are no cells of dimension $>n$, we conclude
        that $\pi_k = 0$ for all $k$. Now the inclusion of
        a point induces a weak homotopy equivalence, hence
        it is a homotopy equivalence by Whitehead's theorem, so
        the space is contractible.\\
        \linebreak
        

        (b) False. If this were true, then
        the homology groups
        of $\Omega X$ would be nontrivial only in finitely
        many dimensions, but
        we have just seen that
        $H_n \left( \Omega \left( S^{3} \vee S^{3} \right)  \right) 
        \cong \mathbb{Z}^{n}$ in all even dimensions.\\
        \linebreak
        
        (c) False: $S^{1}$ is $2$-truncated, but
        $S^2$ is not $3$-truncated since
        $\pi_3 \left( S^2 \right) \cong \mathbb{Z}$ (and
        $S^2 = \Sigma S^{1}$ ).
    \end{solution}

    \begin{problem}[]
        Let $X$ be a simply connected space with
        homology
        \[
        H_n(X) \cong \mathbb{Z}/n \quad \text{for all }
        n\ge 1.
        \] 
        \begin{enumerate}
            \item Show that $\pi_k(X)$ is finite for all
                $k$.
            \item Show that there are infinitely many
                $k \in \mathbb{N} $ with $\pi_k(X) \not \cong
                0$.
        \end{enumerate}
    \end{problem}

    \begin{proof}
        (1) Let $P$ be the set of all primes.
        Then $\mathcal{F}_P$ is the set of all
        finite abelian groups.
        Now by theorem 1.7 in Hatcher's spectral sequences text,
        since $X$ is simply-connected and
        $H_n(X) \in \mathcal{C} = \mathcal{F}_p$ for all
        $n>0$, we have $\pi_n(X) \in \mathcal{F}_p$ for all
        $n>0$.\\
        \linebreak
        (2) Suppose for contradiction that
        there are only finitely many nontrivial homotopy
        groups. Since these are also finite abelian, this
        means that there is a maximal finite collection $P$ of primes
        such that each $p \in P$ divides the order of some
        nontrivial homotopy group of $X$.
        Let now $p \not\in P$. Then
        by assumption,
        $\pi_k \in \mathcal{C} = \mathcal{F}_P$ for
        all $k < p$, so
        the Hurewicz homomorphism
        $h \colon \pi_p (X) \to H_p(X) \cong \mathbb{Z} / p$ is
        an isomorphism mod $\mathcal{C} = \mathcal{F}_P$.
        Hence the cokernel is in $\mathcal{F}_P$. But
        $\mathbb{Z} / p \not\in \mathcal{F}_P$ by assumption, so
        the image of $h$ must be all of $\mathbb{Z} / p$.
        But then by the first isomorphism theorem, we obtain
        $\pi_p(X) / \ker h \cong \mathbb{Z} / p$, so in particular,
        \[
        \left| \pi_p(X) \right| \cong 
        \left| \ker h \right| \left| \mathbb{Z}/p \right| 
        = p \left| \ker h \right| > 1
        \] 
        so $\pi_p (X)$ is nontrivial, and also has
        order a multiple of $p$, so
        $p \in P$, which is a contradiction.



        

    \end{proof}

    \begin{problem}[Whitehead tower computation ]
        Let $\mathbb{R}\mathbb{P}^{1} \subset 
        \mathbb{R}\mathbb{P}^{\infty}$ be the inclusion. Define
        $X = \mathbb{R}\mathbb{P}^{\infty} / 
        \mathbb{R}\mathbb{P}^{1}$ by collapsing
        $\mathbb{R}\mathbb{P}^{1}$ to a point.
        \begin{enumerate}
            \item Show that $X$ is simply connected,
                that
                \[
                H^{*}(X;\mathbb{Z}) \cong
                \begin{cases}
                    \mathbb{Z},& *=0,2\\
                    \mathbb{Z}/2,& *>3 \text{ even}\\
                    0,& \text{otherwise}
                \end{cases}
                \] 
                and that as a graded ring,
                $H^{*}(X;\mathbb{Z}) \cong
                \mathbb{Z} \left[ a \right] / 2a^2$ with
                $\left| a \right| = 2$.
            \item Show that there is a homotopy fiber sequence
                \[
                S^{1} \to \tau_{>2}X \to X
                \] 
                and use the cohomological Leray-Serre spectral
                sequence to compute $H^{*}\left( \tau_{>2}X;
                \mathbb{Z}\right) $.
            \item Show that $\tau_{>2} X$ is homotopy equivalent
                to a finite CW complex.
            \item Is there a finite CW complex with the same
                homotopy groups as $X$? Briefly justify your
                answer.
        \end{enumerate}
    \end{problem}

    \begin{solution}
        (1) Since $\mathbb{R}\mathbb{P}^{1}$ is a subcomplex
        of $\mathbb{R}\mathbb{P}^{\infty}$, we can use the LES
        associated to this pair.
        In homotopy groups, we get
        \[
            \underbrace{
            \pi_1\left( \mathbb{R}\mathbb{P}^{1} \right)}_{
        \cong 0} \to 
        \underbrace{\pi_1\left( \mathbb{R}\mathbb{P}^{\infty} \right)
        }_{\cong 0}
        \to \pi_1 \left( \mathbb{R}\mathbb{P}^{\infty},
        \mathbb{R}\mathbb{P}^{1}\right) \to 
        \underbrace{\pi_0 \left( 
        \mathbb{R}\mathbb{P}^{1}\right)}_{\cong 0}
        \] 
        hence we find
        $0 \cong \pi_1 \left( \mathbb{R}\mathbb{P}^{\infty},
        \mathbb{R}\mathbb{P}^{1}\right) 
        \cong \pi_1 \left( 
        \mathbb{R}\mathbb{P}^{\infty} / 
    \mathbb{R}\mathbb{P}^{1}\right) $.
    Next, we again get
    $H^{n}\left( \mathbb{R}\mathbb{P}^{\infty},
    \mathbb{R}\mathbb{P}^{1}\right) 
    \cong H^{n}(\mathbb{R}\mathbb{P}^{\infty}/
    \mathbb{R}\mathbb{P}^{1})$.

    Recall that
    \[
    H^{*}\left( \mathbb{R}\mathbb{P}^{\infty} \right) 
    \cong \mathbb{Z} \left[ \alpha \right] / (2\alpha)
    \] 
    where $\left| \alpha \right| = 2$, and
    \[
    H^{n}(\mathbb{R}\mathbb{P}^{1})
    \cong 
    \begin{cases}
        \mathbb{Z},& n=0,1\\
        0,& \text{else}
    \end{cases}
    \] 

    Now, $X = \mathbb{R}\mathbb{P}^{\infty} / \mathbb{R}\mathbb{P}^{1}$ 
    has the cell structure of $\mathbb{R}\mathbb{P}^{\infty}$ 
    with one cell in each dimension, but with the $1$-cell collapsed,
    so the singular chain complex becomes
    \[
        \ldots \to \mathbb{Z} \stackrel{2}{\to} 
        \mathbb{Z} \stackrel{0}{\to} \mathbb{Z}
        \stackrel{2}{\to} \stackrel{0}{\to} \ldots
        \stackrel{2}{\to} \mathbb{Z}
        \stackrel{0}{\to} \mathbb{Z} 
        \to 0 \to \mathbb{Z}
    \] 
    where the rightmost $\mathbb{Z}$ is in degree
    $0$.
    Dualizing, we get that, since
    $\Hom \left( \mathbb{Z},\mathbb{Z} \right) 
    \cong \mathbb{Z}$, and a map
    $f \colon \mathbb{Z} \to \mathbb{Z}$ induces the
    map $f^{*} \colon \mathbb{Z} \to \mathbb{Z}$, we obtain
    \[
    \mathbb{Z} \to 0 \to \mathbb{Z} \stackrel{0}{\to} 
    \mathbb{Z} \stackrel{2}{\to} \mathbb{Z}
    \stackrel{0}{\to} \mathbb{Z}
    \stackrel{0}{\to} \mathbb{Z} \stackrel{2}{\to} \ldots
    \] 
    so
    \[
    H^{n}\left( X;\mathbb{Z} \right) 
    \cong
    \begin{cases}
        \mathbb{Z},& n=0,2\\
        \mathbb{Z}/2,& n>2 \text{ even}\\
        0,& \text{else}
    \end{cases}
    \] 
    The quotient map
    $\varphi \colon \mathbb{R}\mathbb{P}^{\infty}
    \to X$ induces a map
    $\varphi^{*} \colon
    H^{*}(X;\mathbb{Z}) \to 
    H^{*}\left( \mathbb{R}\mathbb{P}^{\infty};
    \mathbb{Z}\right) 
    \cong \mathbb{Z}\left[ \alpha \right] 
    / \left( 2 \alpha \right) $ which is an isomorphism in degrees
    $\neq 2$ as it is the identity outside this degree.
    In degree $2$, we want to show that
    it maps the generator $1$ for
    $\mathbb{Z} = H^2 \left( X ; \mathbb{Z} \right) $ 
    to a generator
    $a \in H^2 \left( \mathbb{R}\mathbb{P}^\infty
    ; \mathbb{Z}\right) \cong
    \mathbb{Z} / 2$. But the map
    $H^{*}(X; \mathbb{Z}) \to 
    H^{*}\left( \mathbb{R}\mathbb{P}^{\infty};
    \mathbb{Z}\right) $ is the induces
    map from the LES of the pair
    $\left( \mathbb{R}\mathbb{P}^{\infty},
    \mathbb{R}\mathbb{P}^{1} \right) $.
    Note that
    \[
    H^{*}\left( \mathbb{R}\mathbb{P}^{1};\mathbb{Z} \right) 
    \cong
    \mathbb{Z}\left[ y \right] /
    \left( y^2 \right) 
    \] 
    with $\left| y \right| =1$.
    Thus
    \[
    0 \to \underbrace{
    H^1 \left( \mathbb{R}\mathbb{P}^{1};\mathbb{Z} \right)}_{\cong
\mathbb{Z} y}
    \to H^2 (X;\mathbb{Z}) 
    \to H^2 (\mathbb{R}\mathbb{P}^{\infty};\mathbb{Z})
    \to \underbrace{H^2 \left( \mathbb{R}\mathbb{P}^{1} \right)}_{
    \cong 0}
    \] 
    is exact, so
    so $\varphi^{*}$ maps
    the image of $y$ to $0$.
    So we obtain
    \[
    H^2 (X;\mathbb{Z}) 
    \] 

    On chain complexes, we have
    \begin{equation*}
    \begin{tikzcd}
        \mathbb{Z} \ar[r] \ar[d, "\varphi^{*} = \id"]
        & 0 \ar[r] \ar[d, "\varphi^{*} = 0"] & \mathbb{Z} \ar[r, "0"]
        \ar[d, "\varphi^{*} = \id"] &
        \mathbb{Z} \ar[r, "2"] \ar[d, "\varphi^{*} = \id"]
                              & \mathbb{Z}\ar[r, "0"] 
        \ar[d, "\varphi^{*} = \id"]&
        \cdots \\
        \mathbb{Z} \ar[r, "0"] & \mathbb{Z} \ar[r, "2"] &
        \mathbb{Z} \ar[r, "0"] & \mathbb{Z} \ar[r, "2"] &
        \mathbb{Z} \ar[r, "0"] & \cdots
    \end{tikzcd}
    \end{equation*}
    Thus in cohomology, the map in degree $2$ becomes
    the map
    $\mathbb{Z} \stackrel{1 \mapsto 1}{\to} 
    \mathbb{Z} / 2$.
    So letting $x$ be a generator for
    $H^2 (X; \mathbb{Z})$, we get
    $\varphi^{*}(x) = a \in 
    \mathbb{Z} \left[ a \right] / (2a)$.
    Hence
    $\varphi^{*}(x^{n}) = a^{n}$ which is a generator
    of $H^{2n}\left( \mathbb{R}\mathbb{P}^{\infty};
    \mathbb{Z}\right) $, so we
    find that
    \[
    H^{*}(X;\mathbb{Z}) \cong
    \mathbb{Z} \left[ x \right] / \left( 2x^2 \right) 
    \] 

    
    (2) In general, we have a homotopy fiber sequence

    \[
    K\left( \pi_2(X), 1 \right) \to 
    \tau_{>2}X \to \tau_{>1}X \cong X.
    \] 
    Now, $H_1(X)\cong ab (\pi_1(X)) \cong 0$ by construction,
    and so
    \[
        0 \to \underbrace{\Ext (H_1(X),\mathbb{Z})}_{\cong 0} \to 
    \mathbb{Z} \to 
    \Hom \left( H_2 (X),\mathbb{Z} \right) \to 0
    \] 
    and since $\Ext \left( H_2 (X) ,\mathbb{Z} \right) 
    \cong 0$, $H_2(X)$ has no torsion,
    so $H_2(X) \cong \mathbb{Z}$ which is the first
    nontrivial homology group, so
    $\pi_2 (X) \cong H_2(X) \cong \mathbb{Z}$.
    Thus $K\left( \pi_2(X),1 \right) 
    = K\left( \mathbb{Z},1 \right) \cong
    S^{1}$.\\
    The cohomological Leray-Serre spectral sequence
    now gives
    
    \[\begin{tikzcd}
	\vdots \\
	{\mathbb{Z}a} && {\mathbb{Z}ax} && {\mathbb{Z}/2ax^2} && {\mathbb{Z}/2ax^3} \\
	{\mathbb{Z}} && {\mathbb{Z}x} && {\mathbb{Z}/2x^2} && {\mathbb{Z}/2x^3} \\
	&& 2 && 4 && 6
	\arrow[no head, from=2-1, to=1-1]
	\arrow[from=2-1, to=3-3]
	\arrow[from=2-3, to=3-5]
	\arrow[from=2-5, to=3-7]
	\arrow[no head, from=3-1, to=2-1]
	\arrow[no head, from=3-1, to=3-3]
	\arrow[no head, from=3-3, to=3-5]
	\arrow[no head, from=3-5, to=3-7]
\end{tikzcd}\]
   Now, the homotopy group of
   $\tau_{>2}X$ is trivial in dimension $1$ and $2$, so
   the map
   $\mathbb{Z}a \to \mathbb{Z} x$ must be an isomorphism.
   Furthermore, we may assume it maps
   $a$ to $x$ by a change of sign.
   Now we get
   $d\left( ax \right) =
   d(a)x + a d(x) = x^2$. Hence the map
   $\mathbb{Z} ax \to \mathbb{Z}/2 x^2$ is a surjection, so
   the kernel is $2\mathbb{Z} \cong
   \mathbb{Z}$. Likewise,
   $d\left( ax^{k} \right) 
   = x^{k+1} + a d(x^{k})$ and
   $d(x^{k}) = x d(x^{k-1}) = \ldots
   = x^{k-1} d(x) = 0$, so
   $d(ax^{k}) = x^{k+1}$. Hence all the maps are surjective, and
   in the cases where we have maps
   $\mathbb{Z}/2 \to \mathbb{Z}/2$, it is therefore an isomorphism.
   Thus,
   \[
   H^{k}(\tau_{>2}X) \cong
   \begin{cases}
       \mathbb{Z},& k=0,3\\
       0,& \text{else}
   \end{cases}
   \] 

   (3) By UCT,
   \[
   H_k \left( \tau_{>2}X \right) \cong
   \begin{cases}
       \mathbb{Z},& k=0,3\\
       0,& \text{else}
   \end{cases}
   \] 
   so by uniqueness of Moore spaces,
   $\tau_{>2}X \simeq S^3$.\\
   \linebreak
   (4) No, a finite CW complex can only have finitely
   many nonzero cohomology groups, while
   $X$ has infinitely many.














    \end{solution}









    %\printbibliography
\end{document}
