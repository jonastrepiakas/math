\section{Homotopy Groups}

\subsection{Homotopy}
We follow chapter 14 of \cite{Bredon} for this subsection.\\

To start of, we recall the basic definitions of homotopies.

\begin{definition}[Homotopy]
    Two maps $f_0, f_1 \colon X \to Y$ are said to
    be \textit{homotopic} if there exists a homotopy
    $F \colon X \times I \to Y$ such that
    $F(x,0) = f_0(x)$ and $F(x,1) = f_1(x)$ for
    all $x \in X$.
\end{definition}

\begin{definition}[Homotopy equivalence]
    A map $f \colon X \to Y$ is said to be a \textit{homotopy
    equivalence} if it is an isomorphism in
    $\hTop$.
\end{definition}

\begin{lemma}[Reparametrization Lemma]
    Let $\varphi_1, \varphi_2$ be maps
    $\left( I, \partial I \right) \to 
    \left( I, \partial I \right) $ which are equal on
    $\partial I$. Let
    $F \colon X \times I \to Y$ be a homotopy and let
    $G_i (x,t) = F\left( x, \varphi_i(t) \right) $ for
    $i = 1,2$. Then $G_1 \simeq G_2 \rel
    X \times \partial I$.
\end{lemma}

We shall use $c$ to denote the constant homotopy.

\begin{proposition}[]
    $F * c \simeq F \rel X \times \partial I$ and
    $c * F \simeq F \rel X \times \partial I$.
\end{proposition}

\begin{definition}[]
    If $F \colon X \times I \to Y$ is a homotopy, then we
    define $F^{-1} \colon X \times I \to Y$ by
    $F^{-1}\left( x,t \right) = F(x,1-t)$. 
\end{definition}

Note that $F^{-1}$ is precisely the inverse
to $F$ in $\hTop$.

\begin{proposition}[]
    For any homotopies $F,G,H$ for which the
    concatenations 
     are defined, we have
     \[
         \left( F * G  \right) * H
         \simeq F * \left( G * H \right) 
         \rel X \times \partial I.
     \] 
\end{proposition}


\begin{proposition}[]
    For homotopies $F_1, F_2, G_1, G_2$,
    if $F_1 \simeq F_2 \rel X \times \partial I$ and
    $G_1 \simeq G_2 \rel X \times \partial I$, then
    $F_1 * G_1 \simeq F_2 * G_2 \rel X \times \partial I$.
\end{proposition}

Note that all of the discussion of concatenation of
homotopies goes through with no difficulties for the cases
in which all homotopies are relative to some subspace
$A \subset X$ or are homotopies of pairs
$\left( X, A  \right) \to \left( Y, B \right) $.\\
It follows that homotopy between maps of
pairs $\left( X,A \right) \to \left( Y,B \right) $ is
an equivalence relation. The set of homotopy classes
of these maps is commonly denoted by
$\left[ X,A ; Y ,B \right] $ or just
$\left[ X;Y \right] $ if $A = \varnothing$.

\begin{theorem}[]\label{Thm:299221}
    If $f_0 \simeq f_1 \colon X \to Y$ then
    $M_{f_0} \simeq M_{f_1} \rel
    X + Y$ and
    $C_{f_0} \simeq C_{f_1} \rel
    Y + \text{vertex}$.
\end{theorem}


To show this, one needs the following basic topological
proposition:
\begin{proposition}[] \label{prop:92031999}
    If $f \colon X \to Y$ is a quotient map and
    $K$ is locally compact Hausdorff, then
    $f \times 1 \colon X \times K \to Y \times K$ is
    a quotient map.
\end{proposition}

\begin{proof}[Proof of Theorem \ref{Thm:299221}]
    First, let $F \colon X \times I \to Y$ be the homotopy
    between $f_0$ and $f_1$. Now define $h \colon
    M_{f_0} \to M_{f_1}$ by $h(y) = y$ for
    $y \in Y$ and
    \[
    h\left( x,t \right) = 
    \begin{cases}
        F\left( x,2t \right) ,& t\le \frac{1}{2}\\
        (x, 2t-1),& \frac{1}{2} \le t.
    \end{cases}
    \] 
    Define
    $k \colon M_{f_1} \to M_{f_0}$ likewise by
    the identity on $Y$ nad
    \[
    k\left( x,t \right) =
    \begin{cases}
        F^{-1}\left( x,2t \right) ,& t\le \frac{1}{2}\\
        (x,2t-1),& \frac{1}{2}\le t
    \end{cases}.
    \] 
    Then the composition
    $kh \colon M_{f_0} \to M_{f_1}$ is the identity
    on $Y$ and 
    $F * \left( F^{-1} * E \right) $ on
    the cylinder portion, where $E \colon X \times I \to 
    M_{f_0}$ is induced by the identity on
    $X \times I \to X \times I$.
    This is homotopic to the identity 
    $\rel X \times \left\{ 1 \right\} + Y$.
    Similarly for $hk$.
    In now remains to check the continuity of this homotopy.
    We have a homotopy $M_{f_0} \times I \to 
    M_{f_0}$. We now claim that
    $M_{f_0} \times I \cong M_{f_0 \times I}$. Indeed
    then, using that
    $M_{f_0 \times I} = 
    \frac{X \times I \times I \sqcup Y \times I}{
    \left( (x,0,k) \sim (f_0(x),k \right) }$, it suffices
    to show continuity of the composition
    $X \times I \times I \sqcup Y \times I
    \to M_{f_0} \times I \to M_{f_0}$. 
    For on $Y \times I$, it is the constant homotopy and
    on $X \times I \times I$ it is
    $F * \left( F^{-1} * E \right) \simeq E
    \rel X \times \partial I$. 
    Now, that $M _{f_0} \times I
    \cong M_{f_0 \times I}$ follows from
    Proposition \ref{prop:92031999}.

\end{proof}

Let $f \colon X \to Y$. If $\varphi  \colon Y \to Y'$ is a map,
then there is the induced map
$F \colon M_{f} \to M_{\varphi \circ f}$ induced from
$\varphi $ on $Y$ and the identity on $X \times I$.

\begin{theorem}[]
    If $\varphi  \colon Y \to Y'$ is a homotopy equivalence
    then so is 
    $F \colon \left( M_f , X \right) \to 
    \left( M_{\varphi  \circ f}, X \right) $ and hence
    so is $F \colon C_f \to C_{\varphi  \circ f}$.
\end{theorem}

\begin{proof}
    Let $\psi  \colon Y' \to Y$ be a homotopy inverse
    of $\varphi $ and let $G \colon 
    M_{\varphi \circ f} \to M_{\psi \circ
    \varphi \circ f}$ be the map induced by
    $\psi $ on $Y'$ and the identity on $X \times I$.
    The composition $GF \colon M_f \to M_{\psi \circ \varphi 
    \circ f}$ is induced from $\psi \circ \varphi \colon
    Y \to Y$ and the identity on $X \times I$.
    Let $H \colon Y \times I \to Y$ be a homotopy from
    $\id$ to $\psi \circ \varphi $ ; i.e.,
    $H(y,0) = y$ and $H(y,1) = 
    \psi \varphi (y)$. 
    By the proof of Theorem \ref{Thm:299221}, there is a
    homotopy equivalence
    $h \colon M_f \to M_{\psi \circ \varphi \circ f}\rel X$ given
    by $h(y) = y$ and
     \[
    h(x,t) = 
    \begin{cases}
        H\left( f(x), 2t \right) ,& t\le \frac{1}{2}\\
        (x,2t-1),& t\ge \frac{1}{2}
    \end{cases}.
    \] 
    We claim that
    $h \simeq GF \rel X$. Indeed, the homotopy $H$ can
    be extended to 
    $M_f \times I \to M_{\psi \circ \varphi \circ f}$ by
    putting
    \[
    H\left( (x,s),t \right) 
    =
    \begin{cases}
        H\left( f(x), 2s+t \right) ,& 2s+t \le 1\\
        \left( x, \frac{2s+t-1}{t+1} \right) ,& 2s+t\ge 1
    \end{cases}.
    \] 

    Then $H\left( -,0 \right) = h$ and
    $H\left( -,1 \right) = GF$, so
    since $GF$ is a homotopy equvalence, so is
    $h$.
    Define $F' \colon
    M_{\psi \circ \varphi \circ f} \to 
    M_{\varphi \circ \psi \circ \varphi \circ f}$ 
    as the induced map on mapping cones
    with $\varphi $ on $Y$ and
    the identity on $X \times I$. Then similarly,
    $F' G$ is a homotopy equivalence.\\
    If $k$ is a homotopy inverse of $GF$ then
    $GF k \simeq \id$. If
    $k'$ is a homotopy inverse of $F'G$ then
    $k' F' G \simeq \id$. Thus $G$ has a right
    and left homotopy inverse: $R = Fk$ and
    $L = k'F'$. Then
    $R = \id \circ R \simeq 
    \left( LG \right) R =
    L \left( GR \right) \simeq L \circ \id = L$, so
    $R \simeq L$. That is, 
    $G$ has a homotopy inverse. Therefore,
    $G$ is a homotopy equivalence. Since $G$ and $GF$ are
    homotopy equivalences, so is $F$.
\end{proof}


\begin{problem}[]
    \cite[Ex 14.1]{Bredon} Let $S^2 \cup A$ denote the
    union of the unit $2$-sphere and the line segment
    joining the north and south poles. Show that
    $S^2 \vee S^{1} \simeq
    S^2 \cup A$.
\end{problem}

\begin{proof}
    Define two maps
    $f_0,f_1 \colon \left\{ 0,1 \right\}  \to 
    S^2$ where
    $f_0 (t) = \left( \cos (2\pi t), \sin(2\pi t), 0 \right) $ 
    and $f_1$ is the constant map at $(1,0,0)$. Then
    $f_0 \simeq f_1$, so $C_{f_0} \simeq C_{f_1}$. Now,
    $C_{f_0} = S^2 \cup  A$ while
    $C_{f_1} = S^2 \vee S^{1}$.
\end{proof}

\begin{problem}[]
    \cite[Ex 14.2]{Bredon} 
    Show that the union of a $2$-sphere and a flat
    unit  $2$-cell through the origin is homotopically
    equivalent to the one-point union of two $2$-spheres.
\end{problem}

\begin{proof}
    A $2$-cell is contractible, an
    a $2$-sphere with a $2$-cell inside it is precisely the
    cone of the map
    $S^1 \sqcup S^1 \to S^1$ with the identity on both.
    By \cite[Thm 14.19]{Bredon},
    this is homotopy equivalent to the cone on
    $S^1 \sqcup S^1 \to \left\{ * \right\} $ which is
    $S^2 \vee S^2$.
\end{proof}

\begin{problem}[]
    Show that the union of a standard $2$-torus with two disks,
    one spanning a latitudinal circle and the
    other spanning a longitudinal circle of the torus, is
    homotopically equivalent to a $2$-sphere.
\end{problem}

\begin{proof}
     Using the identification of the torus as the
     quotient space of $I^2$ in the usual way, we can choose
     on spanning circle to be a $2$-cell attached
     along $\left\{ 0 \right\} \times I$ and the
     other to be a $2$-cell attached along
     $I \times \left\{ 0 \right\} $. These are contractible, 
     and the quotient space becomes a $2$-sphere.
\end{proof}


\subsection{Homotopy Groups}

Recall that $\left[ X,A ; Y ,B \right] $ denotes the
set of homotopy classes of maps $X \to Y$ carrying $A$ into
$B$ such that $A$ goes into $B$ during the entire homotopy.

To make a group then, we can select a point $y_0 \in Y$ and
consider the set
\[
\left[ X \times I, X \times \partial I ;
Y , \left\{ y_0 \right\} \right] 
\] 
In this case, the operation of concatenation of homotopies
makes this set into a group.
It is technically also better to choose a basepoint 
$x_0 \in X$ and consider
\[
\left[ X \times I, \left\{ x_0 \right\} \times I
\cup X \times \partial I ; Y , \left\{ y_0 \right\} \right] .
\] 

For the moment, let us set
$A = \left\{ x_0 \right\} \times I \cup 
X \times \partial I$. Then maps
$X \times I \to Y$ which carry $A$ into $\left\{ y_0 \right\} $ 
are in bijective correspondence with maps 
$\left( X \times I \right) / A \to Y$ which take
 the point $\left\{ A \right\} $ into 
 $\left\{ y_0 \right\} $. 
 
 \begin{definition}[Reduced Suspension]
     We define the \textit{reduced suspension} of
     $X$ to be
     \[
     SX = (X \times I) / A =
     \left( X \times I \right) /
     \left( \left\{ x_0 \right\} \times I
     \cup X \times \partial I \right) 
     \] 
 \end{definition}

 The set of homotopy classes of pointed maps
 of a pointed space $X$ to a pointed space $Y$ with
 homotopies preserving the base points will
 be denoted by $\left[ X;Y \right]_* $. 

 Thus
 $\left[ SX;Y \right]_* $ is in canonical bijective
 correspondence with
 $\left[ X \times I, A ; Y , \left\{ y_0 \right\}  \right] $.


 Now, suppose we have pointed maps
 $f,g \colon SX \to Y$. Then they
 induce homotopies
 $f',g' \colon X \times I \to Y$ by precomposing with the
 quotient map
  $X\times I \to SX$. We can then define
  $f' * g' \colon X \times I \to Y$ as usual.
  The resulting pointed map
  $SX \to Y$ will be denoted $f * g$.
  Geometrically, $f * g$ is obtained by
  putting $f$ on the bottom and $g$ on the top
  of the one-point union $SX \vee SX$ and composing
  the resulting map $SX \vee SX \to Y$ with the
  map $SX \to SX \vee SX$ obtained by collapsing the
  middle parameter value $\frac{1}{2}$ copy of
  $X$ in $SX$ to the base point.
  
  \begin{figure}[htpb]
      \centering
      \includegraphics[width=0.6\textwidth]{Figures/TKISO0932.png}
      \caption{The product of two map classes
      $SX \to Y$.}
      \label{fig:TKISO0932-png}
  \end{figure}


  For a map $f \colon \left( SX, \left\{ A \right\}  \right) 
  \to \left( Y, \left\{ y_0 \right\}  \right) $, we denote its
  homotopy class in
  $\left[ SX; Y \right]_{*}$ by
  $\left[ f \right] $, and we define
  \[
  \left[ f \right] \left[ g \right] =
  \left[ f*g\right] 
  \] 
  Under this operation, the set
  $\left[ SX;Y \right]_*$ becomes a group.

  \begin{proposition}[]
      The reduced suspension gives
      $S S^{n-1}\cong S^{n}$.
  \end{proposition}

  Thus, we can define $S^{n}$ as the $n$-fold reduced
  suspension of $S^{0}$. As a special case,
  the set $\left[ S^{n};Y \right]_*$ then becomes
  a group for $n>0$. 

  \begin{definition}[$n$ th homotopy group]
      We define
      \[
      \pi_n \left( Y, y_0 \right) =
      \left[ S^{n}; Y \right]_*
      \] 
      with this operation.
  \end{definition}

  \subsection{Homotopy Groups using H-Spaces/Groups/Cogroups}

  From now on, unless otherwise indicated, we regard
  the $n$-sphere $S^{n}$ as having the cogroup
  structure as the reduced suspension
  $S^{n} = S S^{n-1} = S^{n-1} \wedge S^{1}$ - i.e.,
  the map $\gamma$ in the definition of an H-cogroup will
  be $\gamma \colon S^{n} \to S^{n} \vee S^{n}$ 
  given by
  \[
  \gamma(t,x) = 
  \begin{cases}
      \left( 2t,x \right)_{1},& t\le \frac{1}{2}\\
      \left( 2t-1, x \right)_2,& t\ge \frac{1}{2}.
  \end{cases}
  \] 
  The
  $0$-sphere $S^{0}$ is $\left\{ 0,1 \right\} $ 
  with base point $\left\{ 0 \right\} $.\\

  For a based space $X$ with base point $x_0$, we define
  the nth homotopy group
  \[
  \pi_n (X,x_0) = 
  \left[ S^{n},* ; X, x_0 \right] .
  \] 
  This is a group with the product defined by
  Theorem \ref{Thm:H-group-cogroup}.(2).

  \begin{theorem}[]\label{Theorem:Htpy-Groups-Abelian}
      If $X$ is an H-space then the multiplication
      in $\pi_n (X,x_0)$ is induced by
      the H-space multiplication and is abelian for
      $n\ge 1$.
  \end{theorem}

  \begin{proof}
      This follows directly from Theorem \ref{Thm:H-group-cogroup}.
  \end{proof}

  \begin{lemma}[]\label{Lemma:Suspension-Loop-Space}
      In the pointed category,
      $\left[ SX ; Y \right] \cong
      \left[ X ; \Omega Y \right]$ as groups.
  \end{lemma}
  
  \begin{proof}
      Recall the characteristic correspondence for
      the compact-open topology (Theorem \ref{Thm:Compact-Open-Top}):
      \[
      f\colon X \times S^{1} \to Y 
      \leftrightarrow f' \colon X \to Y^{S^{1}}
      \] 
      given by
      $f'(x) (t) = f(x,t)$.
      Recall that $SX = X \wedge S^{1}$, so
      a map
      $g \colon X \wedge S^{1} \to Y$ induces a map
      $f \colon X \times S^{1} \to Y$ by
      the composition 
      $X \times S^{1} \to X \wedge S^{1} \stackrel{g}{\to} Y$.
      Then $f$ is continuous if and only if
      the map $f' \colon X \to Y^{S^{1}}$ is continuous, where
      $f'(x)(*) = f(x,*) = *$, so
      $f'(*) = * \in Y^{S^{1}}$, the basepoint
      of $\Omega Y$.
      So the correspondence induces a bijective
      correspondence between pointed maps
      $SX \to Y$ and pointed maps $X \to \Omega Y$, and
      pointed homotopies correspond as well.\\
      It remains to show that the correspondence
      is a group homomorphism.
      Recall that $SX$ is an H-cogroup and
      $\Omega X$ is an H-group, so using
      Theorem \ref{Thm:H-group-cogroup}, we
      get that 
      for $f,g \colon SX \to Y$, the product in
      $\left[ SX;Y \right] $ is induced by
      \[
          \left( f*g \right) (x,t) = 
          \begin{cases}
              f(x,2t),& t\le \frac{1}{2}\\
              g\left( x,2t-1 \right),& t\ge \frac{1}{2},
          \end{cases}
      \] 
      which is equal to
      $\left( f*g \right)'(x)(t)$, 
      while the multiplication in
      $\left[ X ; \Omega Y \right] $ is given by
      $\left( f' \cdot g' \right) (x) = 
      f'(x) * g'(x)$ where $*$ is loop concatenation. 
      At time $t$, this is
      $f'(x) (2t)$ for $t\le \frac{1}{2}$ and
      $g'(x) (2t-1)$ for $t\ge \frac{1}{2}$. Thus
      $\left( f*g \right) ' = f' \cdot  g'$.
  \end{proof}


  All of the above immediately carries over to pointed
  pairs $(X,A)$ with a base point in $A$, so
  $\left[ SX, SA ; Y,B \right] $ is a group that
  is canonically isomorphic to
  $\left[ X,A; \Omega Y, \Omega B \right] $.

  Note in particular that
  $D^{n} \cong S D^{n-1}$ for all $n\ge 2$, so
  $D^{n} = D^{1} \wedge S^{n-1} \supset 
  S^{0} \wedge S^{n-1} = S^{n-1}$, so
  $\left( D^{n}, S^{n-1} \right) 
  = S^{n-1} \left( D^{1}, S^{0} \right) $, the
  $(n-1)$-fold reduced suspension. Hence
  we can define the relative homotopy group by
  \[
  \pi_n \left( Y,B,* \right) =
  \left[ D^{n}, S^{n-1}; Y ,B \right] 
  = \left[ S^{n-1}\left( D^{1}, S^{0} \right) ;
  Y,B\right] .
  \] 
  Note that this is defined on pointed spaces and pointed maps,
  so this set is really
  $\left[ D^{n},S^{n-1},s_0; Y,B, * \right] $.
  This becomes a group for $n\ge 2$. 

  Next note that we have a quotienting map
  \[
      I^{n} = \underbrace{D^{1}}_{=I} \times I \times \ldots \times I
      \to D^{1} \wedge S^{1} \wedge \ldots
      \wedge S^{1} =
      D^{n}.
  \] 
  Under this map, 
  $\partial I^{n}$ corresponds to
  $S^{n-1}$ and
  the base point corresponds to
  $J^{n-1} = \left( I \times \partial I^{n-1} \right) 
  \cup  \left( \left\{ 0 \right\} \times I^{n-1} \right) $.
  Thus under this quotienting map, we obtain a
  bijection
  \[
  \pi_n(Y,B,*) =
  \left[ D^{n}, S^{n-1},s_0; Y,B, * \right] 
  \cong \left[ I^{n}, \partial I^{n}, J^{n-1};
  Y,B,*\right].
  \] 

  \begin{corollary}
      $\pi_n(Y,*)$ is abelian for
      $n\ge 2$ and $\pi_n (Y,B,*)$ is abelian for
      $n\ge 3$. Moreover, the group structure is
      independent of the suspension coordinate used
      to define it.
  \end{corollary}

  \begin{proof}
      Recall from Lemma \ref{Lemma:Suspension-Loop-Space}, that
      $\left[ S^{n};Y \right] 
      = \left[ S^{1} \wedge \ldots \wedge S^{1};Y \right] 
      \cong \left[ S^{n-1} ; \Omega Y \right] $ as
      groups. Now
      the loop structure corresponds to the suspension in
      the last coordinate by definition, and
      by Theorem \ref{Thm:H-group-cogroup}, this
      is the same as the group 
      $\left[ S^{n};Y \right] $ with the suspension
      operation on any of the coordinates, since
      the choice of coordinate is arbitrary, this
      shows that
      the group structure on $\pi_n(Y,*) =
      \left[ S S^{n-1};Y \right]
      = \left[ S^{1} \wedge \ldots \wedge
      S^{1}; Y \right] $ is independent
      of the suspension coordinate used to define it.\\
      The product in $\left[ S^{n-1}; \Omega Y \right] $ is
      furthermore abelian for $n-1 \ge 1$ by
      Theorem \ref{Theorem:Htpy-Groups-Abelian} (using that
      $\Omega Y$ is an H-space), and
      the relative case is similar.
  \end{proof}

  \begin{corollary}
      \[
      \pi_n\left( Y,* \right) \cong
      \pi_{n-1}\left( \Omega Y, * \right) 
      \cong \ldots \cong
      \pi_1\left( \Omega^{n-1}Y,* \right) 
      \cong \pi_0 \left( \Omega^{n}Y ,* \right) 
      \] 
      and similarly in the relative case.
  \end{corollary}

  \begin{theorem}[]\label{Thm:Bredon-4.5}
      Let $A$ be a closed subspace of $X$ containing the
      base point $*$. Suppose that $F \colon X \times I \to X$ 
      is a deformation of $X$ contracting $A$ to
      $*$ ; i.e.,
      \begin{align*}
          F(A \times I) 
          &\subset A\\
          F(x,0) 
          &= x\\
          F\left( A \times \left\{ 1 \right\}  \right) 
          &= *\\
          F\left( \left\{ * \right\} \times I \right) 
          &= *
      \end{align*}
      then the quotient map $X \to X /A$ is a homotopy equivalence.
      Similarly for pairs $\left( X,X' \right) $ with
      $A \subset X'$.
  \end{theorem}

  \begin{proof}
      Let
      $\psi \colon X / A \to X$ be defined by the
      commutative diagram
      \begin{equation*}
      \begin{tikzcd}
          X \ar[d, "\varphi "] 
          \ar[dr, "F_{X \times \left\{ 1 \right\} } "]& \\
          X / A \ar[r, "\psi"] & X
      \end{tikzcd}
      \end{equation*}
      We claim that. Let $\varphi \colon X \to X / A$ be
      the quotienting map. We claim that
      $\psi \varphi \simeq \id_{X}$ and
      $\varphi \psi \simeq \id_{X / A}$.
      Since $F\left( A \times I \right) \subset A$, 
      $F$ induces a homotopy
      $F' \colon X / A \times I \to X / A$, where
      $F_{X / A \times \left\{ 1 \right\} }
      = \varphi \psi $, so
      since $F_{X / A \times \left\{ 0 \right\} }
      = \id_{X / A}$, we get the result.
  \end{proof}


  \newpage




  \subsubsection{A different way of defining
  $\pi_n \left( Y, y_0 \right) $}
  Note that reduced suspension supplies a parameter in
  $\left[ 0,1 \right] $ and the space
  $S^{n}$ as constructed is the quotient space of
  $I^{n}$ obtained by collapsing the boundary of the cube to a
  point.
  Pointed maps $S^{n}\to Y$ are in bijective correspondence
  with maps $I^{n}\to Y$ taking $\partial I^{n}$ to
  the base point of $Y$. This is a more traditional way
  of defining $\pi_n(Y)$. This becomes the group
  of homotopy classes of maps
  $\left( I^{n},\partial I^{n} \right) \to 
  \left( Y, \left\{ y_0 \right\}  \right) $ with the
  operation being
  \[
  f*g \left( t_1, \ldots, t_n \right) =
  \begin{cases}
      f\left( 2t_1, t_2, \ldots, t_n \right) ,& t_1 \in 
      \left[ 0,\frac{1}{2} \right] \\
      g\left( 2t_1-1, t_2, \ldots, t_n \right) ,& t_1 \in 
      \left[ \frac{1}{2},1 \right] 
  \end{cases}.
  \] 

  \begin{proposition}[]
      For $n\ge 2$, $\pi_n\left( X, x_0 \right)$ is abelian.
  \end{proposition}

  \begin{proof}
      Consider the homotopy in Figure \ref{fig:JIDWOOL0290L-png}.
      We begin by shrinking the domains of $f$ and $g$ to smaller
      subcubes of $I^{n}$, where the region outside is
      mapped to the basepoint. This allows us to move the boxes
      around in a continuous manner. The rest is clear.
      \begin{figure}[htpb]
          \centering
          \includegraphics[width=0.8\textwidth]{Figures/JIDWOOL0290L.png}
          \caption{The homotopy in question}
          \label{fig:JIDWOOL0290L-png}
      \end{figure}
  \end{proof}

  Next, we want to show that following:
  \begin{proposition}[]\label{Prop:SwjiaKKDNW1102}
      If $X$ is path-connected, then
      $\pi_n\left( X, x_0 \right) \cong
      \pi_n (X, x_1)$ for any two $x_0,x_1 \in X$.
  \end{proposition}

  For this, we introduce an action of
  $\pi_1$ on $\pi_n$.

  \begin{definition}[The action of $\pi_1$ on $\pi_n$]
      Given a path
      $\gamma \colon I \to X$ from
      $x_0$ to $x_1$, we associate to a map
      $f \colon \left( I^{n}, \partial I^{n} \right) \to 
      \left( X, x_1 \right) $ the map
      $\gamma f \colon \left( I^{n}, \partial I^{n} \right) 
      \to \left( X,x_0 \right) $ by shrinking the domain
      of $f$ to a smaller concentric cube in $I^{n}$, then
      inserting the path $\gamma$ on each radial segment
      in the shell between this smaller cube and $\partial
      I^{n}$.
      See Figure \ref{fig:JDWIXHHX011SJ-png}

      \begin{figure}[htpb]
          \centering
          \includegraphics[width=0.25\textwidth]{Figures/JDWIXHHX011SJ.png}
          \caption{Depiction of $\gamma f$.}
          \label{fig:JDWIXHHX011SJ-png}
      \end{figure}

  \begin{note}
      We have the following properties
      \begin{enumerate}
          \item $\gamma \left( f+ g \right) 
              \simeq \gamma f + \gamma g$.
          \item $\left( \gamma \eta \right) f \simeq
              \gamma \left( \eta f \right) $.
          \item $\id f \simeq f$, where
              $\id$ denotes the constant path.
      \end{enumerate}

      To see $(1)$, first deform $f$ and $g$ to be
      constant on the right and left halves of
      $I^{n}$, respectively, producing maps
      which we may call $f+0$ and $0+g$, then we 
      can excise a progressively wider symmetric middle slab
      of $\gamma (f+0) + \gamma(0+g)$ (which can be
      seen on the left in Figure \ref{fig:WIWIWSSK11-png})
      until it becomes $\gamma \left( f+g \right) $ (shown on the
      right).

      \begin{figure}[htpb]
          \centering
          \includegraphics[width=0.8\textwidth]{Figures/WIWIWSSK11.png}
          \caption{}
          \label{fig:WIWIWSSK11-png}
      \end{figure}
  \end{note}

  Now if $\beta_{\gamma} \colon \pi_n(X,x_1) \to 
  \pi_n(X, x_0)$ is the change-of-basepoint transformation,
   $\beta_{\gamma}\left[ f \right] =
   \left[ \gamma f \right] $, then
   the above note shows that $\beta_\gamma$ is a group isomorphism.
   This proves Proposition \ref{Prop:SwjiaKKDNW1102}. 
   If we restrict attention to loops
   $\gamma$ at $x_0$, then since $\beta_{\gamma \eta}=
   \beta_{\gamma} \beta_{\eta}$, the map
   $\left[ \gamma \right] \mapsto \beta_{\gamma}$ 
   defines a homomorphism from
   $\pi_1\left( X, x_0 \right) $ to
   $\Aut \left( \pi_n \left( X,x_0 \right)  \right) $ 
   called the \textit{action of $\pi_1$ on $\pi_n$ }.
  \end{definition}

  \begin{note}
  For $n>1$, this action makes
  $\pi_n(X,x_0)$ into a module over the group ring
  $\mathbb{Z}\left[ \pi_1 \left( X,x_0 \right)  \right] $.
  \end{note}  

  \begin{definition}[Simple/abelian spaces]
      A space with trivial $\pi_1$ action on $\pi_n$ is called
      '$n$-simple', and 'simple' means
      ' $n$-simple for all $n$ '. We call
      a space \textit{abelian} if it has
      trivial action of $\pi_1$ on all homotopy groups
      $\pi_n$.
  \end{definition}

  \begin{proposition}[$\pi_n$ is a functor]
      A map $\varphi  \colon \left( X, x_0 \right) \to 
      \left( Y, y_0 \right) $ induces a map
      $\varphi_* \colon \pi_n \left( X, x_0 \right) \to 
      \pi_n \left( Y, y_0 \right) $ defined by
      $\varphi_* \left[ f \right] = \left[ \varphi  f \right] $.
      It is immediate from the definitions that
      $\varphi_*$ is well-defined and a homomorphism
      for $n\ge 1$. The functorial properties are also clear.
  \end{proposition}

  \begin{corollary}
      Homotopy equivalent spaces have isomorphic
      homotopy groups.
  \end{corollary}

  \begin{proposition}[]
      A covering space projection
      $p \colon \left( \tilde{X}, \tilde{x}_0 \right) \to 
      \left( X, x_0 \right) $ induces isomorphisms
      $p_* \colon \pi_n \left( \tilde{X}, \tilde{x}_0 \right) 
      \to \pi_n \left( X, x_0 \right) $ for all
      $n \ge 2$.
  \end{proposition}

  \begin{proof}
      Since 
      $S^{n}$ is path-connected and locally path-connected,
      and simply connected for $n\ge 2$, we find that
      any map
      $\left( S^{n},s_0 \right) 
      \to \left( X, x_0 \right) $ lifts to a 
      map $\left( S^{n},s_0 \right) \to 
      \left( \tilde{X},\tilde{x}_0 \right) $ when
      $n\ge 2$. This gives surjectivity of
      $p_*$.
      For injectivity, suppose
      $p_* \left[ f \right] = \left[ 0 \right] $ where
      $f \colon \left( S^{n}, s_0 \right) \to 
      \left( \tilde{X},\tilde{x}_0 \right) $.
      Let $c_{\tilde{x}_0}$ be the constant map at
      $\tilde{x}_0$. Then
      $p_* \left[ \tilde{x}_0 \right] =
      \left[ 0 \right] $, so by uniqueness of the
      lifting theorem, 
      $\left[ f \right] = \left[ c_{\tilde{x}_0} \right] =
      \left[ 0 \right] $.
  \end{proof}

  \begin{definition}[Aspherical]
      Spaces with $\pi_n = 0$ for all
      $n\ge 2$ are called \textit{aspherical}.
  \end{definition}

  \begin{corollary}
      $S^{1}, T^{n}$ and $K$ are aspherical since
      they have contractible covering spaces.
  \end{corollary}


  \begin{proposition}[]
      \[
      \pi_n \left( \prod_{\alpha} X_{\alpha} \right) 
      \cong \prod_{\alpha} \pi_n \left( X_{\alpha} \right) 
      \] 
  \end{proposition}

  Next we define relative homotopy groups.

  \begin{definition}[Relative homotopy groups]
      Regard $I^{n-1}$ as a face of $I^{n}$ with the last
      coordinate $s_n = 0$ and let
      $J^{n-1}$ be the closure of
      $\partial I^{n}- I^{n-1}$. Then
      we define 
      \[
      \pi_n \left( X, A, x_0 \right) 
      := \left[ I^{n},\partial I^{n}, J^{n-1};
      X , A , x_0\right] 
      \] 
      We shall leave $\pi_0 \left( X, A, x_0 \right) $ undefined
      for now.
  \end{definition}

  We can define a sum operation on $\pi_n \left( X, A, x_0 \right) $ 
  in the same way as for $\pi_n \left( X, x_0 \right) $, except
  now the coordinate $s_n$ now must remain free, so
  we must use one of the other coordinates. Thus
  we must have at least one other coordinate to define
  the same operation. So $\pi_n \left( X, A, x_0 \right) $ is
  a group for $n\ge 2$, and it is abelian for
  $n\ge 3$. For $n=1$, we have
  $I^{1} = \left[ 0,1 \right] , I^{0} = \left\{ 0 \right\} $ 
  and $J^{0} = \left\{ 1 \right\} $, so
  $\pi_1 \left( X, A, x_0 \right) 
  = \left[ I, \left\{ 0 \right\} , \left\{ 1 \right\} ;
  X, A, x_0 \right] $ is the set of homotopy classes of paths in
  $X$ from a varying point in $A$ to the fixed basepoint
  $x_0 \in A$. In general, this is not a group in any
  natural way. \\
  \linebreak
  Now, we saw before that
  $\pi_n \left( X, x_0 \right) $ can be regarded as
  homotopy classes of maps $\left( S^{n}, x_0 \right) \to 
  \left( X, x_0 \right) $. Similarly, collapsing
  $J^{n-1}$ to a point, converts
  $\left( I^{n} , \partial I^{n}, J^{n-1} \right) $ 
  to $\left( D^{n}, S^{n-1}, s_0 \right) $.
  In this case, addition is done by
  the map $c \colon D^{n} \to D^{n} \vee D^{n}$ collapsing
  $D^{n-1} \subset D^{n}$ to a point.\\
  \linebreak
  \begin{theorem}[Compression criterion]\label{Thm:Compression}
      A map $f \colon \left( D^{n}, S^{n-1}, s_0 \right) 
      \to \left( X, A, x_0 \right) $ represents zero
      in $\pi_n \left( X, A, x_0 \right) $ if and only if
      it is homotopic $\rel S^{n-1}$ to a map with image
      contained in $A$.
  \end{theorem}
  
  \begin{proof}
      Suppose we have a homotopy
      $\rel S^{n-1}$ from $f$ to a map
      $g$, so
      $\left[ f \right] = \left[ g \right] $ in
      $\pi_n \left( X, A, x_0 \right) $. 
      Viewing $g$ as a map
      $\left( D^{n}, S^{n-1}, s_0 \right) 
      \to \left( X, A, x_0 \right) $ whose
      image is contained in $A$, we
      can construct the homotopy
      $H \colon D^{n} \times I \to X$ by
      $H(x,t) = g\left( (1-t) x + s_0 t \right) $ 
      which is a homotopy from $g$ to the
      constant map at $x_0$, hence
      $\left[ g \right]  = 0$ in $\pi_n (X, A, x_0)$.\\
      Conversely, if $\left[ f \right] = 0$ via
      a homotopy $F \colon D^{n} \times I \to X$ such that
      $F(x,0) = f(x)$ and
      $F(x,1) = x_0$ for all $x \in D^{n}$ and
      $F(x,t) \in A$ for all
      $x$ with $\left| x \right| = 1$ as well
      as $F(s_0,t) = x_0$ for all $t$. We can
      construct a homotopy
      using $F$ by restricting $F$ to a family of
      $n$-disks in $D^{n} \times I$ starting with
      $D^{n}\times \left\{ 0 \right\} $ and ending
      with the disk $D^{n} \times \left\{ 1 \right\} 
      \cup S^{n-1} \times I$, and where all the disks
      throughout the family have the same boundary.
      See Figure \ref{fig:DJIMMXKXO0O-jpeg} for a depiction
      of this homotopy.

      \begin{figure}[htpb]
          \centering
          \includegraphics[width=0.8\textwidth]{Figures/DJIMMXKXO0O.jpeg}
          \caption{}
          \label{fig:DJIMMXKXO0O-jpeg}
      \end{figure}
      This completes the proof.
  \end{proof}

  Next, some things that carry over:
  a map $\varphi \colon \left( X, A, x_0 \right) 
  \to \left( Y, B, y_0 \right) $ induces maps
  $\varphi_* \colon \pi_n \left( X, A, x_0 \right) 
  \to \pi_n \left( Y, B, y_0 \right) $ which are
  homomorphisms when $n\ge 2$ and have properties analogous
  to those in the absolute case: 
  $\left( \varphi \psi  \right)_* = 
  \varphi_* \psi_*, (\id_{(X,A,x_0)})_{*} = \id_{\pi_n (X, A, x_0)}$,
  and if  $\varphi \simeq \psi $ through maps
  $\left( X,A,x_0 \right) \to \left( Y,B,y_0 \right) $,
  then $\varphi_* = \psi_*$. 

  \subsubsection{LES of relative homotopy groups}
  Probably the most useful feature of relative homotopy
  groups $\pi_n (X,A,x_0)$ is that they 
  fit into a long exact sequence
  \[
  \ldots \to \pi_n (A,x_0)
  \stackrel{i_*}{\to} \pi_n(X,x_0)
  \stackrel{j_*}{\to} \pi_n (X,A,x_0)
  \stackrel{\partial}{\to} \pi_{n-1}(A,x_0) \to 
  \ldots \to \pi_0 (X,x_0).
  \] 
  Here $i$ and $j$ are the inclusions
  $\left( A, x_0 \right) \hookrightarrow
  (X,x_0)$ and
  $\left( X, x_0, x_0 \right) \hookrightarrow
  \left( X,A,x_0 \right) $. The map
  $\partial$ comes from restricting maps
  $\left( I^{n},\partial I^{n}, J^{n-1} \right) \to 
  \left( X,A,x_0 \right) $ to
  $I^{n-1}$ (the face of $I^{n}$ with the last
  coordinate $s_n = 0$ ),
  or equivalently, by restricting maps
  $\left( D^{n},S^{n-1},s_0 \right) \to 
  \left( X,A,x_0 \right) $ to $S^{n-1}$. The map $\partial$,
  called the \textit{boundary map}, is a homomorphism
  when $n>1$. In fact, we can show the following theorem

  \begin{theorem}[LES of relative homotopy groups]
      Given 
      $x_0 \in B \subset A \subset X$,
      the sequence of relative homotopy groups
  \[
      \ldots \to 
      \pi_n \left( A,B, x_0 \right) 
      \stackrel{i_*}{\to} 
      \pi_n \left( X, B, x_0 \right) 
      \stackrel{j_*}{\to} 
      \pi_n (X, A, x_0)
      \stackrel{\partial}{\to} 
      \pi_{n-1} \left( A,B, x_0 \right) 
      \to \ldots \to 
      \pi_1 (X,A,x_0)
  \] 
  is exact and natural.
  In the case when $B = \left\{ x_0 \right\} $, we have that
  the LES
  \[
  \ldots \to \pi_n (A,x_0)
  \stackrel{i_*}{\to} \pi_n(X,x_0)
  \stackrel{j_*}{\to} \pi_n (X,A,x_0)
  \stackrel{\partial}{\to} \pi_{n-1}(A,x_0) \to 
  \ldots \to \pi_0 (X,x_0).
  \] 
  is exact and natural.
  \end{theorem}

  \begin{proof}
      \textit{Exactness at $\pi_n
      \left( X,B,x_0 \right) $ :} the composition
      $j_* i_*$ is zero because any
      map $\left( I^{n}, \partial I^{n},
      J^{n-1}\right) \to \left( A,B,x_0 \right) $ 
      is zero in $\pi_n \left( X, A, x_0 \right) $ by the
      compression criterion (Theorem \ref{Thm:Compression}).
      To see that $\ker j_* \subset 
      \im i_*$, let
      $f \colon \left( I^{n}, \partial I^{n},
      J^{n-1}\right) \to \left( X, B, x_0 \right) $ 
      represent zero in $\pi_n \left( X, A, x_0 \right) $.
      Using the compression criterion again, we
      then get that $f$ is homotopic $\rel \partial I^{n}$ 
      to a map with image in $A$, hence the class
      $\left[ f \right] \in \pi_n \left( X, B, x_0 \right) $ 
      is indeed in the image of $i_*$. We conclude that
      $\ker j_* = \im i_*$, obtaining exactness
      at $\pi_n \left( X,B, x_0 \right) $.\\
      \textit{Exactness at $\pi_n (X,A,x_0)$:} 
      for a map  $\left[ f \right] 
      \in \im j_*$, we have that
      $j_*$ maps $\partial I^{n}$ into $B$, hence
      in particular $I^{n-1} \subset \partial I^{n}$ into
      $B$, so $\partial j_* \left[ f \right] $ 
      represents a homotopy class
      in $\pi_{n-1}\left( A,B,x_0 \right) $ with
      image in $B$, but then by the compression criterion,
      $\partial j_* \left[ f \right] = 0$ in
      $\pi_{n-1} \left( A,B,x_0 \right) $, so
      $\im j_* \subset \ker \partial $.
      Conversely, suppose
      $\partial \left[ f \right] = 0$. By the compression
      criterion, representatives of $\partial \left[ f \right] $
      are homotopic $\rel \partial I^{n-1}$ to a map
      with image in $B$. In particular,
      $f|_{I^{n-1}}$ is homotopic to a map with
      image in $B$ via a homotopy $F \colon
      I^{n-1} \times I \to A \rel \partial I^{n-1}$.
      We can tack $F$ onto $f$ to get a new map
      $\left( I^{n}, \partial I^{n},
      J^{n-1}\right) \to 
      \left( X, B, x_0 \right) $ which, as
      a map
      $\left( I^{n}, \partial I^{n}, J^{n-1} \right) \to 
      \left( X, A, x_0 \right) $ is homotopic to
      $f$ by the homotopy that tacks on increasingly longer
      initial segments of $F$. See Figure
      \ref{fig:IDIWKAKX-png}. Hence
      $\left[ f \right] \in \im j_*$.

      \begin{figure}[htpb]
          \centering
          \includegraphics[width=0.2\textwidth]{Figures/IDIWKAKX.png}
          \caption{}
          \label{fig:IDIWKAKX-png}
      \end{figure}

      \textit{Exactness at $\pi_n (A,B,x_0)$ :} 
      First, $i_* \partial$ is zero since
      the restriction of a map
      $f \colon \left( I^{n+1}, \partial I^{n+1},
      J^{n}\right) \to \left( X,A,x_0 \right) $ 
      to $I^{n}$ is homotopic $\rel \partial I^{n}$ to a 
      constant map via $f$ itself (a similar picture
      to Figure \ref{fig:DJIMMXKXO0O-jpeg} works).\\
      Conversely, if $B$ is a point, then
      a nullhomotopy $f_t \colon
      \left( I^{n}, \partial I^{n} \right) 
      \to \left( X, x_0 \right) $ of
      $f_0 \colon \left( I^{n},\partial I^{n} \right) 
      \to \left( A,x_0 \right) $ gives a map
      $F \colon \left( I^{n+1},\partial I^{n+1},J^{n} \right) 
      \to \left( X,A,x_0 \right) $ with
      $\partial \left( \left[ F \right]  \right) 
      = \left[ f_0 \right] $. So in this case, the proof is
      finished.
      For a general $B$, let
      $F$ be a nullhomotopy of
      $f \colon \left( I^{n},\partial I^{n},J^{n-1} \right) 
      \to \left( A,B,x_0 \right) $ through maps
      $\left( I^{n}, \partial I^{n}, J^{n-1} \right) 
      \to \left( X,B,x_0 \right) $ and
      let $g$ be the restriction of
      $F$ to $I^{n-1}$ in $I^{n-1} \times I = I^{n}$ (see
      the first of the pictures in
      Figure \ref{fig:USIIOOQ-png}).
      Next reparametrize the $n$ th and
      $(n+1)$ st coordinates as in the
      second picture. Then 
       we find that $f$ with $g$ tacked on
       is in the image of $\partial$. But
       as before, tacking $g$ onto $f$ gives the
       same element of $\pi_n (A,B,x_0)$

      \begin{figure}[htpb]
          \centering
          \includegraphics[width=0.4\textwidth]{Figures/USIIOOQ.png}
          \caption{}
          \label{fig:USIIOOQ-png}
      \end{figure}
  \end{proof}

  
\begin{corollary}
    Consider the inclusion
    $\iota \colon X = X \times \left\{ 0 \right\} 
    \hookrightarrow CX$.
    Then
    $\pi_n \left( CX, X, x_0 \right) 
    \cong \pi_{n-1}\left( X, x_0 \right) $ for all
    $n\ge 1$. Taking
    $n=2$, we can thus realize an group $G$, abelian
    or not, as a relative $\pi_2$ by
    choosing $X$ to have $\pi_1 (X) \cong G$.
\end{corollary}

There are also change-of-basepoint isomorphisms
$\beta_{\gamma}$ for relative homotopy groups.
One takes a path  $\gamma$ in $A \subset X$ from
$x_0$ to $x_1$ which induces
$\beta_{\gamma} \colon \pi_n (X,A,x_1) \to 
\pi_n (X,A,x_0)$ by setting
$\beta_{\gamma} \left( \left[ f \right]  \right) 
= \left[ \gamma f \right] $, where
$\gamma f$ is depicted in 
Figure \ref{fig:DIWIOA-png}.

\begin{figure}[htpb]
    \centering
    \includegraphics[width=0.2\textwidth]{Figures/DIWIOA.png}
    \caption{}
    \label{fig:DIWIOA-png}
\end{figure}

Restricting to loops at the
basepoint, the association $\gamma \mapsto 
\beta_{\gamma}$ defines an action
of $\pi_1 \left( A, x_0 \right) $ on
$\pi_n \left( X, A, x_0 \right) $ analogous to the
action of $\pi_1 \left( X, x_0 \right) $ on
$\pi_n (X,x_0)$.






%\include{pset1}

\begin{problem}[$n$-connected in the relative case]\label{n-connected-relative}
    The following four conditions are equivalent for
    $i>0$ :
    \begin{enumerate}
        \item Every map
            $\left( D^{i} , \partial D^{i} \right) \to 
            \left( X,A \right) $ is homotopic
            $\rel \partial D^{i}$ to a map $D^{i} \to A$.
        \item Every map $\left( D^{i},\partial D^{i} \right) 
            \to (X,A)$ is homotopic through such maps
            to a map $D^{i} \to A$.
        \item Every map $\left( D^{i}, \partial D^{i} \right) 
            \to \left( X,A \right) $ is homotopic through such
            maps to a constant map $D^{i} \to A$.
        \item $\pi_i \left( X, A, x_0 \right) = 0$ for all
            $x_0 \in A$.
    \end{enumerate}
    When $i = 0$, we did not define the relative $\pi_0$,
    and (1)-(3) are each equivalent to saying that
    each path-component of $X$ contains points
    in $A$ since $D^{0}$ is a point and
    $\partial D^{0}$ is empty. The pair
    $\left( X, A \right) $ is called \textit{$n$-connected}
    if (1)-(4) hold for $0<i\le n$ and
    (1)-(3) hold for  $i=0$.
\end{problem}


\subsection{Whitehead's Theorem}

\begin{theorem}[Whitehead's Theorem]\label{Thm:Whitehead}
    If a map $f \colon X \to Y$ between connected
    $CW$ complexes induces isomorphisms
    $f_* \colon \pi_n (X) \to \pi_n (Y)$ for all
    $n$, then $f$ is a homotopy equivalence.
    In case $f$ is the inclusion of a subcomplex
    $X \hookrightarrow Y$, the conclusion is stronger:
    $X$ is a deformation retract of $Y$.
\end{theorem}

The proof will require the following lemma:

\begin{lemma}[Compression Lemma]
    Let $\left( X,A \right) $ be a CW pair and let
    $\left( Y, B \right) $ be any pair with
    $B \neq \varnothing$. For each  $n$ such that
    $X - A$ has cells of dimension $n$, assume
    that  $\pi_n \left( Y, B, y_0 \right) = 0$ for
    all $y_0 \in B$. Then every map $f \colon
    \left( X,A \right) \to \left( Y,B \right) $ is homotopic
    $\rel A$ to a map $X \to B$.
    When $n = 0$, the condition that
    $\pi_n \left( Y,B,y_0 \right) =0$ for all
    $y_0 \in B$ is to be regarded as saying that
    $\left( Y,B \right) $ is $0$-connected.
\end{lemma}

\begin{proof}[Proof of lemma]
    Assume inductively that $f$ has already been
    homotoped to take the skeleton
    $X^{k-1}$ to $B$. Let
    $\Phi$ be the caracteristic (attaching) map of 
    cell $e^{k}$ of $X - A$. Then the composition
    $f \Phi \colon \left( D^{k} , \partial D^{k} \right) 
    \to \left( Y,B \right) $ is in some class
    in $\pi_k \left( Y, B, y_0 \right) = 0$, so
    it can be homotoped into $B \rel \partial D^{k}$ by
    the compression criterion when
    $k > 0$, or
    by $\left( Y,B \right) $ being $0$-connected for
    $k = 0$ (this is condition (3) in Problem \ref{n-connected-relative}).
    This homotopy of $f \Phi$ induces a homotopy
    $\rel X^{k-1}$ on the quotient space
    $X^{k-1} \cup e^{k}$ of $X^{k-1} \sqcup D^{k}$.
    Doing this for all $k$-cells of $X-A$ simultaneously, and
    taking the constant homotopy on $A$, we obtain a
    homotopy of $f|_{X^{k} \cup A}$ to a map into
    $B$. Since the inclusion of a
    subcomplex into a CW-complex is a cofibration,
    $f|_{X^{k} \cup  A}$ extends to all of $X$ (essentially
    the homotopy extension property).
    This completes the inductive step in the finite dimensional
    CW-complex case.
    In the general case, we perform the
    homotopy of the inductive step during the
    $t$-interval $\left[ 1- \frac{1}{2^{k}},
    1- \frac{1}{2^{k+1}}\right] $. Any finite skeleton
    $X^{k}$ is eventually stationary under these
    homotopies, hence we have a well-defined
    homotopy $f_t, t \in \left[ 0,1 \right] $ with
    $f_1 (X) \subset B$.
\end{proof}


\begin{proof}[Proof of Whitehead's Theorem, \ref{Thm:Whitehead}]
    Let's tackle the case when $f$ is the inclusion
    of a subcomplex first. Consider then
    the LES of the pair $\left( Y,X \right) $. Since
    $f$ by assumption induces isomorphisms
    on all homotopy groups,
    $f_* \colon \pi_* (X) \to \pi_* (Y)$, the
    relative homotopy groups
    $\pi_* (Y,X)$ are zero. Applying the lemma now
    to the identity map $\left( Y,X \right) \to 
    \left( Y,X \right) $, we obtain a homotopy
    of the identity  $\id \colon Y \to Y$ to
    a map $Y \to X$ which is relative to
    $X$. That is, we obtain a deformation retract of
    $Y$ onto $X$.\\
    \linebreak
    For the general case, recall that
    a map $f \colon X \to Y$, can be considered
    as the composition of the
    inclusion $X \hookrightarrow M_f$ and the
    retraction $M_f \to Y$. Since
    the retraction is a homotopy equivalence,
    it suffices to show that $M_f$ deformation retracts
    onto $X$ if $f$ induces isomorphisms on homotopy
    groups, or equivalently, if the relative groups
    $\pi_n \left( M_f, X \right) $ are all zero (since
    $M_f \simeq Y$ ).
    If $f$ is cellular - i.e., takes the $n$-skeleton of
    $X$ to the $n$-skeleton of $Y$ for all
    $n$ - then $\left( M_f, X \right) $ is a CW pair and
    we can apply the first paragraph of the proof.\\
    If $f$ is not cellular, we can either apply
    Theorem 4.8 in \cite{Hatcher} which says
    that $f$ is homotopic to a cellular map, or we can use
    the following argument.

    First, using that 
    $\pi_n \left( M_f, X \right) = 0$ for all $n$, 
    apply the Compression Lemma to
    the inclusion $ \left( X \cup  Y, X \right) 
    \hookrightarrow \left( M_f, X \right) $ to
    obtain a homotopy of the
    inclusion to a map into $X \rel X$.
    The inclusion $X \cup Y \hookrightarrow M_f$ can be
    seen to be a cofibration using 
    Theorem \ref{Thm:SJJDHW29WW}, so
    the pair $\left( M_f, X \cup Y \right) $ satisfies the
    homotopy extension property. So the
    homotopy in question extends to a homotopy
    from the identity of $M_f$ to a
    map $g \colon M_f \to M_f$ taking 
    $X \cup Y$ into $X \rel X$. 
    However, we first of all do not know that this
    homotopy is $\rel X$ nor that 
    $g$ maps all of $M_f$ into $X$.\\
    So we apply the
    Compression lemma again to the
    composition
     \[
         \left( X \times I \sqcup Y,
         X \times \partial I \sqcup Y\right) 
         \to \left( M_f, X \cup Y \right) 
         \stackrel{g}{\to} \left( M_f,X \right) ,
    \]
    to get a homotopy
    $\rel X \times \partial I \sqcup Y$ of
    $g$ to a map 
    $X \times I \sqcup Y \to X$. In particular,
    this homotopy passes through the quotient
    $X \times I \sqcup Y \to M_f$, so we
    get a homotopy of $g \rel X \times \partial I \cup Y$
    to a map $M_f \to X$.\\
    Composing the homotopy
    from the identity of $M_f$ to $g$ with this homotopy,
    we get a deformation retraction of
    $M_f$ onto $X$.
\end{proof}

\begin{note}
    Whitehead's theorem requires a map
    $f \colon X \to Y$ which induces isomorphisms
    on homotopy groups. Thus it does not apply simply
    to any two CW complexes $X$ and $Y$ with isomorphic
    homotopy groups since there might not exist such a map.
    For examples where this is the case, see
    \cite[p. 348]{Hatcher}.
\end{note}

\begin{corollary}
    If $X$ is a CW complex with
    $\pi_n(X) = 0$ for all $n\ge 0$, then
    $X \simeq \left\{ 0 \right\} $.
\end{corollary}

\begin{proof}
    The inclusion of a $0$-cell into the complex
    induces an isomorphism on homotopy
    groups, so by Whitehead's theorem, the complex deformation
    retracts to the $0$-cell.
\end{proof}

\begin{lemma}[Extension Lemma]\label{Extension-Lemma}
    Given a CW pair $\left( X,A \right) $ and a map
    $f \colon A \to Y$ with $Y$-path connected,
    then $f$ can be extended to a map
    $X \to Y$ if $\pi_{n-1}(Y) = 0$ for all $n$ such that
    $X -A$ has cells of dimension $n$.
\end{lemma}

\begin{proof}
    Suppose that $f$ has been extended over the
    $\left( n-1 \right) $-skeleton. Then an extension
    over an $n$-cell exists if and only if
    the composition of the cell's attaching
    map $S^{n-1} \to X^{n-1}$ with $f
    \colon X^{n-1} \to Y$ is nullhomotopic, which
    it is if $\pi_{n-1} (Y) = 0$.
\end{proof}

\subsection{Cellular Approximation}

\begin{definition}[Cellular maps]
    A map $f \colon X \to Y$ between CW complexes,
    satisfying
    $f(X^{n}) \subset Y^{n}$ for all $n$, is called
    a \textit{cellular map}.
\end{definition}

\begin{theorem}[Cellular Approximation Theorem]
    Every map $f \colon X \to Y$ of CW complexes is homotopic
    to a cellular map. If $f$ is already cellular
    on a subcomplex $A \subset X$, then homotopy
    map be taken to be stationary on $A$.
\end{theorem}

\begin{remark}[]
    Cellular approximation tells us
    that $\pi_n(X)$ only depends on
    the $(n+1)$-skeleton.
\end{remark}

Recall the following about simplicial maps and simplicial
approximations:

\begin{definition}[Simplicial map]
    Let $K$ and $L$ be simplicial complexes. A function
    $s \colon \left| K \right| \to \left| L \right| $ 
    is called \textit{simplicial} if it takes simplexes
    of $K$ linearly onto simplexes of $L$.
\end{definition}

\begin{definition}[Carrier of $f(x)$]
    Given a map  $f \colon \left| K \right| 
    \to \left| L \right| $ between polyhedra and a
    point $x \in \left| K \right| $, the point
    $f(x)$ lies in the interior of a unique simplex of
    $L$. Call this simplex the \textit{carrier} of 
    $f(x)$.
\end{definition}

\begin{definition}[Simplicial Approximation]
    A simplicial map $s \colon \left| K \right| 
    \to \left| L \right| $ is a simplicial approximation of 
    $f \colon \left| K \right|  \to \left| L \right| $ 
    if $s(x)$ lies in the carrier of $f(x)$ for
    each $x \in \left| K \right| $.
\end{definition}

\begin{theorem}[Simplicial approximation theorem]
    Let $f \colon \left| K \right| \to 
    \left| L \right| $ be a map between polyhedra.
    If $m$ is chosen large enough, there is a simplicial
    approximation $s \colon \left| K^{m} \right| \to 
    \left| L \right| $ to $f \colon \left| K^{m} \right| 
    \to \left| L \right| $.
\end{theorem}

Thus we may view cellular approximation as
a CW analog of simplicial approximation since simplicial
maps are cellular. Simplicial maps are much more rigid
than cellular maps, however, and the core
proof of cellular approximation will be
a weaker form of simplciial approximation.\\
\linebreak
But first, a nice corollary:

\begin{corollary}
    $\pi_n \left( S^{k} \right) $ for
    $n<k$.
\end{corollary}

\begin{proof}
    If $S^{n}$ and $S^{k}$ are given their usual
    CW structure of a single $0$-cell and
    then an $n$- or $k$-cell, respectively, then by
    the Cellular Approximation Theorem, 
    any pointed map $S^{n} \to S^{k}$ is based homotopic to a 
    cellular map, and hence maps
    the  $n$-skeleton of $S^{n}$ into the $n$-skeleton
    of $S^{k}$. But the $n$-skeleton of $S^{k}$ is
    just the $0$-cell. That is, 
    any map $S^{n} \to S^{k}$ is based nullhomotopic, so
    $\pi_n \left( S^{k} \right) = 0$.
\end{proof}

\begin{proof}[Proof of Cellular Approximation Theorem]
    Long. To do
\end{proof}

\begin{example}[Cellular Approximation for Pairs]
    Every map $f \colon \left( X,A \right) 
    \to \left( Y,B \right) $ of $CW$ pairs can be deformed
    through maps $\left( X,A \right) \to 
    \left( Y,B \right) $ to a cellular map.
    This follows from the theorem by first deforming the
    restriction $f \colon A\to B$ to be cellular,
    then extending this to a homotopy of $f$ on all
    of $X$, then deforming the resulting map
    to be cellular staying fixed on $A$. As a further
    refinement, the homotopy of $f$ can be
    taken to be stationary on any subcomplex of
    $X$ where $f$ is already cellular.
\end{example}

\begin{corollary}[Geometric Version of
    $n$-connectedness]\label{n-connectedness-geometrically}
    A CW pair $\left( X,A \right) $ is $n$-connected
    if all the cells in $X - A$ have dimension
    greater than $n$. In particular, the
    pair  $\left( X, X^{n} \right) $ is
    $n$-connected, hence the
    inclusion $X^{n} \hookrightarrow X$ induces
    isomorphisms on $\pi_i$ for $i < n$ and
    a surjection on $\pi_n$.
\end{corollary}

\begin{proof}
    Recall that $\left( X,A \right) $ is $n$-connected
    if every map
    $\left( D^{i}, \partial D^{i} \right) 
    \to \left( X,A \right) $ is homotopic through
    such maps to a map $D^{i} \to A$.
    Now let
    $f \left( D^{i}, \partial D^{i} \right) 
    \to \left( X,A \right) $ be any map.
    Then by the Cellular Approximation theorem for
    Pairs, $f$ is homotopic through maps
    $\left( D^{i} ,\partial D^{i} \right) \to 
    \left( X,A \right) $ to a cellular map, 
    $\tilde{f} \colon \left( D^{i} , \partial D^{i} \right) 
    \to \left( X,A \right) $. But by assumption, all
    cells in $X-A$ have dimension greater than
    $n \ge i$. Hence $\tilde{f}$ maps
    $D^{i}$ into $A$.
    The last part of the statement now follows from the LES
    \[
    \ldots \to \pi_n (X^{n}) \stackrel{\iota_*}{\to} \pi_n(X) \to 
    \underbrace{\pi_n\left( X,X^{n} \right)}_{0} \to 
    \pi_{n-1}(X^{n}) \stackrel{\iota_*}{\to}  \pi_{n-1}(X) \to 
    \underbrace{\pi_{n-1}\left( X, X^{n} \right)}_{0} \to \ldots
    \] 
\end{proof}

\subsection{CW Approximation}

\begin{definition}[Weak Homotopy Equivalence]
    A map $f \colon X \to Y$ is called a
    \textit{weak homotopy equivalence} if it induces
    isomorphisms $\pi_n \left( X, x_0 \right) 
    \to \pi_n \left( Y, f(x_0) \right) $ for all
    $n \ge 0$ and all choices
    of basepoint $x_0$.
\end{definition}

\begin{remark}[Reformulation of Whitehead's Theorem]
    Whitehead's Theorem thus says that
    a weak homotopy equivalence between
     CW complexes is, in fact, a homotopy equivalence.
\end{remark}

\begin{definition}[CW Approximation]
    For a space $X$, a weak homotopy equivalence
    $f \colon Z \to X$, where $Z$ is a CW complex, is
    called a \textit{CW approximation} to $X$.
\end{definition}

\begin{remark}[]
    CW approximations to a given space $X$ are
    unique up to homotopy equivalence since
    if $f \colon Z \to X$ and 
    $f' \colon Z' \to X$ are CW approximations, then
    consider the composition
    $Z \to X \hookrightarrow M_{f'}$.
    Since $f' \colon Z' \to X$ is assumed to be a weak
    homotopy equivalence, we find by the relative LES that
    $\pi_n (M_f, Z') \cong \pi_n \left( X, Z' \right) = 0$ for all
    $n\ge 0$, so by 
    the Compression Lemma (with $A$ chosen to
    be the basepoint of $Z$), we
    find that the map $Z \to X \hookrightarrow M_{f'}$
    is homotopic to a map
    $Z \to Z' \subset M_{f'}$ relative
    to the basepoint.
    But taking $\pi_n$ of
    $Z \to X \to M_{f'} \to Z'$, we get
    $\pi_n(Z) \stackrel{\cong}{\to} 
    \pi_n(X) \stackrel{\cong}{\to} 
    \pi_n (M_{f'}) \stackrel{\cong}{\to}
    \pi_n \left( Z' \right)$
    where $\pi_n (X) \stackrel{\cong}{\to} \pi_n (M_{f'})$ 
    follows from $\iota \colon X \simeq M_{f'}$ being
    a homotopy equivalence;
    $\pi_n\left( M_{f'} \right) 
    \stackrel{\cong}{\to} \pi_n (Z')$ follows
    from the homotopy that we got from the compression lemma,
    and the first isomorphism
    $f_* \colon \pi_n (Z) \stackrel{\cong}{\to}  \pi_n (X)$ 
    follows from $f$ being a weak homotopy equivalence.
    Applying Whitehead's theorem, we find that this
    composition is a homotopy equivalence
    $Z \simeq Z'$.
\end{remark}

\begin{proposition}[]\label{Prop:CW-Approximation}
    Every space $X$ has a CW approximation
    $f \colon Z \to X$. If $X$ is path-connected,
    $Z$ can be chosen to have a single $0$-cell, with
    all other cells attached by basepoint-preserving maps.
    Thus every connected CW complex is homotopy equivalent to
    a CW complex with these additional properties.
\end{proposition}

\begin{proof}
    The construction of a CW approximation
    $f \colon Z \to X$ is inductive, so we first
    describe the induction step. 
    Suppose we are given a CW complex $A$ with a map
    $f \colon A \to X$ and suppose
    we have chosen a basepoint $0$-cell $a_{\gamma}$ in
    each component of $A$.
    Then for an integer $k\ge 0$, we will attach
    $k$-cells to $A$ to form a CW complex
    $B$ with a map $f \colon B \to X$ extending $f$ such that
    \begin{itemize}
        \item For each basepoint $a_{\gamma}$, the
            induced map $f_* \colon
            \pi_i \left( B, a_{\gamma} \right) 
            \to \pi_i \left( X, f\left( a_{\gamma} \right) 
            \right) $ is injective for
            $i = k-1$ (when $k>0$ ) and surjective
            for $i = k$.
    \end{itemize}

    We do this in two steps (the first step is omitted
    when $k= 0$ ):

    \begin{enumerate}
        \item We have been given
            a CW complex $A$ and a map
            $f \colon A \to X$ alongside basepoints
            $a_{\gamma}$. Now for each
            nontrivial element $\alpha$ of the kernel
            $\ker f_*$ ranging over
            all basepoints, choose
            a map $\varphi_{\alpha}\colon
            \left( S^{k-1}, s_0 \right) \to 
            \left( A, a_{\gamma} \right) $ representing
            $\alpha$. We may assume that the
            $\varphi_{\alpha}$ are all cellular (by
            the Cellular Approximation Theorem) where
            $S^{k-1}$ is given its standard CW structure
            with $s_0$ as a 0-cell. 
            Attaching cells $e_{\alpha}^{k}$ to $A$ via
            the maps $\varphi_{\alpha}$ then produces
            a CW complex. Now, $f \circ \varphi_{\alpha}$ 
            is nullhomotopic, so
            $f$ extends over the cell
            $e_{\alpha}^{k}$.
        \item Choose maps
            $f_{\beta}\colon S^{k}\to X$ representing
            all nontrivial elements of
            $\pi_k \left( X, f(a_{\gamma}) \right) $ for
            all the $a_{\gamma}$ 's Then attach
            cells $e_{\beta}^{k}$ to $A$ via the
            constant maps at the appropriate basepoints
            $a_{\gamma}$ and extend $f$ over the resulting spheres
            $S_{\beta}^{k}$ via $f_{\beta}$.
    \end{enumerate}
    By the construction, then
    surjectivity of
    $f_* \colon \pi_i \left( B, a_{\gamma} \right) 
    \to \pi_i \left( X, f(a_{\gamma}) \right) $ for
    $i = k$ follows. Now
    let $\alpha$ be in the kernel of
    $f_* \colon \pi_{k-1} (B, a_{\gamma}) \to 
    \pi_{k-1}\left( X, f(a_{\gamma}) \right) $, and
    let $h \colon S^{k-1} \to B$ be a cellular
    map that represent $\alpha$. Since
    $h$ is cellular, its image is contained in the
     $\left( n-1 \right) $-skeleton
     of $B$ which is a subskeleton (could be all) of $A$.
     Since $h$ has image in $A$, it is in the kernel
     of $f_* \colon \pi_{k-1}(A, a_{\gamma}) \to 
     \pi_{k-1} \left( X, f(a_{\gamma}) \right) $ and thus
     it is homotopic to some $\varphi_{\alpha}$ and therefore
     nullhomotopic in $B$.\\
     \begin{note}
         In step (1), it suffices to attach cells for
         just the generators of the kernels
         when $k>1$, and just for the generators
         of $\pi_k \left( X, f(a_{\gamma}) \right) $ in
         step (2) when $k>0$.
     \end{note}

     \begin{note}
         If the given map $f \colon A \to X$ happened
         to already be injective or surjective
         on $\pi_i$ for some $i < k-1$ or
         $i < k$, respectively, then this remains
         true after attaching the $k$-cells.
         This is because attaching $k$-cells does
         not affect $\pi_i$ if $i < k-1$, by
         cellular approximation, not does
         it affect surjectivity on
         any $\pi_i$, simply because the same maps
         still hold and work.
     \end{note}

     Now to construct a CW approximation $f \colon Z \to 
     X$, one can start with $A$ consisting of one point for
     each path-component of $X$, with
     $f \colon A \to X$ mapping each of these points
     to the corresponding path-component.
     This gives a bijection on $\pi_0$ by construction, hence
     it provides us with the inductive base case.
     Now we can attach $1$-cells to $A$ to create
     a surjection on $\pi_1$ for each path-component, then
     $2$-cells to improve this to an isomorphism on
     $\pi_1$ and a surjection on $\pi_2$ and so forth
     for each successive $\pi_i$ in turn. After all cells
     have been attached, on has a CW complex $Z$ with
     a weak homotopy equivalence $f \colon Z \to X$.
\end{proof}


\begin{example}[]
    One can apply this technique to produce a CW approximation
    to a pair $(X,X_0)$ also. First one constructs
    a CW approximation $f_0 \colon Z_0 \to X_0$, then
    one starts with the composition
    $Z_0 \to X_0 \hookrightarrow X$ and attaches
    cells to $Z_0$ to create a weak homotopy equivalence
    $f \colon Z \to X$ extending $f_0$.
    Then we get

    \begin{equation*}
    \begin{tikzcd}
        \pi_n(Z_0) \ar[r] \ar[d, "\cong"] & \pi_n (Z) \ar[r] \ar[d,
        "\cong"] & 
        \pi_n (Z, Z_0) \ar[r] \ar[d] &
        \pi_{n-1} (Z_0) \ar[r] 
        \ar[d, "\cong"]&
        \pi_{n-1} (Z) \ar[d, "\cong"] \\
        \pi_n (X_0) \ar[r] & \pi_n(X) \ar[r] & \pi_n (X,X_0) \ar[r] &
        \pi_{n-1} (X_0) \ar[r] & \pi_{n-1}(X)
    \end{tikzcd}
    \end{equation*}
    By the five-lemma, it follows that
    $\pi_n \left( Z,Z_0 \right) \to 
    \pi_n \left( X,X_0 \right) $ is an isomorphism
    for each $n$.
\end{example}

\begin{proposition}[]\label{Prop:Removing-lower-dim-cells}
    If $\left( X,A \right) $ is an $n$-connected
    CW pair, then there exists a CW pair
    $\left( Z,A \right) \simeq \left( X,A \right) \rel A$ 
    such that all cells of $Z - A$ have dimension greater
    than $n$.
\end{proposition}

\begin{proof}
    Starting with the inclusion
    $A \hookrightarrow X$, attach cells of
    dimension $n+1$ and higher to
    $A$ to produce a CW complex $Z$ and
    a map $f \colon Z \to X$ using the
    procedure of Proposition \ref{Prop:CW-Approximation}.
    In particular then
    by the Proposition proof,
    $f_*$ induces an injection of
    $\pi_n$ and isomorphisms on all higher homotopy groups.
    Now, the induced map on $\pi_n$ is also
    surjective since it is true for
    $A \hookrightarrow Z \stackrel{f}{\to} X$ 
    as $\left( X,A \right) $ is $n$-connected and hence
    $\pi_n (A) \stackrel{\cong}{\to}  \pi_n(X)$ is an isomorphism.
    Since $f$ is equal to this inclusion on the
    $n$-skeleton, this gives that $f_*$ is also surjective.
    By cellular approximation
    $A \hookrightarrow Z$ induces an isomorphism
    on homotopy groups in dimensions below $n$, and
    likewise $n$-connectedness does the same for
    $A \hookrightarrow X$. But then since
    \begin{equation*}
    \begin{tikzcd}
        \pi_n (A) \ar[rr, bend right, "\iota_*",
        "\cong"'] 
        \ar[r, "\iota_*", "\cong"'] & 
        \pi_n (Z) \ar[r,"f_*"] & 
        \pi_n (X)
    \end{tikzcd}
    \end{equation*}
    commutes, we find that $f_*$ is also an
    isomorphism on all $n\ge 0$.
    Thus $f$ is a weak homotopy equivalence, and hence
    a homotopy equivalence by Whitehead's theorem.\\

    To see that $f$ is a homotopy equivalence
    $\rel A$, we could apply Proposition 
    \ref{Prop:HEP-Homotopy-Equivalence}, but
    here is an alternative argument. Let
    $W$ be the quotient space of the mapping cylinder
    $M_f$ obtained by collapsing each segment
    $\left\{ a \right\} \times I$ to a point, for
    $a \in A$. Assuming $f$ has been made cellular,
    $W$ is a CW complex (why?) containing $X$ and $Z$ as
    subcomplexes, and $W$ deformation retracts
    to $X$ just as $M_f$ does. Also,
    $\pi_i \left( W,Z \right) = 0$ for all
    $i$ since $f$ induces isomorphisms on all
    homotopy groups (by the LES), so $W$ deformation retracts
    onto $Z$ by Whitehead's Theorem (Theorem \ref{Thm:Whitehead}).
    The deformation retract of $W$ onto $X$ and the
    deformation retract of $W$ onto $Z$ are stationary
    on $A$, hence give a homotopy equivalence
    $X \simeq Z \rel A$.
\end{proof}

\begin{example}[Postnikov Towers]\label{Postnikov-Towers}
    For each connected CW complex $X$ and each
    integer $n\ge 1$, we can construct a 
    CW complex $X_n$ containing $X$ as a subcomplex such that
    \begin{enumerate}
        \item $\pi_i \left( X_n \right) = 0$ for $i>n$.
        \item The inclusion $X \hookrightarrow X_n$ induces
            an isomorphism on $\pi_i$ for $i\le n$.
    \end{enumerate}
    
    \begin{idea}
        Take $X$ and fill out any spheres of dimension
        $>n$ by filling them in.
    \end{idea}
    Indeed, we attach $(n+2)$-cells to
    $X$ using cellular maps $S^{n+1} \to X$ that
    generate $\pi_{n+1}(X)$ to form a 
    space with $\pi_{n+1}$ trivial. Then
    for this space, we attach $(n+3)$-cells to
    make $\pi_{n+2}$ trivial, and so on. The result
    is a CW complex $X_n$ with the desired properties.
    The inclusion $X \hookrightarrow X_n$ extends
    to a map $X_{n+1} \to X$ since
    $X_{n+1}$ is obtained from $X$ by attaching cells
    of dimension $n+3$ and greater, and
    $\pi_i (X_n) = 0$ for $i>n$, so we
    can apply the Extension Lemma (Lemma \ref{Extension-Lemma}).
    Thus we get a commutative diagram as follows:
    \begin{equation*}
    \begin{tikzcd}
        & \vdots \ar[d] \\
        & X_3 \ar[d] \\
        & X_2 \ar[d] \\
        X \ar[r] \ar[ru] \ar[ruu] & X_1
    \end{tikzcd}
    \end{equation*}
    This is called a \textit{Postnikov tower} for $X$.
    One can regard the spaces $X_n$ as truncations of
    $X$ which provides successively better approximations
    to $X$ as $n$ increases.
    


\end{example}
