\subsection{Problem Set 3}
    \begin{problem}[]
        Let $Y$ be a simply-connected CW-complex. Assume
        there exists a finite wedge of spheres
        $\bigvee_{i} S^{n_i}$ together with maps
        $i \colon Y \to \bigvee_{i} S^{n_i},
        r\colon \bigvee_{i} S^{n_i} \to Y$ such that
        $r \circ i$ is homotopic to $\id_Y$.
        Prove that $Y$ is homotopy equivalent to some
        finite wedge of spheres 
        $\bigvee_{j} S^{m_j}$.
    \end{problem}


    \begin{proof}
        We want
        to make use of the Corollary 4.33 in Hatcher which says:
        \begin{corollary}[4.33 Hatcher]
            A map $f \colon X \to Y$ between simply-connected
            CW complexes is a homotopy equivalence if 
            $f_* \colon H_n (X) \to H_n(Y)$ is an isomorphism
            for each $n$.
        \end{corollary}

        Since
        $r_* \circ \iota_* = \id_*$, 
        $\iota_* \colon
        H_n (Y) \to H_n\left( \bigvee_i S^{n_i} \right) $ 
        is injective for each $n$.
        Now
        $H_n \left( \bigvee_i S^{n_i} \right) 
        \cong \bigoplus_{A_n} \mathbb{Z}$ where
        we let $A_n$ be an indexing set
        for the $n$-cells $\left\{ e_{\alpha}^{n} \right\} $ 
        in the CW structure of $\bigvee_{i}S^{n_i}$.
        Let $\mathcal{A}_n$ be a set of representative basis
        elements for $H_n \left( \bigvee_i S^{n_i} \right) 
        \cong \bigoplus_{A_n}\mathbb{Z}$ corresponding
        to the inclusion of a sphere into the
        wedge.
        Any
        subgroup of a free group is free, so
        $H_n(Y) \cong
        \bigoplus_{B_n} \mathbb{Z}$ for some
        indexing set $B_n$ for each $n$.
        Now, starting with a single $0$-cell $* $ and
        attaching $\left| B_1 \right| $ $1$-cells to
        $*$, $\left| B_2 \right| $ $2$-cells to $*$, etc.,
        we obtain a space
        $Z = \bigvee_j S^{m_j}$ which satisfies
        $H_n(Z) = H_n(Y)$.
        Now let
        $\mathcal{C}_n$ denote the basis set for
        $H_n (Y) \cong \bigoplus_{B_n} \mathbb{Z}$.
        Since $r_*
        \colon H_n \left( \bigvee_i S^{n_i} \right) 
        \cong \bigoplus_{A_n} \mathbb{Z} \to 
        \bigoplus_{B_n} \mathbb{Z} \cong
        H_n (Y)$ is surjective,
        we can choose representatives
        $\tilde{\mathcal{A}_n} \subset 
        \mathcal{A}_n$ such that
        $r_* \left( \tilde{\mathcal{A}_n} \right) $ 
        gives a set of (by construction, linearly independent) 
        basis elements which generate
        $H_n (Y)$. Now defining a map
        $f \colon \bigvee_j S^{m_j} \to 
        \bigvee_i S^{n_i}$ by sending, for each $n$,
        all the $n$-spheres to distinct elements
        of $\tilde{\mathcal{A}_n}$ (this map is bijective by
        construction), we obtain a map
        $f$ such that
        $r \circ f$ induces an isomorphism on homology on all
        $n$.
    \end{proof}


    \begin{problem}[]
        Let $X$ be a path-connected CW complex such that
        $H_1 \left( X ; \mathbb{Z} \right) = 0$.
        The goal of this problem is to construct a
        simply connected space $Z$ and a map
        $X \to Z$ inducing an isomorphism in homology.
        \begin{enumerate}
            \item Give an example of such $X$ such that
                $\pi_1 \left( X \right) \neq 1$.
            \item Consider a set of generators
                for $\pi_1 (X)$, construct another
                CW complex $Y$ by attaching cells to $X$,
                so that
                \begin{itemize}
                    \item $Y$ is simply connected.
                    \item The inclusion $X \subset Y$ 
                        induces an isomorphism on homology
                        in degrees $\ge 3$.
                \end{itemize}
            \item Show that $H_2 \left( Y ; \mathbb{Z} \right) $ 
                is a sum of $H_2 \left( X; \mathbb{Z} \right) $ 
                together with a free abelian group.
                Let $A$ be a set of generators for this
                free summand.
        \end{enumerate}
    \end{problem}


    \begin{proof}
        (1) Since $H_1$ is just the abelianization for
        $\pi_1$ for path-connected spaces,
        this is equivalent to finding
        a path-connected CW complex
        $X$ whose fundamental group is nontrivial, but
        becomes trivial when abelianized. 
        By corollary 1.28 in Hatcher, for any
        group $G$, we can construct a $2$-dimensional CW
        complex $X_G$ such that $\pi_1 (X_G) \cong G$.
        So it suffices to find a nontrivial group whose
        abelianization is trivial. Such a group is
        called a perfect group, and we have
        many examples of such groups. For example,
        any non-abelian simple group is perfect, so
        for example $A_5$ is perfect. 
        The construction of $X_{A_5}$  can now be
        carried out as follows: $A_5$ is 
        generated by
        $\left( 123 \right) $ and
        $\left( 12345 \right) $ which do not commute,
        so we can express (as with any other group)
        $A_5$ as
        \[
        A_5 = \left< g_{\alpha} \mid r_{\beta} \right>
        \] 
        So in this case, the number of generators is simply $2$.
        Then we can construct $X_{A_5}$ from
        $\bigvee_{\alpha} S^{1}$ by attaching
        $2$-cells $e_{\beta}^2$ by the loops specified by
        the words $r_{\beta}$. By Proposition 1.26 in Hatcher,
        $\pi_1 \left( X_{A_5} \right) \cong
        A_5$, and
        $H_1 \left( X_{A_5} \right) 
        \cong \text{ab}\left( A_5 \right) \cong 1 $.\\
        \linebreak
        Another example is the example from
        Exercise 5 in problem set 2: namely,
        the Poincaré homology sphere. We showed that
        $H_1 \left( S^3 / 2 I ; \mathbb{Z} \right) \cong
        0$ while
        $\pi_1 \left( S^3 / 2 I  \right) \cong 2 I \not \cong 1$.
        Furthermore, we showed that
        $S^3 / 2I$ is a manifold, hence admits a CW complex
        structure, and furthermore, as the quotient of
        a path-connected space, it is also path-connected, so
        $S^3 / 2I$ satisfies all the 
        criterions of the problem.\\
        \linebreak

        (2) We want to attach cells to $X$ to obtain
        a $CW$-complex $Y$ which is simply
        connected and induce an isomorphism on
        homology in degrees $\ge 3$ under the inclusion.
        To do this, suppose
        $f \colon \left( S^{1}, s_0 \right) 
        \to \left( X, x_0 \right) $ is
        a nontrivial element in $\pi_1 (X, x_0)$. We can assume
        by the Cellular Approximation Theorem that
        $f$ is cellular. Then
        we can attach a $2$-cell along $f$ which renders
        $f$ based nullhomotopic. Attaching $2$-cells for
        each nontrivial element in $\pi_1(X)$ like this simultaneously,
        we let $Y$ be the resulting space.
        Then we claim that $\pi_1 (Y) \cong 0$.
        To see this, suppose 
        $g \colon \left( S^{1}, s_0 \right) \to 
        \left( Y, x_0 \right) $ is some map. By
        giving $S^{1}$ the standard CW stucture of a single
        $0$-cell which is $s_0$ and a single $1$-cell attached,
        we get by cellular approximation, that
        $g$ is based homotopic to 
        a map $\tilde{g} \colon \left( S^{1}, s_0 \right) 
        \to \left( Y, x_0 \right) $ which has image
        in $X$. Thus $\tilde{g}$ represents
        an element of $\pi_1 \left( X, x_0 \right) $,
        but by construction of $Y$, $\tilde{g}$ is then
        based nullhomotopic. Composing these homotopies,
        we find that $g$ is based nullhomotopic, so
        $\pi_1 \left( Y \right) \cong 0$.\\
        \linebreak
        
        It remains to show that the inclusion induces
        isomorphisms in homology in degrees $\ge 3$.
        Let $I$ be an indexing set
        for the attaching maps of the
         $2$-cells 
         $\left\{ e_{\alpha}^2 \right\}_{\alpha \in I}$
         that we attached to obtain $Y$ from $X$.
         Let also $A_n$ be an indexing set for
         the $n$-cells in the CW structure of $X$ (we can also view
         $A_n$ as an indexing set for the
         $n$-simplices in the $\Delta$-complex structure obtained
         from $X$ using its CW structure).
         In either case, we obtain a chain complex
         from this CW/$\Delta$-complex structure 
         along with a chain map induced
         by the inclusion $X \hookrightarrow Y$ which
         is the identity in all degrees except degree
         $2$ :
         \begin{equation*}
    \begin{tikzcd}
        \ldots \ar[r] &\bigoplus_{A_4} \mathbb{Z} \ar[r,
        "\partial_4^{X}"] \ar[d, equal] 
        & \bigoplus_{A_3} \mathbb{Z} \ar[r, "\partial_3^{X}"]
        \ar[d, equal] &
        \bigoplus_{A_2} \mathbb{Z} \ar[d, hookrightarrow] \ar[r,
        "\partial_2^{X}"] &
        \bigoplus_{A_1} \mathbb{Z} \ar[d, equal] 
        \ar[r, "\partial_1^{X}"] & \ldots \\
        \ldots \ar[r] &\bigoplus_{A_4}\mathbb{Z} \ar[r,
        "\partial_4^{Y}"]
        & \bigoplus_{A_3} \mathbb{Z} \ar[r, "\partial_3^{Y}"] &
        \bigoplus_{A_2 \sqcup I} \mathbb{Z} \ar[r, "\partial_2^{Y}"]
        & 
        \bigoplus_{A_1} \mathbb{Z} \ar[r, "\partial_1^{Y}"] & \ldots
    \end{tikzcd}
    \end{equation*}
    Now, recalling that the induced
    map  $\iota_{*} \colon
    H_n \left( X \right) \to 
    H_n (Y)$ is given by
    $\left[ c \right] \mapsto 
    \left[ \iota \circ c \right] $, 
    the maps on homology in degrees $ \ge 3$ will
    simply be the identity since
    for any $n \ge 3$, $\partial_n^{Y} = 
    \partial_n^{X}$, so
    \[
    H_n (Y) = \ker \partial_n^{Y} / \im \partial_{n+1}^{Y}
    = \ker \partial_n^{X} / \im \partial_{n+1}^{X}
    = H_n (X).
    \] 
    (3) Using the LES of the pair $\left( Y,X \right) $, we
    find that
    \[
    H_3 \left( Y,X \right) \stackrel{\partial_*}{\to} 
    H_2 \left( X \right) \stackrel{i_*}{\to} 
    H_2 (Y) \stackrel{j_*}{\to}  H_2 (Y,X) \stackrel{\partial_*}{\to} 
    H_1 \left( X \right) 
    \] 
    is exact. 
    Now, note that since $X$ is a CW subcomplex, it
    is, in particular, closed and
    the inclusion  $X \hookrightarrow Y$ is a cofibration,
    so the quotienting map
    $\left( Y,X \right) \to \left( Y / X, * \right) $ 
    induces an isomorphism
    $H_* \left( Y,X \right) \cong
    H_* \left( Y / X, * \right) \cong
    \tilde{H}_* (Y / X)$
    (Corollary 1.7 together with
    Corollary 1.4, Chapter VII in Bredon's Topology and Geometry).
    Now, $Y / X$ is a wedge of
    $2$-spheres, so 
    $\tilde{H}_3 (Y / X) \cong 0$ by considering
    its chain in cellular or simplicial homology.
    As for $H_1 (X)$, this vanishes by assumption on the
    space $X$, so we finally obtain that
    
    \[
    0 \to H_2 \left( X \right) \stackrel{i_*}{\to} 
    H_2 (Y) \stackrel{j_*}{\to}  H_2 (Y,X) \to 0
    \] 
    is a SES.
    Now, using the exact same argument as above,
    $H_2 \left( Y,X \right) \cong
    \tilde{H}_2 (Y /X)$ and $Y / X$ is a wedge of
    $2$-spheres indexed by $I$, so 
    $\tilde{H}_2 \left( Y / X \right) \cong
    \bigoplus_{I} \mathbb{Z}$.
    In particular, this is a free abelian group, and
    we can let $A$ be a set of generators
    for this free summand.
    Since any free group is projective,
    this SES splits, so we find that
    \[
    H_2 (Y) \cong H_2 (X) \oplus H_2 (Y,X)
    \cong H_2 (X) \oplus \bigoplus_{I} \mathbb{Z}
    \] 


    (4) Since $Y$ is simply-connected, the Hurewicz
    theorem gives us an isomorphism
    $h \colon \pi_2 (Y) \to H_2 (Y)$ given
    by sending $f \colon \left( S^2, s_0 \right) 
    \to \left( Y,x_0 \right) $ to
    $h \left( \left[ f \right]  \right) =
    f_* \left( \alpha \right) $ where $\alpha$ is
    a generator of $H_2 \left( S^2 \right) $.
    In particular, we can represent each
    basis element $\alpha$ in $A$ by some
    map $\psi_{\alpha} \colon \left( S^2,s_0 \right) 
    \to \left( Y,x_0 \right) $ by pulling $\alpha$ back
    along the Hurewicz isomorphism.
    By the Cellular Approximation Theorem, we may
    again assume that each  $\psi_{\alpha}$ is cellular
    (giving $S^2$ its standard CW structure).
    Now we let $Z$ be the space obtained
    by attaching $3$-cells to $Y$ along the maps
    $\left\{ \psi_{\alpha} \right\}_{\alpha \in A}$.\\
    \linebreak
    (5) 

    The inclusions $X \hookrightarrow Y \hookrightarrow Z$ now
    give the following maps of chain complexes:
    
         \begin{equation*}
    \begin{tikzcd}
        \ldots \ar[r] &\bigoplus_{A_4} \mathbb{Z} \ar[r,
        "\partial_4^{X}"] \ar[d, equal] 
        & \bigoplus_{A_3} \mathbb{Z} \ar[r, "\partial_3^{X}"]
        \ar[d, equal] &
        \bigoplus_{A_2} \mathbb{Z} \ar[d, hookrightarrow] \ar[r,
        "\partial_2^{X}"] &
        \bigoplus_{A_1} \mathbb{Z} \ar[d, equal] 
        \ar[r, "\partial_1^{X}"] & \ldots \\
        \ldots \ar[r] &\bigoplus_{A_4}\mathbb{Z} \ar[r,
        "\partial_4^{Y}"] \ar[d, equal]
        & \bigoplus_{A_3} \mathbb{Z} \ar[r, "\partial_3^{Y}"] 
        \ar[d, hookrightarrow]&
        \bigoplus_{A_2 \sqcup I} \mathbb{Z} \ar[r, "\partial_2^{Y}"]
        \ar[d, equal] & 
        \bigoplus_{A_1} \mathbb{Z} \ar[r, "\partial_1^{Y}"]
        \ar[d, equal]& \ldots
        \\
        \ldots \ar[r] &\bigoplus_{A_4}\mathbb{Z} \ar[r,
        "\partial_4^{Z}"]
        & \bigoplus_{A_3 \sqcup I} 
        \mathbb{Z} \ar[r, "\partial_3^{Z}"] &
        \bigoplus_{A_2 \sqcup I} \mathbb{Z} \ar[r, "\partial_2^{Z}"]
        & 
        \bigoplus_{A_1} \mathbb{Z} \ar[r, "\partial_1^{Z}"] & \ldots
    \end{tikzcd}
    \end{equation*}
    By the exact same reasoning as before,
    since $\partial_{n}^{Z} = \partial_n^{Y} = 
    \partial_n^{X}$ for $n\ge 4$, it follows
    that $H_n (Z) = H_n(Y) = H_n(X)$ for $n\ge 4$ with
    the inclusions again, by the exact same reasoning as
    in (2), inducing the isomorphisms (in fact, equalities,
    and the inclusions simply become the identity).
    For $n= 1$, we have that  $H_1(X) = H_1(Y) =  0$, and
    so since 
    \[
    H_1 (Z) = \ker \partial_1^{Z} / \im \partial_2^{Z}
    = \ker \partial_1^{Y} / \im \partial_2^{Y} = 
    H_1 (Y)
    \] 
    we also find that
    $H_1 (Z) = 0$,
    so the inclusion $X \hookrightarrow Y \hookrightarrow Z$ 
    trivially induces an isomorphism
    $H_1 (X) \to H_1(Z)$.\\
    For $n = 2$, note that we have the following commutative
    diagram (which commutes by naturality of the Hurewicz
    isomorphism):
    \begin{equation*}
    \begin{tikzcd}
        \pi_2 (X) \ar[r, "i_{*}"] \ar[d, "h", "\cong"']
        & \pi_2(Y) \ar[r, "j_*"] \ar[d, "h", "\cong"'] &
        \pi_2 (Z) \ar[d, "h", "\cong"'] \\
        H_2(X) \ar[r, "i_*"] & H_2(Y) 
        \cong H_2(X) \oplus \bigoplus_{I}\mathbb{Z}
        \ar[r, "j_*"] & H_2(Z)
    \end{tikzcd}
    \end{equation*}
    First, recall that the splitting
    \[
    H_2(Y) \cong H_2(X) \oplus \bigoplus_I \mathbb{Z}
    \] 
    was given by
    $\left( \alpha, \beta \right) \mapsto 
    i_* \left( \alpha \right) +
    s \left( \beta \right) $ where
    $s$ is the section for
    $H_2 (Y) \stackrel{j_*}{\to} H_2 (Y,X)$, so
    in particular, the
    inclusion
    $X \hookrightarrow Y$, becomes the inclusion
    $H_2(X) \hookrightarrow H_2(X) \oplus \bigoplus_{I} \mathbb{Z}$ 
    into the $H_2(X)$ factor under this isomorphism.
 
    In this diagram,
    each element $\alpha \in A$ is mapped
    to some representative $\left( S^2, s_0 \right) 
    \to \left( Y, x_0 \right) $ in
    $\pi_2 (Y)$ which, by construction, is based nullhomotopic
    in $Z$, so we see that
    $j_* \circ h^{-1} (\alpha) = 0$, hence
    $j_* \left( \alpha \right) =
    h \circ j_* \circ h^{-1}(\alpha) = 0$.
    Meanwhile, any element in
    $H_2 (X) \subset H_2(X) \oplus \bigoplus_{I}\mathbb{Z}$ 
    is pulled back along $h$ to a nontrivial element
    which has nonzero image under
    $j_*$ (by construction, $Z$ eliminates only
    the elements of $A$ ). Thus
    $j_* \colon
    H_2(Y) \cong H_2(X) \oplus
    \bigoplus_{I}\mathbb{Z} \to H_2(Z)$ is injective
    on the $H_2(X)$ factor.\\
    hence $j_* i_* = \left( j \circ i \right)_* \colon
    H_2 (X) \to H_2 (Z)$ is injective. Now,
    for any nontrivial $\gamma \in H_2 (Z)$,
    this pulls back to a nontrivial element in
    $\pi_2(Z)$ which is the image of some
    $\beta \in \pi_2 (Y)$ under $j_*$ since $j_*\colon
    \pi_2(Y) \to \pi_2(Z)$ is surjective. 
    This maps down to some element
    $\left( x,y \right) \in 
    H_2(X) \oplus \bigoplus_{I}\mathbb{Z}$, and
    since $j_*$ is $0$ on all factors in
    $\bigoplus_{I}\mathbb{Z}$, we find that
    $j_* (x,0) = \gamma$. So
    $\left( j \circ i \right)_* (x) = 
    j_* \left( x,0 \right) = \gamma$, which shows
    that $\left( j \circ i \right)_*$ is also surjective.
    Thus $\left( j \circ i \right)_* \colon
    H_2 (X) \to H_2(Z)$ is an isomorphism.\\
    \linebreak
    Lastly, for $n=3$, it suffices to show that the
    inclusion $Y \hookrightarrow Z$ induces an isomorphism
    $H_3(Y) \to H_3(Z)$ since we already showed in
    (2) that $H_3(X) \to H_3(Y)$ induced by the inclusion
    is an isomorphism can be seen as follows:
    for each basis element in the $I$ part of
    the summand of the domain of $\partial_3^{Z}$ :
    $\bigoplus_{A_3 \sqcup I} \mathbb{Z}$, this
    is by construction of $Z$ mapped
    to an element in $A$ under $\partial_3^{Z}$ which
    is nontrivial in
    $\bigoplus_{A_2 \sqcup I}\mathbb{Z}$, hence
    $\ker \partial_3^{Z} = \ker \partial_3^{Y}$, so
    $H_3 (Z) = H_3(Y)$ and hence
    the inclusion induces the identity which is
    an isomorphism.\\
    \linebreak
    This completes the proof.
    \end{proof}


