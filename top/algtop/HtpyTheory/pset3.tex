\documentclass[reqno]{amsart}
\usepackage{amscd, amssymb, amsmath, amsthm}
\usepackage{graphicx}
\usepackage[colorlinks=true,linkcolor=blue]{hyperref}
\usepackage[utf8]{inputenc}
\usepackage[T1]{fontenc}
\usepackage{textcomp}
\usepackage{babel}
%% for identity function 1:
\usepackage{bbm}
%%For category theory diagrams:
\usepackage{tikz-cd}

%\usepackage[backend=biber]{biblatex}
%\addbibresource{.bib}


\setlength\parindent{0pt}

\pdfsuppresswarningpagegroup=1

\newtheorem{theorem}{Theorem}[section]
\newtheorem{lemma}[theorem]{Lemma}
\newtheorem{proposition}[theorem]{Proposition}
\newtheorem{corollary}[theorem]{Corollary}
\newtheorem{conjecture}[theorem]{Conjecture}

\theoremstyle{definition}
\newtheorem{definition}[theorem]{Definition}
\newtheorem{example}[theorem]{Example}
\newtheorem{exercise}[theorem]{Exercise}
\newtheorem{problem}[theorem]{Problem}
\newtheorem{question}[theorem]{Question}

\theoremstyle{remark}
\newtheorem*{remark}{Remark}
\newtheorem*{note}{Note}
\newtheorem*{solution}{Solution}



%Inequalities
\newcommand{\cycsum}{\sum_{\mathrm{cyc}}}
\newcommand{\symsum}{\sum_{\mathrm{sym}}}
\newcommand{\cycprod}{\prod_{\mathrm{cyc}}}
\newcommand{\symprod}{\prod_{\mathrm{sym}}}

%Linear Algebra

\DeclareMathOperator{\Span}{span}
\DeclareMathOperator{\im}{im}
\DeclareMathOperator{\diag}{diag}
\DeclareMathOperator{\Ker}{Ker}
\DeclareMathOperator{\ob}{ob}
\DeclareMathOperator{\Hom}{Hom}
\DeclareMathOperator{\Mor}{Mor}
\DeclareMathOperator{\sk}{sk}
\DeclareMathOperator{\Vect}{Vect}
\DeclareMathOperator{\Set}{Set}
\DeclareMathOperator{\Group}{Group}
\DeclareMathOperator{\Ring}{Ring}
\DeclareMathOperator{\Ab}{Ab}
\DeclareMathOperator{\Top}{Top}
\DeclareMathOperator{\hTop}{hTop}
\DeclareMathOperator{\Htpy}{Htpy}
\DeclareMathOperator{\Cat}{Cat}
\DeclareMathOperator{\CAT}{CAT}
\DeclareMathOperator{\Cone}{Cone}
\DeclareMathOperator{\dom}{dom}
\DeclareMathOperator{\cod}{cod}
\DeclareMathOperator{\Aut}{Aut}
\DeclareMathOperator{\Mat}{Mat}
\DeclareMathOperator{\Fin}{Fin}
\DeclareMathOperator{\rel}{rel}
\DeclareMathOperator{\Int}{Int}
\DeclareMathOperator{\sgn}{sgn}
\DeclareMathOperator{\Homeo}{Homeo}
\DeclareMathOperator{\SHomeo}{SHomeo}
\DeclareMathOperator{\PSL}{PSL}
\DeclareMathOperator{\Bil}{Bil}
\DeclareMathOperator{\Sym}{Sym}
\DeclareMathOperator{\Skew}{Skew}
\DeclareMathOperator{\Alt}{Alt}
\DeclareMathOperator{\Quad}{Quad}
\DeclareMathOperator{\Sin}{Sin}
\DeclareMathOperator{\Supp}{Supp}
\DeclareMathOperator{\Char}{char}
\DeclareMathOperator{\Teich}{Teich}
\DeclareMathOperator{\GL}{GL}
\DeclareMathOperator{\tr}{tr}
\DeclareMathOperator{\codim}{codim}
\DeclareMathOperator{\coker}{coker}
\DeclareMathOperator{\corank}{corank}
\DeclareMathOperator{\rank}{rank}
\DeclareMathOperator{\Diff}{Diff}
\DeclareMathOperator{\Bun}{Bun}
\DeclareMathOperator{\Sm}{Sm}
\DeclareMathOperator{\Fr}{Fr}
\DeclareMathOperator{\Cob}{Cob}
\DeclareMathOperator{\Ext}{Ext}
\DeclareMathOperator{\Tor}{Tor}
\DeclareMathOperator{\colim}{colim}



%Row operations
\newcommand{\elem}[1]{% elementary operations
\xrightarrow{\substack{#1}}%
}

\newcommand{\lelem}[1]{% elementary operations (left alignment)
\xrightarrow{\begin{subarray}{l}#1\end{subarray}}%
}

%SS
\DeclareMathOperator{\supp}{supp}
\DeclareMathOperator{\Var}{Var}

%NT
\DeclareMathOperator{\ord}{ord}

%Alg
\DeclareMathOperator{\Rad}{Rad}
\DeclareMathOperator{\Jac}{Jac}

%Misc
\newcommand{\SL}{{\mathrm{SL}}}
\newcommand{\mobgp}{{\mathrm{PSL}_2(\mathbb{C})}}
\newcommand{\id}{{\mathrm{id}}}
\newcommand{\MCG}{{\mathrm{MCG}}}
\newcommand{\PMCG}{{\mathrm{PMCG}}}
\newcommand{\SMCG}{{\mathrm{SMCG}}}
\newcommand{\ud}{{\mathrm{d}}}
\newcommand{\Vol}{{\mathrm{Vol}}}
\newcommand{\Area}{{\mathrm{Area}}}
\newcommand{\diam}{{\mathrm{diam}}}
\newcommand{\End}{{\mathrm{End}}}


\newcommand{\reg}{{\mathtt{reg}}}
\newcommand{\geo}{{\mathtt{geo}}}

\newcommand{\tori}{{\mathcal{T}}}
\newcommand{\cpn}{{\mathtt{c}}}
\newcommand{\pat}{{\mathtt{p}}}

\let\Cap\undefined
\newcommand{\Cap}{{\mathcal{C}}ap}
\newcommand{\Push}{{\mathcal{P}}ush}
\newcommand{\Forget}{{\mathcal{F}}orget}




\begin{document}
    \begin{problem}[]
        Let $Y$ be a simply-connected CW-complex. Assume
        there exists a finite wedge of spheres
        $\bigvee_{i} S^{n_i}$ together with maps
        $i \colon Y \to \bigvee_{i} S^{n_i},
        r\colon \bigvee_{i} S^{n_i} \to Y$ such that
        $r \circ i$ is homotopic to $\id_Y$.
        Prove that $Y$ is homotopy equivalent to some
        finite wedge of spheres 
        $\bigvee_{j} S^{m_j}$.
    \end{problem}

    \begin{proof}
        Since $r \circ \iota \simeq \id_Y$, we
        have that the induced map
        $r_* \colon H_n \left( \bigvee_i S^{n_i} \right) 
        \to H_n (Y)$ is surjective for all $n$. We want
        to make use of the Corollary 4.33 in Hatcher which says:
        \begin{corollary}[4.33 Hatcher]
            A map $f \colon X \to Y$ between simply-connected
            CW complexes is a homotopy equivalence if 
            $f_* \colon H_n (X) \to H_n(Y)$ is an isomorphism
            for each $n$.
        \end{corollary}
        By assumption $Y$ is a simply-connected CW complex,
        and $\bigvee_{i} S^{n_i}$ is also a simply-connected
        CW complex. Attaching cells of
        dimension $\ge 2$ to $\bigvee_{i}S^{n_i}$ 
        does not change either of these properties - i.e., the
        space remains a simply-connected CW complex.\\
        We would like to
        modify $\bigvee_i S^{n_i}$ such that
        $r_*$ becomes injective also on each
        homology group.
        Let $\left[ f \right] 
        \in H_n \left( \bigvee_i S^{n_i} \right) $ 
        be in the kernel of $r_*$.
        For the sake 

        


        \newpage
        Since $r \circ i \simeq \id_Y$, we have
        $r_* \circ i_* = \id_{\pi_n (Y)}$ for any $n \ge 0$, so
        $r_* \colon \pi_n \left( \bigvee_i S^{n_i} \right) 
        \to \pi_n (Y)$ is surjective for each $n$.
        We will extend $r$ inductively to obtain
        a weak homotopy equivalence
        $\bigvee_j S^{m_j} \to Y$. By Whitehead's theorem, it
        will then follow that
        $\bigvee_j S^{m_j} \simeq Y$.\\
        Now, $r_*$ is surjective on
        $\pi_0$, and since both
        $\bigvee_i S^{n_i}$ and $Y$ are path-connected,
        $r_*$ is, in fact, bijective on $\pi_0$.\\
        
        We will now inductively attach cells onto
        $\bigvee_{i}S^{n_i}$ so as to make
        $r_*$ a weak homotopy equivalence. 
        In particular, we will inductively attach
         cells by the constant attaching map
         $e_{\alpha}^{k} \to \bigvee_{i} S^{n_i}$ to
         the base point of $\bigvee_i S^{n_i}$. But this
         will simply be a wedge of spheres again, so
         if we let $Z_k$ denote the space obtained after
         the $k$ th inductive step (i.e., after having attached
         $n$-cells for $n= 1,\ldots, k$), then 
         $Z_k$ will again have the form
         $Z_k = \bigvee_{j} S^{n_j}$.\\
        Suppose we have already made 
        $r_*$ an isomorphism on
        $\pi_n$ for $n = 0, \ldots, k-1$.\\
        Next, choose maps
        $\varphi_{\alpha} \colon \left( S^{k}, s_0 \right) \to 
        \left( \bigvee_{i} S^{n_i}, * \right)$ representing
        all nontrivial elements of the kernel of
        $r_* \colon \pi_{k} (\bigvee_i S^{n_i}) \to 
        \pi_k \left( Y \right)  $. By the Cellular Approximation
        Theorem, if we let $S^{k}$ have the
        CW structure with $s_0$ being the single $0$-cell with a
        $k$-cell attached, then $\varphi_{\alpha}$ may
        be assumed to be cellular. Next
        we can attach $e_{\alpha}^{k+1}$ cells to
        $Z_k= \bigvee_i S^{n_i}$ via the maps $\varphi_{\alpha}$ which
        produces a new CW complex $Z_{k+1}$, which still is
        of the form $\bigvee_i S^{n_i}$, so we shall continue
        to denote $Z_k = \bigvee_i S^{n_i}$. Since
        $r \circ \varphi_{\alpha} $ is based nullhomotopic by
        construction, $r$ extends over the new cells, so
        $r$ extends to a map
        $Z_{k+1} \to Y$. Note that
        by assumption, $r_*$ was an isomorphism
        on $\pi_n$ for $n\le k-1$, and attaching
        $k+1$-cells to $Z_{k}$ has not changed this (the same
        maps from before still work for surjectivity).
        However, we now claim that $r_*$ is injective on
        $\pi_k$ also.
        Suppose $h \colon \left( S^{k}, s_0 \right) 
        \to \left( Z_{k+1}, * \right) $ represents
        an element of the kernel
        of $r_* \colon \pi_{k} \left( Z_{k+1} \right) \to 
        \pi_k\left( Y \right) $. By the Cellular Approximation Theorem
        again (using the standard CW structure on $S^{k}$ as above),
        we may assume that $h$ is cellular, so
        in particular, the image of
        $h$ lies in $Z_{k}$, and thus
        $h$ is in the kernel
        of $r_* \colon
        \pi_{k}(Z_k) \to \pi_k(Y)$ 
        and is thus based homotopic to some $\varphi_{\alpha}$,
        which is based nullhomotopic in $Z_{k+1}$, so
        $h$ represents zero in
        $\pi_k \left( Z_{k+1} \right) $.
        Thus the kernel
        of $r_* \colon \pi_k \left( Z_{k+1} \right) 
        \to \pi_k (Y)$ is trivial, so
        $r$ induces isomorphisms on $\pi_n$ for 
        $n\le k$ now.\\
        Note now that
        since $\bigvee_i S^{n_i} = Z_0$ was assumed 
        to be a finite wedge of spheres, so
        there exists largest $n_M$ in the wedge.
        In particular, then for any $k > n_M$ and
        any map $\left( S^{k}, s_0 \right) \to 
        (Z_{n_M}, *)$ is based homotopic to a cellular
        map by the Cellular Approximation Theorem, and
        hence maps all of $S^{k}$ 
    \end{proof}


    \begin{problem}[]
        Let $X$ be a path-connected CW complex such that
        $H_1 \left( X ; \mathbb{Z} \right) = 0$.
        The goal of this problem is to construct a
        simply connected space $Z$ and a map
        $X \to Z$ inducing an isomorphism in homology.
        \begin{enumerate}
            \item Give an example of such $X$ such that
                $\pi_1 \left( X \right) \neq 1$.
            \item Consider a set of generators
                for $\pi_1 (X)$, construct another
                CW complex $Y$ by attaching cells to $X$,
                so that
                \begin{itemize}
                    \item $Y$ is simply connected.
                    \item The inclusion $X \subset Y$ 
                        induces an isomorphism on homology
                        in degrees $\ge 3$.
                \end{itemize}
            \item Show that $H_2 \left( Y ; \mathbb{Z} \right) $ 
                is a sum of $H_2 \left( X; \mathbb{Z} \right) $ 
                together with a free abelian group.
                Let $A$ be a set of generators for this
                free summand.
        \end{enumerate}
    \end{problem}

    \textbf{See Prop 4.40 in Hatcher}

    \begin{proof}
        (1) Since $H_1$ is just the abelianization for
        $\pi_1$ for path-connected spaces,
        this is equivalent to finding
        a path-connected CW complex
        $X$ whose fundamental group is nontrivial, but
        becomes trivial when abelianized. 
        By corollary 1.28 in Hatcher, for any
        group $G$, we can construct a $2$-dimensional CW
        complex $X_G$ such that $\pi_1 (X_G) \cong G$.
        So it suffices to find a nontrivial group whose
        abelianization is trivial. Such a group is
        called a perfect group, and we have
        many examples of such groups. For example,
        any non-abelian simple group is perfect, so
        for example $A_5$ is perfect. 
        The construction of $X_{A_5}$  can now be
        carried out as follows: $A_5$ is 
        generated by
        $\left( 123 \right) $ and
        $\left( 12345 \right) $ which do not commute,
        so we can express (as with any other group)
        $A_5$ as
        \[
        A_5 = \left< g_{\alpha} \mid r_{\beta} \right>
        \] 
        So in this case, the number of generators is simply $2$.
        Then we can construct $X_{A_5}$ from
        $\bigvee_{\alpha} S^{1}$ by attaching
        $2$-cells $e_{\beta}^2$ by the loops specified by
        the words $r_{\beta}$. By Proposition 1.26 in Hatcher,
        $\pi_1 \left( X_{A_5} \right) \cong
        A_5$, and
        $H_1 \left( X_{A_5} \right) 
        \cong \text{ab}\left( A_5 \right) \cong 1 $.\\
        \linebreak
        (2) We want to attach cells to $X$ to obtain
        a $CW$-complex $Y$ which is simply
        connected and induce an isomorphism on
        homology in degrees $\ge 3$ under the inclusion.
        To do this, suppose
        $f \colon \left( S^{1}, s_0 \right) 
        \to \left( X, x_0 \right) $ is
        in $\pi_1 (X, x_0)$. We can assume
        by the Cellular Approximation Theorem that
        $f$ is cellular. Then
        we can attach a $2$-cell along $f$ which renders
        $f$ based nullhomotopic. Attaching $2$-cells for
        each nontrivial element in $\pi_1(X)$ like this simultaneously,
        we let $Y$ be the resulting space.
        Then we claim that $\pi_1 (Y) \cong 0$.
        To see this, suppose 
        $g \colon \left( S^{1}, s_0 \right) \to 
        \left( Y, x_0 \right) $ is some map. By
        giving $S^{1}$ the standard CW stucture of a single
        $0$-cell which is $s_0$ and a single $1$-cell attached,
        we get by cellular approximation, that
        $g$ is based homotopic to 
        a map $\tilde{g} \colon \left( S^{1}, s_0 \right) 
        \to \left( Y, x_0 \right) $ which has image
        in $X$. Thus $\tilde{g}$ represents
        an element of $\pi_1 \left( X, x_0 \right) $,
        but by construction of $Y$, $\tilde{g}$ is then
        based nullhomotopic. Composing these homotopies,
        we find that $g$ is based nullhomotopic, so
        $\pi_1 \left( Y \right) \cong 0$.\\
        \linebreak
        
        It remains to show that the inclusion induces
        isomorphisms in homology in degrees $\ge 3$.
        Let $I$ be an indexing set
        for the attaching maps of the
         $2$-cells 
         $\left\{ e_{\alpha}^2 \right\}_{\alpha \in I}$
         that we attached to obtain $Y$ from $X$.
         Let also $A_n$ be an indexing set for
         the $n$-cells in the CW structure of $X$ (we can also view
         $A_n$ as an indexing set for the
         $n$-simplices in the $\Delta$-complex structure obtained
         from $X$ using its CW structure).
         In either case, we obtain a chain complex
         from this CW/$\Delta$-complex structure 
         along with a chain map induced
         by the inclusion $X \hookrightarrow Y$ which
         is the identity in all degrees except degree
         $2$ :
         \begin{equation*}
    \begin{tikzcd}
        \ldots \ar[r] &\bigoplus_{A_4} \mathbb{Z} \ar[r,
        "\partial_4^{X}"] \ar[d, equal] 
        & \bigoplus_{A_3} \mathbb{Z} \ar[r, "\partial_3^{X}"]
        \ar[d, equal] &
        \bigoplus_{A_2} \mathbb{Z} \ar[d, hookrightarrow] \ar[r,
        "\partial_2^{X}"] &
        \bigoplus_{A_1} \mathbb{Z} \ar[d, equal] 
        \ar[r, "\partial_1^{X}"] & \ldots \\
        \ldots \ar[r] &\bigoplus_{A_4}\mathbb{Z} \ar[r,
        "\partial_4^{Y}"]
        & \bigoplus_{A_3} \mathbb{Z} \ar[r, "\partial_3^{Y}"] &
        \bigoplus_{A_2 \sqcup I} \mathbb{Z} \ar[r, "\partial_2^{Y}"]
        & 
        \bigoplus_{A_1} \mathbb{Z} \ar[r, "\partial_1^{Y}"] & \ldots
    \end{tikzcd}
    \end{equation*}
    Now, recalling that the induced
    map  $\iota_{*} \colon
    H_n \left( X \right) \to 
    H_n (Y)$ is given by
    $\left[ c \right] \mapsto 
    \left[ \iota \circ c \right] $, 
    the maps on homology in degrees $ \ge 3$ will
    simply be the identity since
    for any $n \ge 3$, $\partial_n^{Y} = 
    \partial_n^{X}$, so
    \[
    H_n (Y) = \ker \partial_n^{Y} / \im \partial_{n+1}^{Y}
    = \ker \partial_n^{X} / \im \partial_{n+1}^{X}
    = H_n (X).
    \] 
    (3) Using the LES of the pair $\left( Y,X \right) $, we
    find that
    \[
    H_3 \left( Y,X \right) \stackrel{\partial_*}{\to} 
    H_2 \left( X \right) \stackrel{i_*}{\to} 
    H_2 (Y) \stackrel{j_*}{\to}  H_2 (Y,X) \stackrel{\partial_*}{\to} 
    H_1 \left( X \right) 
    \] 
    is exact. 
    Now, note that since $X$ is a CW subcomplex, it
    is, in particular, closed and
    the inclusion  $X \hookrightarrow Y$ is a cofibration,
    so the quotienting map
    $\left( Y,X \right) \to \left( Y / X, * \right) $ 
    induces an isomorphism
    $H_* \left( Y,X \right) \cong
    H_* \left( Y / X, * \right) \cong
    \tilde{H}_* (Y / X)$
    (Corollary 1.7 together with
    Corollary 1.4, Chapter VII in Bredon's Topology and Geometry).
    Now, $Y / X$ is a wedge of
    $2$-spheres, so 
    $\tilde{H}_3 (Y / X) \cong 0$ by considering
    its chain in cellular or simplicial homology.
    As for $H_1 (X)$, this vanishes by assumption on the
    space $X$, so we finally obtain that
    
    \[
    0 \to H_2 \left( X \right) \stackrel{i_*}{\to} 
    H_2 (Y) \stackrel{j_*}{\to}  H_2 (Y,X) \to 0
    \] 
    is a SES.
    Now, using the exact same argument as above,
    $H_2 \left( Y,X \right) \cong
    \tilde{H}_2 (Y /X)$ and $Y / X$ is a wedge of
    $2$-spheres indexed by $I$, so 
    $\tilde{H}_2 \left( Y / X \right) \cong
    \bigoplus_{I} \mathbb{Z}$.
    In particular, this is a free abelian group, and
    we can let $A$ be a set of generators
    for this free summand.
    Since any free group is projective,
    this SES splits, so we find that
    \[
    H_2 (Y) \cong H_2 (X) \oplus H_2 (Y,X)
    \cong H_2 (X) \oplus \bigoplus_{I} \mathbb{Z}
    \] 


    (4) Since $Y$ is simply-connected, the Hurewicz
    theorem gives us an isomor
    

 
    \end{proof}



    %\printbibliography
\end{document}
