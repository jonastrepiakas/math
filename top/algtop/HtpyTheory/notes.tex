\documentclass[reqno]{amsart}
\usepackage{amscd, amssymb, amsmath, amsthm}
\usepackage{graphicx}
\usepackage[colorlinks=true,linkcolor=blue]{hyperref}
\usepackage[utf8]{inputenc}
\usepackage[T1]{fontenc}
\usepackage{textcomp}
\usepackage{babel}
%% for identity function 1:
\usepackage{bbm}
%%For category theory diagrams:
\usepackage{tikz-cd}

\usepackage[backend=biber]{biblatex}
\addbibresource{notes.bib}


\setlength\parindent{0pt}

\pdfsuppresswarningpagegroup=1

\newtheorem{theorem}{Theorem}[section]
\newtheorem{lemma}[theorem]{Lemma}
\newtheorem{proposition}[theorem]{Proposition}
\newtheorem{corollary}[theorem]{Corollary}
\newtheorem{conjecture}[theorem]{Conjecture}

\theoremstyle{definition}
\newtheorem{definition}[theorem]{Definition}
\newtheorem{example}[theorem]{Example}
\newtheorem{exercise}[theorem]{Exercise}
\newtheorem{problem}[theorem]{Problem}
\newtheorem{question}[theorem]{Question}

\theoremstyle{remark}
\newtheorem*{remark}{Remark}
\newtheorem*{note}{Note}
\newtheorem*{idea}{Idea}
\newtheorem*{solution}{Solution}



%Inequalities
\newcommand{\cycsum}{\sum_{\mathrm{cyc}}}
\newcommand{\symsum}{\sum_{\mathrm{sym}}}
\newcommand{\cycprod}{\prod_{\mathrm{cyc}}}
\newcommand{\symprod}{\prod_{\mathrm{sym}}}

%Linear Algebra

\DeclareMathOperator{\Span}{span}
\DeclareMathOperator{\im}{im}
\DeclareMathOperator{\diag}{diag}
\DeclareMathOperator{\Ker}{Ker}
\DeclareMathOperator{\ob}{ob}
\DeclareMathOperator{\Hom}{Hom}
\DeclareMathOperator{\Mor}{Mor}
\DeclareMathOperator{\sk}{sk}
\DeclareMathOperator{\Vect}{Vect}
\DeclareMathOperator{\Set}{Set}
\DeclareMathOperator{\Group}{Group}
\DeclareMathOperator{\Ring}{Ring}
\DeclareMathOperator{\Ab}{Ab}
\DeclareMathOperator{\Top}{Top}
\DeclareMathOperator{\hTop}{hTop}
\DeclareMathOperator{\Htpy}{Htpy}
\DeclareMathOperator{\Cat}{Cat}
\DeclareMathOperator{\CAT}{CAT}
\DeclareMathOperator{\Cone}{Cone}
\DeclareMathOperator{\dom}{dom}
\DeclareMathOperator{\cod}{cod}
\DeclareMathOperator{\Aut}{Aut}
\DeclareMathOperator{\Mat}{Mat}
\DeclareMathOperator{\Fin}{Fin}
\DeclareMathOperator{\rel}{rel}
\DeclareMathOperator{\Int}{Int}
\DeclareMathOperator{\sgn}{sgn}
\DeclareMathOperator{\Homeo}{Homeo}
\DeclareMathOperator{\SHomeo}{SHomeo}
\DeclareMathOperator{\PSL}{PSL}
\DeclareMathOperator{\Bil}{Bil}
\DeclareMathOperator{\Sym}{Sym}
\DeclareMathOperator{\Skew}{Skew}
\DeclareMathOperator{\Alt}{Alt}
\DeclareMathOperator{\Quad}{Quad}
\DeclareMathOperator{\Sin}{Sin}
\DeclareMathOperator{\Supp}{Supp}
\DeclareMathOperator{\Char}{char}
\DeclareMathOperator{\Teich}{Teich}
\DeclareMathOperator{\GL}{GL}
\DeclareMathOperator{\tr}{tr}
\DeclareMathOperator{\codim}{codim}
\DeclareMathOperator{\coker}{coker}
\DeclareMathOperator{\corank}{corank}
\DeclareMathOperator{\rank}{rank}
\DeclareMathOperator{\Diff}{Diff}
\DeclareMathOperator{\Bun}{Bun}
\DeclareMathOperator{\Sm}{Sm}
\DeclareMathOperator{\Fr}{Fr}
\DeclareMathOperator{\Cob}{Cob}
\DeclareMathOperator{\Ext}{Ext}
\DeclareMathOperator{\Tor}{Tor}



%Row operations
\newcommand{\elem}[1]{% elementary operations
\xrightarrow{\substack{#1}}%
}

\newcommand{\lelem}[1]{% elementary operations (left alignment)
\xrightarrow{\begin{subarray}{l}#1\end{subarray}}%
}

%SS
\DeclareMathOperator{\supp}{supp}
\DeclareMathOperator{\Var}{Var}

%NT
\DeclareMathOperator{\ord}{ord}

%Alg
\DeclareMathOperator{\Rad}{Rad}
\DeclareMathOperator{\Jac}{Jac}

%Misc
\newcommand{\SL}{{\mathrm{SL}}}
\newcommand{\mobgp}{{\mathrm{PSL}_2(\mathbb{C})}}
\newcommand{\id}{{\mathrm{id}}}
\newcommand{\MCG}{{\mathrm{MCG}}}
\newcommand{\PMCG}{{\mathrm{PMCG}}}
\newcommand{\SMCG}{{\mathrm{SMCG}}}
\newcommand{\ud}{{\mathrm{d}}}
\newcommand{\Vol}{{\mathrm{Vol}}}
\newcommand{\Area}{{\mathrm{Area}}}
\newcommand{\diam}{{\mathrm{diam}}}
\newcommand{\End}{{\mathrm{End}}}


\newcommand{\reg}{{\mathtt{reg}}}
\newcommand{\geo}{{\mathtt{geo}}}

\newcommand{\tori}{{\mathcal{T}}}
\newcommand{\cpn}{{\mathtt{c}}}
\newcommand{\pat}{{\mathtt{p}}}

\let\Cap\undefined
\newcommand{\Cap}{{\mathcal{C}}ap}
\newcommand{\Push}{{\mathcal{P}}ush}
\newcommand{\Forget}{{\mathcal{F}}orget}


\title{Homotopy Theory}

\author{Jonas Trepiakas}
\date{}


\begin{document}
    
\maketitle

For these notes, we will follow \cite{Hatcher}, \cite{Bredon}
and \cite{ORW}.

%\section{Cofibrations}
For this section, we will follow chapter VII.1 in \cite{Bredon}.\\
\linebreak
One of the fundamental questions in topology is the
"extension problem". Namely, given a map
$g \colon A \to Y$ defined on a subspace $A$ of $X$, when
can we extend this map to all of $X$.

This cannot always be done - for example, as is the case
with $A = Y = S^{n}$ and $X = D^{n+1}$ choosing the
map to be any degree $-1$ map.\\
\linebreak
\begin{question}
    Is the extension problem a \textit{homotopy-theoretic} problem?
    That is, does the answer depend only on the homotopy
    class of $g$?
\end{question}
The answer is: generally not. 
For example, we can take $X = \left[ 0,1 \right] ,
A = \left\{ 0 \right\} \cup \left\{ \frac{1}{n} \mid 
n=1, 2, \ldots \right\} $ and $Y = CA$, the cone
on $A$. Choosing $g$ to be the inclusion of
$A$ into $Y$, this cannot be extended to $X$ as the
extension would be discontinuous at $\left\{ 0 \right\} $.
However, $g \simeq g'$ with $g'$ being the constant
map of $A$ to the vertex of the cone, and $g'$ easily
extends to $X$ by the constant map.\\
\linebreak
It turns out, however, that under some very mild conditions
on the spaces, the problem becomes homotopy theoretic. 
We will now discuss this.

\begin{definition}[Homotopy extension property]
    Let $\left( X,A \right) $ and $Y $ be given spaces.
    Then $\left( X, A \right) $ is said to have
    the \textit{homotopy extension property} with respect to
    $Y$ if the following diagram can always be completed
    to be commutative.
    \begin{equation*}
    \begin{tikzcd}
        A \times I \cup X \times \left\{ 0 \right\} 
        \ar[r] \ar[d, hookrightarrow] & Y\\
        X \times I \ar[ru, dashed]
    \end{tikzcd}
    \end{equation*}

    One can also depict this by the following diagram:
    \begin{equation*}
    \begin{tikzcd}
        A \times \left\{ 0 \right\} \ar[rr, hookrightarrow]
        \ar[dd, hookrightarrow] 
        & & A \times I \ar[dd, hookrightarrow] \ar[ld] \\
        & Y & \\
        X \times \left\{ 0 \right\} \ar[ru] \ar[rr] && X \times I
        \ar[lu, dashed]
    \end{tikzcd}
    \end{equation*}
\end{definition}

If $\left( X,A \right) $ has the homotopy extension property
with respect to $Y$, then the extensibility of maps
$g \colon A \to Y$ depends only on the homotopy class of
$g$. For suppose $H \colon g \simeq g'$ and $g'$ can be
extended to  $\tilde{g'} \colon X \to Y$, 
then define the map
$A \times I \cup  X \times \left\{ 0 \right\} $ by
$\tilde{g'} \times \left\{ 0 \right\} $ on
$X \times \left\{ 0 \right\} $ and
$H$ on $A \times I$. The homotopy extension property for the
pair $(X,A)$ then guarantees the existence of a map
$G \colon X \times I \to Y$ which equals
$g$ on $A \times \left\{ 1 \right\} $, so
$H \left( -,1 \right) \colon X \to Y$ extends $g$.

\begin{definition}[Cofibration]
    Let $f \colon A \to X$ be a map. Then $f$ is called
    a \textit{cofibration} if one can always fill in the following
    commutative diagram given the solid arrows:
    \begin{equation*}
    \begin{tikzcd}
        A \times \left\{ 0 \right\} \ar[dd, "f \times \id"] 
        \ar[rr, hookrightarrow]
        & & A\times I \ar[dd, "f \times \id"] \ar[dl] \\
            & Y &\\
        X \times \left\{ 0 \right\} \ar[ru]
        \ar[rr, hookrightarrow]& & X \times I 
        \ar[lu, dashed]
    \end{tikzcd}
    \end{equation*}
    for \textit{any} space $Y$.
\end{definition}

\begin{note}
    If $f$ is an inclusion, the this is the same
    as the homotopy extension property for all $Y$. That attribute
    is sometimes referred to as the 
    \textit{absolute homotopy extension property}.
\end{note}

\begin{lemma}[]\label{Lemma:Cofibration-Inclusion-2}
    If $f \colon A \to X$ is a cofibration, then
    the inclusion  $\iota \colon f(A) \hookrightarrow X$ 
    is a cofibration with $f\left( A \right) $ inheriting
    the subspace topology.
\end{lemma}

\begin{proof}
    If $f$ is a cofibration, then for any $Y$, there
    following diagram can be filled out given the solid arrows

    \begin{equation*}
    \begin{tikzcd}
        A \times \left\{ 0 \right\} \ar[dd, "f \times \id"] 
        \ar[rr, hookrightarrow]
        & & A\times I \ar[dd, "f \times \id"] \ar[dl] \\
            & Y &\\
        X \times \left\{ 0 \right\} \ar[ru]
        \ar[rr]& & X \times I 
        \ar[lu, dashed, "F"]
    \end{tikzcd}
    \end{equation*}
    And thus we can fill the following diagram as well

    \begin{equation*}
    \begin{tikzcd}
        f(A) \times \left\{ 0 \right\} \ar[dd, "\iota \times \id",
        hookrightarrow] 
        \ar[rr, hookrightarrow]
        & & f(A) \times I \ar[dd, "\iota \times \id", 
        hookrightarrow, bend left] \ar[dl, "F|_{f(A) \times I}"] \\
            & Y &\\
        X \times \left\{ 0 \right\} \ar[ru]
        \ar[rr, hookrightarrow]& & X \times I 
        \ar[lu, dashed, "F"]
    \end{tikzcd}
    \end{equation*}
    By definition then $\iota \colon
    f(A) \hookrightarrow X$ is a cofibration.
\end{proof}

\begin{note}
    Note that the converse is not true since
    we will see later in a problem that
    a cofibration is an embedding, so it is
    easy to construct a counter example, for example
    by choosing a well-pointed space (see definition later)
    and then choosing any space
    $A$ which is not a single point and the collapsing map
    $A \to X$ to the base point.
\end{note}

\begin{theorem}[]\label{Thm:Retract-cofibration}
    For an inclusion $A \subset X$, the following are equivalent:
    \begin{enumerate}
        \item The inclusion map $A \hookrightarrow X$ is a 
            cofibration.
        \item $A \times I \cup  X \times  \left\{ 0 \right\} $ 
            is a retract of $X \times I$.
    \end{enumerate}
\end{theorem}

\begin{proof}
    If the inclusion is a cofibration, then choosing
    $Y = A \times I \cup  X \times \left\{ 0 \right\} $ 
    with all arrows being inclusions in the
    diagram of a cofibration, we obtain a map
    $X \times I \to A \times I \cup  X \times \left\{ 0 \right\} $ 
    which is the identity on
    $A \times I \cup  X \times \left\{ 0 \right\} $.\\
    Conversely, if $A \times I \cup  X \times \left\{ 0 \right\} $ 
    is a retract of $X \times I$, then
    we can always complete the diagram by
    mapping $X \times I \to 
    A \times I \cup  X \times  \left\{ 0 \right\} 
    \to Y$ where the second map
    takes the maps $A \times I \to Y$ and
    $X \times \left\{ 0 \right\} \to Y$ from the diagram.
\end{proof}

\begin{corollary}\label{Cor:Subcomplex-Cofibration}
    If $A$ is a subcomplex of a CW-complex $X$, then
    the inclusion $A \hookrightarrow X$ is a cofibration.
\end{corollary}

\begin{proof}
    We want to construct a retraction
    $X \times I \to A \times I \cup  X \times \left\{ 0 \right\} $.
    We will do so by constructing a retraction
    $\left( \left( A \cup X^{(r)}  \right)\times I  \right) \cup 
    \left( X \times \left\{ 0 \right\}  \right) 
    \to \left( A \times I \right) \cup \left( X \times 
    \left\{ 0 \right\} \right) $ by induction on $r$.
    If it has been defined on the
    $(r-1)$-skeleton, then extending it over an
    $r$-cell is simply a matter of extending a map
    on $S^{r-1} \times I \cup D^{r} \times \left\{ 0 \right\} $ 
    over $D^{r} \times I$ which can be done
    since the pair
    $\left( D^{r} \times I, S^{r-1} \times I
    \cup D^{r} \times \left\{ 0 \right\} \right) $ is homeomorphic
    to $\left( D^{r} \times I, D^{r}\times \left\{ 0 \right\} 
    \right) $.
    See Figure \ref{fig:DUWUWUJK122-png}

    \begin{figure}[htpb]
        \centering
        \includegraphics[width=0.8\textwidth]{Figures/DUWUWUJK122.png}
        \caption{A homeomorphism of pairs.}
        \label{fig:DUWUWUJK122-png}
    \end{figure}
    These maps for each cell fit together to
    give a map on the $r$-skeleton because of the
    weak topology on $X \times I$. The union of these
    maps for all $r$ gives a map on $X \times I$, again because
    of the weak topology on $X \times I$.
\end{proof}

\begin{theorem}[]\label{Thm:SJJDHW29WW}
    Assume that $A \subset X$ is closed and that there
    exists a neighborhood $U$ of $A$ and a map
    $\varphi  \colon X \to I$ such that
    \begin{enumerate}
        \item $A = \varphi ^{-1} (0)$.
        \item $\varphi \left( X-U \right) = \left\{ 1 \right\} $.
        \item $U$ deforms to $A$ through $X$ with $A$ fixed.
            That is, there is a map $H \colon U \times I \to X$ 
            such that $H(a,t) = a$ for all $a \in A, H
            (u,0) = 0$, and $H(u,1) \in A$ for all $u \in U$.
    \end{enumerate}
    Then the inclusion
    $A \hookrightarrow X$ is a cofibration. The converse also
    holds.
\end{theorem}

\begin{proof}
    We may assume that $\varphi  = 1$ on a neighborhood
    of $X - U$ by replacing $\varphi $ with
    $\min \left( 2 \varphi , 1 \right) $.
    It suffices to show that there exists a retract
    $\Phi \colon U \times I \to X \times \left\{ 0 \right\} 
    \cup A \times I$ since then the
    map
    \[
    r\left( x,t \right) 
    =
    \begin{cases}
        \Phi\left( x, t \left( 1-\varphi (x) \right)  \right),&
        x \in U\\
        (x,0),& x \not\in U
    \end{cases}
    \] 
    gives a retraction $X \times I \to A \times I \cup 
    X \times \left\{ 0 \right\} $.\\
    We define $\Phi$ by
    \[
    \Phi(u,t) = 
    \begin{cases}
        H\left( u, \frac{t}{\varphi (u)} \right) \times 
        \left\{ 0 \right\},& \varphi (u) > t\\
        H\left( u,1 \right) \times 
        \left\{ t- \varphi (u) \right\} ,& \varphi (u)\le t.
    \end{cases}
    \] 
    The only thing that needs checking here is
    that $\Phi$ is continuous at
    points $\left( u,0 \right) $ such that
    $\varphi (u) = 0$, i.e., points $(a,0)$ for 
    $a \in A$ - indeed here the expression for
    $\varphi (u) > t$ is not defined.

    Recall that a map $f \colon X \to Y$ is continuous
    if for every point $x \in X$ and any neighborhood
    $U$ of $f(x)$, there exists a neighborhood
    $V$ of $x$ such that $f(V) \subset U$.\\
    So let $W$ be a neighborhood of $a = 
    H(a,t)$. Then there exists a neighborhood
    $V \subset W$ containing $a$ such that
    $H\left( V \times I \right) \subset W$, by
    assumption of $H$ being continuous.
    So for $t < \varepsilon$ for some $\varepsilon$ and
    $u \in V$, we have
    $\Phi \left( u,t \right) 
    \in W \times \left[ 0,\varepsilon \right] $.
    Hence $\Phi$ is continuous.\\
    \linebreak
    To prove the converse, suppose that the
    inclusion $A \hookrightarrow X$ is a cofibration. Equivalently,
    $A \times I \cup  X \times \left\{ 0 \right\} $ is
    a retract of $X \times I$. Let $r
    \colon X \times I \to  A \times I \cup X \times
    \left\{ 0 \right\} $ be this retraction.
    Let $s(x) = r\left( x,1 \right) $ and set
    $U = s^{-1}\left( A \times (0,1] \right) $.
    Let $p_X, p_I$ be the projections of
    $X \times I$ to its factors. Then
    put $H = p_X \circ r|_{U \times I} \colon U \times I \to X$.
    Now, $H(a,t) = p_X \circ r |_{U \times I}(a,t)
    = p_X \left( a,t \right) = a$ for all
    $a \in A$ and $t \in I$;
    $H(u,0) = p_X \circ r|_{U \times I}(u,0) =
    p_X \left( u,0 \right) = u$, and
    $H\left( u,1 \right) =
    p_X \circ r|_{U \times I}(u,1) = 
    u$ forces $(u,1) \in A \times I$, hence
    $u \in A$. Thus, $H$ satisfies condition (3).\\
    For (1) and (2), let
    $\varphi (x) = 
    \max_{t \in I} \left| t - p_I r(x,t) \right| $ which
    is possible since $I$ is compact. Then
    $x \in \varphi^{-1}(0)$ implies that
    $\max_{t \in I} \left| t - p_I r(x,t) \right| = 0$, so
    for all $t \in I$, we have
    $\left| t - p_I r(x,t) \right| = 0$, so
    $r(x,t) \in A \times \left\{ t \right\} $ for all
    $t \in (0,1]$. Then
    $r\left( x,0 \right) = \lim_{n \to \infty}
    r\left( x, \frac{1}{n} \right) 
    \in A \times I$ since $A \times I$ is closed. But
    $(x,0) = r(x,0)$, so $x \in A$. Conversely, for any
    $x \in A$, clearly, $\varphi (x) = 0$ since
    $r(x,t) = (x,t)$ for all $t \in I$. This shows that
    $\varphi $ satisfies (1). 
    For (2), we have that for
    $x \in X -U$, with $U = 
    s^{-1}\left( A \times (0,1] \right) $, we
    have $r(x,1) = s(x) \not\in A \times (0,1]$, so
    $r(x,1) \in X \times \left\{ 0 \right\} $. Hence
    $\varphi (x) = 
    \max_{t \in I} \left| t - p_I r(x,t) \right| = 1$, giving
    (2).\\
    It remains to show that $\varphi $ is continuous.
    Let $f(x,t) = \left| t - p_I r(x,t) \right| $ and
    $f_t = (x,t)$ all of which are continuous.
    Then
    \[
        \varphi^{-1} \left( (- \infty, b] \right) 
        = \left\{ x  \mid f(x,t) \le b \text{ for all }t 
        \right\} = 
        \bigcap_{i \in  I} f_t^{-1}\left( (- \infty, b] \right) .
    \]
    is an intersection of closed sets and so is closed.
    Similarly,
    \[
    \varphi^{-1} \left( [a, \infty) \right) 
    = \left\{ x  \mid  f(x,t) \ge a \text{ for some }t \right\} 
    = p_X \left( f^{-1} \left( [a, \infty) \right)  \right) 
    \] 
    which si also closed since $p_X$ is closed as
    a projection and
    $I$ is compact.
    Since the complements of the intervals of the
    form $[a, \infty)$ and 
    $(-\infty, b]$ give a subbase for the topology of
    $\mathbb{R}$, this shows that
    $\varphi $ is continuous.





\end{proof}


Next, we recall that for a map
$f \colon X \to Y$, the mapping cylinder
$M_f$ is defined as
\[
M_f = \left( \left( X \times I \right) \sqcup Y \right) 
/ \left( \left( x,0 \right) \sim f(x) \right) .
\] 
Consider the inclusion
$\iota \colon X \hookrightarrow 
M_f$ where we include $X$ as
$X \times \left\{ 1 \right\} $.
Consider the map
$ \varphi \colon
M f \to I$ given by
$ \varphi (x,t)  =  1 - 2t$ for
$t \ge \frac{1}{2}$ and
$\varphi (x,t) = 1$ on the rest of $M_f$. Choosing
$U = X \times (\frac{1}{3}, 1]$, $U$ clearly
deformations retracts to $X \times \left\{ 1 \right\} $ and
satisfies the conditions of 
Theorem \ref{Thm:SJJDHW29WW}, hence
the inclusion $X \hookrightarrow M_f$ is a cofibration.
Also, the retraction
$r \colon M_f \to Y$ is a homotopy equivalence
with the homotopy inverse being the inclusion
$Y \hookrightarrow M_f$. The diagram
\begin{equation*}
\begin{tikzcd}
    X \ar[rr, "\iota"] \ar[ddr, "f"'] & & M_f \ar[ddl, "r", "\simeq"']\\ 
                                     &&\\
                                     & Y &
\end{tikzcd}
\end{equation*}
commutes.
Thus any map $f$ is a cofibration up to
a homotopy equivalence of spaces.

Recall also that the mapping cone of a
map $f \colon X \to Y$ is defined as
\[
C_f := M_f / X \times \left\{ 1 \right\} 
\cong M_f \cup CX.
\] 
In the case of an inclusion
$\iota \colon A \hookrightarrow X$, we have
$C_{\iota} = X \cup  CA$.\\
There is a map
$C_{\iota} \stackrel{h}{\to} X / A$, defined
as the composite of the quotient map
$X \cup  CA \to X \cup  CA /CA$ composed with the
inverse of the homeomorphism $X / A \to 
X \cup  CA /CA$.

\begin{question}
    Is $h$ a homotopy equivalence?
\end{question}

\begin{theorem}[]\label{Thm:2030akKAK}
    If $A \subset X$ is closed and the inclusion
    $\iota \colon A \to X$ is a cofibration, then
    $h \colon C_{\iota} \to X /A$ is a homotopy equivalence.
    In fact, it is a homotopy equivalence of pairs
    \[
        \left( X /A, * \right) \simeq
        \left( C_{\iota}, CA \right) \simeq
        \left( C_{\iota}, v \right) ,
    \] 
    where $v$ is the vertex of the cone.
\end{theorem}

\begin{proof}
    The mapping cone $C_{\iota} = X
    \cup CA$ consists of three different types of
    points: the vertex $v = \left\{ A \times \left\{ 1 \right\} 
    \right\} $, the rest of the cone
    $\left\{ \left( a,t \right)  \mid 0\le t < 1 \right\} $ 
    where $\left( a,0 \right) = a \in A \subset X$, and
    points in $X$ itself, which we identify with $X \times 
    \left\{ 0 \right\} $.\\
    Define $f \colon A \times I \cup  X \times \left\{ 0 \right\} 
    \to C_{\iota}$ as the collapsing map and extend
    $f$ to $\overline{f} \colon X \times I \to 
    C_{\iota}$ using that $f$ is a cofibration. 
    Then $\overline{f}\left( a,1 \right) =
    v, \overline{f}(a,t) = (a,t)$ and
    $\overline{f}(x,0) = x$.\\
    Let $\overline{f}_t = 
    \overline{f}|_{X \times \left\{ t \right\} }$.
    Since $\overline{f}_1 (A) = \left\{ v \right\} $, 
    we can factorize $\overline{f}_1 \colon
    X \to C_{\iota}$ as
    $g \circ j$ where
    $j \colon X \to X /A$ is the quotient map
    and $g \colon X / A \to C_{\iota}$ is
    the induced map
    \begin{equation*}
    \begin{tikzcd}
        X \ar[d, "j"] \ar[dr, "\overline{f}_1"] & \\
        X / A \ar[r, "g", dashed] & C_{\iota}.
    \end{tikzcd}
    \end{equation*}
    where $g$ is induced and continuous by
    definition of the quotient topology.\\
    We claim that $g$ is a homotopy equivalence with
    homotopy inverse $h$.
    First, we prove that $hg \simeq \id_{X / A}$.

    Note that taking the composite
    $h \overline{f}_t \colon X \to X / A$ gives a homotopy
    between $h \overline{f}_0$ and $h \overline{f}_1$.
    For all $t$, this homotopy takes
    $A$ to the point $\left\{ A \right\} $. Thus, it
    factors to give a homotopy
    \[
    hgj = h \overline{f}_1 
    \simeq h \overline{f}_0 = j
    \] 
    Let $H \colon X \times I \to X / A$ be the homotopy
    between $hgj$ and $j$, so
    $H(x, 0) = hgj(x)$ and
    $H(x,1) = j(x)$. Then
    the map
    $\overline{H} \colon X / A \times I \to X / A$ defined
    by
    $\overline{H}(\left[ x \right] ,t) = 
    H\left( x,t \right) $ defines a homotopy
    between $hg$ and $\id_{X / A}$, so
    $hg \simeq \id_{X / A}$.

    Next, we will show that $gh \simeq \id_{C_{\iota}}$.
    Consider $W = \left( X \times I \right) / 
    \left( A \times \left\{ 1 \right\}  \right) $ 
    and the maps illustrated in Figure \ref{fig:DWIJJXNXJNXJUI-png}.

    \begin{figure}[htpb]
        \centering
        \includegraphics[width=0.8\textwidth]{Figures/DWIJJXNXJNXJUI.png}
        \caption{}
        \label{fig:DWIJJXNXJNXJUI-png}
    \end{figure}
    The map $\overline{f}'$ is induced by
    $\overline{f}$. The map
    $k$ is the "top face" map.
    From this, we see that
    \begin{align*}
        \overline{f}' \circ l 
        &= \id \\
        \pi \circ k 
        &= \id \\
        k \circ \pi 
        &\simeq \id \\
        \overline{f}' \circ k
        &= g \\
        \pi \circ l 
        &= l.
    \end{align*}
   Hence $g h = \overline{f}' k \pi l 
   \simeq \overline{f}' l = \id$. 
\end{proof}


\begin{example}[A non example]
    An example of when the result
    of Theorem 1.6 does not hold is with
    $A = \left\{ 0 \right\} \cup 
    \left\{ \frac{1}{n} \mid n = 1, 2, \ldots \right\} $ 
    and $X = \left[ 0,1 \right] $.
    In this case, $C_{\iota}$ is not homotopy equivalent
    to $X / A$ which is a one-point union of a countably infinite
    sequence of circles with radii going to zero.

    $C_i$ has homeomorphs of circles joined along edges. However,
    the circles do not tend to a point ,so any prospective homotopy
    equivalence $X / A \to C_{\iota}$ would be discontinuous at
    the image of $\left\{ 0 \right\} $ in $X / A$.
\end{example}

\begin{corollary}\label{Cor:Cofibration-Homology}
    If $A \subset X$ is closed and the inclusion
    $A \hookrightarrow X$ is a cofibration, then the map
    $j \colon \left( X, A \right) \to 
    \left( X / A, * \right) $ induces isomorphisms
    \[
    H_* (X,A) \stackrel{\cong}{\to} 
    H_* \left( X / A, * \right) 
    \cong \tilde{H}_* \left( X / A \right) 
    \] 
    and
    \[
    \tilde{H}^{*}(X /A) \cong
    H^{*} (X / A, *) 
    \stackrel{\cong}{\to} H^{*} (X , A).
    \] 
\end{corollary}

\begin{proof}
    We have
    $H_* \left( X/A , *\right) \cong
    H_* \left( C_{\iota}, CA \right) $ by Theorem
    \ref{Thm:2030akKAK}. And since
    $C_{\iota} = X \cup A \times \left[ 0,\frac{1}{2} \right] $ 
    and $CA = A \times \left[ 0,\frac{1}{2} \right] $, where
    we collapse $A \times \left\{ \frac{1}{2} \right\} $ in
    both, and attach  $A \times \left[ 0,\frac{1}{2} \right] $ 
    along $A \times \left\{ 0 \right\} $ in
    $X \cup A \times \left[ 0,\frac{1}{2} \right] $, we obtain
    \[
    H_*(C_{\iota}, CA) \cong
    H_* \left( X \cup A \times \left[ 0,\frac{1}{2} \right] ,
    A \times \left[ 0, \frac{1}{2} \right] \right) 
    \cong H_* (X, A)
    \] 
    since 
    $\left( X \cup A \times \left[ 0,\frac{1}{2} \right] ,
    A \times \left[ 0,\frac{1}{2} \right] \right) 
    \simeq \left( X,A \right) $ by deformation retracting
    $A \times \left[ 0,\frac{1}{2} \right] $ down to
    $A \times \left\{ 0 \right\} \subset X$.
\end{proof}


\subsubsection{Interlude on pointed-spaces and
operations on spaces}

We recall some important constructions:

\begin{definition}[Unreduced Suspension]
    For a space $X$, the \textit{unreduced suspension} 
    $\Sigma X$ is the
    quotient obtained from $X \times I$ by
    collapsing $X \times \left\{ 0 \right\} $ to one
    point and $X \times \left\{ 1 \right\} $ to another
    point.
\end{definition}

\begin{note}
    We have $\Sigma S^{n} = S^{n+1}$.
\end{note}



\begin{definition}[Suspension of a map]
    Given a map
    $f \colon X \to Y$, we can suspend $f$ to
    $\Sigma f \colon \Sigma X \to \Sigma Y$
    by letting $\Sigma f$ be the induced
    map on the quotients:
    \begin{equation*}
    \begin{tikzcd}
        X \times I \ar[r, "f \times \id"] \ar[d]  & Y \times I \ar[d] \\
        \Sigma X \ar[r, "\Sigma f"] & \Sigma Y
    \end{tikzcd}
    \end{equation*}
    
\end{definition}

\begin{definition}[Reduced Suspension]
    For a space $X$, the \textit{reduced suspension}
    $SX$ is the quotient
    \[
    SX = X \times I / \left( X \times \partial I
    \cup \left\{ * \right\} \times I \right) .
    \] 
\end{definition}

\begin{definition}[Reduced Suspension of a map]
    The reduced suspension of a map $f \colon X \to Y$ is
    the induced map on the reduced suspensions of
    $X$ and $Y$ :
    \begin{equation*}
    \begin{tikzcd}
        X \times I \ar[r, "f \times \id"]
        \ar[d] & Y \times I \ar[d] \\
        SX \ar[r, "Sf"] & SY
    \end{tikzcd}
    \end{equation*}
    
\end{definition}

\begin{exercise}[]
    For any homology theory, show that
    there is a natural isomorphism
    $\tilde{H}_I (X) \stackrel{\cong}{\to} 
    \tilde{H}_{i+1} \left( \Sigma X \right) $. Here,
    natural means that for a map $f \colon X \to Y$,
    and its suspension $\Sigma f \colon \Sigma X \to 
    \Sigma Y$, the following diagram commutes:
    \begin{equation*}
    \begin{tikzcd}
        \tilde{H}_i (X) \ar[r, "\cong"] \ar[d, "f_*"]& 
        \tilde{H}_{i+1} \left( \Sigma X \right)
        \ar[d, "(\Sigma f)_*"] \\
        \tilde{H}_i (Y) \ar[r, "\cong"] & \tilde{H}_{i+1}
        \left( \Sigma Y \right) 
    \end{tikzcd}
    \end{equation*}
    
\end{exercise}


\begin{definition}[Wedge Sum/one-point union]
    Given two pointed spaces $(X, x_0), \left( Y, y_0 \right) $,
    we define
    the \textit{wedge sum}  $X \vee Y$ to be
    \[
    X \vee Y = X \sqcup Y / \left( x_0 \sim y_0 \right) ,
    \] 
    i.e., the quotient of the disjoint union identifying
    $x_0$ and $y_0$ to a single point.
\end{definition}

\begin{definition}[Smash Product]
    Inside the product $X \times Y$
    of two pointed space $(X,x_0), (Y,y_0)$,
    we have natural copies of $X$ and $Y$ by
    $X \times \left\{ y_0 \right\} $ and
    $\left\{ x_0 \right\} \times Y$, respectively.
    These two copies intersect only at the point
    $\left( x_0,y_0 \right) $, so their union can
    be identified with
    the wedge sum $X \vee Y$. I.e., $X \vee Y = 
    X \times \left\{ y_0 \right\} \cup 
    \left\{ x_0 \right\} \times Y$. We define
    the \textit{smash product} $X \wedge Y$ to be
    the quotient $X \times Y / X \vee Y$.
\end{definition}


If $f \colon X \to Y$ is a pointed map, then
the reduced mapping cylinder of $f$ is defined
as the quotient space $M_f$ of
$\left( X \times I \right) \cup  Y$ modulo the relations
identifying
$\left( x,0 \right) \sim f(x)$ and the set
$\left\{ * \right\} \times I$ to the base point of $M_f$.\\
The reduced mapping cone is the quotient of the reduced
mapping cylinder $M_f$ obtained by identifying the image of
$X \times \left\{ 1 \right\} $ to a point, the
base point.\\

The circle $S^{1}$ is defined as
$I / \partial I$ with base point $\left\{ \partial I \right\} $.\\
The reduced suspension of a pointed space $X$ is
$SX = X \wedge S^{1}$. It can also be considered
as the quotient space $X \times I / \left( X \times 
\partial I \cup  \left\{ * \right\} \times I\right) $


\begin{definition}[Well-pointed space]
    A base point $x_0 \in X$ is said to be
    \textit{nondegenerate} if the inclusion
    $\left\{ x_0 \right\} \hookrightarrow X$ is a cofibration.
    A pointed Hausdorff space $X$ with nondegenerate
    base point is said to be \textit{well-pointed}.
\end{definition}

It is clear that any manifold or CW-complex satisfies
Theorem \ref{Thm:SJJDHW29WW} with $A$ being any point of
the space. Hence any manifold or CW-complex is
well-pointed.\\

\begin{example}[Pointed space that is not well-pointed]
    Taking the pointed space  $X = 
    \left\{ 0 \right\} \cup  \left\{ \frac{1}{n} \mid 
    n \in \mathbb{N} \right\} $ with base point $0$, this
    space is not well-pointed. This can for example be
    seen because it fails to satisfy Theorem
    \ref{Thm:Retract-cofibration} - any retraction
    would break continuity at
    $\left( 0,1 \right) $.
\end{example}

\begin{example}[]
    If $A \hookrightarrow X$ is a cofibration, then
    $X / A$ with base point $\left\{ A \right\} $ is well-pointed,
    as follows from Theorem \ref{Thm:SJJDHW29WW}.
\end{example}

\begin{theorem}[]
    If $X$ is well-pointed, then so are the reduced
    cone $CX$ and the reduced suspension $SX$. Moreover,
    the collapsing map  $\Sigma X \to SX$, of the unreduced
    suspension to the reduced suspension, is a homotopy
    equivalence.
\end{theorem}

\begin{proof}
    Denote the base point of $X$ by $*$.
    Consider the homeomorphism
    \[
    h \colon \left( I \times I,
    I \times \left\{ 0 \right\} \cup 
\partial I \times I \right) 
\stackrel{\cong}{\to} \left( I \times I, I \times 
\left\{ 0 \right\} \right) 
    \] 
    which clearly exists. For example, take Figure
    \ref{fig:IWIDK01-jpeg}
    \begin{figure}[htpb]
        \centering
        \includegraphics[width=0.5\textwidth]{Figures/IWIDK01.jpeg}
        \caption{}
        \label{fig:IWIDK01-jpeg}
    \end{figure}

    Then the induced homeomorphism
    \[
    \id_X \times h\colon X \times I \times I 
    \stackrel{\cong}{\to} X \times I \times I
    \] 
    carries 
    $X \times I \times \left\{ 0 \right\} \cup 
    X \times \partial I\times I$ to
    $X \times I \times \left\{ 0 \right\} $.
    Hence it takes
    $A = X \times I \times \left\{ 0 \right\} \cup 
    X \times \partial I \times I \cup 
    \left\{ * \right\} \times I \times I$ to
    $X \times I \times \left\{ 0 \right\} 
    \cup \left\{ * \right\} \times I \times I$. 
    Therefore, the pair
    $\left( X \times I \times I, A \right) $ is homeomorphic
    to the pair
    $I \times \left( X \times I, 
    X \times \left\{ 0 \right\} \cup 
\left\{ * \right\} \times I \right) $.
Now, $X$ is well-pointed, so
$X \times \left\{ 0 \right\} \cup 
\left\{ * \right\} \times I$ is a retract of
$X \times I$ by
Theorem \ref{Thm:Retract-cofibration} and the
definition of well-pointed.
It follows that
$A$ is a retract of
$X \times I \times I$.
By another application of
\ref{Thm:Retract-cofibration}, then
the inclusion
$X \times \partial I \cup \left\{ * \right\} \times I
\hookrightarrow X \times I$ is a cofibration. 
Hence the quotient by this,
$SX = X \times I / \left( X \times \partial I
\cup \left\{ * \right\} \times I \right) $ is well-pointed,
using the quotient of the above inclusion.\\
\linebreak
Next consider the homeomorphism
$\left( I \times I, I \times \left\{ 0 \right\} \cup 
\left\{ 1 \right\} \times I \right) 
\stackrel{\cong}{\to} \left( 
I \times I, I \times \left\{ 0 \right\} \right) $ which
can be seen similarly. The induced
homeomorphism
\[
1 \times h \colon X \times I \times I
\stackrel{\cong}{\to} X \times I \times I
\] 
takes
$A:= X \times \left\{ 1 \right\} \times I \cup 
\left\{ * \right\} \times I \times I
\cup X \times I \times \left\{ 0 \right\} $ to
$X \times I \times \left\{ 0 \right\} \cup 
\left\{ * \right\} \times I \times I$.
Thus the pair
$\left( X \times I \times I,
A\right) $ is homeomorphic to
 $I \times \left( X \times I,
 X \times \left\{ 0 \right\} \cup 
\left\{ * \right\} \times I\right) $.
Just as above, we have that
$X \times \left\{ 0 \right\} \cup 
\left\{ * \right\} \times I$ is a retract
of $X \times I$, so
it follows that
$A$ is a retract of $X \times I \times I$. Thus
the inclusion
$X \times \left\{ 1 \right\} \cup 
\left\{ * \right\} \times I \hookrightarrow
X \times I$ is a cofibration, which shows
that $CX = X \times I / \left( X \times \left\{ 1 \right\} 
\cup \left\{ * \right\} \times I\right) $ is
well-pointed.\\
\linebreak
The fact that
$X \times \partial I \cup \left\{ * \right\} \times I
\hookrightarrow X \times I$ is a cofibration
gives that there exists a neighborhood $U$ of
$X \times \partial I \cup 
\left\{ * \right\} \times I$ and a map
$\varphi \colon X \times I \to I$ 
that satisfy Theorem \ref{Thm:SJJDHW29WW}.
We obtain an induced map
$\overline{\varphi }\colon
\Sigma X \to I$ 
which satisfies the same conditions, so
$I \times \times \left\{ * \right\} \times I
\hookrightarrow X \times I / 
\left\{ X \times \left\{ 0 \right\} ,
X \times \left\{ 1 \right\} \right\} = \Sigma X$ is a
cofibration. Now 
Theorem \ref{Thm:2030akKAK} implies
that $\Sigma X \cup CI = C_{\iota} \to \Sigma X / I$ is a homotopy
equivalence.
Hence we obtain
that 
$\Sigma X \simeq
\Sigma X \cup CI \simeq
\Sigma X / I = SX$, via the collapsing map.

\end{proof}


\begin{problem}[]
    Find $H_* \left( \mathbb{P}^2, \mathbb{P}^{1} \right) $ 
    using methods or results from this section.
\end{problem}

\begin{solution}
    Consider $\mathbb{P}^2$ as 
    $S^2$ quotiented by the relation
    $x \simeq -x$. Then
    we can think of  $\mathbb{P}^{1}$ as
    $S^{1} \subset S^{2}$ under this relation.
    We want to show that the inclusion
    $\mathbb{P}^{1} \hookrightarrow \mathbb{P}^2$ is a
    cofibration. 
    Using Theorem \ref{Thm:SJJDHW29WW}, it suffices to
    find a neighborhood $U$ of $\mathbb{P}^{1} \subset 
    \mathbb{P}^2$ and a map $\overline{\varphi } \colon
    \mathbb{P}^2 \to I$ such that
    the conditions of the theorem are satisfied.
    We construct a preliminary map on
    $S^2$ towards this end. 
    Define $\varphi \colon S^2 \to I$ to be
    $\varphi (x) = 
    \min \left\{ 1, 2 \left| x_3 \right|  \right\} $, where
    $x_3$ is the last coordinate of $x$. Since
    $\varphi (x) = \varphi (-x)$, $\varphi $ induces
    a map $\overline{\varphi }\colon
    \mathbb{P}^2 \to I$ such that the diagram
    \begin{equation*}
    \begin{tikzcd}
        S^2 \ar[d] \ar[dr, "\varphi "] & \\
        \mathbb{P}^2 \ar[r, "\overline{\varphi }"] & I
    \end{tikzcd}
    \end{equation*}
    commutes.
    Letting $U$ be the image under the quotient
    map of 
    $\left\{ x \in S^2  \mid 
    \left| x_3 \right| < \frac{1}{2} \right\} $, this
    becomes an open set in $\mathbb{P}^2$ since 
    the above set is saturated with respect to the quotient
    map. It is also clear that
    $U$ and $\overline{\varphi }$ satisfy the conditions of
    the theorem, hence the inclusion
    $\mathbb{P}^{1} \hookrightarrow \mathbb{P}^2$ is a cofibration.
    By Corollary \ref{Cor:Cofibration-Homology}, we
    obtain that
    $H_* \left( \mathbb{P}^2, \mathbb{P}^{1} \right) 
    \cong \tilde{H}_* \left( \mathbb{P}^2 / \mathbb{P}^{1} \right) $.
    But $\mathbb{P}^2 / \mathbb{P}^{1} \cong
    S^{2}$, so
    $H_* \left( \mathbb{P}^2, \mathbb{P}^{1} \right) 
    \cong \tilde{H}_* \left( S^2 \right)$.
    Now simply recall that
    \[
    \tilde{H}_p \left( S^2 \right) 
    \cong
    \begin{cases}
        \mathbb{Z},& p = 2\\
        0,& p \neq 2.
    \end{cases}
    \] 
    \qed
\end{solution}

\begin{problem}[]
    Find $H_*\left( T^2,
    \left\{ * \right\} \times S^{1} \cup 
S^{1} \times \left\{ * \right\} \right) $ using
methods from this section.
\end{problem}

\begin{solution}
    If we can show that the inclusion
    $A:= \left\{ * \right\} \times S^{1} \cup 
    S^{1} \times \left\{ * \right\} \hookrightarrow
    T^2$ is a cofibration, then
    we will again obtain that
    $H_* (T^2, A) \cong
    \tilde{H}_* \left( T^2 / A \right) \cong
    \tilde{H}_* \left( S^2 \right) $.
    But we have a CW-structure on the
    torus given by the square with identified sides.
    With this identificaiton, $A$ simple becomes
    the $1$-skeleton, hence it is a subcomplex, so
    by Corollary \ref{Cor:Subcomplex-Cofibration}, 
    the inclusion $A \hookrightarrow T^2$ is a cofibration.
    This finishes the solution. \qed
\end{solution}

\begin{problem}[]
    For a space $X$, consider the pair
    $\left( CX, X \right) $. What do the results of this
    section tell you about the homology of these, and related,
    spaces?
\end{problem}


\begin{solution}
    We can define a map
    $\varphi \colon CX \to I$ by
    $\varphi (x,t) = t$. Choosing
    $A = X = X \times \left\{ 0 \right\} \subset 
    CX$ and
    $U = CX - \left\{ v \right\} $ where
    $v$ is the vertex, this satisfies the
    conditions in Theorem \ref{Thm:SJJDHW29WW} 
    ($H$ can be defined by
    $H((x,t_0),t) = \left( x,t_0 \right) (1-t)
    + (x,0) t$).
    Hence the inclusion
    $X \hookrightarrow CX$ is a cofibration, so we
    know that
    $H_* \left( CX, X \right) 
    \cong \tilde{H}_* \left( CX / X \right) $.
    Similarly, one can
    show that the inclusion
    $X \hookrightarrow \Sigma X$ is a cofibration, so
    $H_* \left( \Sigma X , X \right) 
    \cong \tilde{H}_* \left( \Sigma X / X \right) 
    \cong \tilde{H}_* \left( 
    \Sigma X \vee \Sigma X \right) $ and
    $H_*\left( SX, X \right) 
    \cong \tilde{H}_* \left( SX \vee SX \right) $.
\end{solution}


\begin{problem}[]
    If $f \colon A \to X$ is a cofibration then show that
    $f$ is an embedding. If $X$ is also Hausdorff,
    then show that $f(A)$ is closed in $X$.
\end{problem}

\begin{proof}
    Since $f$ is a cofibration, the following diagram
    can be filled out, inducing a map
    $g \colon X \times I \to M_f$ :
    \begin{equation*}
    \begin{tikzcd}
        A \times \left\{ 0 \right\} \ar[dd, "f \times \id"] 
        \ar[rr, hookrightarrow] & & A \times I \ar[dd, "f \times 
        \id"] \ar[dl, "q"] \\
                                & M_f & \\
        X \times \left\{ 0 \right\} \ar[ur, "q"] 
        \ar[rr, hookrightarrow]& & X \times I 
        \ar[ul, "g"]
    \end{tikzcd}
    \end{equation*}
    By construction, we have that
    $q \colon A \times \left\{ 1 \right\} 
    \hookrightarrow M_f$ is an embedding, so
    letting $l \colon
    q \left( A \times \left\{ 1 \right\}  \right) 
    \to A \times \left\{ 1 \right\} $ be the inverse map,
    we have
    $l \circ g|_{f(A) \times \left\{ 1 \right\} } 
    \circ \left( f \times \id \right) 
    = \id_{A \times \left\{ 1 \right\} }$.
    Likewise,
    $\left( f \times \id \right) 
    \circ l \circ g|_{f(A) \times \left\{ 1 \right\} }$,
    since $g \left( f(a), t \right) 
    = q(a,t)$, we have that
    $l \circ g|_{f(A) \times \left\{ 1 \right\} }
    (f(a),1) = (a,t)$, hence
    $\left( f \times \id \right) 
    \circ l \circ g|_{f(A) \times \left\{ 1 \right\} }
    = \id_{f(A) \times \left\{ 1 \right\} }$.
    Therefore,
    $f \times \id$ is a homeomorphism
    $A \times \left\{ 1 \right\}  \stackrel{\cong}{\to} 
    f(A) \times \left\{ 1 \right\} $, so
    $f \colon A \stackrel{\cong}{\to} f(A)$ is a homeomorphism.\\
    \linebreak
    By Lemma \ref{Lemma:Cofibration-Inclusion-2} and
    Theorem \ref{Thm:Retract-cofibration}, we
    have that there exists a retraction
    $r \colon X \times I \to 
    X \times \left\{ 0 \right\} \cup 
    f(A) \times I$.

    \begin{lemma}[]
         If a space $X$ is Hausdorff and
          there exists a retraction
          $r \colon X \to A$, then $A$ is closed.
    \end{lemma}

    \begin{proof}
        Let
        $x \in X -A$ be a limit point of $A$. Let
        $U ,V$ be open disjoint neighborhoods of
        $x$ and $r(x)$. Then
        $r^{-1}(V)$ is open and contains $x$, so
        let $U' = U \cap r^{-1}(V)$.
        Now $U \cap A \cap r^{-1}(V) = \varnothing$ since
        otherwise $U \cap A = r\left( U \cap A \right) 
        \subset V$ contradicting $U \cap V = \varnothing$.
        But then $U'$ is an open neighborhood of $x$
        that is disjoint from $A$, contradicting $x$ being
        a limit point of $A$.
        Thus $\overline{A} = A$.
    \end{proof}

    Using this Lemma, we find that since
    $X \times I$ is Hausdorff and
    $r \colon X \times I \to X \times \left\{ 0 \right\} 
    \cup f(A) \times I$ is a retraction,
    $X \times \left\{ 0 \right\} \cup 
    f(A) \times I$ is closed in
    $X \times I$.
    Now,
    $f(A) \times \left\{ 1 \right\} 
    = X \times \left\{ 1 \right\} 
    \cap \left( X \times \left\{ 0 \right\} \cup 
    f(A) \times I\right) $, so
    since $X \times \left\{ 1 \right\} $ is closed
    in $X \times I$,
    $f(A) \times \left\{ 1 \right\} $ 
    is by definition closed in 
    $X \times \left\{ 0 \right\} \cup 
    f(A) \times I$ in the subspace topology. Hence
    it is also closed in $X \times I$.
    Now we use another lemma:
    \begin{lemma}[]
        If $Y$ is a compact space, then
        the projection $X \times Y \to X$ is a closed map.
    \end{lemma}
    \begin{proof}
        Let $W \subset X \times Y$ be closed and
        set $W' = X \times Y -W$.
        Note that $x_0 \in \pi_X (W)$ if and only
        if $\exists y_0 \in Y$ such that
        $(x_0,y_0) \in W$. Thus
        $x_0 \not\in \pi_X(W)$ if and only if
        $\left\{ x_0 \right\} \times Y \subset 
        W'$.

        By the tube lemma,
        $x_0 \not\in \pi_X (W)$ if and only if
        $W'$ contains some tube
        $N \times Y$ about $\left\{ x_0 \right\} \times Y$ 
        where $N$ is an open neighborhood of $x_0$ in $X$. 
        But then
        $N = \pi(N \times Y) \subset 
        X - \pi(W)$ is an open neighborhood of $x_0$ in
        $X - \pi(W)$. Hence
        $X - \pi(W)$ is open, so $\pi(W)$ is
        closed.
    \end{proof}
    Noting that
    $\left\{ 1 \right\} \subset I$ is compact,
    we can apply this Lemma to
    $f(A) \times \left\{ 1 \right\} $ to obtain that
    $f(A)$ is closed in $X$.
    This completes the proof.

\end{proof}

\newpage
\begin{problem}[]
    Let $\iota \colon A \hookrightarrow X$,
    the inclusion of $A$ in $X$, be a cofibration and
    $A$ be a contractible space. Show that the
    quotient map $X \to X / A$ is a homotopy equivalence.
\end{problem}

\begin{proof}
    Let
    $H \colon A \times I \to A$ be the contraction of
    $A$ where
    $H(a,0) = a$ and $H(a,1) = a_0 \in A$.
    Consider the diagram
    \begin{equation*}
    \begin{tikzcd}
        A \times \left\{ 0 \right\} \ar[dd, "\iota \times \id"]
        \ar[rr] 
        & &
        A \times I \ar[dd, "\iota \times \id"] \ar[dl, "H"] \\
                                                    & X&\\
        X \times \left\{ 0 \right\} \ar[rr] \ar[ru,"\id"]
                                                    & & X \times I
        \ar[ul, "\tilde{H}"] 
    \end{tikzcd}
    \end{equation*}
    Then since 
    $\tilde{H}(a,t) \in A$ for all $t$,
    the composition $q \tilde{H} \colon X \times I \to X / A$ sends
    $A$ to a point at all times, hence factors as
    $X \times I \stackrel{q \times \id}{\to} X / A \times I
    \to X / A$. Denote the
    latter map by $\overline{H} \colon 
    X / A \times I \to X / A$. Then
    $q \tilde{H} = \overline{H} \left( q \times \id \right) $.
    When $t = 1$, we have
    $\tilde{H}\left( A,1 \right) $ equal to a point, so
    $\tilde{H}\left( -,1 \right) $ induces a map
    $g \colon X / A \to X$ with $g q = \tilde{H}
    \left( -,1 \right) $.
    It follows that
    $qg = \overline{H}(-,1)$ since
    $qg\left( \overline{x} \right) 
    =qg q(x) = q \tilde{H}(x,1) 
    = \overline{H} \left( q(x),1 \right) =
    \overline{H}\left( \overline{x},1 \right) $.
    Now the maps $g$ and $q$ are inverse homotopy
    equivalences since
    $gq = \tilde{H}(-,1) \simeq
    \tilde{H}(-,0) = \id_X$ and
    $qg = \overline{H}(-,1) \simeq
    \overline{H}(-,0) = \id_{X / A}$.
\end{proof}

\subsubsection{Some Applications of the HEP}

\begin{proposition}[]\label{Prop:HEP-Homotopy-Equivalence}
    Suppose $\left( X,A \right) $ and
    $\left( Y,A \right) $ satisfy the HEP, and 
    $f \colon X \to Y$ is a homotopy equivalence
    with $f |_{A} = \id$. Then $f$ is a homotopy equivalence
    $\rel A$.
\end{proposition}

\begin{corollary}
    If $\left( X,A \right) $ satisfy the HEP and the
    inclusion $A \hookrightarrow X$ is a homotopy equivalence,
    then $A$ is a deformation retract of $X$.
\end{corollary}

\begin{corollary}
    A map $f \colon X \to Y$ is a homotopy equivalence
    if and only if $X$ is a deformation retract of the
    mapping cylinder $M_f$. Hence, two space $X$ and 
    $Y$ are homotopy equivalence if and only if
    there is a third space containing both $X$ and
    $Y$ as deformation retracts.
\end{corollary}



\section{The Compact-Open Topology}

Recall that $Y^{X}$ denotes the \textit{set} of 
\textit{continuous} functions $X \to Y$.

\begin{definition}[]
    The \textit{compact-open topology} on $Y^{X}$ is the
    topology generated by the sets
    $M \left( K, U \right) 
    = \left\{ f \in Y^{X} \mid 
    f(K) \subset U \right\} $ where
    $K \subset X$ is compact and
    $U \subset Y$ is open.
\end{definition}
Generated here means that these sets form
a \textit{subbasis} for the open sets.\\


\begin{lemma}[]\label{Lemma:Compact-Open-Subbasis}
    Let $\mathcal{K}$ be a collection of compact subsets
    of $X$ containing a neighborhood base
    at each point of $X$. Let
    $\mathcal{B}$ be a subbasis for the open sets
    of $Y$. Then
    the collection
     \[
    \left\{ M\left( K,U \right)  \mid 
    K \in \mathcal{K}, B \in \mathcal{B} \right\} 
    \] 
    forms a subbasis for the compact-open topology
    on $Y^{X}$.
\end{lemma}

\begin{proof}
    Recall first that a subbasis is a collection whose
    union is the whole space and such that the
    collection of finite intersections of elements of the
    subbasis form a basis.\\
    In particular, noting that
    $M \left( K, U \right) \cap M\left( K,V \right) =
    M\left( K, U \cap V \right) $, this implies that it
    suffices to consider the case when $\mathcal{B}$ is
    a basis.\\
    So to show that the collection in question
    is a subbasis, it suffices to show that
    given $f \in M(K,U)$, there exist
    $K_1, \ldots,K_n \in \mathcal{K}$ and
    $U_1, \ldots, U_n \in \mathcal{B}$ such that
    $f \in \bigcap_{i=1}^{n} M\left( K_i,U_i \right) 
    \subset M\left( K,U \right) $.

    For each $x \in K$, there is an open
    set $U_x \in \mathcal{B}$ with
    $f(x) \in U_x \subset U$ (since $\mathcal{B}$ was
    assumed to be a basis), and there
    exists a neighborhood $K_x \in \mathcal{K}$  of
    $x$ such that $f(K_x) \subset U_x$ (since
    $f$ is continuous and $\mathcal{K}$ was assumed
    to contain a neighborhood base at each point of $X$ ).
    Thus $f \in M\left( K_x, U_x \right) $.
    Now, covering $K$ with these sets
    $K \subset \bigcup_{x \in K} K_{x}$. By 
    compactness of $K$, there exists a finite subcover
    $K \subset K_{x_1} \cup \ldots \cup K_{x_n}$. Then
    $f \in \bigcap_{i=1}^{n} M\left( K_{x_i}, U_{x_i} \right) 
    \subset M\left( K,U \right) $.
\end{proof}

\begin{proposition}[]\label{Prop:SJUSIO12}
    For $X$ locally compact Hausdorff, the
    "evaluation map" $e\colon Y^{X} \times X
    \to Y$, defined by $e\left( f,x \right) =
    f(x)$, is continuous.
\end{proposition}

\begin{proof}
    Let $\left( f,x \right) \in 
    Y^{X} \times X$ and
    $U$ a neighborhood of $f(x) \in  Y$.
    Now we make use of the following lemma:
    \begin{lemma}[]\label{Lemma:SIJDLII}
        If $X$ is a locally compact Hausdorff space, then
        each neighborhood of a point $x \in X$ contains
        a compact neighborhood of $X$.
        In particular, $X$ is completely regular.
    \end{lemma}

    \begin{proof}
        Let $C$ be a compact neighborhood of $x$ and
        $U$ an arbitrary neighborhood of $x$.
        Since $X$ is Hausdorff, $C$ is closed, so
        $\left( X - U \right) \cap C$ is a closed
        subspace of $C$, hence compact. Now,
        for each point $z \in (X-U) \cap C$, choose, by Hausdorffness,
        open neighborhoods $U_z',V_z'$ of $z$ and $x$, respectively,
        and consider $W' := \bigcup_{z \in (X-U)\cap C} 
        U_z'$. Since this is open,
        $C - W'$ is closed hence compact. Furthermore,
        it is contained in $U$ and contains $x$.\\
        \linebreak
        \textit{Alternative proof due to Bredon:}
        Let $C$ be a compact neighborhood of $x$ and
        $U$ an arbitrary neighborhood
        of $x$. Let $V \subset C \cap U$ be open with
        $x \in V$. Then $\overline{V} \subset C$ is compact
        Hausdorff, hence regular, so there exists
        a neighborhood $N \subset V$ of $x$ in
        $C$ which is closed in $\overline{V}$ 
        and hence closed in $X$. Since
        $N$ is closed in the compact space $C$, it
        is compact. Since $N$ is a neighborhood of
        $x$ in $\overline{V}$ and since
        $N = N \cap V$, $N$ is a neighborhood of $x$ in
        the open set $V$ and hence in $X$.
    \end{proof}

    By Lemma \ref{Lemma:SIJDLII}, 
    there exists a compact neighborhood $K$ of
    $x$ such that
    $f(K) \subset U$. Hence
    $f \in M\left( K,U \right) $, and
    $e \left( M \left( K, U \right) \times 
    K \right) \subset U$.
    This finishes the proof.
\end{proof}

\begin{theorem}[]\label{Thm:Compact-Open-Top}
    Let $X$ be locally compact Hausdorff and $Y$ and
    $T$ arbitrary Hausdorff spaces. Given a function
    $f \colon X \times T \to Y$, define, for each
    $t \in T$, the function
    $f_t \colon X \to Y$ by
    $f_t (x) = f(x,t)$. Then $f$ is continuous
    if and only if both of the following conditions
    hold:
    \begin{enumerate}
        \item Each $f_t$ is continuous
        \item The function $T \to Y^{X}$ taking
            $t \mapsto f_t$ is continuous.
    \end{enumerate}
\end{theorem}

\begin{proof}
    The "if" implication follows from the fact
    that $f$ is the composition
    \[
    X \times T \stackrel{\left( x,t \right) \mapsto 
    \left( f_t, x \right) }{\longrightarrow} Y^{X} \times X
    \stackrel{e}{\to} Y.
    \] 
    Now the evaluation map is continuous
    by Proposition \ref{Prop:SJUSIO12} since
    $X$ is assumed to be locally compact Hausdorff and
    since $f_t$ is assumed to
    be continuous for all $t$ by condition (1); and
    $\left( x,t \right) \mapsto \left( f_t, x \right) $ 
    is continuous since 
    $t \mapsto f_t$ is assumed to be continuous
    by condition (2).\\
    Conversely, for the "only if" implication,
    (1) follows from the fact that $f_t$ is the
    composition
    \[
    X \stackrel{x \mapsto (x,t)}{\to} X \times T
    \stackrel{f}{\to} Y.
    \] 
    To prove (2), let
    $t \in T$ be given and
    $f_t \in M\left( K, U \right) $. It suffices
    to find a neighborhood $W$ of $t$ in $T$ such that
    $t' \in W$ implies that
    $f_{t'} \in M\left( K,U \right) $ (i.e., it suffices to
    prove conditions for continuity for a subbasis only).
    For $x \in K$, there are open neighborhoods $V_x \subset 
    X$ of $x$ and $W_x \subset T$ of $t$ such that
    $f\left( V_x \times W_x \right) \subset U$.
    By compactness, $K \subset 
    V_{x_1} \cup  \ldots \cup  V_{x_n} =: V$ for
    some $V_{x_i}$. Let
    $W = \bigcap_{i=1}^{n} W_{x_i}$. Then
    $f\left( K \times W \right) \subset 
    f\left( V \times W \right)  \subset U$.
    So $t' \in W$ implies that
    $f_{t'} \in M\left( K,U \right) $ as claimed.
\end{proof}

\begin{note}
    This theorem implies that
    a homotopy $X \times I \to Y$ with $X$ locally
    compact is the same thing as a path
    $I \to Y^{X}$ when we give
    $Y^{X}$ the compact-open topology.
\end{note}

\begin{note}
    This is precisely the reason why, when we define
    $\MCG (X)$, we define it as
    $\pi_0 \Homeo^{+}(X, \partial X)$ where we equip
    $\Homeo^{+}\left( X, \partial X \right) $ with the
    subspace topology inherited from
    $X^{X}$ in the compact-open topology.
    By the above theorem, a path 
    $I \to \Homeo^{+} \left( X, \partial X \right) $ 
    given as $t \mapsto \gamma_t$
    is continuous if and only if
    the associated function
    $\gamma \colon X \times I \to X$ given
    by $\gamma(x,t) = \gamma_t(x)$ is
    continuous. But since each
    $\gamma_t$ is a self-homeomorphism of $X$, this
    just tells us that $\gamma$ is an isotopy of
    $X$. So path components
    of $\Homeo^{+} \left( X , \partial X \right) $ 
    correspond to isotopy classes of orientation-preserving
    self-homeomorphisms of $X$ fixing the boundary point-wise.
\end{note}

\begin{theorem}[The Exponential Law]\label{Exponential-Law}
    Let $X$ and $T$ be locally compact Hausdorff spaces
    and let $Y$ be an arbitrary Hausdorff space. Then
    there is the homeomorphism
    \[
    Y^{X \times T} \stackrel{\cong}{\to} \left( Y^{X} \right)^{T}
    \] 
    taking $f \mapsto f^{*}$ where
    $f^{*}(t) (x) = f(x,t) = f_t(x)$.
\end{theorem}

\begin{proof}
    By Theorem \ref{Thm:Compact-Open-Top}, the assignment
    $f \mapsto f^{*}$ is a bijection.\\
    We must show it and its inverse to be continuous.
    Let $U \subset Y$ be open and
    $K \subset X, K'\subset T$ be compact. Then
    \begin{align*}
        f \in M \left( K \times K', U \right) 
        &\iff \left( t \in K', x \in K
        \implies f_t(x) = f(x,t) \in U \right) \\
        &\iff \left( t \in K' \implies 
        f_t \in M \left( K, U \right) \right) \\
        &\iff f^{*} \in M \left( K', M\left( K ,U \right)  \right).
    \end{align*}
    Now, the $K \times K'$ are compact
    subsets of $X \times T$, and the collection
    of all these over $X \times T$ contain
    a neighborhood basis at each point since
    $X$ and $T$ are both assumed to be locally compact.
    By Lemma \ref{Lemma:Compact-Open-Subbasis},
    the collection
    \[
        \left\{ M\left( K \times K', U \right) 
     \mid U \subset Y \text{ open}, 
 K \subset X, K' \subset T \text{ both compact}\right\} 
\]
forms a subbasis for the compact-open topology on
$Y^{X \times T}$. Also, the
$M\left( K, U \right)  $ give a subbasis for
$Y^{X}$ and therefore the
$M \left( K', M\left( K,U \right)  \right) $ form
a subbasis for the topology on
$\left( Y^{X} \right)^{T}$.
Since we showed that these subbases correspond to one
another under the exponential correspondence, the
theorem is proved.
\end{proof}

\begin{proposition}[]
    If $X$ is locally compact Hausdorff and $Y$ and $W$ are
    Hausdorff, then there is the homeomorphism
    \[
    Y^{X} \times W^{X}
    \stackrel{\cong}{\to} \left( Y \times W \right)^{X}
    \] 
    given by $\left( f,g \right) \mapsto 
    f \times g$.
\end{proposition}

\begin{proof}
    It is clearly a bijection.
    If $K, K' \subset X$ are compact
    and $U \subset Y$ and
    $V \subset W$ are open, then
    \begin{align*}
        \left( f,g \right) 
        \in M \left( K,U \right) \times 
        M\left( K',V \right) 
        &\iff \left( x \in K \implies f(x) \in U \right) 
        \text{ and } \left( x \in K' \implies
        g(x) \in V \right) \\
        &\iff \left( (x,y) \in K \times K' \implies
        f \times g(x,y) \in U \times V \right) \\
        &\iff f\times g \in 
        M\left( K , U \times W \right) \cap
        M\left( K', U \times W \right) .
    \end{align*}
    so $\left( f, g \right) \mapsto f \times g$ is an
    open map.\\
    Also $\left( f,g \right) \in 
    M(K, U) \times M(K,V) \iff
    f \times g \in M \left( K, U \times V \right) $ which
    implies that the function is continuous.
\end{proof}

\begin{proposition}[]
    If $X$ and $T$ are locally compact Hausdorff spaces
    and $Y$ is an arbitrary Hausdorff space, then there
    is the homeomorphism
    \[
    Y^{X \sqcup T} \stackrel{\cong}{\to} 
    Y^{X} \times Y^{T}
    \] 
    taking $f \mapsto \left( f \circ \iota_X,
    f \circ \iota_T \right) $.
\end{proposition}

\begin{proof}
    The map is clearly well-defined and
    injective. Also, given $\left( f,g \right) 
    \in Y^{X} \times Y^{T}$, we can
    define a function $f \cup  g\colon
    X \sqcup T \to Y$ by
    $f$ on $X$ and $g$ on $T$, and
    clearly,
    $f \cup g \mapsto (f,g)$ under the correspondence, giving
    surjectivity. We must show that
    it is continuous and has continuous inverse.\\
    Let $f \colon X \sqcup T \to Y$ and
    suppose $\left( 
    f \circ \iota_X , f \circ \iota_T \right) 
    \in M\left( K, U \right) \times 
    M\left( K', V \right)$.
    Then
    $f \in 
    M \left( K, U \right) \cap
    M\left( K', V \right)$ which is an open set
    that is mapped precisely to
    $M \left( K , U \right) \times 
    M \left( K', V \right) $. Hence
    $f\mapsto \left( f \circ \iota_X, f \circ \iota_T \right) $ 
    is continuous.\\

    Conversely, note that
     under the correspondence,
     $M\left( C \sqcup C',U \right) $ is mapped
     to
     $M(C, U) \times M(C',U)$, so
     the map is also open.
\end{proof}

\begin{theorem}[]
    For $X$ locally compact and both $X$ and $Y$ Hausdorff,
    $Y^{X}$ is a covariant functor of $Y$ and
    a contravariant functor of $X$
    from $\Top$ to $\Top$.
\end{theorem}



\begin{proof}
    A map $\varphi  \colon Y \to Z$ induces
    $\varphi^{X} \colon Y^{X} \to Z^{X}$ (put differently,
    $\varphi $ induces
    $\varphi_* \colon \Hom (X,Y) \to \Hom(X,Z)$.))
    We must show that $\varphi^{X}$ is continuous.
    By Theorem \ref{Thm:Compact-Open-Top}, it suffices
    to show that the map
    $Y^{X} \times X \to Z$ given by
    $\left( f,x \right) \mapsto \varphi \left( f(x) \right) $ 
    is continuous, but this is
    the composition $\varphi \circ e$ which is 
    thus continuous.\\
    Next, for the contravariant part, we must show that
    for $\psi \colon X \to T$, both spaces locally compact, we
    have that 
    $Y^{\psi } \colon Y^{T} \to Y^{X}$ given
    by $\psi^{*} \colon f \mapsto f \circ \psi $ is continuous.
    By the same theorem as above, it suffices to show that
    $Y^{T} \times X \to Y$ taking
    $\left( f,x \right) \mapsto f \left( \psi (x) \right) $ is
    continuous, but this is
    $e \circ \left( \id \times \psi  \right) $, which is
    continuous.
\end{proof}

\begin{corollary}
    For $A \subset X$ both locally compact and $X,Y$ Hausdorff,
    the restriction $Y^{X} \to Y^{A}$ is continuous.
\end{corollary}

\begin{proof}
    Apply the contravariant functor
    $\Hom \left( -,Y \right) = 
    Y^{-}$ to the inclusion $\iota \colon A \hookrightarrow X$.
\end{proof}

\begin{theorem}[]
    For $X,Y$ locally compact, and $X,Y,Z$ Hausdorff, the
    function
    \[
    Z^{Y} \times Y^{X} \to Z^{X}
    \] 
    taking $\left( f,g \right) \mapsto f \circ g$ is continuous.
\end{theorem}

\begin{proof}
    Again, by Theorem \ref{Thm:Compact-Open-Top}, it
    suffices to show that the map
    $Z^{Y} \times Y^{X} \times X \to Z$ taking
    $\left( f,g,x \right) \mapsto \left( f \circ g \right) (x)$ 
    is continuous, but this is simply 
    $e \circ \left( \id \times e \right) $.
\end{proof}



%\section{Homotopy Groups}

\subsection{Homotopy}
We follow chapter 14 of \cite{Bredon} for this subsection.\\

To start of, we recall the basic definitions of homotopies.

\begin{definition}[Homotopy]
    Two maps $f_0, f_1 \colon X \to Y$ are said to
    be \textit{homotopic} if there exists a homotopy
    $F \colon X \times I \to Y$ such that
    $F(x,0) = f_0(x)$ and $F(x,1) = f_1(x)$ for
    all $x \in X$.
\end{definition}

\begin{definition}[Homotopy equivalence]
    A map $f \colon X \to Y$ is said to be a \textit{homotopy
    equivalence} if it is an isomorphism in
    $\hTop$.
\end{definition}

\begin{lemma}[Reparametrization Lemma]
    Let $\varphi_1, \varphi_2$ be maps
    $\left( I, \partial I \right) \to 
    \left( I, \partial I \right) $ which are equal on
    $\partial I$. Let
    $F \colon X \times I \to Y$ be a homotopy and let
    $G_i (x,t) = F\left( x, \varphi_i(t) \right) $ for
    $i = 1,2$. Then $G_1 \simeq G_2 \rel
    X \times \partial I$.
\end{lemma}

We shall use $c$ to denote the constant homotopy.

\begin{proposition}[]
    $F * c \simeq F \rel X \times \partial I$ and
    $c * F \simeq F \rel X \times \partial I$.
\end{proposition}

\begin{definition}[]
    If $F \colon X \times I \to Y$ is a homotopy, then we
    define $F^{-1} \colon X \times I \to Y$ by
    $F^{-1}\left( x,t \right) = F(x,1-t)$. 
\end{definition}

Note that $F^{-1}$ is precisely the inverse
to $F$ in $\hTop$.

\begin{proposition}[]
    For any homotopies $F,G,H$ for which the
    concatenations 
     are defined, we have
     \[
         \left( F * G  \right) * H
         \simeq F * \left( G * H \right) 
         \rel X \times \partial I.
     \] 
\end{proposition}


\begin{proposition}[]
    For homotopies $F_1, F_2, G_1, G_2$,
    if $F_1 \simeq F_2 \rel X \times \partial I$ and
    $G_1 \simeq G_2 \rel X \times \partial I$, then
    $F_1 * G_1 \simeq F_2 * G_2 \rel X \times \partial I$.
\end{proposition}

Note that all of the discussion of concatenation of
homotopies goes through with no difficulties for the cases
in which all homotopies are relative to some subspace
$A \subset X$ or are homotopies of pairs
$\left( X, A  \right) \to \left( Y, B \right) $.\\
It follows that homotopy between maps of
pairs $\left( X,A \right) \to \left( Y,B \right) $ is
an equivalence relation. The set of homotopy classes
of these maps is commonly denoted by
$\left[ X,A ; Y ,B \right] $ or just
$\left[ X;Y \right] $ if $A = \varnothing$.

\begin{theorem}[]\label{Thm:299221}
    If $f_0 \simeq f_1 \colon X \to Y$ then
    $M_{f_0} \simeq M_{f_1} \rel
    X + Y$ and
    $C_{f_0} \simeq C_{f_1} \rel
    Y + \text{vertex}$.
\end{theorem}


To show this, one needs the following basic topological
proposition:
\begin{proposition}[] \label{prop:92031999}
    If $f \colon X \to Y$ is a quotient map and
    $K$ is locally compact Hausdorff, then
    $f \times 1 \colon X \times K \to Y \times K$ is
    a quotient map.
\end{proposition}

\begin{proof}[Proof of Theorem \ref{Thm:299221}]
    First, let $F \colon X \times I \to Y$ be the homotopy
    between $f_0$ and $f_1$. Now define $h \colon
    M_{f_0} \to M_{f_1}$ by $h(y) = y$ for
    $y \in Y$ and
    \[
    h\left( x,t \right) = 
    \begin{cases}
        F\left( x,2t \right) ,& t\le \frac{1}{2}\\
        (x, 2t-1),& \frac{1}{2} \le t.
    \end{cases}
    \] 
    Define
    $k \colon M_{f_1} \to M_{f_0}$ likewise by
    the identity on $Y$ nad
    \[
    k\left( x,t \right) =
    \begin{cases}
        F^{-1}\left( x,2t \right) ,& t\le \frac{1}{2}\\
        (x,2t-1),& \frac{1}{2}\le t
    \end{cases}.
    \] 
    Then the composition
    $kh \colon M_{f_0} \to M_{f_1}$ is the identity
    on $Y$ and 
    $F * \left( F^{-1} * E \right) $ on
    the cylinder portion, where $E \colon X \times I \to 
    M_{f_0}$ is induced by the identity on
    $X \times I \to X \times I$.
    This is homotopic to the identity 
    $\rel X \times \left\{ 1 \right\} + Y$.
    Similarly for $hk$.
    In now remains to check the continuity of this homotopy.
    We have a homotopy $M_{f_0} \times I \to 
    M_{f_0}$. We now claim that
    $M_{f_0} \times I \cong M_{f_0 \times I}$. Indeed
    then, using that
    $M_{f_0 \times I} = 
    \frac{X \times I \times I \sqcup Y \times I}{
    \left( (x,0,k) \sim (f_0(x),k \right) }$, it suffices
    to show continuity of the composition
    $X \times I \times I \sqcup Y \times I
    \to M_{f_0} \times I \to M_{f_0}$. 
    For on $Y \times I$, it is the constant homotopy and
    on $X \times I \times I$ it is
    $F * \left( F^{-1} * E \right) \simeq E
    \rel X \times \partial I$. 
    Now, that $M _{f_0} \times I
    \cong M_{f_0 \times I}$ follows from
    Proposition \ref{prop:92031999}.

\end{proof}

Let $f \colon X \to Y$. If $\varphi  \colon Y \to Y'$ is a map,
then there is the induced map
$F \colon M_{f} \to M_{\varphi \circ f}$ induced from
$\varphi $ on $Y$ and the identity on $X \times I$.

\begin{theorem}[]
    If $\varphi  \colon Y \to Y'$ is a homotopy equivalence
    then so is 
    $F \colon \left( M_f , X \right) \to 
    \left( M_{\varphi  \circ f}, X \right) $ and hence
    so is $F \colon C_f \to C_{\varphi  \circ f}$.
\end{theorem}

\begin{proof}
    Let $\psi  \colon Y' \to Y$ be a homotopy inverse
    of $\varphi $ and let $G \colon 
    M_{\varphi \circ f} \to M_{\psi \circ
    \varphi \circ f}$ be the map induced by
    $\psi $ on $Y'$ and the identity on $X \times I$.
    The composition $GF \colon M_f \to M_{\psi \circ \varphi 
    \circ f}$ is induced from $\psi \circ \varphi \colon
    Y \to Y$ and the identity on $X \times I$.
    Let $H \colon Y \times I \to Y$ be a homotopy from
    $\id$ to $\psi \circ \varphi $ ; i.e.,
    $H(y,0) = y$ and $H(y,1) = 
    \psi \varphi (y)$. 
    By the proof of Theorem \ref{Thm:299221}, there is a
    homotopy equivalence
    $h \colon M_f \to M_{\psi \circ \varphi \circ f}\rel X$ given
    by $h(y) = y$ and
     \[
    h(x,t) = 
    \begin{cases}
        H\left( f(x), 2t \right) ,& t\le \frac{1}{2}\\
        (x,2t-1),& t\ge \frac{1}{2}
    \end{cases}.
    \] 
    We claim that
    $h \simeq GF \rel X$. Indeed, the homotopy $H$ can
    be extended to 
    $M_f \times I \to M_{\psi \circ \varphi \circ f}$ by
    putting
    \[
    H\left( (x,s),t \right) 
    =
    \begin{cases}
        H\left( f(x), 2s+t \right) ,& 2s+t \le 1\\
        \left( x, \frac{2s+t-1}{t+1} \right) ,& 2s+t\ge 1
    \end{cases}.
    \] 

    Then $H\left( -,0 \right) = h$ and
    $H\left( -,1 \right) = GF$, so
    since $GF$ is a homotopy equvalence, so is
    $h$.
    Define $F' \colon
    M_{\psi \circ \varphi \circ f} \to 
    M_{\varphi \circ \psi \circ \varphi \circ f}$ 
    as the induced map on mapping cones
    with $\varphi $ on $Y$ and
    the identity on $X \times I$. Then similarly,
    $F' G$ is a homotopy equivalence.\\
    If $k$ is a homotopy inverse of $GF$ then
    $GF k \simeq \id$. If
    $k'$ is a homotopy inverse of $F'G$ then
    $k' F' G \simeq \id$. Thus $G$ has a right
    and left homotopy inverse: $R = Fk$ and
    $L = k'F'$. Then
    $R = \id \circ R \simeq 
    \left( LG \right) R =
    L \left( GR \right) \simeq L \circ \id = L$, so
    $R \simeq L$. That is, 
    $G$ has a homotopy inverse. Therefore,
    $G$ is a homotopy equivalence. Since $G$ and $GF$ are
    homotopy equivalences, so is $F$.
\end{proof}


\begin{problem}[]
    \cite[Ex 14.1]{Bredon} Let $S^2 \cup A$ denote the
    union of the unit $2$-sphere and the line segment
    joining the north and south poles. Show that
    $S^2 \vee S^{1} \simeq
    S^2 \cup A$.
\end{problem}

\begin{proof}
    Define two maps
    $f_0,f_1 \colon \left\{ 0,1 \right\}  \to 
    S^2$ where
    $f_0 (t) = \left( \cos (2\pi t), \sin(2\pi t), 0 \right) $ 
    and $f_1$ is the constant map at $(1,0,0)$. Then
    $f_0 \simeq f_1$, so $C_{f_0} \simeq C_{f_1}$. Now,
    $C_{f_0} = S^2 \cup  A$ while
    $C_{f_1} = S^2 \vee S^{1}$.
\end{proof}

\begin{problem}[]
    \cite[Ex 14.2]{Bredon} 
    Show that the union of a $2$-sphere and a flat
    unit  $2$-cell through the origin is homotopically
    equivalent to the one-point union of two $2$-spheres.
\end{problem}

\begin{proof}
    A $2$-cell is contractible, an
    a $2$-sphere with a $2$-cell inside it is precisely the
    cone of the map
    $S^1 \sqcup S^1 \to S^1$ with the identity on both.
    By \cite[Thm 14.19]{Bredon},
    this is homotopy equivalent to the cone on
    $S^1 \sqcup S^1 \to \left\{ * \right\} $ which is
    $S^2 \vee S^2$.
\end{proof}

\begin{problem}[]
    Show that the union of a standard $2$-torus with two disks,
    one spanning a latitudinal circle and the
    other spanning a longitudinal circle of the torus, is
    homotopically equivalent to a $2$-sphere.
\end{problem}

\begin{proof}
     Using the identification of the torus as the
     quotient space of $I^2$ in the usual way, we can choose
     on spanning circle to be a $2$-cell attached
     along $\left\{ 0 \right\} \times I$ and the
     other to be a $2$-cell attached along
     $I \times \left\{ 0 \right\} $. These are contractible, 
     and the quotient space becomes a $2$-sphere.
\end{proof}


\subsection{Homotopy Groups}

Recall that $\left[ X,A ; Y ,B \right] $ denotes the
set of homotopy classes of maps $X \to Y$ carrying $A$ into
$B$ such that $A$ goes into $B$ during the entire homotopy.

To make a group then, we can select a point $y_0 \in Y$ and
consider the set
\[
\left[ X \times I, X \times \partial I ;
Y , \left\{ y_0 \right\} \right] 
\] 
In this case, the operation of concatenation of homotopies
makes this set into a group.
It is technically also better to choose a basepoint 
$x_0 \in X$ and consider
\[
\left[ X \times I, \left\{ x_0 \right\} \times I
\cup X \times \partial I ; Y , \left\{ y_0 \right\} \right] .
\] 

For the moment, let us set
$A = \left\{ x_0 \right\} \times I \cup 
X \times \partial I$. Then maps
$X \times I \to Y$ which carry $A$ into $\left\{ y_0 \right\} $ 
are in bijective correspondence with maps 
$\left( X \times I \right) / A \to Y$ which take
 the point $\left\{ A \right\} $ into 
 $\left\{ y_0 \right\} $. 
 
 \begin{definition}[Reduced Suspension]
     We define the \textit{reduced suspension} of
     $X$ to be
     \[
     SX = (X \times I) / A =
     \left( X \times I \right) /
     \left( \left\{ x_0 \right\} \times I
     \cup X \times \partial I \right) 
     \] 
 \end{definition}

 The set of homotopy classes of pointed maps
 of a pointed space $X$ to a pointed space $Y$ with
 homotopies preserving the base points will
 be denoted by $\left[ X;Y \right]_* $. 

 Thus
 $\left[ SX;Y \right]_* $ is in canonical bijective
 correspondence with
 $\left[ X \times I, A ; Y , \left\{ y_0 \right\}  \right] $.


 Now, suppose we have pointed maps
 $f,g \colon SX \to Y$. Then they
 induce homotopies
 $f',g' \colon X \times I \to Y$ by precomposing with the
 quotient map
  $X\times I \to SX$. We can then define
  $f' * g' \colon X \times I \to Y$ as usual.
  The resulting pointed map
  $SX \to Y$ will be denoted $f * g$.
  Geometrically, $f * g$ is obtained by
  putting $f$ on the bottom and $g$ on the top
  of the one-point union $SX \vee SX$ and composing
  the resulting map $SX \vee SX \to Y$ with the
  map $SX \to SX \vee SX$ obtained by collapsing the
  middle parameter value $\frac{1}{2}$ copy of
  $X$ in $SX$ to the base point.
  
  \begin{figure}[htpb]
      \centering
      \includegraphics[width=0.6\textwidth]{Figures/TKISO0932.png}
      \caption{The product of two map classes
      $SX \to Y$.}
      \label{fig:TKISO0932-png}
  \end{figure}


  For a map $f \colon \left( SX, \left\{ A \right\}  \right) 
  \to \left( Y, \left\{ y_0 \right\}  \right) $, we denote its
  homotopy class in
  $\left[ SX; Y \right]_{*}$ by
  $\left[ f \right] $, and we define
  \[
  \left[ f \right] \left[ g \right] =
  \left[ f*g\right] 
  \] 
  Under this operation, the set
  $\left[ SX;Y \right]_*$ becomes a group.

  \begin{proposition}[]
      The reduced suspension gives
      $S S^{n-1}\cong S^{n}$.
  \end{proposition}

  Thus, we can define $S^{n}$ as the $n$-fold reduced
  suspension of $S^{0}$. As a special case,
  the set $\left[ S^{n};Y \right]_*$ then becomes
  a group for $n>0$. 

  \begin{definition}[$n$ th homotopy group]
      We define
      \[
      \pi_n \left( Y, y_0 \right) =
      \left[ S^{n}; Y \right]_*
      \] 
      with this operation.
  \end{definition}

  \subsection{Homotopy Groups using H-Spaces/Groups/Cogroups}

  From now on, unless otherwise indicated, we regard
  the $n$-sphere $S^{n}$ as having the cogroup
  structure as the reduced suspension
  $S^{n} = S S^{n-1} = S^{n-1} \wedge S^{1}$ - i.e.,
  the map $\gamma$ in the definition of an H-cogroup will
  be $\gamma \colon S^{n} \to S^{n} \vee S^{n}$ 
  given by
  \[
  \gamma(t,x) = 
  \begin{cases}
      \left( 2t,x \right)_{1},& t\le \frac{1}{2}\\
      \left( 2t-1, x \right)_2,& t\ge \frac{1}{2}.
  \end{cases}
  \] 
  The
  $0$-sphere $S^{0}$ is $\left\{ 0,1 \right\} $ 
  with base point $\left\{ 0 \right\} $.\\

  For a based space $X$ with base point $x_0$, we define
  the nth homotopy group
  \[
  \pi_n (X,x_0) = 
  \left[ S^{n},* ; X, x_0 \right] .
  \] 
  This is a group with the product defined by
  Theorem \ref{Thm:H-group-cogroup}.(2).

  \begin{theorem}[]\label{Theorem:Htpy-Groups-Abelian}
      If $X$ is an H-space then the multiplication
      in $\pi_n (X,x_0)$ is induced by
      the H-space multiplication and is abelian for
      $n\ge 1$.
  \end{theorem}

  \begin{proof}
      This follows directly from Theorem \ref{Thm:H-group-cogroup}.
  \end{proof}

  \begin{lemma}[]\label{Lemma:Suspension-Loop-Space}
      In the pointed category,
      $\left[ SX ; Y \right] \cong
      \left[ X ; \Omega Y \right]$ as groups.
  \end{lemma}
  
  \begin{proof}
      Recall the characteristic correspondence for
      the compact-open topology (Theorem \ref{Thm:Compact-Open-Top}):
      \[
      f\colon X \times S^{1} \to Y 
      \leftrightarrow f' \colon X \to Y^{S^{1}}
      \] 
      given by
      $f'(x) (t) = f(x,t)$.
      Recall that $SX = X \wedge S^{1}$, so
      a map
      $g \colon X \wedge S^{1} \to Y$ induces a map
      $f \colon X \times S^{1} \to Y$ by
      the composition 
      $X \times S^{1} \to X \wedge S^{1} \stackrel{g}{\to} Y$.
      Then $f$ is continuous if and only if
      the map $f' \colon X \to Y^{S^{1}}$ is continuous, where
      $f'(x)(*) = f(x,*) = *$, so
      $f'(*) = * \in Y^{S^{1}}$, the basepoint
      of $\Omega Y$.
      So the correspondence induces a bijective
      correspondence between pointed maps
      $SX \to Y$ and pointed maps $X \to \Omega Y$, and
      pointed homotopies correspond as well.\\
      It remains to show that the correspondence
      is a group homomorphism.
      Recall that $SX$ is an H-cogroup and
      $\Omega X$ is an H-group, so using
      Theorem \ref{Thm:H-group-cogroup}, we
      get that 
      for $f,g \colon SX \to Y$, the product in
      $\left[ SX;Y \right] $ is induced by
      \[
          \left( f*g \right) (x,t) = 
          \begin{cases}
              f(x,2t),& t\le \frac{1}{2}\\
              g\left( x,2t-1 \right),& t\ge \frac{1}{2},
          \end{cases}
      \] 
      which is equal to
      $\left( f*g \right)'(x)(t)$, 
      while the multiplication in
      $\left[ X ; \Omega Y \right] $ is given by
      $\left( f' \cdot g' \right) (x) = 
      f'(x) * g'(x)$ where $*$ is loop concatenation. 
      At time $t$, this is
      $f'(x) (2t)$ for $t\le \frac{1}{2}$ and
      $g'(x) (2t-1)$ for $t\ge \frac{1}{2}$. Thus
      $\left( f*g \right) ' = f' \cdot  g'$.
  \end{proof}


  All of the above immediately carries over to pointed
  pairs $(X,A)$ with a base point in $A$, so
  $\left[ SX, SA ; Y,B \right] $ is a group that
  is canonically isomorphic to
  $\left[ X,A; \Omega Y, \Omega B \right] $.

  Note in particular that
  $D^{n} \cong S D^{n-1}$ for all $n\ge 2$, so
  $D^{n} = D^{1} \wedge S^{n-1} \supset 
  S^{0} \wedge S^{n-1} = S^{n-1}$, so
  $\left( D^{n}, S^{n-1} \right) 
  = S^{n-1} \left( D^{1}, S^{0} \right) $, the
  $(n-1)$-fold reduced suspension. Hence
  we can define the relative homotopy group by
  \[
  \pi_n \left( Y,B,* \right) =
  \left[ D^{n}, S^{n-1}; Y ,B \right] 
  = \left[ S^{n-1}\left( D^{1}, S^{0} \right) ;
  Y,B\right] .
  \] 
  Note that this is defined on pointed spaces and pointed maps,
  so this set is really
  $\left[ D^{n},S^{n-1},s_0; Y,B, * \right] $.
  This becomes a group for $n\ge 2$. 

  Next note that we have a quotienting map
  \[
      I^{n} = \underbrace{D^{1}}_{=I} \times I \times \ldots \times I
      \to D^{1} \wedge S^{1} \wedge \ldots
      \wedge S^{1} =
      D^{n}.
  \] 
  Under this map, 
  $\partial I^{n}$ corresponds to
  $S^{n-1}$ and
  the base point corresponds to
  $J^{n-1} = \left( I \times \partial I^{n-1} \right) 
  \cup  \left( \left\{ 0 \right\} \times I^{n-1} \right) $.
  Thus under this quotienting map, we obtain a
  bijection
  \[
  \pi_n(Y,B,*) =
  \left[ D^{n}, S^{n-1},s_0; Y,B, * \right] 
  \cong \left[ I^{n}, \partial I^{n}, J^{n-1};
  Y,B,*\right].
  \] 

  \begin{corollary}
      $\pi_n(Y,*)$ is abelian for
      $n\ge 2$ and $\pi_n (Y,B,*)$ is abelian for
      $n\ge 3$. Moreover, the group structure is
      independent of the suspension coordinate used
      to define it.
  \end{corollary}

  \begin{proof}
      Recall from Lemma \ref{Lemma:Suspension-Loop-Space}, that
      $\left[ S^{n};Y \right] 
      = \left[ S^{1} \wedge \ldots \wedge S^{1};Y \right] 
      \cong \left[ S^{n-1} ; \Omega Y \right] $ as
      groups. Now
      the loop structure corresponds to the suspension in
      the last coordinate by definition, and
      by Theorem \ref{Thm:H-group-cogroup}, this
      is the same as the group 
      $\left[ S^{n};Y \right] $ with the suspension
      operation on any of the coordinates, since
      the choice of coordinate is arbitrary, this
      shows that
      the group structure on $\pi_n(Y,*) =
      \left[ S S^{n-1};Y \right]
      = \left[ S^{1} \wedge \ldots \wedge
      S^{1}; Y \right] $ is independent
      of the suspension coordinate used to define it.\\
      The product in $\left[ S^{n-1}; \Omega Y \right] $ is
      furthermore abelian for $n-1 \ge 1$ by
      Theorem \ref{Theorem:Htpy-Groups-Abelian} (using that
      $\Omega Y$ is an H-space), and
      the relative case is similar.
  \end{proof}

  \begin{corollary}
      \[
      \pi_n\left( Y,* \right) \cong
      \pi_{n-1}\left( \Omega Y, * \right) 
      \cong \ldots \cong
      \pi_1\left( \Omega^{n-1}Y,* \right) 
      \cong \pi_0 \left( \Omega^{n}Y ,* \right) 
      \] 
      and similarly in the relative case.
  \end{corollary}

  \begin{theorem}[]\label{Thm:Bredon-4.5}
      Let $A$ be a closed subspace of $X$ containing the
      base point $*$. Suppose that $F \colon X \times I \to X$ 
      is a deformation of $X$ contracting $A$ to
      $*$ ; i.e.,
      \begin{align*}
          F(A \times I) 
          &\subset A\\
          F(x,0) 
          &= x\\
          F\left( A \times \left\{ 1 \right\}  \right) 
          &= *\\
          F\left( \left\{ * \right\} \times I \right) 
          &= *
      \end{align*}
      then the quotient map $X \to X /A$ is a homotopy equivalence.
      Similarly for pairs $\left( X,X' \right) $ with
      $A \subset X'$.
  \end{theorem}

  \begin{proof}
      Let
      $\psi \colon X / A \to X$ be defined by the
      commutative diagram
      \begin{equation*}
      \begin{tikzcd}
          X \ar[d, "\varphi "] 
          \ar[dr, "F_{X \times \left\{ 1 \right\} } "]& \\
          X / A \ar[r, "\psi"] & X
      \end{tikzcd}
      \end{equation*}
      We claim that. Let $\varphi \colon X \to X / A$ be
      the quotienting map. We claim that
      $\psi \varphi \simeq \id_{X}$ and
      $\varphi \psi \simeq \id_{X / A}$.
      Since $F\left( A \times I \right) \subset A$, 
      $F$ induces a homotopy
      $F' \colon X / A \times I \to X / A$, where
      $F_{X / A \times \left\{ 1 \right\} }
      = \varphi \psi $, so
      since $F_{X / A \times \left\{ 0 \right\} }
      = \id_{X / A}$, we get the result.
  \end{proof}


  \newpage




  \subsubsection{A different way of defining
  $\pi_n \left( Y, y_0 \right) $}
  Note that reduced suspension supplies a parameter in
  $\left[ 0,1 \right] $ and the space
  $S^{n}$ as constructed is the quotient space of
  $I^{n}$ obtained by collapsing the boundary of the cube to a
  point.
  Pointed maps $S^{n}\to Y$ are in bijective correspondence
  with maps $I^{n}\to Y$ taking $\partial I^{n}$ to
  the base point of $Y$. This is a more traditional way
  of defining $\pi_n(Y)$. This becomes the group
  of homotopy classes of maps
  $\left( I^{n},\partial I^{n} \right) \to 
  \left( Y, \left\{ y_0 \right\}  \right) $ with the
  operation being
  \[
  f*g \left( t_1, \ldots, t_n \right) =
  \begin{cases}
      f\left( 2t_1, t_2, \ldots, t_n \right) ,& t_1 \in 
      \left[ 0,\frac{1}{2} \right] \\
      g\left( 2t_1-1, t_2, \ldots, t_n \right) ,& t_1 \in 
      \left[ \frac{1}{2},1 \right] 
  \end{cases}.
  \] 

  \begin{proposition}[]
      For $n\ge 2$, $\pi_n\left( X, x_0 \right)$ is abelian.
  \end{proposition}

  \begin{proof}
      Consider the homotopy in Figure \ref{fig:JIDWOOL0290L-png}.
      We begin by shrinking the domains of $f$ and $g$ to smaller
      subcubes of $I^{n}$, where the region outside is
      mapped to the basepoint. This allows us to move the boxes
      around in a continuous manner. The rest is clear.
      \begin{figure}[htpb]
          \centering
          \includegraphics[width=0.8\textwidth]{Figures/JIDWOOL0290L.png}
          \caption{The homotopy in question}
          \label{fig:JIDWOOL0290L-png}
      \end{figure}
  \end{proof}

  Next, we want to show that following:
  \begin{proposition}[]\label{Prop:SwjiaKKDNW1102}
      If $X$ is path-connected, then
      $\pi_n\left( X, x_0 \right) \cong
      \pi_n (X, x_1)$ for any two $x_0,x_1 \in X$.
  \end{proposition}

  For this, we introduce an action of
  $\pi_1$ on $\pi_n$.

  \begin{definition}[The action of $\pi_1$ on $\pi_n$]
      Given a path
      $\gamma \colon I \to X$ from
      $x_0$ to $x_1$, we associate to a map
      $f \colon \left( I^{n}, \partial I^{n} \right) \to 
      \left( X, x_1 \right) $ the map
      $\gamma f \colon \left( I^{n}, \partial I^{n} \right) 
      \to \left( X,x_0 \right) $ by shrinking the domain
      of $f$ to a smaller concentric cube in $I^{n}$, then
      inserting the path $\gamma$ on each radial segment
      in the shell between this smaller cube and $\partial
      I^{n}$.
      See Figure \ref{fig:JDWIXHHX011SJ-png}

      \begin{figure}[htpb]
          \centering
          \includegraphics[width=0.25\textwidth]{Figures/JDWIXHHX011SJ.png}
          \caption{Depiction of $\gamma f$.}
          \label{fig:JDWIXHHX011SJ-png}
      \end{figure}

  \begin{note}
      We have the following properties
      \begin{enumerate}
          \item $\gamma \left( f+ g \right) 
              \simeq \gamma f + \gamma g$.
          \item $\left( \gamma \eta \right) f \simeq
              \gamma \left( \eta f \right) $.
          \item $\id f \simeq f$, where
              $\id$ denotes the constant path.
      \end{enumerate}

      To see $(1)$, first deform $f$ and $g$ to be
      constant on the right and left halves of
      $I^{n}$, respectively, producing maps
      which we may call $f+0$ and $0+g$, then we 
      can excise a progressively wider symmetric middle slab
      of $\gamma (f+0) + \gamma(0+g)$ (which can be
      seen on the left in Figure \ref{fig:WIWIWSSK11-png})
      until it becomes $\gamma \left( f+g \right) $ (shown on the
      right).

      \begin{figure}[htpb]
          \centering
          \includegraphics[width=0.8\textwidth]{Figures/WIWIWSSK11.png}
          \caption{}
          \label{fig:WIWIWSSK11-png}
      \end{figure}
  \end{note}

  Now if $\beta_{\gamma} \colon \pi_n(X,x_1) \to 
  \pi_n(X, x_0)$ is the change-of-basepoint transformation,
   $\beta_{\gamma}\left[ f \right] =
   \left[ \gamma f \right] $, then
   the above note shows that $\beta_\gamma$ is a group isomorphism.
   This proves Proposition \ref{Prop:SwjiaKKDNW1102}. 
   If we restrict attention to loops
   $\gamma$ at $x_0$, then since $\beta_{\gamma \eta}=
   \beta_{\gamma} \beta_{\eta}$, the map
   $\left[ \gamma \right] \mapsto \beta_{\gamma}$ 
   defines a homomorphism from
   $\pi_1\left( X, x_0 \right) $ to
   $\Aut \left( \pi_n \left( X,x_0 \right)  \right) $ 
   called the \textit{action of $\pi_1$ on $\pi_n$ }.
  \end{definition}

  \begin{note}
  For $n>1$, this action makes
  $\pi_n(X,x_0)$ into a module over the group ring
  $\mathbb{Z}\left[ \pi_1 \left( X,x_0 \right)  \right] $.
  \end{note}  

  \begin{definition}[Simple/abelian spaces]
      A space with trivial $\pi_1$ action on $\pi_n$ is called
      '$n$-simple', and 'simple' means
      ' $n$-simple for all $n$ '. We call
      a space \textit{abelian} if it has
      trivial action of $\pi_1$ on all homotopy groups
      $\pi_n$.
  \end{definition}

  \begin{proposition}[$\pi_n$ is a functor]
      A map $\varphi  \colon \left( X, x_0 \right) \to 
      \left( Y, y_0 \right) $ induces a map
      $\varphi_* \colon \pi_n \left( X, x_0 \right) \to 
      \pi_n \left( Y, y_0 \right) $ defined by
      $\varphi_* \left[ f \right] = \left[ \varphi  f \right] $.
      It is immediate from the definitions that
      $\varphi_*$ is well-defined and a homomorphism
      for $n\ge 1$. The functorial properties are also clear.
  \end{proposition}

  \begin{corollary}
      Homotopy equivalent spaces have isomorphic
      homotopy groups.
  \end{corollary}

  \begin{proposition}[]
      A covering space projection
      $p \colon \left( \tilde{X}, \tilde{x}_0 \right) \to 
      \left( X, x_0 \right) $ induces isomorphisms
      $p_* \colon \pi_n \left( \tilde{X}, \tilde{x}_0 \right) 
      \to \pi_n \left( X, x_0 \right) $ for all
      $n \ge 2$.
  \end{proposition}

  \begin{proof}
      Since 
      $S^{n}$ is path-connected and locally path-connected,
      and simply connected for $n\ge 2$, we find that
      any map
      $\left( S^{n},s_0 \right) 
      \to \left( X, x_0 \right) $ lifts to a 
      map $\left( S^{n},s_0 \right) \to 
      \left( \tilde{X},\tilde{x}_0 \right) $ when
      $n\ge 2$. This gives surjectivity of
      $p_*$.
      For injectivity, suppose
      $p_* \left[ f \right] = \left[ 0 \right] $ where
      $f \colon \left( S^{n}, s_0 \right) \to 
      \left( \tilde{X},\tilde{x}_0 \right) $.
      Let $c_{\tilde{x}_0}$ be the constant map at
      $\tilde{x}_0$. Then
      $p_* \left[ \tilde{x}_0 \right] =
      \left[ 0 \right] $, so by uniqueness of the
      lifting theorem, 
      $\left[ f \right] = \left[ c_{\tilde{x}_0} \right] =
      \left[ 0 \right] $.
  \end{proof}

  \begin{definition}[Aspherical]
      Spaces with $\pi_n = 0$ for all
      $n\ge 2$ are called \textit{aspherical}.
  \end{definition}

  \begin{corollary}
      $S^{1}, T^{n}$ and $K$ are aspherical since
      they have contractible covering spaces.
  \end{corollary}


  \begin{proposition}[]
      \[
      \pi_n \left( \prod_{\alpha} X_{\alpha} \right) 
      \cong \prod_{\alpha} \pi_n \left( X_{\alpha} \right) 
      \] 
  \end{proposition}

  Next we define relative homotopy groups.

  \begin{definition}[Relative homotopy groups]
      Regard $I^{n-1}$ as a face of $I^{n}$ with the last
      coordinate $s_n = 0$ and let
      $J^{n-1}$ be the closure of
      $\partial I^{n}- I^{n-1}$. Then
      we define 
      \[
      \pi_n \left( X, A, x_0 \right) 
      := \left[ I^{n},\partial I^{n}, J^{n-1};
      X , A , x_0\right] 
      \] 
      We shall leave $\pi_0 \left( X, A, x_0 \right) $ undefined
      for now.
  \end{definition}

  We can define a sum operation on $\pi_n \left( X, A, x_0 \right) $ 
  in the same way as for $\pi_n \left( X, x_0 \right) $, except
  now the coordinate $s_n$ now must remain free, so
  we must use one of the other coordinates. Thus
  we must have at least one other coordinate to define
  the same operation. So $\pi_n \left( X, A, x_0 \right) $ is
  a group for $n\ge 2$, and it is abelian for
  $n\ge 3$. For $n=1$, we have
  $I^{1} = \left[ 0,1 \right] , I^{0} = \left\{ 0 \right\} $ 
  and $J^{0} = \left\{ 1 \right\} $, so
  $\pi_1 \left( X, A, x_0 \right) 
  = \left[ I, \left\{ 0 \right\} , \left\{ 1 \right\} ;
  X, A, x_0 \right] $ is the set of homotopy classes of paths in
  $X$ from a varying point in $A$ to the fixed basepoint
  $x_0 \in A$. In general, this is not a group in any
  natural way. \\
  \linebreak
  Now, we saw before that
  $\pi_n \left( X, x_0 \right) $ can be regarded as
  homotopy classes of maps $\left( S^{n}, x_0 \right) \to 
  \left( X, x_0 \right) $. Similarly, collapsing
  $J^{n-1}$ to a point, converts
  $\left( I^{n} , \partial I^{n}, J^{n-1} \right) $ 
  to $\left( D^{n}, S^{n-1}, s_0 \right) $.
  In this case, addition is done by
  the map $c \colon D^{n} \to D^{n} \vee D^{n}$ collapsing
  $D^{n-1} \subset D^{n}$ to a point.\\
  \linebreak
  \begin{theorem}[Compression criterion]\label{Thm:Compression}
      A map $f \colon \left( D^{n}, S^{n-1}, s_0 \right) 
      \to \left( X, A, x_0 \right) $ represents zero
      in $\pi_n \left( X, A, x_0 \right) $ if and only if
      it is homotopic $\rel S^{n-1}$ to a map with image
      contained in $A$.
  \end{theorem}
  
  \begin{proof}
      Suppose we have a homotopy
      $\rel S^{n-1}$ from $f$ to a map
      $g$, so
      $\left[ f \right] = \left[ g \right] $ in
      $\pi_n \left( X, A, x_0 \right) $. 
      Viewing $g$ as a map
      $\left( D^{n}, S^{n-1}, s_0 \right) 
      \to \left( X, A, x_0 \right) $ whose
      image is contained in $A$, we
      can construct the homotopy
      $H \colon D^{n} \times I \to X$ by
      $H(x,t) = g\left( (1-t) x + s_0 t \right) $ 
      which is a homotopy from $g$ to the
      constant map at $x_0$, hence
      $\left[ g \right]  = 0$ in $\pi_n (X, A, x_0)$.\\
      Conversely, if $\left[ f \right] = 0$ via
      a homotopy $F \colon D^{n} \times I \to X$ such that
      $F(x,0) = f(x)$ and
      $F(x,1) = x_0$ for all $x \in D^{n}$ and
      $F(x,t) \in A$ for all
      $x$ with $\left| x \right| = 1$ as well
      as $F(s_0,t) = x_0$ for all $t$. We can
      construct a homotopy
      using $F$ by restricting $F$ to a family of
      $n$-disks in $D^{n} \times I$ starting with
      $D^{n}\times \left\{ 0 \right\} $ and ending
      with the disk $D^{n} \times \left\{ 1 \right\} 
      \cup S^{n-1} \times I$, and where all the disks
      throughout the family have the same boundary.
      See Figure \ref{fig:DJIMMXKXO0O-jpeg} for a depiction
      of this homotopy.

      \begin{figure}[htpb]
          \centering
          \includegraphics[width=0.8\textwidth]{Figures/DJIMMXKXO0O.jpeg}
          \caption{}
          \label{fig:DJIMMXKXO0O-jpeg}
      \end{figure}
      This completes the proof.
  \end{proof}

  Next, some things that carry over:
  a map $\varphi \colon \left( X, A, x_0 \right) 
  \to \left( Y, B, y_0 \right) $ induces maps
  $\varphi_* \colon \pi_n \left( X, A, x_0 \right) 
  \to \pi_n \left( Y, B, y_0 \right) $ which are
  homomorphisms when $n\ge 2$ and have properties analogous
  to those in the absolute case: 
  $\left( \varphi \psi  \right)_* = 
  \varphi_* \psi_*, (\id_{(X,A,x_0)})_{*} = \id_{\pi_n (X, A, x_0)}$,
  and if  $\varphi \simeq \psi $ through maps
  $\left( X,A,x_0 \right) \to \left( Y,B,y_0 \right) $,
  then $\varphi_* = \psi_*$. 

  \subsubsection{LES of relative homotopy groups}
  Probably the most useful feature of relative homotopy
  groups $\pi_n (X,A,x_0)$ is that they 
  fit into a long exact sequence
  \[
  \ldots \to \pi_n (A,x_0)
  \stackrel{i_*}{\to} \pi_n(X,x_0)
  \stackrel{j_*}{\to} \pi_n (X,A,x_0)
  \stackrel{\partial}{\to} \pi_{n-1}(A,x_0) \to 
  \ldots \to \pi_0 (X,x_0).
  \] 
  Here $i$ and $j$ are the inclusions
  $\left( A, x_0 \right) \hookrightarrow
  (X,x_0)$ and
  $\left( X, x_0, x_0 \right) \hookrightarrow
  \left( X,A,x_0 \right) $. The map
  $\partial$ comes from restricting maps
  $\left( I^{n},\partial I^{n}, J^{n-1} \right) \to 
  \left( X,A,x_0 \right) $ to
  $I^{n-1}$ (the face of $I^{n}$ with the last
  coordinate $s_n = 0$ ),
  or equivalently, by restricting maps
  $\left( D^{n},S^{n-1},s_0 \right) \to 
  \left( X,A,x_0 \right) $ to $S^{n-1}$. The map $\partial$,
  called the \textit{boundary map}, is a homomorphism
  when $n>1$. In fact, we can show the following theorem

  \begin{theorem}[LES of relative homotopy groups]
      Given 
      $x_0 \in B \subset A \subset X$,
      the sequence of relative homotopy groups
  \[
      \ldots \to 
      \pi_n \left( A,B, x_0 \right) 
      \stackrel{i_*}{\to} 
      \pi_n \left( X, B, x_0 \right) 
      \stackrel{j_*}{\to} 
      \pi_n (X, A, x_0)
      \stackrel{\partial}{\to} 
      \pi_{n-1} \left( A,B, x_0 \right) 
      \to \ldots \to 
      \pi_1 (X,A,x_0)
  \] 
  is exact and natural.
  In the case when $B = \left\{ x_0 \right\} $, we have that
  the LES
  \[
  \ldots \to \pi_n (A,x_0)
  \stackrel{i_*}{\to} \pi_n(X,x_0)
  \stackrel{j_*}{\to} \pi_n (X,A,x_0)
  \stackrel{\partial}{\to} \pi_{n-1}(A,x_0) \to 
  \ldots \to \pi_0 (X,x_0).
  \] 
  is exact and natural.
  \end{theorem}

  \begin{proof}
      \textit{Exactness at $\pi_n
      \left( X,B,x_0 \right) $ :} the composition
      $j_* i_*$ is zero because any
      map $\left( I^{n}, \partial I^{n},
      J^{n-1}\right) \to \left( A,B,x_0 \right) $ 
      is zero in $\pi_n \left( X, A, x_0 \right) $ by the
      compression criterion (Theorem \ref{Thm:Compression}).
      To see that $\ker j_* \subset 
      \im i_*$, let
      $f \colon \left( I^{n}, \partial I^{n},
      J^{n-1}\right) \to \left( X, B, x_0 \right) $ 
      represent zero in $\pi_n \left( X, A, x_0 \right) $.
      Using the compression criterion again, we
      then get that $f$ is homotopic $\rel \partial I^{n}$ 
      to a map with image in $A$, hence the class
      $\left[ f \right] \in \pi_n \left( X, B, x_0 \right) $ 
      is indeed in the image of $i_*$. We conclude that
      $\ker j_* = \im i_*$, obtaining exactness
      at $\pi_n \left( X,B, x_0 \right) $.\\
      \textit{Exactness at $\pi_n (X,A,x_0)$:} 
      for a map  $\left[ f \right] 
      \in \im j_*$, we have that
      $j_*$ maps $\partial I^{n}$ into $B$, hence
      in particular $I^{n-1} \subset \partial I^{n}$ into
      $B$, so $\partial j_* \left[ f \right] $ 
      represents a homotopy class
      in $\pi_{n-1}\left( A,B,x_0 \right) $ with
      image in $B$, but then by the compression criterion,
      $\partial j_* \left[ f \right] = 0$ in
      $\pi_{n-1} \left( A,B,x_0 \right) $, so
      $\im j_* \subset \ker \partial $.
      Conversely, suppose
      $\partial \left[ f \right] = 0$. By the compression
      criterion, representatives of $\partial \left[ f \right] $
      are homotopic $\rel \partial I^{n-1}$ to a map
      with image in $B$. In particular,
      $f|_{I^{n-1}}$ is homotopic to a map with
      image in $B$ via a homotopy $F \colon
      I^{n-1} \times I \to A \rel \partial I^{n-1}$.
      We can tack $F$ onto $f$ to get a new map
      $\left( I^{n}, \partial I^{n},
      J^{n-1}\right) \to 
      \left( X, B, x_0 \right) $ which, as
      a map
      $\left( I^{n}, \partial I^{n}, J^{n-1} \right) \to 
      \left( X, A, x_0 \right) $ is homotopic to
      $f$ by the homotopy that tacks on increasingly longer
      initial segments of $F$. See Figure
      \ref{fig:IDIWKAKX-png}. Hence
      $\left[ f \right] \in \im j_*$.

      \begin{figure}[htpb]
          \centering
          \includegraphics[width=0.2\textwidth]{Figures/IDIWKAKX.png}
          \caption{}
          \label{fig:IDIWKAKX-png}
      \end{figure}

      \textit{Exactness at $\pi_n (A,B,x_0)$ :} 
      First, $i_* \partial$ is zero since
      the restriction of a map
      $f \colon \left( I^{n+1}, \partial I^{n+1},
      J^{n}\right) \to \left( X,A,x_0 \right) $ 
      to $I^{n}$ is homotopic $\rel \partial I^{n}$ to a 
      constant map via $f$ itself (a similar picture
      to Figure \ref{fig:DJIMMXKXO0O-jpeg} works).\\
      Conversely, if $B$ is a point, then
      a nullhomotopy $f_t \colon
      \left( I^{n}, \partial I^{n} \right) 
      \to \left( X, x_0 \right) $ of
      $f_0 \colon \left( I^{n},\partial I^{n} \right) 
      \to \left( A,x_0 \right) $ gives a map
      $F \colon \left( I^{n+1},\partial I^{n+1},J^{n} \right) 
      \to \left( X,A,x_0 \right) $ with
      $\partial \left( \left[ F \right]  \right) 
      = \left[ f_0 \right] $. So in this case, the proof is
      finished.
      For a general $B$, let
      $F$ be a nullhomotopy of
      $f \colon \left( I^{n},\partial I^{n},J^{n-1} \right) 
      \to \left( A,B,x_0 \right) $ through maps
      $\left( I^{n}, \partial I^{n}, J^{n-1} \right) 
      \to \left( X,B,x_0 \right) $ and
      let $g$ be the restriction of
      $F$ to $I^{n-1}$ in $I^{n-1} \times I = I^{n}$ (see
      the first of the pictures in
      Figure \ref{fig:USIIOOQ-png}).
      Next reparametrize the $n$ th and
      $(n+1)$ st coordinates as in the
      second picture. Then 
       we find that $f$ with $g$ tacked on
       is in the image of $\partial$. But
       as before, tacking $g$ onto $f$ gives the
       same element of $\pi_n (A,B,x_0)$

      \begin{figure}[htpb]
          \centering
          \includegraphics[width=0.4\textwidth]{Figures/USIIOOQ.png}
          \caption{}
          \label{fig:USIIOOQ-png}
      \end{figure}
  \end{proof}

  
\begin{corollary}
    Consider the inclusion
    $\iota \colon X = X \times \left\{ 0 \right\} 
    \hookrightarrow CX$.
    Then
    $\pi_n \left( CX, X, x_0 \right) 
    \cong \pi_{n-1}\left( X, x_0 \right) $ for all
    $n\ge 1$. Taking
    $n=2$, we can thus realize an group $G$, abelian
    or not, as a relative $\pi_2$ by
    choosing $X$ to have $\pi_1 (X) \cong G$.
\end{corollary}

There are also change-of-basepoint isomorphisms
$\beta_{\gamma}$ for relative homotopy groups.
One takes a path  $\gamma$ in $A \subset X$ from
$x_0$ to $x_1$ which induces
$\beta_{\gamma} \colon \pi_n (X,A,x_1) \to 
\pi_n (X,A,x_0)$ by setting
$\beta_{\gamma} \left( \left[ f \right]  \right) 
= \left[ \gamma f \right] $, where
$\gamma f$ is depicted in 
Figure \ref{fig:DIWIOA-png}.

\begin{figure}[htpb]
    \centering
    \includegraphics[width=0.2\textwidth]{Figures/DIWIOA.png}
    \caption{}
    \label{fig:DIWIOA-png}
\end{figure}

Restricting to loops at the
basepoint, the association $\gamma \mapsto 
\beta_{\gamma}$ defines an action
of $\pi_1 \left( A, x_0 \right) $ on
$\pi_n \left( X, A, x_0 \right) $ analogous to the
action of $\pi_1 \left( X, x_0 \right) $ on
$\pi_n (X,x_0)$.






%\include{pset1}

\begin{problem}[$n$-connected in the relative case]\label{n-connected-relative}
    The following four conditions are equivalent for
    $i>0$ :
    \begin{enumerate}
        \item Every map
            $\left( D^{i} , \partial D^{i} \right) \to 
            \left( X,A \right) $ is homotopic
            $\rel \partial D^{i}$ to a map $D^{i} \to A$.
        \item Every map $\left( D^{i},\partial D^{i} \right) 
            \to (X,A)$ is homotopic through such maps
            to a map $D^{i} \to A$.
        \item Every map $\left( D^{i}, \partial D^{i} \right) 
            \to \left( X,A \right) $ is homotopic through such
            maps to a constant map $D^{i} \to A$.
        \item $\pi_i \left( X, A, x_0 \right) = 0$ for all
            $x_0 \in A$.
    \end{enumerate}
    When $i = 0$, we did not define the relative $\pi_0$,
    and (1)-(3) are each equivalent to saying that
    each path-component of $X$ contains points
    in $A$ since $D^{0}$ is a point and
    $\partial D^{0}$ is empty. The pair
    $\left( X, A \right) $ is called \textit{$n$-connected}
    if (1)-(4) hold for $0<i\le n$ and
    (1)-(3) hold for  $i=0$.
\end{problem}


\subsection{Whitehead's Theorem}

\begin{theorem}[Whitehead's Theorem]\label{Thm:Whitehead}
    If a map $f \colon X \to Y$ between connected
    $CW$ complexes induces isomorphisms
    $f_* \colon \pi_n (X) \to \pi_n (Y)$ for all
    $n$, then $f$ is a homotopy equivalence.
    In case $f$ is the inclusion of a subcomplex
    $X \hookrightarrow Y$, the conclusion is stronger:
    $X$ is a deformation retract of $Y$.
\end{theorem}

The proof will require the following lemma:

\begin{lemma}[Compression Lemma]
    Let $\left( X,A \right) $ be a CW pair and let
    $\left( Y, B \right) $ be any pair with
    $B \neq \varnothing$. For each  $n$ such that
    $X - A$ has cells of dimension $n$, assume
    that  $\pi_n \left( Y, B, y_0 \right) = 0$ for
    all $y_0 \in B$. Then every map $f \colon
    \left( X,A \right) \to \left( Y,B \right) $ is homotopic
    $\rel A$ to a map $X \to B$.
    When $n = 0$, the condition that
    $\pi_n \left( Y,B,y_0 \right) =0$ for all
    $y_0 \in B$ is to be regarded as saying that
    $\left( Y,B \right) $ is $0$-connected.
\end{lemma}

\begin{proof}[Proof of lemma]
    Assume inductively that $f$ has already been
    homotoped to take the skeleton
    $X^{k-1}$ to $B$. Let
    $\Phi$ be the caracteristic (attaching) map of 
    cell $e^{k}$ of $X - A$. Then the composition
    $f \Phi \colon \left( D^{k} , \partial D^{k} \right) 
    \to \left( Y,B \right) $ is in some class
    in $\pi_k \left( Y, B, y_0 \right) = 0$, so
    it can be homotoped into $B \rel \partial D^{k}$ by
    the compression criterion when
    $k > 0$, or
    by $\left( Y,B \right) $ being $0$-connected for
    $k = 0$ (this is condition (3) in Problem \ref{n-connected-relative}).
    This homotopy of $f \Phi$ induces a homotopy
    $\rel X^{k-1}$ on the quotient space
    $X^{k-1} \cup e^{k}$ of $X^{k-1} \sqcup D^{k}$.
    Doing this for all $k$-cells of $X-A$ simultaneously, and
    taking the constant homotopy on $A$, we obtain a
    homotopy of $f|_{X^{k} \cup A}$ to a map into
    $B$. Since the inclusion of a
    subcomplex into a CW-complex is a cofibration,
    $f|_{X^{k} \cup  A}$ extends to all of $X$ (essentially
    the homotopy extension property).
    This completes the inductive step in the finite dimensional
    CW-complex case.
    In the general case, we perform the
    homotopy of the inductive step during the
    $t$-interval $\left[ 1- \frac{1}{2^{k}},
    1- \frac{1}{2^{k+1}}\right] $. Any finite skeleton
    $X^{k}$ is eventually stationary under these
    homotopies, hence we have a well-defined
    homotopy $f_t, t \in \left[ 0,1 \right] $ with
    $f_1 (X) \subset B$.
\end{proof}


\begin{proof}[Proof of Whitehead's Theorem, \ref{Thm:Whitehead}]
    Let's tackle the case when $f$ is the inclusion
    of a subcomplex first. Consider then
    the LES of the pair $\left( Y,X \right) $. Since
    $f$ by assumption induces isomorphisms
    on all homotopy groups,
    $f_* \colon \pi_* (X) \to \pi_* (Y)$, the
    relative homotopy groups
    $\pi_* (Y,X)$ are zero. Applying the lemma now
    to the identity map $\left( Y,X \right) \to 
    \left( Y,X \right) $, we obtain a homotopy
    of the identity  $\id \colon Y \to Y$ to
    a map $Y \to X$ which is relative to
    $X$. That is, we obtain a deformation retract of
    $Y$ onto $X$.\\
    \linebreak
    For the general case, recall that
    a map $f \colon X \to Y$, can be considered
    as the composition of the
    inclusion $X \hookrightarrow M_f$ and the
    retraction $M_f \to Y$. Since
    the retraction is a homotopy equivalence,
    it suffices to show that $M_f$ deformation retracts
    onto $X$ if $f$ induces isomorphisms on homotopy
    groups, or equivalently, if the relative groups
    $\pi_n \left( M_f, X \right) $ are all zero (since
    $M_f \simeq Y$ ).
    If $f$ is cellular - i.e., takes the $n$-skeleton of
    $X$ to the $n$-skeleton of $Y$ for all
    $n$ - then $\left( M_f, X \right) $ is a CW pair and
    we can apply the first paragraph of the proof.\\
    If $f$ is not cellular, we can either apply
    Theorem 4.8 in \cite{Hatcher} which says
    that $f$ is homotopic to a cellular map, or we can use
    the following argument.

    First, using that 
    $\pi_n \left( M_f, X \right) = 0$ for all $n$, 
    apply the Compression Lemma to
    the inclusion $ \left( X \cup  Y, X \right) 
    \hookrightarrow \left( M_f, X \right) $ to
    obtain a homotopy of the
    inclusion to a map into $X \rel X$.
    The inclusion $X \cup Y \hookrightarrow M_f$ can be
    seen to be a cofibration using 
    Theorem \ref{Thm:SJJDHW29WW}, so
    the pair $\left( M_f, X \cup Y \right) $ satisfies the
    homotopy extension property. So the
    homotopy in question extends to a homotopy
    from the identity of $M_f$ to a
    map $g \colon M_f \to M_f$ taking 
    $X \cup Y$ into $X \rel X$. 
    However, we first of all do not know that this
    homotopy is $\rel X$ nor that 
    $g$ maps all of $M_f$ into $X$.\\
    So we apply the
    Compression lemma again to the
    composition
     \[
         \left( X \times I \sqcup Y,
         X \times \partial I \sqcup Y\right) 
         \to \left( M_f, X \cup Y \right) 
         \stackrel{g}{\to} \left( M_f,X \right) ,
    \]
    to get a homotopy
    $\rel X \times \partial I \sqcup Y$ of
    $g$ to a map 
    $X \times I \sqcup Y \to X$. In particular,
    this homotopy passes through the quotient
    $X \times I \sqcup Y \to M_f$, so we
    get a homotopy of $g \rel X \times \partial I \cup Y$
    to a map $M_f \to X$.\\
    Composing the homotopy
    from the identity of $M_f$ to $g$ with this homotopy,
    we get a deformation retraction of
    $M_f$ onto $X$.
\end{proof}

\begin{note}
    Whitehead's theorem requires a map
    $f \colon X \to Y$ which induces isomorphisms
    on homotopy groups. Thus it does not apply simply
    to any two CW complexes $X$ and $Y$ with isomorphic
    homotopy groups since there might not exist such a map.
    For examples where this is the case, see
    \cite[p. 348]{Hatcher}.
\end{note}

\begin{corollary}
    If $X$ is a CW complex with
    $\pi_n(X) = 0$ for all $n\ge 0$, then
    $X \simeq \left\{ 0 \right\} $.
\end{corollary}

\begin{proof}
    The inclusion of a $0$-cell into the complex
    induces an isomorphism on homotopy
    groups, so by Whitehead's theorem, the complex deformation
    retracts to the $0$-cell.
\end{proof}

\begin{lemma}[Extension Lemma]\label{Extension-Lemma}
    Given a CW pair $\left( X,A \right) $ and a map
    $f \colon A \to Y$ with $Y$-path connected,
    then $f$ can be extended to a map
    $X \to Y$ if $\pi_{n-1}(Y) = 0$ for all $n$ such that
    $X -A$ has cells of dimension $n$.
\end{lemma}

\begin{proof}
    Suppose that $f$ has been extended over the
    $\left( n-1 \right) $-skeleton. Then an extension
    over an $n$-cell exists if and only if
    the composition of the cell's attaching
    map $S^{n-1} \to X^{n-1}$ with $f
    \colon X^{n-1} \to Y$ is nullhomotopic, which
    it is if $\pi_{n-1} (Y) = 0$.
\end{proof}

\subsection{Cellular Approximation}

\begin{definition}[Cellular maps]
    A map $f \colon X \to Y$ between CW complexes,
    satisfying
    $f(X^{n}) \subset Y^{n}$ for all $n$, is called
    a \textit{cellular map}.
\end{definition}

\begin{theorem}[Cellular Approximation Theorem]
    Every map $f \colon X \to Y$ of CW complexes is homotopic
    to a cellular map. If $f$ is already cellular
    on a subcomplex $A \subset X$, then homotopy
    map be taken to be stationary on $A$.
\end{theorem}

\begin{remark}[]
    Cellular approximation tells us
    that $\pi_n(X)$ only depends on
    the $(n+1)$-skeleton.
\end{remark}

Recall the following about simplicial maps and simplicial
approximations:

\begin{definition}[Simplicial map]
    Let $K$ and $L$ be simplicial complexes. A function
    $s \colon \left| K \right| \to \left| L \right| $ 
    is called \textit{simplicial} if it takes simplexes
    of $K$ linearly onto simplexes of $L$.
\end{definition}

\begin{definition}[Carrier of $f(x)$]
    Given a map  $f \colon \left| K \right| 
    \to \left| L \right| $ between polyhedra and a
    point $x \in \left| K \right| $, the point
    $f(x)$ lies in the interior of a unique simplex of
    $L$. Call this simplex the \textit{carrier} of 
    $f(x)$.
\end{definition}

\begin{definition}[Simplicial Approximation]
    A simplicial map $s \colon \left| K \right| 
    \to \left| L \right| $ is a simplicial approximation of 
    $f \colon \left| K \right|  \to \left| L \right| $ 
    if $s(x)$ lies in the carrier of $f(x)$ for
    each $x \in \left| K \right| $.
\end{definition}

\begin{theorem}[Simplicial approximation theorem]
    Let $f \colon \left| K \right| \to 
    \left| L \right| $ be a map between polyhedra.
    If $m$ is chosen large enough, there is a simplicial
    approximation $s \colon \left| K^{m} \right| \to 
    \left| L \right| $ to $f \colon \left| K^{m} \right| 
    \to \left| L \right| $.
\end{theorem}

Thus we may view cellular approximation as
a CW analog of simplicial approximation since simplicial
maps are cellular. Simplicial maps are much more rigid
than cellular maps, however, and the core
proof of cellular approximation will be
a weaker form of simplciial approximation.\\
\linebreak
But first, a nice corollary:

\begin{corollary}
    $\pi_n \left( S^{k} \right) $ for
    $n<k$.
\end{corollary}

\begin{proof}
    If $S^{n}$ and $S^{k}$ are given their usual
    CW structure of a single $0$-cell and
    then an $n$- or $k$-cell, respectively, then by
    the Cellular Approximation Theorem, 
    any pointed map $S^{n} \to S^{k}$ is based homotopic to a 
    cellular map, and hence maps
    the  $n$-skeleton of $S^{n}$ into the $n$-skeleton
    of $S^{k}$. But the $n$-skeleton of $S^{k}$ is
    just the $0$-cell. That is, 
    any map $S^{n} \to S^{k}$ is based nullhomotopic, so
    $\pi_n \left( S^{k} \right) = 0$.
\end{proof}

\begin{proof}[Proof of Cellular Approximation Theorem]
    Long. To do
\end{proof}

\begin{example}[Cellular Approximation for Pairs]
    Every map $f \colon \left( X,A \right) 
    \to \left( Y,B \right) $ of $CW$ pairs can be deformed
    through maps $\left( X,A \right) \to 
    \left( Y,B \right) $ to a cellular map.
    This follows from the theorem by first deforming the
    restriction $f \colon A\to B$ to be cellular,
    then extending this to a homotopy of $f$ on all
    of $X$, then deforming the resulting map
    to be cellular staying fixed on $A$. As a further
    refinement, the homotopy of $f$ can be
    taken to be stationary on any subcomplex of
    $X$ where $f$ is already cellular.
\end{example}

\begin{corollary}[Geometric Version of
    $n$-connectedness]\label{n-connectedness-geometrically}
    A CW pair $\left( X,A \right) $ is $n$-connected
    if all the cells in $X - A$ have dimension
    greater than $n$. In particular, the
    pair  $\left( X, X^{n} \right) $ is
    $n$-connected, hence the
    inclusion $X^{n} \hookrightarrow X$ induces
    isomorphisms on $\pi_i$ for $i < n$ and
    a surjection on $\pi_n$.
\end{corollary}

\begin{proof}
    Recall that $\left( X,A \right) $ is $n$-connected
    if every map
    $\left( D^{i}, \partial D^{i} \right) 
    \to \left( X,A \right) $ is homotopic through
    such maps to a map $D^{i} \to A$.
    Now let
    $f \left( D^{i}, \partial D^{i} \right) 
    \to \left( X,A \right) $ be any map.
    Then by the Cellular Approximation theorem for
    Pairs, $f$ is homotopic through maps
    $\left( D^{i} ,\partial D^{i} \right) \to 
    \left( X,A \right) $ to a cellular map, 
    $\tilde{f} \colon \left( D^{i} , \partial D^{i} \right) 
    \to \left( X,A \right) $. But by assumption, all
    cells in $X-A$ have dimension greater than
    $n \ge i$. Hence $\tilde{f}$ maps
    $D^{i}$ into $A$.
    The last part of the statement now follows from the LES
    \[
    \ldots \to \pi_n (X^{n}) \stackrel{\iota_*}{\to} \pi_n(X) \to 
    \underbrace{\pi_n\left( X,X^{n} \right)}_{0} \to 
    \pi_{n-1}(X^{n}) \stackrel{\iota_*}{\to}  \pi_{n-1}(X) \to 
    \underbrace{\pi_{n-1}\left( X, X^{n} \right)}_{0} \to \ldots
    \] 
\end{proof}

\subsection{CW Approximation}

\begin{definition}[Weak Homotopy Equivalence]
    A map $f \colon X \to Y$ is called a
    \textit{weak homotopy equivalence} if it induces
    isomorphisms $\pi_n \left( X, x_0 \right) 
    \to \pi_n \left( Y, f(x_0) \right) $ for all
    $n \ge 0$ and all choices
    of basepoint $x_0$.
\end{definition}

\begin{remark}[Reformulation of Whitehead's Theorem]
    Whitehead's Theorem thus says that
    a weak homotopy equivalence between
     CW complexes is, in fact, a homotopy equivalence.
\end{remark}

\begin{definition}[CW Approximation]
    For a space $X$, a weak homotopy equivalence
    $f \colon Z \to X$, where $Z$ is a CW complex, is
    called a \textit{CW approximation} to $X$.
\end{definition}

\begin{remark}[]
    CW approximations to a given space $X$ are
    unique up to homotopy equivalence since
    if $f \colon Z \to X$ and 
    $f' \colon Z' \to X$ are CW approximations, then
    consider the composition
    $Z \to X \hookrightarrow M_{f'}$.
    Since $f' \colon Z' \to X$ is assumed to be a weak
    homotopy equivalence, we find by the relative LES that
    $\pi_n (M_f, Z') \cong \pi_n \left( X, Z' \right) = 0$ for all
    $n\ge 0$, so by 
    the Compression Lemma (with $A$ chosen to
    be the basepoint of $Z$), we
    find that the map $Z \to X \hookrightarrow M_{f'}$
    is homotopic to a map
    $Z \to Z' \subset M_{f'}$ relative
    to the basepoint.
    But taking $\pi_n$ of
    $Z \to X \to M_{f'} \to Z'$, we get
    $\pi_n(Z) \stackrel{\cong}{\to} 
    \pi_n(X) \stackrel{\cong}{\to} 
    \pi_n (M_{f'}) \stackrel{\cong}{\to}
    \pi_n \left( Z' \right)$
    where $\pi_n (X) \stackrel{\cong}{\to} \pi_n (M_{f'})$ 
    follows from $\iota \colon X \simeq M_{f'}$ being
    a homotopy equivalence;
    $\pi_n\left( M_{f'} \right) 
    \stackrel{\cong}{\to} \pi_n (Z')$ follows
    from the homotopy that we got from the compression lemma,
    and the first isomorphism
    $f_* \colon \pi_n (Z) \stackrel{\cong}{\to}  \pi_n (X)$ 
    follows from $f$ being a weak homotopy equivalence.
    Applying Whitehead's theorem, we find that this
    composition is a homotopy equivalence
    $Z \simeq Z'$.
\end{remark}

\begin{proposition}[]\label{Prop:CW-Approximation}
    Every space $X$ has a CW approximation
    $f \colon Z \to X$. If $X$ is path-connected,
    $Z$ can be chosen to have a single $0$-cell, with
    all other cells attached by basepoint-preserving maps.
    Thus every connected CW complex is homotopy equivalent to
    a CW complex with these additional properties.
\end{proposition}

\begin{proof}
    The construction of a CW approximation
    $f \colon Z \to X$ is inductive, so we first
    describe the induction step. 
    Suppose we are given a CW complex $A$ with a map
    $f \colon A \to X$ and suppose
    we have chosen a basepoint $0$-cell $a_{\gamma}$ in
    each component of $A$.
    Then for an integer $k\ge 0$, we will attach
    $k$-cells to $A$ to form a CW complex
    $B$ with a map $f \colon B \to X$ extending $f$ such that
    \begin{itemize}
        \item For each basepoint $a_{\gamma}$, the
            induced map $f_* \colon
            \pi_i \left( B, a_{\gamma} \right) 
            \to \pi_i \left( X, f\left( a_{\gamma} \right) 
            \right) $ is injective for
            $i = k-1$ (when $k>0$ ) and surjective
            for $i = k$.
    \end{itemize}

    We do this in two steps (the first step is omitted
    when $k= 0$ ):

    \begin{enumerate}
        \item We have been given
            a CW complex $A$ and a map
            $f \colon A \to X$ alongside basepoints
            $a_{\gamma}$. Now for each
            nontrivial element $\alpha$ of the kernel
            $\ker f_*$ ranging over
            all basepoints, choose
            a map $\varphi_{\alpha}\colon
            \left( S^{k-1}, s_0 \right) \to 
            \left( A, a_{\gamma} \right) $ representing
            $\alpha$. We may assume that the
            $\varphi_{\alpha}$ are all cellular (by
            the Cellular Approximation Theorem) where
            $S^{k-1}$ is given its standard CW structure
            with $s_0$ as a 0-cell. 
            Attaching cells $e_{\alpha}^{k}$ to $A$ via
            the maps $\varphi_{\alpha}$ then produces
            a CW complex. Now, $f \circ \varphi_{\alpha}$ 
            is nullhomotopic, so
            $f$ extends over the cell
            $e_{\alpha}^{k}$.
        \item Choose maps
            $f_{\beta}\colon S^{k}\to X$ representing
            all nontrivial elements of
            $\pi_k \left( X, f(a_{\gamma}) \right) $ for
            all the $a_{\gamma}$ 's Then attach
            cells $e_{\beta}^{k}$ to $A$ via the
            constant maps at the appropriate basepoints
            $a_{\gamma}$ and extend $f$ over the resulting spheres
            $S_{\beta}^{k}$ via $f_{\beta}$.
    \end{enumerate}
    By the construction, then
    surjectivity of
    $f_* \colon \pi_i \left( B, a_{\gamma} \right) 
    \to \pi_i \left( X, f(a_{\gamma}) \right) $ for
    $i = k$ follows. Now
    let $\alpha$ be in the kernel of
    $f_* \colon \pi_{k-1} (B, a_{\gamma}) \to 
    \pi_{k-1}\left( X, f(a_{\gamma}) \right) $, and
    let $h \colon S^{k-1} \to B$ be a cellular
    map that represent $\alpha$. Since
    $h$ is cellular, its image is contained in the
     $\left( n-1 \right) $-skeleton
     of $B$ which is a subskeleton (could be all) of $A$.
     Since $h$ has image in $A$, it is in the kernel
     of $f_* \colon \pi_{k-1}(A, a_{\gamma}) \to 
     \pi_{k-1} \left( X, f(a_{\gamma}) \right) $ and thus
     it is homotopic to some $\varphi_{\alpha}$ and therefore
     nullhomotopic in $B$.\\
     \begin{note}
         In step (1), it suffices to attach cells for
         just the generators of the kernels
         when $k>1$, and just for the generators
         of $\pi_k \left( X, f(a_{\gamma}) \right) $ in
         step (2) when $k>0$.
     \end{note}

     \begin{note}
         If the given map $f \colon A \to X$ happened
         to already be injective or surjective
         on $\pi_i$ for some $i < k-1$ or
         $i < k$, respectively, then this remains
         true after attaching the $k$-cells.
         This is because attaching $k$-cells does
         not affect $\pi_i$ if $i < k-1$, by
         cellular approximation, not does
         it affect surjectivity on
         any $\pi_i$, simply because the same maps
         still hold and work.
     \end{note}

     Now to construct a CW approximation $f \colon Z \to 
     X$, one can start with $A$ consisting of one point for
     each path-component of $X$, with
     $f \colon A \to X$ mapping each of these points
     to the corresponding path-component.
     This gives a bijection on $\pi_0$ by construction, hence
     it provides us with the inductive base case.
     Now we can attach $1$-cells to $A$ to create
     a surjection on $\pi_1$ for each path-component, then
     $2$-cells to improve this to an isomorphism on
     $\pi_1$ and a surjection on $\pi_2$ and so forth
     for each successive $\pi_i$ in turn. After all cells
     have been attached, on has a CW complex $Z$ with
     a weak homotopy equivalence $f \colon Z \to X$.
\end{proof}


\begin{example}[]
    One can apply this technique to produce a CW approximation
    to a pair $(X,X_0)$ also. First one constructs
    a CW approximation $f_0 \colon Z_0 \to X_0$, then
    one starts with the composition
    $Z_0 \to X_0 \hookrightarrow X$ and attaches
    cells to $Z_0$ to create a weak homotopy equivalence
    $f \colon Z \to X$ extending $f_0$.
    Then we get

    \begin{equation*}
    \begin{tikzcd}
        \pi_n(Z_0) \ar[r] \ar[d, "\cong"] & \pi_n (Z) \ar[r] \ar[d,
        "\cong"] & 
        \pi_n (Z, Z_0) \ar[r] \ar[d] &
        \pi_{n-1} (Z_0) \ar[r] 
        \ar[d, "\cong"]&
        \pi_{n-1} (Z) \ar[d, "\cong"] \\
        \pi_n (X_0) \ar[r] & \pi_n(X) \ar[r] & \pi_n (X,X_0) \ar[r] &
        \pi_{n-1} (X_0) \ar[r] & \pi_{n-1}(X)
    \end{tikzcd}
    \end{equation*}
    By the five-lemma, it follows that
    $\pi_n \left( Z,Z_0 \right) \to 
    \pi_n \left( X,X_0 \right) $ is an isomorphism
    for each $n$.
\end{example}

\begin{proposition}[]\label{Prop:Removing-lower-dim-cells}
    If $\left( X,A \right) $ is an $n$-connected
    CW pair, then there exists a CW pair
    $\left( Z,A \right) \simeq \left( X,A \right) \rel A$ 
    such that all cells of $Z - A$ have dimension greater
    than $n$.
\end{proposition}

\begin{proof}
    Starting with the inclusion
    $A \hookrightarrow X$, attach cells of
    dimension $n+1$ and higher to
    $A$ to produce a CW complex $Z$ and
    a map $f \colon Z \to X$ using the
    procedure of Proposition \ref{Prop:CW-Approximation}.
    In particular then
    by the Proposition proof,
    $f_*$ induces an injection of
    $\pi_n$ and isomorphisms on all higher homotopy groups.
    Now, the induced map on $\pi_n$ is also
    surjective since it is true for
    $A \hookrightarrow Z \stackrel{f}{\to} X$ 
    as $\left( X,A \right) $ is $n$-connected and hence
    $\pi_n (A) \stackrel{\cong}{\to}  \pi_n(X)$ is an isomorphism.
    Since $f$ is equal to this inclusion on the
    $n$-skeleton, this gives that $f_*$ is also surjective.
    By cellular approximation
    $A \hookrightarrow Z$ induces an isomorphism
    on homotopy groups in dimensions below $n$, and
    likewise $n$-connectedness does the same for
    $A \hookrightarrow X$. But then since
    \begin{equation*}
    \begin{tikzcd}
        \pi_n (A) \ar[rr, bend right, "\iota_*",
        "\cong"'] 
        \ar[r, "\iota_*", "\cong"'] & 
        \pi_n (Z) \ar[r,"f_*"] & 
        \pi_n (X)
    \end{tikzcd}
    \end{equation*}
    commutes, we find that $f_*$ is also an
    isomorphism on all $n\ge 0$.
    Thus $f$ is a weak homotopy equivalence, and hence
    a homotopy equivalence by Whitehead's theorem.\\

    To see that $f$ is a homotopy equivalence
    $\rel A$, we could apply Proposition 
    \ref{Prop:HEP-Homotopy-Equivalence}, but
    here is an alternative argument. Let
    $W$ be the quotient space of the mapping cylinder
    $M_f$ obtained by collapsing each segment
    $\left\{ a \right\} \times I$ to a point, for
    $a \in A$. Assuming $f$ has been made cellular,
    $W$ is a CW complex (why?) containing $X$ and $Z$ as
    subcomplexes, and $W$ deformation retracts
    to $X$ just as $M_f$ does. Also,
    $\pi_i \left( W,Z \right) = 0$ for all
    $i$ since $f$ induces isomorphisms on all
    homotopy groups (by the LES), so $W$ deformation retracts
    onto $Z$ by Whitehead's Theorem (Theorem \ref{Thm:Whitehead}).
    The deformation retract of $W$ onto $X$ and the
    deformation retract of $W$ onto $Z$ are stationary
    on $A$, hence give a homotopy equivalence
    $X \simeq Z \rel A$.
\end{proof}

\begin{example}[Postnikov Towers]\label{Postnikov-Towers}
    For each connected CW complex $X$ and each
    integer $n\ge 1$, we can construct a 
    CW complex $X_n$ containing $X$ as a subcomplex such that
    \begin{enumerate}
        \item $\pi_i \left( X_n \right) = 0$ for $i>n$.
        \item The inclusion $X \hookrightarrow X_n$ induces
            an isomorphism on $\pi_i$ for $i\le n$.
    \end{enumerate}
    
    \begin{idea}
        Take $X$ and fill out any spheres of dimension
        $>n$ by filling them in.
    \end{idea}
    Indeed, we attach $(n+2)$-cells to
    $X$ using cellular maps $S^{n+1} \to X$ that
    generate $\pi_{n+1}(X)$ to form a 
    space with $\pi_{n+1}$ trivial. Then
    for this space, we attach $(n+3)$-cells to
    make $\pi_{n+2}$ trivial, and so on. The result
    is a CW complex $X_n$ with the desired properties.
    The inclusion $X \hookrightarrow X_n$ extends
    to a map $X_{n+1} \to X$ since
    $X_{n+1}$ is obtained from $X$ by attaching cells
    of dimension $n+3$ and greater, and
    $\pi_i (X_n) = 0$ for $i>n$, so we
    can apply the Extension Lemma (Lemma \ref{Extension-Lemma}).
    Thus we get a commutative diagram as follows:
    \begin{equation*}
    \begin{tikzcd}
        & \vdots \ar[d] \\
        & X_3 \ar[d] \\
        & X_2 \ar[d] \\
        X \ar[r] \ar[ru] \ar[ruu] & X_1
    \end{tikzcd}
    \end{equation*}
    This is called a \textit{Postnikov tower} for $X$.
    One can regard the spaces $X_n$ as truncations of
    $X$ which provides successively better approximations
    to $X$ as $n$ increases.
    


\end{example}




\section{Methods of Calculation}


\subsection{Excision for Homotopy Groups}

\begin{theorem}[]\label{Thm:Homotopy-Excision}
    Let $X$ be a CW complex decomposed as the union of
    subcomplexes $A$ and $B$ with nonempty connected
    intersection $C = A \cap B$. If
    $(A,C)$ is $m$-connected and
    $\left( B,C \right) $ is $n$-connected,
    $m,n \ge 0$, then the map
    $\pi_i (A,C) \to \pi_i (X,B)$ induced
    by inclusion is an isomorphism for
    $i < m+n$ and a surjection for
    $i = m+n$.
\end{theorem}

\begin{corollary}[Freudenthal Suspension Theorem]
    The unreduced suspension map
    $\pi_i (S^{n}) \to 
    \pi_{i+1} \left( S^{n+1} \right) $, induced
    by the suspension map $S^{n} \to 
    \Sigma S^{n} \cong S^{n+1}$, is an
    isomorphism for
    $i < 2n-1$ and a surjection for
    $i = 2n-1$. More generally, this holds for
    the suspension
    $\pi_i (X) \to \pi_{i+1}\left( \Sigma X \right) $ whenever
    $X$ is an $(n-1)$-connected CW complex.
\end{corollary}

\begin{proof}[Proof of Corollary]
    Decompose the unreduced suspension
    $\Sigma X = 
    (X \times I) / \left( 
    X \times \left\{ 0 \right\} ,
X \times \left\{ 1 \right\} \right) $ as the
    union of two cones
    $C_+ X$ and $C_{-}X$ intersecting in a 
    copy of $X$.
    Recall that a map
    $f\colon X \to Y$ induces a suspended map
    $\Sigma f \colon \Sigma X \to \Sigma Y$. Now,
    if we consider 
    $f$ to be any map
    $f \colon \left( S^{n},s_0 \right) \to 
    \left( X, x_0 \right) $, then
    we have
    a suspended map
    \begin{equation*}
    \begin{tikzcd}
        S^{n} \times I \ar[r, "f \times \id"]
        \ar[d] & X \times I \ar[d] \\
        S^{n+1} \cong \Sigma S^{n} \ar[r, "\Sigma f"] & \Sigma X
    \end{tikzcd}
    \end{equation*}
    So, in particular,
    $\Sigma f$ is some class
    in $\pi_{n+1} \left( \Sigma X \right) $.
    Define the suspension homomorphism
    $\pi_i (X) \to \pi_{i+1}\left( \Sigma X \right) $ 
    to be the map that sends
    $f$ to $\Sigma f$. This is a homomorphism (why?).

    The unreduced suspension map is the same
    as the map
    \[
    \pi_i (X) \cong
    \pi_{i+1} \left( C_+ X, X \right) 
    \to \pi_{i+1} (\Sigma X, C_{-}X) \cong
    \pi_{i+1} \left( \Sigma X \right) .
    \] 
    (why?) where the two isomorphisms come
    from the LES of pairs and the middle
    map is induced by inclusion.
    The first map
    $\pi_i (X) \to \pi_{i+1}(C_+ X, X)$ 
    takes a map
    $\left( I^{i}, \partial I^{n} \right) 
    \to \left( X, x_0 \right) $ 
    to the map
    $\left( I^{n+1}, \partial I^{n+1},
    J^{n}\right) \to \left( C_+ X, X, x_0 \right) $
    constructed by extending
    the given map radially to correspond with the
    height of $C_{+}X$. So one face of
    $I^{n+1}$ will be mapped to the vertex of
    $C_{+}X$.\\

    Including this into
    $\left( \Sigma X, C_{-}X \right) $ gives
    the middle homomorphism, and
    then the map
    $\pi_{i+1} \left( \Sigma X, C_{-}X \right) \to 
    \pi_{i+1}(\Sigma X)$ is simply the identity on
    our map. \\
    From the LES of $\left( C_{\pm }X, X \right) $, we
    see that this pair is $n$-connected
    if $X$ is $(n-1)$-connected. Then
    Theorem \ref{Thm:Homotopy-Excision} gives
    that the middle map is an isomorphism
    for $i+1 < 2n$ and surjective for
    $i+1 = 2n$.

\end{proof}





\begin{example}[$\pi_n \left( \bigvee_{\alpha}\label{Example:Wedge-of-Spheres}
    S_{\alpha}^{n} \right) $]
    We want to show that
    $\pi_n \left( \bigvee_{\alpha}S_{\alpha}^{n} \right) $ 
    for $n\ge 2$ is free abelian with basis
    the homotopy classes of the inclusions
    $S_{\alpha}^{n} \hookrightarrow 
    \bigvee_{\alpha} S_{\alpha}^{n}$.\\
    Suppose first that there are only \textit{finitely many}
    summands $S_{\alpha}^{n}$. Then we can regard
    $\bigvee_{\alpha} S_{\alpha}^{n}$ as
    the $n$-skeleton of the product
    $\prod_{\alpha} S_{\alpha}^{n}$, where
    $S_{\alpha}^{n}$ is given the usual CW structure
    and $\prod_{\alpha} S_{\alpha}^{n}$ has the product
    CW structure. (See Hatcher appendix A).\\
    By construction then $\prod_{\alpha} S_{\alpha}^{n}$ 
    has cells only in dimensions a multiple of $n$, so
    the pair $\left( \prod_{\alpha} S_{\alpha}^{n},
    \bigvee_{\alpha} S_{\alpha}^{n} \right) $ is
    $(2n-1)$-connected by Corollary 
    \ref{n-connectedness-geometrically}.
    So from the LES for the pair, we see that
    the inclusion
    $\bigvee_{\alpha} S_{\alpha}^{n}
    \hookrightarrow \prod_{\alpha} S_{\alpha}^{n}$
    induces an isomorphism on homotopy groups
    in dimensions $\le 2n-1$. Next we have
    $\pi_n \left( \prod_{\alpha} S_{\alpha}^{n} \right) 
    \cong \bigoplus_{\alpha} \pi_n \left( S_{\alpha}^{n} \right) 
    \cong \bigoplus_{\alpha} \mathbb{Z}$,
    a free abelian group with basis the inclusions
    $S_{\alpha}^{n} \hookrightarrow \prod_{\alpha}
    S_{\alpha}^{n}$, so pulling this back along the
    isomorphism
    $\pi_{n}\left( \bigvee_{\alpha}S_{\alpha}^{n} \right) 
    \cong \pi_n \left( \prod_{\alpha}S_{\alpha}^{n} \right) $,
    the same is true for $\bigvee_{\alpha}S_{\alpha}^{n}$.
    This proves the claim when there are
    finitely many $S_{\alpha}^{n}$'s.\\
    When there are infinitely many summands
    $S_{\alpha}^{n}$, consider the homomorphism
    $\Phi \colon \bigoplus_{\alpha}
    \pi_n \left( S_{\alpha}^{n} \right) \to 
    \pi_n \left( \bigvee_{\alpha}S_{\alpha}^{n} \right) $ 
    induced by the inclusions
    $S_{\alpha}^{n} \hookrightarrow \bigvee_{\alpha}
    S_{\alpha}^{n}$. Then $\Phi$ is surjective
    since any map $f \colon S^{n} \to 
    \bigvee_{\alpha}S_{\alpha}^{n}$ has compact
    image contained in the wedge sum of finitely many
    $S_{\alpha}^{n}$'s, so by the finite case already proved,
    $\left[ f \right] $ is in the image of $\Phi$.\\
    Similarly, a nullhomotopy of $f$ has compact image
    contained in a finite wedge sum of $S_{\alpha}^{n}$'s,
    so the finite case also implies that $\Phi$ is injective.
\end{example}

\begin{proposition}[]\label{Prop:JXKDAOWJ}
    If a CW pair $\left( X,A \right) $ is $r$-connected
    and $A$ is $s$-connected, with $r,s \ge 0$, then
    the map $\pi_i (X,A) \to \pi_i (X /A)$ induced
    by the quotient map
    $X \to X / A$ is an isomorphism for
    $i \le r +s$ and a surjection for
    $i = r+s+1$.
\end{proposition}

\begin{proof}
    Consider $X \cup  CA$. Since
    $A$ is closed and the inclusion
    $A \hookrightarrow X$ is a cofibration (since these
    are CW complexes), the map
    $h \colon C_{\iota} = X \cup CA \to 
    X / A$ is a homotopy equivalence by Theorem 
    \ref{Thm:2030akKAK}. So we have a commutative
    diagram
    \begin{equation*}
    \begin{tikzcd}
        \pi_i (X,A) \ar[r] & \pi_i \left( X \cup CA, CA \right) 
        \ar[r] & \pi_i \left( X \cup  CA /CA \right) 
        = \pi_i (X / A)\\
               & \pi_i (X \cup  CA) \ar[u, "\cong"] 
               \ar[ru, "\cong"]
    \end{tikzcd}
    \end{equation*}
    where the vertical isomorphism comes from the
    LES of the pair $\left( X \cup CA, CA \right) $.
    Now, applying Theorem \ref{Thm:Homotopy-Excision}
    to $\left( A,B \right) = \left( X, CA \right) $,
    since $\left( X, A \right) $ is
    $r$-connected and
    $\left( CA, A \right) $ is
    $(s+1)$-connected, we find that the
    homomorphism 
    $\pi_i \left( X,A \right) \to 
    \pi_i \left( X \cup CA, CA \right) $ induced
    by the inclusion
    is an isomorphism for
    $i < r+s+1$ and a surjection for
    $i = r+s+1$, which proves the result.
\end{proof}


\begin{example}[Construction of spaces
    with a particular group as $\pi_n$]\label{Example:DJISI02}
    Suppose $X$ is obtained from a wedge of
    spheres $\bigvee_{\alpha} S_{\alpha}^{n}$ by
    attaching cells $e_{\beta}^{n+1}$ via
    basepoint-preserving maps
    $\varphi_{\beta} \colon S^{n} \to 
    \bigvee_{\alpha} S_{\alpha}^{n}, n\ge 2$.
    By cellular approximation, we know that
    $\pi_i (X) = 0$ for $i < n$, and we shall show
    that $\pi_n(X)$ is the quotient of the free
    abelian group
    $\pi_n \left( \bigvee_{\alpha} S_{\alpha}^{n} \right) 
    \cong \bigoplus_{\alpha} \mathbb{Z}$ by
    the subgroup generated by the
    classes $\left[ \varphi_{\alpha} \right] $.
    Any subgroup can be realized in this way, by
    choosing maps $\varphi_{\beta}$ to represent a 
    set of generators for the subgroup.
    Let 
    $X = \left( \bigvee_{\alpha}S_{\alpha}^{n} \right) 
    \bigcup_{\beta } e_{\beta}^{n+1}$.\\
    Then the LES of the pair
    $\left( X, \bigvee_{\alpha} S_{\alpha}^{n} \right) $ gives
    \[
    \pi_{n+1}\left( X, \bigvee_{\alpha}S_{\alpha}^{n} \right) 
    \stackrel{\partial}{\to} \pi_n \left( \bigvee_{\alpha}
    S_{\alpha}^{n} \right) \to 
    \pi_n (X) \to 0.
    \] 
    so 
    \[
    \pi_n (X) \cong \pi_n \left( \bigvee_{\alpha}S_{\alpha}^{n}
    \right) / \im \partial
    \] 
    The quotient
    $X / \bigvee_{\alpha} S_{\alpha}^{n}$ is a wedge
    of spheres $S_{\beta}^{n+1}$, so by Proposition
    \ref{Prop:JXKDAOWJ} and Example \ref{Example:Wedge-of-Spheres},
    the map
     $\pi_{n+1} \left( X, \bigvee_{\alpha}S_{\alpha}^{n} \right) 
     \to \pi_{n+1} \left( X / \bigvee_{\alpha}S_{\alpha}^{n} \right) 
     \cong \pi_{n+1} \left( \bigvee_{\beta}
     S_{\beta}^{n+1} \right) $ 
     is an isomorphism, so
     $\pi_{n+1}\left( X, \bigvee_{\alpha}S_{\alpha}^{n} \right) $ 
     is free with basis the caracteristic
     maps $\varphi_{\beta}$ of the cells $e_{\beta}^{n+1}$.
     The boundary map $\partial$ takes
     these to the classes
     $\left[ \varphi_{\beta} \right] $, so
     the result follows.

\end{example}


\newpage

\subsubsection{Eilenberg-MacLane Spaces}

\begin{definition}[Eilenberg-MacLane space, $K (G,n)$]
    A space $X$ having just one nontrivial homotopy
    group $\pi_n (X) \cong G$ is called
    an \textit{Eilenberg-MacLane space} $K(G,n)$. 
\end{definition}


\textit{Construction of Eilenberg-MacLane Spaces:}\\
Given arbitrary $G$ and $n$, and assuming $G$ is abelian
if $n>1$, we can construct a CW complex
$K(G,n)$. To begin, construct the
CW complex $X$ from Example \ref{Example:DJISI02}. Then
$X$ is an $(n-1) $- conneted CW complex of dimension
$n+1$ such that $\pi_n (X) \cong G$ by construction. 
Alternatively, given the existence of Moore spaces
$M(G,n)$ for any $G$ and $n$, we can take a Moore space
$M(G,n)$ and use the Hurewicz isomorphism
to conclude that $\pi_n (X) \cong H_n(X)$.
Hence
we just need to fix all homotopy groups of dimension
greater than $n$. By
Example \ref{Postnikov-Towers}, we can
construct a CW complex $X_n$ containing $X$ as a subcomplex
such that
$\pi_n (X_n) \cong \pi_n (X) \cong G$ while
$\pi_k(X_n) \cong 0$ for all $k \neq  n$.


\begin{example}[Constructing spaces with arbitrary
    (abelian) homotopy groups]
    Recall that
    \[
    \pi_n \left( \prod_{\alpha}X_{\alpha} \right) 
    \cong \prod_{\alpha} \pi_n \left( X_{\alpha} \right) ,
    \] 
    so if we have a sequence
    of abelian groups $\left\{ G_{n_i} \right\}_{i \in I}$, and let
    $X_{n_i}$ denote that $K(G_{n_i}, n_{i})$ space, then
    we find that
    \[
    \pi_k( \prod_{i \in I} X_{n_i})
    \cong \prod_{i \in I} \pi_k \left( X_{n_i} \right) 
    \cong \begin{cases}
        G_{n_i},& k = n_{i} \text{ for some }i \in I\\
        0,& \text{else}
    \end{cases}
    \] 
\end{example}

Having covered the existence of Eilenberg-MacLane spaces, we
now find the following for uniqueness of these spaces:

\begin{proposition}[Uniqueness of Eilenberg-MacLane spaces]\label{Prop:23SUIHD}
    The homotopy type of a $CW$ complex
    $K(G,n)$ is uniquely determined by $G$ and $n$.
\end{proposition}

The proof is based on the following
lemma giving a condition for when homomorphisms
between homotopy groups are induced by some
map:

\begin{lemma}[]\label{Lemma:NDWUAI288}
    Let $X$ be a $CW$ complex of the form
    $\left( \bigvee_{\alpha} S_{\alpha}^{n} \right) 
    \bigcup_{\beta} e_{\beta}^{n+1}$ for some $n\ge 1$.
    Then for every homomorphism $\psi \colon
    \pi_n (X) \to \pi_n (Y)$ with $Y$ path-connected
    there exists a map $f \colon X \to Y$ with
    $f_* = \psi $.
\end{lemma}

\begin{proof}
    The construction of $f$ is as one would expect:
    first let $f$ send the natural basepoint of
    $\bigvee_{\alpha} S_{\alpha}^{n}$ to a chosen
    basepoint $y_0 \in Y$.
    Now for every sphere $S_{\alpha}^{n}$ in $X$,
    we extend $f$ over the sphere via a map
    representing $\psi \left( \left[ i_{\alpha} \right]  \right) 
    $ where $i_{\alpha}$ is the inclusion
    $S_{\alpha}^{n} \hookrightarrow X$. This defines
    $f$ on the $n$-skeleton of $X$ : $f \colon X^{n} \to Y$.
    Since now $f_* \left[ i_{\alpha} \right] =
    \psi \left[ i_{\alpha} \right] $ for all $\alpha$ and
    the $\left[ i_{\alpha} \right] $ generate
    $\pi_n \left( X^{n} \right) $, this
    defines $f_*$ on all of $\pi_n(X^{n})$.

    To extend $f$ over the $(n+1)$-cells, it will suffice
    to show that $f \circ \varphi_{\beta}$ is nullhomotopic,
    where $\varphi_{\beta} \colon S^{n} \to X^{n}$ is the
    attaching map for the $(n+1)$-cell $e_{\beta}^{n+1}$.
    But $f \circ \varphi_{\beta}$ is a representative
    of $f_* \left[ \varphi_{\beta} \right] =
    \psi \left[ \varphi_{\beta} \right] $.
    Thus we have transformed $f_* \left[ \varphi_{\beta} \right] $ 
    into an element in the image of
    $\psi \colon \pi_n(X) \to \pi_n(Y)$, and
    for this, we can use the extra structure of $X$, not just
    $X^{n}$. In $X$, $\left[ \varphi_{\beta} \right] $ is
    trivial via the
    characteristic map of the cell
    $e_{\beta}^{n+1}$, so $\psi \left[ \varphi_{\beta} \right] =
    \psi (0) = 0$, thus indeed
    $f \circ \varphi_{\beta}$ is nullhomotopic.
    Thus we obtain the desired extension
    $f \colon X \to Y$.
    To see that $f_* = \psi $, simply note that
    by cellular approximation, any element
    of $\pi_n(X)$ can be represented as
    an element in $\pi_n(X^{n})$, and on
    $\pi_n(X^{n})$, $f_*$ agrees with $\psi $ by construction.
\end{proof}


\begin{proof}[Proof of Proposition \ref{Prop:23SUIHD}]
    Let $K'$ be any $K\left( G,n \right) $ CW complex, and
    let $K$ be the specific $K(G,n)$ CW complex constructed
    in Example \ref{Example:DJISI02}. In particular,
    $K$ is of the form of 
    Lemma \ref{Lemma:NDWUAI288}. Since
    $\pi_n(K) = \pi_n(Y)$, we can apply Lemma
    \ref{Lemma:NDWUAI288} to obtain a map
    $f \colon K \to K'$ inducing the identity
    on $\pi_n$. Since all other homotopy groups
    of $K$ and $K'$ are trivial, Whitehead's theorem now gives
    that $f$ is a homotopy equivalence. Since
    homotopy equivalence is an equivalence relation, this finishes
    the proof.
\end{proof}


\subsection{The Hurewicz Theorem}

\begin{theorem}[]
    If a space $X$ is $(n-1)$-connected, $n\ge 2$,
    then $\tilde{H}_i (X) = 0$ for 
    $i < n$ and $\pi_n (X) \cong
    H_n (X)$. If a pair $(X,A)$ is $(n-1)$-connected,
    $n \ge 2$, with $A$ simply connected
    and nonempty, then $H_i (X,A) = 0$ for
    $i < n$ and $\pi_n (X,A) \cong
    H_n (X,A)$.
\end{theorem}

\begin{remark}[]
    This result is, in a sense, the best that we can
    expect. For example, $S^{n}$ has trivial
    homology groups above dimension $n$ but many
    nontrivial homotopy groups in this range
    when $n\ge 2$; and conversely,
    Eilenberg-MacLane spaces such as
    $\mathbb{C}\mathbb{P}^{\infty}$ have trivial
    higher homotopy groups but many nontrivial homology
    groups.
\end{remark}


\begin{corollary}
    A map $f \colon X \to Y$ between simply-connected
    CW complexes is a homotopy equivalence if
    $f_* \colon H_n (X) \to H_n(Y)$ is an isomorphism
    for each $n$.
\end{corollary}

\begin{proof}
    By replacing $Y$ with the mapping cylinder $M_f$, we
    may assume $f$ is the inclusion $X \hookrightarrow Y$.
    Since $X$ and $Y$ are simply-connected,
    $\pi_1 \left( Y,X \right) = 0$. The relative Hurewicz
    theorem says that the first nonzero $\pi_n(Y,X)$ is isomorphic
    to the first nonzero $H_n(Y,X)$, but by the LES
    of the pair $(Y,X)$ in homology,
    $H_n(Y,X) \cong 0$ for all $n\ge 0$, so
    also $\pi_n(Y,X) \cong 0$ for all $n\ge 0$, so
     $f$ induces isomorphisms
     $\pi_n(X) \to \pi_n(Y)$ for all $n$. By
     Whitehead's theorem, $f$ is a homotopy equivalence.
\end{proof}






%\include{pset2}


%\subsection{Problem Set 3}
    \begin{problem}[]
        Let $Y$ be a simply-connected CW-complex. Assume
        there exists a finite wedge of spheres
        $\bigvee_{i} S^{n_i}$ together with maps
        $i \colon Y \to \bigvee_{i} S^{n_i},
        r\colon \bigvee_{i} S^{n_i} \to Y$ such that
        $r \circ i$ is homotopic to $\id_Y$.
        Prove that $Y$ is homotopy equivalent to some
        finite wedge of spheres 
        $\bigvee_{j} S^{m_j}$.
    \end{problem}


    \begin{proof}
        We want
        to make use of the Corollary 4.33 in Hatcher which says:
        \begin{corollary}[4.33 Hatcher]
            A map $f \colon X \to Y$ between simply-connected
            CW complexes is a homotopy equivalence if 
            $f_* \colon H_n (X) \to H_n(Y)$ is an isomorphism
            for each $n$.
        \end{corollary}

        Since
        $r_* \circ \iota_* = \id_*$, 
        $\iota_* \colon
        H_n (Y) \to H_n\left( \bigvee_i S^{n_i} \right) $ 
        is injective for each $n$.
        Now
        $H_n \left( \bigvee_i S^{n_i} \right) 
        \cong \bigoplus_{A_n} \mathbb{Z}$ where
        we let $A_n$ be an indexing set
        for the $n$-cells $\left\{ e_{\alpha}^{n} \right\} $ 
        in the CW structure of $\bigvee_{i}S^{n_i}$.
        Let $\mathcal{A}_n$ be a set of representative basis
        elements for $H_n \left( \bigvee_i S^{n_i} \right) 
        \cong \bigoplus_{A_n}\mathbb{Z}$ corresponding
        to the inclusion of a sphere into the
        wedge.
        Any
        subgroup of a free group is free, so
        $H_n(Y) \cong
        \bigoplus_{B_n} \mathbb{Z}$ for some
        indexing set $B_n$ for each $n$.
        Now, starting with a single $0$-cell $* $ and
        attaching $\left| B_1 \right| $ $1$-cells to
        $*$, $\left| B_2 \right| $ $2$-cells to $*$, etc.,
        we obtain a space
        $Z = \bigvee_j S^{m_j}$ which satisfies
        $H_n(Z) = H_n(Y)$.
        Now let
        $\mathcal{C}_n$ denote the basis set for
        $H_n (Y) \cong \bigoplus_{B_n} \mathbb{Z}$.
        Since $r_*
        \colon H_n \left( \bigvee_i S^{n_i} \right) 
        \cong \bigoplus_{A_n} \mathbb{Z} \to 
        \bigoplus_{B_n} \mathbb{Z} \cong
        H_n (Y)$ is surjective,
        we can choose representatives
        $\tilde{\mathcal{A}_n} \subset 
        \mathcal{A}_n$ such that
        $r_* \left( \tilde{\mathcal{A}_n} \right) $ 
        gives a set of (by construction, linearly independent) 
        basis elements which generate
        $H_n (Y)$. Now defining a map
        $f \colon \bigvee_j S^{m_j} \to 
        \bigvee_i S^{n_i}$ by sending, for each $n$,
        all the $n$-spheres to distinct elements
        of $\tilde{\mathcal{A}_n}$ (this map is bijective by
        construction), we obtain a map
        $f$ such that
        $r \circ f$ induces an isomorphism on homology on all
        $n$.
    \end{proof}


    \begin{problem}[]
        Let $X$ be a path-connected CW complex such that
        $H_1 \left( X ; \mathbb{Z} \right) = 0$.
        The goal of this problem is to construct a
        simply connected space $Z$ and a map
        $X \to Z$ inducing an isomorphism in homology.
        \begin{enumerate}
            \item Give an example of such $X$ such that
                $\pi_1 \left( X \right) \neq 1$.
            \item Consider a set of generators
                for $\pi_1 (X)$, construct another
                CW complex $Y$ by attaching cells to $X$,
                so that
                \begin{itemize}
                    \item $Y$ is simply connected.
                    \item The inclusion $X \subset Y$ 
                        induces an isomorphism on homology
                        in degrees $\ge 3$.
                \end{itemize}
            \item Show that $H_2 \left( Y ; \mathbb{Z} \right) $ 
                is a sum of $H_2 \left( X; \mathbb{Z} \right) $ 
                together with a free abelian group.
                Let $A$ be a set of generators for this
                free summand.
        \end{enumerate}
    \end{problem}


    \begin{proof}
        (1) Since $H_1$ is just the abelianization for
        $\pi_1$ for path-connected spaces,
        this is equivalent to finding
        a path-connected CW complex
        $X$ whose fundamental group is nontrivial, but
        becomes trivial when abelianized. 
        By corollary 1.28 in Hatcher, for any
        group $G$, we can construct a $2$-dimensional CW
        complex $X_G$ such that $\pi_1 (X_G) \cong G$.
        So it suffices to find a nontrivial group whose
        abelianization is trivial. Such a group is
        called a perfect group, and we have
        many examples of such groups. For example,
        any non-abelian simple group is perfect, so
        for example $A_5$ is perfect. 
        The construction of $X_{A_5}$  can now be
        carried out as follows: $A_5$ is 
        generated by
        $\left( 123 \right) $ and
        $\left( 12345 \right) $ which do not commute,
        so we can express (as with any other group)
        $A_5$ as
        \[
        A_5 = \left< g_{\alpha} \mid r_{\beta} \right>
        \] 
        So in this case, the number of generators is simply $2$.
        Then we can construct $X_{A_5}$ from
        $\bigvee_{\alpha} S^{1}$ by attaching
        $2$-cells $e_{\beta}^2$ by the loops specified by
        the words $r_{\beta}$. By Proposition 1.26 in Hatcher,
        $\pi_1 \left( X_{A_5} \right) \cong
        A_5$, and
        $H_1 \left( X_{A_5} \right) 
        \cong \text{ab}\left( A_5 \right) \cong 1 $.\\
        \linebreak
        Another example is the example from
        Exercise 5 in problem set 2: namely,
        the Poincaré homology sphere. We showed that
        $H_1 \left( S^3 / 2 I ; \mathbb{Z} \right) \cong
        0$ while
        $\pi_1 \left( S^3 / 2 I  \right) \cong 2 I \not \cong 1$.
        Furthermore, we showed that
        $S^3 / 2I$ is a manifold, hence admits a CW complex
        structure, and furthermore, as the quotient of
        a path-connected space, it is also path-connected, so
        $S^3 / 2I$ satisfies all the 
        criterions of the problem.\\
        \linebreak

        (2) We want to attach cells to $X$ to obtain
        a $CW$-complex $Y$ which is simply
        connected and induce an isomorphism on
        homology in degrees $\ge 3$ under the inclusion.
        To do this, suppose
        $f \colon \left( S^{1}, s_0 \right) 
        \to \left( X, x_0 \right) $ is
        a nontrivial element in $\pi_1 (X, x_0)$. We can assume
        by the Cellular Approximation Theorem that
        $f$ is cellular. Then
        we can attach a $2$-cell along $f$ which renders
        $f$ based nullhomotopic. Attaching $2$-cells for
        each nontrivial element in $\pi_1(X)$ like this simultaneously,
        we let $Y$ be the resulting space.
        Then we claim that $\pi_1 (Y) \cong 0$.
        To see this, suppose 
        $g \colon \left( S^{1}, s_0 \right) \to 
        \left( Y, x_0 \right) $ is some map. By
        giving $S^{1}$ the standard CW stucture of a single
        $0$-cell which is $s_0$ and a single $1$-cell attached,
        we get by cellular approximation, that
        $g$ is based homotopic to 
        a map $\tilde{g} \colon \left( S^{1}, s_0 \right) 
        \to \left( Y, x_0 \right) $ which has image
        in $X$. Thus $\tilde{g}$ represents
        an element of $\pi_1 \left( X, x_0 \right) $,
        but by construction of $Y$, $\tilde{g}$ is then
        based nullhomotopic. Composing these homotopies,
        we find that $g$ is based nullhomotopic, so
        $\pi_1 \left( Y \right) \cong 0$.\\
        \linebreak
        
        It remains to show that the inclusion induces
        isomorphisms in homology in degrees $\ge 3$.
        Let $I$ be an indexing set
        for the attaching maps of the
         $2$-cells 
         $\left\{ e_{\alpha}^2 \right\}_{\alpha \in I}$
         that we attached to obtain $Y$ from $X$.
         Let also $A_n$ be an indexing set for
         the $n$-cells in the CW structure of $X$ (we can also view
         $A_n$ as an indexing set for the
         $n$-simplices in the $\Delta$-complex structure obtained
         from $X$ using its CW structure).
         In either case, we obtain a chain complex
         from this CW/$\Delta$-complex structure 
         along with a chain map induced
         by the inclusion $X \hookrightarrow Y$ which
         is the identity in all degrees except degree
         $2$ :
         \begin{equation*}
    \begin{tikzcd}
        \ldots \ar[r] &\bigoplus_{A_4} \mathbb{Z} \ar[r,
        "\partial_4^{X}"] \ar[d, equal] 
        & \bigoplus_{A_3} \mathbb{Z} \ar[r, "\partial_3^{X}"]
        \ar[d, equal] &
        \bigoplus_{A_2} \mathbb{Z} \ar[d, hookrightarrow] \ar[r,
        "\partial_2^{X}"] &
        \bigoplus_{A_1} \mathbb{Z} \ar[d, equal] 
        \ar[r, "\partial_1^{X}"] & \ldots \\
        \ldots \ar[r] &\bigoplus_{A_4}\mathbb{Z} \ar[r,
        "\partial_4^{Y}"]
        & \bigoplus_{A_3} \mathbb{Z} \ar[r, "\partial_3^{Y}"] &
        \bigoplus_{A_2 \sqcup I} \mathbb{Z} \ar[r, "\partial_2^{Y}"]
        & 
        \bigoplus_{A_1} \mathbb{Z} \ar[r, "\partial_1^{Y}"] & \ldots
    \end{tikzcd}
    \end{equation*}
    Now, recalling that the induced
    map  $\iota_{*} \colon
    H_n \left( X \right) \to 
    H_n (Y)$ is given by
    $\left[ c \right] \mapsto 
    \left[ \iota \circ c \right] $, 
    the maps on homology in degrees $ \ge 3$ will
    simply be the identity since
    for any $n \ge 3$, $\partial_n^{Y} = 
    \partial_n^{X}$, so
    \[
    H_n (Y) = \ker \partial_n^{Y} / \im \partial_{n+1}^{Y}
    = \ker \partial_n^{X} / \im \partial_{n+1}^{X}
    = H_n (X).
    \] 
    (3) Using the LES of the pair $\left( Y,X \right) $, we
    find that
    \[
    H_3 \left( Y,X \right) \stackrel{\partial_*}{\to} 
    H_2 \left( X \right) \stackrel{i_*}{\to} 
    H_2 (Y) \stackrel{j_*}{\to}  H_2 (Y,X) \stackrel{\partial_*}{\to} 
    H_1 \left( X \right) 
    \] 
    is exact. 
    Now, note that since $X$ is a CW subcomplex, it
    is, in particular, closed and
    the inclusion  $X \hookrightarrow Y$ is a cofibration,
    so the quotienting map
    $\left( Y,X \right) \to \left( Y / X, * \right) $ 
    induces an isomorphism
    $H_* \left( Y,X \right) \cong
    H_* \left( Y / X, * \right) \cong
    \tilde{H}_* (Y / X)$
    (Corollary 1.7 together with
    Corollary 1.4, Chapter VII in Bredon's Topology and Geometry).
    Now, $Y / X$ is a wedge of
    $2$-spheres, so 
    $\tilde{H}_3 (Y / X) \cong 0$ by considering
    its chain in cellular or simplicial homology.
    As for $H_1 (X)$, this vanishes by assumption on the
    space $X$, so we finally obtain that
    
    \[
    0 \to H_2 \left( X \right) \stackrel{i_*}{\to} 
    H_2 (Y) \stackrel{j_*}{\to}  H_2 (Y,X) \to 0
    \] 
    is a SES.
    Now, using the exact same argument as above,
    $H_2 \left( Y,X \right) \cong
    \tilde{H}_2 (Y /X)$ and $Y / X$ is a wedge of
    $2$-spheres indexed by $I$, so 
    $\tilde{H}_2 \left( Y / X \right) \cong
    \bigoplus_{I} \mathbb{Z}$.
    In particular, this is a free abelian group, and
    we can let $A$ be a set of generators
    for this free summand.
    Since any free group is projective,
    this SES splits, so we find that
    \[
    H_2 (Y) \cong H_2 (X) \oplus H_2 (Y,X)
    \cong H_2 (X) \oplus \bigoplus_{I} \mathbb{Z}
    \] 


    (4) Since $Y$ is simply-connected, the Hurewicz
    theorem gives us an isomorphism
    $h \colon \pi_2 (Y) \to H_2 (Y)$ given
    by sending $f \colon \left( S^2, s_0 \right) 
    \to \left( Y,x_0 \right) $ to
    $h \left( \left[ f \right]  \right) =
    f_* \left( \alpha \right) $ where $\alpha$ is
    a generator of $H_2 \left( S^2 \right) $.
    In particular, we can represent each
    basis element $\alpha$ in $A$ by some
    map $\psi_{\alpha} \colon \left( S^2,s_0 \right) 
    \to \left( Y,x_0 \right) $ by pulling $\alpha$ back
    along the Hurewicz isomorphism.
    By the Cellular Approximation Theorem, we may
    again assume that each  $\psi_{\alpha}$ is cellular
    (giving $S^2$ its standard CW structure).
    Now we let $Z$ be the space obtained
    by attaching $3$-cells to $Y$ along the maps
    $\left\{ \psi_{\alpha} \right\}_{\alpha \in A}$.\\
    \linebreak
    (5) 

    The inclusions $X \hookrightarrow Y \hookrightarrow Z$ now
    give the following maps of chain complexes:
    
         \begin{equation*}
    \begin{tikzcd}
        \ldots \ar[r] &\bigoplus_{A_4} \mathbb{Z} \ar[r,
        "\partial_4^{X}"] \ar[d, equal] 
        & \bigoplus_{A_3} \mathbb{Z} \ar[r, "\partial_3^{X}"]
        \ar[d, equal] &
        \bigoplus_{A_2} \mathbb{Z} \ar[d, hookrightarrow] \ar[r,
        "\partial_2^{X}"] &
        \bigoplus_{A_1} \mathbb{Z} \ar[d, equal] 
        \ar[r, "\partial_1^{X}"] & \ldots \\
        \ldots \ar[r] &\bigoplus_{A_4}\mathbb{Z} \ar[r,
        "\partial_4^{Y}"] \ar[d, equal]
        & \bigoplus_{A_3} \mathbb{Z} \ar[r, "\partial_3^{Y}"] 
        \ar[d, hookrightarrow]&
        \bigoplus_{A_2 \sqcup I} \mathbb{Z} \ar[r, "\partial_2^{Y}"]
        \ar[d, equal] & 
        \bigoplus_{A_1} \mathbb{Z} \ar[r, "\partial_1^{Y}"]
        \ar[d, equal]& \ldots
        \\
        \ldots \ar[r] &\bigoplus_{A_4}\mathbb{Z} \ar[r,
        "\partial_4^{Z}"]
        & \bigoplus_{A_3 \sqcup I} 
        \mathbb{Z} \ar[r, "\partial_3^{Z}"] &
        \bigoplus_{A_2 \sqcup I} \mathbb{Z} \ar[r, "\partial_2^{Z}"]
        & 
        \bigoplus_{A_1} \mathbb{Z} \ar[r, "\partial_1^{Z}"] & \ldots
    \end{tikzcd}
    \end{equation*}
    By the exact same reasoning as before,
    since $\partial_{n}^{Z} = \partial_n^{Y} = 
    \partial_n^{X}$ for $n\ge 4$, it follows
    that $H_n (Z) = H_n(Y) = H_n(X)$ for $n\ge 4$ with
    the inclusions again, by the exact same reasoning as
    in (2), inducing the isomorphisms (in fact, equalities,
    and the inclusions simply become the identity).
    For $n= 1$, we have that  $H_1(X) = H_1(Y) =  0$, and
    so since 
    \[
    H_1 (Z) = \ker \partial_1^{Z} / \im \partial_2^{Z}
    = \ker \partial_1^{Y} / \im \partial_2^{Y} = 
    H_1 (Y)
    \] 
    we also find that
    $H_1 (Z) = 0$,
    so the inclusion $X \hookrightarrow Y \hookrightarrow Z$ 
    trivially induces an isomorphism
    $H_1 (X) \to H_1(Z)$.\\
    For $n = 2$, note that we have the following commutative
    diagram (which commutes by naturality of the Hurewicz
    isomorphism):
    \begin{equation*}
    \begin{tikzcd}
        \pi_2 (X) \ar[r, "i_{*}"] \ar[d, "h", "\cong"']
        & \pi_2(Y) \ar[r, "j_*"] \ar[d, "h", "\cong"'] &
        \pi_2 (Z) \ar[d, "h", "\cong"'] \\
        H_2(X) \ar[r, "i_*"] & H_2(Y) 
        \cong H_2(X) \oplus \bigoplus_{I}\mathbb{Z}
        \ar[r, "j_*"] & H_2(Z)
    \end{tikzcd}
    \end{equation*}
    First, recall that the splitting
    \[
    H_2(Y) \cong H_2(X) \oplus \bigoplus_I \mathbb{Z}
    \] 
    was given by
    $\left( \alpha, \beta \right) \mapsto 
    i_* \left( \alpha \right) +
    s \left( \beta \right) $ where
    $s$ is the section for
    $H_2 (Y) \stackrel{j_*}{\to} H_2 (Y,X)$, so
    in particular, the
    inclusion
    $X \hookrightarrow Y$, becomes the inclusion
    $H_2(X) \hookrightarrow H_2(X) \oplus \bigoplus_{I} \mathbb{Z}$ 
    into the $H_2(X)$ factor under this isomorphism.
 
    In this diagram,
    each element $\alpha \in A$ is mapped
    to some representative $\left( S^2, s_0 \right) 
    \to \left( Y, x_0 \right) $ in
    $\pi_2 (Y)$ which, by construction, is based nullhomotopic
    in $Z$, so we see that
    $j_* \circ h^{-1} (\alpha) = 0$, hence
    $j_* \left( \alpha \right) =
    h \circ j_* \circ h^{-1}(\alpha) = 0$.
    Meanwhile, any element in
    $H_2 (X) \subset H_2(X) \oplus \bigoplus_{I}\mathbb{Z}$ 
    is pulled back along $h$ to a nontrivial element
    which has nonzero image under
    $j_*$ (by construction, $Z$ eliminates only
    the elements of $A$ ). Thus
    $j_* \colon
    H_2(Y) \cong H_2(X) \oplus
    \bigoplus_{I}\mathbb{Z} \to H_2(Z)$ is injective
    on the $H_2(X)$ factor.\\
    hence $j_* i_* = \left( j \circ i \right)_* \colon
    H_2 (X) \to H_2 (Z)$ is injective. Now,
    for any nontrivial $\gamma \in H_2 (Z)$,
    this pulls back to a nontrivial element in
    $\pi_2(Z)$ which is the image of some
    $\beta \in \pi_2 (Y)$ under $j_*$ since $j_*\colon
    \pi_2(Y) \to \pi_2(Z)$ is surjective. 
    This maps down to some element
    $\left( x,y \right) \in 
    H_2(X) \oplus \bigoplus_{I}\mathbb{Z}$, and
    since $j_*$ is $0$ on all factors in
    $\bigoplus_{I}\mathbb{Z}$, we find that
    $j_* (x,0) = \gamma$. So
    $\left( j \circ i \right)_* (x) = 
    j_* \left( x,0 \right) = \gamma$, which shows
    that $\left( j \circ i \right)_*$ is also surjective.
    Thus $\left( j \circ i \right)_* \colon
    H_2 (X) \to H_2(Z)$ is an isomorphism.\\
    \linebreak
    Lastly, for $n=3$, it suffices to show that the
    inclusion $Y \hookrightarrow Z$ induces an isomorphism
    $H_3(Y) \to H_3(Z)$ since we already showed in
    (2) that $H_3(X) \to H_3(Y)$ induced by the inclusion
    is an isomorphism can be seen as follows:
    for each basis element in the $I$ part of
    the summand of the domain of $\partial_3^{Z}$ :
    $\bigoplus_{A_3 \sqcup I} \mathbb{Z}$, this
    is by construction of $Z$ mapped
    to an element in $A$ under $\partial_3^{Z}$ which
    is nontrivial in
    $\bigoplus_{A_2 \sqcup I}\mathbb{Z}$, hence
    $\ker \partial_3^{Z} = \ker \partial_3^{Y}$, so
    $H_3 (Z) = H_3(Y)$ and hence
    the inclusion induces the identity which is
    an isomorphism.\\
    \linebreak
    This completes the proof.
    \end{proof}





\newpage

\printbibliography
\end{document}
