\documentclass[reqno]{amsart}
\usepackage{amscd, amssymb, amsmath, amsthm}
\usepackage{graphicx}
\usepackage[colorlinks=true,linkcolor=blue]{hyperref}
\usepackage[utf8]{inputenc}
\usepackage[T1]{fontenc}
\usepackage{textcomp}
\usepackage{babel}
%% for identity function 1:
\usepackage{bbm}
%%For category theory diagrams:
\usepackage{tikz-cd}

\usepackage[backend=biber]{biblatex}
\addbibresource{notes.bib}


\setlength\parindent{0pt}

\pdfsuppresswarningpagegroup=1

\newtheorem{theorem}{Theorem}[section]
\newtheorem{lemma}[theorem]{Lemma}
\newtheorem{proposition}[theorem]{Proposition}
\newtheorem{corollary}[theorem]{Corollary}
\newtheorem{conjecture}[theorem]{Conjecture}

\theoremstyle{definition}
\newtheorem{definition}[theorem]{Definition}
\newtheorem{example}[theorem]{Example}
\newtheorem{exercise}[theorem]{Exercise}
\newtheorem{problem}[theorem]{Problem}
\newtheorem{question}[theorem]{Question}

\theoremstyle{remark}
\newtheorem*{remark}{Remark}
\newtheorem*{note}{Note}
\newtheorem*{solution}{Solution}



%Inequalities
\newcommand{\cycsum}{\sum_{\mathrm{cyc}}}
\newcommand{\symsum}{\sum_{\mathrm{sym}}}
\newcommand{\cycprod}{\prod_{\mathrm{cyc}}}
\newcommand{\symprod}{\prod_{\mathrm{sym}}}

%Linear Algebra

\DeclareMathOperator{\Span}{span}
\DeclareMathOperator{\im}{im}
\DeclareMathOperator{\diag}{diag}
\DeclareMathOperator{\Ker}{Ker}
\DeclareMathOperator{\ob}{ob}
\DeclareMathOperator{\Hom}{Hom}
\DeclareMathOperator{\Mor}{Mor}
\DeclareMathOperator{\sk}{sk}
\DeclareMathOperator{\Vect}{Vect}
\DeclareMathOperator{\Set}{Set}
\DeclareMathOperator{\Group}{Group}
\DeclareMathOperator{\Ring}{Ring}
\DeclareMathOperator{\Ab}{Ab}
\DeclareMathOperator{\Top}{Top}
\DeclareMathOperator{\hTop}{hTop}
\DeclareMathOperator{\Htpy}{Htpy}
\DeclareMathOperator{\Cat}{Cat}
\DeclareMathOperator{\CAT}{CAT}
\DeclareMathOperator{\Cone}{Cone}
\DeclareMathOperator{\dom}{dom}
\DeclareMathOperator{\cod}{cod}
\DeclareMathOperator{\Aut}{Aut}
\DeclareMathOperator{\Mat}{Mat}
\DeclareMathOperator{\Fin}{Fin}
\DeclareMathOperator{\rel}{rel}
\DeclareMathOperator{\Int}{Int}
\DeclareMathOperator{\sgn}{sgn}
\DeclareMathOperator{\Homeo}{Homeo}
\DeclareMathOperator{\SHomeo}{SHomeo}
\DeclareMathOperator{\PSL}{PSL}
\DeclareMathOperator{\Bil}{Bil}
\DeclareMathOperator{\Sym}{Sym}
\DeclareMathOperator{\Skew}{Skew}
\DeclareMathOperator{\Alt}{Alt}
\DeclareMathOperator{\Quad}{Quad}
\DeclareMathOperator{\Sin}{Sin}
\DeclareMathOperator{\Supp}{Supp}
\DeclareMathOperator{\Char}{char}
\DeclareMathOperator{\Teich}{Teich}
\DeclareMathOperator{\GL}{GL}
\DeclareMathOperator{\tr}{tr}
\DeclareMathOperator{\codim}{codim}
\DeclareMathOperator{\coker}{coker}
\DeclareMathOperator{\corank}{corank}
\DeclareMathOperator{\rank}{rank}
\DeclareMathOperator{\Diff}{Diff}
\DeclareMathOperator{\Bun}{Bun}
\DeclareMathOperator{\Sm}{Sm}
\DeclareMathOperator{\Fr}{Fr}
\DeclareMathOperator{\Cob}{Cob}
\DeclareMathOperator{\Ext}{Ext}
\DeclareMathOperator{\Tor}{Tor}



%Row operations
\newcommand{\elem}[1]{% elementary operations
\xrightarrow{\substack{#1}}%
}

\newcommand{\lelem}[1]{% elementary operations (left alignment)
\xrightarrow{\begin{subarray}{l}#1\end{subarray}}%
}

%SS
\DeclareMathOperator{\supp}{supp}
\DeclareMathOperator{\Var}{Var}

%NT
\DeclareMathOperator{\ord}{ord}

%Alg
\DeclareMathOperator{\Rad}{Rad}
\DeclareMathOperator{\Jac}{Jac}

%Misc
\newcommand{\SL}{{\mathrm{SL}}}
\newcommand{\mobgp}{{\mathrm{PSL}_2(\mathbb{C})}}
\newcommand{\id}{{\mathrm{id}}}
\newcommand{\MCG}{{\mathrm{MCG}}}
\newcommand{\PMCG}{{\mathrm{PMCG}}}
\newcommand{\SMCG}{{\mathrm{SMCG}}}
\newcommand{\ud}{{\mathrm{d}}}
\newcommand{\Vol}{{\mathrm{Vol}}}
\newcommand{\Area}{{\mathrm{Area}}}
\newcommand{\diam}{{\mathrm{diam}}}
\newcommand{\End}{{\mathrm{End}}}


\newcommand{\reg}{{\mathtt{reg}}}
\newcommand{\geo}{{\mathtt{geo}}}

\newcommand{\tori}{{\mathcal{T}}}
\newcommand{\cpn}{{\mathtt{c}}}
\newcommand{\pat}{{\mathtt{p}}}

\let\Cap\undefined
\newcommand{\Cap}{{\mathcal{C}}ap}
\newcommand{\Push}{{\mathcal{P}}ush}
\newcommand{\Forget}{{\mathcal{F}}orget}


\title{Homotopy Theory}

\author{Jonas Trepiakas}
\date{}


\begin{document}
    
\maketitle

For these notes, we will follow \cite{Hatcher}, \cite{Bredon}
and \cite{ORW}.

\section{Cofibrations}
For this section, we will follow chapter VII.1 in \cite{Bredon}.\\
\linebreak
One of the fundamental questions in topology is the
"extension problem". Namely, given a map
$g \colon A \to Y$ defined on a subspace $A$ of $X$, when
can we extend this map to all of $X$.

This cannot always be done - for example, as is the case
with $A = Y = S^{n}$ and $X = D^{n+1}$ choosing the
map to be any degree $-1$ map.\\
\linebreak
\begin{question}
    Is the extension problem a \textit{homotopy-theoretic} problem?
    That is, does the answer depend only on the homotopy
    class of $g$?
\end{question}
The answer is: generally not. 
For example, we can take $X = \left[ 0,1 \right] ,
A = \left\{ 0 \right\} \cup \left\{ \frac{1}{n} \mid 
n=1, 2, \ldots \right\} $ and $Y = CA$, the cone
on $A$. Choosing $g$ to be the inclusion of
$A$ into $Y$, this cannot be extended to $X$ as the
extension would be discontinuous at $\left\{ 0 \right\} $.
However, $g \simeq g'$ with $g'$ being the constant
map of $A$ to the vertex of the cone, and $g'$ easily
extends to $X$ by the constant map.\\
\linebreak
It turns out, however, that under some very mild conditions
on the spaces, the problem becomes homotopy theoretic. 
We will now discuss this.

\begin{definition}[Homotopy extension property]
    Let $\left( X,A \right) $ and $Y $ be given spaces.
    Then $\left( X, A \right) $ is said to have
    the \textit{homotopy extension property} with respect to
    $Y$ if the following diagram can always be completed
    to be commutative.
    \begin{equation*}
    \begin{tikzcd}
        A \times I \cup X \times \left\{ 0 \right\} 
        \ar[r] \ar[d, hookrightarrow] & Y\\
        X \times I \ar[ru, dashed]
    \end{tikzcd}
    \end{equation*}

    One can also depict this by the following diagram:
    \begin{equation*}
    \begin{tikzcd}
        A \times \left\{ 0 \right\} \ar[rr, hookrightarrow]
        \ar[dd, hookrightarrow] 
        & & A \times I \ar[dd, hookrightarrow] \ar[ld] \\
        & Y & \\
        X \times \left\{ 0 \right\} \ar[ru] \ar[rr] && X \times I
        \ar[lu, dashed]
    \end{tikzcd}
    \end{equation*}
\end{definition}

If $\left( X,A \right) $ has the homotopy extension property
with respect to $Y$, then the extensibility of maps
$g \colon A \to Y$ depends only on the homotopy class of
$g$. For suppose $H \colon g \simeq g'$ and $g'$ can be
extended to  $\tilde{g'} \colon X \to Y$, 
then define the map
$A \times I \cup  X \times \left\{ 0 \right\} $ by
$\tilde{g'} \times \left\{ 0 \right\} $ on
$X \times \left\{ 0 \right\} $ and
$H$ on $A \times I$. The homotopy extension property for the
pair $(X,A)$ then guarantees the existence of a map
$G \colon X \times I \to Y$ which equals
$g$ on $A \times \left\{ 1 \right\} $, so
$H \left( -,1 \right) \colon X \to Y$ extends $g$.

\begin{definition}[Cofibration]
    Let $f \colon A \to X$ be a map. Then $f$ is called
    a \textit{cofibration} if one can always fill in the following
    commutative diagram given the solid arrows:
    \begin{equation*}
    \begin{tikzcd}
        A \times \left\{ 0 \right\} \ar[dd, "f \times \id"] 
        \ar[rr]
        & & A\times I \ar[dd, "f \times \id"] \ar[dl] \\
            & Y &\\
        X \times \left\{ 0 \right\} \ar[ru]
        \ar[rr]& & X \times I 
        \ar[lu, dashed]
    \end{tikzcd}
    \end{equation*}
    for \textit{any} space $Y$.
\end{definition}

\begin{note}
    If $f$ is an inclusion, the this is the same
    as the homotopy extension property for all $Y$. That attribute
    is sometimes referred to as the 
    \textit{absolute homotopy extension property}.
\end{note}

\begin{theorem}[]\label{Thm:Retract-cofibration}
    For an inclusion $A \subset X$, the following are equivalent:
    \begin{enumerate}
        \item The inclusion map $A \hookrightarrow X$ is a 
            cofibration.
        \item $A \times I \cup  X \times  \left\{ 0 \right\} $ 
            is a retract of $X \times I$.
    \end{enumerate}
\end{theorem}

\begin{proof}
    If the inclusion is a cofibration, then choosing
    $Y = A \times I \cup  X \times \left\{ 0 \right\} $ 
    with all arrows being inclusions in the
    diagram of a cofibration, we obtain a map
    $X \times I \to A \times I \cup  X \times \left\{ 0 \right\} $ 
    which is the identity on
    $A \times I \cup  X \times \left\{ 0 \right\} $.\\
    Conversely, if $A \times I \cup  X \times \left\{ 0 \right\} $ 
    is a retract of $X \times I$, then
    we can always complete the diagram by
    mapping $X \times I \to 
    A \times I \cup  X \times  \left\{ 0 \right\} 
    \to Y$ where the second map
    takes the maps $A \times I \to Y$ and
    $X \times \left\{ 0 \right\} \to Y$ from the diagram.
\end{proof}

\begin{corollary}\label{Cor:Subcomplex-Cofibration}
    If $A$ is a subcomplex of a CW-complex $X$, then
    the inclusion $A \hookrightarrow X$ is a cofibration.
\end{corollary}

\begin{proof}
    We want to construct a retraction
    $X \times I \to A \times I \cup  X \times \left\{ 0 \right\} $.
    We will do so by constructing a retraction
    $\left( \left( A \cup X^{(r)}  \right)\times I  \right) \cup 
    \left( X \times \left\{ 0 \right\}  \right) 
    \to \left( A \times I \right) \cup \left( X \times 
    \left\{ 0 \right\} \right) $ by induction on $r$.
    If it has been defined on the
    $(r-1)$-skeleton, then extending it over an
    $r$-cell is simply a matter of extending a map
    on $S^{r-1} \times I \cup D^{r} \times \left\{ 0 \right\} $ 
    over $D^{r} \times I$ which can be done
    since the pair
    $\left( D^{r} \times I, S^{r-1} \times I
    \cup D^{r} \times \left\{ 0 \right\} \right) $ is homeomorphic
    to $\left( D^{r} \times I, D^{r}\times \left\{ 0 \right\} 
    \right) $.
    See Figure \ref{fig:DUWUWUJK122-png}

    \begin{figure}[htpb]
        \centering
        \includegraphics[width=0.8\textwidth]{DUWUWUJK122.png}
        \caption{A homeomorphism of pairs.}
        \label{fig:DUWUWUJK122-png}
    \end{figure}
    These maps for each cell fit together to
    give a map on the $r$-skeleton because of the
    weak topology on $X \times I$. The union of these
    maps for all $r$ gives a map on $X \times I$, again because
    of the weak topology on $X \times I$.
\end{proof}

\begin{theorem}[]\label{Thm:SJJDHW29WW}
    Assume that $A \subset X$ is closed and that there
    exists a neighborhood $U$ of $A$ and a map
    $\varphi  \colon X \to I$ such that
    \begin{enumerate}
        \item $A = \varphi ^{-1} (0)$.
        \item $\varphi \left( X-U \right) = \left\{ 1 \right\} $.
        \item $U$ deforms to $A$ through $X$ with $A$ fixed.
            That is, there is a map $H \colon U \times I \to X$ 
            such that $H(a,t) = a$ for all $a \in A, H
            (u,0) = 0$, and $H(u,1) \in A$ for all $u \in U$.
    \end{enumerate}
    Then the inclusion
    $A \hookrightarrow X$ is a cofibration. The converse also
    holds.
\end{theorem}

\begin{proof}
    We may assume that $\varphi  = 1$ on a neighborhood
    of $X - U$ by replacing $\varphi $ with
    $\min \left( 2 \varphi , 1 \right) $.
    It suffices to show that there exists a retract
    $\Phi \colon U \times I \to X \times \left\{ 0 \right\} 
    \cup A \times I$ since then the
    map
    \[
    r\left( x,t \right) 
    =
    \begin{cases}
        \Phi\left( x, t \left( 1-\varphi (x) \right)  \right),&
        x \in U\\
        (x,0),& x \not\in U
    \end{cases}
    \] 
    gives a retraction $X \times I \to A \times I \cup 
    X \times \left\{ 0 \right\} $.\\
    We define $\Phi$ by
    \[
    \Phi(u,t) = 
    \begin{cases}
        H\left( u, \frac{t}{\varphi (u)} \right) \times 
        \left\{ 0 \right\},& \varphi (u) > t\\
        H\left( u,1 \right) \times 
        \left\{ t- \varphi (u) \right\} ,& \varphi (u)\le t.
    \end{cases}
    \] 
    The only thing that needs checking here is
    that $\Phi$ is continuous at
    points $\left( u,0 \right) $ such that
    $\varphi (u) = 0$, i.e., points $(a,0)$ for 
    $a \in A$ - indeed here the expression for
    $\varphi (u) > t$ is not defined.

    Recall that a map $f \colon X \to Y$ is continuous
    if for every point $x \in X$ and any neighborhood
    $U$ of $f(x)$, there exists a neighborhood
    $V$ of $x$ such that $f(V) \subset U$.\\
    So let $W$ be a neighborhood of $a = 
    H(a,t)$. Then there exists a neighborhood
    $V \subset W$ containing $a$ such that
    $H\left( V \times I \right) \subset W$, by
    assumption of $H$ being continuous.
    So for $t < \varepsilon$ for some $\varepsilon$ and
    $u \in V$, we have
    $\Phi \left( u,t \right) 
    \in W \times \left[ 0,\varepsilon \right] $.
    Hence $\Phi$ is continuous.\\
    \linebreak
    To prove the converse, suppose that the
    inclusion $A \hookrightarrow X$ is a cofibration. Equivalently,
    $A \times I \cup  X \times \left\{ 0 \right\} $ is
    a retract of $X \times I$. Let $r
    \colon X \times I \to  A \times I \cup X \times
    \left\{ 0 \right\} $ be this retraction.
    Let $s(x) = r\left( x,1 \right) $ and set
    $U = s^{-1}\left( A \times (0,1] \right) $.
    Let $p_X, p_I$ be the projections of
    $X \times I$ to its factors. Then
    put $H = p_X \circ r|_{U \times I} \colon U \times I \to X$.
    Now, $H(a,t) = p_X \circ r |_{U \times I}(a,t)
    = p_X \left( a,t \right) = a$ for all
    $a \in A$ and $t \in I$;
    $H(u,0) = p_X \circ r|_{U \times I}(u,0) =
    p_X \left( u,0 \right) = u$, and
    $H\left( u,1 \right) =
    p_X \circ r|_{U \times I}(u,1) = 
    u$ forces $(u,1) \in A \times I$, hence
    $u \in A$. Thus, $H$ satisfies condition (3).\\
    For (1) and (2), let
    $\varphi (x) = 
    \max_{t \in I} \left| t - p_I r(x,t) \right| $ which
    is possible since $I$ is compact. Then
    $x \in \varphi^{-1}(0)$ implies that
    $\max_{t \in I} \left| t - p_I r(x,t) \right| = 0$, so
    for all $t \in I$, we have
    $\left| t - p_I r(x,t) \right| = 0$, so
    $r(x,t) \in A \times \left\{ t \right\} $ for all
    $t \in (0,1]$. Then
    $r\left( x,0 \right) = \lim_{n \to \infty}
    r\left( x, \frac{1}{n} \right) 
    \in A \times I$ since $A \times I$ is closed. But
    $(x,0) = r(x,0)$, so $x \in A$. Conversely, for any
    $x \in A$, clearly, $\varphi (x) = 0$ since
    $r(x,t) = (x,t)$ for all $t \in I$. This shows that
    $\varphi $ satisfies (1). 
    For (2), we have that for
    $x \in X -U$, with $U = 
    s^{-1}\left( A \times (0,1] \right) $, we
    have $r(x,1) = s(x) \not\in A \times (0,1]$, so
    $r(x,1) \in X \times \left\{ 0 \right\} $. Hence
    $\varphi (x) = 
    \max_{t \in I} \left| t - p_I r(x,t) \right| = 1$, giving
    (2).\\
    It remains to show that $\varphi $ is continuous.
    Let $f(x,t) = \left| t - p_I r(x,t) \right| $ and
    $f_t = (x,t)$ all of which are continuous.
    Then
    \[
        \varphi^{-1} \left( (- \infty, b] \right) 
        = \left\{ x  \mid f(x,t) \le b \text{ for all }t 
        \right\} = 
        \bigcap_{i \in  I} f_t^{-1}\left( (- \infty, b] \right) .
    \]
    is an intersection of closed sets and so is closed.
    Similarly,
    \[
    \varphi^{-1} \left( [a, \infty) \right) 
    = \left\{ x  \mid  f(x,t) \ge a \text{ for some }t \right\} 
    = p_X \left( f^{-1} \left( [a, \infty) \right)  \right) 
    \] 
    which si also closed since $p_X$ is closed as
    a projection and
    $I$ is compact.
    Since the complements of the intervals of the
    form $[a, \infty)$ and 
    $(-\infty, b]$ give a subbase for the topology of
    $\mathbb{R}$, this shows that
    $\varphi $ is continuous.





\end{proof}


Next, we recall that for a map
$f \colon X \to Y$, the mapping cylinder
$M_f$ is defined as
\[
M_f = \left( \left( X \times I \right) \sqcup Y \right) 
/ \left( \left( x,0 \right) \sim f(x) \right) .
\] 
Consider the inclusion
$\iota \colon X \hookrightarrow 
M_f$ where we include $X$ as
$X \times \left\{ 1 \right\} $.
Consider the map
$ \varphi \colon
M f \to I$ given by
$ \varphi (x,t)  =  1 - 2t$ for
$t \ge \frac{1}{2}$ and
$\varphi (x,t) = 1$ on the rest of $M_f$. Choosing
$U = X \times (\frac{1}{3}, 1]$, $U$ clearly
deformations retracts to $X \times \left\{ 1 \right\} $ and
satisfies the conditions of 
Theorem \ref{Thm:SJJDHW29WW}, hence
the inclusion $X \hookrightarrow M_f$ is a cofibration.
Also, the retraction
$r \colon M_f \to Y$ is a homotopy equivalence
with the homotopy inverse being the inclusion
$Y \hookrightarrow M_f$. The diagram
\begin{equation*}
\begin{tikzcd}
    X \ar[rr, "\iota"] \ar[ddr, "f"'] & & M_f \ar[ddl, "r", "\simeq"']\\ 
                                     &&\\
                                     & Y &
\end{tikzcd}
\end{equation*}
commutes.
Thus any map $f$ is a cofibration up to
a homotopy equivalence of spaces.

Recall also that the mapping cone of a
map $f \colon X \to Y$ is defined as
\[
C_f := M_f / X \times \left\{ 1 \right\} 
\cong M_f \cup CX.
\] 
In the case of an inclusion
$\iota \colon A \hookrightarrow X$, we have
$C_{\iota} = X \cup  CA$.\\
There is a map
$C_{\iota} \stackrel{h}{\to} X / A$, defined
as the composite of the quotient map
$X \cup  CA \to X \cup  CA /CA$ composed with the
inverse of the homeomorphism $X / A \to 
X \cup  CA /CA$.

\begin{question}
    Is $h$ a homotopy equivalence?
\end{question}

\begin{theorem}[]\label{Thm:2030akKAK}
    If $A \subset X$ is closed and the inclusion
    $\iota \colon A \to X$ is a cofibration, then
    $h \colon C_{\iota} \to X /A$ is a homotopy equivalence.
    In fact, it is a homotopy equivalence of pairs
    \[
        \left( X /A, * \right) \simeq
        \left( C_{\iota}, CA \right) \simeq
        \left( C_{\iota}, v \right) ,
    \] 
    where $v$ is the vertex of the cone.
\end{theorem}

\begin{proof}
    The mapping cone $C_{\iota} = X
    \cup CA$ consists of three different types of
    points: the vertex $v = \left\{ A \times \left\{ 1 \right\} 
    \right\} $, the rest of the cone
    $\left\{ \left( a,t \right)  \mid 0\le t < 1 \right\} $ 
    where $\left( a,0 \right) = a \in A \subset X$, and
    points in $X$ itself, which we identify with $X \times 
    \left\{ 0 \right\} $.\\
    Define $f \colon A \times I \cup  X \times \left\{ 0 \right\} 
    \to C_{\iota}$ as the collapsing map and extend
    $f$ to $\overline{f} \colon X \times I \to 
    C_{\iota}$ using that $f$ is a cofibration. 
    Then $\overline{f}\left( a,1 \right) =
    v, \overline{f}(a,t) = (a,t)$ and
    $\overline{f}(x,0) = x$.\\
    Let $\overline{f}_t = 
    \overline{f}|_{X \times \left\{ t \right\} }$.
    Since $\overline{f}_1 (A) = \left\{ v \right\} $, 
    we can factorize $\overline{f}_1 \colon
    X \to C_{\iota}$ as
    $g \circ j$ where
    $j \colon X \to X /A$ is the quotient map
    and $g \colon X / A \to C_{\iota}$ is
    the induced map
    \begin{equation*}
    \begin{tikzcd}
        X \ar[d, "j"] \ar[dr, "\overline{f}_1"] & \\
        X / A \ar[r, "g", dashed] & C_{\iota}.
    \end{tikzcd}
    \end{equation*}
    where $g$ is induced and continuous by
    definition of the quotient topology.\\
    We claim that $g$ is a homotopy equivalence with
    homotopy inverse $h$.
    First, we prove that $hg \simeq \id_{X / A}$.

    Note that taking the composite
    $h \overline{f}_t \colon X \to X / A$ gives a homotopy
    between $h \overline{f}_0$ and $h \overline{f}_1$.
    For all $t$, this homotopy takes
    $A$ to the point $\left\{ A \right\} $. Thus, it
    factors to give a homotopy
    \[
    hgj = h \overline{f}_1 
    \simeq h \overline{f}_0 = j
    \] 
    Let $H \colon X \times I \to X / A$ be the homotopy
    between $hgj$ and $j$, so
    $H(x, 0) = hgj(x)$ and
    $H(x,1) = j(x)$. Then
    the map
    $\overline{H} \colon X / A \times I \to X / A$ defined
    by
    $\overline{H}(\left[ x \right] ,t) = 
    H\left( x,t \right) $ defines a homotopy
    between $hg$ and $\id_{X / A}$, so
    $hg \simeq \id_{X / A}$.

    Next, we will show that $gh \simeq \id_{C_{\iota}}$.
    Consider $W = \left( X \times I \right) / 
    \left( A \times \left\{ 1 \right\}  \right) $ 
    and the maps illustrated in Figure \ref{fig:DWIJJXNXJNXJUI-png}.

    \begin{figure}[htpb]
        \centering
        \includegraphics[width=0.8\textwidth]{DWIJJXNXJNXJUI.png}
        \caption{DWIJJXNXJNXJUI.png}
        \label{fig:DWIJJXNXJNXJUI-png}
    \end{figure}
    The map $\overline{f}'$ is induced by
    $\overline{f}$. The map
    $k$ is the "top face" map.
    From this, we see that
    \begin{align*}
        \overline{f}' \circ l 
        &= \id \\
        \pi \circ k 
        &= \id \\
        k \circ \pi 
        &\simeq \id \\
        \overline{f}' \circ k
        &= g \\
        \pi \circ l 
        &= l.
    \end{align*}
   Hence $g h = \overline{f}' k \pi l 
   \simeq \overline{f}' l = \id$. 
\end{proof}


\begin{example}[A non example]
    An example of when the result
    of Theorem 1.6 does not hold is with
    $A = \left\{ 0 \right\} \cup 
    \left\{ \frac{1}{n} \mid n = 1, 2, \ldots \right\} $ 
    and $X = \left[ 0,1 \right] $.
    In this case, $C_{\iota}$ is not homotopy equivalent
    to $X / A$ which is a one-point union of a countably infinite
    sequence of circles with radii going to zero.

    $C_i$ has homeomorphs of circles joined along edges. However,
    the circles do not tend to a point ,so any prospective homotopy
    equivalence $X / A \to C_{\iota}$ would be discontinuous at
    the image of $\left\{ 0 \right\} $ in $X / A$.
\end{example}

\begin{corollary}\label{Cor:Cofibration-Homology}
    If $A \subset X$ is closed and the inclusion
    $A \hookrightarrow X$ is a cofibration, then the map
    $j \colon \left( X, A \right) \to 
    \left( X / A, * \right) $ induces isomorphisms
    \[
    H_* (X,A) \stackrel{\cong}{\to} 
    H_* \left( X / A, * \right) 
    \cong \tilde{H}_* \left( X / A \right) 
    \] 
    and
    \[
    \tilde{H}^{*}(X /A) \cong
    H^{*} (X / A, *) 
    \stackrel{\cong}{\to} H^{*} (X , A).
    \] 
\end{corollary}

\begin{proof}
    We have
    $H_* \left( X/A , *\right) \cong
    H_* \left( C_{\iota}, CA \right) $ by Theorem
    \ref{Thm:2030akKAK}. And since
    $C_{\iota} = X \cup A \times \left[ 0,\frac{1}{2} \right] $ 
    and $CA = A \times \left[ 0,\frac{1}{2} \right] $, where
    we collapse $A \times \left\{ \frac{1}{2} \right\} $ in
    both, and attach  $A \times \left[ 0,\frac{1}{2} \right] $ 
    along $A \times \left\{ 0 \right\} $ in
    $X \cup A \times \left[ 0,\frac{1}{2} \right] $, we obtain
    \[
    H_*(C_{\iota}, CA) \cong
    H_* \left( X \cup A \times \left[ 0,\frac{1}{2} \right] ,
    A \times \left[ 0, \frac{1}{2} \right] \right) 
    \cong H_* (X, A)
    \] 
    since 
    $\left( X \cup A \times \left[ 0,\frac{1}{2} \right] ,
    A \times \left[ 0,\frac{1}{2} \right] \right) 
    \simeq \left( X,A \right) $ by deformation retracting
    $A \times \left[ 0,\frac{1}{2} \right] $ down to
    $A \times \left\{ 0 \right\} \subset X$.
\end{proof}


\subsubsection{Interlude on pointed-spaces and
operations on spaces}

We recall some important constructions:

\begin{definition}[Unreduced Suspension]
    For a space $X$, the \textit{unreduced suspension} 
    $\Sigma X$ is the
    quotient obtained from $X \times I$ by
    collapsing $X \times \left\{ 0 \right\} $ to one
    point and $X \times \left\{ 1 \right\} $ to another
    point.
\end{definition}

\begin{note}
    We have $\Sigma S^{n} = S^{n+1}$.
\end{note}



\begin{definition}[Suspension of a map]
    Given a map
    $f \colon X \to Y$, we can suspend $f$ to
    $\Sigma f \colon \Sigma X \to \Sigma Y$
    by letting $\Sigma f$ be the induced
    map on the quotients:
    \begin{equation*}
    \begin{tikzcd}
        X \times I \ar[r, "f \times \id"] \ar[d]  & Y \times I \ar[d] \\
        \Sigma X \ar[r, "\Sigma f"] & \Sigma Y
    \end{tikzcd}
    \end{equation*}
    
\end{definition}

\begin{exercise}[]
    For any homology theory, show that
    there is a natural isomorphism
    $\tilde{H}_I (X) \stackrel{\cong}{\to} 
    \tilde{H}_{i+1} \left( \Sigma X \right) $. Here,
    natural means that for a map $f \colon X \to Y$,
    and its suspension $\Sigma f \colon \Sigma X \to 
    \Sigma Y$, the following diagram commutes:
    \begin{equation*}
    \begin{tikzcd}
        \tilde{H}_i (X) \ar[r, "\cong"] \ar[d, "f_*"]& 
        \tilde{H}_{i+1} \left( \Sigma X \right)
        \ar[d, "(\Sigma f)_*"] \\
        \tilde{H}_i (Y) \ar[r, "\cong"] & \tilde{H}_{i+1}
        \left( \Sigma Y \right) 
    \end{tikzcd}
    \end{equation*}
    
\end{exercise}


\begin{definition}[Wedge Sum/one-point union]
    Given two pointed spaces $(X, x_0), \left( Y, y_0 \right) $,
    we define
    the \textit{wedge sum}  $X \vee Y$ to be
    \[
    X \vee Y = X \sqcup Y / \left( x_0 \sim y_0 \right) ,
    \] 
    i.e., the quotient of the disjoint union identifying
    $x_0$ and $y_0$ to a single point.
\end{definition}

\begin{definition}[Smash Product]
    Inside the product $X \times Y$
    of two pointed space $(X,x_0), (Y,y_0)$,
    we have natural copies of $X$ and $Y$ by
    $X \times \left\{ y_0 \right\} $ and
    $\left\{ x_0 \right\} \times Y$, respectively.
    These two copies intersect only at the point
    $\left( x_0,y_0 \right) $, so their union can
    be identified with
    the wedge sum $X \vee Y$. I.e., $X \vee Y = 
    X \times \left\{ y_0 \right\} \cup 
    \left\{ x_0 \right\} \times Y$. We define
    the \textit{smash product} $X \wedge Y$ to be
    the quotient $X \times Y / X \vee Y$.
\end{definition}


If $f \colon X \to Y$ is a pointed map, then
the reduced mapping cylinder of $f$ is defined
as the quotient space $M_f$ of
$\left( X \times I \right) \cup  Y$ modulo the relations
identifying
$\left( x,0 \right) \sim f(x)$ and the set
$\left\{ * \right\} \times I$ to the base point of $M_f$.\\
The reduced mapping cone is the quotient of the reduced
mapping cylinder $M_f$ obtained by identifying the image of
$X \times \left\{ 1 \right\} $ to a point, the
base point.\\

The circle $S^{1}$ is defined as
$I / \partial I$ with base point $\left\{ \partial I \right\} $.\\
The reduced suspension of a pointed space $X$ is
$SX = X \wedge S^{1}$. It can also be considered
as the quotient space $X \times I / \left( X \times 
\partial I \cup  \left\{ * \right\} \times I\right) $


\begin{definition}[Well-pointed space]
    A base point $x_0 \in X$ is said to be
    \textit{nondegenerate} if the inclusion
    $\left\{ x_0 \right\} \hookrightarrow X$ is a cofibration.
    A pointed Hausdorff space $X$ with nondegenerate
    base point is said to be \textit{well-pointed}.
\end{definition}

It is clear that any manifold or CW-complex satisfies
Theorem \ref{Thm:SJJDHW29WW} with $A$ being any point of
the space. Hence any manifold or CW-complex is
well-pointed.\\

\begin{example}[Pointed space that is not well-pointed]
    Taking the pointed space  $X = 
    \left\{ 0 \right\} \cup  \left\{ \frac{1}{n} \mid 
    n \in \mathbb{N} \right\} $ with base point $0$, this
    space is not well-pointed. This can for example be
    seen because it fails to satisfy Theorem
    \ref{Thm:Retract-cofibration} - any retraction
    would break continuity at
    $\left( 0,1 \right) $.
\end{example}

\begin{example}[]
    If $A \hookrightarrow X$ is a cofibration, then
    $X / A$ with base point $\left\{ A \right\} $ is well-pointed,
    as follows from Theorem \ref{Thm:SJJDHW29WW}.
\end{example}

\begin{theorem}[]
    If $X$ is well-pointed, then so are the reduced
    cone $CX$ and the reduced suspension $SX$. Moreover,
    the collapsing map  $\Sigma X \to SX$, of the unreduced
    suspension to the reduced suspension, is a homotopy
    equivalence.
\end{theorem}

\begin{proof}
    Denote the base point of $X$ by $*$.
    Consider the homeomorphism
    \[
    h \colon \left( I \times I,
    I \times \left\{ 0 \right\} \cup 
\partial I \times I \right) 
\stackrel{\cong}{\to} \left( I \times I, I \times 
\left\{ 0 \right\} \right) 
    \] 
    which clearly exists. For example, take Figure
    \ref{fig:IWIDK01-jpeg}
    \begin{figure}[htpb]
        \centering
        \includegraphics[width=0.5\textwidth]{IWIDK01.jpeg}
        \caption{}
        \label{fig:IWIDK01-jpeg}
    \end{figure}

    Then the induced homeomorphism
    \[
    \id_X \times h\colon X \times I \times I 
    \stackrel{\cong}{\to} X \times I \times I
    \] 
    carries 
    $X \times I \times \left\{ 0 \right\} \cup 
    X \times \partial I\times I$ to
    $X \times I \times \left\{ 0 \right\} $.
    Hence it takes
    $A = X \times I \times \left\{ 0 \right\} \cup 
    X \times \partial I \times I \cup 
    \left\{ * \right\} \times I \times I$ to
    $X \times I \times \left\{ 0 \right\} 
    \cup \left\{ * \right\} \times I \times I$. 
    Therefore, the pair
    $\left( X \times I \times I, A \right) $ is homeomorphic
    to the pair
    $I \times \left( X \times I, 
    X \times \left\{ 0 \right\} \cup 
\left\{ * \right\} \times I \right) $.
Now, $X$ is well-pointed, so
$X \times \left\{ 0 \right\} \cup 
\left\{ * \right\} \times I$ is a retract of
$X \times I$ by
Theorem \ref{Thm:Retract-cofibration} and the
definition of well-pointed.
It follows that
$A$ is a retract of
$X \times I \times I$.
By another application of
\ref{Thm:Retract-cofibration}, then
the inclusion
$X \times \partial I \cup \left\{ * \right\} \times I
\hookrightarrow X \times I$ is a cofibration. 
Hence the quotient by this,
$SX = X \times I / \left( X \times \partial I
\cup \left\{ * \right\} \times I \right) $ is well-pointed,
using the quotient of the above inclusion.\\
\linebreak
Next consider the homeomorphism
$\left( I \times I, I \times \left\{ 0 \right\} \cup 
\left\{ 1 \right\} \times I \right) 
\stackrel{\cong}{\to} \left( 
I \times I, I \times \left\{ 0 \right\} \right) $ which
can be seen similarly. The induced
homeomorphism
\[
1 \times h \colon X \times I \times I
\stackrel{\cong}{\to} X \times I \times I
\] 
takes
$A:= X \times \left\{ 1 \right\} \times I \cup 
\left\{ * \right\} \times I \times I
\cup X \times I \times \left\{ 0 \right\} $ to
$X \times I \times \left\{ 0 \right\} \cup 
\left\{ * \right\} \times I \times I$.
Thus the pair
$\left( X \times I \times I,
A\right) $ is homeomorphic to
 $I \times \left( X \times I,
 X \times \left\{ 0 \right\} \cup 
\left\{ * \right\} \times I\right) $.
Just as above, we have that
$X \times \left\{ 0 \right\} \cup 
\left\{ * \right\} \times I$ is a retract
of $X \times I$, so
it follows that
$A$ is a retract of $X \times I \times I$. Thus
the inclusion
$X \times \left\{ 1 \right\} \cup 
\left\{ * \right\} \times I \hookrightarrow
X \times I$ is a cofibration, which shows
that $CX = X \times I / \left( X \times \left\{ 1 \right\} 
\cup \left\{ * \right\} \times I\right) $ is
well-pointed.\\
\linebreak
The fact that
$X \times \partial I \cup \left\{ * \right\} \times I
\hookrightarrow X \times I$ is a cofibration
gives that there exists a neighborhood $U$ of
$X \times \partial I \cup 
\left\{ * \right\} \times I$ and a map
$\varphi \colon X \times I \to I$ 
that satisfy Theorem \ref{Thm:SJJDHW29WW}.
We obtain an induced map
$\overline{\varphi }\colon
\Sigma X \to I$ 
which satisfies the same conditions, so
$I \times \times \left\{ * \right\} \times I
\hookrightarrow X \times I / 
\left\{ X \times \left\{ 0 \right\} ,
X \times \left\{ 1 \right\} \right\} = \Sigma X$ is a
cofibration. Now 
Theorem \ref{Thm:2030akKAK} implies
that $\Sigma X \cup CI = C_{\iota} \to \Sigma X / I$ is a homotopy
equivalence.
Hence we obtain
that 
$\Sigma X \simeq
\Sigma X \cup CI \simeq
\Sigma X / I = SX$, via the collapsing map.

\end{proof}


\begin{problem}[]
    Find $H_* \left( \mathbb{P}^2, \mathbb{P}^{1} \right) $ 
    using methods or results from this section.
\end{problem}

\begin{solution}
    Consider $\mathbb{P}^2$ as 
    $S^2$ quotiented by the relation
    $x \simeq -x$. Then
    we can think of  $\mathbb{P}^{1}$ as
    $S^{1} \subset S^{2}$ under this relation.
    We want to show that the inclusion
    $\mathbb{P}^{1} \hookrightarrow \mathbb{P}^2$ is a
    cofibration. 
    Using Theorem \ref{Thm:SJJDHW29WW}, it suffices to
    find a neighborhood $U$ of $\mathbb{P}^{1} \subset 
    \mathbb{P}^2$ and a map $\overline{\varphi } \colon
    \mathbb{P}^2 \to I$ such that
    the conditions of the theorem are satisfied.
    We construct a preliminary map on
    $S^2$ towards this end. 
    Define $\varphi \colon S^2 \to I$ to be
    $\varphi (x) = 
    \min \left\{ 1, 2 \left| x_3 \right|  \right\} $, where
    $x_3$ is the last coordinate of $x$. Since
    $\varphi (x) = \varphi (-x)$, $\varphi $ induces
    a map $\overline{\varphi }\colon
    \mathbb{P}^2 \to I$ such that the diagram
    \begin{equation*}
    \begin{tikzcd}
        S^2 \ar[d] \ar[dr, "\varphi "] & \\
        \mathbb{P}^2 \ar[r, "\overline{\varphi }"] & I
    \end{tikzcd}
    \end{equation*}
    commutes.
    Letting $U$ be the image under the quotient
    map of 
    $\left\{ x \in S^2  \mid 
    \left| x_3 \right| < \frac{1}{2} \right\} $, this
    becomes an open set in $\mathbb{P}^2$ since 
    the above set is saturated with respect to the quotient
    map. It is also clear that
    $U$ and $\overline{\varphi }$ satisfy the conditions of
    the theorem, hence the inclusion
    $\mathbb{P}^{1} \hookrightarrow \mathbb{P}^2$ is a cofibration.
    By Corollary \ref{Cor:Cofibration-Homology}, we
    obtain that
    $H_* \left( \mathbb{P}^2, \mathbb{P}^{1} \right) 
    \cong \tilde{H}_* \left( \mathbb{P}^2 / \mathbb{P}^{1} \right) $.
    But $\mathbb{P}^2 / \mathbb{P}^{1} \cong
    S^{2}$, so
    $H_* \left( \mathbb{P}^2, \mathbb{P}^{1} \right) 
    \cong \tilde{H}_* \left( S^2 \right)$.
    Now simply recall that
    \[
    \tilde{H}_p \left( S^2 \right) 
    \cong
    \begin{cases}
        \mathbb{Z},& p = 2\\
        0,& p \neq 2.
    \end{cases}
    \] 
    \qed
\end{solution}

\begin{problem}[]
    Find $H_*\left( T^2,
    \left\{ * \right\} \times S^{1} \cup 
S^{1} \times \left\{ * \right\} \right) $ using
methods from this section.
\end{problem}

\begin{solution}
    If we can show that the inclusion
    $A:= \left\{ * \right\} \times S^{1} \cup 
    S^{1} \times \left\{ * \right\} \hookrightarrow
    T^2$ is a cofibration, then
    we will again obtain that
    $H_* (T^2, A) \cong
    \tilde{H}_* \left( T^2 / A \right) \cong
    \tilde{H}_* \left( S^2 \right) $.
    But we have a CW-structure on the
    torus given by the square with identified sides.
    With this identificaiton, $A$ simple becomes
    the $1$-skeleton, hence it is a subcomplex, so
    by Corollary \ref{Cor:Subcomplex-Cofibration}, 
    the inclusion $A \hookrightarrow T^2$ is a cofibration.
    This finishes the solution. \qed
\end{solution}

\begin{problem}[]
    For a space $X$, consider the pair
    $\left( CX, X \right) $. What do the results of this
    section tell you about the homology of these, and related,
    spaces?
\end{problem}

\begin{solution}
    We can define a map
    $\varphi \colon CX \to I$ by
    $\varphi (x,t) = t$. Choosing
    $A = X = X \times \left\{ 0 \right\} \subset 
    CX$ and
    $U = CX - \left\{ v \right\} $ where
    $v$ is the vertex, this satisfies the
    conditions in Theorem \ref{Thm:SJJDHW29WW} 
    ($H$ can be defined by
    $H((x,t_0),t) = \left( x,t_0 \right) (1-t)
    + (x,0) t$).
    Hence the inclusion
    $X \hookrightarrow CX$ is a cofibration, so we
    know that
    $H_* \left( CX, X \right) 
    \cong \tilde{H}_* \left( CX / X \right) $.
    Similarly, one can
    show that the inclusion
    $X \hookrightarrow \Sigma X$ is a cofibration, so
    $H_* \left( \Sigma X , X \right) 
    \cong \tilde{H}_* \left( \Sigma X / X \right) 
    \cong \tilde{H}_* \left( 
    \Sigma X \vee \Sigma X \right) $ and
    $H_*\left( SX, X \right) 
    \cong \tilde{H}_* \left( SX \vee SX \right) $.
\end{solution}








\newpage

\section{Homotopy Groups}

\subsection{Homotopy}
We follow chapter 14 of \cite{Bredon} for this subsection.\\

To start of, we recall the basic definitions of homotopies.

\begin{definition}[Homotopy]
    Two maps $f_0, f_1 \colon X \to Y$ are said to
    be \textit{homotopic} if there exists a homotopy
    $F \colon X \times I \to Y$ such that
    $F(x,0) = f_0(x)$ and $F(x,1) = f_1(x)$ for
    all $x \in X$.
\end{definition}

\begin{definition}[Homotopy equivalence]
    A map $f \colon X \to Y$ is said to be a \textit{homotopy
    equivalence} if it is an isomorphism in
    $\hTop$.
\end{definition}

\begin{lemma}[Reparametrization Lemma]
    Let $\varphi_1, \varphi_2$ be maps
    $\left( I, \partial I \right) \to 
    \left( I, \partial I \right) $ which are equal on
    $\partial I$. Let
    $F \colon X \times I \to Y$ be a homotopy and let
    $G_i (x,t) = F\left( x, \varphi_i(t) \right) $ for
    $i = 1,2$. Then $G_1 \simeq G_2 \rel
    X \times \partial I$.
\end{lemma}

We shall use $c$ to denote the constant homotopy.

\begin{proposition}[]
    $F * c \simeq F \rel X \times \partial I$ and
    $c * F \simeq F \rel X \times \partial I$.
\end{proposition}

\begin{definition}[]
    If $F \colon X \times I \to Y$ is a homotopy, then we
    define $F^{-1} \colon X \times I \to Y$ by
    $F^{-1}\left( x,t \right) = F(x,1-t)$. 
\end{definition}

Note that $F^{-1}$ is precisely the inverse
to $F$ in $\hTop$.

\begin{proposition}[]
    For any homotopies $F,G,H$ for which the
    concatenations 
     are defined, we have
     \[
         \left( F * G  \right) * H
         \simeq F * \left( G * H \right) 
         \rel X \times \partial I.
     \] 
\end{proposition}


\begin{proposition}[]
    For homotopies $F_1, F_2, G_1, G_2$,
    if $F_1 \simeq F_2 \rel X \times \partial I$ and
    $G_1 \simeq G_2 \rel X \times \partial I$, then
    $F_1 * G_1 \simeq F_2 * G_2 \rel X \times \partial I$.
\end{proposition}

Note that all of the discussion of concatenation of
homotopies goes through with no difficulties for the cases
in which all homotopies are relative to some subspace
$A \subset X$ or are homotopies of pairs
$\left( X, A  \right) \to \left( Y, B \right) $.\\
It follows that homotopy between maps of
pairs $\left( X,A \right) \to \left( Y,B \right) $ is
an equivalence relation. The set of homotopy classes
of these maps is commonly denoted by
$\left[ X,A ; Y ,B \right] $ or just
$\left[ X;Y \right] $ if $A = \varnothing$.

\begin{theorem}[]\label{Thm:299221}
    If $f_0 \simeq f_1 \colon X \to Y$ then
    $M_{f_0} \simeq M_{f_1} \rel
    X + Y$ and
    $C_{f_0} \simeq C_{f_1} \rel
    Y + \text{vertex}$.
\end{theorem}


To show this, one needs the following basic topological
proposition:
\begin{proposition}[] \label{prop:92031999}
    If $f \colon X \to Y$ is a quotient map and
    $K$ is locally compact Hausdorff, then
    $f \times 1 \colon X \times K \to Y \times K$ is
    a quotient map.
\end{proposition}

\begin{proof}[Proof of Theorem \ref{Thm:299221}]
    First, let $F \colon X \times I \to Y$ be the homotopy
    between $f_0$ and $f_1$. Now define $h \colon
    M_{f_0} \to M_{f_1}$ by $h(y) = y$ for
    $y \in Y$ and
    \[
    h\left( x,t \right) = 
    \begin{cases}
        F\left( x,2t \right) ,& t\le \frac{1}{2}\\
        (x, 2t-1),& \frac{1}{2} \le t.
    \end{cases}
    \] 
    Define
    $k \colon M_{f_1} \to M_{f_0}$ likewise by
    the identity on $Y$ nad
    \[
    k\left( x,t \right) =
    \begin{cases}
        F^{-1}\left( x,2t \right) ,& t\le \frac{1}{2}\\
        (x,2t-1),& \frac{1}{2}\le t
    \end{cases}.
    \] 
    Then the composition
    $kh \colon M_{f_0} \to M_{f_1}$ is the identity
    on $Y$ and 
    $F * \left( F^{-1} * E \right) $ on
    the cylinder portion, where $E \colon X \times I \to 
    M_{f_0}$ is induced by the identity on
    $X \times I \to X \times I$.
    This is homotopic to the identity 
    $\rel X \times \left\{ 1 \right\} + Y$.
    Similarly for $hk$.
    In now remains to check the continuity of this homotopy.
    We have a homotopy $M_{f_0} \times I \to 
    M_{f_0}$. We now claim that
    $M_{f_0} \times I \cong M_{f_0 \times I}$. Indeed
    then, using that
    $M_{f_0 \times I} = 
    \frac{X \times I \times I \sqcup Y \times I}{
    \left( (x,0,k) \sim (f_0(x),k \right) }$, it suffices
    to show continuity of the composition
    $X \times I \times I \sqcup Y \times I
    \to M_{f_0} \times I \to M_{f_0}$. 
    For on $Y \times I$, it is the constant homotopy and
    on $X \times I \times I$ it is
    $F * \left( F^{-1} * E \right) \simeq E
    \rel X \times \partial I$. 
    Now, that $M _{f_0} \times I
    \cong M_{f_0 \times I}$ follows from
    Proposition \ref{prop:92031999}.

\end{proof}

Let $f \colon X \to Y$. If $\varphi  \colon Y \to Y'$ is a map,
then there is the induced map
$F \colon M_{f} \to M_{\varphi \circ f}$ induced from
$\varphi $ on $Y$ and the identity on $X \times I$.

\begin{theorem}[]
    If $\varphi  \colon Y \to Y'$ is a homotopy equivalence
    then so is 
    $F \colon \left( M_f , X \right) \to 
    \left( M_{\varphi  \circ f}, X \right) $ and hence
    so is $F \colon C_f \to C_{\varphi  \circ f}$.
\end{theorem}

\begin{proof}
    Let $\psi  \colon Y' \to Y$ be a homotopy inverse
    of $\varphi $ and let $G \colon 
    M_{\varphi \circ f} \to M_{\psi \circ
    \varphi \circ f}$ be the map induced by
    $\psi $ on $Y'$ and the identity on $X \times I$.
    The composition $GF \colon M_f \to M_{\psi \circ \varphi 
    \circ f}$ is induced from $\psi \circ \varphi \colon
    Y \to Y$ and the identity on $X \times I$.
    Let $H \colon Y \times I \to Y$ be a homotopy from
    $\id$ to $\psi \circ \varphi $ ; i.e.,
    $H(y,0) = y$ and $H(y,1) = 
    \psi \varphi (y)$. 
    By the proof of Theorem \ref{Thm:299221}, there is a
    homotopy equivalence
    $h \colon M_f \to M_{\psi \circ \varphi \circ f}\rel X$ given
    by $h(y) = y$ and
     \[
    h(x,t) = 
    \begin{cases}
        H\left( f(x), 2t \right) ,& t\le \frac{1}{2}\\
        (x,2t-1),& t\ge \frac{1}{2}
    \end{cases}.
    \] 
    We claim that
    $h \simeq GF \rel X$. Indeed, the homotopy $H$ can
    be extended to 
    $M_f \times I \to M_{\psi \circ \varphi \circ f}$ by
    putting
    \[
    H\left( (x,s),t \right) 
    =
    \begin{cases}
        H\left( f(x), 2s+t \right) ,& 2s+t \le 1\\
        \left( x, \frac{2s+t-1}{t+1} \right) ,& 2s+t\ge 1
    \end{cases}.
    \] 

    Then $H\left( -,0 \right) = h$ and
    $H\left( -,1 \right) = GF$, so
    since $GF$ is a homotopy equvalence, so is
    $h$.
    Define $F' \colon
    M_{\psi \circ \varphi \circ f} \to 
    M_{\varphi \circ \psi \circ \varphi \circ f}$ 
    as the induced map on mapping cones
    with $\varphi $ on $Y$ and
    the identity on $X \times I$. Then similarly,
    $F' G$ is a homotopy equivalence.\\
    If $k$ is a homotopy inverse of $GF$ then
    $GF k \simeq \id$. If
    $k'$ is a homotopy inverse of $F'G$ then
    $k' F' G \simeq \id$. Thus $G$ has a right
    and left homotopy inverse: $R = Fk$ and
    $L = k'F'$. Then
    $R = \id \circ R \simeq 
    \left( LG \right) R =
    L \left( GR \right) \simeq L \circ \id = L$, so
    $R \simeq L$. That is, 
    $G$ has a homotopy inverse. Therefore,
    $G$ is a homotopy equivalence. Since $G$ and $GF$ are
    homotopy equivalences, so is $F$.
\end{proof}


\begin{problem}[]
    \cite[Ex 14.1]{Bredon} Let $S^2 \cup A$ denote the
    union of the unit $2$-sphere and the line segment
    joining the north and south poles. Show that
    $S^2 \vee S^{1} \simeq
    S^2 \cup A$.
\end{problem}

\begin{proof}
    Define two maps
    $f_0,f_1 \colon \left\{ 0,1 \right\}  \to 
    S^2$ where
    $f_0 (t) = \left( \cos (2\pi t), \sin(2\pi t), 0 \right) $ 
    and $f_1$ is the constant map at $(1,0,0)$. Then
    $f_0 \simeq f_1$, so $C_{f_0} \simeq C_{f_1}$. Now,
    $C_{f_0} = S^2 \cup  A$ while
    $C_{f_1} = S^2 \vee S^{1}$.
\end{proof}

\begin{problem}[]
    \cite[Ex 14.2]{Bredon} 
    Show that the union of a $2$-sphere and a flat
    unit  $2$-cell through the origin is homotopically
    equivalent to the one-point union of two $2$-spheres.
\end{problem}

\begin{proof}
    A $2$-cell is contractible, an
    a $2$-sphere with a $2$-cell inside it is precisely the
    cone of the map
    $S^1 \sqcup S^1 \to S^1$ with the identity on both.
    By \cite[Thm 14.19]{Bredon},
    this is homotopy equivalent to the cone on
    $S^1 \sqcup S^1 \to \left\{ * \right\} $ which is
    $S^2 \vee S^2$.
\end{proof}

\begin{problem}[]
    Show that the union of a standard $2$-torus with two disks,
    one spanning a latitudinal circle and the
    other spanning a longitudinal circle of the torus, is
    homotopically equivalent to a $2$-sphere.
\end{problem}

\begin{proof}
     Using the identification of the torus as the
     quotient space of $I^2$ in the usual way, we can choose
     on spanning circle to be a $2$-cell attached
     along $\left\{ 0 \right\} \times I$ and the
     other to be a $2$-cell attached along
     $I \times \left\{ 0 \right\} $. These are contractible, 
     and the quotient space becomes a $2$-sphere.
\end{proof}


\subsection{Homotopy Groups}

Recall that $\left[ X,A ; Y ,B \right] $ denotes the
set of homotopy classes of maps $X \to Y$ carrying $A$ into
$B$ such that $A$ goes into $B$ during the entire homotopy.

To make a group then, we can select a point $y_0 \in Y$ and
consider the set
\[
\left[ X \times I, X \times \partial I ;
Y , \left\{ y_0 \right\} \right] 
\] 
In this case, the operation of concatenation of homotopies
makes this set into a group.
It is technically also better to choose a basepoint 
$x_0 \in X$ and consider
\[
\left[ X \times I, \left\{ x_0 \right\} \times I
\cup X \times \partial I ; Y , \left\{ y_0 \right\} \right] .
\] 

For the moment, let us set
$A = \left\{ x_0 \right\} \times I \cup 
X \times \partial I$. Then maps
$X \times I \to Y$ which carry $A$ into $\left\{ y_0 \right\} $ 
are in bijective correspondence with maps 
$\left( X \times I \right) / A \to Y$ which take
 the point $\left\{ A \right\} $ into 
 $\left\{ y_0 \right\} $. 
 
 \begin{definition}[Reduced Suspension]
     We define the \textit{reduced suspension} of
     $X$ to be
     \[
     SX = (X \times I) / A =
     \left( X \times I \right) /
     \left( \left\{ x_0 \right\} \times I
     \cup X \times \partial I \right) 
     \] 
 \end{definition}

 The set of homotopy classes of pointed maps
 of a pointed space $X$ to a pointed space $Y$ with
 homotopies preserving the base points will
 be denoted by $\left[ X;Y \right]_* $. 

 Thus
 $\left[ SX;Y \right]_* $ is in canonical bijective
 correspondence with
 $\left[ X \times I, A ; Y , \left\{ y_0 \right\}  \right] $.


 Now, suppose we have pointed maps
 $f,g \colon SX \to Y$. Then they
 induce homotopies
 $f',g' \colon X \times I \to Y$ by precomposing with the
 quotient map
  $X\times I \to SX$. We can then define
  $f' * g' \colon X \times I \to Y$ as usual.
  The resulting pointed map
  $SX \to Y$ will be denoted $f * g$.
  Geometrically, $f * g$ is obtained by
  putting $f$ on the bottom and $g$ on the top
  of the one-point union $SX \vee SX$ and composing
  the resulting map $SX \vee SX \to Y$ with the
  map $SX \to SX \vee SX$ obtained by collapsing the
  middle parameter value $\frac{1}{2}$ copy of
  $X$ in $SX$ to the base point.
  
  \begin{figure}[htpb]
      \centering
      \includegraphics[width=0.6\textwidth]{TKISO0932.png}
      \caption{The product of two map classes
      $SX \to Y$.}
      \label{fig:TKISO0932-png}
  \end{figure}


  For a map $f \colon \left( SX, \left\{ A \right\}  \right) 
  \to \left( Y, \left\{ y_0 \right\}  \right) $, we denote its
  homotopy class in
  $\left[ SX; Y \right]_{*}$ by
  $\left[ f \right] $, and we define
  \[
  \left[ f \right] \left[ g \right] =
  \left[ f*g\right] 
  \] 
  Under this operation, the set
  $\left[ SX;Y \right]_*$ becomes a group.

  \begin{proposition}[]
      The reduced suspension gives
      $S S^{n-1}\cong S^{n}$.
  \end{proposition}

  Thus, we can define $S^{n}$ as the $n$-fold reduced
  suspension of $S^{0}$. As a special case,
  the set $\left[ S^{n};Y \right]_*$ then becomes
  a group for $n>0$. 

  \begin{definition}[$n$ th homotopy group]
      We define
      \[
      \pi_n \left( Y, y_0 \right) =
      \left[ S^{n}; Y \right]_*
      \] 
      with this operation.
  \end{definition}

  \subsubsection{A different way of defining
  $\pi_n \left( Y, y_0 \right) $}
  Note that reduced suspension supplies a parameter in
  $\left[ 0,1 \right] $ and the space
  $S^{n}$ as constructed is the quotient space of
  $I^{n}$ obtained by collapsing the boundary of the cube to a
  point.
  Pointed maps $S^{n}\to Y$ are in bijective correspondence
  with maps $I^{n}\to Y$ taking $\partial I^{n}$ to
  the base point of $Y$. This is a more traditional way
  of defining $\pi_n(Y)$. This becomes the group
  of homotopy classes of maps
  $\left( I^{n},\partial I^{n} \right) \to 
  \left( Y, \left\{ y_0 \right\}  \right) $ with the
  operation being
  \[
  f*g \left( t_1, \ldots, t_n \right) =
  \begin{cases}
      f\left( 2t_1, t_2, \ldots, t_n \right) ,& t_1 \in 
      \left[ 0,\frac{1}{2} \right] \\
      g\left( 2t_1-1, t_2, \ldots, t_n \right) ,& t_1 \in 
      \left[ \frac{1}{2},1 \right] 
  \end{cases}.
  \] 

  \begin{proposition}[]
      For $n\ge 2$, $\pi_n\left( X, x_0 \right)$ is abelian.
  \end{proposition}

  \begin{proof}
      Consider the homotopy in Figure \ref{fig:JIDWOOL0290L-png}.
      We begin by shrinking the domains of $f$ and $g$ to smaller
      subcubes of $I^{n}$, where the region outside is
      mapped to the basepoint. This allows us to move the boxes
      around in a continuous manner. The rest is clear.
      \begin{figure}[htpb]
          \centering
          \includegraphics[width=0.8\textwidth]{JIDWOOL0290L.png}
          \caption{The homotopy in question}
          \label{fig:JIDWOOL0290L-png}
      \end{figure}
  \end{proof}

  Next, we want to show that following:
  \begin{proposition}[]\label{Prop:SwjiaKKDNW1102}
      If $X$ is path-connected, then
      $\pi_n\left( X, x_0 \right) \cong
      \pi_n (X, x_1)$ for any two $x_0,x_1 \in X$.
  \end{proposition}

  For this, we introduce an action of
  $\pi_1$ on $\pi_n$.

  \begin{definition}[The action of $\pi_1$ on $\pi_n$]
      Given a path
      $\gamma \colon I \to X$ from
      $x_0$ to $x_1$, we associate to a map
      $f \colon \left( I^{n}, \partial I^{n} \right) \to 
      \left( X, x_1 \right) $ the map
      $\gamma f \colon \left( I^{n}, \partial I^{n} \right) 
      \to \left( X,x_0 \right) $ by shrinking the domain
      of $f$ to a smaller concentric cube in $I^{n}$, then
      inserting the path $\gamma$ on each radial segment
      in the shell between this smaller cube and $\partial
      I^{n}$.
      See Figure \ref{fig:JDWIXHHX011SJ-png}

      \begin{figure}[htpb]
          \centering
          \includegraphics[width=0.25\textwidth]{JDWIXHHX011SJ.png}
          \caption{Depiction of $\gamma f$.}
          \label{fig:JDWIXHHX011SJ-png}
      \end{figure}

  \begin{note}
      We have the following properties
      \begin{enumerate}
          \item $\gamma \left( f+ g \right) 
              \simeq \gamma f + \gamma g$.
          \item $\left( \gamma \eta \right) f \simeq
              \gamma \left( \eta f \right) $.
          \item $\id f \simeq f$, where
              $\id$ denotes the constant path.
      \end{enumerate}

      To see $(1)$, first deform $f$ and $g$ to be
      constant on the right and left halves of
      $I^{n}$, respectively, producing maps
      which we may call $f+0$ and $0+g$, then we 
      can excise a progressively wider symmetric middle slab
      of $\gamma (f+0) + \gamma(0+g)$ (which can be
      seen on the left in Figure \ref{fig:WIWIWSSK11-png})
      until it becomes $\gamma \left( f+g \right) $ (shown on the
      right).

      \begin{figure}[htpb]
          \centering
          \includegraphics[width=0.8\textwidth]{WIWIWSSK11.png}
          \caption{}
          \label{fig:WIWIWSSK11-png}
      \end{figure}
  \end{note}

  Now if $\beta_{\gamma} \colon \pi_n(X,x_1) \to 
  \pi_n(X, x_0)$ is the change-of-basepoint transformation,
   $\beta_{\gamma}\left[ f \right] =
   \left[ \gamma f \right] $, then
   the above note shows that $\beta_\gamma$ is a group isomorphism.
   This proves Proposition \ref{Prop:SwjiaKKDNW1102}. 
   If we restrict attention to loops
   $\gamma$ at $x_0$, then since $\beta_{\gamma \eta}=
   \beta_{\gamma} \beta_{\eta}$, the map
   $\left[ \gamma \right] \mapsto \beta_{\gamma}$ 
   defines a homomorphism from
   $\pi_1\left( X, x_0 \right) $ to
   $\Aut \left( \pi_n \left( X,x_0 \right)  \right) $ 
   called the \textit{action of $\pi_1$ on $\pi_n$ }.
  \end{definition}

  \begin{note}
  For $n>1$, this action makes
  $\pi_n(X,x_0)$ into a module over the group ring
  $\mathbb{Z}\left[ \pi_1 \left( X,x_0 \right)  \right] $.
  \end{note}  

  \begin{definition}[Simple/abelian spaces]
      A space with trivial $\pi_1$ action on $\pi_n$ is called
      '$n$-simple', and 'simple' means
      ' $n$-simple for all $n$ '. We call
      a space \textit{abelian} if it has
      trivial action of $\pi_1$ on all homotopy groups
      $\pi_n$.
  \end{definition}

  \begin{proposition}[$\pi_n$ is a functor]
      A map $\varphi  \colon \left( X, x_0 \right) \to 
      \left( Y, y_0 \right) $ induces a map
      $\varphi_* \colon \pi_n \left( X, x_0 \right) \to 
      \pi_n \left( Y, y_0 \right) $ defined by
      $\varphi_* \left[ f \right] = \left[ \varphi  f \right] $.
      It is immediate from the definitions that
      $\varphi_*$ is well-defined and a homomorphism
      for $n\ge 1$. The functorial properties are also clear.
  \end{proposition}

  \begin{corollary}
      Homotopy equivalent spaces have isomorphic
      homotopy groups.
  \end{corollary}

  \begin{proposition}[]
      A covering space projection
      $p \colon \left( \tilde{X}, \tilde{x}_0 \right) \to 
      \left( X, x_0 \right) $ induces isomorphisms
      $p_* \colon \pi_n \left( \tilde{X}, \tilde{x}_0 \right) 
      \to \pi_n \left( X, x_0 \right) $ for all
      $n \ge 2$.
  \end{proposition}

  \begin{proof}
      Since 
      $S^{n}$ is path-connected and locally path-connected,
      and simply connected for $n\ge 2$, we find that
      any map
      $\left( S^{n},s_0 \right) 
      \to \left( X, x_0 \right) $ lifts to a 
      map $\left( S^{n},s_0 \right) \to 
      \left( \tilde{X},\tilde{x}_0 \right) $ when
      $n\ge 2$. This gives surjectivity of
      $p_*$.
      For injectivity, suppose
      $p_* \left[ f \right] = \left[ 0 \right] $ where
      $f \colon \left( S^{n}, s_0 \right) \to 
      \left( \tilde{X},\tilde{x}_0 \right) $.
      Let $c_{\tilde{x}_0}$ be the constant map at
      $\tilde{x}_0$. Then
      $p_* \left[ \tilde{x}_0 \right] =
      \left[ 0 \right] $, so by uniqueness of the
      lifting theorem, 
      $\left[ f \right] = \left[ c_{\tilde{x}_0} \right] =
      \left[ 0 \right] $.
  \end{proof}

  \begin{definition}[Aspherical]
      Spaces with $\pi_n = 0$ for all
      $n\ge 2$ are called \textit{aspherical}.
  \end{definition}

  \begin{corollary}
      $S^{1}, T^{n}$ and $K$ are aspherical since
      they have contractible covering spaces.
  \end{corollary}


  \begin{proposition}[]
      \[
      \pi_n \left( \prod_{\alpha} X_{\alpha} \right) 
      \cong \prod_{\alpha} \pi_n \left( X_{\alpha} \right) 
      \] 
  \end{proposition}

  Next we define relative homotopy groups.

  \begin{definition}[Relative homotopy groups]
      Regard $I^{n-1}$ as a face of $I^{n}$ with the last
      coordinate $s_n = 0$ and let
      $J^{n-1}$ be the closure of
      $\partial I^{n}- I^{n-1}$. Then
      we define 
      \[
      \pi_n \left( X, A, x_0 \right) 
      := \left[ I^{n},\partial I^{n}, J^{n-1};
      X , A , x_0\right] 
      \] 
      We shall leave $\pi_0 \left( X, A, x_0 \right) $ undefined
      for now.
  \end{definition}

  We can define a sum operation on $\pi_n \left( X, A, x_0 \right) $ 
  in the same way as for $\pi_n \left( X, x_0 \right) $, except
  now the coordinate $s_n$ now must remain free, so
  we must use one of the other coordinates. Thus
  we must have at least one other coordinate to define
  the same operation. So $\pi_n \left( X, A, x_0 \right) $ is
  a group for $n\ge 2$, and it is abelian for
  $n\ge 3$. For $n=1$, we have
  $I^{1} = \left[ 0,1 \right] , I^{0} = \left\{ 0 \right\} $ 
  and $J^{0} = \left\{ 1 \right\} $, so
  $\pi_1 \left( X, A, x_0 \right) 
  = \left[ I, \left\{ 0 \right\} , \left\{ 1 \right\} ;
  X, A, x_0 \right] $ is the set of homotopy classes of paths in
  $X$ from a varying point in $A$ to the fixed basepoint
  $x_0 \in A$. In general, this is not a group in any
  natural way. \\
  \linebreak
  Now, we saw before that
  $\pi_n \left( X, x_0 \right) $ can be regarded as
  homotopy classes of maps $\left( S^{n}, x_0 \right) \to 
  \left( X, x_0 \right) $. Similarly, collapsing
  $J^{n-1}$ to a point, converts
  $\left( I^{n} , \partial I^{n}, J^{n-1} \right) $ 
  to $\left( D^{n}, S^{n-1}, s_0 \right) $.
  In this case, addition is done by
  the map $c \colon D^{n} \to D^{n} \vee D^{n}$ collapsing
  $D^{n-1} \subset D^{n}$ to a point.\\
  \linebreak
  \begin{theorem}[Compression criterion]\label{Thm:Compression}
      A map $f \colon \left( D^{n}, S^{n-1}, s_0 \right) 
      \to \left( X, A, x_0 \right) $ represents zero
      in $\pi_n \left( X, A, x_0 \right) $ if and only if
      it is homotopic $\rel S^{n-1}$ to a map with image
      contained in $A$.
  \end{theorem}
  
  \begin{proof}
      Suppose we have a homotopy
      $\rel S^{n-1}$ from $f$ to a map
      $g$, so
      $\left[ f \right] = \left[ g \right] $ in
      $\pi_n \left( X, A, x_0 \right) $. 
      Viewing $g$ as a map
      $\left( D^{n}, S^{n-1}, s_0 \right) 
      \to \left( X, A, x_0 \right) $ whose
      image is contained in $A$, we
      can construct the homotopy
      $H \colon D^{n} \times I \to X$ by
      $H(x,t) = g\left( (1-t) x + s_0 t \right) $ 
      which is a homotopy from $g$ to the
      constant map at $x_0$, hence
      $\left[ g \right]  = 0$ in $\pi_n (X, A, x_0)$.\\
      Conversely, if $\left[ f \right] = 0$ via
      a homotopy $F \colon D^{n} \times I \to X$ such that
      $F(x,0) = f(x)$ and
      $F(x,1) = x_0$ for all $x \in D^{n}$ and
      $F(x,t) \in A$ for all
      $x$ with $\left| x \right| = 1$ as well
      as $F(s_0,t) = x_0$ for all $t$. We can
      construct a homotopy
      using $F$ by restricting $F$ to a family of
      $n$-disks in $D^{n} \times I$ starting with
      $D^{n}\times \left\{ 0 \right\} $ and ending
      with the disk $D^{n} \times \left\{ 1 \right\} 
      \cup S^{n-1} \times I$, and where all the disks
      throughout the family have the same boundary.
      See Figure \ref{fig:DJIMMXKXO0O-jpeg} for a depiction
      of this homotopy.

      \begin{figure}[htpb]
          \centering
          \includegraphics[width=0.8\textwidth]{DJIMMXKXO0O.jpeg}
          \caption{}
          \label{fig:DJIMMXKXO0O-jpeg}
      \end{figure}
      This completes the proof.
  \end{proof}

  Next, some things that carry over:
  a map $\varphi \colon \left( X, A, x_0 \right) 
  \to \left( Y, B, y_0 \right) $ induces maps
  $\varphi_* \colon \pi_n \left( X, A, x_0 \right) 
  \to \pi_n \left( Y, B, y_0 \right) $ which are
  homomorphisms when $n\ge 2$ and have properties analogous
  to those in the absolute case: 
  $\left( \varphi \psi  \right)_* = 
  \varphi_* \psi_*, (\id_{(X,A,x_0)})_{*} = \id_{\pi_n (X, A, x_0)}$,
  and if  $\varphi \simeq \psi $ through maps
  $\left( X,A,x_0 \right) \to \left( Y,B,y_0 \right) $,
  then $\varphi_* = \psi_*$. 

  \subsubsection{LES of relative homotopy groups}
  Probably the most useful feature of relative homotopy
  groups $\pi_n (X,A,x_0)$ is that they 
  fit into a long exact sequence
  \[
  \ldots \to \pi_n (A,x_0)
  \stackrel{i_*}{\to} \pi_n(X,x_0)
  \stackrel{j_*}{\to} \pi_n (X,A,x_0)
  \stackrel{\partial}{\to} \pi_{n-1}(A,x_0) \to 
  \ldots \to \pi_0 (X,x_0).
  \] 
  Here $i$ and $j$ are the inclusions
  $\left( A, x_0 \right) \hookrightarrow
  (X,x_0)$ and
  $\left( X, x_0, x_0 \right) \hookrightarrow
  \left( X,A,x_0 \right) $. The map
  $\partial$ comes from restricting maps
  $\left( I^{n},\partial I^{n}, J^{n-1} \right) \to 
  \left( X,A,x_0 \right) $ to
  $I^{n-1}$ (the face of $I^{n}$ with the last
  coordinate $s_n = 0$ ),
  or equivalently, by restricting maps
  $\left( D^{n},S^{n-1},s_0 \right) \to 
  \left( X,A,x_0 \right) $ to $S^{n-1}$. The map $\partial$,
  called the \textit{boundary map}, is a homomorphism
  when $n>1$. In fact, we can show the following theorem

  \begin{theorem}[LES of relative homotopy groups]
      Given 
      $x_0 \in B \subset A \subset X$,
      the sequence of relative homotopy groups
  \[
      \ldots \to 
      \pi_n \left( A,B, x_0 \right) 
      \stackrel{i_*}{\to} 
      \pi_n \left( X, B, x_0 \right) 
      \stackrel{j_*}{\to} 
      \pi_n (X, A, x_0)
      \stackrel{\partial}{\to} 
      \pi_{n-1} \left( A,B, x_0 \right) 
      \to \ldots \to 
      \pi_1 (X,A,x_0)
  \] 
  is exact and natural.
  In the case when $B = \left\{ x_0 \right\} $, we have that
  the LES
  \[
  \ldots \to \pi_n (A,x_0)
  \stackrel{i_*}{\to} \pi_n(X,x_0)
  \stackrel{j_*}{\to} \pi_n (X,A,x_0)
  \stackrel{\partial}{\to} \pi_{n-1}(A,x_0) \to 
  \ldots \to \pi_0 (X,x_0).
  \] 
  is exact and natural.
  \end{theorem}

  \begin{proof}
      \textit{Exactness at $\pi_n
      \left( X,B,x_0 \right) $ :} the composition
      $j_* i_*$ is zero because any
      map $\left( I^{n}, \partial I^{n},
      J^{n-1}\right) \to \left( A,B,x_0 \right) $ 
      is zero in $\pi_n \left( X, A, x_0 \right) $ by the
      compression criterion (Theorem \ref{Thm:Compression}).
      To see that $\ker j_* \subset 
      \im i_*$, let
      $f \colon \left( I^{n}, \partial I^{n},
      J^{n-1}\right) \to \left( X, B, x_0 \right) $ 
      represent zero in $\pi_n \left( X, A, x_0 \right) $.
      Using the compression criterion again, we
      then get that $f$ is homotopic $\rel \partial I^{n}$ 
      to a map with image in $A$, hence the class
      $\left[ f \right] \in \pi_n \left( X, B, x_0 \right) $ 
      is indeed in the image of $i_*$. We conclude that
      $\ker j_* = \im i_*$, obtaining exactness
      at $\pi_n \left( X,B, x_0 \right) $.\\
      \textit{Exactness at $\pi_n (X,A,x_0)$:} 
      for a map  $\left[ f \right] 
      \in \im j_*$, we have that
      $j_*$ maps $\partial I^{n}$ into $B$, hence
      in particular $I^{n-1} \subset \partial I^{n}$ into
      $B$, so $\partial j_* \left[ f \right] $ 
      represents a homotopy class
      in $\pi_{n-1}\left( A,B,x_0 \right) $ with
      image in $B$, but then by the compression criterion,
      $\partial j_* \left[ f \right] = 0$ in
      $\pi_{n-1} \left( A,B,x_0 \right) $, so
      $\im j_* \subset \ker \partial $.
      Conversely, suppose
      $\partial \left[ f \right] = 0$. By the compression
      criterion, representatives of $\partial \left[ f \right] $
      are homotopic $\rel \partial I^{n-1}$ to a map
      with image in $B$. In particular,
      $f|_{I^{n-1}}$ is homotopic to a map with
      image in $B$ via a homotopy $F \colon
      I^{n-1} \times I \to A \rel \partial I^{n-1}$.
      We can tack $F$ onto $f$ to get a new map
      $\left( I^{n}, \partial I^{n},
      J^{n-1}\right) \to 
      \left( X, B, x_0 \right) $ which, as
      a map
      $\left( I^{n}, \partial I^{n}, J^{n-1} \right) \to 
      \left( X, A, x_0 \right) $ is homotopic to
      $f$ by the homotopy that tacks on increasingly longer
      initial segments of $F$. See Figure
      \ref{fig:IDIWKAKX-png}. Hence
      $\left[ f \right] \in \im j_*$.

      \begin{figure}[htpb]
          \centering
          \includegraphics[width=0.2\textwidth]{IDIWKAKX.png}
          \caption{}
          \label{fig:IDIWKAKX-png}
      \end{figure}

      \textit{Exactness at $\pi_n (A,B,x_0)$ :} 
      First, $i_* \partial$ is zero since
      the restriction of a map
      $f \colon \left( I^{n+1}, \partial I^{n+1},
      J^{n}\right) \to \left( X,A,x_0 \right) $ 
      to $I^{n}$ is homotopic $\rel \partial I^{n}$ to a 
      constant map via $f$ itself (a similar picture
      to Figure \ref{fig:DJIMMXKXO0O-jpeg} works).\\
      Conversely, if $B$ is a point, then
      a nullhomotopy $f_t \colon
      \left( I^{n}, \partial I^{n} \right) 
      \to \left( X, x_0 \right) $ of
      $f_0 \colon \left( I^{n},\partial I^{n} \right) 
      \to \left( A,x_0 \right) $ gives a map
      $F \colon \left( I^{n+1},\partial I^{n+1},J^{n} \right) 
      \to \left( X,A,x_0 \right) $ with
      $\partial \left( \left[ F \right]  \right) 
      = \left[ f_0 \right] $. So in this case, the proof is
      finished.
      For a general $B$, let
      $F$ be a nullhomotopy of
      $f \colon \left( I^{n},\partial I^{n},J^{n-1} \right) 
      \to \left( A,B,x_0 \right) $ through maps
      $\left( I^{n}, \partial I^{n}, J^{n-1} \right) 
      \to \left( X,B,x_0 \right) $ and
      let $g$ be the restriction of
      $F$ to $I^{n-1}$ in $I^{n-1} \times I = I^{n}$ (see
      the first of the pictures in
      Figure \ref{fig:USIIOOQ-png}).
      Next reparametrize the $n$ th and
      $(n+1)$ st coordinates as in the
      second picture. Then 
       we find that $f$ with $g$ tacked on
       is in the image of $\partial$. But
       as before, tacking $g$ onto $f$ gives the
       same element of $\pi_n (A,B,x_0)$

      \begin{figure}[htpb]
          \centering
          \includegraphics[width=0.4\textwidth]{USIIOOQ.png}
          \caption{}
          \label{fig:USIIOOQ-png}
      \end{figure}
  \end{proof}

  
\begin{corollary}
    Consider the inclusion
    $\iota \colon X = X \times \left\{ 0 \right\} 
    \hookrightarrow CX$.
    Then
    $\pi_n \left( CX, X, x_0 \right) 
    \cong \pi_{n-1}\left( X, x_0 \right) $ for all
    $n\ge 1$. Taking
    $n=2$, we can thus realize an group $G$, abelian
    or not, as a relative $\pi_2$ by
    choosing $X$ to have $\pi_1 (X) \cong G$.
\end{corollary}

There are also change-of-basepoint isomorphisms
$\beta_{\gamma}$ for relative homotopy groups.
One takes a path  $\gamma$ in $A \subset X$ from
$x_0$ to $x_1$ which induces
$\beta_{\gamma} \colon \pi_n (X,A,x_1) \to 
\pi_n (X,A,x_0)$ by setting
$\beta_{\gamma} \left( \left[ f \right]  \right) 
= \left[ \gamma f \right] $, where
$\gamma f$ is depicted in 
Figure \ref{fig:DIWIOA-png}.

\begin{figure}[htpb]
    \centering
    \includegraphics[width=0.2\textwidth]{DIWIOA.png}
    \caption{}
    \label{fig:DIWIOA-png}
\end{figure}

Restricting to loops at the
basepoint, the association $\gamma \mapsto 
\beta_{\gamma}$ defines an action
of $\pi_1 \left( A, x_0 \right) $ on
$\pi_n \left( X, A, x_0 \right) $ analogous to the
action of $\pi_1 \left( X, x_0 \right) $ on
$\pi_n (X,x_0)$.








\include{psets/pset1/pset1}

\begin{problem}[$n$-connected in the relative case]\label{n-connected-relative}
    The following four conditions are equivalent for
    $i>0$ :
    \begin{enumerate}
        \item Every map
            $\left( D^{i} , \partial D^{i} \right) \to 
            \left( X,A \right) $ is homotopic
            $\rel \partial D^{i}$ to a map $D^{i} \to A$.
        \item Every map $\left( D^{i},\partial D^{i} \right) 
            \to (X,A)$ is homotopic through such maps
            to a map $D^{i} \to A$.
        \item Every map $\left( D^{i}, \partial D^{i} \right) 
            \to \left( X,A \right) $ is homotopic through such
            maps to a constant map $D^{i} \to A$.
        \item $\pi_i \left( X, A, x_0 \right) = 0$ for all
            $x_0 \in A$.
    \end{enumerate}
    When $i = 0$, we did not define the relative $\pi_0$,
    and (1)-(3) are each equivalent to saying that
    each path-component of $X$ contains points
    in $A$ since $D^{0}$ is a point and
    $\partial D^{0}$ is empty. The pair
    $\left( X, A \right) $ is called \textit{$n$-connected}
    if (1)-(4) hold for $0<i\le n$ and
    (1)-(3) hold for  $i=0$.
\end{problem}


\subsection{Whitehead's Theorem}

\begin{theorem}[Whitehead's Theorem]\label{Thm:Whitehead}
    If a map $f \colon X \to Y$ between connected
    $CW$ complexes induces isomorphisms
    $f_* \colon \pi_n (X) \to \pi_n (Y)$ for all
    $n$, then $f$ is a homotopy equivalence.
    In case $f$ is the inclusion of a subcomplex
    $X \hookrightarrow Y$, the conclusion is stronger:
    $X$ is a deformation retract of $Y$.
\end{theorem}

The proof will require the following lemma:

\begin{lemma}[Compression Lemma]
    Let $\left( X,A \right) $ be a CW pair and let
    $\left( Y, B \right) $ be any pair with
    $B \neq \varnothing$. For each  $n$ such that
    $X - A$ has cells of dimension $n$, assume
    that  $\pi_n \left( Y, B, y_0 \right) = 0$ for
    all $y_0 \in B$. Then every map $f \colon
    \left( X,A \right) \to \left( Y,B \right) $ is homotopic
    $\rel A$ to a map $X \to B$.
    When $n = 0$, the condition that
    $\pi_n \left( Y,B,y_0 \right) =0$ for all
    $y_0 \in B$ is to be regarded as saying that
    $\left( Y,B \right) $ is $0$-connected.
\end{lemma}

\begin{proof}[Proof of lemma]
    Assume inductively that $f$ has already been
    homotoped to take the skeleton
    $X^{k-1}$ to $B$. Let
    $\Phi$ be the caracteristic (attaching) map of 
    cell $e^{k}$ of $X - A$. Then the composition
    $f \Phi \colon \left( D^{k} , \partial D^{k} \right) 
    \to \left( Y,B \right) $ is in some class
    in $\pi_k \left( Y, B, y_0 \right) = 0$, so
    it can be homotoped into $B \rel \partial D^{k}$ by
    the compression criterion when
    $k > 0$, or
    by $\left( Y,B \right) $ being $0$-connected for
    $k = 0$ (this is condition (3) in Problem \ref{n-connected-relative}).
    This homotopy of $f \Phi$ induces a homotopy
    $\rel X^{k-1}$ on the quotient space
    $X^{k-1} \cup e^{k}$ of $X^{k-1} \sqcup D^{k}$.
    Doing this for all $k$-cells of $X-A$ simultaneously, and
    taking the constant homotopy on $A$, we obtain a
    homotopy of $f|_{X^{k} \cup A}$ to a map into
    $B$. Since the inclusion of a
    subcomplex into a CW-complex is a cofibration,
    $f|_{X^{k} \cup  A}$ extends to all of $X$ (essentially
    the homotopy extension property).
    This completes the inductive step in the finite dimensional
    CW-complex case.
    In the general case, we perform the
    homotopy of the inductive step during the
    $t$-interval $\left[ 1- \frac{1}{2^{k}},
    1- \frac{1}{2^{k+1}}\right] $. Any finite skeleton
    $X^{k}$ is eventually stationary under these
    homotopies, hence we have a well-defined
    homotopy $f_t, t \in \left[ 0,1 \right] $ with
    $f_1 (X) \subset B$.
\end{proof}


\begin{proof}[Proof of Whitehead's Theorem, \ref{Thm:Whitehead}]
    Let's tackle the case when $f$ is the inclusion
    of a subcomplex first. Consider then
    the LES of the pair $\left( Y,X \right) $. Since
    $f$ by assumption induces isomorphisms
    on all homotopy groups,
    $f_* \colon \pi_* (X) \to \pi_* (Y)$, the
    relative homotopy groups
    $\pi_* (Y,X)$ are zero. Applying the lemma now
    to the identity map $\left( Y,X \right) \to 
    \left( Y,X \right) $, we obtain a homotopy
    of the identity  $\id \colon Y \to Y$ to
    a map $Y \to X$ which is relative to
    $X$. That is, we obtain a deformation retract of
    $Y$ onto $X$.\\
    \linebreak
    For the general case, recall that
    a map $f \colon X \to Y$, can be considered
    as the composition of the
    inclusion $X \hookrightarrow M_f$ and the
    retraction $M_f \to Y$. Since
    the retraction is a homotopy equivalence,
    it suffices to show that $M_f$ deformation retracts
    onto $X$ if $f$ induces isomorphisms on homotopy
    groups, or equivalently, if the relative groups
    $\pi_n \left( M_f, X \right) $ are all zero (since
    $M_f \simeq Y$ ).
    If $f$ is cellular - i.e., takes the $n$-skeleton of
    $X$ to the $n$-skeleton of $Y$ for all
    $n$ - then $\left( M_f, X \right) $ is a CW pair and
    we can apply the first paragraph of the proof.\\
    If $f$ is not cellular, we can either apply
    Theorem 4.8 in \cite{Hatcher} which says
    that $f$ is homotopic to a cellular map, or we can use
    the following argument.

    First, using that 
    $\pi_n \left( M_f, X \right) = 0$ for all $n$, 
    apply the Compression Lemma to
    the inclusion $ \left( X \cup  Y, X \right) 
    \hookrightarrow \left( M_f, X \right) $ to
    obtain a homotopy of the
    inclusion to a map into $X \rel X$.
    The inclusion $X \cup Y \hookrightarrow M_f$ can be
    seen to be a cofibration using 
    Theorem \ref{Thm:SJJDHW29WW}, so
    the pair $\left( M_f, X \cup Y \right) $ satisfies the
    homotopy extension property. So the
    homotopy in question extends to a homotopy
    from the identity of $M_f$ to a
    map $g \colon M_f \to M_f$ taking 
    $X \cup Y$ into $X \rel X$. Applying the
    Compression lemma again to the
    composition
     \[
         \left( X \times I \sqcup Y,
         X \times \partial I \sqcup Y\right) 
         \to \left( M_f, X \cup Y \right) 
         \stackrel{g}{\to} \left( M_f,X \right) ,
    \]
    we get a deformation retraction of
    $M_f$ onto $X$.
\end{proof}





\newpage

\printbibliography
\end{document}
