\documentclass[reqno]{amsart}
\usepackage{amscd, amssymb, amsmath, amsthm}
\usepackage{graphicx}
\usepackage[colorlinks=true,linkcolor=blue]{hyperref}
\usepackage[utf8]{inputenc}
\usepackage[T1]{fontenc}
\usepackage{textcomp}
\usepackage{babel}
%% for identity function 1:
\usepackage{bbm}
%%For category theory diagrams:
\usepackage{tikz-cd}

%\usepackage[backend=biber]{biblatex}
%\addbibresource{.bib}


\setlength\parindent{0pt}

\pdfsuppresswarningpagegroup=1

\newtheorem{theorem}{Theorem}[section]
\newtheorem{lemma}[theorem]{Lemma}
\newtheorem{proposition}[theorem]{Proposition}
\newtheorem{corollary}[theorem]{Corollary}
\newtheorem{conjecture}[theorem]{Conjecture}

\theoremstyle{definition}
\newtheorem{definition}[theorem]{Definition}
\newtheorem{example}[theorem]{Example}
\newtheorem{exercise}[theorem]{Exercise}
\newtheorem{problem}[theorem]{Problem}
\newtheorem{question}[theorem]{Question}

\theoremstyle{remark}
\newtheorem*{remark}{Remark}
\newtheorem*{note}{Note}
\newtheorem*{solution}{Solution}



%Inequalities
\newcommand{\cycsum}{\sum_{\mathrm{cyc}}}
\newcommand{\symsum}{\sum_{\mathrm{sym}}}
\newcommand{\cycprod}{\prod_{\mathrm{cyc}}}
\newcommand{\symprod}{\prod_{\mathrm{sym}}}

%Linear Algebra

\DeclareMathOperator{\Span}{span}
\DeclareMathOperator{\im}{im}
\DeclareMathOperator{\diag}{diag}
\DeclareMathOperator{\Ker}{Ker}
\DeclareMathOperator{\ob}{ob}
\DeclareMathOperator{\Hom}{Hom}
\DeclareMathOperator{\Mor}{Mor}
\DeclareMathOperator{\sk}{sk}
\DeclareMathOperator{\Vect}{Vect}
\DeclareMathOperator{\Set}{Set}
\DeclareMathOperator{\Group}{Group}
\DeclareMathOperator{\Ring}{Ring}
\DeclareMathOperator{\Ab}{Ab}
\DeclareMathOperator{\Top}{Top}
\DeclareMathOperator{\hTop}{hTop}
\DeclareMathOperator{\Htpy}{Htpy}
\DeclareMathOperator{\Cat}{Cat}
\DeclareMathOperator{\CAT}{CAT}
\DeclareMathOperator{\Cone}{Cone}
\DeclareMathOperator{\dom}{dom}
\DeclareMathOperator{\cod}{cod}
\DeclareMathOperator{\Aut}{Aut}
\DeclareMathOperator{\Mat}{Mat}
\DeclareMathOperator{\Fin}{Fin}
\DeclareMathOperator{\rel}{rel}
\DeclareMathOperator{\Int}{Int}
\DeclareMathOperator{\sgn}{sgn}
\DeclareMathOperator{\Homeo}{Homeo}
\DeclareMathOperator{\SHomeo}{SHomeo}
\DeclareMathOperator{\PSL}{PSL}
\DeclareMathOperator{\Bil}{Bil}
\DeclareMathOperator{\Sym}{Sym}
\DeclareMathOperator{\Skew}{Skew}
\DeclareMathOperator{\Alt}{Alt}
\DeclareMathOperator{\Quad}{Quad}
\DeclareMathOperator{\Sin}{Sin}
\DeclareMathOperator{\Supp}{Supp}
\DeclareMathOperator{\Char}{char}
\DeclareMathOperator{\Teich}{Teich}
\DeclareMathOperator{\GL}{GL}
\DeclareMathOperator{\tr}{tr}
\DeclareMathOperator{\codim}{codim}
\DeclareMathOperator{\coker}{coker}
\DeclareMathOperator{\corank}{corank}
\DeclareMathOperator{\rank}{rank}
\DeclareMathOperator{\Diff}{Diff}
\DeclareMathOperator{\Bun}{Bun}
\DeclareMathOperator{\Sm}{Sm}
\DeclareMathOperator{\Fr}{Fr}
\DeclareMathOperator{\Cob}{Cob}
\DeclareMathOperator{\Ext}{Ext}
\DeclareMathOperator{\Tor}{Tor}
\DeclareMathOperator{\Conf}{Conf}
\DeclareMathOperator{\UConf}{UConf}



%Row operations
\newcommand{\elem}[1]{% elementary operations
\xrightarrow{\substack{#1}}%
}

\newcommand{\lelem}[1]{% elementary operations (left alignment)
\xrightarrow{\begin{subarray}{l}#1\end{subarray}}%
}

%SS
\DeclareMathOperator{\supp}{supp}
\DeclareMathOperator{\Var}{Var}

%NT
\DeclareMathOperator{\ord}{ord}

%Alg
\DeclareMathOperator{\Rad}{Rad}
\DeclareMathOperator{\Jac}{Jac}

%Misc
\newcommand{\SL}{{\mathrm{SL}}}
\newcommand{\mobgp}{{\mathrm{PSL}_2(\mathbb{C})}}
\newcommand{\id}{{\mathrm{id}}}
\newcommand{\MCG}{{\mathrm{MCG}}}
\newcommand{\PMCG}{{\mathrm{PMCG}}}
\newcommand{\SMCG}{{\mathrm{SMCG}}}
\newcommand{\ud}{{\mathrm{d}}}
\newcommand{\Vol}{{\mathrm{Vol}}}
\newcommand{\Area}{{\mathrm{Area}}}
\newcommand{\diam}{{\mathrm{diam}}}
\newcommand{\End}{{\mathrm{End}}}


\newcommand{\reg}{{\mathtt{reg}}}
\newcommand{\geo}{{\mathtt{geo}}}

\newcommand{\tori}{{\mathcal{T}}}
\newcommand{\cpn}{{\mathtt{c}}}
\newcommand{\pat}{{\mathtt{p}}}

\let\Cap\undefined
\newcommand{\Cap}{{\mathcal{C}}ap}
\newcommand{\Push}{{\mathcal{P}}ush}
\newcommand{\Forget}{{\mathcal{F}}orget}



\title{Assignment 8}
\author{Jonas Trepiakas}
\date{}

\begin{document}
\maketitle
    Recall that the cohomology with integral coefficients
    of $K(\mathbb{Z} /3,2)$ up to degree $6$ is:
    \[
    H^{*}\left( K(\mathbb{Z}/3,2); \mathbb{Z} \right) 
    \cong
    \begin{cases}
        \mathbb{Z},& * = 0,\\
        \mathbb{Z} /3,& *=3,5,\\
        0,& * = 1,2,4,6.
    \end{cases}
    \] 
    With $\mathbb{F}_3$-coefficients we have
    \[
    H^{*}\left( K\left( \mathbb{Z}/3, 2 \right) ;
    \mathbb{F}_3 \right) \cong \mathbb{F}_3
    \left<1, a,b,a^2, ab \right> \quad \text{for } * \le 5
    \] 
    where
    $\left| a \right| =2$ and $\left| b \right| =3$.


    \begin{problem}[]
        The goal is to classify up to homotopy equivalence all
        CW complexes $X$ with the following properties:
        \begin{enumerate}
            \item $X$ has a single cell in each
                dimension $0,3,4,6$, and no cells in
                other dimensions.
            \item $H_3(X) \cong \mathbb{Z} /3$.
        \end{enumerate}
        Let $Y = X^{(4)}$ be the $4$-skeleton of $X$.
        \begin{enumerate}
            \item Show that $Y$ is uniquely determined up to
                homotopy equivalence.
            \item Compute the $E_4$-page of the cohomology
                spectral sequence (up to degree $6$ ) for the
                fiber sequence
                \[
                K\left( \mathbb{Z}/3, 2 \right) \to 
                \tau_{>3}Y \to Y.
                \] 
                Assuming the differential 
                $d_3 \colon E_3^{0,5} \to E_{3}^{3,3}$ is non-trivial,
                show that
                \[
                \pi_k(Y) \cong
                \begin{cases}
                    0,& *=0,1,2,4,5\\
                    \mathbb{Z} /3,& *=3
                \end{cases}
                \] 
                and that $\pi_6(Y)$ has at least three elements.
            \item Redo step 3 with $\mathbb{F}_3$-coefficients.
                Deduce that the $d_3$-differential in step 2
                must indeed have been non-trivial.
            \item Show that $X$ is unique up to homotopy.
        \end{enumerate}
    \end{problem}

    
    \begin{solution}
        (1) The $\Delta$-chain complex for $Y$ has the form
        \[
        \ldots \to 0 \to \mathbb{Z} \stackrel{a}{\to} \mathbb{Z}
        \to  0 \to  0 \to \mathbb{Z} \to  0 \to  \ldots
        \] 
        where $\coker a = \mathbb{Z}/3$, hence
        $a$ must be multiplication by $3$.
        But then taking homology, we obtain
        \[
       \ldots \to 
       0 \to 0 \to \mathbb{Z} /3 \to 0 \to  0 \to \mathbb{Z} \to 0
       \to \ldots
        \] 
        I.e.,
        \[
        \tilde{H}_k(Y) \cong
        \begin{cases}
            \mathbb{Z}/3,& k=3\\
            0,& k\neq 3
        \end{cases}
        \] 
        so $Y$ is a Moore space
        $M(\mathbb{Z}/3, 3)$.
        We have seen, (Hatcher, example 4.34), that
        Moore spaces are unique up to homotopy equivalent from
        which the claim follows.\\
        \linebreak
        (2) 
        Firstly, since
        $X$ and hence $Y$ only have
        cells in dimension $0$ and then
        $>1$, we obtain by the cellular approximation theorem
        that $\pi_1(Y) \cong \pi_1(X) \cong 0$, so
        that $\pi_1$ acts trivially on homology.
        Hence we can use the LSSS.
        By the UCT, 
        \[
        \tilde{H}^{k}(Y)
        \cong
        \begin{cases}
            \mathbb{Z}/3,& k=4\\
            0,& k\neq 4
        \end{cases}
        \] 
        so we obtain a double complex as follows:



\[\begin{tikzcd}
	& \vdots \\
	5 & {\mathbb{Z}/3} && {\mathbb{Z}/3} & {\mathbb{Z}/3} \\
	\\
	3 & {\mathbb{Z}/3} && {\mathbb{Z}/3} & {\mathbb{Z}/3} \\
	\\
	& {\mathbb{Z}} &&& {\mathbb{Z}/3} & \cdots \\
	&&& 3 & 4
	\arrow[no head, from=2-2, to=1-2]
	\arrow[dashed, from=2-2, to=4-4]
	\arrow[no head, from=4-2, to=2-2]
	\arrow[dashed, from=4-2, to=6-5]
	\arrow[no head, from=6-2, to=4-2]
	\arrow[no head, from=6-2, to=6-5]
	\arrow[no head, from=6-5, to=6-6]
\end{tikzcd}\]
where
$H^{3}\left( Y; H^{k}\left( K\left( \mathbb{Z}/3,2 \right) ;
\mathbb{Z} \right)  \right) 
\cong \Hom \left( H_3(Y) ,
H^{k} \left( K\left( \mathbb{Z}/3,2 \right)  \right) \right) \cong
\Hom \left( \mathbb{Z}/3, \mathbb{Z}/3 \right) 
\cong \mathbb{Z}/3$ for $k=3,5$ from the UCT and
$H^{4}\left( Y; H^{k}\left( 
K\left( \mathbb{Z}/3,2 \right) ; \mathbb{Z}\right)  \right) 
\cong \Ext \left( H_{3}(Y),
H^{k}\left( K\left( \mathbb{Z}/3,2 \right) ;\mathbb{Z}
\right) \right) \cong
\Ext \left( \mathbb{Z}/3, \mathbb{Z}/3 \right) 
\cong \mathbb{Z}/3$ again from the UCT.\\

Since there can only be trivial maps
in $E_2$, the same double complex forms
$E_3$ where we also obtain the topmost indicated dashed
map as the only possible nontrivial map.

We assumed that this map is nontrivial, hence
must be an isomorphism, so the $E_4$ page will
be as follows:

\[\begin{tikzcd}
	& \vdots \\
	5 &&& {\mathbb{Z}/3} & {\mathbb{Z}/3} \\
	\\
	3 & {\mathbb{Z}/3} &&& {\mathbb{Z}/3} \\
	\\
	& {\mathbb{Z}} &&& {\mathbb{Z}/3} & \cdots \\
	&&& 3 & 4
	\arrow[no head, from=4-2, to=1-2]
	\arrow[dashed, from=4-2, to=6-5]
	\arrow[no head, from=6-2, to=4-2]
	\arrow[no head, from=6-2, to=6-5]
	\arrow[no head, from=6-5, to=6-6]
\end{tikzcd}\]
Since $H^{3}\left( \tau_{>3}Y;\mathbb{Z} \right) 
\cong \Hom \left( H_3(\tau_{>3}Y), \mathbb{Z} \right) 
\cong 0$ by UCT, we must have that the indicated map is
injective, hence an isomorphism. Thus
$E^{4}$ will look as follows:

\[\begin{tikzcd}
	& \vdots \\
	5 &&& {\mathbb{Z}/3} & {\mathbb{Z}/3} \\
	\\
	3 &&&& {\mathbb{Z}/3} \\
	\\
	& {\mathbb{Z}} &&&& \cdots \\
	&&& 3 & 4
	\arrow[no head, from=6-2, to=1-2]
	\arrow[no head, from=6-2, to=6-6]
\end{tikzcd}\]
From this, we can read off that
$H^{7}\left( \tau_{>3}Y; \mathbb{Z} \right) 
\cong \mathbb{Z}/3$. 
Now,
recall that $\tau_{>3}Y$ is a CW-complex, hence
its homology and cohomology groups are
finitely generated abelian, so we can
use the structure theorem for
finitely generated abelian groups and analyse
its torsion part and torsion free part, into which it
separates by a direct product. Now
\[
0 \to \Ext \left(H_6 (\tau_{>3}Y), \mathbb{Z} \right) 
\to \underbrace{H^{7}\left( \tau_{>3}Y;\mathbb{Z} \right)}_{\cong
\mathbb{Z}/3} \to 
\Hom \left( H_7\left( \tau_{>3}Y \right) ,\mathbb{Z} \right) \to 
0
\] 
and
$\Hom \left( H_7 \left( \tau_{>3}Y \right) ,\mathbb{Z} \right) $ 
is the torsion-free part of
$H_7 \left( \tau_{>3}Y \right) $, so
since $\mathbb{Z}/3$ surjects onto this part, it
must be $0$. Thus
$\Ext\left( H_6\left( \tau_{>3}Y \right) ,\mathbb{Z} \right) 
\cong H^{7}\left( \tau_{>3}Y;\mathbb{Z} \right) 
\cong \mathbb{Z}/3$, but
$\Ext \left( H_6 \left( \tau_{>3}Y \right) ,\mathbb{Z} \right) $ 
is the torsion part of $H_6 \left( \tau_{>3}Y \right) $, so
$H_6 \left( \tau_{>3}Y \right) 
\cong \mathbb{Z}/3 \bigoplus \left( \text{free part} \right) $,
where the free part could be trivial. In any case,
$H_6 \left( \tau_{>3}Y \right) $ is nontrivial,
while all
$H_k \left( \tau_{>3}Y \right) \cong 0$ for 
$1\le k \le 5$ by the UCT. Hence
by Hurewicz,  
\[
\pi_k \left( \tau_{>3}Y \right) 
\cong
\begin{cases}
    \mathbb{Z}/3 \bigoplus \left( \text{free part} \right),& k=6\\
    0,& 1\le k\le 5
\end{cases}
\] 
Combining this with
$\pi_k \left( Y \right) $ for
$k \le 3$ which by Hurewicz and the previous problem is
\[
\pi_k(Y) \cong
\begin{cases}
    \mathbb{Z}/3,& k=3\\
    0,& 0\le k\le 2
\end{cases}
\] 
we obtain
\[
\pi_k(Y) \cong
\begin{cases}
    \mathbb{Z}/3 \bigoplus \left( \text{free part} \right),& k=6\\
    \mathbb{Z}/3,& k=3\\
    0,& k=0,1,2,4,5
\end{cases}
\] 


(3) By UCT,
\[
0 \to \Ext \left( H_{n-1}(Y), \mathbb{Z}/3 \right) 
\to H^{n}(Y;\mathbb{Z}/3) \to 
\Hom \left( H_n (Y), \mathbb{Z}/3 \right) \to 0
\] 
so when $n=3$,
$\Ext$ vanishes, so
$H^{3}(Y;\mathbb{Z}/3) \cong
\Hom \left( \mathbb{Z}/3, \mathbb{Z}/3 \right) 
\cong \mathbb{Z}/3$ and
when $n=4$,
$\Hom$ vanishes, so
$H^{4}(Y;\mathbb{Z}/3) \cong
\Ext \left( H_3(Y),\mathbb{Z}/3 \right) 
\cong \mathbb{Z}/3$.
Furthermore,
$H^{0}(Y;\mathbb{Z}/3) \cong
\mathbb{Z}/3$. In all
other columns, $H^{p}$ vanishes.

So consider the double complex
\[\begin{tikzcd}
	& \vdots \\
	5 & {\mathbb{Z}/3 ab} &&& {\mathbb{Z}/3abx} & {\mathbb{Z}/3aby} \\
	4 & {\mathbb{Z}/3 a^2} &&& {\mathbb{Z}/3a^2x} & {\mathbb{Z}/3a^2y} \\
	3 & {\mathbb{Z}/3b} &&& {\mathbb{Z}/3bx} & {\mathbb{Z}/3by} \\
	2 & {\mathbb{Z}/3 a} &&& {\mathbb{Z}/3 ax} & {\mathbb{Z}/3ay} \\
	\\
	& {\mathbb{Z}/3} &&& {\mathbb{Z}/3 x} & {\mathbb{Z}/3y} & \cdots \\
	&&&& 3 & 4
	\arrow[no head, from=2-2, to=1-2]
	\arrow["{d_3}"{description}, from=2-2, to=4-5]
	\arrow["{d_4}"{description}, from=2-2, to=5-6]
	\arrow[no head, from=3-2, to=2-2]
	\arrow[from=3-2, to=5-5]
	\arrow[no head, from=4-2, to=3-2]
	\arrow["\cong"{description}, from=4-2, to=7-6]
	\arrow[no head, from=5-2, to=4-2]
	\arrow["\cong"{description}, from=5-2, to=7-5]
	\arrow[no head, from=7-2, to=5-2]
	\arrow[no head, from=7-2, to=7-5]
	\arrow[no head, from=7-5, to=7-6]
	\arrow[no head, from=7-6, to=7-7]
\end{tikzcd}\]
Since
$H^{p}(\tau_{>3}Y; \mathbb{Z}/3) \cong 0$ for
$p = 2,3$, we must have that the
maps eminating from
$\mathbb{Z} /3a$ and $\mathbb{Z}/3b$ are injective, hence
they must be isomorphisms since any
nontrivial group homomorphism
$\mathbb{Z}/3 \to \mathbb{Z}/3$ is an isomorphism.
Thus $d_3(a) = x$ and
$d_4(b) = y$. These are both forced since all other maps eminating
from these groups or terminating at them are $0$-maps.
Now, using the multiplicative structure, we can
calculate $d$ evaluated at the other generators.
Using the Leibniz rule, we get
$d_3 (a^2) = d(a) a + (-1)^{\left| a \right| } a d(a)
= 2ax \in \mathbb{Z}/3 ax
$ which still generates
$\mathbb{Z}/3ax$, hence
$d_3 \colon
\mathbb{Z}/3a^2 \to \mathbb{Z}/3 ax$ is an isomorphism.
Likewise
\[
d_3(ab) = 
x b + a d_3(b) = xb \in \mathbb{Z}/3 xb
\] 
so since $d_3 \colon \mathbb{Z}/3ab \to 
\mathbb{Z}/3 xb$ maps generators to generators,
it is an isomorphism.

Thus, the page $E_4$ looks as follows:

\[\begin{tikzcd}
	& \vdots \\
	5 &&&& {\mathbb{Z}/3abx} & {\mathbb{Z}/3aby} \\
	4 &&&& {\mathbb{Z}/3a^2x} & {\mathbb{Z}/3a^2y} \\
	3 &&&&& {\mathbb{Z}/3by} \\
	2 &&&&& {\mathbb{Z}/3ay} \\
	\\
	& {\mathbb{Z}/3} &&&&& \cdots \\
	&&&& 3 & 4
	\arrow[no head, from=7-2, to=1-2]
	\arrow[no head, from=7-2, to=7-7]
\end{tikzcd}\]

Clearly, on this and all subsequent pages, all
maps are trivial, so
$E_4 = E_{\infty}$, and we get that
$H^{p}\left( \tau_{>3}Y; \mathbb{Z}/3 \right) 
\cong
\begin{cases}
    \mathbb{Z}/3,& p=0,6\\
    0,& 1\le p\le 5
\end{cases}$


Now, suppose that the map
$\mathbb{Z}/3 \to \mathbb{Z}/3$ in the
integral case from part (2) of the problem were
not an isomorphism - i.e., suppose it were trivial.
Then since that group was the only group of total degree
$5$ and since all maps from it or terminating at it on subsequent
pages would be $0$-maps, we would get that
$H^{5}(\tau_{>3}Y;\mathbb{Z}) \cong \mathbb{Z}/3$.
But then by UCT, and the above part of this problem,
$0 = H^{5}\left( \tau_{>3}Y;\mathbb{Z}/3 \right) 
\cong H^{5}\left( \tau_{>3}Y ; \mathbb{Z} \right) \otimes
\mathbb{Z}/3 \cong \mathbb{Z}/3 \otimes \mathbb{Z}/3 
\neq \mathbb{Z}/3 \neq 0$
gives a contradiction. Hence the map
must have been nontrivial, which was what we wanted to show.\\
\linebreak
Now proceeding as in step (2), we can conclude all the same things.\\
\linebreak
(4) 

We will make use of the following proposition a few times:

\begin{proposition}[Hatcher, Proposition 0.18]
    If $\left( X_1,A \right) $ is a CW pair and we have
    attaching maps $f,g \colon
    A \to X_0$ that are homotopic, then
    $X_0 \sqcup_f X_1 \simeq X_0 \sqcup_g X_1 \rel X_0$.
\end{proposition}

Now, a CW complex can be constructed by successively
adding cells
of higher and higher dimension to the previously
constructed skeleton (as described in Hatcher
p. 5), so
suppose $X$ and $X'$ are spaces which have
a single $0,3,4$ and $6$ cell
and such that $H_3(X) \cong H_3(X') \cong \mathbb{Z}/3$.

Since $X$ and $X'$ are CW complexes, cellular homology tells us from
the cellular chain complex that the
attaching maps for each
CW complex of the $4$-cell is a degree $3$ map.\\

Now, to construct $X$ and $X'$, start with a single
$0$-cell. Then we attach
the $3$-cell via the unique attachig map
$S^2 \to *$ where
$*$ is the $0$-cell.

This space is necessarily a $3$-sphere.

The attaching map of the $4$-cell is now a map
$S^3 \to S^3$ which is of degree $3$.
We use the following lemma:
\begin{lemma}[]
    Two maps $f ,g \colon S^{n} \to S^{n}$ have the
    same degree if and only if they are homotopic.
\end{lemma}

\begin{proof}
    If they are homotopic, they have the same degree
    as degree is a construction on homology.

    Conversely, suppose
    $\deg f = \deg g$.
    Recal that the Hurewicz isomorphism is
    natural with the commutative diagram
    \begin{equation*}
    \begin{tikzcd}
        \pi_n (X,x_0) \ar[d, "h"] \ar[r, "\varphi_*"] &
        \pi_n(Y,y_0) \ar[d, "h"] \\
        H_n(X) \ar[r, "\varphi_*"] & H_n(Y)
    \end{tikzcd}
    \end{equation*}
    where $\varphi  \colon X \to Y$ is a map.
    
    
    Now, since $f$ and $g$ might not be
    basepoint preserving maps (once we choose a basepoint),
    we need to fix this issue. One way to do this would
    be for example to compose $f$ and $g$ with a 
    rotation of the sphere such that it maps the
    basepoint to itself - noting that rotations are homotopic
    to the identity, hence also do not change the degrees.
    Thus we can without issue assume that
    $f$ and  $g$
    Using this, we obtain a commutative diagram as follows:

    \begin{equation*}
    \begin{tikzcd}
        \pi_n(S^{n}) \ar[r, "f_*"] \ar[d, "\cong"] &
        \pi_n(S^{n}) \ar[d, "\cong"] \\
        H_n(S^{n}) \ar[r, "f_* = g_*"] & 
        H_n(S^{n}) \\
        \pi_n(S^{n}) \ar[r, "g_*"] \ar[u, "\cong"] &
        \pi_n(S^{n}) \ar[u, "\cong"]
    \end{tikzcd}
    \end{equation*}
    Thus we find that
    $f$ and $g$ induce the same
    maps on $\pi_n$, so
    in particular,
    $\left[ f \right]  =
    f_* \left[ 1 \right] =
    g_* \left[ 1 \right] = \left[ g \right] $, hence
    $f$ and $g$ are homotopic.\\
    \linebreak
    For $n=1$, where we cannot use the Hurewicz map, we can
    use that the abelianization map is an isomorphism and then
    note that $\pi_1(S^{1}) \cong \mathbb{Z}$ is an isomorphism.

\end{proof}

Using this lemma, we can conclude that
the attaching map
for the $4$-cell on $X'$ is homotopic to the
degree $3$-map for the attaching map on $X$, hence using the
Proposition, we conclude that
$X^{(4)} \simeq (X')^{(4)} (\rel X^{(3)} = X^{'(3)})$.
Next, we attach a $6$-cell to each complex.
But note that we showed in part (3) that
$\pi_5 (X) = 0 = \pi_5(X')$, so
suppose $\varphi $ is the attaching map
$S^{5} \to X^{(4)}$ and
$\varphi '$ the attaching map
$S^{5} \to X^{'(4)}$. Then
both a homotopic to a constant map, say
$\varphi \simeq c_{p}$ and
$\varphi ' \simeq c_{p'}$. But also
$\pi_0$ of either space is trivial, so
they are path-connected, hence
$c_{p} \simeq c_{*}$ where
$*$ is the $0$-cell which, in particular, is contained
in the $3$-skeleton of both
$X$ and $X'$. Thus also
$c_{p'} \simeq c_*$. Let
$G \colon X^{(4)} \to (X')^{(4)}$ be the homotopy equivalence
$\rel X^{(3)} = X^{'(3)}$ with homotopy
inverse $H$, so
both $G$ and $H$ restrict to the identity on
the $3$-skeleton. 
Then $G$ and $H$ extend to maps
$\tilde{G} \colon X^{(4)} \cup_{c_* }D^{6} \to
 (X')^{(4)} \cup_{c_*} D^{6}$ and $\tilde{H} \colon
 (X')^{(4)} \cup_{c_*} D^{6}
\to X^{(4)} \cup_{c_*}D^{6}$, respectively, by
being the identity on $D^{6}$.
Since the homotopies
$GH \simeq \id$ and $HG \simeq \id$ are also
the identity on the $3$-skeleton at all times,
we can extends these homotopies
to 
$\tilde{G} \tilde{H} \simeq \id$ and
$\tilde{H} \tilde{G} \simeq \id$.

And now we are done because

\[
X = X^{(4)} \cup_{\varphi }D^{6}
\simeq X^{(4)} \cup_{c_*} D^{6}
\simeq (X')^{(4)} \cup_{c_*} D^{6}
\simeq (X')^{(4)} \cup_{\varphi '} D^{6}
= X'.
\] 




    \end{solution}











    %\printbibliography
\end{document}
