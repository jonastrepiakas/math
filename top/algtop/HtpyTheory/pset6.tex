\documentclass[reqno]{amsart}
\usepackage{amscd, amssymb, amsmath, amsthm}
\usepackage{graphicx}
\usepackage[colorlinks=true,linkcolor=blue]{hyperref}
\usepackage[utf8]{inputenc}
\usepackage[T1]{fontenc}
\usepackage{textcomp}
\usepackage{babel}
%% for identity function 1:
\usepackage{bbm}
%%For category theory diagrams:
\usepackage{tikz-cd}

\usepackage{adjustbox}

%\usepackage[backend=biber]{biblatex}
%\addbibresource{.bib}


\setlength\parindent{0pt}

\pdfsuppresswarningpagegroup=1

\newtheorem{theorem}{Theorem}[section]
\newtheorem{lemma}[theorem]{Lemma}
\newtheorem{proposition}[theorem]{Proposition}
\newtheorem{corollary}[theorem]{Corollary}
\newtheorem{conjecture}[theorem]{Conjecture}

\theoremstyle{definition}
\newtheorem{definition}[theorem]{Definition}
\newtheorem{example}[theorem]{Example}
\newtheorem{exercise}[theorem]{Exercise}
\newtheorem{problem}[theorem]{Problem}
\newtheorem{question}[theorem]{Question}

\theoremstyle{remark}
\newtheorem*{remark}{Remark}
\newtheorem*{note}{Note}
\newtheorem*{solution}{Solution}



%Inequalities
\newcommand{\cycsum}{\sum_{\mathrm{cyc}}}
\newcommand{\symsum}{\sum_{\mathrm{sym}}}
\newcommand{\cycprod}{\prod_{\mathrm{cyc}}}
\newcommand{\symprod}{\prod_{\mathrm{sym}}}

%Linear Algebra

\DeclareMathOperator{\Span}{span}
\DeclareMathOperator{\im}{im}
\DeclareMathOperator{\diag}{diag}
\DeclareMathOperator{\Ker}{Ker}
\DeclareMathOperator{\ob}{ob}
\DeclareMathOperator{\Hom}{Hom}
\DeclareMathOperator{\Mor}{Mor}
\DeclareMathOperator{\sk}{sk}
\DeclareMathOperator{\Vect}{Vect}
\DeclareMathOperator{\Set}{Set}
\DeclareMathOperator{\Group}{Group}
\DeclareMathOperator{\Ring}{Ring}
\DeclareMathOperator{\Ab}{Ab}
\DeclareMathOperator{\Top}{Top}
\DeclareMathOperator{\hTop}{hTop}
\DeclareMathOperator{\Htpy}{Htpy}
\DeclareMathOperator{\Cat}{Cat}
\DeclareMathOperator{\CAT}{CAT}
\DeclareMathOperator{\Cone}{Cone}
\DeclareMathOperator{\dom}{dom}
\DeclareMathOperator{\cod}{cod}
\DeclareMathOperator{\Aut}{Aut}
\DeclareMathOperator{\Mat}{Mat}
\DeclareMathOperator{\Fin}{Fin}
\DeclareMathOperator{\rel}{rel}
\DeclareMathOperator{\Int}{Int}
\DeclareMathOperator{\sgn}{sgn}
\DeclareMathOperator{\Homeo}{Homeo}
\DeclareMathOperator{\SHomeo}{SHomeo}
\DeclareMathOperator{\PSL}{PSL}
\DeclareMathOperator{\Bil}{Bil}
\DeclareMathOperator{\Sym}{Sym}
\DeclareMathOperator{\Skew}{Skew}
\DeclareMathOperator{\Alt}{Alt}
\DeclareMathOperator{\Quad}{Quad}
\DeclareMathOperator{\Sin}{Sin}
\DeclareMathOperator{\Supp}{Supp}
\DeclareMathOperator{\Char}{char}
\DeclareMathOperator{\Teich}{Teich}
\DeclareMathOperator{\GL}{GL}
\DeclareMathOperator{\tr}{tr}
\DeclareMathOperator{\codim}{codim}
\DeclareMathOperator{\coker}{coker}
\DeclareMathOperator{\corank}{corank}
\DeclareMathOperator{\rank}{rank}
\DeclareMathOperator{\Diff}{Diff}
\DeclareMathOperator{\Bun}{Bun}
\DeclareMathOperator{\Sm}{Sm}
\DeclareMathOperator{\Fr}{Fr}
\DeclareMathOperator{\Cob}{Cob}
\DeclareMathOperator{\Ext}{Ext}
\DeclareMathOperator{\Tor}{Tor}
\DeclareMathOperator{\Conf}{Conf}
\DeclareMathOperator{\UConf}{UConf}



%Row operations
\newcommand{\elem}[1]{% elementary operations
\xrightarrow{\substack{#1}}%
}

\newcommand{\lelem}[1]{% elementary operations (left alignment)
\xrightarrow{\begin{subarray}{l}#1\end{subarray}}%
}

%SS
\DeclareMathOperator{\supp}{supp}
\DeclareMathOperator{\Var}{Var}

%NT
\DeclareMathOperator{\ord}{ord}

%Alg
\DeclareMathOperator{\Rad}{Rad}
\DeclareMathOperator{\Jac}{Jac}

%Misc
\newcommand{\SL}{{\mathrm{SL}}}
\newcommand{\mobgp}{{\mathrm{PSL}_2(\mathbb{C})}}
\newcommand{\id}{{\mathrm{id}}}
\newcommand{\MCG}{{\mathrm{MCG}}}
\newcommand{\PMCG}{{\mathrm{PMCG}}}
\newcommand{\SMCG}{{\mathrm{SMCG}}}
\newcommand{\ud}{{\mathrm{d}}}
\newcommand{\Vol}{{\mathrm{Vol}}}
\newcommand{\Area}{{\mathrm{Area}}}
\newcommand{\diam}{{\mathrm{diam}}}
\newcommand{\End}{{\mathrm{End}}}


\newcommand{\reg}{{\mathtt{reg}}}
\newcommand{\geo}{{\mathtt{geo}}}

\newcommand{\tori}{{\mathcal{T}}}
\newcommand{\cpn}{{\mathtt{c}}}
\newcommand{\pat}{{\mathtt{p}}}

\let\Cap\undefined
\newcommand{\Cap}{{\mathcal{C}}ap}
\newcommand{\Push}{{\mathcal{P}}ush}
\newcommand{\Forget}{{\mathcal{F}}orget}




\begin{document}
    Conventions for this assignment: We assume
    all topological spaces to be nice enough for covering
    theory (we can even assume locally contractible).
    Basepoints are assumed to be good basepoints, namely
    the inclusion $\left\{ x \right\} \subset 
    X$ is assumed to have the homotopy extension property.
    If
    $X$ is a space, then $\Omega X$ denotes
    its loop space and there is a fiber sequence
    \[
    \Omega X \to PX \to X
    \] 
    where $PX$ is a contractible space.


    \begin{problem}[]
        Show that the homology of
        $\Omega\left( S^2 \vee S^3 \right) $ is
        \[
        H_* \left( \Omega \left( S^2 \vee S^3 \right) ;
        \mathbb{Z}\right) \cong
        \mathbb{Z}^{F_n}
        \] 
        where $F_n$ is the $n$ th Fibbonaci number
        (We set $F_0 = F_1 = 1$ and
        $F_n = F_{n-1} + F_{n-2}$ ).
    \end{problem}

    \begin{proof}
        Using the fiber sequence
        \[
        \Omega \left( S^2 \vee S^3 \right) 
        \to P \left( S^2 \vee S^3 \right) 
        \to S^2 \vee S^3
        \] 
        we obtain the following quadrant double complex:
        \[\begin{tikzcd}
	\vdots && \vdots & \vdots \\
	{H_3(\Omega (S^2 \vee S^3))} && {H_3(\Omega (S^2 \vee S^3))} & {H_3(\Omega (S^2 \vee S^3))} \\
	{H_2(\Omega (S^2 \vee S^3))} && {H_2(\Omega (S^2 \vee S^3))} & {H_2(\Omega (S^2 \vee S^3))} \\
	{H_1(\Omega (S^2 \vee S^3))} && {H_1(\Omega (S^2 \vee S^3))} & {H_1(\Omega (S^2 \vee S^3))} \\
	\mathbb{Z} && \mathbb{Z} & \mathbb{Z} & \cdots
	\arrow[no head, from=2-1, to=1-1]
	\arrow[no head, from=3-1, to=2-1]
	\arrow[from=3-1, to=4-3]
	\arrow[from=3-1, to=5-4]
	\arrow[no head, from=4-1, to=3-1]
	\arrow["\cong"{description}, from=4-1, to=5-3]
	\arrow[no head, from=5-1, to=4-1]
	\arrow[no head, from=5-1, to=5-3]
	\arrow[no head, from=5-3, to=5-4]
	\arrow[no head, from=5-4, to=5-5]
\end{tikzcd}\]

In this complex, representing
the $E^2$ page, we obtain that
the $d_2$ emminating from
$H_1 \left( \Omega \left( S^2 \vee S^3 \right)  \right) $ 
to $\mathbb{Z}$ must be an isomorphism,
since in $E^{\infty}$ which represents
the homology of the total space 
$P \left( S^2 \vee S^3 \right) $ which is
contractible,
we have that the $\mathbb{Z}$ at
$(2,1)$ vanishes, hence
it must vanish in $E^2$ as
$d_2$ is the only nontrivial map emminating
or terminating at $(2,1)$. This gives surjectivity
of this map, and the same argument
on $H_1 \left( \Omega
\left( S^2 \vee S^3 \right) \right) $, which must also
vanish, gives that
$d_2$ must be injective as well.

Next, we come to the inductive part of the
diagram.
Note that in $E^{\infty}$, all
$H_n \left( \Omega \left( S^2 \vee S^3 \right)  \right) $ 
must vanish for $n \ge 1$. Furthermore,
any map in
$E^{k}$ for $k\ge 4$ has horizontal
length greater than the greatest horizontal distance
between nontrivial objects of the double complex,
hence all maps in $E^{k}$, for $k\ge 4$, must be
trivial, so
$E^{4} = E^{\infty}$. Hence
all the homologies
of $\Omega \left( S^2 \vee S^3 \right) $ must
vanish because of the maps
$d_2$ and $d_3$.
Firstly, note that
$d_2$ maps
$d_2 \colon
H_i \left( \Omega \left( S^2 \vee S^3 \right)  \right) 
\to H_{i-1} 
\left( \Omega \left( S^2 \vee S^3 \right)  \right) $, and
in particular, this map must be surjective
since it is the only nontrivial map
terminating at the homologies
in the second column which all must vanish.
Next note that
we similarly can see that
the maps
$ d_3$ must be surjective (killing off the
terms in the third column) and
injective as they must kill of the objects in the
$0$ th column.

Hence we find that we obtain a SES
\[
0 \to  H_{i-1} \left( \Omega \left( S^2 \vee S^3 \right)  \right) 
\to H_{i+1} \left( \Omega \left( 
S^2 \vee S^3 \right)  \right) 
\to H_i \left( \Omega \left( S^2 \vee S^3 \right)  \right) 
\to 0
\] 
        
Inserting $\mathbb{Z}$ for
$H_{0}$ and
$H_1$ when $i = 1$,
we obtain, since $\mathbb{Z}$ is projective,
that the sequence splits
and $H_2 \left( \Omega
\left( S^2 \vee S^3 \right) \right) \cong
\mathbb{Z}^2$. Assume that
$H_k \left( \Omega
\left( S^2 \vee S^3 \right) \right) \cong
\mathbb{Z}^{F_k}$ for $k\le N-1$.
Then again
\[
0 \to \mathbb{Z}^{F_{N-2}}
\to H_{N}\left( \Omega \left( S^2 \vee S^3 \right)  \right) 
\to \mathbb{Z}^{F_{N-1}} \to 0
\] 
Again $\mathbb{Z}^{F_{N-1}}$ is projective, so the
SES splits, so
\[
    H_{N}\left( \Omega \left( S^2 \vee S^3 \right)  \right) 
\cong
\mathbb{Z}^{F_{N-2}}
\oplus \mathbb{Z}^{F_{N-1}}
\cong \mathbb{Z}^{F_{N-2} + F_{N-1}}
\cong \mathbb{Z}^{F_N}.
\]
Induction now finishes the proof.






    \end{proof}





    \begin{problem}[]
        Let $G$ be a finite group acting freely on a space
        $X$. There is a homotopy fiber sequence
        \[
        X \to X /G \to BG
        \] 
        in which the action of $\pi_1 (BG)$ on
        $H_i(X)$ is induced by the action
        of $G$ on $X$.
        \begin{enumerate}
            \item Show that if $G$ acts freely
                on $S^{n}$ via orientation preserving
                maps (i.e., the action on
                $H_n(S^{n})$ is trivial), then
                there is an isomorphism
                \[
                H_{k+n+1}(BG) \stackrel{\cong}{\to} 
                H_k(BG)
                \] 
                for $k>0$ ($H_*(BG)$ is periodic).\\
                You may use without proof that
                $n$-manifolds admit $n$-dimensional CW structures.
            \item Compute $H_* (BC_3 ; \mathbb{F}_3)$.
                By considering the Serre spectral sequence
                over $\mathbb{F}_3$, show that
                $C_3$ does not acts freely on 
                $\mathbb{R}\mathbb{P}^{2n}$.
            \item Compute $H_{*}\left( B
                \left( C_2 \times C_2 \right) ; \mathbb{F}_2\right) $.
                Is there a free action of $C_2 \times C_2$ on
                 $S^{n}$?
        \end{enumerate}
    \end{problem}

    \begin{proof}
        (1) By assumption, the action of
        $\pi_1 (BG)$ on
        $H_i (S^{n})$ is induced from the action
        of $G$ on $X$ which is by orientation
        preserving maps, so the action on
        $H_i(S^{n})$ is trivial.

        If $BG$ is path-connected, we can use Theorem 1.3.

        Since $G$ is finite and acts freely
        on $S^{n}$, the
        action of $G$ on $S^{n}$ is continuous, free and
        proper, so the quotient space
        $S^{n} / G$ is an $n$-manifold, and thus
        admits an $n$-dimensional CW structure. 

        In particular, all homology groups
        of dimension greater than $n$ vanish
        for $S^{n} /G$.

        
        In the double complex,
        we plot 
        $H_s(BG,H_s(S^{n}))$ on
        the $(s,t)$ th coordinate.
        In particular,
        for $t \neq 0,n$, the coefficient group
        vanishes, hence 
        $H_s(BG, H_t(S^{n}))$ vanishes
        for $t \neq 0,n$.\\
        For $t = 0,n$, $H_s(S^{n}) \cong \mathbb{Z}$, so
        letting
        $H_s(BG) = H_s (BG; \mathbb{Z})$ denote the
        integral homology, we obtain the following
        diagram from
        the double complex using the Serre spectral sequence

\adjustbox{scale=0.8,center}{
\begin{tikzcd}
	\vdots \\
	{H_n(S^n) \cong \mathbb{Z}} & {H_1(BG)} & {H_2(BG)} & \cdots & {H_k(BG)} \\
	\\
	{\mathbb{Z}} & {H_1(BG)} & {H_2(BG)} & \cdots & {H_{1+n}(BG)} & {H_{2+n}(BG)} & {H_{3+n}(BG)} & \cdots & {H_{k+n+1}(BG)}
	\arrow[no head, from=2-1, to=1-1]
	\arrow["\cong", from=2-2, to=4-6]
	\arrow["\cong", from=2-3, to=4-7]
	\arrow["\cong", from=2-5, to=4-9]
	\arrow[no head, from=4-1, to=2-1]
	\arrow[no head, from=4-1, to=4-2]
	\arrow[from=4-2, to=4-3]
	\arrow[no head, from=4-3, to=4-4]
	\arrow[no head, from=4-4, to=4-5]
\end{tikzcd}
}
In the diagram, we can conclude the isomorphisms because
we know that the direct sum of the
objects on the diagonals $n+1+k$ for $k\ge 0$
must be $0$ (as this is the
$(n+k+1)$ st homology of
$S^{n} / G$ which vanishes), and thus
$H_{k+n+1}(BG)$ on the horizontal axis must
be eliminated for $k\ge 0$. The only possibilities
for nontrivial map terminating at
these objects are the maps drawn in the diagram, hence
these are surjective.

But similarly, the domain objects of these maps
must be eliminated since they lie on
the diagonals for $n+k+1$ with $k\ge 0$, hence
all these maps must be injective too since these are
the only nontrivial maps emminating or terminating from
the objects.

From this, we obtain the desired isomorphisms
$H_k (BG) \cong H_{n+k+1}(BG)$ for all
$k\ge 1$.\\
\linebreak
(2) Since $\mathbb{F}_3$ is a field, there is no torsion and
it is abelian,
so $E_{s,t}^2 = 
H_s \left( B, H_t(F) \right) $ simplifies
to $E_{s,t}^2 =
H_s(B) \otimes_{\mathbb{Z}} H_t(F)$.

For $C_3 = \mathbb{Z}_3$, we obtain using the
free action of $\mathbb{Z}_3$ on
$S^{1}$ by
$z \mapsto e^{\frac{2 \pi i}{3}} z$, a
quotient space which is the lens space
$L(3;1) $.
Using (1), we now have that
$H_k(B\mathbb{Z}_3) \cong
H_{k+2}(B\mathbb{Z}_3)$ for all $k>0$.


We know the homology
of the lens space is

\[
H_k \left( L(3;1) \right) 
\cong 
\begin{cases}
    \mathbb{Z},& k=0,1\\
    0,& \text{else}
\end{cases}
\] 
so
\[
H_k \left( L(3;1); \mathbb{F}_3 \right) 
\cong
\begin{cases}
    \mathbb{Z} /3,& k=0,1\\
    0,& \text{else}
\end{cases}
\] 
Since $\mathbb{F}_3$ is a field,
the homology of
$L(3;1)$ in degree $k$ will be the direct sum of the diagonal entries
on the $k$ diagonal.\\
So for the double complex, we want
a $\mathbb{Z}/3$ in the $0$ and $1$ diagonal, and trivial groups
in the rest.

We obtain the following diagram, and conclude that

 \[
H_n\left( B\mathbb{Z}_3 \right) 
\cong
\begin{cases}
    \mathbb{Z},& n=0 \text{ or } n \text{ odd}\\
    \mathbb{Z}/m,& \text{else}
\end{cases}
\] 
for some $m \in \mathbb{N}$

\begin{tikzcd}
	\vdots \\
	{\mathbb{Z}} & {H_1(B\mathbb{Z}_3)} & {H_2(B\mathbb{Z}_3)} & {H_3(B\mathbb{Z}_3)} \\
	{\mathbb{Z}} & {H_1(B\mathbb{Z}_3)} & {H_2(B\mathbb{Z}_3)} & {H_3(B\mathbb{Z}_3)} & \cdots
	\arrow[no head, from=2-1, to=1-1]
	\arrow[two heads, from=2-1, to=3-3]
	\arrow["\cong"{description}, from=2-2, to=3-4]
	\arrow["\cong", from=2-3, to=3-5]
	\arrow[no head, from=3-1, to=2-1]
	\arrow[no head, from=3-1, to=3-2]
	\arrow[no head, from=3-2, to=3-3]
	\arrow[no head, from=3-3, to=3-4]
	\arrow[no head, from=3-4, to=3-5]
\end{tikzcd}


Consider now the free action of
$\mathbb{Z}_3$ on $S^3$ giving
 the lens space $L(3;1,1)$.
 Note that
  \[
  H_k(L(3;1,1)) \cong
  \begin{cases}
      \mathbb{Z},& k=0,3\\
      \mathbb{Z} / 3,& 0<k<3 \text{ odd}\\
     0,& \text{else}
  \end{cases}
  \] 
  so
  \[
  H_k\left( L(3;1,1); \mathbb{F}_3 \right) 
  \cong
  \begin{cases}
      \mathbb{Z} /3,& k=0,1,3\\
      0,& \text{else}
  \end{cases}
  \] 

So we obtain the following double complex:
\[
    \begin{tikzcd}
	{\mathbb{Z}/3} & {H_1(B\mathbb{Z}_3;\mathbb{F}_3)}
                   & \cdots & {H_4(B\mathbb{Z}_3;\mathbb{F}_3)}
                   & {H_5(B\mathbb{Z}_3;\mathbb{F}_3)} \\
	\\
	{\mathbb{Z}/3} & {H_1(B\mathbb{Z}_3;\mathbb{F}_3)}
                   & \cdots & {H_4(B\mathbb{Z}_3;\mathbb{F}_3)}
                   & {H_5(B\mathbb{Z}_3;\mathbb{F}_3)}
	\arrow[from=1-1, to=3-4]
	\arrow[from=1-2, to=3-5]
	\arrow[no head, from=3-1, to=1-1]
	\arrow[no head, from=3-1, to=3-2]
	\arrow[no head, from=3-2, to=3-3]
	\arrow[no head, from=3-3, to=3-4]
	\arrow[no head, from=3-4, to=3-5]
\end{tikzcd}
\]
It is then clear that there are no maps that can
cancel the $H_1\left( B\mathbb{Z}_3 ;\mathbb{F}_3\right) $ and
$H_2\left( B\mathbb{Z}_3; \mathbb{F}_3 \right) $ and
that these terms are the only contributions to the
degree $1$ and $2$ homology of the
lens space $L(3;1,1)$, respectively.
In particular, this immediately gives that
$H_1\left( B\mathbb{Z}_3; \mathbb{F}_3 \right)
\cong \mathbb{Z} /3$ and
$H_2 \left( B\mathbb{Z}_3 ; \mathbb{F}_3 \right) \cong
0$
so by the first part,

\[
H_k \left( B\mathbb{Z}_3 ; \mathbb{F}_3 \right) 
\cong
\begin{cases}
    \mathbb{Z}/3,& k=0 \text{ or } k \text{ odd and positive}\\
    0,& \text{else}
\end{cases}
\] 

Now suppose that
$\mathbb{Z}_3$ acted freely on
$\mathbb{R}\mathbb{P}^{2n}$.
Then the quotient space, just as in the first problem,
is a $2n$-manifold, hence admits
a $2n$-dimensional CW structure, hence
$H_k(\mathbb{R}\mathbb{P}^{2n}/ \mathbb{Z}_3)$ vanishes
for $k>2n$.
Now the double complex looks like the following diagram:


\adjustbox{scale=0.8,center}{
\begin{tikzcd}
	{\mathbb{Z}/3} & {\underbrace{H_1(B\mathbb{Z}/3;\mathbb{F}_3)}_{\cong \mathbb{Z}/3}} & {\underbrace{H_2(B\mathbb{Z}/3;\mathbb{F}_3)}_{\cong 0}} \\
	\\
	{\mathbb{Z}/3} & {\underbrace{H_1(B\mathbb{Z}/3;\mathbb{F}_3)}_{\cong \mathbb{Z}/3}} & \cdots & {\underbrace{H_{2n+1}(B\mathbb{Z}/3;\mathbb{F}_3)}_{\cong \mathbb{Z}/3}} & {\underbrace{H_{2n+2}(B\mathbb{Z}/3;\mathbb{F}_3)}_{\cong 0}} & {\underbrace{H_{2n+3}(B\mathbb{Z}/3;\mathbb{F}_3)}_{\cong \mathbb{Z}/3}}
	\arrow["{d_{2n}}", from=1-1, to=3-4]
	\arrow[from=1-2, to=3-5]
	\arrow[from=1-3, to=3-6]
	\arrow[no head, from=3-1, to=1-1]
\end{tikzcd}
}
In particular,
since the homologies
of $\mathbb{R}\mathbb{P}^{2n}/ \mathbb{Z}_3$ vanish
in dimensions $>2n$, we must have that
the maps
into $H_{2n+k}\left( B\mathbb{Z}/3, \mathbb{F}_3 \right) $ 
for $k>0$
on the horizontal line must be surjective when
$k$ is odd (to cancel the $\mathbb{Z}/3$ ).
But consider for example the map
$0 \cong H_2 \left( B \mathbb{Z}/3; \mathbb{F}_3 \right) 
\to H_{2n+3}\left( B\mathbb{Z}/3; \mathbb{F}_3 \right) \cong
\mathbb{Z} / 3$. This cannot be surjective, so
we obtain a contradiction. So the
assumption of $\mathbb{Z}_3$ acting freely
on $\mathbb{R}\mathbb{P}^{2n}$ must be false.\\
\linebreak




    \end{proof}








    %\printbibliography
\end{document}
