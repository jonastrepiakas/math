\documentclass[reqno]{amsart}
\usepackage{amscd, amssymb, amsmath, amsthm}
\usepackage{graphicx}
\usepackage[colorlinks=true,linkcolor=blue]{hyperref}
\usepackage[utf8]{inputenc}
\usepackage[T1]{fontenc}
\usepackage{textcomp}
\usepackage{babel}
%% for identity function 1:
\usepackage{bbm}
%%For category theory diagrams:
\usepackage{tikz-cd}

%\usepackage[backend=biber]{biblatex}
%\addbibresource{.bib}


\setlength\parindent{0pt}

\pdfsuppresswarningpagegroup=1

\newtheorem{theorem}{Theorem}[section]
\newtheorem{lemma}[theorem]{Lemma}
\newtheorem{proposition}[theorem]{Proposition}
\newtheorem{corollary}[theorem]{Corollary}
\newtheorem{conjecture}[theorem]{Conjecture}

\theoremstyle{definition}
\newtheorem{definition}[theorem]{Definition}
\newtheorem{example}[theorem]{Example}
\newtheorem{exercise}[theorem]{Exercise}
\newtheorem{problem}[theorem]{Problem}
\newtheorem{question}[theorem]{Question}

\theoremstyle{remark}
\newtheorem*{remark}{Remark}
\newtheorem*{note}{Note}
\newtheorem*{solution}{Solution}



%Inequalities
\newcommand{\cycsum}{\sum_{\mathrm{cyc}}}
\newcommand{\symsum}{\sum_{\mathrm{sym}}}
\newcommand{\cycprod}{\prod_{\mathrm{cyc}}}
\newcommand{\symprod}{\prod_{\mathrm{sym}}}

%Linear Algebra

\DeclareMathOperator{\Span}{span}
\DeclareMathOperator{\im}{im}
\DeclareMathOperator{\diag}{diag}
\DeclareMathOperator{\Ker}{Ker}
\DeclareMathOperator{\ob}{ob}
\DeclareMathOperator{\Hom}{Hom}
\DeclareMathOperator{\Mor}{Mor}
\DeclareMathOperator{\sk}{sk}
\DeclareMathOperator{\Vect}{Vect}
\DeclareMathOperator{\Set}{Set}
\DeclareMathOperator{\Group}{Group}
\DeclareMathOperator{\Ring}{Ring}
\DeclareMathOperator{\Ab}{Ab}
\DeclareMathOperator{\Top}{Top}
\DeclareMathOperator{\hTop}{hTop}
\DeclareMathOperator{\Htpy}{Htpy}
\DeclareMathOperator{\Cat}{Cat}
\DeclareMathOperator{\CAT}{CAT}
\DeclareMathOperator{\Cone}{Cone}
\DeclareMathOperator{\dom}{dom}
\DeclareMathOperator{\cod}{cod}
\DeclareMathOperator{\Aut}{Aut}
\DeclareMathOperator{\Mat}{Mat}
\DeclareMathOperator{\Fin}{Fin}
\DeclareMathOperator{\rel}{rel}
\DeclareMathOperator{\Int}{Int}
\DeclareMathOperator{\sgn}{sgn}
\DeclareMathOperator{\Homeo}{Homeo}
\DeclareMathOperator{\SHomeo}{SHomeo}
\DeclareMathOperator{\PSL}{PSL}
\DeclareMathOperator{\Bil}{Bil}
\DeclareMathOperator{\Sym}{Sym}
\DeclareMathOperator{\Skew}{Skew}
\DeclareMathOperator{\Alt}{Alt}
\DeclareMathOperator{\Quad}{Quad}
\DeclareMathOperator{\Sin}{Sin}
\DeclareMathOperator{\Supp}{Supp}
\DeclareMathOperator{\Char}{char}
\DeclareMathOperator{\Teich}{Teich}
\DeclareMathOperator{\GL}{GL}
\DeclareMathOperator{\tr}{tr}
\DeclareMathOperator{\codim}{codim}
\DeclareMathOperator{\coker}{coker}
\DeclareMathOperator{\corank}{corank}
\DeclareMathOperator{\rank}{rank}
\DeclareMathOperator{\Diff}{Diff}
\DeclareMathOperator{\Bun}{Bun}
\DeclareMathOperator{\Sm}{Sm}
\DeclareMathOperator{\Fr}{Fr}
\DeclareMathOperator{\Cob}{Cob}
\DeclareMathOperator{\Ext}{Ext}
\DeclareMathOperator{\Tor}{Tor}
\DeclareMathOperator{\Conf}{Conf}
\DeclareMathOperator{\UConf}{UConf}



%Row operations
\newcommand{\elem}[1]{% elementary operations
\xrightarrow{\substack{#1}}%
}

\newcommand{\lelem}[1]{% elementary operations (left alignment)
\xrightarrow{\begin{subarray}{l}#1\end{subarray}}%
}

%SS
\DeclareMathOperator{\supp}{supp}
\DeclareMathOperator{\Var}{Var}

%NT
\DeclareMathOperator{\ord}{ord}

%Alg
\DeclareMathOperator{\Rad}{Rad}
\DeclareMathOperator{\Jac}{Jac}

%Misc
\newcommand{\SL}{{\mathrm{SL}}}
\newcommand{\mobgp}{{\mathrm{PSL}_2(\mathbb{C})}}
\newcommand{\id}{{\mathrm{id}}}
\newcommand{\MCG}{{\mathrm{MCG}}}
\newcommand{\PMCG}{{\mathrm{PMCG}}}
\newcommand{\SMCG}{{\mathrm{SMCG}}}
\newcommand{\ud}{{\mathrm{d}}}
\newcommand{\Vol}{{\mathrm{Vol}}}
\newcommand{\Area}{{\mathrm{Area}}}
\newcommand{\diam}{{\mathrm{diam}}}
\newcommand{\End}{{\mathrm{End}}}


\newcommand{\reg}{{\mathtt{reg}}}
\newcommand{\geo}{{\mathtt{geo}}}

\newcommand{\tori}{{\mathcal{T}}}
\newcommand{\cpn}{{\mathtt{c}}}
\newcommand{\pat}{{\mathtt{p}}}

\let\Cap\undefined
\newcommand{\Cap}{{\mathcal{C}}ap}
\newcommand{\Push}{{\mathcal{P}}ush}
\newcommand{\Forget}{{\mathcal{F}}orget}




\begin{document}
    Conventions for this assignment: We assume
    all topological spaces to be nice enough for covering
    theory (we can even assume locally contractible).
    Basepoints are assumed to be good basepoints, namely
    the inclusion $\left\{ x \right\} \subset 
    X$ is assumed to have the homotopy extension property.
    If
    $X$ is a space, then $\Omega X$ denotes
    its loop space and there is a fiber sequence
    \[
    \Omega X \to PX \to X
    \] 
    where $PX$ is a contractible space.


    \begin{problem}[]
        Show that the homology of
        $\Omega\left( S^2 \vee S^3 \right) $ is
        \[
        H_* \left( \Omega \left( S^2 \vee S^3 \right) ;
        \mathbb{Z}\right) \cong
        \mathbb{Z}^{F_n}
        \] 
        where $F_n$ is the $n$ th Fibbonaci number
        (We set $F_0 = F_1 = 1$ and
        $F_n = F_{n-1} + F_{n-2}$ ).
    \end{problem}

    \begin{proof}
        Using the fiber sequence
        \[
        \Omega \left( S^2 \vee S^3 \right) 
        \to P \left( S^2 \vee S^3 \right) 
        \to S^2 \vee S^3
        \] 
        we obtain the following quadrant double complex:
        \[\begin{tikzcd}
	\vdots && \vdots & \vdots \\
	{H_3(\Omega (S^2 \vee S^3))} && {H_3(\Omega (S^2 \vee S^3))} & {H_3(\Omega (S^2 \vee S^3))} \\
	{H_2(\Omega (S^2 \vee S^3))} && {H_2(\Omega (S^2 \vee S^3))} & {H_2(\Omega (S^2 \vee S^3))} \\
	{H_1(\Omega (S^2 \vee S^3))} && {H_1(\Omega (S^2 \vee S^3))} & {H_1(\Omega (S^2 \vee S^3))} \\
	\mathbb{Z} && \mathbb{Z} & \mathbb{Z} & \cdots
	\arrow[no head, from=2-1, to=1-1]
	\arrow[no head, from=3-1, to=2-1]
	\arrow[from=3-1, to=4-3]
	\arrow[from=3-1, to=5-4]
	\arrow[no head, from=4-1, to=3-1]
	\arrow["\cong"{description}, from=4-1, to=5-3]
	\arrow[no head, from=5-1, to=4-1]
	\arrow[no head, from=5-1, to=5-3]
	\arrow[no head, from=5-3, to=5-4]
	\arrow[no head, from=5-4, to=5-5]
\end{tikzcd}\]

In this complex, representing
the $E^2$ page, we obtain that
the $d_2$ emminating from
$H_1 \left( \Omega \left( S^2 \vee S^3 \right)  \right) $ 
to $\mathbb{Z}$ must be an isomorphism,
since in $E^{\infty}$ which represents
the homology of the total space 
$P \left( S^2 \vee S^3 \right) $ which is
contractible,
we have that the $\mathbb{Z}$ at
$(2,1)$ vanishes, hence
it must vanish in $E^2$ as
$d_2$ is the only nontrivial map emminating
or terminating at $(2,1)$. This gives surjectivity
of this map, and the same argument
on $H_1 \left( \Omega
\left( S^2 \vee S^3 \right) \right) $, which must also
vanish, gives that
$d_2$ must be injective as well.

Next, we come to the inductive part of the
diagram.
Note that in $E^{\infty}$, all
$H_n \left( \Omega \left( S^2 \vee S^3 \right)  \right) $ 
must vanish for $n \ge 1$. Furthermore,
any map in
$E^{k}$ for $k\ge 4$ has horizontal
length greater than the greatest horizontal distance
between nontrivial objects of the double complex,
hence all maps in $E^{k}$, for $k\ge 4$, must be
trivial, so
$E^{4} = E^{\infty}$. Hence
all the homologies
of $\Omega \left( S^2 \vee S^3 \right) $ must
vanish because of the maps
$d_2$ and $d_3$.
Firstly, note that
$d_2$ maps
$d_2 \colon
H_i \left( \Omega \left( S^2 \vee S^3 \right)  \right) 
\to H_{i-1} 
\left( \Omega \left( S^2 \vee S^3 \right)  \right) $, and
in particular, this map must be surjective
since it is the only nontrivial map
terminating at the homologies
in the second column which all must vanish.
Next note that
we similarly can see that
the maps
$ d_3$ must be surjective (killing off the
terms in the third column) and
injective as they must kill of the objects in the
$0$ th column.

Hence we find that we obtain a SES
\[
0 \to  H_{i-1} \left( \Omega \left( S^2 \vee S^3 \right)  \right) 
\to H_{i+1} \left( \Omega \left( 
S^2 \vee S^3 \right)  \right) 
\to H_i \left( \Omega \left( S^2 \vee S^3 \right)  \right) 
\to 0
\] 
        
Inserting $\mathbb{Z}$ for
$H_{0}$ and
$H_1$ when $i = 1$,
we obtain, since $\mathbb{Z}$ is projective,
that the sequence splits
and $H_2 \left( \Omega
\left( S^2 \vee S^3 \right) \right) \cong
\mathbb{Z}^2$. Assume that
$H_k \left( \Omega
\left( S^2 \vee S^3 \right) \right) \cong
\mathbb{Z}^{F_k}$ for $k\le N-1$.
Then again
\[
0 \to \mathbb{Z}^{F_{N-2}}
\to H_{N}\left( \Omega \left( S^2 \vee S^3 \right)  \right) 
\to \mathbb{Z}^{F_{N-1}} \to 0
\] 
Again $\mathbb{Z}^{F_{N-1}}$ is projective, so the
SES splits, so
\[
    H_{N}\left( \Omega \left( S^2 \vee S^3 \right)  \right) 
\cong
\mathbb{Z}^{F_{N-2}}
\oplus \mathbb{Z}^{F_{N-1}}
\cong \mathbb{Z}^{F_{N-2} + F_{N-1}}
\cong \mathbb{Z}^{F_N}.
\]
Induction now finishes the proof.






    \end{proof}














    %\printbibliography
\end{document}
