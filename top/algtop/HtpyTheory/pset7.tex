\documentclass[reqno]{amsart}
\usepackage{amscd, amssymb, amsmath, amsthm}
\usepackage{graphicx}
\usepackage[colorlinks=true,linkcolor=blue]{hyperref}
\usepackage[utf8]{inputenc}
\usepackage[T1]{fontenc}
\usepackage{textcomp}
\usepackage{babel}
%% for identity function 1:
\usepackage{bbm}
%%For category theory diagrams:
\usepackage{tikz-cd}

%\usepackage[backend=biber]{biblatex}
%\addbibresource{.bib}


\setlength\parindent{0pt}

\pdfsuppresswarningpagegroup=1

\newtheorem{theorem}{Theorem}[section]
\newtheorem{lemma}[theorem]{Lemma}
\newtheorem{proposition}[theorem]{Proposition}
\newtheorem{corollary}[theorem]{Corollary}
\newtheorem{conjecture}[theorem]{Conjecture}

\theoremstyle{definition}
\newtheorem{definition}[theorem]{Definition}
\newtheorem{example}[theorem]{Example}
\newtheorem{exercise}[theorem]{Exercise}
\newtheorem{problem}[theorem]{Problem}
\newtheorem{question}[theorem]{Question}

\theoremstyle{remark}
\newtheorem*{remark}{Remark}
\newtheorem*{note}{Note}
\newtheorem*{solution}{Solution}



%Inequalities
\newcommand{\cycsum}{\sum_{\mathrm{cyc}}}
\newcommand{\symsum}{\sum_{\mathrm{sym}}}
\newcommand{\cycprod}{\prod_{\mathrm{cyc}}}
\newcommand{\symprod}{\prod_{\mathrm{sym}}}

%Linear Algebra

\DeclareMathOperator{\Span}{span}
\DeclareMathOperator{\im}{im}
\DeclareMathOperator{\diag}{diag}
\DeclareMathOperator{\Ker}{Ker}
\DeclareMathOperator{\ob}{ob}
\DeclareMathOperator{\Hom}{Hom}
\DeclareMathOperator{\Mor}{Mor}
\DeclareMathOperator{\sk}{sk}
\DeclareMathOperator{\Vect}{Vect}
\DeclareMathOperator{\Set}{Set}
\DeclareMathOperator{\Group}{Group}
\DeclareMathOperator{\Ring}{Ring}
\DeclareMathOperator{\Ab}{Ab}
\DeclareMathOperator{\Top}{Top}
\DeclareMathOperator{\hTop}{hTop}
\DeclareMathOperator{\Htpy}{Htpy}
\DeclareMathOperator{\Cat}{Cat}
\DeclareMathOperator{\CAT}{CAT}
\DeclareMathOperator{\Cone}{Cone}
\DeclareMathOperator{\dom}{dom}
\DeclareMathOperator{\cod}{cod}
\DeclareMathOperator{\Aut}{Aut}
\DeclareMathOperator{\Mat}{Mat}
\DeclareMathOperator{\Fin}{Fin}
\DeclareMathOperator{\rel}{rel}
\DeclareMathOperator{\Int}{Int}
\DeclareMathOperator{\sgn}{sgn}
\DeclareMathOperator{\Homeo}{Homeo}
\DeclareMathOperator{\SHomeo}{SHomeo}
\DeclareMathOperator{\PSL}{PSL}
\DeclareMathOperator{\Bil}{Bil}
\DeclareMathOperator{\Sym}{Sym}
\DeclareMathOperator{\Skew}{Skew}
\DeclareMathOperator{\Alt}{Alt}
\DeclareMathOperator{\Quad}{Quad}
\DeclareMathOperator{\Sin}{Sin}
\DeclareMathOperator{\Supp}{Supp}
\DeclareMathOperator{\Char}{char}
\DeclareMathOperator{\Teich}{Teich}
\DeclareMathOperator{\GL}{GL}
\DeclareMathOperator{\tr}{tr}
\DeclareMathOperator{\codim}{codim}
\DeclareMathOperator{\coker}{coker}
\DeclareMathOperator{\corank}{corank}
\DeclareMathOperator{\rank}{rank}
\DeclareMathOperator{\Diff}{Diff}
\DeclareMathOperator{\Bun}{Bun}
\DeclareMathOperator{\Sm}{Sm}
\DeclareMathOperator{\Fr}{Fr}
\DeclareMathOperator{\Cob}{Cob}
\DeclareMathOperator{\Ext}{Ext}
\DeclareMathOperator{\Tor}{Tor}
\DeclareMathOperator{\Conf}{Conf}
\DeclareMathOperator{\UConf}{UConf}



%Row operations
\newcommand{\elem}[1]{% elementary operations
\xrightarrow{\substack{#1}}%
}

\newcommand{\lelem}[1]{% elementary operations (left alignment)
\xrightarrow{\begin{subarray}{l}#1\end{subarray}}%
}

%SS
\DeclareMathOperator{\supp}{supp}
\DeclareMathOperator{\Var}{Var}

%NT
\DeclareMathOperator{\ord}{ord}

%Alg
\DeclareMathOperator{\Rad}{Rad}
\DeclareMathOperator{\Jac}{Jac}

%Misc
\newcommand{\SL}{{\mathrm{SL}}}
\newcommand{\mobgp}{{\mathrm{PSL}_2(\mathbb{C})}}
\newcommand{\id}{{\mathrm{id}}}
\newcommand{\MCG}{{\mathrm{MCG}}}
\newcommand{\PMCG}{{\mathrm{PMCG}}}
\newcommand{\SMCG}{{\mathrm{SMCG}}}
\newcommand{\ud}{{\mathrm{d}}}
\newcommand{\Vol}{{\mathrm{Vol}}}
\newcommand{\Area}{{\mathrm{Area}}}
\newcommand{\diam}{{\mathrm{diam}}}
\newcommand{\End}{{\mathrm{End}}}


\newcommand{\reg}{{\mathtt{reg}}}
\newcommand{\geo}{{\mathtt{geo}}}

\newcommand{\tori}{{\mathcal{T}}}
\newcommand{\cpn}{{\mathtt{c}}}
\newcommand{\pat}{{\mathtt{p}}}

\let\Cap\undefined
\newcommand{\Cap}{{\mathcal{C}}ap}
\newcommand{\Push}{{\mathcal{P}}ush}
\newcommand{\Forget}{{\mathcal{F}}orget}




\begin{document}
    \begin{problem}[]
        Let $F$ be the homotopy fibre of the map
        $S^{n} \to S^{n}$ of degree $k$, for
        $n\ge 2$.
        \begin{enumerate}
            \item Show that
                $H^{i}(F) = 0$ for $0 < i < n$.
            \item Using the Serre spectral sequence,
                compute that
                \[
                H^{i}(F) = 
                \begin{cases}
                    \mathbb{Z},& i=0\\
                    \mathbb{Z} / k,& i = 1 + m(n-1), \, m>0\\
                    0,& \text{otherwise}
                \end{cases}.
                \] 
            \item Show that for $x,y \in H^{*}(F)$, if
                $\deg (x), \deg (y) > 0$, then
                 $x \smile y = 0$.
        \end{enumerate}
    \end{problem}

        \begin{proof}
            (1) Since $\pi_1 S^{n} = 0$, the Serre spectral
            sequence to the homotopy fiber sequence
            \[
            F \to S^{n} \to S^{n}
            \] 
            gives the following double complex:



\[\begin{tikzcd}
	{H^{n-1}(F)} \\
	\vdots \\
	{H^2(F)} \\
	{H^1(F)} \\
	{\mathbb{Z}} && {\mathbb{Z}} & \cdots \\
	&& n
	\arrow[from=1-1, to=5-3]
	\arrow[no head, from=2-1, to=1-1]
	\arrow[no head, from=3-1, to=2-1]
	\arrow[no head, from=4-1, to=3-1]
	\arrow[no head, from=5-1, to=4-1]
	\arrow[no head, from=5-1, to=5-3]
	\arrow[no head, from=5-3, to=5-4]
\end{tikzcd}\]

We apply the LSSS for cohomology and find that
$H^{i}(S^{n}) = F_0^{n}$, and since
$H^{i}(F)$ is the only nontrivial entry on the
antidiagonal in degree $i$, and 
since there are no maps to kill off 
$H^{i}(F)$ for $0<i<n-1$, we obtain that
$H^{i}(F) =
H^{i}(S^{n}) = 0$ for $0 < i < n-1$.

All that's missing is $i = n-1$. For this, note that by the LES
for the fibration, we get the following exact sequence:
\[
    \underbrace{\mathbb{Z}}_{\pi_n(S^{n})}
    \stackrel{\cdot k}{\to} \underbrace{\mathbb{Z}}_{\pi_n(S^{n})}
    \to \pi_{n-1}(F) \to \underbrace{0}_{\pi_{n-1}(S^{n})}
\] 
hence $\pi_{n-1}\left(F \right) \cong
\coker \left( \mathbb{Z} \stackrel{\cdot k}{\to} \mathbb{Z} \right) 
\cong \mathbb{Z} / k$, and by the Hurewicz theorem, we
get $H_{n-1}(F) \cong \pi_{n-1}(F) \cong
\mathbb{Z} / k$.
Now using the UCT, we obtain 
\[
    0 \to \underbrace{\Ext \left( H_{n-2}(F), \mathbb{Z} \right)}_{=0}
    \to H^{n-1} (F) \to \underbrace{
        \Hom \left( \underbrace{H_{n-1}(F)}_{=
    \mathbb{Z} / k},\mathbb{Z} \right)}_{=0} \to 0
\] 
so $H^{n-1}(F) = 0$ as we wanted.



(2)

By the LSSS, the $E^{\infty}$ page has the
form $E_{0,0}^{\infty} = E_{n,0}^{\infty} \cong
\mathbb{Z}$, so in particular,
on the $E^{k}$ page, we get the following double complex:


\[\begin{tikzcd}
	\vdots \\
	{H^{3(n-1)}(F)} && {H^{3(n-1)}(F)} \\
	{H^{2(n-1)}(F)} && {H^{2(n-1)}(F)} \\
	{\mathbb{Z}\cong H^{n-1}(F)} && {\mathbb{Z}\cong H^{n-1}(F)} \\
	{\mathbb{Z}} && {\mathbb{Z}}
	\arrow[no head, from=2-1, to=1-1]
	\arrow[from=2-1, to=3-3]
	\arrow[no head, from=3-1, to=2-1]
	\arrow[from=3-1, to=4-3]
	\arrow[no head, from=4-1, to=3-1]
	\arrow[from=4-1, to=5-3]
	\arrow[no head, from=5-1, to=4-1]
	\arrow[no head, from=5-1, to=5-3]
\end{tikzcd}\]

This is the only page on which the horizontal maps
can be nontrivial, so given the
$E^{\infty}$ page, we conclude that the
maps must be isomorphisms (including the trivial ones
by just inductively shifting down
by $n-1$ enough times).
Hence we get periodicity, so
\[
H^{i}(F) =
\begin{cases}
    \mathbb{Z},& i=0\\
    \mathbb{Z} /k,& i = 1+m(n-1), \, m>0\\
    0,& \text{otherwise}
\end{cases},
\] 
which was what we wanted to show.\\
\linebreak





(3) Suppose $\deg (x) + \deg (y) = 2$ so both are of 
degree $1$, then
since $H^{1}(F) = 0$, we have $x=0=y$ so
$x \smile y = 0$.
Suppose we have shown it for
$\deg (x) + \deg (y) \le N-1$ now. If
$\deg x + \deg y = N$, then
firstly we can assume
$x,y\neq 0$ since otherwise
$x \smile y = 0$. Hence
$x \in H^{1+ m(n-1)}(F)$ and
$y \in H^{1+m'(n-1)}$, so
$x \smile y \in 
H^{2 + (m+m')(n-1)}(F) = 0$, so directly,
$x \smile y = 0$.



\end{proof}


\begin{problem}[]
    Use the path-loop fibration to deduce the
    cohomo
\end{problem}






    %\printbibliography
\end{document}
