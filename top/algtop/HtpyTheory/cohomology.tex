\documentclass[reqno]{amsart}
\usepackage{amscd, amssymb, amsmath, amsthm}
\usepackage{graphicx}
\usepackage[colorlinks=true,linkcolor=blue]{hyperref}
\usepackage[utf8]{inputenc}
\usepackage[T1]{fontenc}
\usepackage{textcomp}
\usepackage{babel}
%% for identity function 1:
\usepackage{bbm}
%%For category theory diagrams:
\usepackage{tikz-cd}

%\usepackage[backend=biber]{biblatex}
%\addbibresource{.bib}


\setlength\parindent{0pt}

\pdfsuppresswarningpagegroup=1

\newtheorem{theorem}{Theorem}[section]
\newtheorem{lemma}[theorem]{Lemma}
\newtheorem{proposition}[theorem]{Proposition}
\newtheorem{corollary}[theorem]{Corollary}
\newtheorem{conjecture}[theorem]{Conjecture}

\theoremstyle{definition}
\newtheorem{definition}[theorem]{Definition}
\newtheorem{example}[theorem]{Example}
\newtheorem{exercise}[theorem]{Exercise}
\newtheorem{problem}[theorem]{Problem}
\newtheorem{question}[theorem]{Question}

\theoremstyle{remark}
\newtheorem*{remark}{Remark}
\newtheorem*{note}{Note}
\newtheorem*{solution}{Solution}



%Inequalities
\newcommand{\cycsum}{\sum_{\mathrm{cyc}}}
\newcommand{\symsum}{\sum_{\mathrm{sym}}}
\newcommand{\cycprod}{\prod_{\mathrm{cyc}}}
\newcommand{\symprod}{\prod_{\mathrm{sym}}}

%Linear Algebra

\DeclareMathOperator{\Span}{span}
\DeclareMathOperator{\im}{im}
\DeclareMathOperator{\diag}{diag}
\DeclareMathOperator{\Ker}{Ker}
\DeclareMathOperator{\ob}{ob}
\DeclareMathOperator{\Hom}{Hom}
\DeclareMathOperator{\Mor}{Mor}
\DeclareMathOperator{\sk}{sk}
\DeclareMathOperator{\Vect}{Vect}
\DeclareMathOperator{\Set}{Set}
\DeclareMathOperator{\Group}{Group}
\DeclareMathOperator{\Ring}{Ring}
\DeclareMathOperator{\Ab}{Ab}
\DeclareMathOperator{\Top}{Top}
\DeclareMathOperator{\hTop}{hTop}
\DeclareMathOperator{\Htpy}{Htpy}
\DeclareMathOperator{\Cat}{Cat}
\DeclareMathOperator{\CAT}{CAT}
\DeclareMathOperator{\Cone}{Cone}
\DeclareMathOperator{\dom}{dom}
\DeclareMathOperator{\cod}{cod}
\DeclareMathOperator{\Aut}{Aut}
\DeclareMathOperator{\Mat}{Mat}
\DeclareMathOperator{\Fin}{Fin}
\DeclareMathOperator{\rel}{rel}
\DeclareMathOperator{\Int}{Int}
\DeclareMathOperator{\sgn}{sgn}
\DeclareMathOperator{\Homeo}{Homeo}
\DeclareMathOperator{\SHomeo}{SHomeo}
\DeclareMathOperator{\PSL}{PSL}
\DeclareMathOperator{\Bil}{Bil}
\DeclareMathOperator{\Sym}{Sym}
\DeclareMathOperator{\Skew}{Skew}
\DeclareMathOperator{\Alt}{Alt}
\DeclareMathOperator{\Quad}{Quad}
\DeclareMathOperator{\Sin}{Sin}
\DeclareMathOperator{\Supp}{Supp}
\DeclareMathOperator{\Char}{char}
\DeclareMathOperator{\Teich}{Teich}
\DeclareMathOperator{\GL}{GL}
\DeclareMathOperator{\tr}{tr}
\DeclareMathOperator{\codim}{codim}
\DeclareMathOperator{\coker}{coker}
\DeclareMathOperator{\corank}{corank}
\DeclareMathOperator{\rank}{rank}
\DeclareMathOperator{\Diff}{Diff}
\DeclareMathOperator{\Bun}{Bun}
\DeclareMathOperator{\Sm}{Sm}
\DeclareMathOperator{\Fr}{Fr}
\DeclareMathOperator{\Cob}{Cob}
\DeclareMathOperator{\Ext}{Ext}
\DeclareMathOperator{\Tor}{Tor}
\DeclareMathOperator{\Conf}{Conf}
\DeclareMathOperator{\UConf}{UConf}



%Row operations
\newcommand{\elem}[1]{% elementary operations
\xrightarrow{\substack{#1}}%
}

\newcommand{\lelem}[1]{% elementary operations (left alignment)
\xrightarrow{\begin{subarray}{l}#1\end{subarray}}%
}

%SS
\DeclareMathOperator{\supp}{supp}
\DeclareMathOperator{\Var}{Var}

%NT
\DeclareMathOperator{\ord}{ord}

%Alg
\DeclareMathOperator{\Rad}{Rad}
\DeclareMathOperator{\Jac}{Jac}

%Misc
\newcommand{\SL}{{\mathrm{SL}}}
\newcommand{\mobgp}{{\mathrm{PSL}_2(\mathbb{C})}}
\newcommand{\id}{{\mathrm{id}}}
\newcommand{\MCG}{{\mathrm{MCG}}}
\newcommand{\PMCG}{{\mathrm{PMCG}}}
\newcommand{\SMCG}{{\mathrm{SMCG}}}
\newcommand{\ud}{{\mathrm{d}}}
\newcommand{\Vol}{{\mathrm{Vol}}}
\newcommand{\Area}{{\mathrm{Area}}}
\newcommand{\diam}{{\mathrm{diam}}}
\newcommand{\End}{{\mathrm{End}}}


\newcommand{\reg}{{\mathtt{reg}}}
\newcommand{\geo}{{\mathtt{geo}}}

\newcommand{\tori}{{\mathcal{T}}}
\newcommand{\cpn}{{\mathtt{c}}}
\newcommand{\pat}{{\mathtt{p}}}

\let\Cap\undefined
\newcommand{\Cap}{{\mathcal{C}}ap}
\newcommand{\Push}{{\mathcal{P}}ush}
\newcommand{\Forget}{{\mathcal{F}}orget}




\begin{document}


\subsection{Cohomology in terms of Homological Algebra}

    Recall the Universal Coefficient Theorem for Cohomology:

    \begin{theorem}[Universal Coefficient Theorme for Cohomology]
        Let $R$ be a ring and $A$ an $R-$ module. Let
        $C_*$ be a complex of projective $R$-modules
        such that the subcomplex of boundaries 
        $B_*$ is also a complex of projective modules.
        \begin{enumerate}
            \item For all $n$, there is a SES
                \[
                0 \to \Ext_R^{1} \left( 
                H_{n-1}(C_*),A\right) \stackrel{\lambda_n}{\to} 
                H^{n} \left( \Hom_R \left( C_*,A \right)  \right) 
                \stackrel{\mu_n}{\to} \Hom_R \left( 
                H_n \left( C_* \right), A \right) \to 0
                \] 
                where both $\lambda_n$ and $\mu_n$ are
                natural in $C_*$ and $A$.
            \item If $R$ is a PID, then the SES in (1) is
                split, but it is not always naturally split.
        \end{enumerate}
    \end{theorem}

    Also recall the basic properties:

    \begin{lemma}[]
        For a finitely generated $H$, we have
        \begin{itemize}
            \item $\Ext \left( 
                H \oplus H' , G\right) \cong
                \Ext \left( H, G \right) \oplus
                \Ext(H',G)$
            \item $\Ext (H,G) = 0$ if $H$ is free.
            \item $\Ext (\mathbb{Z} / n, G) \cong
                G / n G$.
        \end{itemize}
    \end{lemma}

    \begin{corollary}
        If the homology groups $H_n$ and
        $H_{n-1}$ of a chain complex
        $C$ of free abelian groups are finitely
        generated, with torsion subgroups
        $T_n \subset H_n$ and 
        $T_{n-1} \subset H_{n-1}$, then
        $H^{n}( \Hom_\mathbb{Z} (C_*, \mathbb{Z}) ) \cong
        \left( H_n / T_n \right) \oplus T_{n-1}$.
    \end{corollary}

    \begin{proof}
        By the Universal Coefficient theorem for cohomology, we
        have that
        \[
        H^{n} \left( 
        \Hom_\mathbb{Z} \left( C_*, \mathbb{Z} \right) \right) 
        \cong
        \Ext_{\mathbb{Z}}^{1} 
        \left( H_{n-1}(C_*), \mathbb{Z} \right) 
        \oplus \Hom_{\mathbb{Z}} \left( 
        H_n \left( C_*\right), \mathbb{Z} \right) 
        \] 
        Now,
        $\Ext_{\mathbb{Z}}^{1} \left( H_{n-1}(C_*),\mathbb{Z}
        \right) \cong T_{n-1}$ and
        $\Hom_{\mathbb{Z}} \left( H_n
        \left( C_* \right), \mathbb{Z} \right) 
        \cong H_n / T_n$.
    \end{proof}

    \begin{proposition}[]
        If a chain map between chain complexes of free
        abelian groups induces an isomorphism on homology
        groups, then it induces an isomorphism
        on cohomology groups with any coefficient
        group $G$.
    \end{proposition}

    \begin{proof}
        Suppose
        $\alpha \colon C_* \to C_*'$ is the chain map
        such that $\alpha_* \colon
        H_n (C_*) \to H_n (C_*')$ is an isomorphism
        for all $n$.
        Consider the diagram

        \begin{equation*}
        \begin{tikzcd}
            0 \ar[r] & 
            \Ext \left( H_{n-1}(C),G \right) 
            \ar[r] & H^{n}(C;G) \ar[r, "h"] &
            \Hom(H_n(C),G) \ar[r] & 0\\
            0 \ar[r] & 
            \Ext(H_{n-1}(C'),G) \ar[r] \ar[u, "(\alpha_*)^{*}",
            "\cong"']
                     & H^{n}(C';G) \ar[r, "h"] \ar[u, "\alpha^*"] 
                     & \Hom(H_n (C'),G) \ar[r] 
            \ar[u, "(\alpha_*)^{*}", "\cong"'] & 0
        \end{tikzcd}
        \end{equation*}
        which follows form naturality of the
        Universal Coefficient theorem.
        Then by the $5$-lemma, we obtain that
        $\alpha^*$ is an isomorphism also.
        
    \end{proof}



    \subsection{Cohomology of Spaces}


    Define
    $S^{-n} (X;G) :=
    \Hom_{\mathbb{Z}} \left( S_n(X),A \right) $,
    so $S^{*}(X;A)$ is a chain complex.
    We define
    $H^{n}(X;A) :=
    H_{-n}\left( S^{*}(X;A) \right) $, called \textit{singular
    cohomology of $X$ with coefficients in $A$.}\\

    Thus an $n$-cochain $\varphi 
    \in S^{-n}(X;G)$ assigns to each
    $n$-simplex $\sigma \colon \Delta^{n} \to X$ a
    value $\varphi \left( \sigma \right) \in G$.
    Since the $n$-simplicies form a basis for
    $S_n(X)$, these values can be chosen arbitrarily,
    hence $n$-cochains are exactly equivalent to functions
    from singular $n$-simplices to $G$.\\
    The \textit{coboundary map} $\delta
    \colon S^{-n}(X;G) \to S^{-(n+1)}(X;G)$ is the dual
     $\partial^{*}$, so for
     a cochain $\varphi \in 
     S^{-n}(X;G)$, its coboundary $\delta
     \varphi  $ is the composition
     $\delta \varphi  = 
     \partial^{*} \varphi =
     \varphi \circ \partial$, i.e., the composition
     $C_{n+1} (X) \stackrel{\partial}{\to} 
     C_n(X) \stackrel{\varphi }{\to} G$.

     Hence for a singular $(n+1)$-simplex
     $\sigma \colon \Delta^{n+1} \to X$, we have
     \[
     \delta \varphi (\sigma) = 
     \sum_{i} (-1)^{i} 
     \varphi \left( \sigma |_{\left[ v_0, \ldots,
     \widehat{v_i}, \ldots, v_{n+1}\right] } \right).
     \] 
     Since $\delta^2$ is the dual of
     $\partial^2 = 0$, we have
     $\delta^2 = 0$ also, so
     $H^{n}(X;G)$ can be defined as above.\\
     \linebreak
     \begin{note}
         For a cochain $\varphi \in 
         S^{-n}(X;G)$ to be a cocyle means that
         $\delta \varphi = \varphi \partial = 0$, i.e.,
         it means that $\varphi $ vanishes on boundaries.
     \end{note}











    %\printbibliography
\end{document}
