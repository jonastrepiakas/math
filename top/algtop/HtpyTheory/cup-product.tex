\documentclass[reqno]{amsart}
\usepackage{amscd, amssymb, amsmath, amsthm}
\usepackage{graphicx}
\usepackage[colorlinks=true,linkcolor=blue]{hyperref}
\usepackage[utf8]{inputenc}
\usepackage[T1]{fontenc}
\usepackage{textcomp}
\usepackage{babel}
%% for identity function 1:
\usepackage{bbm}
%%For category theory diagrams:
\usepackage{tikz-cd}

%\usepackage[backend=biber]{biblatex}
%\addbibresource{.bib}


\setlength\parindent{0pt}

\pdfsuppresswarningpagegroup=1

\newtheorem{theorem}{Theorem}[section]
\newtheorem{lemma}[theorem]{Lemma}
\newtheorem{proposition}[theorem]{Proposition}
\newtheorem{corollary}[theorem]{Corollary}
\newtheorem{conjecture}[theorem]{Conjecture}

\theoremstyle{definition}
\newtheorem{definition}[theorem]{Definition}
\newtheorem{example}[theorem]{Example}
\newtheorem{exercise}[theorem]{Exercise}
\newtheorem{problem}[theorem]{Problem}
\newtheorem{question}[theorem]{Question}

\theoremstyle{remark}
\newtheorem*{remark}{Remark}
\newtheorem*{note}{Note}
\newtheorem*{solution}{Solution}
\newtheorem*{construction}{Construction}



%Inequalities
\newcommand{\cycsum}{\sum_{\mathrm{cyc}}}
\newcommand{\symsum}{\sum_{\mathrm{sym}}}
\newcommand{\cycprod}{\prod_{\mathrm{cyc}}}
\newcommand{\symprod}{\prod_{\mathrm{sym}}}

%Linear Algebra

\DeclareMathOperator{\Span}{span}
\DeclareMathOperator{\im}{im}
\DeclareMathOperator{\diag}{diag}
\DeclareMathOperator{\Ker}{Ker}
\DeclareMathOperator{\ob}{ob}
\DeclareMathOperator{\Hom}{Hom}
\DeclareMathOperator{\Mor}{Mor}
\DeclareMathOperator{\sk}{sk}
\DeclareMathOperator{\Vect}{Vect}
\DeclareMathOperator{\Set}{Set}
\DeclareMathOperator{\Group}{Group}
\DeclareMathOperator{\Ring}{Ring}
\DeclareMathOperator{\Ab}{Ab}
\DeclareMathOperator{\Top}{Top}
\DeclareMathOperator{\hTop}{hTop}
\DeclareMathOperator{\Htpy}{Htpy}
\DeclareMathOperator{\Cat}{Cat}
\DeclareMathOperator{\CAT}{CAT}
\DeclareMathOperator{\Cone}{Cone}
\DeclareMathOperator{\dom}{dom}
\DeclareMathOperator{\cod}{cod}
\DeclareMathOperator{\Aut}{Aut}
\DeclareMathOperator{\Mat}{Mat}
\DeclareMathOperator{\Fin}{Fin}
\DeclareMathOperator{\rel}{rel}
\DeclareMathOperator{\Int}{Int}
\DeclareMathOperator{\sgn}{sgn}
\DeclareMathOperator{\Homeo}{Homeo}
\DeclareMathOperator{\SHomeo}{SHomeo}
\DeclareMathOperator{\PSL}{PSL}
\DeclareMathOperator{\Bil}{Bil}
\DeclareMathOperator{\Sym}{Sym}
\DeclareMathOperator{\Skew}{Skew}
\DeclareMathOperator{\Alt}{Alt}
\DeclareMathOperator{\Quad}{Quad}
\DeclareMathOperator{\Sin}{Sin}
\DeclareMathOperator{\Supp}{Supp}
\DeclareMathOperator{\Char}{char}
\DeclareMathOperator{\Teich}{Teich}
\DeclareMathOperator{\GL}{GL}
\DeclareMathOperator{\tr}{tr}
\DeclareMathOperator{\codim}{codim}
\DeclareMathOperator{\coker}{coker}
\DeclareMathOperator{\corank}{corank}
\DeclareMathOperator{\rank}{rank}
\DeclareMathOperator{\Diff}{Diff}
\DeclareMathOperator{\Bun}{Bun}
\DeclareMathOperator{\Sm}{Sm}
\DeclareMathOperator{\Fr}{Fr}
\DeclareMathOperator{\Cob}{Cob}
\DeclareMathOperator{\Ext}{Ext}
\DeclareMathOperator{\Tor}{Tor}
\DeclareMathOperator{\Conf}{Conf}
\DeclareMathOperator{\UConf}{UConf}



%Row operations
\newcommand{\elem}[1]{% elementary operations
\xrightarrow{\substack{#1}}%
}

\newcommand{\lelem}[1]{% elementary operations (left alignment)
\xrightarrow{\begin{subarray}{l}#1\end{subarray}}%
}

%SS
\DeclareMathOperator{\supp}{supp}
\DeclareMathOperator{\Var}{Var}

%NT
\DeclareMathOperator{\ord}{ord}

%Alg
\DeclareMathOperator{\Rad}{Rad}
\DeclareMathOperator{\Jac}{Jac}

%Misc
\newcommand{\SL}{{\mathrm{SL}}}
\newcommand{\mobgp}{{\mathrm{PSL}_2(\mathbb{C})}}
\newcommand{\id}{{\mathrm{id}}}
\newcommand{\MCG}{{\mathrm{MCG}}}
\newcommand{\PMCG}{{\mathrm{PMCG}}}
\newcommand{\SMCG}{{\mathrm{SMCG}}}
\newcommand{\ud}{{\mathrm{d}}}
\newcommand{\Vol}{{\mathrm{Vol}}}
\newcommand{\Area}{{\mathrm{Area}}}
\newcommand{\diam}{{\mathrm{diam}}}
\newcommand{\End}{{\mathrm{End}}}


\newcommand{\reg}{{\mathtt{reg}}}
\newcommand{\geo}{{\mathtt{geo}}}

\newcommand{\tori}{{\mathcal{T}}}
\newcommand{\cpn}{{\mathtt{c}}}
\newcommand{\pat}{{\mathtt{p}}}

\let\Cap\undefined
\newcommand{\Cap}{{\mathcal{C}}ap}
\newcommand{\Push}{{\mathcal{P}}ush}
\newcommand{\Forget}{{\mathcal{F}}orget}




\begin{document}
    We will cover the cup product following Hatcher.\\

    \begin{definition}[Cup Product]
        For a ring $R$, let
        $\varphi \in C^{k}(X;R)$ and
        $\psi \in C^{l}(X;R)$. Then
        the \textit{cup product}
        $\varphi \smile \psi \in C^{k+l}(X;R)$ is the cochain
        whose value
        on $\sigma \colon \Delta^{k+l} \to X$ is given
        by
        \[
            \left( \varphi \smile \psi  \right) (\sigma)
            = \varphi \left( \sigma |_{\left[ v_0,\ldots,
            v_k \right] } \right) 
            \psi \left( \sigma|_{\left[ v_k,\ldots,
            v_{k+l} \right] } \right) 
        \] 
        where the right-hand side is the product
        in $R$.
    \end{definition}

    To see that this induces a
    cup product on cohomology, we
    need the following lemma:

    \begin{lemma}[]
        $\delta \left( \varphi \smile
        \psi \right) = 
        \delta \varphi \smile \psi +
        (-1)^{k} \varphi \smile \delta \psi $ 
        for $\varphi \in C^{k}(X;R)$ and
        $\psi \in C^{l}(X;R)$.
    \end{lemma}

    Using the lemma, it is clear
    that the cup product
    of two cocycles is again a cocycles, and
    that the cup product of a
    cocycle and a coboundary, in either order,
    is a coboundary.
    It follows that there is an induced cup product
    \[
    H^{k}(X;R) \times H^{l}(X;R)
    \stackrel{\smile}{\to} H^{k+l}(X;R).
    \] 
    This is associative and distributive since at the level
    of cochains the cup product has these properties.\\
    If $R$ has an identity, then there is an identity
    elements for the cup product, the class
    $1 \in H^{0}(X;R)$ defined by the
    $0$-cocycle taking the value $1$ on each singular
    $0$-simplex.

    \subsubsection{Relative cup product}
    The cup product formula $\left( \varphi \smile
    \psi \right) (\sigma) = 
    \varphi \left( \sigma |_{\left[ v_0, \ldots, v_k \right] 
    }\right) \psi \left( 
    \sigma |_{\left[ v_k, \ldots, v_{k+l} \right] }\right) $ also
    gives relative cup products
    \begin{align*}
        H^{k}(X;R) \times H^{l}(X,A;R)
        &\stackrel{\smile}{\to} H^{k+l}(X,A;R)\\
        H^{k}(X,A;R) \times H^{l}(X;R)
        &\stackrel{\smile}{\to} H^{k+l}(X,A;R)\\
        H^{k}(X,A;R) \times H^{l}(X,A;R)
        &\stackrel{\smile}{\to} H^{k+l}(X,A;R)
    \end{align*}
    since if $\varphi $ or $\psi $ vanishes
    on chains in $A$, then so does
    $\varphi \smile \psi $.\\
    \linebreak
    We can also define an even more general relative cup product
    \[
    H^{k}(X,A;R) \times H^{l}(X,B;R)
    \stackrel{\smile}{\to} H^{k+l}(X, A \cup B;R)
    \] 
    when $A$ and $B$ are open subsets of $X$ or subcomplexes
    of the CW complex $X$.

    \begin{construction}
        The absolute cup product restricts to a cup product
        $C^{k}(X,A;R) \times C^{l}(X,B;R)
        \to C^{k+l}(X,A \sqcup  B; R)$ where
        $C^{n}\left( X,A \sqcup  B;R \right) $ is the subgroup
        of $C^{n}(X;R)$ consisting of cochains vanishing
        on sums of chains in $A$ and chains in $B$.
        If $A$ and $B$ are open in $X$, then
        the inclusions
        $C^{n} (X,A \cup B;R) \hookrightarrow 
        C^{n}(X, A\sqcup B;R)$ induces
        isomorphisms on cohomology:
        
    \end{construction}











    \begin{proposition}[]
        For a map $f \colon X \to Y$, the induced
        map $f^{*} \colon
        H^{n} (Y;R) \to H^{n}(X;R)$ satisfies
        $f^{*}\left( \alpha \smile \beta \right) 
        = f^{*}(\alpha) \smile
        f^{*}(\beta)$, and similarly in the relative
        case.
    \end{proposition}

    \begin{theorem}[]
        The identity $\alpha \smile \beta =
        (-1)^{kl} \beta \smile \alpha$ holds
        for all $\alpha \in H^{k}\left( X,A;R \right) $ and
        $\beta \in H^{l}(X,A;R)$, when $R$ is commutative.
    \end{theorem}

    \section{The Cohomology Ring}

    Since the cup product is associative and distributive,
    it is natural to try to make it the
    multiplication in a ring structure on
    the cohomology groups of a space $X$.
    This is easy to do if we define
    $H^{*}(X;R) = \bigoplus_{k \in \mathbb{Z}} H^{k}(X;R)$.
    That is, if we define $H^{*}(X;R)$ as the
    direct sum of the cohomology
    groups of the space. Then
    elements of $H^{*}(X;R)$ are
    finite sums
    $\Sigma_i \alpha_i$ with
    $\alpha_i \in H^{i}(X;R)$ and the
    product of two such sums
    is defined to be
    $\left( \Sigma_i \alpha_i \right) 
    \left( \Sigma_j \beta_j \right) =
    \Sigma_{i,j} \alpha_i \beta_j$.
    
    \begin{exercise}[]
        Show that this makes $H^{*}(X;R)$ into a ring,w
        with identity if $R$ has an identity.
        Similarly for $H^{*}(X,A;R)$ with the relative
        cup product.

        Taking scalar multiplication by elements
        of $R$ into account, these rings
        can also be regarded as $R$-algebras.
    \end{exercise}

    \begin{example}[]
        Recall that
        $H^{k}(\mathbb{R}\mathbb{P}^{2};\mathbb{Z}_2)
        \cong \mathbb{Z}_2$ for
        $k = 0, 1,2$ and is $0$ otherwise.
        Also by example 3.8 in Hatcher on Cohomology,
        for a generator
        $\alpha \in H^{1}(\mathbb{R}\mathbb{P}^2;\mathbb{Z}_2)$,
        $\alpha^2 
        =\alpha \smile \alpha$ is a generator
        of $H^{2}(\mathbb{R}\mathbb{P}^2; \mathbb{Z}_2)$,
        hence
        $H^{*}(\mathbb{R}\mathbb{P}^2;\mathbb{Z}_2) \cong
        \mathbb{Z}_2 \left[ \alpha \right] /
        \left( \alpha^3 \right) $.
    \end{example}

    Adding cohomology classes of different dimensions
    to form $H^{*}(X;R)$ is convenient, but it has
    little topological significance. One
    can always regard the
    cohomology ring as a \textit{graded ring}:

    \begin{definition}[Graded Ring]
        A ring $A$ with a decomposition
        $\bigoplus_{k\ge 0}A_k$  into additive
        subgroups $A_k \le A$ such that the
        multiplication takes
        $A_k \times A_l$ to $A_{k+l}$ is called
        a \textit{graded ring}.
    \end{definition}

    To indicate that $\alpha \in A$ lies
    in $A_k$, we write
    $\left| a \right| = k$.

    \begin{definition}[Degree/dimension]
        The number $\left| a \right| $ 
        is called the \textit{degree} or 
        \textit{dimension} of $a$.
    \end{definition}

    \begin{definition}[Commutative/anticommutative/graded commutative]
        A graded ring satisfying the commutativity
        property that
        $ab = (-1)^{\left| a \right| \left| b \right| } ba$ 
        is usually called \textit{commutative} or any of
        the following less ambiguous terms:
        \textit{graded commutative},
        \textit{anticommutative}, or 
        \textit{skew commutative}.
    \end{definition}

    \begin{example}[Polynomial Rings]
        An example of a graded ring is
        $R \left[ \alpha \right] $ or
        the truncated version: $R \left[ \alpha \right] /
        \left( \alpha^n \right) $.

        We have seen that
        $H^{*}\left( \mathbb{R}\mathbb{P}^{2}; \mathbb{Z}/2 \right) 
        \cong \mathbb{Z}/2 \left[ \alpha \right] /
        \left( \alpha^{3} \right) $.
        More generally, we can show that
        $H^{*}\left( \mathbb{R}\mathbb{P}^{n};
        \mathbb{Z}/2\right) \cong
        \mathbb{Z}/2 \left[ \alpha \right] /
        \left( \alpha^{n+1} \right) $ and
        $H^{*}\left( \mathbb{R}\mathbb{P}^{\infty};
        \mathbb{Z}/2\right) \cong
        \mathbb{Z}/2 \left[ \alpha \right] $, where,
        in these cases, $\left| \alpha \right| = 1$.
    \end{example}

    \begin{example}[Exterior Algebras]
        The \textit{exterior algebra}
        $\Lambda_R \left[ \alpha_1, \ldots, \alpha_n \right] $ 
        over a commutative ring $R$ with identity
        is the free $R$-module with basis
        the finite products
        $\alpha_{i_1} \cdots \alpha_{i_k},
        i_1 < \ldots < i_k$, with associative,
        distributive multiplication defined by
        the rules
        $\alpha_i \alpha_j = - \alpha_j \alpha_i$ for
        $i\neq j$ and
        $\alpha_i^2 = 0$ for all $i$.
        The empty producto f $\alpha_i$ 's is the identity
        element $1$ in
        $\Lambda_R \left[ \alpha_1, \ldots, \alpha_n \right] $.

        In view of
        $\alpha_i \alpha_j = - \alpha_j \alpha_i$,
        the exterior algebra becomes an
        anticommutative graded ring by specifying
        odd dimensions for the generators
        $\alpha$.

        By the Künneth formula, we have
        \[
        H^{*}\left( S^{k_1} \times 
        \ldots \times S^{k_n} ; \mathbb{Z}\right) 
        \cong
        H^{*}(S^{k_1} ; \mathbb{Z}) \otimes_{\mathbb{Z}} \ldots
        \otimes_{\mathbb{Z}} H^{*}(S^{k_n};\mathbb{Z})
        \cong \Lambda_{\mathbb{Z}}
        \left[ \alpha_1, \ldots, \alpha_n \right] 
        \] 
        when all the $k_i$ are odd,
        since the first isomorphism
        is given by the cross product.

        When some $k_i$ 's are even, one obtains
        the tensor product of an exterior algebra
        for the odd-dimensional speheres and truncated
        polynomial rings
        $\mathbb{Z} \left[ \alpha \right] /
        \left( \alpha^2 \right) $ for the even
        dimensional spheres.
    \end{example}

    \subsection{The Cross Product}
    
    \begin{definition}[First definition of cross product,
        external cup product]
        We define the \textit{cross product}, or
        \textit{external cup product} as it is
        sometimes called, by
        the map
        \[
        H^{*}(X;R) \times H^{*}(Y;R) \stackrel{\times }{\to} 
        H^{*}\left( X \times Y; R \right) 
        \] 
        given by
        $a \times b = p_1^{*}(a) \smile p_2^{*}(b)$ where
        $p_1, p_2$ are the projections of
        $X \times Y$ onto $X$ and $Y$, respectively.
    \end{definition}

    \begin{definition}[Cross Product, second definition]
    Since the cup product is distributive, the cross
    product is bilinear, hence it induces an
    $R$-module homomorphism

    \begin{equation*}
    \begin{tikzcd}
        H^{*}(X;R) \times H^{*}(Y;R) \ar[dr, "\times "]
        \ar[d] &\\
        H^{*}(X;R) \otimes_R H^{*}(Y;R) 
        \ar[r, "\times "] & H^{*}(X \times Y;R)
    \end{tikzcd}
    \end{equation*}
    which we also call the cross product, given
    by $a \otimes b \mapsto a \times b$.
    \end{definition}

    This module homomorphism becomes
    a ring homomorphism if we define
    the multiplication in a tensor product
    of graded rings by
    $\left( a \otimes b \right) \left( c \otimes d \right) 
    = (-1)^{\left| b \right| \left| c \right| }
    ac \otimes bd$ where
    $\left| x \right| $ denotes the dimension of $x$.
    
    This can be seen as follows (note that
    $ac = a \smile c$ and $bd = b \smile d$ ):
    \begin{align*}
        \mu \left( \left( a \otimes b \right) 
        \left( c \otimes d \right) \right) 
        &= (-1)^{\left| b \right| \left| c \right| }
        \mu \left( ac \otimes bd \right) \\
        &= (-1)^{\left| b \right| \left| c \right| }
        (a \smile c) \times (b \smile d)\\
        &= (-1)^{\left| b \right| \left| c \right| }
        p_1^{*} (a \smile c) \smile
        p_2^{*}\left( b \smile d \right) \\
        &= (-1)^{\left| b \right| \left| c \right| }
        p_1^{*}(a) \smile p_1^{*}(c) \smile
        p_2^{*}(b) \smile p_2^{*}(d)\\
        &= p_1^{*}(a) \smile p_2^{*}(b) \smile
        p_1^{*}(c) \smile p_2^{*}(d)\\
        &= \left( a \times b \right) \left( c
        \times d\right) = \mu (a \otimes b) 
        \mu (c \otimes d)
    \end{align*}



    \begin{theorem}[]
        The cross product $H^{*}(X;R) \otimes_R
        H^{*}(Y;R) \to H^{*}(X \times Y; R)$ is an
        isomorphism of rings
        if $X$ and $Y$ are CW complexes and
        $H^{k}(Y;R)$ is a finitely generated
        free $R$-module for all $k$.
    \end{theorem}
    




    %\printbibliography
\end{document}
