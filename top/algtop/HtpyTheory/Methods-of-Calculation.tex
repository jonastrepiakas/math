\section{Methods of Calculation}


\subsection{Excision for Homotopy Groups}

\begin{theorem}[]\label{Thm:Homotopy-Excision}
    Let $X$ be a CW complex decomposed as the union of
    subcomplexes $A$ and $B$ with nonempty connected
    intersection $C = A \cap B$. If
    $(A,C)$ is $m$-connected and
    $\left( B,C \right) $ is $n$-connected,
    $m,n \ge 0$, then the map
    $\pi_i (A,C) \to \pi_i (X,B)$ induced
    by inclusion is an isomorphism for
    $i < m+n$ and a surjection for
    $i = m+n$.
\end{theorem}

\begin{corollary}[Freudenthal Suspension
    Theorem]\label{Thm:Freudenthal-Suspension}
    The unreduced suspension map
    $\pi_i (S^{n}) \to 
    \pi_{i+1} \left( S^{n+1} \right) $, induced
    by the suspension map $S^{n} \to 
    \Sigma S^{n} \cong S^{n+1}$, is an
    isomorphism for
    $i < 2n-1$ and a surjection for
    $i = 2n-1$. More generally, this holds for
    the suspension
    $\pi_i (X) \to \pi_{i+1}\left( \Sigma X \right) $ whenever
    $X$ is an $(n-1)$-connected CW complex.
\end{corollary}

\begin{proof}[Proof of Corollary]
    Decompose the unreduced suspension
    $\Sigma X = 
    (X \times I) / \left( 
    X \times \left\{ 0 \right\} ,
X \times \left\{ 1 \right\} \right) $ as the
    union of two cones
    $C_+ X$ and $C_{-}X$ intersecting in a 
    copy of $X$.
    Recall that a map
    $f\colon X \to Y$ induces a suspended map
    $\Sigma f \colon \Sigma X \to \Sigma Y$. Now,
    if we consider 
    $f$ to be any map
    $f \colon \left( S^{n},s_0 \right) \to 
    \left( X, x_0 \right) $, then
    we have
    a suspended map
    \begin{equation*}
    \begin{tikzcd}
        S^{n} \times I \ar[r, "f \times \id"]
        \ar[d] & X \times I \ar[d] \\
        S^{n+1} \cong \Sigma S^{n} \ar[r, "\Sigma f"] & \Sigma X
    \end{tikzcd}
    \end{equation*}
    So, in particular,
    $\Sigma f$ is some class
    in $\pi_{n+1} \left( \Sigma X \right) $.
    Define the suspension homomorphism
    $\pi_i (X) \to \pi_{i+1}\left( \Sigma X \right) $ 
    to be the map that sends
    $f$ to $\Sigma f$. This is a homomorphism (why?).

    The unreduced suspension map is the same
    as the map
    \[
    \pi_i (X) \cong
    \pi_{i+1} \left( C_+ X, X \right) 
    \to \pi_{i+1} (\Sigma X, C_{-}X) \cong
    \pi_{i+1} \left( \Sigma X \right) .
    \] 
    (why?) where the two isomorphisms come
    from the LES of pairs and the middle
    map is induced by inclusion.
    The first map
    $\pi_i (X) \to \pi_{i+1}(C_+ X, X)$ 
    takes a map
    $\left( I^{i}, \partial I^{n} \right) 
    \to \left( X, x_0 \right) $ 
    to the map
    $\left( I^{n+1}, \partial I^{n+1},
    J^{n}\right) \to \left( C_+ X, X, x_0 \right) $
    constructed by extending
    the given map radially to correspond with the
    height of $C_{+}X$. So one face of
    $I^{n+1}$ will be mapped to the vertex of
    $C_{+}X$.\\

    Including this into
    $\left( \Sigma X, C_{-}X \right) $ gives
    the middle homomorphism, and
    then the map
    $\pi_{i+1} \left( \Sigma X, C_{-}X \right) \to 
    \pi_{i+1}(\Sigma X)$ is simply the identity on
    our map. \\
    From the LES of $\left( C_{\pm }X, X \right) $, we
    see that this pair is $n$-connected
    if $X$ is $(n-1)$-connected. Then
    Theorem \ref{Thm:Homotopy-Excision} gives
    that the middle map is an isomorphism
    for $i+1 < 2n$ and surjective for
    $i+1 = 2n$.

\end{proof}


\begin{corollary}
    $\pi_n\left( S^{n} \right) \cong \mathbb{Z}$, generated
    by the identity map, for all all
    $n\ge 1$. In particular, the degree map
    $\deg \colon \pi_n \left( S^{n} \right) \to \mathbb{Z}$ is an
    isomorphism.
\end{corollary}

\begin{proof}
    From Corollary \ref{Thm:Freudenthal-Suspension}, we
    obtain a sequence of suspension homomorphisms
    \[
    \pi_1 \left( S^{1} \right) \to \pi_2\left( S^2 \right) 
    \to \pi_3 \left( S^3 \right) \to \ldots
    \] 
    where the first map is surjective and all
    subsequent maps are isomorphisms.
    Since $\pi_1 \left( S^{1} \right) \cong \mathbb{Z}$,
    this implies that $\pi_i \left( S^{i} \right) $ 
    is either $\mathbb{Z}$ or a cyclic finite group for
    all  $i\ge 2$.
    Note that there exist basepoint-preserving maps
    $S^{n} \to S^{n}$ for arbitrary degree and since
    degree is a homotopy invariant, this implies that
    $\pi_n \left( S^{n} \right) $ must be infinite for
    all $n\ge 2$ since otherwise there would be
    maps $S^{n} \to S^{n}$ that are based homotopic but
    of different degrees which is a contradiction.\\
    Lastly, we claim that the degree map
    $\pi_n \left( S^{n} \right) \to \mathbb{Z}$ is an
    isomorphism.

    One way to see this is first to note
    that $\deg$ is a homomorphism by
    Problem \ref{pset1-degree}, where we also
    showed it to be surjective.\\
    A different way to show surjectivity is to
    note that

    \begin{proposition}[]
        $\deg \Sigma f = \deg f$, where
        $\Sigma f \colon S^{n+1} \to S^{n+1}$ is the suspension
        of the map $f\colon S^{n} \to S^{n}$.
    \end{proposition}

    \begin{proof}
        Let $C S^{n} = \left( S^{n} \times I \right) /
        \left( S^{n} \times \left\{ 1 \right\}  \right) $ 
        be the cone on $S^{n}$ and let
        $S^{n} = S^{n}\times \left\{ 0 \right\} \subset 
        CS^{n}$ be the base, so $CS^{n} / S^{n} \cong
        \Sigma S^{n}$. First the map
        $f$ induces $Cf \colon \left( CS^{n}, S^{n} \right) 
        \to \left( CS^{n} , S^{n} \right) $ with
        quotient $\Sigma f$. The naturality of the boundary
        maps in the homology LES of the pair $\left( CS^{n},
        S^{n}\right) $ then gives commutativity
        of the diagram
        \begin{equation*}
        \begin{tikzcd}
            \tilde{H}_{n+1}\left( S^{n+1} \right) 
            \ar[r, "\partial", "\cong"']
            \ar[d, "\Sigma f_*"]& \tilde{H}_n(S^{n}) 
            \ar[d, "f_*"]\\
            \tilde{H}_{n+1}(S^{n+1}) \ar[r, "\partial", 
            "\cong"'] & \tilde{H}_n \left( S^{n} \right) 
        \end{tikzcd}
        \end{equation*}
        So if $f_*$ is multiplication by
        $\deg f$, then so is $\Sigma f_*$.
    \end{proof}

    Now since 
    $\gamma_k \colon z \mapsto z^{k}$ is a degree $k$ map of
    $S^{1}$, this shows that $\deg$ is surjective
    as a map $\pi_1\left( S^{1} \right) \to \mathbb{Z}$, and
    now by the above Proposition, the degree
    map on
    $\Sigma \gamma_k \colon S^{n+1} \to S^{n+1}$ has
    degree $k$ also, so by repeated suspension, we
    have degree $k$ maps in all $\pi_n\left( S^{n} \right) $,
    giving surjectivity of
    $\pi_n \left( S^{n} \right) \to \mathbb{Z}$.




\end{proof}



\begin{example}[$\pi_n \left( \bigvee_{\alpha}\label{Example:Wedge-of-Spheres}
    S_{\alpha}^{n} \right) $]
    We want to show that
    $\pi_n \left( \bigvee_{\alpha}S_{\alpha}^{n} \right) $ 
    for $n\ge 2$ is free abelian with basis
    the homotopy classes of the inclusions
    $S_{\alpha}^{n} \hookrightarrow 
    \bigvee_{\alpha} S_{\alpha}^{n}$.\\
    Suppose first that there are only \textit{finitely many}
    summands $S_{\alpha}^{n}$. Then we can regard
    $\bigvee_{\alpha} S_{\alpha}^{n}$ as
    the $n$-skeleton of the product
    $\prod_{\alpha} S_{\alpha}^{n}$, where
    $S_{\alpha}^{n}$ is given the usual CW structure
    and $\prod_{\alpha} S_{\alpha}^{n}$ has the product
    CW structure. (See Hatcher appendix A).\\
    By construction then $\prod_{\alpha} S_{\alpha}^{n}$ 
    has cells only in dimensions a multiple of $n$, so
    the pair $\left( \prod_{\alpha} S_{\alpha}^{n},
    \bigvee_{\alpha} S_{\alpha}^{n} \right) $ is
    $(2n-1)$-connected by Corollary 
    \ref{n-connectedness-geometrically}.
    So from the LES for the pair, we see that
    the inclusion
    $\bigvee_{\alpha} S_{\alpha}^{n}
    \hookrightarrow \prod_{\alpha} S_{\alpha}^{n}$
    induces an isomorphism on homotopy groups
    in dimensions $\le 2n-1$. Next we have
    $\pi_n \left( \prod_{\alpha} S_{\alpha}^{n} \right) 
    \cong \bigoplus_{\alpha} \pi_n \left( S_{\alpha}^{n} \right) 
    \cong \bigoplus_{\alpha} \mathbb{Z}$,
    a free abelian group with basis the inclusions
    $S_{\alpha}^{n} \hookrightarrow \prod_{\alpha}
    S_{\alpha}^{n}$, so pulling this back along the
    isomorphism
    $\pi_{n}\left( \bigvee_{\alpha}S_{\alpha}^{n} \right) 
    \cong \pi_n \left( \prod_{\alpha}S_{\alpha}^{n} \right) $,
    the same is true for $\bigvee_{\alpha}S_{\alpha}^{n}$.
    This proves the claim when there are
    finitely many $S_{\alpha}^{n}$'s.\\
    When there are infinitely many summands
    $S_{\alpha}^{n}$, consider the homomorphism
    $\Phi \colon \bigoplus_{\alpha}
    \pi_n \left( S_{\alpha}^{n} \right) \to 
    \pi_n \left( \bigvee_{\alpha}S_{\alpha}^{n} \right) $ 
    induced by the inclusions
    $S_{\alpha}^{n} \hookrightarrow \bigvee_{\alpha}
    S_{\alpha}^{n}$. Then $\Phi$ is surjective
    since any map $f \colon S^{n} \to 
    \bigvee_{\alpha}S_{\alpha}^{n}$ has compact
    image contained in the wedge sum of finitely many
    $S_{\alpha}^{n}$'s, so by the finite case already proved,
    $\left[ f \right] $ is in the image of $\Phi$.\\
    Similarly, a nullhomotopy of $f$ has compact image
    contained in a finite wedge sum of $S_{\alpha}^{n}$'s,
    so the finite case also implies that $\Phi$ is injective.
\end{example}

\begin{proposition}[]\label{Prop:JXKDAOWJ}
    If a CW pair $\left( X,A \right) $ is $r$-connected
    and $A$ is $s$-connected, with $r,s \ge 0$, then
    the map $\pi_i (X,A) \to \pi_i (X /A)$ induced
    by the quotient map
    $X \to X / A$ is an isomorphism for
    $i \le r +s$ and a surjection for
    $i = r+s+1$.
\end{proposition}

\begin{proof}
    Consider $X \cup  CA$. Since
    $A$ is closed and the inclusion
    $A \hookrightarrow X$ is a cofibration (since these
    are CW complexes), the map
    $h \colon C_{\iota} = X \cup CA \to 
    X / A$ is a homotopy equivalence by Theorem 
    \ref{Thm:2030akKAK}. So we have a commutative
    diagram
    \begin{equation*}
    \begin{tikzcd}
        \pi_i (X,A) \ar[r] & \pi_i \left( X \cup CA, CA \right) 
        \ar[r] & \pi_i \left( X \cup  CA /CA \right) 
        = \pi_i (X / A)\\
               & \pi_i (X \cup  CA) \ar[u, "\cong"] 
               \ar[ru, "\cong"]
    \end{tikzcd}
    \end{equation*}
    where the vertical isomorphism comes from the
    LES of the pair $\left( X \cup CA, CA \right) $.
    Now, applying Theorem \ref{Thm:Homotopy-Excision}
    to $\left( A,B \right) = \left( X, CA \right) $,
    since $\left( X, A \right) $ is
    $r$-connected and
    $\left( CA, A \right) $ is
    $(s+1)$-connected, we find that the
    homomorphism 
    $\pi_i \left( X,A \right) \to 
    \pi_i \left( X \cup CA, CA \right) $ induced
    by the inclusion
    is an isomorphism for
    $i < r+s+1$ and a surjection for
    $i = r+s+1$, which proves the result.
\end{proof}


\begin{example}[Construction of spaces
    with a particular group as $\pi_n$]\label{Example:DJISI02}
    Suppose $X$ is obtained from a wedge of
    spheres $\bigvee_{\alpha} S_{\alpha}^{n}$ by
    attaching cells $e_{\beta}^{n+1}$ via
    basepoint-preserving maps
    $\varphi_{\beta} \colon S^{n} \to 
    \bigvee_{\alpha} S_{\alpha}^{n}, n\ge 2$.
    By cellular approximation, we know that
    $\pi_i (X) = 0$ for $i < n$, and we shall show
    that $\pi_n(X)$ is the quotient of the free
    abelian group
    $\pi_n \left( \bigvee_{\alpha} S_{\alpha}^{n} \right) 
    \cong \bigoplus_{\alpha} \mathbb{Z}$ by
    the subgroup generated by the
    classes $\left[ \varphi_{\alpha} \right] $.
    Any subgroup can be realized in this way, by
    choosing maps $\varphi_{\beta}$ to represent a 
    set of generators for the subgroup.
    Let 
    $X = \left( \bigvee_{\alpha}S_{\alpha}^{n} \right) 
    \bigcup_{\beta } e_{\beta}^{n+1}$.\\
    Then the LES of the pair
    $\left( X, \bigvee_{\alpha} S_{\alpha}^{n} \right) $ gives
    \[
    \pi_{n+1}\left( X, \bigvee_{\alpha}S_{\alpha}^{n} \right) 
    \stackrel{\partial}{\to} \pi_n \left( \bigvee_{\alpha}
    S_{\alpha}^{n} \right) \to 
    \pi_n (X) \to 0.
    \] 
    so 
    \[
    \pi_n (X) \cong \pi_n \left( \bigvee_{\alpha}S_{\alpha}^{n}
    \right) / \im \partial
    \] 
    The quotient
    $X / \bigvee_{\alpha} S_{\alpha}^{n}$ is a wedge
    of spheres $S_{\beta}^{n+1}$, so by Proposition
    \ref{Prop:JXKDAOWJ} and Example \ref{Example:Wedge-of-Spheres},
    the map
     $\pi_{n+1} \left( X, \bigvee_{\alpha}S_{\alpha}^{n} \right) 
     \to \pi_{n+1} \left( X / \bigvee_{\alpha}S_{\alpha}^{n} \right) 
     \cong \pi_{n+1} \left( \bigvee_{\beta}
     S_{\beta}^{n+1} \right) $ 
     is an isomorphism, so
     $\pi_{n+1}\left( X, \bigvee_{\alpha}S_{\alpha}^{n} \right) $ 
     is free with basis the caracteristic
     maps $\varphi_{\beta}$ of the cells $e_{\beta}^{n+1}$.
     The boundary map $\partial$ takes
     these to the classes
     $\left[ \varphi_{\beta} \right] $, so
     the result follows.

\end{example}


\newpage

\subsubsection{Moore Spaces}

\begin{definition}[Moore Space]
    Given an abelian group $G$ and an integer $n\ge 1$,
    a CW complex $X$ such that
    $H_n(X) \cong G$ and $\tilde{H}_i(X) \cong 0$ for
    $i\neq n$, then $X$ is said to be a \textit{Moore space},
    commonly written $M(G,n)$ to indicate the
    dependence on $G$ and $n$. It is
    also sensible to require that $X$ be simply connected when
    $n>1$.
\end{definition}

\begin{lemma}[Existence of Moore spaces]\label{Moore-space-existence}
    For any abelian group $G$ and any integer $n\ge 1$,
    we can construct a $M(G,n)$-space.
\end{lemma}

\begin{proof}
As a simple case, when $G = \mathbb{Z} / m$, we can
take $X$ to be $S^{n}$ with a cell $e^{n+1}$ attached
by a map $S^{n} \to S^{n}$ of degree $m$. More
generally, any finite generated $G$ can be realized
by taking wedges of examples of this type for finite cyclic
summands of $G$ together with copies of $S^{n}$ for
infinite cyclic summands of $G$.\\
\linebreak
For the general nonfinitely generated case, let
$F \to G$ be a homomorphism of a free abelian group
$F$ onto $G$, sending a basis for $F$ onto some
set of generators of $G$. The kernel $K$ of this
homomorphism is a subgroup of a free abelian group, hence
is itself free abelian. Choose
bases $\left\{ x_{\alpha} \right\} $ for $F$ and
$\left\{ \gamma_{\beta} \right\} $ for $K$, and
write $y_{\beta} = \sum_{\alpha} d_{\beta \alpha} x_{\alpha}$.
Let $X^{n} = \bigvee_{\alpha} S_{\alpha}^{n}$, so
$H_n \left( X^{n} \right) \cong F$ . We will construct
$X$ from $X^{n}$ by attaching cells $e_{\beta}^{n+1}$ 
via maps $f_{\beta} \colon S^{n} \to X^{n}$ such that the
composition of $f_{\beta}$ with the projection onto the
summand $S_{\alpha}^{n}$ has degree $d_{\beta \alpha}$. Then
the cellular boundary map $d_{n+1}$ will be the inclusion
$K \hookrightarrow F$, hence $X$ will have the desired
homotopy groups.\\

We let the map $f_{\beta}$ be the
map which maps the complement of
$\sum_{\alpha} \left| d_{\beta \alpha} \right| $ 
disjoint balls in $S^{n}$ to the $0$-cell of $X^{n}$ while
sending $\left| d_{\beta \alpha} \right| $ of the
balls onto the summand $S_{\alpha}^{n}$ by maps of degree
$+1$ if $d_{\beta \alpha}> 0$ or degree
$-1$ if $d_{\beta \alpha}<0$.
By Theorem \ref{Thm:Degree-of-sum-of-maps}, we get
that the composition of
$f_{\beta}$ with the projection onto
the summand $S_{\alpha}^{n}$ has degree
$d_{\beta \alpha}$.\\
This finishes the construction of  a $M(G,n)$ space.
\end{proof}

\begin{corollary}
    For any abelian groups 
    $\left\{ G_{n} \right\}_{n \in \mathbb{N} }$, we
    can construct a space $X$ such that
    $H_n \left( X \right) \cong G_n$ for all $n$.
\end{corollary}

\begin{proof}
    Take the wedge of the Moore spaces $M\left( G_n,n \right) $ for
    $n \in \mathbb{N} $.
\end{proof}



\subsubsection{Eilenberg-MacLane Spaces}

\begin{definition}[Eilenberg-MacLane space, $K (G,n)$]
    A space $X$ having just one nontrivial homotopy
    group $\pi_n (X) \cong G$ is called
    an \textit{Eilenberg-MacLane space} $K(G,n)$. 
\end{definition}


\textit{Construction of Eilenberg-MacLane Spaces:}\\
Given arbitrary $G$ and $n$, and assuming $G$ is abelian
if $n>1$, we can construct a CW complex
$K(G,n)$. To begin, construct the
CW complex $X$ from Example \ref{Example:DJISI02}. Then
$X$ is an $(n-1) $- conneted CW complex of dimension
$n+1$ such that $\pi_n (X) \cong G$ by construction. 
Alternatively, given the existence of Moore spaces
$M(G,n)$ for any $G$ and $n$, we can take a Moore space
$M(G,n)$ and use the Hurewicz isomorphism
to conclude that $\pi_n (X) \cong H_n(X)$.
Hence
we just need to fix all homotopy groups of dimension
greater than $n$. By
Example \ref{Postnikov-Towers}, we can
construct a CW complex $X_n$ containing $X$ as a subcomplex
such that
$\pi_n (X_n) \cong \pi_n (X) \cong G$ while
$\pi_k(X_n) \cong 0$ for all $k \neq  n$.


\begin{example}[Constructing spaces with arbitrary
    (abelian) homotopy groups]
    Recall that
    \[
    \pi_n \left( \prod_{\alpha}X_{\alpha} \right) 
    \cong \prod_{\alpha} \pi_n \left( X_{\alpha} \right) ,
    \] 
    so if we have a sequence
    of abelian groups $\left\{ G_{n_i} \right\}_{i \in I}$, and let
    $X_{n_i}$ denote that $K(G_{n_i}, n_{i})$ space, then
    we find that
    \[
    \pi_k( \prod_{i \in I} X_{n_i})
    \cong \prod_{i \in I} \pi_k \left( X_{n_i} \right) 
    \cong \begin{cases}
        G_{n_i},& k = n_{i} \text{ for some }i \in I\\
        0,& \text{else}
    \end{cases}
    \] 
\end{example}

Having covered the existence of Eilenberg-MacLane spaces, we
now find the following for uniqueness of these spaces:

\begin{proposition}[Uniqueness of Eilenberg-MacLane spaces]\label{Prop:23SUIHD}
    The homotopy type of a $CW$ complex
    $K(G,n)$ is uniquely determined by $G$ and $n$.
\end{proposition}

The proof is based on the following
lemma giving a condition for when homomorphisms
between homotopy groups are induced by some
map:

\begin{lemma}[]\label{Lemma:NDWUAI288}
    Let $X$ be a $CW$ complex of the form
    $\left( \bigvee_{\alpha} S_{\alpha}^{n} \right) 
    \bigcup_{\beta} e_{\beta}^{n+1}$ for some $n\ge 1$.
    Then for every homomorphism $\psi \colon
    \pi_n (X) \to \pi_n (Y)$ with $Y$ path-connected
    there exists a map $f \colon X \to Y$ with
    $f_* = \psi $.
\end{lemma}

\begin{proof}
    The construction of $f$ is as one would expect:
    first let $f$ send the natural basepoint of
    $\bigvee_{\alpha} S_{\alpha}^{n}$ to a chosen
    basepoint $y_0 \in Y$.
    Now for every sphere $S_{\alpha}^{n}$ in $X$,
    we extend $f$ over the sphere via a map
    representing $\psi \left( \left[ i_{\alpha} \right]  \right) 
    $ where $i_{\alpha}$ is the inclusion
    $S_{\alpha}^{n} \hookrightarrow X$. This defines
    $f$ on the $n$-skeleton of $X$ : $f \colon X^{n} \to Y$.
    Since now $f_* \left[ i_{\alpha} \right] =
    \psi \left[ i_{\alpha} \right] $ for all $\alpha$ and
    the $\left[ i_{\alpha} \right] $ generate
    $\pi_n \left( X^{n} \right) $, this
    defines $f_*$ on all of $\pi_n(X^{n})$.

    To extend $f$ over the $(n+1)$-cells, it will suffice
    to show that $f \circ \varphi_{\beta}$ is nullhomotopic,
    where $\varphi_{\beta} \colon S^{n} \to X^{n}$ is the
    attaching map for the $(n+1)$-cell $e_{\beta}^{n+1}$.
    But $f \circ \varphi_{\beta}$ is a representative
    of $f_* \left[ \varphi_{\beta} \right] =
    \psi \left[ \varphi_{\beta} \right] $.
    Thus we have transformed $f_* \left[ \varphi_{\beta} \right] $ 
    into an element in the image of
    $\psi \colon \pi_n(X) \to \pi_n(Y)$, and
    for this, we can use the extra structure of $X$, not just
    $X^{n}$. In $X$, $\left[ \varphi_{\beta} \right] $ is
    trivial via the
    characteristic map of the cell
    $e_{\beta}^{n+1}$, so $\psi \left[ \varphi_{\beta} \right] =
    \psi (0) = 0$, thus indeed
    $f \circ \varphi_{\beta}$ is nullhomotopic.
    Thus we obtain the desired extension
    $f \colon X \to Y$.
    To see that $f_* = \psi $, simply note that
    by cellular approximation, any element
    of $\pi_n(X)$ can be represented as
    an element in $\pi_n(X^{n})$, and on
    $\pi_n(X^{n})$, $f_*$ agrees with $\psi $ by construction.
\end{proof}


\begin{proof}[Proof of Proposition \ref{Prop:23SUIHD}]
    Let $K'$ be any $K\left( G,n \right) $ CW complex, and
    let $K$ be the specific $K(G,n)$ CW complex constructed
    in Example \ref{Example:DJISI02}. In particular,
    $K$ is of the form of 
    Lemma \ref{Lemma:NDWUAI288}. Since
    $\pi_n(K) = \pi_n(Y)$, we can apply Lemma
    \ref{Lemma:NDWUAI288} to obtain a map
    $f \colon K \to K'$ inducing the identity
    on $\pi_n$. Since all other homotopy groups
    of $K$ and $K'$ are trivial, Whitehead's theorem now gives
    that $f$ is a homotopy equivalence. Since
    homotopy equivalence is an equivalence relation, this finishes
    the proof.
\end{proof}


\subsection{The Hurewicz Theorem}

\begin{theorem}[The Little Hurewicz Theorem]\label{Theorem:Little-Hurewicz}
    If a space $X$ is $(n-1)$-connected, $n\ge 2$,
    then $\tilde{H}_i (X) = 0$ for 
    $i < n$ and $\pi_n (X) \cong
    H_n (X)$. If a pair $(X,A)$ is $(n-1)$-connected,
    $n \ge 2$, with $A$ simply connected
    and nonempty, then $H_i (X,A) = 0$ for
    $i < n$ and $\pi_n (X,A) \cong
    H_n (X,A)$.
\end{theorem}

\begin{remark}[]
    This result is, in a sense, the best that we can
    expect. For example, $S^{n}$ has trivial
    homology groups above dimension $n$ but many
    nontrivial homotopy groups in this range
    when $n\ge 2$; and conversely,
    Eilenberg-MacLane spaces such as
    $\mathbb{C}\mathbb{P}^{\infty}$ have trivial
    higher homotopy groups but many nontrivial homology
    groups.
\end{remark}

\begin{proof}
    We may assume $X$ is a CW complex and $(X,A)$ is a CW
    pair by taking $CW$ approximations to $X$ and
    $\left( X,A \right) $. For
    CW pairs, the relative case then reduces
    to the absolute case since
    $\pi_i \left( X,A \right) \cong
    \pi_i (X / A)$ for $i\le n$ by
    Proposition \ref{Prop:JXKDAOWJ}, while
    $H_i(X,A) \cong \tilde{H}_I \left( X / A \right) $ 
    as $A \hookrightarrow X$ is a cofibration of
    a closed subspace.\\
    In the absolute case, using Proposition 
    \ref{Prop:Removing-lower-dim-cells}, we
    can replace $X$ by a homotopy equivalence
    CW complex with  $(n-1)$-skeleton a point, hence
    $\tilde{H}_i(X) = 0$ for
    $i < n$. To show that
    $\pi_n (X) = H_n(X)$, we can take the
    Postnikov truncation $X_{n+1} = \tau_{\le n+1} X$, obtained
    by throwing away all cells of dimension $> n+1$ from
    $X$. Then $X$ has the form
    $\left( \bigvee_{\alpha}S_{\alpha}^{n} \right) \cup_{\beta}
    e_{\beta}^{n+1}$. We may assume that the
    attaching maps $\varphi_{\beta}$ of the
    cells $e_{\beta}^{n+1}$ are basepoint-preserving (we
    can do this since we used Proposition 
    \ref{Prop:Removing-lower-dim-cells} which used
    Proposition \ref{Prop:CW-Approximation} and its
    proof, and in this proof, we could choose
    the cells to be basepoint-preserving.)
    But now by Example \ref{Example:DJISI02}, we get
    that 
    $\pi_n(X) = \coker \left( 
    \pi_{n+1}(X,X^{n}) \to 
\pi_n \left( X^{n} \right) \right) $ which is the cokernel
of a map $\bigoplus_{\beta} \mathbb{Z} \to 
\bigoplus_{\alpha} \mathbb{Z}$.
We claim that this is the
same as the cellular boundary map
$d \colon \tilde{H}_{n+1}\left( X^{n+1} / X^{n} \right) 
\cong H_{n+1}\left( X^{n+1}, X^{n} \right) \to 
H_{n}\left( X^{n}, X^{n-1} \right)\cong 
\tilde{H}_n \left( X^{n} / X^{n-1} \right) $.
For a cell $e_{\beta}^{n+1}$, the coefficients
of $d \left( e_{\beta}^{n+1} \right) $ are
precisely the degrees of the compositions
$q_{\alpha} \varphi_{\beta}$ where
$q_{\alpha}$ collapses all $n$-cells except
$e_{\alpha}^{n}$ to a point.
Similarly, since the isomorphism
$\pi_n \left( S^{n} \right) \cong \mathbb{Z}$ is
obtained by the degree map, and
the map $\pi_{n+1}\left( X / X^{n} \right) 
\cong \pi_{n+1}\left( X , X^{n} \right) \stackrel{\partial}{\to} 
\pi_n \left( X^{n} \right) $ sends
the $(n+1)$-cell $e_{\beta}^{n+1}$ to
$\deg \varphi_{\beta}^{n}$ also (the degree of the attaching
map), by construction. Hence the maps
are the same, so 
$\pi_n (X) \cong \coker \left( 
\pi_{n+1}\left( X,X^{n} \right) \to 
\pi_n \left( X^{n} \right) \right) \cong
\coker d = H_n (X)$ where the last equality holds since
there are no $(n-1)$-cells.


\end{proof}


\begin{corollary}[Homology version of Whitehead's Theorem]
    A map $f \colon X \to Y$ between simply-connected
    CW complexes is a homotopy equivalence if
    $f_* \colon H_n (X) \to H_n(Y)$ is an isomorphism
    for each $n$.
\end{corollary}

\begin{proof}
    By replacing $Y$ with the mapping cylinder $M_f$, we
    may assume $f$ is the inclusion $X \hookrightarrow Y$.
    Since $X$ and $Y$ are simply-connected,
    $\pi_1 \left( Y,X \right) = 0$. The relative Hurewicz
    theorem says that the first nonzero $\pi_n(Y,X)$ is isomorphic
    to the first nonzero $H_n(Y,X)$, but by the LES
    of the pair $(Y,X)$ in homology,
    $H_n(Y,X) \cong 0$ for all $n\ge 0$, so
    also $\pi_n(Y,X) \cong 0$ for all $n\ge 0$, so
     $f$ induces isomorphisms
     $\pi_n(X) \to \pi_n(Y)$ for all $n$. By
     Whitehead's theorem, $f$ is a homotopy equivalence.
\end{proof}

\begin{lemma}[Uniqueness of Moore Spaces]
    The homotopy type of a CW complex Moore 
    space $M(G,n)$ is uniquely determined by $G$ and
    $n$ if $n>1$, so $M(G,n)$ is simply-connected.
\end{lemma}

\begin{proof}
    Let $Y$ be any $M(G,n)$ space and $X $ be the Moore
    space constructed in Lemma \ref{Moore-space-existence}.
    By Hurewicz,
    $G \cong H_n(X) \cong \pi_n(X)$ and
    similarly for $Y$.\\
    Since $X$ is a CW complex of the
    form $\left( \bigvee_{\alpha}S_{\alpha}^{n} \right) 
    \cup_{\beta} e_{\beta}^{n+1}$, it
    follows from Lemma \ref{Lemma:NDWUAI288} that
    there exists a map $f \colon X \to Y$ 
    inducing the identity on homotopy groups.
    We would like to be able to conclude that this
    map then also induces the identity on
    homology, or just an isomorphism, however,
    this requires the stronger Hurewicz theorem.
    Instead, we can fix this problem differently:
    there is an induced map
    $\tilde{f}  \colon X
    \to M_f$, and we have
    that this induces an isomorphism on homotopy groups
    in degrees $\le n$, so
    $\left( M_f, X \right) $ is $n$-connected.
    Since $X$ is simply connected, the Little Hurewicz Theorem
    further gives us that
    $H_{n+1} \left( M_f, X \right) \cong
    \pi_{n+1} \left( M_f, X \right) $, so
    if $\left( M_f, X \right) $ is furthermore
    $(n+1)$-connected, then we would find that
    $f$ also also induces an isomorphism on
    $H_n$ by the LES.
    So see this, we can first attach
    $(n+2)$-cells to $Y$ so that
    $\pi_{n+1}(Y) = 0$. With this $Y$,
    $\left( M_f,X \right) $ becomes $(n+1)$-connected, so
    for the enlarged $Y$, $f$ induces an
    isomorphism on $H_n$, and
    now we can note that
    adding $(n+2)$-cells to $Y$ did not affect
    the homology of $Y$ in degree $n$, so
    $f$ induced an isomorphism to begin with.
\end{proof}


\begin{definition}[Hurewicz map]
    Thinking of $\pi_n \left( X,A,x_0 \right) $ for
    $n>0$ as
    $\left[ D^{n},\partial D^{n}, s_0 ; X,A,x_0 \right] $, the
    \textit{Hurewicz map} $h \colon
    \pi_n \left( X,A,x_0 \right) \to 
    H_n (X,A)$ is defined by
    $h \left( \left[ f \right]  \right) 
    = f_* \left( \alpha \right) $ where
    $\alpha$ is a fixed generator
    of $H_n \left( D^{n}, \partial D^{n} \right) \cong
    \mathbb{Z}$, and
    $f_* \colon H_n \left( D^{n},\partial D^{n} \right) 
    \to H_n (X,A)$ is induced by $f$.
\end{definition}

If we have a homotopy
    $f \simeq g$ through maps $\left( D^{n},
    \partial D^{n}, s_0\right) \to \left( X,A,x_0 \right) $,
    through maps
    $\left( D^{n},\partial D^{n}, s_0 \right) 
    \to \left( X,A,x_0 \right) $ or even
    through maps $\left( D^{n},\partial D^{n} \right) \to 
    \left( X,A \right) $, we have
    $f_* = g_*$, so $h$ is well-defined.

    \begin{proposition}[]
        The Hurewicz maps $h \colon 
        \pi_n \left( X,A,x_0 \right) \to 
        H_n (X,A)$  is a homomorphism, assuming
        $n>1$ so that $\pi_n \left( X,A,x_0 \right) $ is a group.
    \end{proposition}




