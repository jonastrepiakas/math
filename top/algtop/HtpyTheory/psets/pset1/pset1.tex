
\subsection{Problem set 1}



\subsubsection{Exercises}


\begin{exercise}[The action of the fundamental gorup, part
    2]
    Let $X$ be a path-connected, semi-locally simply-connected
    space with basepoint $x$ and $p \colon \tilde{X}\to X$ its
    universal cover. Show that for $n\ge 2$ and
     $\tilde{x} \in X$ with $p (\tilde{x}) = x$, the isomorphism
     $p_* = \pi_n (p) \colon \pi_n \left( \tilde{X}, \tilde{x} \right) 
     \cong \pi_n(X, x)$ allows us to identify the
     action of $\pi_1 \left( X, x \right) $ on $\pi_n (X, x)$ with
     the action of $\pi_1 \left( X, x \right) $ on
     $\pi_n \left( \tilde{X}, \tilde{x} \right) $ induced by
     the group of deck transformations, i.e., the natural action
     of $\pi_1 (X, x)$ on $\tilde{X}$. In particular, make the
     statement precise.
\end{exercise}

\begin{proof}
    We want to show that for $\left[ \gamma \right] 
    \in \pi_1(X, x)$ and
    $\left[ f \right] \in \pi_n \left( X, x \right) $,
    if $\tilde{g}$ is the lift for $\gamma$ starting at
    $\tilde{x}_0$, and
    $\tilde{f} \colon \left( S^{n}, s_0 \right) 
    \to \left( \tilde{X}, \tilde{x}_0 \right) $ is the
    lift of $f$, then
    $p_* \left( \tilde{\gamma} \tilde{f} \right) 
    = \gamma f$. But this follows directly from how
    $\tilde{\gamma} \tilde{f}$ and 
    $\gamma f$ we constructed. Namely, applying
    $p$ to the square used in the definition, we see that we
    obtain $\gamma f$ from $\tilde{\gamma} \tilde{f}$ since
    $p \circ \tilde{\gamma} = \gamma$ and
    $p \circ \tilde{f} = f$.

\end{proof}


\begin{exercise}[]
    Let $X$ and $Y$ be pointed spaces and $n \ge 2$. Show that the
    inclusion $X \vee Y \hookrightarrow X \times Y$ induces
    a surjection $\pi_n \left( X \vee Y \right) 
    \to \pi_n \left( X \times Y \right) $ for all $n$. Furthermore,
    this exhibits $\pi_n \left( X \times Y \right) $ as
    a retract of $\pi_n \left( X \vee Y \right) $ for all $n$.
    (Is this also true for $n=1$?)
\end{exercise}

\begin{proof}
    a
\end{proof}






\subsubsection{Problems}

    \begin{problem}[]
        Fix an isomorphism $H_n\left( S^{n} \right) \cong \mathbb{Z}$.
        We define the degree $\deg f$ of a map
        $f \colon S^{n} \to S^{n}$ to be the integer such that
        $f_* \colon H_n(S^{n}) \to H_n(S^{n})$ sends $1$ to
        $\deg f \in \mathbb{Z}$.
        \begin{enumerate}
            \item Show that taking the degree of a map
                $S^{n} \to S^{n}$ induces a well-defined
                map
                \[
                \deg \colon \pi_n(S^{n}) \to \mathbb{Z}
                \] 
            \item Show that $\deg $ is a group homomorphism.
            \item Show that the map $\deg$ is surjective.
            \item Suppose that $n\ge 2$. Show that
                $\pi_n\left( S^{n} \right) \cong
                \mathbb{Z} \times A$ for some abelian group $A$.
        \end{enumerate}
    \end{problem}

        \begin{proof}


        \begin{enumerate}
            \item Let
                $ \left[ f \right] 
                \in \pi_n \left( S^{n} \right) $ and
                suppose $f, f'$ are two representatives of
                this class. Then
                $f$ and $f'$ are homotopic by definition,
                so $f_* = \left( f' \right)_* \colon
                \mathbb{Z} = H_n\left( S^{n} \right) \to 
                H_n\left( S^{n} \right) = \mathbb{Z} $ are equal.
                In particular,
                $\deg f = f_*(1) = \left( f' \right)_* (1)
                = \deg f'$. So the map is well-defined.
            \item To show that degree is a group homomorphism,
                we must show that
                $\deg \left( f+g \right) = 
                \deg f + \deg g$.

                For this, we will show a couple of results.

                \begin{proposition}[]
                    Let $X = S_1^{n} \vee \ldots \vee
                    S_k^{n}$ for $n > 0$. Then
                    the homomorphism 
                    $H_n\left( S_1^{n} \right) \oplus \ldots
                    \oplus H_n\left( S_k^{n} \right) 
                    \to H_n(X)$ induced by the inclusion maps
                    is an isomorphism whose inverse
                    is induced by the projections
                    $X \to S_i^{n}$.
                \end{proposition}
                
                To prove this proposition, we must show the following
                lemma.

                \begin{lemma}[]
                    Let $X$ be a Hausdorff space and let
                    $x_0 \in X$ be a point having a closed
                    neighborhood $N$ in $X$ of which
                    $\left\{ x_0 \right\} $ is a strong deformation
                    retract. Let $Y$ be a Hausdorff space and
                    let $y_0 \in Y$. Define
                    $X \vee Y = X \times \left\{ y_0 \right\} 
                    \cup \left\{ x_0 \right\} \times Y$.
                    Then the inclusion maps induce
                    isomorphisms 
                    $\tilde{H}_i(X) \oplus \tilde{H}_i(Y)
                    \cong \tilde{H}_i\left( X \vee Y \right) $
                    whose inverse is induced by the projections
                    of $X \vee Y$ to $X$ and $Y$.
                \end{lemma}
                
                \begin{proof}[Proof of lemma]
                    Consider $A = X$ and
                    $U = X - N$ which is open, and
                    $\overline{U} \subset A$. Then by
                    excision,
                    $H_* \left( X \vee Y, X \right) 
                    \cong H_*\left( N \cup Y, N \right) 
                    \cong \tilde{H}_*\left( Y \right) $

                    Consider the LES of the triple
                    $\left( X \vee Y,
                    \left\{ x_0 \right\} \times Y,
                \left\{ x_0 \right\} \times 
            \left\{ y_0 \right\} \right) $. We obtain
            \[
            \ldots \to 
            H_p\left( \left\{ x_0 \right\} \times Y,
            \left( x_0,y_0 \right) \right)
            \stackrel{i_*}{\to}
            H_p\left( X \vee Y, \left( x_0,y_0 \right)  \right) 
            \stackrel{j_*}{\to} H_p\left( X \vee Y,
            \left\{ x_0 \right\} \times Y \right) \to \ldots
            \] 
            Since $\pi_Y \circ i = \id_{
            \left\{ x_0 \right\} \times Y} $, 
            $i_*$ is injective.

            Furthermore, we have
            \[
            H_p\left( X \vee Y, 
            \left( x_0, y_0 \right) \right) 
            \stackrel{\left( \pi_X \right)_* }{\to } 
            H_p \left( \left\{ x_0 \right\} \times Y,
            \left( x_0, y_0 \right) \right) 
            \cong H_p\left( X \vee Y , 
            \left\{ x_0 \right\} \times Y \right) 
            \] 
            so $j_* = \left( \pi_X \right)_*$ under these
            identifications, so, in particular, 
            $j_*$ is surjective. Therefore, our exact sequence
            is a SES:
            \[
            0 \to  H_p\left( Y, pt \right)
            \stackrel{i_*}{\to }
            H_p \left( X \vee Y, pt \right) 
            \stackrel{j_*}{\to }
            \underbrace{H_p \left( X \vee Y, Y \right)}_{
            \cong H_p\left( X, pt \right) } \to 0
            \] 
            It remains to show that this SES is split, but
            since
            $\pi_X \circ \iota_{X} = 
            \id_{\left\{ x_0 \right\} \times X}$,
            we have that $\iota_{X*}$ provides a section.

        \end{proof}

        \begin{proof}[Proof of proposition]
            This follows by induction on the lemma.
        \end{proof}

        Next, suppose that $E_1, \ldots, E_k$ are
        disjoint open subsets of $S^{n}$, each homeomorphic to
        $\mathbb{R}^{n}$ for $n>0$.
        Let $f \colon S^{n} \to Y$ be a map which takes
        $S^{n} - \bigcup E_i $ to $y_0$. Then
        $f$ factors through the quotient space
        $S^{n} / \left( S^{n} - \bigcup E_i  \right) 
        \cong S_1^{n} \vee \ldots \vee S_k^{n}$ where
        $S_i^{n} = S^{n} / \left( S^{n} - E_i \right) $:
        \[
            f \colon S^{n} \stackrel{g}{\to }
            S_1^{n} \vee \ldots \vee S_k^{n}
            \stackrel{h}{\to } Y
        \] 
        Let $\iota_{j} \colon S_j^{n} 
        \hookrightarrow S_1^{n} \vee \ldots \vee
        S_k^{n}$ be the $j$ th inclusion and let
        $p_j \colon S_1^{n} \vee \ldots \vee
        S_k^{n} \to S_j^{n}$ be the $j$ th projection.
        Then by the proposition,
        $\sum_j \iota_{j*} p_{j*} = \id_* \colon
        H_n \left( S_1^{n} \vee \ldots \vee S_k^{n} \right) 
        \to H_n \left( S_1^{n} \vee \ldots \vee
        S_k^{n}\right) $. Let
        $g_j = p_j \circ g \colon S^{n} \to S_j^{n}$ and
        $h_j = h \circ \iota_j \colon
        S_j^{n} \to Y$ and let
        $f_j = h_j \circ g_j \colon  S^{n} \to Y$.
        That is, $f_j$ is the map which is $f$ on
        $E_j$ and maps the complement of $E_j$ to the basepoint
        $y_0$.

        \begin{theorem}[]
            In the above situation, $f_*
            = \sum_{j=1}^{k} f_{j*} \colon
            H_n\left( S^{n} \right) \to H_n(Y)$.
        \end{theorem}

        \begin{proof}[Proof of theorem]
            We have
            $f_* = 
            h_* \circ g_* = 
            \sum_j h_* i_{j*} p_{j*} g_*
            = \sum_j h_{j*} g_{j*}
            = \sum_j f_{j*}$.
        \end{proof}

        Now we get back to showing that
        $\deg \left( f+g \right) = \deg f + \deg g$.

        Note that by way of defining
        $f+g$, this essentially maps $I^{n}$ by
        $f$ on the left half and $g$ on the right half with
        the boundary mapping to the base point $x_0$.
        In particular, this factors through the
        quotient $I^{n} \to I^{n} / \partial I^{n} 
        \cong S^{n}$, where now the two halves can be interpreted
        as, say, the upper and lower hemispheres. In particular,
        the equator is by assumption also mapped to
        $x_0$, so we can quotient further by
        $S^{n} \to S^{n} \vee S^{n}$ by "pinching" the equator
        to a point. This is essentially what the proposition
        above describes. In particular, $f+g$ can be covered
        by the two open hemispheres and maps the equator
        to $x_0$, so by the theorem,
        we have $\left( f+g \right)_*
        = f_* + g_*$, i.e.,
        $\deg (f+g) = \left( f+g \right)_*(1)
        = f_*(1) + g_*(1) = \deg f + \deg g$, as we wanted
        to show.

    \item 
        Next we show that $\deg$ is surjective. 
        First note that
        $\deg \id = \id_* (1) = 1$ by functoriality since
        $\id_* = \id_{H_n(S^{n})}$.
        By functoriality, we thus hit
        all of $\mathbb{Z}$. More precisely,
        $\deg \left( \ast_{n} \id \right) = n$ for
        $n \in \mathbb{N} $ as
        $\deg$ is a homomorphism.
        Also $\deg \left( \ast_{n} (-\id) \right) 
        = -n$ for $n \in \mathbb{N} $ and
        $\deg (c_{x_0}) = 0$, so
        $\deg$ is surjective.
    \item Let $n\ge 2$. We have a SES
        \[
        0 \to \ker \deg \to 
        \pi_n\left( S^{n} \right) \stackrel{\deg}{\to }
        \mathbb{Z} \to 0.
        \] 
        Since $\mathbb{Z}$ is projective, 
        this splits, so
        $\pi_n\left( S^{n} \right) 
        \cong \mathbb{Z} \oplus \ker \deg$.
        But $\ker \deg$ is a subgroup of
        $\pi_n\left( S^{n} \right) $ which is abelian, hence
        is itself abelian.







        \end{enumerate}

    \end{proof}


    \begin{problem}[]
        Fix $n\ge 1$. We say that a space $X$ is
        \textit{$n$-connected} if it is non-empty, path-connected,
        and $\pi_k (X, x) = 0$ for all $1 \le k \le n$ and
        $x \in X$.
        For $\left( X, x_0 \right) $ a pointed, path-connected space,
        show that the following are equivalent:
        \begin{enumerate}
            \item $X$ is $n$-connected.
            \item $\pi_k\left( X,x_0 \right) = 0$ for all
                $1 \le k \le n$.
            \item Every map
                $S^{k} \to X$ can be extended to a map
                $D^{k+1} \to X$ for all $k \le n$.
            \item Every map
                $S^{k} \to X$ is homotopic to a constant
                map for all
                $k \le n$.
        \end{enumerate}
    \end{problem}

    \begin{proof}
        $\left( 1 \implies 2 \right) $: this follows since
        $X$ being $n$-connected means that $\pi_k\left( X,x \right) 
        = 0$ for all $x \in X$ and all $1\le k\le n$,
        hence in particular for $x_0$.\\
        $(2 \implies 3)$ : Let
        $f \colon S^{k} \to X$ be a map. 
        Then $f$ represents some homotopy class
        $\left[ f \right] \in \pi_k\left( X,x_0 \right) $. But
        since $\pi_k(X,x_0) = 0$, $f$ is homotopic to
        the constant map at $x_0 \rel s_0$.
        Let $H \colon S^{k} \times I \to X$ be this homotopy.
        Define
        $\tilde{f} \colon D^{k+1} \to X$ by
        $\tilde{f}(x) = H \left( x, \|x\| \right) $.
        Then $\tilde{f}$ is continuous as a composite of
        continuous maps and
        $\tilde{f}|_{S^{k}}(-) = 
        H\left( - , 1 \right) =
        f(-)$, so $\tilde{f}$ indeed extends $f$.\\
        $(3 \implies 4)$ : Let $f \colon S^{k} \to X$ be
        a map. Extends $f$ to a map
        $\tilde{f} \colon D^{k+1} \to X$. Define now
        a homotopy
        $H \colon S^{k} \times I \to X$ by
        $H\left( x,t \right) =
        \tilde{f}\left( xt \right) $. This is continuous
        and $H\left( x,1 \right) =
        \tilde{f}(x) = f(x)$ while
        $H\left( x,0 \right) = 
        \tilde{f}(0) \in X$ is constant. Hence this gives
        a homotopy between $f$ and $c_{\tilde{f}(0)}$.\\
        \linebreak
        $(4 \implies 3)$ : 
        Let $f \colon S^{k} \to X$ be a given map.
        By assumption, there exists
        a homotopy
        $H \colon S^{k} \times I \to X$ such that
        $H\left( -,1 \right) = f(-)$ and
        $H(-,0) = c$ where $c$ is some constant map
        at a point in $X$.
        But then $H$ factors through the quotient
        \begin{equation*}
        \begin{tikzcd}
            S^{k} \times I \ar[d] \ar[dr, "H"] & \\
            D^{k+1} \ar[r, "\tilde{H}"'] & X
        \end{tikzcd}
        \end{equation*}
        where we identify
        $S^{k} \times \left\{ 0 \right\} $ to a point.
        But then $\tilde{H}|_{S^{k}} (-) = 
        H\left( -, 1 \right) =
        f(-)$, so $\tilde{H}$ extends $f$.\\
        \linebreak
        $(3 \implies 2)$ : 
        Let $\left[ f \right] \in \pi_k(X, x_0)$ and
        $f$ a representative. 
        We want to show that $f$ is homotopic to the
        constant map at $x_0$ relative $\partial I^{k}$.
        Extend $f$ to a map 
        $\tilde{f} \colon D^{k+1} \to X$, and let
        $H \colon S^{k} \times I \to X$ be given by
        $H\left( x,t \right) =
        \tilde{f}\left( ts_0 + (1-t) x \right) $. This gives
        a homotopy between $f$ and the constant map
        at  $x_0$.\\
        \linebreak
        $\left( 2 \implies 1 \right) $ : the only thing that
        requires showing is that
        given that $\pi_k(X, x_0) = 0$ for all $k$, we then
        have $\pi_k \left( X, x \right) = 0$ for all $k$ and
        all $x \in X$. But this is precisely what the given
        hint says we are allowed to assume since
        $X$ is path connected. So we are done.
    \end{proof}





    %\printbibliography
