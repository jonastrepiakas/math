\documentclass[reqno]{amsart}
\usepackage{amscd, amssymb, amsmath, amsthm}
\usepackage{graphicx}
\usepackage[colorlinks=true,linkcolor=blue]{hyperref}
\usepackage[utf8]{inputenc}
\usepackage[T1]{fontenc}
\usepackage{textcomp}
\usepackage{babel}
%% for identity function 1:
\usepackage{bbm}
%%For category theory diagrams:
\usepackage{tikz-cd}

%\usepackage[backend=biber]{biblatex}
%\addbibresource{.bib}


\setlength\parindent{0pt}

\pdfsuppresswarningpagegroup=1

\newtheorem{theorem}{Theorem}[section]
\newtheorem{lemma}[theorem]{Lemma}
\newtheorem{proposition}[theorem]{Proposition}
\newtheorem{corollary}[theorem]{Corollary}
\newtheorem{conjecture}[theorem]{Conjecture}

\theoremstyle{definition}
\newtheorem{definition}[theorem]{Definition}
\newtheorem{example}[theorem]{Example}
\newtheorem{exercise}[theorem]{Exercise}
\newtheorem{problem}[theorem]{Problem}
\newtheorem{question}[theorem]{Question}

\theoremstyle{remark}
\newtheorem*{remark}{Remark}
\newtheorem*{note}{Note}
\newtheorem*{solution}{Solution}



%Inequalities
\newcommand{\cycsum}{\sum_{\mathrm{cyc}}}
\newcommand{\symsum}{\sum_{\mathrm{sym}}}
\newcommand{\cycprod}{\prod_{\mathrm{cyc}}}
\newcommand{\symprod}{\prod_{\mathrm{sym}}}

%Linear Algebra

\DeclareMathOperator{\Span}{span}
\DeclareMathOperator{\im}{im}
\DeclareMathOperator{\diag}{diag}
\DeclareMathOperator{\Ker}{Ker}
\DeclareMathOperator{\ob}{ob}
\DeclareMathOperator{\Hom}{Hom}
\DeclareMathOperator{\Mor}{Mor}
\DeclareMathOperator{\sk}{sk}
\DeclareMathOperator{\Vect}{Vect}
\DeclareMathOperator{\Set}{Set}
\DeclareMathOperator{\Group}{Group}
\DeclareMathOperator{\Ring}{Ring}
\DeclareMathOperator{\Ab}{Ab}
\DeclareMathOperator{\Top}{Top}
\DeclareMathOperator{\hTop}{hTop}
\DeclareMathOperator{\Htpy}{Htpy}
\DeclareMathOperator{\Cat}{Cat}
\DeclareMathOperator{\CAT}{CAT}
\DeclareMathOperator{\Cone}{Cone}
\DeclareMathOperator{\dom}{dom}
\DeclareMathOperator{\cod}{cod}
\DeclareMathOperator{\Aut}{Aut}
\DeclareMathOperator{\Mat}{Mat}
\DeclareMathOperator{\Fin}{Fin}
\DeclareMathOperator{\rel}{rel}
\DeclareMathOperator{\Int}{Int}
\DeclareMathOperator{\sgn}{sgn}
\DeclareMathOperator{\Homeo}{Homeo}
\DeclareMathOperator{\SHomeo}{SHomeo}
\DeclareMathOperator{\PSL}{PSL}
\DeclareMathOperator{\Bil}{Bil}
\DeclareMathOperator{\Sym}{Sym}
\DeclareMathOperator{\Skew}{Skew}
\DeclareMathOperator{\Alt}{Alt}
\DeclareMathOperator{\Quad}{Quad}
\DeclareMathOperator{\Sin}{Sin}
\DeclareMathOperator{\Supp}{Supp}
\DeclareMathOperator{\Char}{char}
\DeclareMathOperator{\Teich}{Teich}
\DeclareMathOperator{\GL}{GL}
\DeclareMathOperator{\tr}{tr}
\DeclareMathOperator{\codim}{codim}
\DeclareMathOperator{\coker}{coker}
\DeclareMathOperator{\corank}{corank}
\DeclareMathOperator{\rank}{rank}
\DeclareMathOperator{\Diff}{Diff}
\DeclareMathOperator{\Bun}{Bun}
\DeclareMathOperator{\Sm}{Sm}
\DeclareMathOperator{\Fr}{Fr}
\DeclareMathOperator{\Cob}{Cob}
\DeclareMathOperator{\Ext}{Ext}
\DeclareMathOperator{\Tor}{Tor}



%Row operations
\newcommand{\elem}[1]{% elementary operations
\xrightarrow{\substack{#1}}%
}

\newcommand{\lelem}[1]{% elementary operations (left alignment)
\xrightarrow{\begin{subarray}{l}#1\end{subarray}}%
}

%SS
\DeclareMathOperator{\supp}{supp}
\DeclareMathOperator{\Var}{Var}

%NT
\DeclareMathOperator{\ord}{ord}

%Alg
\DeclareMathOperator{\Rad}{Rad}
\DeclareMathOperator{\Jac}{Jac}

%Misc
\newcommand{\SL}{{\mathrm{SL}}}
\newcommand{\mobgp}{{\mathrm{PSL}_2(\mathbb{C})}}
\newcommand{\id}{{\mathrm{id}}}
\newcommand{\MCG}{{\mathrm{MCG}}}
\newcommand{\PMCG}{{\mathrm{PMCG}}}
\newcommand{\SMCG}{{\mathrm{SMCG}}}
\newcommand{\ud}{{\mathrm{d}}}
\newcommand{\Vol}{{\mathrm{Vol}}}
\newcommand{\Area}{{\mathrm{Area}}}
\newcommand{\diam}{{\mathrm{diam}}}
\newcommand{\End}{{\mathrm{End}}}


\newcommand{\reg}{{\mathtt{reg}}}
\newcommand{\geo}{{\mathtt{geo}}}

\newcommand{\tori}{{\mathcal{T}}}
\newcommand{\cpn}{{\mathtt{c}}}
\newcommand{\pat}{{\mathtt{p}}}

\let\Cap\undefined
\newcommand{\Cap}{{\mathcal{C}}ap}
\newcommand{\Push}{{\mathcal{P}}ush}
\newcommand{\Forget}{{\mathcal{F}}orget}




\begin{document}
    \begin{problem}[]
        Let $T = S^{1} \times S^{1}$ be the torus and
        $i \colon D^2 \hookrightarrow T$ an embedding
        of the unit disk that is disjoint from $S^{1} \times 
        \left\{ s_0 \right\} $. Define
        $A := \left( S^{1} \times \left\{ s_0 \right\}  \right) 
        \cup i \left( S^{1} \right) \subset T$. 
        Let $x_0 = \left( s_0 , s_0 \right) $ and
        $x_1 \in i \left( S^{1} \right) $.
        \begin{enumerate}
            \item Draw a picture of $\left( X,A \right) $ 
                and the two points $x_0$ and $x_1$.
            \item Construct an explicit bijection of sets
                $\pi_1 \left( T,A, x_1 \right) \cong
                \mathbb{Z}^2 \sqcup \mathbb{Z}$.
            \item Compute the relative homotopy groups
                $\pi_2 \left( T,A, x_0 \right) $ and
                $\pi_2 \left( T,A,x_1 \right) $.
        \end{enumerate}
    \end{problem}

    \begin{problem}[]
        \begin{enumerate}
            \item Compute $\pi_1 \left( S^{1} \vee S^2 \right) $ 
                and describe the universal cover of
                $S^{1} \vee S^2$.
            \item Show that $\pi_2 \left( S^{1} \vee S^2 \right) $ 
                is isomorphic to $\bigoplus_{\mathbb{Z}}\mathbb{Z}$.
            \item Explicitly describe the action of
                $\pi_1 \left( S^{1}\vee S^2 \right) $ on
                $\bigoplus_{\mathbb{Z}} \mathbb{Z} \cong
                \pi_2 \left( S^{1} \vee S^2 \right) $.
        \end{enumerate}
    \end{problem}

    \begin{problem}[]
        Let $\left( X, A, x_0 \right) $ be a pointed pair
        such that the inclusion $i \colon A \to X$ is based
        nullhomotopic (the nullhomotopy preserves the basepoint).
        The goal is to show that for $n\ge 2$, there is an
        isomorphism of groups:
        \[
        \pi_n \left( X,A, x_0 \right) \cong
        \pi_n \left( X, x_0 \right) \times 
        \pi_{n-1} \left( A, x_0 \right) .
        \] 
        \begin{enumerate}
            \item Show that there is an exact sequence of
                groups
                \[
                1 \to \pi_n \left( X, x_0 \right) 
                \stackrel{j_*}{\to } \pi_n \left( X,A,
                x_0\right) \stackrel{\partial_{*}}{\to }
                \pi_{n-1}\left( A, x_0 \right) \to 1.
                \] 
            \item Using a based nullhomotopy 
                $H \colon A \times \left[ 0,1 \right] 
                \to X$, construct a natural group morphism
                \[
                r_* \colon \pi_n\left( X,A,x_0 \right) 
                \to \pi_n (X, x_0)
                \] 
                such that $r_* \circ j_* = 1$.
            \item Show that for any SES of groups
                \[
                1 \to A \stackrel{\alpha}{\to} B
                \stackrel{\beta}{\to} C\to 1.
                \] 
                such that $\alpha$ admits a retraction, there
                is a group isomorphism
                \[
                B \cong A \times C.
                \] 
                Conclude the desired isomorphism.
        \end{enumerate}
    \end{problem}









    %\printbibliography
\end{document}
