\documentclass[reqno]{amsart}

\usepackage[margin=2.5cm]{geometry}
\usepackage[pdftex]{graphicx}
\usepackage[utf8]{inputenc}
\usepackage[T1]{fontenc}
\usepackage{textcomp}
\usepackage{babel}
\usepackage{amsmath, amssymb, amsthm, amscd}
\usepackage[colorlinks=true,linkcolor=blue]{hyperref}
\usepackage{float}
\usepackage{mathrsfs}
%\usepackage{enumitem}
%% for identity function 1:
\usepackage{bbm}
%%For category theory diagrams:
\usepackage{tikz-cd}
%%For code (e.g. python) in latex:
%\usepackage{listings}
%
%Usage: 
%\begin{lstlisting}[language=Python]
%\end{lstlisting}

\newcommand{\incfig}[2][1]{%
\def\svgwidth{#1\columnwidth}
\import{./figures/}{#2.pdf_tex}
}


\theoremstyle{plain}% default
\newtheorem{theorem}{Theorem}[section]
\newtheorem{lemma}[theorem]{Lemma}
\newtheorem{proposition}[theorem]{Proposition}
\newtheorem{corollary}[theorem]{Corollary}


\theoremstyle{definition}
\newtheorem{definition}[theorem]{Definition}
\newtheorem{example}[theorem]{Example}
\newtheorem{exercise}[theorem]{Exercise}
\newtheorem{problem}[theorem]{Problem}


\theoremstyle{remark}
\newtheorem*{remark}{Remark}
\newtheorem*{note}{Note}
\newtheorem*{solution}{Solution}






% figure support
\usepackage{import}
\usepackage{xifthen}
\pdfminorversion=7
\usepackage{pdfpages}
\usepackage{transparent}

\pdfsuppresswarningpagegroup=1

\setlength\parindent{0pt}

\newcommand{\qedwhite}{\hfill \ensuremath{\Box}}

%Inequalities
\newcommand{\cycsum}{\sum_{\mathrm{cyc}}}
\newcommand{\symsum}{\sum_{\mathrm{sym}}}
\newcommand{\cycprod}{\prod_{\mathrm{cyc}}}
\newcommand{\symprod}{\prod_{\mathrm{sym}}}

%Linear Algebra

\DeclareMathOperator{\Span}{span}
\DeclareMathOperator{\Ima}{Im}
\DeclareMathOperator{\diag}{diag}
\DeclareMathOperator{\Ker}{Ker}
\DeclareMathOperator{\ob}{ob}
\DeclareMathOperator{\sk}{sk}
\DeclareMathOperator{\Vect}{Vect}
\DeclareMathOperator{\Set}{Set}
\DeclareMathOperator{\Group}{Group}
\DeclareMathOperator{\Ring}{Ring}
\DeclareMathOperator{\Ab}{Ab}
\DeclareMathOperator{\Top}{Top}
\DeclareMathOperator{\hTop}{hTop}
\DeclareMathOperator{\Htpy}{Htpy}
\DeclareMathOperator{\Cat}{Cat}
\DeclareMathOperator{\CAT}{CAT}
\DeclareMathOperator{\Cone}{Cone}
\DeclareMathOperator{\dom}{dom}
\DeclareMathOperator{\cod}{cod}
\DeclareMathOperator{\Aut}{Aut}
\DeclareMathOperator{\Mat}{Mat}
\DeclareMathOperator{\Fin}{Fin}
\DeclareMathOperator{\rel}{rel}
\DeclareMathOperator{\Int}{Int}
\DeclareMathOperator{\sgn}{sgn}
\newcommand{\SL}{{\mathrm{SL}}}
\newcommand{\mobgp}{{\mathrm{PSL}_2(\mathbb{C})}}
\newcommand{\Hom}{{\mathrm{Hom}}}
\newcommand{\id}{{\mathrm{id}}}
\newcommand{\Mod}{{\mathrm{Mod}}}
\newcommand{\ud}{{\mathrm{d}}}
\newcommand{\Vol}{{\mathrm{Vol}}}
\newcommand{\Area}{{\mathrm{Area}}}
\newcommand{\diam}{{\mathrm{diam}}}

\newcommand{\reg}{{\mathtt{reg}}}
\newcommand{\geo}{{\mathtt{geo}}}

\newcommand{\tori}{{\mathcal{T}}}
\newcommand{\cpn}{{\mathtt{c}}}
\newcommand{\pat}{{\mathtt{p}}}


%Row operations
\newcommand{\elem}[1]{% elementary operations
\xrightarrow{\substack{#1}}%
}

\newcommand{\lelem}[1]{% elementary operations (left alignment)
\xrightarrow{\begin{subarray}{l}#1\end{subarray}}%
}

%SS
\DeclareMathOperator{\supp}{supp}
\DeclareMathOperator{\Var}{Var}

%NT
\DeclareMathOperator{\ord}{ord}

%Alg
\DeclareMathOperator{\Rad}{Rad}
\DeclareMathOperator{\Jac}{Jac}

\DeclareMathAlphabet{\pazocal}{OMS}{zplm}{m}{n}
\newcommand{\unif}{\pazocal{U}}

\begin{document}
    \begin{proposition}[13.19]
        If $f  \colon X \to Y$ is an identification map and $K$ is
        a locally compact Hausdorff space then $f\times 1  \colon
        X \times K \to Y \times K$ is an identification map.
    \end{proposition}

    \begin{proof}
        Let $\pi = \left( f \times 1 \right) $.
        Let $A \subset Y \times K$ such that $\pi^{-1}(A) $ is open.
        Let $\left( x,y \right) \in \pi^{-1}(A)$ and choose open sets
         $U_1,V$ such that $\left( x,y \right) \in U_1 \times V$
         and $U_1 \times \overline{V} \subset \pi^{-1}(A)$.
         Now suppose $u \in f^{-1}\left( f \left( U_1 \right)  \right) $. Then
         $f(u) \times \overline{V} \subset A$, so
         $u \times \overline{V} \subset  \pi^{-1}(A)$ which is open in
         $U_1 \times \overline{V}$. By the tube lemma, we can find
         an open set $U_u$ around $u$ in $X$ such that
         $U_u \times \overline{V} \subset \pi^{-1}(A)$. 
         Let $U_2 = \bigcup_{u \in f^{-1}\left( f(U_1) \right) } U_u$. This is
         open and $U_2 \times \overline{V} \subset \pi^{-1}(A)$. Continuing
         for $U_i$ with $i>2 $ in the same way, we let
         $U = \bigcup_{i \in \mathbb{N}} U_i$. Then
         clearly $U \times \overline{V} \subset \pi^{-1}(A)$.\\
         Now suppose $u' \in f^{-1}\left( f(U) \right) $, so
         there exists $u \in U$ such that $f(u') = f(u)$. But then
         there exists  $i \in \mathbb{N}$ such that
         $u \in U_i$ and hence $u' \in U_{i+1}$ by construction, so
         $u' \in U$. Hence $f^{-1}\left( f\left( U \right)  \right) = U$, so
         $U$ is saturated. Therefore
         $\pi \left( U \times V \right) $ is an open set contained in $A$ which
         contains $\left( f(x), y \right) $. As $f$ is surjective,
         we can for any $(y',y) \in A$ find $(x',y) \in X$ and repeat the above
         to obtain some open set whose image is contained in $A$ and contains 
         $\left( y',y \right) $. Therefore $A$ is open.
     \end{proof}

     \begin{exercise}[13.2]
         If $X,Y$ are normal, $A \subset X$ is closed, and $f  \colon A \to Y$ 
         is a map, show that $Y \cup_f X$ is normal.
     \end{exercise}

     \begin{proof}
        \[
        Y \cup_f X = Y + X / \sim
        \] 

        \subsection{Homotopy}
        Let $C$ denote the constant homotopy, whichever one makes sense
        in the current context, i.e., $C  \colon X \times I \to Y$ is $\rel X$.
        \begin{proposition}[14.13]
            We have $F * C \simeq F \rel X \times \partial I$, and, similarly,
            $C * F \simeq F \rel X \times \partial I$.
        \end{proposition}
        
        \begin{proof}
        We have a map $\varphi_1  \colon \left( I, \partial I \right) \to 
        \left( I, \partial I \right) $ given by
        \[
        \varphi_1 \left( t \right) 
        =
        \begin{cases}
            2t,& t \in \left[ 0, \frac{1}{2} \right] \\
            1,& t\in \left[ \frac{1}{2},1 \right] 
        \end{cases}
        \] 
        Then $\varphi_1$ is continuous, and we have
        \[
        F * C \simeq F * C \left( x, \varphi_1 (t) \right) 
        =
        \] 
        
        \end{proof}
         



     \end{proof}



%\bibliography{refs}
\end{document}
