\documentclass[reqno]{amsart}

\usepackage[margin=2.5cm]{geometry}
\usepackage[pdftex]{graphicx}
\usepackage[utf8]{inputenc}
\usepackage[T1]{fontenc}
\usepackage{textcomp}
\usepackage{babel}
\usepackage{amsmath, amssymb, amsthm, amscd}
\usepackage[colorlinks=true,linkcolor=blue]{hyperref}
\usepackage{float}
\usepackage{mathrsfs}
%\usepackage{enumitem}
%% for identity function 1:
\usepackage{bbm}
%%For category theory diagrams:
\usepackage{tikz-cd}
%%For code (e.g. python) in latex:
%\usepackage{listings}
%
%Usage: 
%\begin{lstlisting}[language=Python]
%\end{lstlisting}

\newcommand{\incfig}[2][1]{%
\def\svgwidth{#1\columnwidth}
\import{./figures/}{#2.pdf_tex}
}


\theoremstyle{plain}% default
\newtheorem{theorem}{Theorem}[section]
\newtheorem{lemma}[theorem]{Lemma}
\newtheorem{proposition}[theorem]{Proposition}
\newtheorem{corollary}[theorem]{Corollary}


\theoremstyle{definition}
\newtheorem{definition}[theorem]{Definition}
\newtheorem{example}[theorem]{Example}
\newtheorem{exercise}[theorem]{Exercise}
\newtheorem{problem}[theorem]{Problem}


\theoremstyle{remark}
\newtheorem*{remark}{Remark}
\newtheorem*{note}{Note}
\newtheorem*{solution}{Solution}






% figure support
\usepackage{import}
\usepackage{xifthen}
\pdfminorversion=7
\usepackage{pdfpages}
\usepackage{transparent}

\pdfsuppresswarningpagegroup=1

\setlength\parindent{0pt}

\newcommand{\qedwhite}{\hfill \ensuremath{\Box}}

%Inequalities
\newcommand{\cycsum}{\sum_{\mathrm{cyc}}}
\newcommand{\symsum}{\sum_{\mathrm{sym}}}
\newcommand{\cycprod}{\prod_{\mathrm{cyc}}}
\newcommand{\symprod}{\prod_{\mathrm{sym}}}

%Linear Algebra

\DeclareMathOperator{\Span}{span}
\DeclareMathOperator{\Ima}{Im}
\DeclareMathOperator{\diag}{diag}
\DeclareMathOperator{\Ker}{Ker}
\DeclareMathOperator{\ob}{ob}
\DeclareMathOperator{\sk}{sk}
\DeclareMathOperator{\Vect}{Vect}
\DeclareMathOperator{\Set}{Set}
\DeclareMathOperator{\Group}{Group}
\DeclareMathOperator{\Ring}{Ring}
\DeclareMathOperator{\Ab}{Ab}
\DeclareMathOperator{\Top}{Top}
\DeclareMathOperator{\hTop}{hTop}
\DeclareMathOperator{\Htpy}{Htpy}
\DeclareMathOperator{\Cat}{Cat}
\DeclareMathOperator{\CAT}{CAT}
\DeclareMathOperator{\Cone}{Cone}
\DeclareMathOperator{\dom}{dom}
\DeclareMathOperator{\cod}{cod}
\DeclareMathOperator{\Aut}{Aut}
\DeclareMathOperator{\Mat}{Mat}
\DeclareMathOperator{\Fin}{Fin}
\DeclareMathOperator{\rel}{rel}
\DeclareMathOperator{\Int}{Int}
\DeclareMathOperator{\sgn}{sgn}
\DeclareMathOperator{\Arg}{Arg}
\DeclareMathOperator{\Arctan}{Arctan}
\DeclareMathOperator{\Arcsin}{Arcsin}
\DeclareMathOperator{\Arccos}{Arccos}
\newcommand{\SL}{{\mathrm{SL}}}
\newcommand{\mobgp}{{\mathrm{PSL}_2(\mathbb{C})}}
\newcommand{\Hom}{{\mathrm{Hom}}}
\newcommand{\id}{{\mathrm{id}}}
\newcommand{\Mod}{{\mathrm{Mod}}}
\newcommand{\ud}{{\mathrm{d}}}
\newcommand{\Vol}{{\mathrm{Vol}}}
\newcommand{\Area}{{\mathrm{Area}}}
\newcommand{\diam}{{\mathrm{diam}}}

\newcommand{\reg}{{\mathtt{reg}}}
\newcommand{\geo}{{\mathtt{geo}}}

\newcommand{\tori}{{\mathcal{T}}}
\newcommand{\cpn}{{\mathtt{c}}}
\newcommand{\pat}{{\mathtt{p}}}


%Row operations
\newcommand{\elem}[1]{% elementary operations
\xrightarrow{\substack{#1}}%
}

\newcommand{\lelem}[1]{% elementary operations (left alignment)
\xrightarrow{\begin{subarray}{l}#1\end{subarray}}%
}

%SS
\DeclareMathOperator{\supp}{supp}
\DeclareMathOperator{\Var}{Var}

%NT
\DeclareMathOperator{\ord}{ord}

%Alg
\DeclareMathOperator{\Rad}{Rad}
\DeclareMathOperator{\Jac}{Jac}

\DeclareMathAlphabet{\pazocal}{OMS}{zplm}{m}{n}
\newcommand{\unif}{\pazocal{U}}

\begin{document}
    \begin{definition}[]
        For $z \in \mathbb{C} - \left\{ 0 \right\} $, we let
        \[
        \arg z = \left\{ \theta \in \mathbb{R}  \mid |z| e^{i \theta} = z \right\} 
        \] 
    \end{definition}

    \begin{definition}[Principal argument]
        We define the principal argument of $z \in \mathbb{C} - \left\{
        0 \right\} $ as the unique element
        \[
            \Arg z \in \arg z \cap (- \pi , \pi ]
        \] 
    \end{definition}

    \begin{definition}[Argument function]
        An argument function for a subset $A \subset \mathbb{C} - \left\{
        0 \right\} $ is a function 
        $\theta  \colon A \to \mathbb{R}$ such that 
        $\theta (z) \in \arg z$ for all $z \in A$.
    \end{definition}

    \begin{lemma}[]
        $\Arg $ is continuous on $\mathbb{C}_\pi = \left\{ r e^{i \pi}  \mid
        r \ge 0 \right\} $.
    \end{lemma}

    \begin{proof}
        We have that $\Arg z$ maps $\mathbb{C}_\pi$ onto $\left( -\pi , \pi
        \right) $, and
        \begin{align*}
            \Arg z &= \Arccos \frac{x}{|z|}, \quad z = x+iy, y > 0\\
            \Arg z &= \Arctan \frac{x}{y}, \quad z = x+iy, x>0\\
            \Arg z &= \Arcsin \frac{y}{|z|}, \quad z = x+iy, y<0
        \end{align*}
        We have that $\Arccos \frac{x}{|z|}$ and
        $\Arctan \frac{x}{y}$ agree on $\left\{ x+iy \in \mathbb{C}  \mid 
        x,y >0 \right\} $ and
        that $\Arctan \frac{x}{y}$ and
        $\Arcsin \frac{y}{|z|}$ agree on
        $\left\{ x+iy \in \mathbb{C}  \mid  x>0, y <0 \right\} $. All these
        are $C^{\infty}$ functions, so in particular continuous, hence
        they define a continuous function on
        $\mathbb{C}_\pi$.
    \end{proof}

    \begin{definition}[Argument function for $\mathbb{C}_{\alpha}$]
       Likewise, if $\alpha \in \mathbb{R}$, we can define
        \[
       \mathbb{C}_{\alpha} = \mathbb{C} - \left\{ 
       r e^{i \alpha}  \mid  r \ge 0 \right\}.
       \] 
       Then we can define 
       \[
       \Arg_{\alpha}  \colon \mathbb{C}_{\alpha} \to \mathbb{R}
       \] 
       by
       \[
       \Arg_{\alpha} (z) = \Arg \left( e^{i \left( \pi - \alpha \right) }z
       \right) + \alpha - \pi.
       \] 
    \end{definition}
    As a composition of continuous maps, $\Arg_{\alpha}$ is continuous
    on $\mathbb{C}_{\pi}$.
    
    \begin{proposition}[]
        There exist a continuous argument function on
        $ A \subset S^{1} \subset \mathbb{C}$ if and only if
        $A \neq S^{1}$.
    \end{proposition}
    
    \begin{proof}
        We first show that there does not exist a continuous argument function
        on $S^{1}$. Suppose there exists a continuous argument function
        $\theta  \colon S^{1} \to \mathbb{R}$. Since
        $S^{1}$ is compact and path-connected, 
        the image of $S^{1}$ under $\theta$ must be a closed interval, $\left[
        a,b\right] $. As $\theta$ is bijective too, it is a homeomorphism. But
        removing any point of $S^{1}$ leaves it path-connected, while removing
        a point in the interior of $\left[ a,b \right] $ leaves it separated,
        and thus not connected. Since connectedness and path-connectedness are
        topological properties, $S^{1}$ is not homeomorphic to $\left[ a,b
        \right] $, so no such argument function exists.\\
        Now, conversely, if $A \neq S^{1}$, then we can pick a point
        $e^{i \alpha} \in S^{1} - A$. Then
        $\Arg_{\alpha}$ is a continuous argument function for $A$.
    \end{proof}


%\bibliography{refs}
\end{document}
