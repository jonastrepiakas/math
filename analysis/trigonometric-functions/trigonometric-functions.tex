\documentclass[reqno]{amsart}

\usepackage[margin=2.5cm]{geometry}
\usepackage[pdftex]{graphicx}
\usepackage[utf8]{inputenc}
\usepackage[T1]{fontenc}
\usepackage{textcomp}
\usepackage{babel}
\usepackage{amsmath, amssymb, amsthm, amscd}
\usepackage[colorlinks=true,linkcolor=blue]{hyperref}
\usepackage{float}
\usepackage{mathrsfs}
%\usepackage{enumitem}
%% for identity function 1:
\usepackage{bbm}
%%For category theory diagrams:
\usepackage{tikz-cd}
%%For code (e.g. python) in latex:
%\usepackage{listings}
%
%Usage: 
%\begin{lstlisting}[language=Python]
%\end{lstlisting}

\newcommand{\incfig}[2][1]{%
\def\svgwidth{#1\columnwidth}
\import{./figures/}{#2.pdf_tex}
}


\theoremstyle{plain}% default
\newtheorem{theorem}{Theorem}[section]
\newtheorem{lemma}[theorem]{Lemma}
\newtheorem{proposition}[theorem]{Proposition}
\newtheorem{corollary}[theorem]{Corollary}


\theoremstyle{definition}
\newtheorem{definition}[theorem]{Definition}
\newtheorem{example}[theorem]{Example}
\newtheorem{exercise}[theorem]{Exercise}
\newtheorem{problem}[theorem]{Problem}


\theoremstyle{remark}
\newtheorem*{remark}{Remark}
\newtheorem*{note}{Note}
\newtheorem*{solution}{Solution}






% figure support
\usepackage{import}
\usepackage{xifthen}
\pdfminorversion=7
\usepackage{pdfpages}
\usepackage{transparent}

\pdfsuppresswarningpagegroup=1

\setlength\parindent{0pt}

\newcommand{\qedwhite}{\hfill \ensuremath{\Box}}

%Inequalities
\newcommand{\cycsum}{\sum_{\mathrm{cyc}}}
\newcommand{\symsum}{\sum_{\mathrm{sym}}}
\newcommand{\cycprod}{\prod_{\mathrm{cyc}}}
\newcommand{\symprod}{\prod_{\mathrm{sym}}}

%Linear Algebra

\DeclareMathOperator{\Span}{span}
\DeclareMathOperator{\Ima}{Im}
\DeclareMathOperator{\diag}{diag}
\DeclareMathOperator{\Ker}{Ker}
\DeclareMathOperator{\ob}{ob}
\DeclareMathOperator{\sk}{sk}
\DeclareMathOperator{\Vect}{Vect}
\DeclareMathOperator{\Set}{Set}
\DeclareMathOperator{\Group}{Group}
\DeclareMathOperator{\Ring}{Ring}
\DeclareMathOperator{\Ab}{Ab}
\DeclareMathOperator{\Top}{Top}
\DeclareMathOperator{\hTop}{hTop}
\DeclareMathOperator{\Htpy}{Htpy}
\DeclareMathOperator{\Cat}{Cat}
\DeclareMathOperator{\CAT}{CAT}
\DeclareMathOperator{\Cone}{Cone}
\DeclareMathOperator{\dom}{dom}
\DeclareMathOperator{\cod}{cod}
\DeclareMathOperator{\Aut}{Aut}
\DeclareMathOperator{\Mat}{Mat}
\DeclareMathOperator{\Fin}{Fin}
\DeclareMathOperator{\rel}{rel}
\DeclareMathOperator{\Int}{Int}
\DeclareMathOperator{\Arctan}{Arctan}
\newcommand{\SL}{{\mathrm{SL}}}
\newcommand{\mobgp}{{\mathrm{PSL}_2(\mathbb{C})}}
\newcommand{\Hom}{{\mathrm{Hom}}}
\newcommand{\id}{{\mathrm{id}}}
\newcommand{\Mod}{{\mathrm{Mod}}}
\newcommand{\ud}{{\mathrm{d}}}
\newcommand{\Vol}{{\mathrm{Vol}}}
\newcommand{\Area}{{\mathrm{Area}}}
\newcommand{\diam}{{\mathrm{diam}}}

\newcommand{\reg}{{\mathtt{reg}}}
\newcommand{\geo}{{\mathtt{geo}}}

\newcommand{\tori}{{\mathcal{T}}}
\newcommand{\cpn}{{\mathtt{c}}}
\newcommand{\pat}{{\mathtt{p}}}



%Row operations
\newcommand{\elem}[1]{% elementary operations
\xrightarrow{\substack{#1}}%
}

\newcommand{\lelem}[1]{% elementary operations (left alignment)
\xrightarrow{\begin{subarray}{l}#1\end{subarray}}%
}

%SS
\DeclareMathOperator{\supp}{supp}
\DeclareMathOperator{\Var}{Var}

%NT
\DeclareMathOperator{\ord}{ord}

%Alg
\DeclareMathOperator{\Rad}{Rad}
\DeclareMathOperator{\Jac}{Jac}

\DeclareMathAlphabet{\pazocal}{OMS}{zplm}{m}{n}
\newcommand{\unif}{\pazocal{U}}

\begin{document}
    \begin{definition}[Tangent]
        We define the tangent function as $\tan  \colon
        \mathbb{C} - \left\{ \frac{\pi}{2} + \pi \mathbb{Z} \right\} 
        \to \mathbb{C}$ by
        \[
        \tan z = \frac{\sin z}{\cos z}
        \] 
    \end{definition}
    The function $\tan z$ is holomorphic on its whole domain
    $\mathbb{C} - \left\{ \frac{\pi}{2} + \pi \mathbb{Z} \right\} $ by
    theorem 1.18 in. \cite{Berg}

    \begin{problem}[Problem 3.5 in \cite{Berg}]
        Use Goursat's lemma to show the following: Let
        $G$ be a star-shaped open set around $z_0$. For every
        $z \in G$, let $\left[ z_0, z \right] $ be the straight
        line from $z_0$ to $z$ parametrized by
        $\gamma (t) = \left( 1-t \right) z_0 + tz$. Show that
        if $f$ is holomorphic on $G$, then
        $F  \colon G \to \mathbb{C}$ defined by
        \[
        F(z) = \int_{\left[ z_0, z \right] } f
        \] 
        is an antiderivative to $f$ which satisfies
        $F(z_0) = 0$.
    \end{problem}
    

    \begin{problem}[Inverse of tangent, Arctan, problem 3.8 in
        \cite{Berg}]
    Consider the path-connected open set
    \[
    G = \mathbb{C} - \left\{ iy  \mid y\in \mathbb{R},
    \left| y \right| \ge 1 \right\} .
    \] 
    Define the function $\Arctan  \colon G \to \mathbb{C}$ by
    \[
    \Arctan z = \int_{0}^{1}  \frac{z dt}{1+ t^2 z^2}. 
    \] 
    Show that $\Arctan$ is holomorphic on $G$ with derivative
    \[
    \frac{d}{dz} \Arctan z = \frac{1}{1+ z^2}.
    \] 
    Show that $\Arctan |_{\mathbb{R}}$ is the inverse
    function to $\tan  \colon \left( -\frac{\pi}{2},\frac{\pi}{2} \right)
    \to \mathbb{R}$ by showing that $\Arctan$ maps $G$ bijectively
    onto the strip
    \[
    \left\{ z = x+iy  \mid 
    -\frac{\pi}{2} < x < \frac{\pi}{2} \right\} ,
    \] 
    and showing that it is inverse to $\tan $.\\
    Now show that
    \[
    \Arctan z = z - \frac{z^3}{3} + \frac{z^{5}}{5} -\ldots 
    \quad \text{for} \quad \left| z \right| <1.
    \] 
    
    
    
    
    \end{problem}

\begin{remark}
    Since holomorphic functions are analytic, $\Arctan|_{\mathbb{R}}$ 
    is in particular smooth, and since $\tan  \colon
    \left( - \frac{\pi}{2}, \frac{\pi}{2} \right) \to \mathbb{R}$ 
    is smooth, we see that both are diffeomorphisms.\\
    Thus we get that the open cube
    $\left( -\frac{\pi}{2},\frac{\pi}{2} \right)^{n}$ is diffeomorphic
    with $\mathbb{R}^{n}$ by the map
    $f (x_1,\ldots, x_n) = \left( \tan x_1, \ldots, \tan x_n \right) $.
\end{remark}










\newpage
\bibliography{trigonometric-functions}
\end{document}
