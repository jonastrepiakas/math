\documentclass[reqno]{amsart}

\usepackage[margin=2.5cm]{geometry}
\usepackage[pdftex]{graphicx}
\usepackage[utf8]{inputenc}
\usepackage[T1]{fontenc}
\usepackage{textcomp}
\usepackage{babel}
\usepackage{amsmath, amssymb, amsthm, amscd}
\usepackage[colorlinks=true,linkcolor=blue]{hyperref}
\usepackage{float}
\usepackage{mathrsfs}
%\usepackage{enumitem}
%% for identity function 1:
\usepackage{bbm}
%%For category theory diagrams:
\usepackage{tikz-cd}
%%For code (e.g. python) in latex:
%\usepackage{listings}
%
%Usage: 
%\begin{lstlisting}[language=Python]
%\end{lstlisting}

\newcommand{\incfig}[2][1]{%
\def\svgwidth{#1\columnwidth}
\import{./figures/}{#2.pdf_tex}
}


\theoremstyle{plain}% default
\newtheorem{theorem}{Theorem}[section]
\newtheorem{lemma}[theorem]{Lemma}
\newtheorem{proposition}[theorem]{Proposition}
\newtheorem{corollary}[theorem]{Corollary}


\theoremstyle{definition}
\newtheorem{definition}[theorem]{Definition}
\newtheorem{example}[theorem]{Example}
\newtheorem{exercise}[theorem]{Exercise}
\newtheorem{problem}[theorem]{Problem}


\theoremstyle{remark}
\newtheorem*{remark}{Remark}
\newtheorem*{note}{Note}
\newtheorem*{solution}{Solution}






% figure support
\usepackage{import}
\usepackage{xifthen}
\pdfminorversion=7
\usepackage{pdfpages}
\usepackage{transparent}

\pdfsuppresswarningpagegroup=1

\setlength\parindent{0pt}

\newcommand{\qedwhite}{\hfill \ensuremath{\Box}}

%Inequalities
\newcommand{\cycsum}{\sum_{\mathrm{cyc}}}
\newcommand{\symsum}{\sum_{\mathrm{sym}}}
\newcommand{\cycprod}{\prod_{\mathrm{cyc}}}
\newcommand{\symprod}{\prod_{\mathrm{sym}}}

%Linear Algebra

\DeclareMathOperator{\Span}{span}
\DeclareMathOperator{\Ima}{Im}
\DeclareMathOperator{\diag}{diag}
\DeclareMathOperator{\Ker}{Ker}
\DeclareMathOperator{\ob}{ob}
\DeclareMathOperator{\sk}{sk}
\DeclareMathOperator{\Vect}{Vect}
\DeclareMathOperator{\Set}{Set}
\DeclareMathOperator{\Group}{Group}
\DeclareMathOperator{\Ring}{Ring}
\DeclareMathOperator{\Ab}{Ab}
\DeclareMathOperator{\Top}{Top}
\DeclareMathOperator{\hTop}{hTop}
\DeclareMathOperator{\Htpy}{Htpy}
\DeclareMathOperator{\Cat}{Cat}
\DeclareMathOperator{\CAT}{CAT}
\DeclareMathOperator{\Cone}{Cone}
\DeclareMathOperator{\dom}{dom}
\DeclareMathOperator{\cod}{cod}
\DeclareMathOperator{\Aut}{Aut}
\DeclareMathOperator{\Mat}{Mat}
\DeclareMathOperator{\Fin}{Fin}
\DeclareMathOperator{\rel}{rel}
\DeclareMathOperator{\Int}{Int}
\DeclareMathOperator{\sgn}{sgn}
\newcommand{\SL}{{\mathrm{SL}}}
\newcommand{\mobgp}{{\mathrm{PSL}_2(\mathbb{C})}}
\newcommand{\Hom}{{\mathrm{Hom}}}
\newcommand{\id}{{\mathrm{id}}}
\newcommand{\Mod}{{\mathrm{Mod}}}
\newcommand{\ud}{{\mathrm{d}}}
\newcommand{\Vol}{{\mathrm{Vol}}}
\newcommand{\Area}{{\mathrm{Area}}}
\newcommand{\diam}{{\mathrm{diam}}}

\newcommand{\reg}{{\mathtt{reg}}}
\newcommand{\geo}{{\mathtt{geo}}}

\newcommand{\tori}{{\mathcal{T}}}
\newcommand{\cpn}{{\mathtt{c}}}
\newcommand{\pat}{{\mathtt{p}}}


%Row operations
\newcommand{\elem}[1]{% elementary operations
\xrightarrow{\substack{#1}}%
}

\newcommand{\lelem}[1]{% elementary operations (left alignment)
\xrightarrow{\begin{subarray}{l}#1\end{subarray}}%
}

%SS
\DeclareMathOperator{\supp}{supp}
\DeclareMathOperator{\Var}{Var}

%NT
\DeclareMathOperator{\ord}{ord}

%Alg
\DeclareMathOperator{\Rad}{Rad}
\DeclareMathOperator{\Jac}{Jac}

\DeclareMathAlphabet{\pazocal}{OMS}{zplm}{m}{n}
\newcommand{\unif}{\pazocal{U}}

\begin{document}
\section{Baxandall and Liebeck}

    \begin{exercise}[]
        Prove that $f  \colon \mathbb{R} \to \mathbb{R}^2$ defined
        by $f(t) = \left( \left| t \right| ,t \right) $ is not
        differentiable at $0$.
    \end{exercise}

    \begin{proof}
        If $f$ were differentiable at $0$, then there
        would exist a linear function $L  \colon \mathbb{R} \to 
        \mathbb{R}^2$ and a function
        $\eta  \colon \mathbb{R} \to \mathbb{R}^2$ such that
        $\lim_{h\to 0} \eta(h)=0$ and
        \[
        f\left( h \right) -f(0) = L(h)+ h \eta(h)
        \] 
        Now $f(h) - f(0) = \left( \left| h \right| ,h \right) $.
        Suppose $h = -\varepsilon <0$. Then
        $f(h) - f(0) = \left( \varepsilon, - \varepsilon \right) $ 
        and $L(h)+ h \eta \left( h \right) 
        = - \varepsilon L(1) - \varepsilon \eta \left( - \varepsilon
        \right) =
        -\varepsilon \left( L(1) + 
        \eta \left( -\varepsilon \right) \right) $.
        Thus 
        \[
           L(1) =\lim_{h\to 0-}
           \frac{f(h)-f(0)}{h} =
           \lim_{-\varepsilon \to 0, -\varepsilon < 0}
           \frac{\left( \varepsilon, - \varepsilon
           \right) }{- \varepsilon}= \left( -1,1 \right) 
       \]
        
        And for $h = \varepsilon > 0$, we have
        $f\left( h \right) -f(0) = 
        \left( \varepsilon, \varepsilon \right) $ and
        $L\left( h \right) + h \eta (h)=
        \varepsilon \left( L(1) + \eta (\varepsilon) \right) $, so
        \[
        L(1) = 
        \lim_{h\to 0+} \frac{f(h) - f(0)}{h} 
        \lim_{\varepsilon \to 0, \varepsilon > 0} 
        \frac{\left( \varepsilon, \varepsilon \right) }{\varepsilon}
        = \left( 1,1 \right) 
        \] 
        giving a contradiction for
        $L(1)$.
    \end{proof}

    \begin{exercise}[3.3.1]
        Let $f  \colon \mathbb{R}^2 \to \mathbb{R}$ be
        $f\left( x_1, x_2 \right) = x_1+ x_2$. Show that
        for any $p \in \mathbb{R}^2$,
        \[
        \lim_{h\to 0} \frac{f\left( p+h \right) -f(p)}{\|h\|}
        \] 
        does not exist.
    \end{exercise}

    \begin{solution}
        We have for $h\neq 0$,
        \[
        \frac{f\left( p+h \right) -f(p)}{\|h\|}
        = \frac{p_1 + h_1 + p_2 + h_2 - p_1 - p_2}{\|h\|}
        = \frac{h_1 + h_2}{\sqrt{h_1^2 + h_2^2} }
        \] 
        Supposing $h_1 = h_2$, we get
        $\frac{2 h_1}{\sqrt{2} h_1} = \frac{2}{\sqrt{2} }$ which is the
        limit along the linear $x=y$ in $\mathbb{R}^2$ approaching from
        above. However, if $h_1 = -h_2$, then
        the expression is $0$, so the limit is $0$ as well. Thus
        this limit does not exist.\\
        However, 
        \[
        f(p+h) - f(p) = h_1 + h_2
        = h_1 \frac{\partial f}{\partial x^{1}} (p) + 
        h_2 \frac{\partial f}{\partial x^{2}} (p)
        \] 
        so by theorem 3.3.22, $f$ is differentiable at
        $p$.\\
        This illustrates that the simple definition for functions
         $\mathbb{R} \to \mathbb{R}$ of
         \[
         f'(x)= \lim_{h\to 0} \frac{f(x+h) - f(x)}{h}
         \] 
         does not generalize to $\mathbb{R}^{m} \to \mathbb{R}$ for
         $m\ge 2$. 
    \end{solution}

    \begin{exercise}[]
        Prove that a linear function $L  \colon \mathbb{R}^{m} \to \mathbb{R}$ 
        is differentiable everywhere, and that it is equal to its own 
        differential at all points in $\mathbb{R}^{m}$.
    \end{exercise}

    \begin{proof}
        Clearly $L(p+h) - L(p) = L(p) + L(h) - L(p) = L(h)$, so
        $\delta_{L,p}$ is closely approximated near $0$ by
        $L$, so $L$ is differentiable everywhere, and
        $L_{L,p} = L$ by uniqueness, so $L$ is its own differential.\\
        Putting this in the context of differential geometry, we have
        that the differential of $L$ is
        \[
        dL = \sum \frac{\partial L}{\partial x^{i}} d x^{i}
        = \sum L\left( e_i \right) dx^{i}
        \] 
        Now, applying it to $e_i = \frac{\partial }{\partial x^{i}} $,
        we get
        $dL(e_i) = \sum L(e_j) dx^{j} \frac{\partial }{\partial x^{i}} 
        = \sum L(e_j) \frac{\partial x^{j}}{\partial x^{i}} 
        = \sum L(e_j) \delta_{i,j}
        = L(e_i)$. As both are linear maps
        $\mathbb{R}^{m} \to \mathbb{R}$ and agree on the standard
        basis, we conclude that $dL = L = L_{L,p}$, so indeed
        the differentials agree.
    \end{proof}

    \begin{exercise}[3.3.3]
        Prove that the function $f  \colon \mathbb{R}^2 \to \mathbb{R}$ 
        defined by $f\left( x_1, x_2 \right) =
        x_1^3 + x_2^3$ is differentiable everywhere. Find the linear
        function $L_{f,p}$ that closely approximates $\delta_{f,p}$ 
        near $0$.
    \end{exercise}

    \begin{solution}
        Recalling that if such an $L_{f,p}$ exists, then
        $L_{f,p}(e_i) = \frac{\partial f}{\partial x^{i}}(p) $, we
        find $L_{f,p}(e_1) = 3 p_1^2$ and
        $L_{f,p}(e_2) = 3p_2^2$, so
        $L_{f,p}(h) = h_1 3p_1^2 + h_2 3 p_2^2$. Now
        \begin{align*}
        \delta_{f,p}(h) = \left( h_1+p_1 \right)^3 + \left( h_2
        +p_2\right)^^3 - p_1^3 - p_2^3 
        &= 
        h_1^3 + h_2^3 + 3 \left( h_1^2 p_1 + h_2^2 p_2 +
        h_1p_1^2 + h_2p_2^2 \right)\\ 
        &=
        h_1^3 + h_2^3 + 3h_1^2 p_1 + 3h_2^2 p_2 + 
        L_{f,p}(h)\\
        &= L_{f,p}(h) + \|h \| \eta (h)
        \end{align*}
        where         
        \[
        \eta (h) =
        \begin{cases}
            \frac{h_1^3 + h_2^3 + 3h_1^2 p_1 + 3h_2^2 p_2}{
            \|h\|},& h\neq 0\\
            0,& h=0
        \end{cases}
    \]
        Now, since $\left| h_1 \right| ,
        \left| h_2 \right| \le \|h\|$, we find that
        $\eta (h) \to 0$ as $h\to 0$, so
        $f$ is differentiable.\\
        The Jacobian $J_{f,p}$ becomes
        \[
        J_{f,p} = \begin{pmatrix} 
            3p_1^2 & 3p_2^2
        \end{pmatrix} 
        \] 
        

        
    \end{solution}





%\bibliography{refs}
\end{document}
