\documentclass[a4paper]{article}

\usepackage[margin=2.5cm]{geometry}
\usepackage[pdftex]{graphicx}
\usepackage[utf8]{inputenc}
\usepackage[T1]{fontenc}
\usepackage{textcomp}
\usepackage{babel}
\usepackage{amsmath, amssymb}
\usepackage[colorlinks=true,linkcolor=blue]{hyperref}
\usepackage{float}
\usepackage{mathrsfs}
%\usepackage{enumitem}
%% for identity function 1:
%\usepackage{bbm}
%%For category theory diagrams:
%\usepackage{tikz-cd}
%%For code (e.g. python) in latex:
%\usepackage{listings}
%
%Usage: 
%\begin{lstlisting}[language=Python]
%\end{lstlisting}

\newcommand{\incfig}[2][1]{%
\def\svgwidth{#1\columnwidth}
\import{./figures/}{#2.pdf_tex}
}


% figure support
\usepackage{import}
\usepackage{xifthen}
\pdfminorversion=7
\usepackage{pdfpages}
\usepackage{transparent}

\pdfsuppresswarningpagegroup=1

\setlength\parindent{0pt}

\newcommand{\qed}{\tag*{$\blacksquare$}}
\newcommand{\qedwhite}{\hfill \ensuremath{\Box}}

%Inequalities
\newcommand{\cycsum}{\sum_{\mathrm{cyc}}}
\newcommand{\symsum}{\sum_{\mathrm{sym}}}
\newcommand{\cycprod}{\prod_{\mathrm{cyc}}}
\newcommand{\symprod}{\prod_{\mathrm{sym}}}

%Linear Algebra

%Redeclaring Span and image
\DeclareMathOperator{\Span}{span}
\DeclareMathOperator{\Ima}{Im}
\DeclareMathOperator{\diag}{diag}
\DeclareMathOperator{\Ker}{Ker}
\DeclareMathOperator{\ob}{ob}


%Row operations
\newcommand{\elem}[1]{% elementary operations
\xrightarrow{\substack{#1}}%
}

\newcommand{\lelem}[1]{% elementary operations (left alignment)
\xrightarrow{\begin{subarray}{l}#1\end{subarray}}%
}

%SS
\DeclareMathOperator{\supp}{supp}
\DeclareMathOperator{\Var}{Var}

%NT
\DeclareMathOperator{\ord}{ord}

%Alg
\DeclareMathOperator{\Rad}{Rad}
\DeclareMathOperator{\Jac}{Jac}

\DeclareMathAlphabet{\pazocal}{OMS}{zplm}{m}{n}
\newcommand{\unif}{\pazocal{U}}

\begin{document}
\textbf{1.1.1:} 
If $d_E$ and $d_T$ induce the same topology, then any function is continuous
with respect to one if and only if it is continuous with respect to the
other.\\
We show that the topologies are each finer than the other:\\
\linebreak
Let $x = (x_1, \ldots, x_n), y = (y_1, \ldots, y_n)$. 
Let $z_k = \left( x_1, \ldots, x_k, y_{k+1}, y_{k+2}, \ldots y_n \right) $. Then
by repeated application of the triangle inequality, we get
\begin{align*}
    \sqrt{\sum_{i=1}^{n} \left( x_i - y_i \right)^2} 
    = d_E (x,y) &\le d_E \left( x, \left( x_1, y_2, \ldots, y_n \right)  \right) 
+ d_E \left( \left( x_1, y_2, \ldots, y_n \right) , y \right) \\
&= d_E\left( x, z_1 \right) + d_E \left( z_1, y \right) \\
&\le d(x, z_2) + d(z_2, z_1) + d\left( z_1, y \right) \\
&\vdots\\
&\le d(x,z_{n-1}) + d\left( z_{n-1}, z_{n-2} \right) +
\ldots + d\left( z_2, z_1 \right) + d(z_1, y)\\
&= \sum_{i}^{n} \left| x_i - y_i \right| 
\end{align*}



By Cauchy-Schwarz, we have
\[
\sum_{i=1}^{n} \left| x_i-y_i \right| \le \sqrt{\sum_{i=1}^{n} \left( x_i - y_i
\right)^2} \sqrt{n},
\] 
so we have $B_{d_T}(x,\varepsilon) \subset B_{d_E}(x,\varepsilon)$ and
$B_{d_E}(x, \frac{\varepsilon}{\sqrt{n} }) \subset B_{d_T}(x, \varepsilon)$
. Since $x$ was arbitrary, we thus have
$\tau_{d_T} \subset \tau_{d_E} \subset \tau_{d_T}$ by lemma 13.3 in Munkres, so the topologies are
equal.\\
\linebreak




    \textbf{1.1.2:} Assume first it is continuous in the metric sense. 
    Let $f  \colon X \to Y$ be continuous. Let $V \subset Y$ be open. Then
    it is the union of basis elements: $V = \bigcup_{i \in  I} B_i(x_i,
    \varepsilon_i)$. Let $U = f^{-1} (V) = \bigcup_{i \in I} f^{-1} \left( 
    B_i (x_i, \varepsilon_i\right) )$. Let $x \in U$. Then $f(x) \in B_i (x_i,
    \varepsilon_i)$ for some $i \in I$. Since $B_i\left( x_i, \varepsilon_i
    \right) $ is open, we can choose a ball $B\left( f(x), \varepsilon_x \right)  \subset B_i \left( x_i,
    \varepsilon_i \right) $. By definition of our continuity, we now can find
    a $\delta_x$ such that $B\left( x, \delta_x \right) \subset 
    f^{-1} \left( B \left( f(x), \varepsilon_x \right)  \right) 
    \subset f^{-1}(V) = U$. Now since $x$ was arbitrary, we can find such
    a $\delta_x$ for any $x \in U$. Then
    \[
    U = \bigcup_{x \in U} \left\{ x \right\} \subset \bigcup_{x \in U} B\left(
    x, \delta_x \right) \subset U
    \] 
    so $\bigcup_{x \in U} B\left( x, \delta_x \right) = U$ and thus $U$ is
    a union of open sets and hence open.\\
    Hence $f$ is continuous the topological sense also.\\
    \linebreak
    Now assume $f$ is continuous in the topological sense and we show it is
    continuous in the metric sense.\\
    \linebreak
    Let $x \in X$ and $\varepsilon > 0$. Then  $B\left( f(x), \varepsilon
    \right) $ is an open set and thus $ U = f^{-1}\left( B\left( f(x),\varepsilon
    \right)  \right) $ is open. Thus we can write it as a union of open balls
    $U = \bigcup_{i \in  I} B\left( x_i, \delta_i \right) $. Then there exists
    some $i \in I$ such that $x \in B\left( x_i, \delta_i \right) $. Now since
    $B\left( x_i, \delta_i \right) $ is open, we can choose a basis element
    $B\left( x, \delta \right) \subset B\left( x_i, \delta_i \right) $. Now
    $f\left( B\left( x, \delta \right)  \right) 
    \subset f\left( B\left( x_i, \delta_i \right)  \right) 
    \subset f(U) \subset B\left( f(x), \varepsilon \right) $, which satisfies
    the requirement.\\
    \linebreak
    \textbf{1.2.1:} Firstly, since  $\varnothing, X \in \tau_i$ for all $i \in
    I$, we have
    $\varnothing, X \in \bigcap_{i \in  I} \tau_i$.\\
    \linebreak
    Now let $\left\{ U_{\alpha} \right\}_{\alpha \in J}$ be a collection of
    sets in $\bigcap_{i \in  I} \tau_i$. Then 
    for all $\alpha \in J$, we have $U_{\alpha} \in \tau_i$ for all $i$, and
    hence $\bigcup_{\alpha \in J} U_{\alpha} \in \tau_i$ for all $i$ since
    each $\tau_i$ is a topology and hence closed under arbitrary unions. But
    then
    $\bigcup_{\alpha \in J} U_{\alpha} \in \bigcap_{i \in  I} \tau_i$.
    So $\bigcap_{i \in I} \tau_i$ is closed under arbitrary unions of open
    sets.\\
    \linebreak
    Now let $U_1, \ldots, U_n$ be a finite collection of open sets in
    $\bigcap_{i \in  I} \tau_i$.
    Then for all $i \in I$, we have $U_k \in \tau_i$ for all $k = 1,\ldots ,n$,
    and therefore
    $\bigcap_{k=1}^{n}U_k \in \tau_i$ for all $i$ since each $\tau_i$ is
    a topology and thus closed under finite intersections. Then
    $\bigcap_{k=1}^{n} U_k \in \bigcap_{i \in  I} \tau_i$, so $\bigcap_{i \in
    I} \tau_i$ is closed under finite intersections.\\
    So $\bigcap_{i \in  I} \tau_i$ is a topology on $X$.
















    
\end{document}
