\documentclass[a4paper]{article}

\usepackage[margin=2.5cm]{geometry}
\usepackage[pdftex]{graphicx}
\usepackage[utf8]{inputenc}
\usepackage[T1]{fontenc}
\usepackage{textcomp}
\usepackage{babel}
\usepackage{amsmath, amssymb}
\usepackage[colorlinks=true,linkcolor=blue]{hyperref}
\usepackage{float}
\usepackage{mathrsfs}
%\usepackage{enumitem}
%% for identity function 1:
%\usepackage{bbm}
%%For category theory diagrams:
%\usepackage{tikz-cd}
%%For code (e.g. python) in latex:
%\usepackage{listings}
%
%Usage: 
%\begin{lstlisting}[language=Python]
%\end{lstlisting}

\newcommand{\incfig}[2][1]{%
\def\svgwidth{#1\columnwidth}
\import{./figures/}{#2.pdf_tex}
}


% figure support
\usepackage{import}
\usepackage{xifthen}
\pdfminorversion=7
\usepackage{pdfpages}
\usepackage{transparent}

\pdfsuppresswarningpagegroup=1

\setlength\parindent{0pt}

\newcommand{\qed}{\tag*{$\blacksquare$}}
\newcommand{\qedwhite}{\hfill \ensuremath{\Box}}

%Inequalities
\newcommand{\cycsum}{\sum_{\mathrm{cyc}}}
\newcommand{\symsum}{\sum_{\mathrm{sym}}}
\newcommand{\cycprod}{\prod_{\mathrm{cyc}}}
\newcommand{\symprod}{\prod_{\mathrm{sym}}}

%Linear Algebra

%Redeclaring Span and image
\DeclareMathOperator{\Span}{span}
\DeclareMathOperator{\Ima}{Im}
\DeclareMathOperator{\diag}{diag}
\DeclareMathOperator{\Ker}{Ker}
\DeclareMathOperator{\ob}{ob}


%Row operations
\newcommand{\elem}[1]{% elementary operations
\xrightarrow{\substack{#1}}%
}

\newcommand{\lelem}[1]{% elementary operations (left alignment)
\xrightarrow{\begin{subarray}{l}#1\end{subarray}}%
}

%SS
\DeclareMathOperator{\supp}{supp}
\DeclareMathOperator{\Var}{Var}

%NT
\DeclareMathOperator{\ord}{ord}

%Alg
\DeclareMathOperator{\Rad}{Rad}
\DeclareMathOperator{\Jac}{Jac}

\DeclareMathAlphabet{\pazocal}{OMS}{zplm}{m}{n}
\newcommand{\unif}{\pazocal{U}}

\begin{document}
    \textbf{3.0.1:}\\
    (i) Let $f  \colon \tilde{X} \to Y$ be a map, and 
    $q  \colon X \to \tilde{X}$ be a quotient map.\\
    If $f$ is continuous then
    $f \circ q$ is continuous as the composition of continuous
    maps.\\
    If conversely $f \circ q$ is continuous, let
    $V \subset Y$ be open. Then
     $\left( f \circ q \right)^{-1}(V) = 
     q^{-1} \circ f^{-1}(V)$ is open, and
     by definition $q^{-1}\left( f^{-1}(V) \right) $ is
     open if and only if $f^{-1}(V)$ is open
     in $\tilde{X}$, so $f^{-1}(V)$ is open, hence
     $f$ is continuous.\\
     \linebreak
     (ii) 
     Choose arbitrary $x \neq y$ of $X$ and let
     $f$ be the corresponding continuous function satisfying
     $f(x)\neq f(y)$. \\
     Since $f$ is continuous, 
     $U = f^{-1}\left( (-\infty, \frac{x+y}{2} ) \right) $ 
     and $ V = f^{-1}\left( \left( \frac{x+y}{2}, \infty \right)  \right) $ 
     are open, disjoint and 
     $U$ contains $\min \{x,y\}$ while
     $V$ contains $\max \{x,y\}$. Hence
     $X$ is Hausdorff as $x,y$ were arbitrary.\\
     \linebreak
     (iii) Define
     $\rho_{w} (P)
     = d(P,w) =  \min \left\{ d(w,p) \mid p \in P \right\}  $
     where $d$ denotes standard metric
     inherited from $\mathbb{R}^{n}$.\\
     This is continuous and gives the euclidean distance
     from $w$ to $P$ - it is in particular well defined as
     a minimum since there exists some
     closed ball $D^{n}$ containing
     $w$ and points from $P$ - the intersection of
     $D^{n}$ and $P$ is closed hence compact, and 
     thus by the extreme value theorem, the distance function
     attains a maximum on the compact subspace - which is clearly
     smaller than any distance outside the ball.\\
     By (i) we have that
     $\rho_w$ is continuous if and only if
     $\rho_w \circ q$ is continuous where $q  \colon
     V_{k}^{O} \left( \mathbb{R}^{n} \right) 
     \to Gr_{k}\left( \mathbb{R}^{n} \right) $ is the
     quotient map.\\
     Now, for any interval
     $\left( a,b \right) \subset R$, we have
     $\left( \rho_w \circ q \right)^{-1}
     \left( a,b \right) $ is precisely
     all $k$-tuples of northonormal vectors in
     $\mathbb{R}^{n}$ such that their span contains
     a point whose distance to $w$ is in $(a,b)$.\\
     \linebreak
     $\rho_w$ is naturally equal to the
     orthogonal projection $P_U$ sending
     a vector $v = u +w$ with $u \in U$ and $w \in U^{\perp}$ 
     with $U$ as subspace of $V$ to $u$, i.e. $P_U(v) = u$.\\
     By 6.55 in LADR, we have
     that if we pick an orthonormal basis $\left( v_1, \ldots, v_k \right)
     $ representing the $k$-plane, we have
     $\rho_w (P) = P_P (w) = \langle w, v_1 \rangle v_1 +
     \ldots + \langle w, v_k \rangle v_k $ which is continuous
     as it is linear (6.55.(i) and (a)).\\
     \linebreak
     
     






     (iv):
     Let $P_1, P_2 \in Gr_{k}\left( \mathbb{R}^{n} \right) $
     be distinct. Choose any point $w \in P_1 - P_2$. Then
     $\rho_w (P_1) = 0 < \rho_w (P_2)$ where we can conclude strict
     inequality since we have $\min$ in our definition for
     $\rho_w$.\\
     \linebreak
     (i) 
     For a $1$-plane in $Gr_1 \left( \mathbb{R}^2 \right) $,
     we have $P \in U_X$ if it has trivial intersection
     with $X^{\perp}$ which is the orthogonal line of $X$ through
     $0$. Any line that is not $X^{\perp}$ will trivial
     intersection with $X^{\perp}$, so any
     1-dimensional subspace of $\mathbb{R}^2$ that
     is not $X^{\perp}$ will be in $U_X$.\\
     \linebreak
     (ii) 
     Any
     set in
     $\left\{ P \in Gr_{k}\left( \mathbb{R}^{n} \right) 
      \colon P \cap X^{\perp} = \left\{ 0 \right\} \right\} $ 
      is of the form
      $\Gamma (A) = \{ Ax + x$ with $x \in X\}$
      where $A  \colon X \to X^{\perp}$ is
      a linear map. Defining
      $\varphi (A) = \Gamma (A)$, we find
      this is a bijection, so since
      the dimension of  $\mathcal{L}(X, X^{\perp})$ is
      $(n-k) \cdot k$ by identifying it with
      its matrix representation getting
      $M \left( (n-k) \times k, \mathbb{R} \right) $, we find that
      $U_X \cong \mathbb{R}^{(n-k) k}$.























\end{document}
