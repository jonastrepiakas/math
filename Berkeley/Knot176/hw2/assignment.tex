\documentclass[a4paper]{article}

\usepackage[margin=2.5cm]{geometry}
\usepackage[pdftex]{graphicx}
\usepackage[utf8]{inputenc}
\usepackage[T1]{fontenc}
\usepackage{textcomp}
\usepackage{babel}
\usepackage{amsmath, amssymb}
\usepackage[colorlinks=true,linkcolor=blue]{hyperref}
\usepackage{float}
\usepackage{mathrsfs}
%\usepackage{enumitem}
%% for identity function 1:
%\usepackage{bbm}
%%For category theory diagrams:
%\usepackage{tikz-cd}
%%For code (e.g. python) in latex:
%\usepackage{listings}
%
%Usage: 
%\begin{lstlisting}[language=Python]
%\end{lstlisting}

\newcommand{\incfig}[2][1]{%
\def\svgwidth{#1\columnwidth}
\import{./figures/}{#2.pdf_tex}
}


% figure support
\usepackage{import}
\usepackage{xifthen}
\pdfminorversion=7
\usepackage{pdfpages}
\usepackage{transparent}

\pdfsuppresswarningpagegroup=1

\setlength\parindent{0pt}

\newcommand{\qed}{\tag*{$\blacksquare$}}
\newcommand{\qedwhite}{\hfill \ensuremath{\Box}}

%Inequalities
\newcommand{\cycsum}{\sum_{\mathrm{cyc}}}
\newcommand{\symsum}{\sum_{\mathrm{sym}}}
\newcommand{\cycprod}{\prod_{\mathrm{cyc}}}
\newcommand{\symprod}{\prod_{\mathrm{sym}}}

%Linear Algebra

%Redeclaring Span and image
\DeclareMathOperator{\Span}{span}
\DeclareMathOperator{\Ima}{Im}
\DeclareMathOperator{\diag}{diag}
\DeclareMathOperator{\Ker}{Ker}
\DeclareMathOperator{\ob}{ob}


%Row operations
\newcommand{\elem}[1]{% elementary operations
\xrightarrow{\substack{#1}}%
}

\newcommand{\lelem}[1]{% elementary operations (left alignment)
\xrightarrow{\begin{subarray}{l}#1\end{subarray}}%
}

%SS
\DeclareMathOperator{\supp}{supp}
\DeclareMathOperator{\Var}{Var}

%NT
\DeclareMathOperator{\ord}{ord}

%Alg
\DeclareMathOperator{\Rad}{Rad}
\DeclareMathOperator{\Jac}{Jac}

\DeclareMathAlphabet{\pazocal}{OMS}{zplm}{m}{n}
\newcommand{\unif}{\pazocal{U}}

\begin{document}
   \textbf{2.1.1:} Let $\left\{ U_i \right\}_{i \in I}$ be an open covering of $X \times Y$ equipped
   with the product topology. Fix an $x \in X$.\\
   Then for each $y \in Y$, there exists some $U_{y}$ containing
    $(x,y)$, and since this is open, and 
    $$\left\{ V_i \times W_i  \mid V_i \text{ open in } X \text{ and }
    W_i \text{ open in } Y \right\} $$
    forms a basis for the product topology,
    we can find $V_{y} \times W_y$ with $V_y$ open in $X$ and $W_y$ open in
    $Y$ such that $(x,y) \in V_y \times W_y \subset U_y$. Then
    since $\bigcup_{y \in Y} W_y = Y$, and $Y$ is compact, we can find a finite
    subcollection $y_1, \ldots, y_n$ such that
    $Y = W_{y_1} \cup \ldots \cup W_{y_n}$. Then
    $V_x = V_{y_1} \cap \ldots V_{y_n}$ is nonempty and open as $x \in  V_x$ 
    and
    it is the intersection of a finite collection of open sets. 
    We then get
    \[
    V_x \times Y = V_x \times W_{y_1} \cup \ldots \cup V_x \times W_{y_n}
    \subset U_{y_1} \cup \ldots \cup U_{y_n}.
    \] 
    What this means is that for any $x \in X$, there exists
    a finite subcollection $\mathcal{A}_x \subset 
    \left\{ U_i \right\}_{i \in I}$, such that
    $V_x \times Y \subset \bigcup_{A \in \mathcal{A}_x} A$.\\
    Now since $\bigcup_{x \in X} V_x = X$ and $X$ is compact, there exist some 
    finite subcollection $V_{x_1}, \ldots, V_{x_n}$ such that
    $V_{x_1} \cup \ldots \cup V_{x_n} = X$.
    Hence
    \[
    X \times Y = \bigcup_{i=1}^{n} V_x \times Y
    \subset \bigcup_{i}^{n} \bigcup_{A \in \mathcal{A}_{x_i}} A \subset
    X \times Y,
    \] 
    so
    \[
    X \times Y = \bigcup_{i}^{n} \bigcup_{A \in \mathcal{A}_{x_i}} A
    \] 
    which is a finite subcover of $\left\{ U_i \right\}_{i \in I}$.
    Thus $X \times Y$ is compact.\\
    \linebreak
    \textbf{2.1.2:} Let
    $\left( x_n \right) \subset I$ be any sequence in $I$.
    If $(x_n)$ converges in $I$, we are done, so we may assume $(x_n)$ does not
    converge. In particular, $(x_n)$ thus consists of infinitely many distinct
    points. Therefore, either $\left[ 0, \frac{1}{2} \right] $ or
    $\left[ \frac{1}{2},1 \right] $ must contain infinitely many distinct
    points
    of the sequence. Let $I_1$ denote a half of $I_0$ satisfying this.\\
    Now define for a general  $n \in \mathbb{Z}_+$, $I_n$ to be
    a half of $I_{n-1}$ which contains infinitely many distinct points
    - such an $I_n$ must exist, as otherwise
    $I_{n-1}$ contains only finitely many points in contradiciton with
    its construction.\\
    We have a countable collection
    $\left\{ I_n \right\}_{n \in \mathbb{N}_{0}}$ with
    $I_0 \supset I_1 \supset I_2 \supset \ldots$.
    By the axiom of countable choice, we can choose a
    sequence $\left( x_{n_{k}} \right)_{k \in \mathbb{N}_{0}}$ such that
    $x_{n_k} \in I_{k}$. We can assume this to be
    a subsequence of $(x_n)$, i.e. $n_i < n_k$ if $i < k$, since
    otherwise if $i < k$ with $n_k \le n_i$, we can replace $x_{n_k}$ by $x_{n_i}$ since
    $I_{n_i} \subset I_{n_k}$.\\
    \linebreak
    Now let $\varepsilon > 0$ be arbitrary fixed. By the Archimedean property,
    there exists $N \in Z_{+}$ such that $\frac{1}{\varepsilon} < N$ and hence
    $0 < \frac{1}{2^{N}} < \frac{1}{N} < \varepsilon$.
    For any $r,s \ge N$, assuming without loss of generality that $r \le s$,
    we have $x_{n_{r}}, x_{n_{s}} \in I_{r}$ and since the diameter
    of $I_r$ is $\frac{1}{2^{r}}$, we have
    \[
    |x_{n_r} - x_{n_s}| \le \frac{1}{2^{r}} \le \frac{1}{2^{N}} < \varepsilon.
    \] 
    So $\left( x_{n_k} \right)_{k \in \mathbb{N}_{0}}$ is a cauchy sequence,
    and
    since $\mathbb{R}$ is complete, it follows that 
    $\left( x_{n_k} \right) $ converges to a point in $I$ (since $I$ is closed
    and the sequence is in $I$ ).\\
    \linebreak
    
    
    \textbf{2.1.3:} An equatorial cut would produce a mobius strip which is
    essentially twice the length and has twice the twists of the original
    strip. An equatorial cut along each of the resulting pieces would create
    two strips of the same length and same number of twists - i.e. two twists.
    
    
    
    
   

   
   
   













































\end{document}
