\documentclass[a4paper]{article}

\usepackage[margin=2.5cm]{geometry}
\usepackage[pdftex]{graphicx}
\usepackage[utf8]{inputenc}
\usepackage[T1]{fontenc}
\usepackage{textcomp}
\usepackage{babel}
\usepackage{amsmath, amssymb}
\usepackage[colorlinks=true,linkcolor=blue]{hyperref}
\usepackage{float}
\usepackage{mathrsfs}
\usepackage{enumitem}
%% for identity function 1:
\usepackage{bbm}
%%For category theory diagrams:
\usepackage{tikz-cd}
%%For code (e.g. python) in latex:
%\usepackage{listings}
%
%Usage: 
%\begin{lstlisting}[language=Python]
%\end{lstlisting}

\newcommand{\incfig}[2][1]{%
\def\svgwidth{#1\columnwidth}
\import{./figures/}{#2.pdf_tex}
}


% figure support
\usepackage{import}
\usepackage{xifthen}
\pdfminorversion=7
\usepackage{pdfpages}
\usepackage{transparent}

\pdfsuppresswarningpagegroup=1

\setlength\parindent{0pt}

\newcommand{\qed}{\tag*{$\blacksquare$}}
\newcommand{\qedwhite}{\hfill \ensuremath{\Box}}

%Inequalities
\newcommand{\cycsum}{\sum_{\mathrm{cyc}}}
\newcommand{\symsum}{\sum_{\mathrm{sym}}}
\newcommand{\cycprod}{\prod_{\mathrm{cyc}}}
\newcommand{\symprod}{\prod_{\mathrm{sym}}}

%Linear Algebra

\DeclareMathOperator{\Span}{span}
\DeclareMathOperator{\Ima}{Im}
\DeclareMathOperator{\diag}{diag}
\DeclareMathOperator{\Ker}{Ker}
\DeclareMathOperator{\ob}{ob}
\DeclareMathOperator{\Hom}{Hom}
\DeclareMathOperator{\sk}{sk}
\DeclareMathOperator{\Vect}{Vect}
\DeclareMathOperator{\Set}{Set}
\DeclareMathOperator{\Group}{Group}
\DeclareMathOperator{\Ring}{Ring}
\DeclareMathOperator{\Ab}{Ab}
\DeclareMathOperator{\Top}{Top}
\DeclareMathOperator{\Htpy}{Htpy}
\DeclareMathOperator{\Cat}{Cat}
\DeclareMathOperator{\CAT}{CAT}


%Row operations
\newcommand{\elem}[1]{% elementary operations
\xrightarrow{\substack{#1}}%
}

\newcommand{\lelem}[1]{% elementary operations (left alignment)
\xrightarrow{\begin{subarray}{l}#1\end{subarray}}%
}

%SS
\DeclareMathOperator{\supp}{supp}
\DeclareMathOperator{\Var}{Var}

%NT
\DeclareMathOperator{\ord}{ord}

%Alg
\DeclareMathOperator{\Rad}{Rad}
\DeclareMathOperator{\Jac}{Jac}

\DeclareMathAlphabet{\pazocal}{OMS}{zplm}{m}{n}
\newcommand{\unif}{\pazocal{U}}

\begin{document}
    \textbf{2.1.ii} Prove that if $F  \colon C \to \Set$ is representable, then
    $F$ preserves monomorphisms, i.e., sends every monomorphism in $C$ to an
    injective function. Use the contrapositive to find a covariant set-valued
    functor defined on your favorite concrete category that is not
    representable.\\
    \linebreak
    \textit{Solution:} Suppose $F$ is representable by $c \in C$. Thus there
    is a natural isomorphism between $F$ and
    $C(c,-)$. Thus
    
    \[\begin{tikzcd}
	{C(c,x)} && {F(x)} \\
	\\
	{C(c,y)} && {F(y)}
	\arrow["{F(f)}", from=1-3, to=3-3]
	\arrow["{f_*}"', from=1-1, to=3-1]
	\arrow["\cong", from=1-1, to=1-3]
	\arrow["\cong"', from=3-1, to=3-3]
\end{tikzcd}\]
commutes for any $x,y \in C$ with $f  \colon x\to y$. Now, if
$f  \colon x\to y$ is a monomorphism, then for any
$z,w \in C$ with $g  \colon z\to x$ and $h  \colon w \to x$,
\[
fg = fh \implies g=h.
\] 
This is equivalent to saying that for any
$z \in C$, post-composition with $f$ defines an injection
$f_*  \colon C(z,x) \to C(z,y)$. Hence, in the square above, if
we denote the top isomorphism by $\alpha_x$ and the bottom one by
$\alpha_y$, then we find that
$\alpha_y \circ f_* \circ \alpha_x^{-1} = F(f)$, and since isomorphisms between
sets are bijective functions, and as $f_*$ is injective, we have that
$F(f) = \alpha_y \circ f_* \circ \alpha_x^{-1}$ is injective.\\
Taking the contrapositive, we have that if a covariant functor
$F  \colon C \to \Set$ does not preserve monomorphisms, then
$F$ is not representable.\\
\linebreak
Consider the category $C = \mathbbm{2} = \left\{ 0,1 \right\} $ with the single
non-identity map $0 \to 1$. Now define a functor
$F  \colon C \to \Set$ sending $0 \to \left\{ 1,2 \right\} $ and
$1 \to \left\{ 3 \right\} $. Then $F(0\to 1)$ is not mono since
the maps $\alpha, \beta  \colon \left\{ -1,-2 \right\} \to \left\{ 1,2 \right\}
$ by
$\alpha (-1) = 1, \alpha(-2) = 2$ and $\beta (-1) = 2, \beta(-2) = 1$ each give
that for  $\gamma  \colon \left\{ 1,2 \right\} \to \left\{ 3 \right\} $,
$\gamma \alpha = \gamma \beta$, yet $\alpha \neq  \beta$, so
$\gamma$ is not a monomorphism.\\
\linebreak
\textbf{2.1.iii:} Suppose $F  \colon C \to \Set$ is equivalent to $G
 \colon D \to \Set$ in the sense that there is an equivalence of categories
 $H  \colon C \to D$ so that $GH$ and $F$ are naturally isomorphic.\\
 \begin{enumerate}[label=(\roman*)]
     \item If $G$ is representable, then is $F$ representable?
     \item If $F$ is representable, then is $G$ representable?
 \end{enumerate}

 \textit{Solution:}\\
 We claim both (i) and (ii) are true.\\
 \linebreak
 (i) We have that $GH$ and $F$ are naturally isomorphic, so
 
 \[\begin{tikzcd}
	{GH(c)} && {F(c)} \\
	\\
	{GH(c')} && {F(c')}
	\arrow["{GH(f)}"', from=1-1, to=3-1]
	\arrow["{F(f)}", from=1-3, to=3-3]
	\arrow["\cong", from=1-1, to=1-3]
	\arrow["\cong"', from=3-1, to=3-3]
\end{tikzcd}\]
commutes.\\
Suppose $G$ is representable, so
there exists some $d \in D$ such that
\[\begin{tikzcd}
	{C(d,x)} && {G(x)} \\
	\\
	{C(d,y)} && {G(y)}
	\arrow["{f_*}"', from=1-1, to=3-1]
	\arrow["{G(f)}", from=1-3, to=3-3]
	\arrow["\cong", from=1-1, to=1-3]
	\arrow["\cong"', from=3-1, to=3-3]
\end{tikzcd}\]
commutes.\\
Since $H$ is one part of an equivalence of categories, it is full, faithful
and essentially surjective on objects by theorem 1.5.9, so there
exists some $\tilde{c} \in C$ such that
$H(\tilde{c}) \cong d$. Furthermore, by theorem 1.5.9, it is full and
faithful, so
for any $a,b \in C$, 
$|\Hom(a,b)| = \left| \Hom(H(a),H(b)) \right| $, s
$C(a,b) \cong C(H(a), H(b))$.\\
Composing the two commutative squares, and using this last bijection, we find that
\[\begin{tikzcd}
	{C(\tilde{c},c)} && {C(H(\tilde{c}),H(c))} && {GH(c)} && {F(c)} \\
	\\
	{C(\tilde{c},c')} && {C(H(\tilde{c}),H(c'))} && {GH(c')} && {F(c')}
	\arrow["{F(f)}", from=1-7, to=3-7]
	\arrow["\cong", from=1-5, to=1-7]
	\arrow["{GH(f)}"', from=1-5, to=3-5]
	\arrow["\cong"', from=3-5, to=3-7]
	\arrow["\cong", from=1-3, to=1-5]
	\arrow["{(H(f))_*}"', from=1-3, to=3-3]
	\arrow["\cong"', from=3-3, to=3-5]
	\arrow["\cong", from=1-1, to=1-3]
	\arrow["\cong"', from=3-1, to=3-3]
	\arrow["{f_*}"', from=1-1, to=3-1]
\end{tikzcd}\]
commutes, and since each square commutes, we have that the outer rectangle
commutes, giving that $F$ is represented by $\tilde{c}$.\\
\linebreak
(ii) Suppose $GH$ and $F$ are naturally isomorphic as before, and suppose
$F$ is representable - suppose it is represented by $c \in C$, so
 \[\begin{tikzcd}
	{GH(c)} && {F(c)} \\
	\\
	{GH(c')} && {F(c')}
	\arrow["{GH(f)}"', from=1-1, to=3-1]
	\arrow["{F(f)}", from=1-3, to=3-3]
	\arrow["\cong", from=1-1, to=1-3]
	\arrow["\cong"', from=3-1, to=3-3]
\end{tikzcd}\]
and


\[\begin{tikzcd}
	{C(c,x)} && {F(x)} \\
	\\
	{C(c,y)} && {F(y)}
	\arrow["{f_*}"', from=1-1, to=3-1]
	\arrow["{F(f)}", from=1-3, to=3-3]
	\arrow["\cong", from=1-1, to=1-3]
	\arrow["\cong"', from=3-1, to=3-3]
\end{tikzcd}\]
commute.\\
Letting $\alpha_x$ and $\alpha_y$ denote the top and bottom isomorphism,
respectively, we have
 \[
f_* \alpha_x^{-1} =
\alpha_y^{-1} \alpha_y f_* \alpha_x^{-1}
= \alpha_y^{-1} F(f) \alpha_x \alpha_x^{-1}
= \alpha_y^{-1}F(f)
\] 
so we can rewrite it as the following diagram commuting:
\[\begin{tikzcd}
	{F(x)} && {C(c,x)} \\
	\\
	{F(y)} && {C(c,y)}
	\arrow["{f_*}", from=1-3, to=3-3]
	\arrow["{F(f)}"', from=1-1, to=3-1]
	\arrow["\cong", from=1-1, to=1-3]
	\arrow["\cong"', from=3-1, to=3-3]
\end{tikzcd}\]

Now let $d',\tilde{d} \in D$ be arbitrary with $g  \colon d' \to \tilde{d}$ a morphism between
them. As $H$ is essentially surjective on objects and full and faithful by
theorem 1.5.9., there exist $c', \tilde{c} \in C$ such that
$H(c') \cong d'$ and $H(\tilde{c}) \cong \tilde{d}$. By fullness and
faithfullness, $C(c', \tilde{c}) \cong C\left( d', \tilde{d} \right) $, so
there exists $g'  \colon c' \to \tilde{c}$ with
$H(g') = g$.  We thus find

\[\begin{tikzcd}
	{G(d')} && {F(c')} && {C(c,c')} && {C(H(c),d')} \\
	\\
	{G(\tilde{d})} && {F(\tilde{c})} && {C(c,\tilde{c})} && {C(H(c),\tilde{d})}
	\arrow["{G(g)}"', from=1-1, to=3-1]
	\arrow["{F(g')}", from=1-3, to=3-3]
	\arrow["\cong", from=1-1, to=1-3]
	\arrow["\cong"', from=3-1, to=3-3]
	\arrow["{g'_*}", from=1-5, to=3-5]
	\arrow["\cong", from=1-3, to=1-5]
	\arrow["\cong"', from=3-3, to=3-5]
	\arrow["{g_*}", from=1-7, to=3-7]
	\arrow["\cong", from=1-5, to=1-7]
	\arrow["\cong"', from=3-5, to=3-7]
\end{tikzcd}\]
commutes with each square commuting. Here the last square
again follows as $H$ is part of an equivalence and hence both
faithful and full, so
$C(c,x) \cong C(H(c), H(x))$ for any $x \in C$.\\

Thus the outer square commutes, so
$G$ is represented by $G(c)$.









\end{document}
