\documentclass[a4paper]{article}

\usepackage[margin=2.5cm]{geometry}
\usepackage[pdftex]{graphicx}
\usepackage[utf8]{inputenc}
\usepackage[T1]{fontenc}
\usepackage{textcomp}
\usepackage{babel}
\usepackage{amsmath, amssymb}
\usepackage[colorlinks=true,linkcolor=blue]{hyperref}
\usepackage{float}
\usepackage{mathrsfs}
%\usepackage{enumitem}
%% for identity function 1:
\usepackage{bbm}
%%For category theory diagrams:
%\usepackage{tikz-cd}
%%For code (e.g. python) in latex:
%\usepackage{listings}
%
%Usage: 
%\begin{lstlisting}[language=Python]
%\end{lstlisting}

\newcommand{\incfig}[2][1]{%
\def\svgwidth{#1\columnwidth}
\import{./figures/}{#2.pdf_tex}
}


% figure support
\usepackage{import}
\usepackage{xifthen}
\pdfminorversion=7
\usepackage{pdfpages}
\usepackage{transparent}

\pdfsuppresswarningpagegroup=1

\setlength\parindent{0pt}

\newcommand{\qed}{\tag*{$\blacksquare$}}
\newcommand{\qedwhite}{\hfill \ensuremath{\Box}}

%Inequalities
\newcommand{\cycsum}{\sum_{\mathrm{cyc}}}
\newcommand{\symsum}{\sum_{\mathrm{sym}}}
\newcommand{\cycprod}{\prod_{\mathrm{cyc}}}
\newcommand{\symprod}{\prod_{\mathrm{sym}}}

%Linear Algebra

%Redeclaring Span and image
\DeclareMathOperator{\Span}{span}
\DeclareMathOperator{\Ima}{Im}
\DeclareMathOperator{\diag}{diag}
\DeclareMathOperator{\Ker}{Ker}
\DeclareMathOperator{\ob}{ob}
\DeclareMathOperator{\sk}{sk}
\DeclareMathOperator{\Hom}{Hom}


%Row operations
\newcommand{\elem}[1]{% elementary operations
\xrightarrow{\substack{#1}}%
}

\newcommand{\lelem}[1]{% elementary operations (left alignment)
\xrightarrow{\begin{subarray}{l}#1\end{subarray}}%
}

%SS
\DeclareMathOperator{\supp}{supp}
\DeclareMathOperator{\Var}{Var}

%NT
\DeclareMathOperator{\ord}{ord}

%Alg
\DeclareMathOperator{\Rad}{Rad}
\DeclareMathOperator{\Jac}{Jac}

\DeclareMathAlphabet{\pazocal}{OMS}{zplm}{m}{n}
\newcommand{\unif}{\pazocal{U}}

\begin{document}
    \textbf{1.5.ix:} Show that any category that is equivalent to a locally
    small category is locally small.\\
    \linebreak
    \textit{Solution:} Let $C$ be a locally small category that is equivalent
    to a category $D$. Let
    $d$ and $d'$ be objects of $D$. Since $C$ and $D$ are equivalent, we have
    by theorem 1.5.9 that
    the functor defining an equivalence between $C$ and $D$ is fully faithful
    and
    essentially surjective. Let $F  \colon C \to D$ be a functor defining
    an equivalence between $C$ and $D$.\\
    Since $F$ is essentially surjective, we can find objects
    $c, c' \in C$ such that $F(c) \cong d$ and $F(c') \cong d'$. Since
    $F$ is fully faithful, we have
    that the map $C\left( c,c' \right) \to D\left( d,d' \right) $ is
    bijective, and since $C\left( c,c' \right) $ is a set, $D(d,d')$ is
    a set.\\
    Since $d$ and $d'$ were arbitrary, we find that $D$ is locally small.\\
    \linebreak
    \textbf{1.5.x:} Characterize the categories that are equivalent to discrete
    categories. A category that is connected and essentially discrete is called
    chaotic.\\
    \linebreak
    \textit{Solution:} Let $C$ be a category that is equivalent to a discrete category $D$.\\
    The skeleton of $C$ and $D$ are isomorphic, so since the skeleton of $D$ is
    $D$ itself - since each isomorphism class only contains a single object -,
    we have $\sk C \cong \sk D \cong D$, so the skeleton of $C$ is discrete.
    If for two objects $c, c' \in C$ there was a non-isomorphic morphism
    $c \to c'$, there would be a morphism in the skeleton from the isomorphism
    class of $c$ to the isomorphism class of $c'$, so since the skeleton is
    discrete, we conclude that all morphisms in $C$ are isomorphisms.\\
    Furthermore, since the functor taking $C$ to $D$ in the equivalence is
    faithful, we have that since $\left| \Hom (d,d) \right| = 1$ for all $d \in
    D$, $\left| \Hom (c,c') \right| = 1$ for any two isomorphic $c,c'$ in
    $C$.\\
    \linebreak
    \textit{Claim:} The categories that are equivalent to discrete categories
    are precisely all groupoids with hom-sets of order less than or equal to $1$.\\
    \linebreak
    \textit{Proof:} We have shown  the inclusion $\subset $. For the converse,
    suppose $C$ is a groupoid with hom-sets of order $\le 1$.
    Let $d, d' \in \sk C$ be distinct. Then since
    the skeleton of $C$ is equivalent to $C$, let $F  \colon \sk C \to C$ be
    a functor defining the equivalence. If there were a morphism
    $d \to d' \in \Hom (d,d')$, then
    there exists a morphism $Fd \to Fd' \in \Hom \left( Fd, Fd' \right) $, but
    since $C$ is a groupoid, this is then an isomorphism. Now let  $G$ be the
    other part of the functor defining the equivalence  with $F$. Then
    $G$ maps isomorphisms to isomorphisms, so
    $G \left( Fd \right) \to G\left( Fd' \right) $ is an isomorphism, but
    by equivalence $\alpha  \colon GF \simeq \mathbbm{1}_{\sk C}$, so
    letting  $f  \colon d \to d'$, we have
    $\alpha_{d'} GF(f) \alpha_{d}^{-1} = f$, so since each of the morphisms on
    the left are isomorphisms, $f$ is an isomorphism. But this contradicts
    that $d$ and $d'$ are distinct.\\
    \linebreak
    Chaotic categories is thus a connected groupoid with hom-sets of order less
    than or equal to $1$. This groupoid is thus all mapped to the discrete
    category of a single object
    with the identity as the only morphism. Since the functor
    defining the equivalence is full, there must furthermore exist a morphism
    between each pair of objects in the connected groupoid. So it is simply
    the groupoid where each object is isomorphic to every other object. 
    






\end{document}
