\documentclass[a4paper]{article}

\usepackage[margin=2.5cm]{geometry}
\usepackage[pdftex]{graphicx}
\usepackage[utf8]{inputenc}
\usepackage[T1]{fontenc}
\usepackage{textcomp}
\usepackage{babel}
\usepackage{amsmath, amssymb}
\usepackage[colorlinks=true,linkcolor=blue]{hyperref}
\usepackage{float}
\usepackage{mathrsfs}
%\usepackage{enumitem}
%% for identity function 1:
\usepackage{bbm}
%%For category theory diagrams:
\usepackage{tikz-cd}
%%For code (e.g. python) in latex:
%\usepackage{listings}
%
%Usage: 
%\begin{lstlisting}[language=Python]
%\end{lstlisting}

\newcommand{\incfig}[2][1]{%
\def\svgwidth{#1\columnwidth}
\import{./figures/}{#2.pdf_tex}
}


% figure support
\usepackage{import}
\usepackage{xifthen}
\pdfminorversion=7
\usepackage{pdfpages}
\usepackage{transparent}

\pdfsuppresswarningpagegroup=1

\setlength\parindent{0pt}

\newcommand{\qed}{\tag*{$\blacksquare$}}
\newcommand{\qedwhite}{\hfill \ensuremath{\Box}}

%Inequalities
\newcommand{\cycsum}{\sum_{\mathrm{cyc}}}
\newcommand{\symsum}{\sum_{\mathrm{sym}}}
\newcommand{\cycprod}{\prod_{\mathrm{cyc}}}
\newcommand{\symprod}{\prod_{\mathrm{sym}}}

%Linear Algebra

\DeclareMathOperator{\Span}{span}
\DeclareMathOperator{\Ima}{Im}
\DeclareMathOperator{\diag}{diag}
\DeclareMathOperator{\Ker}{Ker}
\DeclareMathOperator{\ob}{ob}
\DeclareMathOperator{\Hom}{Hom}
\DeclareMathOperator{\sk}{sk}
\DeclareMathOperator{\Vect}{Vect}
\DeclareMathOperator{\Set}{Set}
\DeclareMathOperator{\Group}{Group}
\DeclareMathOperator{\Ring}{Ring}
\DeclareMathOperator{\Ab}{Ab}
\DeclareMathOperator{\Top}{Top}
\DeclareMathOperator{\Htpy}{Htpy}
\DeclareMathOperator{\Cat}{Cat}
\DeclareMathOperator{\CAT}{CAT}
\DeclareMathOperator{\Cone}{Cone}


%Row operations
\newcommand{\elem}[1]{% elementary operations
\xrightarrow{\substack{#1}}%
}

\newcommand{\lelem}[1]{% elementary operations (left alignment)
\xrightarrow{\begin{subarray}{l}#1\end{subarray}}%
}

%SS
\DeclareMathOperator{\supp}{supp}
\DeclareMathOperator{\Var}{Var}

%NT
\DeclareMathOperator{\ord}{ord}

%Alg
\DeclareMathOperator{\Rad}{Rad}
\DeclareMathOperator{\Jac}{Jac}

\DeclareMathAlphabet{\pazocal}{OMS}{zplm}{m}{n}
\newcommand{\unif}{\pazocal{U}}

\begin{document}
\textbf{Exercise 3.1.ix:} Show that if $J$ has an initial object, then the
limit of any functor indexed by $J$ is the value of that functor at an initial
object. Apply the dual of this result to describe the colimit of a diagram
indexed by a successor ordinal.\\
\linebreak
\textit{Solution:} Suppose $J$ has initial object $1$ and
let $F  \colon J \to C$ be a functor indexed by $J$.
We want to show that $F(1) = \lim F$, i.e., that there exists
a universal cone $\lambda  \colon F(1) \implies F$.\\
Firstly, we define the cone $\lambda$ by letting
$\lambda_j = F(f_j)$ where
$f_j$ is the unique morphism $1 \to j$ in $J$ (uniqueness by $1$ being initial,
and hence $\left| \Hom(1,j) \right| = 1$ for all $j \in J$.\\
Then we want to show that
for any $j,j' \in J$ with a morphism $f  \colon j \to j'$, the following commutes

\begin{equation}
\begin{tikzcd}
     & F(1) \ar[dl, "F(f_j)"] \ar[dr, "F(f_{j'})"] &\\
    Fj \ar[rr, "Ff"] & & Fj'
\end{tikzcd}
\end{equation}
To show commutativity, we note that since
$f \circ f_j$ is a map $1 \mapsto j \mapsto j'$, we have
$f \circ f_j \in \Hom\left( 1,j' \right) = \left\{ f_{j'} \right\} $, so
$f \circ f_j = f_{j'}$, and since functors distribute over composition, we have
\[
    F(f) \circ F\left( f_j \right) 
    = F(f \circ f_j) = F(f_{j'})
\] 
giving commutativity of (1).\\
\linebreak
To show uniqueness, suppose there exists
an element $A \in J$ and a cone  of $F$ over $A$ with
components $A_j  \colon A \to F(j)$ such that
for any $j,j' \in J$ with a morphism $f  \colon j \to j'$, the following
diagram commutes
\begin{equation}
\begin{tikzcd}
    & A \ar[dl, "A_j"] \ar[dr, "A_{j'}"] & \\
    Fj \ar[rr, "Ff"] & & Fj'
\end{tikzcd}
\end{equation}
Then in particular,  there exists a map
$A_1  \colon A \to F(1)$. We claim that $A_1$ is the unique morphism such that
\begin{equation}
\begin{tikzcd}
    & A \ar[ddl, "A_j"] \ar[ddr, "A_{j'}"] \ar[d, "A_1"] &\\
    & F(1) \ar[dl, "Ff_j"] \ar[dr, "Ff_{j'}"] &\\
    Fj \ar[rr, "Ff"] & & Fj'
\end{tikzcd}
\end{equation}
commutes.\\
Firstly, $F f_{j} \circ A_1 = A_j$ and
$F f_{j'} \circ A_1 = A_{j'}$ by (2), and
$F f \circ Ff_{j} = Ff_{j'}$ by (1), so (3) commutes.\\
Suppose there exists another morphism
$B  \colon A \to F(1)$ such that

\begin{equation}
\begin{tikzcd}
    & A \ar[ddl, "A_j"] \ar[ddr, "A_{j'}"] \ar[d, "B"] &\\
    & F(1) \ar[dl, "Ff_j"] \ar[dr, "Ff_{j'}"] &\\
    Fj \ar[rr, "Ff"] & & Fj'
\end{tikzcd}
\end{equation}
Then choosing $j$ to be $1$, and
letting $f$ be the unique morphism $f  \colon 1 \to j'$, we get that
the following commutes

\begin{equation}
\begin{tikzcd}
    & A \ar[ddl, "A_1"] \ar[ddr, "A_{j'}"] \ar[d, "B"] &\\
    & F(1) \ar[dl, "F \mathbbm{1}_{1}"] \ar[dr, "Ff_{j'}"] &\\
    F(1) \ar[rr, "Ff"] & & Fj'
\end{tikzcd}
\end{equation}
where we can conclude $F f_j = F f_{1}$ to be $F \mathbbm{1}_1$ since
$f_1 \in \Hom\left( 1, 1 \right) 
= \left\{ \mathbbm{1}_1 \right\} $ as $1$ is initial.\\
Hence we get $B = B \circ \mathbbm{1}_{F(1)} = B \circ F \mathbbm{1}_1 = A_1$,
where
the first and second equalities follow by definition of a functor. Thus we get
uniqueness. So
$F(1)$ is the limit of the functor $F$ with
$\left\{ F(f_j) \right\}_{j \in J}$ being the legs of the universal cone
$\lambda  \colon F(1) \implies F$.\\
\linebreak
The dual of this result then states that
if $J$ has a terminal object, then the colimit of any functor
indexed by $J$ is the value of that functor at a terminal object.\\
\linebreak
Now, suppose we have a finite successor ordinal $\mathbbm{n}$ freely generated
by
\begin{equation*}
\begin{tikzcd}
    0 \to 1 \to 2 \to \ldots \to n-1
\end{tikzcd}
\end{equation*}
Then $n-1$ is a terminal object, since by (iv) on page 5, every non-identity
morphism can be uniquely factored as a composite of morphisms in the displayed
graph, so
if $f_j  \colon j \to n-1$ is a morphism, then
$f_j$ factors uniquely as the composite
$j \to j+1 \to \ldots \to  n-1$ in the graph above, and hence
if we call the composition $j\to j+1 \to \ldots \to n-1$ for $g_j$ then
$\Hom\left( j, n-1 \right) 
= \left\{ g_j \right\} $. In particular, $g_j$ exists for all
$j$ (since $n-1$ must also have an identity morphism and for any other $j$, we
have a morphism as a composition of morphisms in the graph). Since
$\left| \Hom(j, n-1) \right| = 1$ for all $j$, $n-1$ is a terminal object.\\
Since an infinite successor ordinal, $\alpha$, is a successor 
of the infinite ordinal, we do not run into the problem of not having a
right-most object, as in for $\mathbbm{\omega}$, instead we get
 \begin{equation*}
\begin{tikzcd}
    0 \to 1 \to 2 \to \ldots \to \alpha -1
\end{tikzcd}
\end{equation*}
for $\alpha$, so $\alpha-1$ becomes the terminal object with a completely
equivalent reasoning as for the finite case.\\
\linebreak
Now the dual version states that the colimit of any functor $F$
indexed by a successor ordinal, call it $\alpha$, is the value
$F\left( \alpha -1 \right) $ where
the legs of the universal cone are
$F(g_j)$.\\
\linebreak
\textbf{Exercise 3.2.iii}: For any pair of morphisms $f  \colon a \to b,
g  \colon c \to d$ in a locally small category $C$, construct the set of
commutative squares $\text{Sq}(f,g)$ :
\begin{equation*}
\begin{tikzcd}
    a \ar[d, "f"] \ar[r] & c \ar[d, "g"]\\
    b \ar[r] & d
\end{tikzcd}
\end{equation*}
from $f$ to $g $ as a pullback in $\Set$.\\
\linebreak
\textit{Solution:} 
We wish to find the set
\[
\left\{ (k,l) \in 
\Hom(b,d) \times  \Hom(a,c)  \mid l \circ f = g \circ k \right\} 
\] 
since any such $(k,l)$ define a commutative square of $\text{Sq}(f,g)$ and
conversely, any commutative square must define such a tuple of morphisms.\\
\linebreak

Consider the diagram
\begin{equation*}
\begin{tikzcd}
    \Hom(b,d) \ar[r, "f^{*}"] & \Hom(a,d) & \ar[l, "g_*"] \Hom(a,c)
\end{tikzcd}
\end{equation*}
By example 3.2.11, elements of the pullback of this pair of functions are
cones
\begin{equation*}
\begin{tikzcd}
    1 \ar[r] \ar[d] & \Hom(a,c) \ar[d, "g_*"] \\
    \Hom(b,d) \ar[r, "f^{*}"] & \Hom(a,d)
\end{tikzcd}
\end{equation*}
The data of this consists by example 3.2.11 of a pair of morphisms
$k \in \Hom(a,c)$ and $l \in \Hom(b,d)$ such that
$g \circ k = l \circ f$. I.e. the pullback is
 \[
\Hom(b,d) \times_{\Hom(a,d)} \Hom(a,c)
:= \left\{ (k,l) \in 
\Hom(b,d) \times  \Hom(a,c)  \mid l \circ f = g \circ k \right\} 
\] 
which is in bijective correspondence given by to the set of commutative squares
$\text{Sq}(f,g)$ :
 \[
     \left( k, l \right) 
     \in \Hom(b,d) \times_{\Hom(a,d)} \Hom(a,c)
     \leftrightarrow 
     \begin{tikzcd}
         a \ar[r, "k"] \ar[d,"f"] & c \ar[d, "g"]\\
         b \ar[r, "l"] & d
     \end{tikzcd}
\] 
by the definition since $l \circ f = g \circ k$ for all
$(k,l)  
     \in \Hom(b,d) \times_{\Hom(a,d)} \Hom(a,c)$.

























\end{document}
