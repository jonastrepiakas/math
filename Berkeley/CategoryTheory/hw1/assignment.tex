\documentclass[a4paper]{article}

\usepackage[margin=2.5cm]{geometry}
\usepackage[pdftex]{graphicx}
\usepackage[utf8]{inputenc}
\usepackage[T1]{fontenc}
\usepackage{textcomp}
\usepackage{babel}
\usepackage{amsmath, amssymb}
\usepackage[colorlinks=true,linkcolor=blue]{hyperref}
\usepackage{float}
\usepackage{mathrsfs}
%\usepackage{enumitem}
%% for identity function 1:
\usepackage{bbm}
%%For category theory diagrams:
%\usepackage{tikz-cd}

\newcommand{\incfig}[2][1]{%
\def\svgwidth{#1\columnwidth}
\import{./figures/}{#2.pdf_tex}
}


% figure support
\usepackage{import}
\usepackage{xifthen}
\pdfminorversion=7
\usepackage{pdfpages}
\usepackage{transparent}

\pdfsuppresswarningpagegroup=1

\setlength\parindent{0pt}

\newcommand{\qed}{\tag*{$\blacksquare$}}
\newcommand{\qedwhite}{\hfill \ensuremath{\Box}}

%Inequalities
\newcommand{\cycsum}{\sum_{\mathrm{cyc}}}
\newcommand{\symsum}{\sum_{\mathrm{sym}}}
\newcommand{\cycprod}{\prod_{\mathrm{cyc}}}
\newcommand{\symprod}{\prod_{\mathrm{sym}}}

%Linear Algebra

%Redeclaring Span and image
\DeclareMathOperator{\Span}{span}
\DeclareMathOperator{\Ima}{Im}
\DeclareMathOperator{\diag}{diag}
\DeclareMathOperator{\Ker}{Ker}
\DeclareMathOperator{\ob}{ob}

%Row operations
\newcommand{\elem}[1]{% elementary operations
\xrightarrow{\substack{#1}}%
}

\newcommand{\lelem}[1]{% elementary operations (left alignment)
\xrightarrow{\begin{subarray}{l}#1\end{subarray}}%
}

%SS
\DeclareMathOperator{\supp}{supp}
\DeclareMathOperator{\Var}{Var}

%NT
\DeclareMathOperator{\ord}{ord}

%Alg
\DeclareMathOperator{\Rad}{Rad}
\DeclareMathOperator{\Jac}{Jac}

\DeclareMathAlphabet{\pazocal}{OMS}{zplm}{m}{n}
\newcommand{\unif}{\pazocal{U}}

\begin{document}
    \textbf{1.1.i:} (i) 
Let $\mathcal{A}$ be a category and $A,B \in \ob(\mathcal{A})$. Let
$f \in \mathcal{A}\left( A,B \right) $ and assume 
$g,h \in \mathcal{B,A}$ such that
$fg = \mathbbm{1}_{B} = fh$ and $gf = \mathbbm{1}_{A} = hf$.
Then
\[
    h = h \mathbbm{1}_{B} = h (fg) = (hf) g = \mathbbm{1}_{A} g = g.
\] 

(ii): Let $f \colon x \to y$ and $g,h  \colon y \to x$ such that
$gf = \mathbbm{1}_{x}$ and $fh = \mathbbm{1}_{y}$. Then
\[
    g = g \mathbbm{1}_{y} = g \left( fh \right) = (gf) h = \mathbbm{1}_{x} h = h
\] 
Thus we can denote $g = f^{-1} = h$ and we get
$f f^{-1} = \mathbbm{1}_{y}$ and $f^{-1}f = \mathbbm{1}_{x}$, so 
$f$ is an isomorphism by definition with $f^{-1}$ as its inverse.\\
\linebreak
\textbf{1.1.iii:}\\
(i) Let $\ob \left( c /C \right) $ be all morphisms in $C$ with domain $c$.
Let $c /C \left( f,g \right) $ be all maps in $C$ from the codomain of  $f$ to
the codomain of $g$. For any $f,g,h \in \ob \left( c /C \right) $ and for any
$\alpha \in  c /C \left( f,g \right) $ and $\beta \in c /C \left( g,h \right)
$,
define the composition of $\alpha$ with $\beta$ as the map
$\beta \circ \alpha$ in $C$ whose existence is  guaranteed by $C$ being
a category.\\
For the identity: since $C$ is a category, we have that for each $x \in \ob(C)$,
there exists an identity on $x$, and thus since
for any map  $f \in \ob(c /C)$, say $f  \colon c \to x$, we have $f
= \mathbbm{1}_{x} f$ in $C$, so  $\mathbbm{1}_{x} \in c /C (f,f)$ where
$\mathbbm{1}_{x}$ represents the morphism $\left( f \colon c \to x \right) \to 
\left( f  \colon c \to x \right) $; and since
$x$ was arbitary, all objects in $c /C$ have an identity.\\
\linebreak
For associativity: let $\alpha \in c /C(f,g), \beta \in c /C (g,h), \gamma \in
c /C (h,k)$. Then the $(\gamma \beta) \alpha = \gamma (\beta \alpha)$ follows
from associativity of morphism composition in $C$.\\
Similarly, for the identity laws follow from the identity in $C$ : for any
$\alpha \in  c /C(f,g)$ where say $c \xrightarrow{f} x_f$ and
$c \xrightarrow{g} x_g$, we have $\mathbbm{1}_{x_f} \in c /C(f,f)$ and $\mathbbm{1}_{x_g}
\in  c /C(g,g)$ and $\mathbbm{1}_{x_g} \alpha = \alpha = \alpha
\mathbbm{1}_{x_f}$ since this is true in $C$.






















\end{document}
