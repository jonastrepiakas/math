\documentclass[a4paper]{article}

\usepackage[margin=2.5cm]{geometry}
\usepackage[pdftex]{graphicx}
\usepackage[utf8]{inputenc}
\usepackage[T1]{fontenc}
\usepackage{textcomp}
\usepackage{babel}
\usepackage{amsmath, amssymb}
\usepackage[colorlinks=true,linkcolor=blue]{hyperref}
\usepackage{float}
\usepackage{mathrsfs}
%\usepackage{enumitem}
%% for identity function 1:
%\usepackage{bbm}
%%For category theory diagrams:
\usepackage{tikz-cd}
%%For code (e.g. python) in latex:
%\usepackage{listings}
%
%Usage: 
%\begin{lstlisting}[language=Python]
%\end{lstlisting}

\newcommand{\incfig}[2][1]{%
\def\svgwidth{#1\columnwidth}
\import{./figures/}{#2.pdf_tex}
}


% figure support
\usepackage{import}
\usepackage{xifthen}
\pdfminorversion=7
\usepackage{pdfpages}
\usepackage{transparent}

\pdfsuppresswarningpagegroup=1

\setlength\parindent{0pt}

\newcommand{\qed}{\tag*{$\blacksquare$}}
\newcommand{\qedwhite}{\hfill \ensuremath{\Box}}

%Inequalities
\newcommand{\cycsum}{\sum_{\mathrm{cyc}}}
\newcommand{\symsum}{\sum_{\mathrm{sym}}}
\newcommand{\cycprod}{\prod_{\mathrm{cyc}}}
\newcommand{\symprod}{\prod_{\mathrm{sym}}}

%Linear Algebra

%Redeclaring Span and image
\DeclareMathOperator{\Span}{span}
\DeclareMathOperator{\Ima}{Im}
\DeclareMathOperator{\diag}{diag}
\DeclareMathOperator{\Ker}{Ker}
\DeclareMathOperator{\ob}{ob}


%Row operations
\newcommand{\elem}[1]{% elementary operations
\xrightarrow{\substack{#1}}%
}

\newcommand{\lelem}[1]{% elementary operations (left alignment)
\xrightarrow{\begin{subarray}{l}#1\end{subarray}}%
}

%SS
\DeclareMathOperator{\supp}{supp}
\DeclareMathOperator{\Var}{Var}

%NT
\DeclareMathOperator{\ord}{ord}

%Alg
\DeclareMathOperator{\Rad}{Rad}
\DeclareMathOperator{\Jac}{Jac}

\DeclareMathAlphabet{\pazocal}{OMS}{zplm}{m}{n}
\newcommand{\unif}{\pazocal{U}}

\begin{document}
    \textbf{1.3.iii:} Let $\mathcal{C}$ be a category consisting of
    objects $1,2,3,4$ and morphisms $1\to 2$ and $3\to 4$ with the identities
    on each object as well. This is clearly a category.\\
    Now define a category $\mathcal{D}$ consisting of $a,b,c$ with
    morphisms $a\to b, b\to c, a\to c$ and all identities. This is also
    a category.\\
    Define the funcotr $F  \colon \mathcal{C} \to \mathcal{D}$ collapsing
    $2,3$ to $b$ - i.e. $1 \to a$, $2,3 \to b$ and $4 \to c$ with
    $\left( 1\to 2 \right)  \to (a\to b)$ and
    $(3\to 4) \to ( b\to c)$ and identities mapped to identiteis.
    This satisfies the functoriality axioms since
    the only composable maps in $\mathcal{C}$ were with identities, and
    since identities are mapped to identities. However,
    $a\to c$ is not in the image of $F$ while $a\to b$ and $b\to c$ are in the
    image, hence the image is not a category.\\
    \linebreak
    \textbf{1.3.iv:} The functors $F=C \left( c, - \right) $ and
    $G=C\left( -, c \right) $ have been described
    in terms of objects and morphisms, so it remains to check the functoriality
    axioms.\\
    Let $f = x \to y$ and $g= y \to z$ and $h = gf = x \to z$ be morphisms in $C$. Then
    $F(gf) = F\left( h \right) = h_*$ and $F(g) F(f) = g_* f_*$.
    Now, for any  $\alpha \in C\left( c, x \right) $, we have
    \[
    h_* \left( \alpha \right) = h \alpha = gf \alpha = g\left( f \alpha \right) 
    =g_* \left( f \alpha \right) 
    = g_* \left( f_* \left( \alpha \right)  \right) 
    = g_* f_* \left( \alpha \right) 
    \] 
    so $F(gf) = F(g) F(f)$ for any composable $f,g$ in $C$.\\
    For each object  $a \in C$, we have
    $F(a) = C(c,a)$ and
    $F(1_{a}) = 1_{a*}$ given by: for any $\gamma \in C(c,a)$, we have
    $1_{a*}(\gamma) = 1_{a}\gamma = \gamma$ by composition in $C$, so $
    F(1_a) = 1_{a*}= 
    1_{C \left( c,a \right) } = 1_{F(a)}$.\\
    \linebreak
    Taking the dual, we get that the opposite functor
    $C^{op}$ is also a functor.\\
    \linebreak
    \textbf{1.4.i:} By the natural transformation $F \implies G$, we have
    that since $F(f) \alpha_c = F(f) \alpha_{c'}$ for an arbitrary morphism $f
     \colon c \to c'$, we have also
    $F(f) \alpha_c^{-1} =
    \alpha_{c'}^{-1} \alpha_{c'}F(f) \alpha_{c}^{-1}
    = \alpha_{c'}^{-1}G(f) \alpha_{c}\alpha_{c}^{-1}
    = \alpha_{c'}^{-1}G(f)$, so the following diagram commutes, and hece
    $\alpha^{-1}  \colon G \implies F$ is a natural transformation.


    % https://q.uiver.app/#q=WzAsNCxbMCwwLCJHKGMpIl0sWzIsMCwiRihjKSJdLFswLDIsIkcoYycpIl0sWzIsMiwiRihjJykiXSxbMCwyLCJHKGYpIl0sWzEsMywiRihmKSJdLFswLDEsIlxcYWxwaGFfY157LTF9IiwxXSxbMiwzLCJcXGFscGhhX3tjJ31eey0xfSIsMV1d
\[\begin{tikzcd}
	{G(c)} && {F(c)} \\
	\\
	{G(c')} && {F(c')}
	\arrow["{G(f)}", from=1-1, to=3-1]
	\arrow["{F(f)}", from=1-3, to=3-3]
	\arrow["{\alpha_c^{-1}}"{description}, from=1-1, to=1-3]
	\arrow["{\alpha_{c'}^{-1}}"{description}, from=3-1, to=3-3]
\end{tikzcd}\]

















\end{document}
