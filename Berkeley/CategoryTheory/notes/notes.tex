\documentclass[a4paper]{article}

\usepackage[margin=2.5cm]{geometry}
\usepackage[pdftex]{graphicx}
\usepackage[utf8]{inputenc}
\usepackage[T1]{fontenc}
\usepackage{textcomp}
\usepackage{babel}
\usepackage{amsmath, amssymb}
\usepackage[colorlinks=true,linkcolor=blue]{hyperref}
\usepackage{float}
\usepackage{mathrsfs}
%\usepackage{enumitem}
%% for identity function 1:
\usepackage{bbm}
%%For category theory diagrams:
\usepackage{tikz-cd}
%%For code (e.g. python) in latex:
%\usepackage{listings}
%
%Usage: 
%\begin{lstlisting}[language=Python]
%\end{lstlisting}

\newcommand{\incfig}[2][1]{%
\def\svgwidth{#1\columnwidth}
\import{./figures/}{#2.pdf_tex}
}


% figure support
\usepackage{import}
\usepackage{xifthen}
\pdfminorversion=7
\usepackage{pdfpages}
\usepackage{transparent}

\pdfsuppresswarningpagegroup=1

\setlength\parindent{0pt}

\newcommand{\qed}{\tag*{$\blacksquare$}}
\newcommand{\qedwhite}{\hfill \ensuremath{\Box}}

%Inequalities
\newcommand{\cycsum}{\sum_{\mathrm{cyc}}}
\newcommand{\symsum}{\sum_{\mathrm{sym}}}
\newcommand{\cycprod}{\prod_{\mathrm{cyc}}}
\newcommand{\symprod}{\prod_{\mathrm{sym}}}

%Linear Algebra

\DeclareMathOperator{\Span}{span}
\DeclareMathOperator{\Ima}{Im}
\DeclareMathOperator{\diag}{diag}
\DeclareMathOperator{\Ker}{Ker}
\DeclareMathOperator{\ob}{ob}
\DeclareMathOperator{\Hom}{Hom}
\DeclareMathOperator{\sk}{sk}
\DeclareMathOperator{\Vect}{Vect}
\DeclareMathOperator{\Set}{Set}
\DeclareMathOperator{\Group}{Group}
\DeclareMathOperator{\Ring}{Ring}
\DeclareMathOperator{\Ab}{Ab}
\DeclareMathOperator{\Top}{Top}
\DeclareMathOperator{\Htpy}{Htpy}
\DeclareMathOperator{\Cat}{Cat}
\DeclareMathOperator{\CAT}{CAT}


%Row operations
\newcommand{\elem}[1]{% elementary operations
\xrightarrow{\substack{#1}}%
}

\newcommand{\lelem}[1]{% elementary operations (left alignment)
\xrightarrow{\begin{subarray}{l}#1\end{subarray}}%
}

%SS
\DeclareMathOperator{\supp}{supp}
\DeclareMathOperator{\Var}{Var}

%NT
\DeclareMathOperator{\ord}{ord}

%Alg
\DeclareMathOperator{\Rad}{Rad}
\DeclareMathOperator{\Jac}{Jac}

\DeclareMathAlphabet{\pazocal}{OMS}{zplm}{m}{n}
\newcommand{\unif}{\pazocal{U}}

\begin{document}
    \subsection*{Representable functors}
P. 52\\
\textbf{Example (iv):} Is the forgetful functor
$U  \colon \Ring \to \Set$ representable?\\
\linebreak
\textit{Solution:} We want to find an element $R$
in $\Ring$ such that
the forgetful functor is naturally isomorphic to
the functor represented by $R$.\\
What are some natural considerations? $\mathbb{Z}$ again?\\
Suppose we map each
$f \in C(\mathbb{Z}, R)$ to the unique
element $f(1) \in U(R)$. This induces a well defined
map $\alpha_R  \colon C(\mathbb{Z}\to R) \to U(R)$ which is
bijective and hence an isomorphism. We have the following natural isomorphism
\[\begin{tikzcd}
	{C(\mathbb{Z}\to R)} && {U(R)} \\
	\\
	{C(\mathbb{Z}\to S)} && {U(S)}
	\arrow["{\alpha_R}", from=1-1, to=1-3]
	\arrow["{\phi_*}"', from=1-1, to=3-1]
	\arrow["{U(\phi)}", from=1-3, to=3-3]
	\arrow["{\alpha_{S}}"', from=3-1, to=3-3]
\end{tikzcd}\]
Naturality follows as
for an $f  \colon \mathbb{Z} \to R$ a ring homomorphism,
$\alpha_{S} \circ \varphi_* f$ maps to
the element
$\varphi \left( f(1) \right) \in U(S)$ and
$U(\varphi) \circ \alpha_R (f) =
U(\varphi) (f(1)) =
\varphi \left( f(1) \right) \in U(S)$, so it is indeed a natural isomorphism.

\textbf{Example (v):} The functor $U(-)^{n}  \colon
\Group \to \Set$ that sends a group $G$ to the set of $n$-tuples of
elements of $G$ is represented by the free group $F_n$ on $n$ generators
where $\alpha_G$ sends
the map which sends the generator $x_i \to g_i$ to
the set $(g_1, \ldots, g_n) \in \Ima U(-)^{n}$.
\[\begin{tikzcd}
	{C(F_n,G)} && {U(G)^n} \\
	\\
	{C(F_n,H)} && {U(H)^n}
	\arrow["{\varphi_*}", from=1-1, to=3-1]
	\arrow["{U(\varphi)^n}", from=1-3, to=3-3]
	\arrow["{\alpha_G}"{description}, from=1-1, to=1-3]
	\arrow["{\alpha_H}"{description}, from=3-1, to=3-3]
\end{tikzcd}\]

Now, since for a map
$f \in C(F_n, G)$ which maps
$x_i \to g_i$, we have
$\alpha_H \circ \varphi_* (f) = 
\alpha_H \left( \varphi \circ f \right) 
= \left( \varphi(g_1), \ldots, \varphi(g_n) \right) $ and
$U\left( \varphi \right)^{n} \circ
\alpha_G \left( f \right) = 
U\left( \varphi \right)^{n} \left( g_1, \ldots, g_n \right) 
= \left( \varphi(g_1), \ldots, \varphi(g_n) \right) $, the
square is indeed commutative, so as $\alpha_G$ is a bijection, 
we indeed have a natural isomorphism.\\
\linebreak

\textbf{Example (vi)} Given any group presentation such as
\[
S_3 := \langle s,t  \mid s^2 = t^2 =1, sts =tst \rangle
\] 
defines a functor $F  \colon \Group \to \Set$ that carries a group $G$ to the set
\[
\left\{ \left( g_1, g_2 \right) \in G^2  \mid 
g_1^2 = g_2^2 = e, g_1g_2g_1 = g_2g_1g_2 \right\} .
\] 
The functor is represented by the group admitting the given presentation, in
this case by $S_3$.\\
The presentation tells us that homomorphisms $S_3 \to G$ are
classified by pairs of elements $g_1, g_2 \in G$ satisfying the listed
relations.\\
\linebreak
\textbf{(vii)} The functor $(-)^{\times }  \colon \Ring \to \Set$ that sends
a unital ring to its set of units is represented by the ring
$\mathbb{Z} \left[ x, x^{-1} \right] $ of Laurent polynomials in one variable.

\[\begin{tikzcd}
	{C(\mathbb{Z}[x,x^{-1}],G)} && {G^{\times}} \\
	\\
	{C(\mathbb{Z}[x,x^{-1}],H)} && {H^{\times}}
	\arrow["{\varphi_*}", from=1-1, to=3-1]
	\arrow["{\varphi^{\times}}", from=1-3, to=3-3]
	\arrow["{(x \to u) \to u}"{description}, from=1-1, to=1-3]
	\arrow["{(x \to v) \to v}"{description}, from=3-1, to=3-3]
\end{tikzcd}\]

Let $f$ be the map sending $\varphi \in C\left( \mathbb{Z}[x,x^{-1}] ,G \right) $ 
to $\varphi(u)$ and $g$ similarly but with $H$ instead of $G$.\\
Then for $k \in C\left( \mathbb{Z}[x,x^{-1}],G \right) $,
$g \circ \varphi_* (k) = 
g \left( \varphi \circ k \right) 
= \varphi \left( k(x) \right) $ and
$\varphi^{\times }\circ f (k) =
\varphi^{\times } \left( k(x) \right) $ and since 
$k(x)$ is a unit in $G$, this last part is well-defined and equals
$\varphi \left( k(x) \right) $, hence we have naturality if $f,g$ are
bijections.\\
It suffices, by symmetry, to show it for $f$.\\
If $\varphi  \colon \mathbb{Z}[x,x^{-1}] \to G$ is a homomorphism, then
$1 = \varphi(1) = \varphi(x) \varphi(x^{-1})$, so
$\varphi(x)$ is a unit; so $f$ is indeed well-defined.\\
Injectivity follows since if $f(\varphi) = f (\psi)$, then
$\varphi(x) = \psi(x)$, so given any
$p \in \mathbb{Z}\left[ x,x^{-1} \right] $, we have
$p = \sum_{k\neq 0} a_k x^{k}$, so
$\varphi(p) = \sum_{k\neq 0} a_k \varphi(x)^{k}
= \sum_{k\neq 0} a_k \psi(x)^{k} = \psi(p)$, so
$\varphi = \psi$.\\
For surjectivity, simply note that
letting $\varphi(x) = u$ for some $u \in G^{\times }$, determines
a well-defined homomorphism, so $f$ is surjective.\\
This gives the result.\\
\linebreak
Thus this representation gives that any ring homomorphism
$\mathbb{Z}\left[ x,x^{-1} \right] \to R$ may be defined by sending 
$x$ to any unit of $R$ and is completely determined by the assignment,
moreover, there are no ring homomorphisms that carry $x$ to a non-unit.\\
\linebreak
\textbf{Example (viii)} The forgetful functor
$U  \colon \Top \to \Set$ is represented by the singleton space: there is
a natural bijection between elements of a topological space and continuous
functions from the one-point space:

\[\begin{tikzcd}
	{C(\{x\}, X)} && {U(X)} \\
	\\
	{C(\{x\},Y)} && {U(Y)}
	\arrow["{f_*}", from=1-1, to=3-1]
	\arrow["{U(f)}", from=1-3, to=3-3]
	\arrow[from=1-1, to=1-3]
	\arrow["{(x \to y') \to \{y' \}}"{description}, from=3-1, to=3-3]
	\arrow["{(x \to x') \to \{x' \}}"{description}, from=1-1, to=1-3]
\end{tikzcd}\]

First, we check commutativity. Let $f$ denote the map
sending a $\varphi \in C(\left\{ x \right\} ,X)$ to
$\left\{ \varphi(x) \right\} \in U(X)$ and
$g$ likewise for $Y$.\\
Now for a continuous map $h  \colon \left\{ x \right\} \to X$, we have
$g \circ f_* (h) = g \left( f \circ h \right) 
= f(h(x))$ and
$U(f) \circ f (h) = U(f) \circ h(x)
= f(h(x))$.\\
\linebreak
Bijectivity is clear, so we have a natural isomorphism.

\textbf{Example xiii:} The forgetful funcotr $U  \colon \Set_* \to \Set$ is
represented by the two-element based set:

\[\begin{tikzcd}
	{C(\{*,y\}\to U_*)} && U \\
	\\
	{C(\{*,y\}\to V_*)} && V
	\arrow["{f_*}", from=1-1, to=3-1]
	\arrow["f", from=1-3, to=3-3]
	\arrow["{((*,y)\to(*,u))\to u}", from=1-1, to=1-3]
	\arrow["{((*,y)\to(*,v))\to v}"', from=3-1, to=3-3]
\end{tikzcd}\]


\textbf{Example (xiv)} The functor $\text{Path}  \colon \Top \to \Set$ that
carries a topological space to its set of paths and the functor
$\text{Loop} \colon \Top_* \to \Set$ that carries a based space to its set of
based loops are each representable by definition, by the unit interval $I$ and
the based circle $S^{1}$, respectively.\\
\linebreak
\subsubsection*{Free categories}
\textbf{Example (ix), (x) and (xi):}\\
The functors $\ob, \text{mor}, \text{iso}  \colon \Cat \to \Set $ that take
a locally small category to, respectively, its set of objects,
its set of morphisms and its set of isomorphisms (point in a specified
direction) are represented, respectively, by the terminal category
$\mathbbm{1} $, the category $\mathbbm{2}$ and
the category $\mathbbm{I}$. In this sense, $\mathbbm{2}$ is the \textbf{free}
or \textbf{walking arrow}, and the category $\mathbbm{I}$ is the \textbf{free}
or \textbf{walking isomorphism}.\\
\linebreak
The adjective "free" is reserved for universal properties expressed by
covariant represented functors. It could be applied to any of the objects
listed above: e.g. $\mathbbm{2}$ is the free category with an arrow,
$S^{1}$ is the free space containing a loop. The dual term
"cofree" for universal properties expressed by contravariant represented
functors is less commonly used.\\
\linebreak
The following contravariant functors are representable:\\
\linebreak
\textbf{Example (i)} The contravariant power set functor
$P  \colon \Set^{op} \to \Set$ is represented by the set
$\\Omega = \left\{ \perp, \top \right\} $ with two elements.\\
The natural isomorphism $\Set (A, \Omega) \cong PA$ is defined by
the bijection that associates a function $A \to \Omega$ with the subset that is
the preimage of $\perp$ ; reversing perspectives, a subset $A' \subset A$ is
identified iwth its \textbf{classifying function}
$\chi_{A'}  \colon A \to \Omega$ which sends exactly the elements
of $A'$ to the element $\perp$. The naturality condition
stipulates that for any function $f  \colon A \to B$, the diagram

\[\begin{tikzcd}
	{\text{Set}(B, \Omega)} && PB \\
	\\
	{\text{Set}(A, \Omega)} && PA
	\arrow["{f^*}", from=1-1, to=3-1]
	\arrow["{f^{-1}}", from=1-3, to=3-3]
	\arrow["\cong", from=1-1, to=1-3]
	\arrow["\cong"', from=3-1, to=3-3]
\end{tikzcd}\]
commutes. That is, naturality asserts that given a function
$\chi_{B'}  \colon B \to \Omega$ classifying the subset
$B' \subset B$, the composite function
$A \stackrel{f}{\to } B \stackrel{\chi_{B'}}{\to } \Omega$ classifies the
subset
$f^{-1}(B') \subset A$. That is, that
$f^{-1}(B')$ is the same as
$P\circ f^{*} (\chi_{B'}) = 
P \left( \chi_{B'}\circ f \right) 
$ which is presicely
$(\chi_{B'} \circ f)^{-1}(\perp) =
f^{-1} \left( \chi_{B'}^{-1}(\perp) \right) 
= f^{-1} \left( B' \right) $.\\
\linebreak
\textbf{Example (ii):} The functor
$\mathcal{O}  \colon \Top^{op} \to \Set$ that sends a space to its set of open
subsets is represented by the \textbf{Sierpinski space} $S$, the topological
space with two points, one closed and one open. The natural bijection
$\Top (X, S) \cong \mathcal{O}(X)$ associates a continuous function
$X \to S$ to the preimage of the open points.

\[\begin{tikzcd}
	{Top(X,S)} && {O(X)} \\
	\\
	{Top(Y,S)} && {O(Y)}
	\arrow["{f^{-1}}", from=3-3, to=1-3]
	\arrow["{f^*}", from=3-1, to=1-1]
	\arrow["\cong"', from=3-1, to=3-3]
	\arrow["\cong", from=1-1, to=1-3]
\end{tikzcd}\]

Let $S = \left\{ 0,1 \right\} $ where $0$ is open. Then
naturality asserts that
for a map $g  \colon Y \to S$, the set
$f^{-1}(g^{-1}(0))$ is the same as
$(f^{*} (g))^{-1}(0) =
(g \circ f)^{-1}(0)
= f^{-1} \left( g^{-1}(0) \right) $.\\
\linebreak
\textbf{Example (iii):} The Sierpinski space also represents the functor
$\mathcal{C}  \colon \Top^{op} \to \Set$ that sends a space to its set of
closed subsets. Composing the natural isomorphism
$\mathcal{O} \cong \Top (-,S) \cong \mathcal{C}$, we see that
the closed set and open set functors are naturally isomorphic. The
composite natural isomorphism carries an open subset to its complement, which
is closed. This recovers the natural isomorphism described in Example
1.4.3(v).\\
\linebreak
\textbf{Example (iv)} The functor
$\Hom (-\times A, B)  \colon \Set^{op} \to \Set$ that sends
a set $X$ to the set of functions
$X \times A \to B$ is represented by the set
$B^{A}$ of functions from $A$ to $B$. That is, there is a natural bijection
between functions $X \times A \to B$ and functions $X \to B^{A}$. This
natural isomorphism is referred to as
\textbf{currying} in computer science; by fixing a variable in a two-variable
function, one obtains a family of functions in a single variable.

\[\begin{tikzcd}
	{C(X, B^A)} && {\text{Hom}(X \times A,B)} \\
	\\
	{C(Y,B^A)} && {\text{Hom}(Y \times A,B)}
	\arrow["{f_*}", from=1-1, to=3-1]
	\arrow["{(f,id)_*}", from=1-3, to=3-3]
	\arrow["\cong", from=1-1, to=1-3]
	\arrow["\cong"', from=3-1, to=3-3]
\end{tikzcd}\]

Suppose $g  \colon C(X,B^{A}) \to \Hom(X\times A,B)$ sends
$k \in C(X, B^{A})$ to
the map $X \times A \to B$ defined by
$(x,a) \to k(x)(a)$. This gives for any map
in  $C(X, B^{A})$ a map in $\Hom(X \times A,B)$.\\
Define $h  \colon C(Y, B^{A}) \to \Hom(Y \times A,B)$ similarly.\\
\linebreak
Conversely, for any $\varphi \in \Hom(X \times A,B)$, we can define
$k \in C(X, B^{A})$ by
sending $x \to \varphi(x, -)  \colon A \to B$.\\
We thus have an injection both ways, so by Cantor-Bernstein, the
sets are in bijections.\\
Naturality then asserts that for a $k \in C(X, B^{A})$,
$h \circ f_* (k) = h\left( k \circ f \right) $ which sends
$(y,a)$ to $k\circ f(y)(a)$ is the same as
$(f,id)_* g(k)$ which sends
$(y,a)$ to $(f(y), a)$ and then to
$k(f(y))(a) = k\circ f (y) (a)$.\\
\linebreak
\textbf{Example (v)} The functor $U(-)^{*}  \colon 
\Vect_{k}^{op} \to \Set$ that sends a vector space to the set of vectors in its
dual space is represented by the vector space $k$, i.e., linear maps
$V \to k$ are, by definition, precisely the vectors in the dual space
$V^{*}$.

\[\begin{tikzcd}
	{C(V,k)} && {U(V^*)} \\
	\\
	{C(W,k)} && {U(W^*)}
	\arrow["{f_*}", from=3-1, to=1-1]
	\arrow["{U(f_*)}"', from=3-3, to=1-3]
	\arrow[from=3-1, to=3-3]
	\arrow[from=1-1, to=1-3]
\end{tikzcd}\]


\textbf{2.2.2:} Suppose we have a $G$-set $X  \colon BG \to \Set$. Suppose now
we have a natural transformation $\varphi  \colon G \implies X$, which is a 
$G$-equivariant map $\varphi  \colon G \to X$. 
Now $\varphi (g) = g \cdot \varphi(e)$. Is the functor $G$ representable?

We have

\begin{equation*}
\begin{tikzcd}
    C(e,g) \arrow[d, "f_*"] \arrow[r, "\alpha \to G(\alpha( e))"] & G(g) \arrow[d, "G(f)"]\\
    C(e,h) \arrow[r] & G(h)
\end{tikzcd}
\end{equation*}

\textbf{Exercise 1.4.iv:} Prove that distinct parallel morphisms
$f, g  \colon c \to d$ define distinct natural transformations
\[
f_*, g_*  \colon C(-,c) \implies C(-,d) \quad
\text{and} \quad f^{*}, g^{*}  \colon C(d,-) \implies C(c,-)
\] 
\textit{Solution:}

\begin{equation*}
\begin{tikzcd}
    C(x,c) \arrow[r, "f_{*}"] \arrow[d, "h^{*}"] & C(x,d) \arrow[d,
    "h^{*}"]\\
    C(y,c) \arrow[r, "f_*"] & C(y,d)
\end{tikzcd}
\end{equation*}
Two natural transformations are distinct if the components are distinct.
But all components for $f_*$ are $f_*$ and similarly for $g_*$.\\
So in particular, $\alpha_c = f_*$, so
if the components were the same, then
$f =f \circ 1_{c} = f_* \left( 1_c \right) 
= g_* \left( 1_c \right) 
= g \circ 1_c = g$, contradiction.\\
\linebreak
\textbf{Proof of Cayley's theorem using the Yoneda embedding:}\\
Regard a group $G$ as a one-object category
$BG$. Since there is a single object in $BG$ whose endomorphisms define the
group $G$, there is a unique contravariant represented functor
$BG^{op} \to \Set$ which by example 1.3.9 corresponds to a right
$G$-set, with $G$ acting by right multiplication.\\
This is the image of the covariant Yoneda embedding
$BG \to \Set^{BG^{op}}$. Corollary 2.2.8 then tells us that the only
$G$-equivariant endomorphisms of the right $G$-set $G$ are those maps defined
by left multiplication by a fixed element of $G$, i.e. it is of the form
$g_*$ with $g \in G$.\\
In particular, any $G$-equivariant endomorphism of $G$ must be an automorphism,
a fact that is not otherwise obvious.\\
The Yoneda embedding thus defines an isomorphism between
$G$ and the automorphism group of the right $G$-set $G$, an object in
$\Set^{BG^{op}}$. Composing with the faithful forgetful functor
$\Set^{BG^{op}} \to \Set$, we obtain an isomorphism between
$G$ and a subgroup of the automorphism group
$\text{Sym}(G)$ of the set $G$.\\
\linebreak
\textbf{Exercise 2.2.iv:} Prove that the following are equivalent.\\
(i) $f  \colon x \to y$ is an iso in $C$.\\
(ii) $f_*  \colon C(-,x) \implies C(-,y)$ is a natural iso.\\
(iii) $f^{*}  \colon
C(y,-) \implies C(x,-)$ is a natural iso.\\
\linebreak
\textit{Solution:} 
$\left(i \right) \implies (ii)$ : 
for a morphism $h  \colon c \to d$ in $C$,
\begin{equation*}
\begin{tikzcd}
    C(d,x) \arrow[r, "f_*"] \arrow[d, "h^{*}"] & C(d,y) 
    \arrow[d, "h^{*}"]\\
    C(c,x) \arrow[r, "f_*"] & C(c,y)
\end{tikzcd}
\end{equation*}
and $(f^{-1})_* f_* = \mathbbm{1}$, so
$f_*$ is an iso. Hence $f_*$ is a natural iso with clear commutativity.\\
\linebreak
$(i) \implies (iii)$ similarly.\\
\linebreak
$(iii) \implies (i)$ : the Yoneda embedding gives a full and faithful embedding
sending $f$ to $f^{*}$. Since $f^{*}$ i a natural iso,
there exists a morphism $g'$ in 
$\Hom \left( C(-,d), C(-,c) \right) \cong \Hom (d,c)$, with corresponding
$g$ in $\Hom(d,c)$ such that $g_* = g$. Thus
$1 = g_* f_* = (gf)_*$ and similarly  $1 = (fg)_*$. Hence
$f$ is an iso.\\
Similarly for $(ii) \implies (i)$.\\
\linebreak
\textbf{Universal properties and universal elements:} this shows that
isomorphic objects $x \cong y$ are \textbf{representably isomorphic}, meaning
that $C(-,x) \cong C(-,y)$ and $C(x,-) \cong C(y,-)$ (exercise 2.2.iv).\\
The Yoneda lemma supplies the converse:\\
\textbf{Proposition 2.3.1}: Consider a pair of objects $x$ and $y$ in a locally
small category $C$.\\
(i) If either the co- or contravariant functors represented by $x$ and $y$ are
naturally isomorphic, then $x$ and $y$ are isomorphic.\\
(ii) In particular, if $x$ and $y$ represent the same functor, then $x$ and $y$
are isomorphic.\\
\linebreak
\textit{Proof:} The full and faithful Yoneda embeddings $C\to 
\Set^{C^{op}}$ and
$C^{op} \to \Set^{C}$ create isomorphism since fully faithful functors create
isomorphisms (exercise 1.5.iv), so an (natural) isomorphism between represented
functors is induced by a unique isomorphism between their representing objects,
which in particular must be isomorphic, proving (i) (uniqueness by
the fact that the Yoneda embeddings are faithful).\\
Now (ii) is an immediate consequence since given a functor represented by
both $x$ and $y$, the representing natural isomorphisms compose to demonstate
that $x$ and $y$ are represntably isomorphic.\\
 \linebreak
\begin{center}
    There may be many isomorphisms between the objects $x$ and $y$ appearing in
    the proof of proposition 2.3.1, but there is a unique natural isomorphism
    commuting with the chosen representations. On account of this, one
    typically refers to \textit{the} representing object of a representable
    functor. Category theorists often use the definite article "the" in
    contexts where the object in question is well-defined up to canonical
    isomorphism.
\end{center}
\textbf{Corollary 2.3.2:} The full subcategory of $C$ spanned by its terminal
objects is either empty or is a contractible groupoid. In particular, any two
terminal objects in $C$ are uniquely isomorphic.\\
\textbf{Def:} A \textbf{contractible groupoid} is a category that is equivalent
to the terminal category $\mathbbm{1}$. Explicitly, a contractible groupoid is
a category with a unique morphism in each hom-set.\\
\linebreak

\textbf{Proof:} For any terminal objects $t,t'$,
$C(t,t')$ is a singleton by definition. Proposition 2.3.1 then implies that
this morphism is an isomorphism: the objects $t$ and $t'$ are terminal iff they
represent the functor $*  \colon C^{op} \to \Set$ that is constant at the
singleton set.\\
\linebreak
\textbf{Def 2.3.3:} A \textbf{universal property} of an object $c \in C$ is
expressed by a representable functor $F$ together with a \textbf{universal
element} $x \in Fc$ that defines a natural isomorphism
$C(c, -) \cong F$ or $C(-,c) \cong F$, as appropriate, via the Yoneda lemma.\\
\linebreak
\subsection*{Limits and colimits}

\textbf{Def 3.1.1:} For any object $c \in C$ and any category
$J$, the \textbf{constant functor} $c  \colon J \to C$ sneds every object of
$J$ to $c$ and every morphism in $J$ to the identity morphism $1_c$. The
constant functors define an embedding $\Delta  \colon C 
\to C^{J}$ (recall an embedding is a faithful functor that is injective on
objects) that sends an object $c$ to the constant functor at $c$ and a morphism
$f  \colon c \to c'$ to the \textbf{constant natural transformation}, in
which each component is defined to be the morphism $f$; i.e.
$\Delta (f)  \colon \text{constant }c \implies \text{constant }c'$ has
components $f = \Delta(f)_j  \colon
\text{constant }c (j) \to \text{constant }c'(j)
= c \to c'$.\\
\linebreak
\textbf{Def 3.1.2:} A \textbf{cone over} a diagram
$F  \colon J \to C$ with \textbf{summit} or \textbf{apex} $c \in C$ is
a natural transformation $\lambda  \colon c \implies F$ whose domain is the
constant functor at $c$. The components 
$\left( \lambda_j  \colon c \to Fj \right)_{j \in J}$ of the natural
transformation are called the \textbf{legs} of the cone.\\
Explicitly:
\begin{itemize}
    \item The data of a cone over $F  \colon J \to C$ with summit $c$ is
        a collection of morphisms $\lambda_j  \colon C \to Fj$, indexed by the
        objects $j \in J$.
    \item A family of morphisms $\left( 
        \lambda_j  \colon c \to Fj \right)_{j \in J}$ defines a cone over $F$ 
        if and only if, for each morphism $f  \colon j\to k$ in $J$, the
        following triangle commutes in $C$ :
\end{itemize}

\begin{equation*}
\begin{tikzcd}
    & c \arrow[dl, "\lambda_j"] \arrow[dr, "\lambda_k"] & \\
    Fj \arrow[rr, "Ff"] & & Fk 
\end{tikzcd}
\end{equation*}
Dually, a \textbf{cone under} $F$ with \textbf{nadir} $c$ is a natural
transformation $\lambda F \implies c$, whose \textbf{legs} are the components 
$\left( \lambda_j  \colon F_j \to c \right)_{j \in J}$. The naturality
condition asserts that, for each morphism $f  \colon j \to k$ of $J$, the
triangle
\begin{equation*}
\begin{tikzcd}
    Fj \arrow[rr, "Ff"] \arrow[dr, "\lambda_j"] & & Fk \arrow[dl,
    "\lambda_k"]\\
                                                & c &
\end{tikzcd}
\end{equation*}
commutes in $C$.\\
Cones under a diagram are also called \textbf{cocones} - a cone under
$F  \colon J \to C$ is precisely a cone over
$F  \colon J^{op} \to C^{op}$.



\subsection*{Questions}

\begin{enumerate}
    \item On page 75, it says: "By the Yoneda lemma, a limit consists of an
        object $\text{lim}F \in C$ together with a universal cone
        $\lambda  \colon \text{lim} F \implies F$, called the limit cone,
        . . ." What does "universal" specifically mean here?
    \item 
\end{enumerate}

\textbf{Note:} I believe the proof of proposition 2.3.1 shows that an
isomorphism between represented functors is induced by a \textit{unique}
isomorphism between their representing objects, which in particular must be
isomorphic.



\end{document}
