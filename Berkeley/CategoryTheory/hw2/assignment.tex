\documentclass[a4paper]{article}

\usepackage[margin=2.5cm]{geometry}
\usepackage[pdftex]{graphicx}
\usepackage[utf8]{inputenc}
\usepackage[T1]{fontenc}
\usepackage{textcomp}
\usepackage{babel}
\usepackage{amsmath, amssymb}
\usepackage[colorlinks=true,linkcolor=blue]{hyperref}
\usepackage{float}
\usepackage{mathrsfs}
%\usepackage{enumitem}
%% for identity function 1:
\usepackage{bbm}
%%For category theory diagrams:
%\usepackage{tikz-cd}
%%For code (e.g. python) in latex:
%\usepackage{listings}
%
%Usage: 
%\begin{lstlisting}[language=Python]
%\end{lstlisting}

\newcommand{\incfig}[2][1]{%
\def\svgwidth{#1\columnwidth}
\import{./figures/}{#2.pdf_tex}
}


% figure support
\usepackage{import}
\usepackage{xifthen}
\pdfminorversion=7
\usepackage{pdfpages}
\usepackage{transparent}

\pdfsuppresswarningpagegroup=1

\setlength\parindent{0pt}

\newcommand{\qed}{\tag*{$\blacksquare$}}
\newcommand{\qedwhite}{\hfill \ensuremath{\Box}}

%Inequalities
\newcommand{\cycsum}{\sum_{\mathrm{cyc}}}
\newcommand{\symsum}{\sum_{\mathrm{sym}}}
\newcommand{\cycprod}{\prod_{\mathrm{cyc}}}
\newcommand{\symprod}{\prod_{\mathrm{sym}}}

%Linear Algebra

%Redeclaring Span and image
\DeclareMathOperator{\Span}{span}
\DeclareMathOperator{\Ima}{Im}
\DeclareMathOperator{\diag}{diag}
\DeclareMathOperator{\Ker}{Ker}
\DeclareMathOperator{\ob}{ob}


%Row operations
\newcommand{\elem}[1]{% elementary operations
\xrightarrow{\substack{#1}}%
}

\newcommand{\lelem}[1]{% elementary operations (left alignment)
\xrightarrow{\begin{subarray}{l}#1\end{subarray}}%
}

%SS
\DeclareMathOperator{\supp}{supp}
\DeclareMathOperator{\Var}{Var}

%NT
\DeclareMathOperator{\ord}{ord}

%Alg
\DeclareMathOperator{\Rad}{Rad}
\DeclareMathOperator{\Jac}{Jac}

\DeclareMathAlphabet{\pazocal}{OMS}{zplm}{m}{n}
\newcommand{\unif}{\pazocal{U}}

\begin{document}
    \textbf{1.2.ii:} \\
    (i) If $f  \colon x \to y$ is a split epimorphism then
    there exists $g  \colon y \to x$ such that $fg = \mathbbm{1}_{y}$.\\
    Let $c \in C$ be arbitrary, then for any
    $h \in C(c,y)$, we have
    $h = \mathbbm{1}_y h = fg h = f_* \left( gh \right) $ with $gh \in C(c,x)$
    by
    composition. So $f_*$ is surjective.\\
    \linebreak
    Conversely, if $f_*  \colon C(c,x) \to C(c,y)$ is surjective for
    all $c \in C$, then
    letting $c = y$, we get a map $f_*  \colon C(y,x) \to C(y,y)$ that is
    surjective. Since  $\mathbbm{1}_y \in C(y,y)$, let $
    g \in C(y,x)$ such that $fg = f_* (g) = \mathbbm{1}_y$. Then
    $f$ is a split epimorphism.\\
    \linebreak
    \textbf{1.2.iii:} \\
    (i): if  $f \colon x \to y$ and $g  \colon y\to x$ are monomorphisms, then
    if for maps $h,k$ we have
    $fg h = fg k$, then $gh = gk$ since $f$ is monic and then
    $h = k$ since $g$ is monic. Thus $gf  \colon x \to z$ is
    monic.\\
    \linebreak
    (ii) Assume $f  \colon x \to y$ and $g  \colon y \to z$ are morphisms so
    that
    $gf$ is monic. Now assume
    $fh = fk$ for some maps $h,k$ with codomain $x$. Then
    composing with $g$ on the left we get
    $gf h = gfk$ and since $gf$ is monic, we get $h =  k$. Therefore
    $f$ is monic.\\
    \linebreak
    We thus have
    if for  $f^{op}  \colon y \to x$ and $g^{op}  \colon z\to y$ monic, we have
    $\left( fg \right)^{op} = g^{op} f^{op}  \colon z \to x$ is
    monic.\\
    Taking the dual, noting that the dual of a monic function is an epic
    function, we get:
    if $f  \colon x \to y$ and $g  \colon y \to z$ are epic, then
    $fg  \colon x \to z$ is epic too.\\
    \linebreak
    Similarly, we have from (ii):
    if $f^{op}  \colon y\to x$ and $g^{op}  \colon z \to y$ are morphisms so
    that
    $\left( gf \right)^{op} =f^{op}g^{op}$ is monic, then
    $g^{op}$ is monic.\\
    \linebreak
    Taking the dual we get:
    if $f \colon x \to y$ and $g  \colon y \to z$ are morphisms so that
    $gf$ is epic then $g$ is epic.\\
    \linebreak
    Restricting the morphisms of a category to only its monomorphisms or
    restricting to only its epimorphisms thus gives a subcategory since
    composition is well defined by (i) and (i') and the identities are trivial
    monomorphisms and epimorphisms; the rest of the requirements follow
    directly from the parent category.\\
    \linebreak
    \textbf{1.3.i:} Any functor between groups must send the single group
    object to the single group object - so there is one possible map of
    objects. By the functoriality axioms, we must have that the identity
    of the group is mapped to the identity of the other group.\\
    The mapping of any morphism (any group element) must also fulfill 
    $F (g_1 g_2) = F(g_1) F(g_2)$ which simply says that $F$ is a group
    homomorphism.\\
    \linebreak
    Conversely, any group homomorphism also gives a functor between groups:
    if $\varphi  \colon G \to H$ is a group homomorphism. Then the
    corresponding
    functor, $F$, maps the single object $G$ to the single object $H$ and
    the morphism $g_1$ of $G$ is mapped to $\varphi (g_1)$. By this, we
    have $F(e_G) = \varphi(e_G) = e_H$ and
    $F(g_1 g_2) = \varphi (g_1 g_2) = \varphi (g_1) \varphi(g_2) = F(g_1)
    F(g_2)$
    so the functoriality
    axioms are satisfied. Hence $F$ is a functor, so group homomorphisms
    correspond to functors between group categories.
    
    





















\end{document}
