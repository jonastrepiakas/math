\documentclass[a4paper]{article}

\usepackage[margin=2.5cm]{geometry}
\usepackage[pdftex]{graphicx}
\usepackage[utf8]{inputenc}
\usepackage[T1]{fontenc}
\usepackage{textcomp}
\usepackage{babel}
\usepackage{amsmath, amssymb}
\usepackage[colorlinks=true,linkcolor=blue]{hyperref}
\usepackage{float}
\usepackage{mathrsfs}
%\usepackage{enumitem}
%% for identity function 1:
\usepackage{bbm}
%%For category theory diagrams:
\usepackage{tikz-cd}
%%For code (e.g. python) in latex:
%\usepackage{listings}
%
%Usage: 
%\begin{lstlisting}[language=Python]
%\end{lstlisting}

\newcommand{\incfig}[2][1]{%
\def\svgwidth{#1\columnwidth}
\import{./figures/}{#2.pdf_tex}
}


% figure support
\usepackage{import}
\usepackage{xifthen}
\pdfminorversion=7
\usepackage{pdfpages}
\usepackage{transparent}

\pdfsuppresswarningpagegroup=1

\setlength\parindent{0pt}

\newcommand{\qed}{\tag*{$\blacksquare$}}
\newcommand{\qedwhite}{\hfill \ensuremath{\Box}}

%Inequalities
\newcommand{\cycsum}{\sum_{\mathrm{cyc}}}
\newcommand{\symsum}{\sum_{\mathrm{sym}}}
\newcommand{\cycprod}{\prod_{\mathrm{cyc}}}
\newcommand{\symprod}{\prod_{\mathrm{sym}}}

%Linear Algebra

\DeclareMathOperator{\Span}{span}
\DeclareMathOperator{\Ima}{Im}
\DeclareMathOperator{\diag}{diag}
\DeclareMathOperator{\Ker}{Ker}
\DeclareMathOperator{\ob}{ob}
\DeclareMathOperator{\Hom}{Hom}
\DeclareMathOperator{\sk}{sk}
\DeclareMathOperator{\Vect}{Vect}
\DeclareMathOperator{\Set}{Set}
\DeclareMathOperator{\Group}{Group}
\DeclareMathOperator{\Ring}{Ring}
\DeclareMathOperator{\Ab}{Ab}
\DeclareMathOperator{\Top}{Top}
\DeclareMathOperator{\Htpy}{Htpy}
\DeclareMathOperator{\Cat}{Cat}
\DeclareMathOperator{\CAT}{CAT}


%Row operations
\newcommand{\elem}[1]{% elementary operations
\xrightarrow{\substack{#1}}%
}

\newcommand{\lelem}[1]{% elementary operations (left alignment)
\xrightarrow{\begin{subarray}{l}#1\end{subarray}}%
}

%SS
\DeclareMathOperator{\supp}{supp}
\DeclareMathOperator{\Var}{Var}

%NT
\DeclareMathOperator{\ord}{ord}

%Alg
\DeclareMathOperator{\Rad}{Rad}
\DeclareMathOperator{\Jac}{Jac}

\DeclareMathAlphabet{\pazocal}{OMS}{zplm}{m}{n}
\newcommand{\unif}{\pazocal{U}}

\begin{document}
    \textbf{2.2.i} State and prove the dual to Theorem 2.2.4, characterizing
    natural transformations $C(-, c) \implies F$ for a contravariant functor
    $F  \colon C^{op} \to \Set$.\\
    \linebreak
    \textit{Solution:} Taking the dual of the theorem, we have that
    for any contravariant functor 
    $F  \colon C^{op} \to \Set$, whose domain
    $C^{op}$ is locally small and any object $c \in C^{op}$, there
    is a bijection
    \[
    \Hom \left( C^{op}(-,c), F \right) \cong Fc
    \] 
    that associates a natural transformation
    $\alpha  \colon C^{op}\left( -, c \right) 
    \implies F$ to the element $\alpha_c (1_c^{op}) \in Fc$. Moreover, this
    correspondence is natural in both $c$ and $F$.\\
    \linebreak
    For the proof, we insert $C^{op}$ in the Yoneda lemma and
    choosing a covariant functor
    $F  \colon C^{op} \to \Set$, so we get
    \[
    \Hom \left( C^{op}(-,c),F \right) 
    = \Hom\left( C(c,-),F\right) \cong
     Fc.
    \] 
    But a covariant functor $F  \colon C^{op} \to \Set$ is just a contravariant
    functor $C \to \Set$ (which we usually just denote
    $F  \colon C^{op} \to \Set$ as in the statement of the dual theorem above),
    giving the desired bijection.\\
    We check naturality in the functor and object.\\

    
    Naturality in the functor asserts that given
    a natural transformation
    $\gamma  \colon F \implies G$, the element
    of $Gc$ representing the composite natural transformation
    $\gamma \alpha  \colon C^{op}(-,c) \implies F \implies G$ is the image
    under $\gamma_c  \colon Fc \to Gc$ of the element
    $Fc$ representing
    $\alpha  \colon C^{op}(-,c) \implies F$, i.e the diagram
    \begin{equation*}
    \begin{tikzcd}
        \Hom\left( C^{op}(-,c),F \right) 
        \arrow[r, "\Phi_F"] \arrow[d, "\gamma_*"] & Fc \arrow[d,
        "\gamma_c"] \\
        \Hom\left( C^{op}(-,c), G \right) \arrow[r, "\Phi_G"]
                            & Gc
    \end{tikzcd}
    \end{equation*}
    commutes in $\Set$. By definition,
    $\Phi_G \left( \gamma \alpha \right) 
    = \left( \gamma \alpha \right)_{c}\mathbbm{1}_{c}$ which is
    $\gamma_c (\alpha_c \mathbbm{1}_c)$ by the definition of vertical
    composition, and this is
    $\gamma_c \left( \Phi_F \left( \alpha \right)  \right) $.\\
    \linebreak
    Naturality in the object asserts that given a morphism
    $f  \colon d \to c$ in $C$, the element of
    $Fd$ representing the composite natural transformation
    $\alpha f_*  \colon C^{op} \left( -,d \right) 
    \implies C^{op} \left( -,c \right) \implies F$ is the image
    under $Ff  \colon Fc \to Fd$ of the element of $Fc$ that represents
    $\alpha$, i.e. the diagram
    
     \begin{equation*}
    \begin{tikzcd}
        \Hom\left( C^{op}(-,d),F \right) \arrow[r, "\Phi_d"] \arrow[d, "(f_*)^{*}"] &
        Fd \arrow[d, "Ff"] \\
        \Hom\left( C^{op}(-,c),F \right) \arrow[r, "\Phi_c"] & Fc
    \end{tikzcd}
    \end{equation*}
    commutes. Letting
    $\beta \in \Hom\left( C^{op}(-,d),F \right) $, we get
    $Ff \left( \beta_d (\mathbbm{1}_d) \right) $ along the top right, and
    $\Phi_c \left( \beta f_* \right) 
    = \left( \beta f_* \right)_c \left( \mathbbm{1}_c \right) $ along the
    bottom left. But
    $(\beta f_*)_c \left( \mathbbm{1}_c \right) $ by definition of vertical
    composition is the function
    $\mathbbm{1}_c \mapsto f \mapsto \beta_c (f)$ and
    $\beta_c (f) = Ff \left( \beta_d \left( \mathbbm{1}_d \right)  \right) $
    by the commutative square (2.2.6) on page 58 in Riehl.\\
    This gives naturality in the object.\\
    \linebreak
    
    
    



    



    \textbf{2.2.ii:} Explain why the Yoneda lemma does not dualize to classify
    natural transformations from an arbitrary set-valued functor to
    a represented functor.\\
    \linebreak
    \textit{Solution:}
    When we formulated the dual version, we replaced $C$ with $C^{op}$ and
    considered covariant functors $C^{op} \to \Set$. Taking the dual
    thus changed the direction of the morphisms in our category and
    the variance of our functor (as covariant $C^{op}\to \Set$ is the same as
    contravariant $C \to \Set$ ), but the direction of the functors
    (i.e. going $C \to \Set$ or $C^{op}\to \Set$ ) and natural
    transformations did not change, so the statement does not dualize to natural
    transformations going
    $F \implies C(c,-)$.
    
    
    
    















\end{document}
