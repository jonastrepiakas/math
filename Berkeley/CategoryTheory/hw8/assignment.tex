\documentclass[a4paper]{article}

\usepackage[margin=2.5cm]{geometry}
\usepackage[pdftex]{graphicx}
\usepackage[utf8]{inputenc}
\usepackage[T1]{fontenc}
\usepackage{textcomp}
\usepackage{babel}
\usepackage{amsmath, amssymb}
\usepackage[colorlinks=true,linkcolor=blue]{hyperref}
\usepackage{float}
\usepackage{mathrsfs}
%\usepackage{enumitem}
%% for identity function 1:
\usepackage{bbm}
%%For category theory diagrams:
\usepackage{tikz-cd}
%%For code (e.g. python) in latex:
%\usepackage{listings}
%
%Usage: 
%\begin{lstlisting}[language=Python]
%\end{lstlisting}

\newcommand{\incfig}[2][1]{%
\def\svgwidth{#1\columnwidth}
\import{./figures/}{#2.pdf_tex}
}


% figure support
\usepackage{import}
\usepackage{xifthen}
\pdfminorversion=7
\usepackage{pdfpages}
\usepackage{transparent}

\pdfsuppresswarningpagegroup=1

\setlength\parindent{0pt}

\newcommand{\qed}{\tag*{$\blacksquare$}}
\newcommand{\qedwhite}{\hfill \ensuremath{\Box}}

%Inequalities
\newcommand{\cycsum}{\sum_{\mathrm{cyc}}}
\newcommand{\symsum}{\sum_{\mathrm{sym}}}
\newcommand{\cycprod}{\prod_{\mathrm{cyc}}}
\newcommand{\symprod}{\prod_{\mathrm{sym}}}

%Linear Algebra

\DeclareMathOperator{\Span}{span}
\DeclareMathOperator{\Ima}{Im}
\DeclareMathOperator{\diag}{diag}
\DeclareMathOperator{\Ker}{Ker}
\DeclareMathOperator{\ob}{ob}
\DeclareMathOperator{\Hom}{Hom}
\DeclareMathOperator{\sk}{sk}
\DeclareMathOperator{\Vect}{Vect}
\DeclareMathOperator{\Set}{Set}
\DeclareMathOperator{\Group}{Group}
\DeclareMathOperator{\Ring}{Ring}
\DeclareMathOperator{\Ab}{Ab}
\DeclareMathOperator{\Top}{Top}
\DeclareMathOperator{\Htpy}{Htpy}
\DeclareMathOperator{\Cat}{Cat}
\DeclareMathOperator{\CAT}{CAT}
\DeclareMathOperator{\Cone}{Cone}


%Row operations
\newcommand{\elem}[1]{% elementary operations
\xrightarrow{\substack{#1}}%
}

\newcommand{\lelem}[1]{% elementary operations (left alignment)
\xrightarrow{\begin{subarray}{l}#1\end{subarray}}%
}

%SS
\DeclareMathOperator{\supp}{supp}
\DeclareMathOperator{\Var}{Var}

%NT
\DeclareMathOperator{\ord}{ord}

%Alg
\DeclareMathOperator{\Rad}{Rad}
\DeclareMathOperator{\Jac}{Jac}

\DeclareMathAlphabet{\pazocal}{OMS}{zplm}{m}{n}
\newcommand{\unif}{\pazocal{U}}

\begin{document}
    \textbf{B (3.1.ii):} For a fixed diagram $F \in C^{J}$, show that the cone
    functor $\Cone (-, F)$ is naturally isomorphic to
    $\Hom \left( \Delta (-), F \right) $, the restriction of the hom functor
    for the category $C^{J}$ along the constant functor embedding defined in
    3.1.1. Do you find this result surprising? Why or why not?\\
    \linebreak
    \textit{Solution:} We want to define a natural isomorphism
    $\alpha  \colon \Cone(-,F) \implies \Hom \left( \Delta (-),F \right) $.\\
    When we look at the definition of a cone over a diagram $F$ with apex $c$,
    we find it is a natural transformation $c \implies F$ which is
    in the set $\Hom \left( \Delta(c) , F \right) $, so we suspect that the
    natural isomorphism will be quite obvious when we write things out.\\
    \linebreak
    Suppose we have a cone $\lambda
    = \left( \lambda_i  \colon c \to Fi \right)_{i \in J}
    \in \Cone(c,F)$. This is locally, for
    a morphism $f  \colon i \to j$ in $J$, is of the form
    \begin{equation*}
    \begin{tikzcd}
        & c \arrow[dl, "\lambda_i"] \arrow[dr, "\lambda_j"] &\\
        Fi \arrow[rr, "Ff"] & & Fj
    \end{tikzcd}
    \end{equation*}
    Similarly, a natural transformation
    $\beta  \colon \Delta(c) \implies F$ is a collection of morphisms
    $\beta_i  \colon c \to F i$ such that for any $f  \colon i \to j$ in $J$, the following diagram commutes
    \begin{equation*}
    \begin{tikzcd}
        c \arrow[r, "\beta_i"] \arrow[d, "\mathbbm{1}_{\Delta c}"] & F(i) \arrow[d,
        "Ff"] \\
        c \arrow[r, "\beta_j"] & F(j)
    \end{tikzcd}
    \end{equation*}
    We note that the data of the cone $\Cone(c,F)$ is a collection of morphisms
    $\left( \lambda_i  \colon c \to F i \right)_{i \in J}$ which by definition
    define a natural transformation $\Delta(c) \implies F$. Now, since
    $J$ is small by definition of a diagram, we have that the collection
    $J$ has only a set's worth of arrow. Since any object has an identity
    morphism, we thus deduce that $J$ also only has a set's worth of objects,
    so we can conclude that
    $\left( \lambda_i  \colon c \to F_i \right)_{i \in J}$ is of set size.
    Therefore,
    we can consider the collection of morphisms as a set
    $\left\{ \lambda_i  \colon c \to F_i \right\}_{i \in J}
    \in \Set$ which is
    an element of $\Hom\left( \Delta(c), F \right) $.\\
    Now, define a collection of maps
    \begin{align*}
        \alpha_c  \colon \Cone(c,F) &\to \Hom (\Delta(c),F)\\
        \lambda = \left( \lambda_i  \colon c \to Fi \right)_{i \in J} 
        &\mapsto \left\{ \lambda_i  \colon c \to F_i  \colon i\in J \right\} 
    \end{align*}
    We claim that $\alpha$ is a natural isomorphism.\\
    \linebreak
    Consider a cone $\lambda = \left( \lambda_i  \colon c \to F_i \right)_{i
    \in J} \in \Cone(c,F)$. Then naturality asserts that for a morphism
    $f  \colon d \to c$ in $C$, we have
    $\Hom\left( \Delta(-), F \right)(f) \circ \alpha_c \left( \lambda \right) 
    = \alpha_d \circ \Cone(-,F)(f) (\lambda)$.\\
    Now
    \begin{align*}
        \Hom\left( \Delta(-), F \right) (f) \circ \alpha_c (\lambda)
        &= \Hom\left( \Delta(-),F \right)(f) \left( \left\{ \lambda_i  \colon c \to
        F i \right\}_{i \in J} \right) \\
        &= \left\{\lambda_i \circ f  \colon d \to Fi  \right\}_{i \in J}.
    \end{align*}
    And
    \begin{align*}
        \alpha_d \circ \Cone(-,F)(f)(\lambda)
        &= \alpha_d \left( \lambda_i \circ f  \colon d \to F_i \right)_{i \in
        J}\\
        &= \left\{ \lambda_i \circ f  \colon d \to F_i \right\}_{i \in J}.
    \end{align*}
    This gives naturality.\\
    \linebreak
    To see that $\alpha$ is an isomorphism, we must check that $\alpha_c$ is
    a bijection. We note that
    for any set $\left\{ \lambda_i  \colon c \to Fi  \colon
    i \in J\right\} \in \Hom\left( \Delta(c),F \right) $, the collection
    $\left( \lambda_i  \colon c \to Fi \right)_{i \in J}$ defines
    a cone over $F$ with summit $c$ since for any morphism $f  \colon j\to k$ in $J$, the
    square
    \begin{equation*}
    \begin{tikzcd}
        \Delta(c)(j) = c \arrow[r, "\lambda_j"] \arrow[d
        , "\Delta(c)(f) = \mathbbm{1}_c"] & F(j) \arrow[d,
        "Ff"]\\
        \Delta(c)(k) = c \arrow[r, "\lambda_k"] & F(k)
    \end{tikzcd}
    \end{equation*}
    commutes, and hence the following triangle commutes:
    \begin{equation*}
    \begin{tikzcd}
        & c \arrow[dl, "\lambda_i"] \arrow[dr, "\lambda_j"] &\\
        Fi \arrow[rr, "Ff"] & & Fj
    \end{tikzcd}
    \end{equation*}
    which by the comment on page 74 means that the family of
    morphisms  $\left( \lambda_i  \colon c \to Fi \right)_{i \in J}$ defines
    a cone over $F$ with summit $c$. So define a map
    $\beta_c  \colon \Hom(\Delta(c), F) \to \Cone(c,F)$ which maps
    $\left\{ \lambda_i  \colon c \to F i \right\}_{i \in J}$ to
    the cone $\lambda$. Then this is a two-sided inverse to  $\alpha_c$, so
    $\alpha_c$ is a bijection. Hence each
    $\alpha_c$ is invertible. So $\alpha$ is a natural isomorphism.\\
    \linebreak
    I don't find this result particularly surprising since cones over $F$ with
    summit $c$ is indeed just a natural transformation
    $c \implies F$, i.e. elements of
    $\text{Mor}(c, F)$. The only thing to note is that
    this seems to boil down to the fact that we are dealing with a diagram
    which is defined on a small category, thus making our collection of
    morphisms into a set; however, this is also clear from the
    definition of a cone, so it's still not particularly surprising.\\
    \linebreak
    \textbf{2.4.x:} Fixing two objects $A,B$ in a locally small category $C$,
    we define a functor
    \[
    C(A,-) \times C(B,-)  \colon C \to \Set
    \] 
    that carries an object $X$ to the set
    $C(A,X) \times C(B,X)$ whose elements are pairs of maps
    $a  \colon A \to X$ and $b  \colon B \to X$ in $C$. What would it mean for
    this functor to be representable?\\
    \linebreak
    \textit{Solution:} We will write $f \times g$ for $(f,g)$.\\
    Suppose
    $C\left( A,- \right) \times C\left( B,- \right) $ is representable by
    an element $c \in C$, so there is a natural isomorphism
    \[
        C(c,-) \stackrel{\alpha}{\cong} C(A,-) \times C(B,-).
    \] 
    This means that for any morphism
    $f  \colon X\to Y$ between two objects $X,Y \in C$, the following square commutes
    \begin{equation*} \begin{tikzcd}
        C(c,X) \arrow[r, "\alpha_X"] \arrow[d, "f_*"] & C(A,X) \times C(B,X) \arrow[d,
        "f_* \times f_*"]\\
        C(c,Y) \arrow[r, "\alpha_Y"] & C(A,Y) \times C(B,Y)
    \end{tikzcd}
    \end{equation*}
    Now, since $C$ is locally small, Yoneda gives that
    any such natural isomorphism $\alpha$ is in
    bijection with some element in
    $C(A,c) \times C(B,c)$, namely $\alpha_c \left( \mathbbm{1}_c \right) $.
    Thus there exists some morphisms $g  \in  C(A,c)$ and
    $h \in C(B,c)$ that uniquely represent and determine
    $\alpha$, so $\alpha_c \left( \mathbbm{1}_c \right) 
    = g \times h$.\\
    Now, suppose we have morphisms $p  \colon A \to Y$ and
    $q  \colon B \to Y$. Since $\alpha$ was an isomorphism, we can let
    $k = \alpha_Y^{-1}(p \times q) \in C(c,Y)$. Then letting
    $X = c$ we get that

    \begin{equation*} \begin{tikzcd}
        C(c,c) \arrow[r, "\alpha_c"] \arrow[d, "k_*"] & C(A,c) \times C(B,c) \arrow[d,
        "k_* \times k_*"]\\
        C(c,Y) \arrow[r, "\alpha_Y"] & C(A,Y) \times C(B,Y)
    \end{tikzcd}
    \end{equation*}
    commutes,
    so 
    \[
    k \circ g \times  k \circ h =
    \alpha_Y (k) = p \times q
    \] 
    We can depict this as
    \begin{equation*}
    \begin{tikzcd}
        A \arrow[r, "g"] \arrow[dr, "p"] & c \arrow[d, "\exists ! k"] &  B \arrow[l, "h"]
        \arrow[dl,
        "q"]\\
         & Y &
    \end{tikzcd}
    \end{equation*}
    commuting. Here $k$ is unique since if $k'$ also makes the above commute,
    then
    \[
    k' = k'_{*} \mathbbm{1}_c =
    \alpha_Y^{-1} \left( k_* \times k_* \right) \alpha_c \mathbbm{1}_c
    = \alpha_Y^{-1} \left( k_* \times k_* \right) 
    \left( g \times h \right) 
    = \alpha_Y^{-1} \left( k \circ g \times k \circ h \right) 
= \alpha_Y^{-1} \left( \alpha_Y (k) \right)  = k\]
    This universal property precisely determines
    $c$ together with the unique maps
    $g  \colon A \to c$ and
    $h  \colon B \to c$ to be the coproduct of $A$ and $B$, i.e.,
    $c = A \sqcup B$, where
    the underlying diagram is a discrete category of
    two elements.\\
    So a representation of
    $C(A,-) \times C(B,-)$ corresponds to a coproduct of
    $A$ and $B$.
    
    
    
    
    
    
    

    
    

















\end{document}
