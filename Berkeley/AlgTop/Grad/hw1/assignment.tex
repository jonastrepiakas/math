\documentclass[a4paper]{article}

\usepackage[margin=2.5cm]{geometry}
\usepackage[pdftex]{graphicx}
\usepackage[utf8]{inputenc}
\usepackage[T1]{fontenc}
\usepackage{textcomp}
\usepackage{babel}
\usepackage{amsmath, amssymb}
\usepackage[colorlinks=true,linkcolor=blue]{hyperref}
\usepackage{float}
\usepackage{mathrsfs}
%\usepackage{enumitem}

\newcommand{\incfig}[2][1]{%
\def\svgwidth{#1\columnwidth}
\import{./figures/}{#2.pdf_tex}
}


% figure support
\usepackage{import}
\usepackage{xifthen}
\pdfminorversion=7
\usepackage{pdfpages}
\usepackage{transparent}
\usepackage{bbm}

\pdfsuppresswarningpagegroup=1

\setlength\parindent{0pt}

\newcommand{\qed}{\tag*{$\blacksquare$}}
\newcommand{\qedwhite}{\hfill \ensuremath{\Box}}

%Inequalities
\newcommand{\cycsum}{\sum_{\mathrm{cyc}}}
\newcommand{\symsum}{\sum_{\mathrm{sym}}}
\newcommand{\cycprod}{\prod_{\mathrm{cyc}}}
\newcommand{\symprod}{\prod_{\mathrm{sym}}}

%Linear Algebra

%Redeclaring Span and image
\DeclareMathOperator{\Span}{span}
\DeclareMathOperator{\Ima}{Im}
\DeclareMathOperator{\diag}{diag}
\DeclareMathOperator{\Ker}{Ker}

%Row operations
\newcommand{\elem}[1]{% elementary operations
\xrightarrow{\substack{#1}}%
}

\newcommand{\lelem}[1]{% elementary operations (left alignment)
\xrightarrow{\begin{subarray}{l}#1\end{subarray}}%
}

%SS
\DeclareMathOperator{\supp}{supp}
\DeclareMathOperator{\Var}{Var}

%NT
\DeclareMathOperator{\ord}{ord}

%Alg
\DeclareMathOperator{\Rad}{Rad}
\DeclareMathOperator{\Jac}{Jac}

\DeclareMathAlphabet{\pazocal}{OMS}{zplm}{m}{n}
\newcommand{\unif}{\pazocal{U}}

\title{Assignment 1}
\author{Jonas Trepiakas - 3039733855 - jtrepiakas@berkeley.edu}
\date{}

\begin{document}
\maketitle
\newpage



\textbf{5:} Let $F  \colon X \times I \to X$ be a deformation retraction of $X$ onto
$x_0$ such that $F(x,0) = \mathbbm{1}$ and $F (X,1) = x_0$ and $F(x_0,t)= x_0$
for all $t$. Let $U \subset X$ be a neighborhood of $x_0$ in $X$. Since
$F$ is continuous, $F^{-1}(U)$ is open and nonempty as $x_0 \in F^{-1}(U)$,
and since $U$ is open and $I$ is open
in $I$, we have by definition of the product topology that
$U \times I$ is open in $X \times I$ and thus $F^{-1}(U) \cap U \times I$ is
open and nonempty as $\left\{ x_0 \right\}  \times I \subset F^{-1}(U) \cap
U \times I$. We can thus write $F^{-1}(U) \cap U \times I$ as a union of basis
elements $\bigcup_{\alpha} 
V_{\alpha} \cup W_{\alpha}$ where all $V_{\alpha}$ are open in $X$ and
$W_{\alpha}$ open in $I$. Since $\left\{ x_0 \right\} \times I \subset
F^{-1}(U)$, we have $\bigcup_{\alpha} W_{\alpha} = I$, and since $I$ is
compact, there exist $\alpha_1, \ldots, \alpha_n$ such that
$W_{\alpha_1} \cup \ldots \cup W_{\alpha_n} = I$. Then
$V = V_{\alpha_1} \cap \ldots \cap V_{\alpha_n} $ is open as the finite
intersection of open sets and nonempty as all contain $x_0$, and we find that
$F (V \times I) \subset U$ since for any $(x,t) \in V \times I$, there exists a 
$1 \le i \le n$ such that $(x,t) \in V_{\alpha_i} \times W_{\alpha_i}$ and
$F\left( V_{\alpha_1} \times W_{\alpha_1} \right) \subset F \left( F^{-1}\left(
U\right) \cap U \times I \right) \subset U$. Now $V \times I$ is open and satisfies
that the restriction $F|_{V \times I}  \colon V \times I \to U$ is well-defined and a homotopy between the
inclusion map
$\mathbbm{1}_{V}  \colon V \to U$  and the constant function sending $V$ to
$x_0$ since
$F(x,0) = \mathbbm{1}_{V}$ and $F(x,1) = x_0$, and it is continuous since
 for any open set $W$ of the subspace $U$, we have
 $F|_{V \times I}^{-1}\left( W \right) =
 F^{-1}\left( W \right) \cap V \times I$ which is the intersection of open sets
 and thus open.\\
 Hence the inclusion map $\mathbbm{1}_{V}  \colon V \to U$ is nullhomotopic.\\
\linebreak
\textbf{15:} We have $S^{\infty} = \bigcup_{n} S^{n}$ and each
$S^{n}$ can be built inductively from $S^{n-1}$ by attaching two $n$-cells (by
the procedure on page 7, Hatcher). If now $A$ is a subcomplex of  $S^{\infty}$ 
that contains an $n$-cell, it must also contain the boundary of this $n$-cell
since a subcomplex is a closed subspace. The boundary of an $n$-cell,
$\partial e^{n}$, is an $(n-1)$-cell which in the $n$-skeleton $S^{n}$ would be
$S^{n-1}$ in this cell structure. So if the subcomplex $A$ has finite dimension
$n$, then it is of the form $e_1^{n} \bigcup_{i<n} e_1^{i}\cup e_2^{i}$ or
$e_1^{n}\cup e_2^{n} \bigcup_{i<n} e_1^{i}\cup e_2^{i}$. If $A$ is infinite
dimensional, it must be $S^{\infty}$ since it is a union of cells and assume it
does not contain a cell $e_{z}^{l}$. Since it is not finite dimensional, it must
contain a cell $e^{j}$ with $j>l$ and hence by the finite case, it must contain
$\bigcup_{i < j} e_1^{i}\cup e_2^{i}$ which contains $e_{z}^{j}$.
          










\end{document}
