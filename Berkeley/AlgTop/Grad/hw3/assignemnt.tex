\documentclass[a4paper]{article}

\usepackage[margin=2.5cm]{geometry}
\usepackage[pdftex]{graphicx}
\usepackage[utf8]{inputenc}
\usepackage[T1]{fontenc}
\usepackage{textcomp}
\usepackage{babel}
\usepackage{amsmath, amssymb}
\usepackage[colorlinks=true,linkcolor=blue]{hyperref}
\usepackage{float}
\usepackage{mathrsfs}
%\usepackage{enumitem}
%% for identity function 1:
%\usepackage{bbm}
%%For category theory diagrams:
%\usepackage{tikz-cd}
%%For code (e.g. python) in latex:
%\usepackage{listings}
%
%Usage: 
%\begin{lstlisting}[language=Python]
%\end{lstlisting}

\newcommand{\incfig}[2][1]{%
\def\svgwidth{#1\columnwidth}
\import{./figures/}{#2.pdf_tex}
}


% figure support
\usepackage{import}
\usepackage{xifthen}
\pdfminorversion=7
\usepackage{pdfpages}
\usepackage{transparent}

\pdfsuppresswarningpagegroup=1

\setlength\parindent{0pt}

\newcommand{\qed}{\tag*{$\blacksquare$}}
\newcommand{\qedwhite}{\hfill \ensuremath{\Box}}

%Inequalities
\newcommand{\cycsum}{\sum_{\mathrm{cyc}}}
\newcommand{\symsum}{\sum_{\mathrm{sym}}}
\newcommand{\cycprod}{\prod_{\mathrm{cyc}}}
\newcommand{\symprod}{\prod_{\mathrm{sym}}}

%Linear Algebra

%Redeclaring Span and image
\DeclareMathOperator{\Span}{span}
\DeclareMathOperator{\Ima}{Im}
\DeclareMathOperator{\diag}{diag}
\DeclareMathOperator{\Ker}{Ker}
\DeclareMathOperator{\ob}{ob}


%Row operations
\newcommand{\elem}[1]{% elementary operations
\xrightarrow{\substack{#1}}%
}

\newcommand{\lelem}[1]{% elementary operations (left alignment)
\xrightarrow{\begin{subarray}{l}#1\end{subarray}}%
}

%SS
\DeclareMathOperator{\supp}{supp}
\DeclareMathOperator{\Var}{Var}

%NT
\DeclareMathOperator{\ord}{ord}

%Alg
\DeclareMathOperator{\Rad}{Rad}
\DeclareMathOperator{\Jac}{Jac}

\DeclareMathAlphabet{\pazocal}{OMS}{zplm}{m}{n}
\newcommand{\unif}{\pazocal{U}}

\begin{document}

    \textbf{1.2.4:} 
    Firstly project all points in $\mathbb{R}^3 - X$ 
    radially onto $S^{2}$ by the map
    $(x,t) \to x(1-t) + t \frac{x}{\|x \|}$. This is a deformation retraction, so
    the fundamental group is isomorphic to
    $\pi_1 \left( \mathbb{R}^3 -X \right) $.
    Our resulting space is the $2$-sphere with $2n$ 
    points removed, where $n$ is the number of lines
    in $X$. Choose any removed point and stereohraphically project the sphere
    with $2n$ holes onto $\mathbb{R}^2$ with
    $2n-1$ holes removed - this is a homeomorphism, so the fundamental group is
    invariant.\\
    \linebreak
    

    \textbf{Solution 1:} Let $X'$ denote this space. 
    We can now define a path $\gamma  \colon I
    \to X'$ which in the interval $\left[ \frac{k}{2n-1},
    \frac{k+1}{2n-1} \right] $ completes a loop around the $k$ th hole
    where we enumerate the holes in some arbitrary way.
    The resulting image is  $\lor_{2n-1}S^{1}$. Now, we define a 
    deformation retraction of $\mathbb{R}^2$ minus the holes onto
     $\gamma(I)$. For points not inside one of the circles, we
     send it to the closest point on $\gamma(I)$. For
     points inside $\gamma(I)$, we project from the hole radially out
     onto $\gamma(I)$, and on $\gamma(I)$ we let the deformation retraction be
     the identity. Then this function deformation retracts
     our space $X'$ onto $\lor_{2n-1}S^{1}$, so in particular,
     the fundamental groups are isomorphic, and by van Kampen, we
     have $\pi_1 (X') \cong \pi \left( \lor_{2n-1}S^{1} \right) 
     \cong *_{2n-1} \mathbb{Z} $ by example 1.21.\\
     \linebreak
     




     I hope it is okay if I include a second solution which I originally wrote
     that I'm not completely sure whether works.\\
     \textbf{Solution 2:}
    Now let $d$ denote the minimal distance between
    the $2n-1$ points. Then for each hole, $x_i$, we define
    $(x,t) \to \begin{cases}
        (t-1) x + t (\frac{x-x_0}{\| x - x_0 \|} \frac{d}{3} + x_0), & \|x
        - x_i\| \le \frac{d}{3}\\
        x, & \|x - x_i\| > \frac{d}{3}
    \end{cases} $
    for all $x$ in the plane with $2n-1$ holes,
    which is a deformation
    retraction that expands the holes into removed disks in the plane.\\
    \linebreak
    Now, the closed disks are compact, so there exists $R>0$ such that
    all the removed disks are within
    $B(0, R)$ in our space. Now deformation retract the space
    by
    $(x,t) \to x $ for all $t$ if $x \in B(0,R)$ and
    $(x,t) \to x (1-t) + R \frac{x}{\|x\|}t$ if $x \in 
    B\left( 0,R \right)^{c}$.

 Let $Y$ denote the space $D(0,R)$ with $2n-1$ disks removed. $Y$ is
 path-connected, so choose a basepoint $x_0 \in Y$
 and connect it to one circle which we then connect to the next circle by
 a path and so on with a path from the last circle to $x_0$, forming a loop.
 We wish to build $Y$ as a CW-complex, so to attach a 2-cell, we attach
 another edge with a point on its free end to any point on the 1-skeleton.
 To the free point, we attach an edge whose endpoints are both the free point,
 thereby creating a loop.\\
 \linebreak
 
 Thus $Y$ can be built as a CW-complex with $2n+1$ 0-cells
 (one for each circle that is the boundary of the removed disk, and one for the
 chosen basepoint and one for the free point), $4n$ 1-cells connecting, $2n-1$ used to construct circles and
 $2n-1$ used to connect the circle to the basepoint and $2$ used for the
 attaching map to attach to. To this
 $1$-skeleton, we attach a $2$-cell to get a disk with $4$
 open disks removed where we can view the disk as a square, letting one edge
 wrap around the original 1-skeleton, 2 edges fusing onto the last added edge
 and the fourth edge fusing onto the last added loop.\\
 Since $X^{1}$ is path-connected, we then find by proposition
 1.26 that $X^{1} \to X^{2}$ induces a surjection
 $\pi_1 \left( X^{1} \right) \to \pi_1 \left( X^2 \right) $
 whose kernel is generated by the attaching loop which can be decomposed as
 $a= a_1 a_2 \ldots a_{2n}$ where each $a_i$ is a generating loop for the
 corresponding circle's fundamental group. Thus
 the fundamental group of the space 
 is isomorphic to the fundamental group of
 the $1$-skeleton modulo $a$. Now, as each edge in the 1-skeleton is
 contractible, we get by page 11 that the quotient space obtained by collapsing
 this to a point is homopy equivalent to the one skeleton $X^{1}$; and
 by proposition 1.18, $\pi_1 (X^{1})$ is isomorphic to the fundamental group of
 the quotient space which is $\pi_1 \left( \lor_{2n} S^{1} \right) 
 = *_{2n} \mathbb{Z}$.  Therefore the fundamental group of the
 space is
 \[
 \pi_1 \left( X^2 \right) \cong
 *_{2n}\mathbb{Z} / \left( a_1 a_2 \ldots a_{2n} \right) 
 \cong *_{2n-1} \mathbb{Z}.
 \] 
 
 \textbf{1.2.22:} Since $X$ is a deformation retract
 of $\mathbb{R}^3 - K$ by assumption, we wish to calculate
 $\pi_1 (X)$. Firstly, $X$ is path connected, so pick any
  $p \in X$ on the plane. Next deformation retract
  the plane onto $p$ by
  $(x,t) \to (1-t)x + tp$. This collapses the edges of
  the $R_i$ to $p$ as well. Now, each
  $R_i$ is a rectangular piece, and we can deformation retract
  this to its edge - i.e. we deformation retract it along its length.
  Since the edges connect to the plane $T$ are identified, we end up with
  a circle for each $R_i$, wrapping around $K$ once. For the relation,
  choose a concrete point, say on the plane in the image where
  $R_i$ intersects $R_k$, i.e. the leftmost corner in the diagram from
  Hatcher. Then taking the path $x_i x_j x_i^{-1}$ gives a loop
  that is homotopy equivalent to the path $x_k$ by sliding the loop
  across the crossing using $S_l$.
  Thus $x_i x_j x_i^{-1} = x_k$ for each square $S_l$ where
  the indices are as in the figure in Hatcher.\\
  \linebreak
  (b) We have
  \[
  \pi_1 \left( \mathbb{R}^3 - K \right) 
  \cong \pi_1 (X) \cong
  \langle x_1, x_2, \ldots, x_n  \mid x_i x_j x_i^{-1} x_k^{-1}
  , \text{ for all } S_{l} \rangle
  \] 
  
  Abelianization of this results from letting the basis be the basis of a free
  abelian group. In this case, all relations
  for $S_l$ become
  \[
  x_i x_j x_i^{-1} x_k^{-1} = x_i x_i^{-1} x_j x_k^{-1}
  = x_j x_k^{-1}
  \] 
  So for any bridge, identify loops on each side.
  Since all rectangles are separated by some amount of rectangles and bridges,
this makes all loops identify, so we simply get
  $\pi_1 \left( \mathbb{R}^3 - K \right) \cong
  \langle x \rangle = \mathbb{Z}$
  
  under abelianization.

  



















\end{document}
