\documentclass[a4paper]{article}

\usepackage[margin=2.5cm]{geometry}
\usepackage[pdftex]{graphicx}
\usepackage[utf8]{inputenc}
\usepackage[T1]{fontenc}
\usepackage{textcomp}
\usepackage{babel}
\usepackage{amsmath, amssymb}
\usepackage[colorlinks=true,linkcolor=blue]{hyperref}
\usepackage{float}
\usepackage{mathrsfs}
%\usepackage{enumitem}
%% for identity function 1:
\usepackage{bbm}
%%For category theory diagrams:
%\usepackage{tikz-cd}
%%For code (e.g. python) in latex:
%\usepackage{listings}
%
%Usage: 
%\begin{lstlisting}[language=Python]
%\end{lstlisting}

\newcommand{\incfig}[2][1]{%
\def\svgwidth{#1\columnwidth}
\import{./figures/}{#2.pdf_tex}
}


% figure support
\usepackage{import}
\usepackage{xifthen}
\pdfminorversion=7
\usepackage{pdfpages}
\usepackage{transparent}

\pdfsuppresswarningpagegroup=1

\setlength\parindent{0pt}

\newcommand{\qed}{\tag*{$\blacksquare$}}
\newcommand{\qedwhite}{\hfill \ensuremath{\Box}}

%Inequalities
\newcommand{\cycsum}{\sum_{\mathrm{cyc}}}
\newcommand{\symsum}{\sum_{\mathrm{sym}}}
\newcommand{\cycprod}{\prod_{\mathrm{cyc}}}
\newcommand{\symprod}{\prod_{\mathrm{sym}}}

%Linear Algebra

%Redeclaring Span and image
\DeclareMathOperator{\Span}{span}
\DeclareMathOperator{\Ima}{Im}
\DeclareMathOperator{\diag}{diag}
\DeclareMathOperator{\Ker}{Ker}
\DeclareMathOperator{\ob}{ob}


%Row operations
\newcommand{\elem}[1]{% elementary operations
\xrightarrow{\substack{#1}}%
}

\newcommand{\lelem}[1]{% elementary operations (left alignment)
\xrightarrow{\begin{subarray}{l}#1\end{subarray}}%
}

%SS
\DeclareMathOperator{\supp}{supp}
\DeclareMathOperator{\Var}{Var}

%NT
\DeclareMathOperator{\ord}{ord}

%Alg
\DeclareMathOperator{\Rad}{Rad}
\DeclareMathOperator{\Jac}{Jac}

\DeclareMathAlphabet{\pazocal}{OMS}{zplm}{m}{n}
\newcommand{\unif}{\pazocal{U}}

\begin{document}
    \textbf{16:}\\
    (a) Assume there is a retraction $r  \colon \mathbb{R}^3 \to A$ where $A
    \cong S^{1}$. Then by proposition 1.17, the 
    homomorphism $i_*  \colon \pi_1 \left( \mathbb{R}^3 \right) 
    \to \pi_1\left( A \right) $ induced by the inclusion 
    $i  \colon A \to X$ is injective. However, since $A \cong S^{1}$, we have
    $\pi_1 \left( A \right) \approx \pi_1 \left( S^{1} \right) \approx
    \mathbb{Z} $,
    while $\pi\left( \mathbb{R}^3 \right) = 0$ since $\mathbb{R}^3$ is convex
    (example 1.4) - we dropped basepoints because both spaces are
    path-connected.\\
    Therefore there would be an injective map $\mathbb{Z} \to \left\{
    0 \right\}  $ by proposition 1.17 which is impossible.\\
    \linebreak
    (b) We have $\pi \left( S^{1} \times D^2 \right) \cong \pi\left( S^{1} \right)
    \times \pi \left( D^2 \right) \cong \pi(S_1) \cong \mathbb{Z}$.\\
    On the other hand, $\pi \left( S^{1} \times S^{1} \right) 
    \cong \pi\left( S^{1} \right) \times  \pi\left( S^{1} \right) \cong
    \mathbb{Z} \times \mathbb{Z}$.\\

    Where the first isomorphism in both cases follows from proposition 1.12 and the fact that
    $S^{1}$ and $D^2$ are path-connected, and the last isomorphism follows from
    theorem 1.7.\\
    If there existed a retraction from $S^{1}\times D^2$ to $S^{1}\times
    S^{1}$, then by proposition 1.17, there would exist an injective
    homomorphism from $\pi_1 \left( S^{1}\times S^{1} \right) \cong \mathbb{Z}
    \times \mathbb{Z}$ to
    $\pi_1 \left( S_1 \times D^2 \right) \cong \mathbb{Z}$ which is impossible:
    assume $\varphi  \colon \mathbb{Z} \times \mathbb{Z} \to \mathbb{Z}$ is an
    injective homomorphism. 
    Let $\varphi(1,0) = a, \varphi(0,1) = b$ with $a,b \neq 0$ since
    $\varphi$ is assumed to be injective. Then
    $\varphi (x,y) = ax + by$, and hence $\varphi(b,-a) = ab - ab = 0$, so 
    $a,b = 0$, contradiction.\\
    \linebreak
    (c) We first have that
    $\pi_1 \left( S^{1} \times D^2 \right) 
    \cong \pi_1 \left( S^{1} \right) \times \pi_1 \left( D^2 \right)  
    \cong \pi_1 \left( S_1 \right) \cong \mathbb{Z}$.\\
    Explicitly, we have:
    let $\varphi  \colon S^{1} \times D^2 \to S^{1}$ be the map of
    the filled torus to its central circle, i.e. the map which collapses each
    meridian circle
    $\{x\} \times S^{1} $ to a point. This is a deformation retraction,
    and we can thus for any loop $f  \colon I \to S^{1} \times D^2$
    compose $f$ with $\varphi$ to get a loop on
    $S^{1}$.\\
    Now take the
    loop in $A$ : $\gamma  \colon I \to A$ that completes exactly one cycle. Then
    with the inclusion $i  \colon A \to S^{1} \times D^2$, we
     have $i \gamma  \colon I \to S^{1} \times D^2$ is a loop in
     $S^{1} \times D^2$. Therefore
     $\left[ i \gamma \right] \in  \pi_1 \left( S^{1}\times D^2 \right) $.
     Now $\varphi$ induces a homomorphism $\varphi_*  \colon \pi_1 \left(
     S^{1}\times D^2 \right) \to \pi_1 \left( S^{1} \right)$ by
     $\varphi_* \left[ f \right] = \left[ \varphi f \right] $. So
     $\varphi_* \left[ i \gamma \right] = \left[ \varphi i \gamma \right] $.\\
     Thus $\varphi i \gamma$ is generated by the generating element
     of $\pi_1 \left( S^{1} \times D^2 \right) $, call it $a$ - where we have
     used theorem 1.7. By the
     projection, we see that $\varphi i \gamma$ corresponds to $a a^{-1}$ which
     is nullhomotopic to the constant loop at  $\varphi i \gamma (0)$ which we
     choose freely as our basepoint as $S^{1}$ is path-connected. Therefore
     $\left[ \varphi i \gamma \right] = \left[ 0 \right] $ where $0$ denotes
     the constant loop at the basepoint. Since $\varphi$ is
     a deformation retraction, the induced homomorphism is an isomorphism, so
      $\left[ i \gamma \right] = \left[ 0 \right] $.\\
      Now, if $S^{1} \times D^2$ were retractible to $A$, then the induced
      inclusion homomorphism $i_*  \colon \pi_1 \left( A \right) \to \pi_1
      \left( S^{1}\times D^2 \right) $ would map $\left[ 0 \right] $ and
      $\left[ \gamma \right] $ to different homotopy classes, but as we have
      seen, $\left[ i \gamma \right] = \left[ 0 \right] $ in $\pi_1 \left(
      S^{1} \times D^2 \right) $, and thus $i_*$ is not injective, so
      $S^{1}\times D^2$ is not retractible to $A$.
    \\
     \linebreak
     (d) Both $D^2 \lor D^2$ and $S^{1} \lor S^{1}$ are path-connected, so we
     can consider $\pi_1 \left( D^2 \lor D^2 \right) $ and $\pi_1 \left( 
     S^{1} \lor S^{1} \right) $. Since $D^2 \lor D^2$ is star shaped with
     respect to the connecting point, it is deformation retractible to a point
     and thus has trivial fundamental group. If we can show that $\pi_1 \left(
     S^{1} \lor S^{1} \right) $ is non-trivial, then we are done since the
     induced inclusion  $i_*  \colon \pi_1 \left( S^{1}\lor S^{1} \right) \to
     \left\{ 0 \right\} $ cannot be injective, and then the result follows
     by proposition 1.17.\\
     \linebreak
     Let $r|_{S_1^{1}}  \colon S_1^{1} \to S^{1}\lor S^{1}$ be the map sending one of
     the spheres of $S^{1}\lor S^{1}$ to the connecting point of $S^{1}\lor
     S^{1}$. Let $r|_{S_2^{1}} \colon S_2^{1} \to S^{1}\lor S^{1}$ be the
     identity on the other sphere. Since $S^{1}$ is closed and the intersection
     of the domains is the connecting point which is a closed set, we find by
     the pasting lemma a retraction $r \colon S^{1}\lor S^{1} \to S^{1} \lor
     S^{1}$ where $r(S^{1} \lor S^{1}) = S^{1}$ and $r|_{S_2^{1}}
     = \mathbbm{1}$.\\
      So there is
     a retraction onto $S^{1}$, but thus we get an injective inclusion
     $i_*  \colon \pi_1 \left( S^{1} \right)  \to  \pi_1 \left( S^{1} \lor
     S^{1} \right) $ from proposition 1.17, and since $\pi_1 \left( S^{1} \right) \cong \mathbb{Z}$, 
     $\pi_1 \left( S^{1} \lor S^{1} \right) $ cannot be trivial.\\
     \linebreak
     (e) Assume that $X$ is $S_1$ where $(0,1)$ and $(0,-1)$ are identified.
     Then there is a deformation retraction 
     $F((x,y),t) = t(\sqrt{1 - y^2} , y) + (1-t) \left( x,y \right) $ sending
     $S^{1}$ to the right side of $S^{1}$.\\ 
     Since the ends of this curve are identified, this is just a $1$-cell
     attached to a $0$-cell which is $S^{1}$.\\
     \linebreak
     Hence we have that if there is a retraction from the disk with two points
     on its boundary identified to its boundary $S^{1} \lor S^{1}$, then
     by proposition 1.17, it induces an injective homomorphism
     $i_*  \colon \pi_1 \left( S^{1} \lor S^{1} \right) \to \pi_1 \left( S^{1}
     \right) $; however, by the van Kampen Theorem (Example 1.21), we have
     $\pi_1 \left( S^{1} \lor S^{1} \right) \cong \mathbb{Z} * \mathbb{Z}$, and
     a map from $\mathbb{Z} * \mathbb{Z}$ into $\mathbb{Z}$ cannot be
     injective since the image would have to be an abelian subgroup, e.g.\\
     \linebreak
     
     
     

 
\textbf{20:} By lemma 1.19, we have $f_{0*} = \beta_h f_{1*}$. Let $x_0 \in X$
be any point and let $\left[ g \right]  \in \pi_1 \left( X, x_0 \right) $. Then
since $f_0$ and $f_1$ are identity maps, we have $f_{0*}$ and $f_{1*}$ are
identity maps, so
\[
\left[ g \right] = f_{0*} \left[ g \right] = \beta_h f_{1*}\left[ g \right] 
= \beta_h \left[ g \right] = \left[ h \cdot g \cdot \overline{h} \right]
= \left[ h \right] \left[ g \right] \left[ \overline{h} \right]
\] 

If we apply $\left[ h \right] $ on the right side, we get
\[
\left[ g \right] \left[ f_t(x_0) \right]
= \left[ g \right] \left[ h \right] = \left[ h \right] \left[ g \right] 
\left[ \overline{h} \right] \left[ h \right] 
= \left[ h \right] \left[ g \right] = \left[ f_t(x_0) \right] \left[ g \right] 
\] 
Since $\left[ g \right] \in \pi_1 \left( X, x_0 \right) $ and $x_0 \in X$ were
arbitrary,
we find that $f_t(x_0)$ is in the center of $\pi_1 \left( X, x_0 \right) $ for
any $x_0 \in X$.













\end{document}
