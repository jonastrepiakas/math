\documentclass[a4paper]{article}

\usepackage[margin=2.5cm]{geometry}
\usepackage[pdftex]{graphicx}
\usepackage[utf8]{inputenc}
\usepackage[T1]{fontenc}
\usepackage{textcomp}
\usepackage{babel}
\usepackage{amsmath, amssymb}
\usepackage[colorlinks=true,linkcolor=blue]{hyperref}
\usepackage{float}
\usepackage{mathrsfs}
%\usepackage{enumitem}

\newcommand{\incfig}[2][1]{%
\def\svgwidth{#1\columnwidth}
\import{./figures/}{#2.pdf_tex}
}


% figure support
\usepackage{import}
\usepackage{xifthen}
\pdfminorversion=7
\usepackage{pdfpages}
\usepackage{transparent}
\usepackage{bbm}

\pdfsuppresswarningpagegroup=1

\setlength\parindent{0pt}

\newcommand{\qed}{\tag*{$\blacksquare$}}
\newcommand{\qedwhite}{\hfill \ensuremath{\Box}}

%Inequalities
\newcommand{\cycsum}{\sum_{\mathrm{cyc}}}
\newcommand{\symsum}{\sum_{\mathrm{sym}}}
\newcommand{\cycprod}{\prod_{\mathrm{cyc}}}
\newcommand{\symprod}{\prod_{\mathrm{sym}}}

%Linear Algebra

%Redeclaring Span and image
\DeclareMathOperator{\Span}{span}
\DeclareMathOperator{\Ima}{Im}
\DeclareMathOperator{\diag}{diag}
\DeclareMathOperator{\Ker}{Ker}

%Row operations
\newcommand{\elem}[1]{% elementary operations
\xrightarrow{\substack{#1}}%
}

\newcommand{\lelem}[1]{% elementary operations (left alignment)
\xrightarrow{\begin{subarray}{l}#1\end{subarray}}%
}

%SS
\DeclareMathOperator{\supp}{supp}
\DeclareMathOperator{\Var}{Var}

%NT
\DeclareMathOperator{\ord}{ord}

%Alg
\DeclareMathOperator{\Rad}{Rad}
\DeclareMathOperator{\Jac}{Jac}

\DeclareMathAlphabet{\pazocal}{OMS}{zplm}{m}{n}
\newcommand{\unif}{\pazocal{U}}

\begin{document}
    

\textbf{Problem 1:} Show homotopy equivalence is an equivalence relation.\\
\linebreak
\textit{Solution:}
Let $A \simeq B \simeq C$. Let $f  \colon A \to B, g  \colon B\to A$ and
$\overline{f}  \colon B \to C, \overline{g}  \colon C \to B$ with
$fg \simeq \mathbbm{1}$ and $\overline{f}\overline{g} \simeq \mathbbm{1}$ by
$\gamma_1 = \mathbbm{1}, \gamma_0 = fg$ and $\alpha_1 = \mathbbm{1}, \alpha_0 =
\overline{f}\overline{g}$. Then
\[
F_t = \begin{cases}
    g \alpha_{2t} f,& t \in \left[ 0, \frac{1}{2} \right]\\
    \gamma_{2t -1},& t \in \left[ \frac{1}{2},1 \right] 
\end{cases}
\] 
is a homotopy equivalence between $A$ and $C$ hence showing transitivity.\\
\linebreak







\textbf{1.1.7:} The homotopy
\[
F  \colon \left( S^{1} \times  I \right) \times I \to S^{1} \times I
\] 
by
\[
F \left( \theta , s, t \right) =  \left( \theta + 2 \pi s t, s \right) 
\] 
Then $F\left( \theta , s , 0 \right) = \mathbbm{1}$ while
$F\left( \theta, s , 1 \right) = f\left( \theta , s \right) $ and
$F$ is continuous hence a homotopy - also, it fixed the circle for $s=0$.\\
\linebreak
Rest done on paper.\\
\linebreak

\textbf{1.1.8:} No, the proof of borsuk-ulam fails
when we conclude $\eta$ is nullhomotopic on $S^{1} \times S^{1}$ which it is
not.\\
More specifically, the natural projection of the torus onto the
2d-plane works.





\textbf{1.1.18:} Assume $\alpha \in \pi_1 \left( X \right) $.
Let $\varphi  \colon S^{1} \to X$ be the attaching map.
We can let $p  \colon I \to S^{1}$ by $t \to \left( \cos 2\pi t, \sin
2 \pi t\right) $. Then $\varphi \circ p$ is the attaching loop.\\
Let $x_0 = \left( \varphi \circ p \right) (0) \in A$.
Now, since $A$ is path-connected, there exists
a path $\gamma  \colon I \to A$ from $\alpha (0)$ to $x_0$.\\
Then $\gamma \alpha \overline{\gamma}$ is a loop
at $x_0$ in $A$.\\
\linebreak

Now choose a neighborhood around $x_0$ and let
$A_1$ be the union of this neighborhood and the attached
$n$-cell. Let $A_2$ be an $\varepsilon$-neighborhood of
$A$ in $X$ that deformation retracts onto $A$ - this
becomes $A$ union an annulus of the attached $n$-cell. 
And the annulus deformation retracts onto the attaching
circle of $A$.\\


The intersection of $A_1$ and $A_2$ is path-connected
 - the union of the annulus with the neighborhood around $x_0$ -, so
by lemma 1.15, the loop can be decomposed into running
in $A_1$ and $A_2$. Any loop in $A_1$ however is 
nullhomotopic in $X$, so it can be decomposed into loops
running in $A_2$ only. Now deformation retract $A_2$ to
$A$ and we find the loop  $\alpha$ can be decomposed
into loops running in $A$ only.\\
\linebreak









\textbf{1.1.19:} Let $\gamma  \colon I \to X$ be a 
loop in $X$. The CW complex is composed of
a 0-skeleton with edges which are path connected.


















\end{document}
