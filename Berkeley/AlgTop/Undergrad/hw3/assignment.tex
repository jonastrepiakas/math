\documentclass[a4paper]{article}

\usepackage[margin=2.5cm]{geometry}
\usepackage[pdftex]{graphicx}
\usepackage[utf8]{inputenc}
\usepackage[T1]{fontenc}
\usepackage{textcomp}
\usepackage{babel}
\usepackage{amsmath, amssymb}
\usepackage[colorlinks=true,linkcolor=blue]{hyperref}
\usepackage{float}
\usepackage{mathrsfs}
%\usepackage{enumitem}
%% for identity function 1:
\usepackage{bbm}
%%For category theory diagrams:
%\usepackage{tikz-cd}
%%For code (e.g. python) in latex:
%\usepackage{listings}
%
%Usage: 
%\begin{lstlisting}[language=Python]
%\end{lstlisting}

\newcommand{\incfig}[2][1]{%
\def\svgwidth{#1\columnwidth}
\import{./figures/}{#2.pdf_tex}
}


% figure support
\usepackage{import}
\usepackage{xifthen}
\pdfminorversion=7
\usepackage{pdfpages}
\usepackage{transparent}

\pdfsuppresswarningpagegroup=1

\setlength\parindent{0pt}

\newcommand{\qed}{\tag*{$\blacksquare$}}
\newcommand{\qedwhite}{\hfill \ensuremath{\Box}}

%Inequalities
\newcommand{\cycsum}{\sum_{\mathrm{cyc}}}
\newcommand{\symsum}{\sum_{\mathrm{sym}}}
\newcommand{\cycprod}{\prod_{\mathrm{cyc}}}
\newcommand{\symprod}{\prod_{\mathrm{sym}}}

%Linear Algebra

\DeclareMathOperator{\Span}{span}
\DeclareMathOperator{\Ima}{Im}
\DeclareMathOperator{\diag}{diag}
\DeclareMathOperator{\Ker}{Ker}
\DeclareMathOperator{\ob}{ob}
\DeclareMathOperator{\Hom}{Hom}
\DeclareMathOperator{\sk}{sk}
\DeclareMathOperator{\diam}{diam}


%Row operations
\newcommand{\elem}[1]{% elementary operations
\xrightarrow{\substack{#1}}%
}

\newcommand{\lelem}[1]{% elementary operations (left alignment)
\xrightarrow{\begin{subarray}{l}#1\end{subarray}}%
}

%SS
\DeclareMathOperator{\supp}{supp}
\DeclareMathOperator{\Var}{Var}

%NT
\DeclareMathOperator{\ord}{ord}

%Alg
\DeclareMathOperator{\Rad}{Rad}
\DeclareMathOperator{\Jac}{Jac}

\DeclareMathAlphabet{\pazocal}{OMS}{zplm}{m}{n}
\newcommand{\unif}{\pazocal{U}}

\begin{document}
    \textbf{p. 47, 3:} Use the Heine-Borel theorem to show that an infinite subset of
    a closed interval must have a limit point.\\
    \linebreak
   \textit{Solution:} Assume for contradiction that $I$ is an closed interval
   and an infinite subset $A \subset I$ has no limit point.\\
   Now, a point $x \in I$ is a limit point of $A$ if and only if for any
   neighborhood $N$ of $x$ in $I$, there exists a point $a \in A \cap (N- \left\{
   x \right\} )$, so negating each side we find that
   $x \in I$ is not a limit point of $A$ if and only if there exists
   a neighborhood $N$ of $x$ in $I$ such that 
   $A \cap \left( N-\left\{ x \right\}  \right) = \varnothing$. Now, since
   $A$ has no limit point in $I$, we have that for all $x \in I$, we can find
   a neighborhood  $N_x$ of $x$ in $I$ such that $A\cap (N_x - \left\{ x \right\}
   ) = \varnothing$; hence
   $A \cap N_x \subset \left\{ x \right\} $. Now, $I = \bigcup_{x \in I} \left\{ x \right\} 
   \subset \bigcup_{x \in I} N_x \subset I$, so
   $I = \bigcup_{x \in I} N_x$. Now, $I$ is compact by Heine-Borell, so
   there exists a finite subcover $N_{x_1}\cup \ldots \cup N_{x_n} = I$. 
   Then by construction
   \[
       A =  A \cap I = A \cap \left( N_{x_1} \cup \ldots \cup N_{x_n} \right) 
       = \left( A \cap N_{x_1} \right) \cup \ldots
       \cup \left( A \cap N_{x_n} \right) 
   \subset \left\{ x_1 , \ldots , x_n \right\},
   \] 
   contradicting $A$ being infinite.\\
   \linebreak
   
   



















\textbf{p. 50., 14:} Let $f  \colon X \to Y$ be an injective continuous map.
If we restrict it to a function  $f  \colon X \to f(X)$ then
$f$ is injective and surjective. We have that $X$ is Hausdorff; now we claim
$f(Y)$ is Hausdorff with the induced subspace topology.\\
Let $x,y \in f(X)$. Then $x,y \in Y$, so there exist open sets
$U,V$ in $Y$ such that $x \in U, y \in V$ and $U \cap V = \varnothing$.
Then the sets $U' = U \cap f(X)$ and $V' = V \cap f(X)$ are open in
the subspace topology by definition, and as $x,y \in f(X)$, also
$x \in U'$ and $y \in V'$, and 
$U' \cap V' \subset U \cap V = \varnothing$, so $U' \cap V' = \varnothing$.
Hence
$f(X)$ is Hausdorff. Now by theorem 3.7, 
$f  \colon X \to f(X)$ is a homeomorphism, so by definition, 
$f $ is an embedding of $X$ in $Y$.\\
\linebreak
\textbf{p. 55. 21:} If $A$ and $B$ are compact, and if $W$ is a neighborhood of
$A \times B$ in $X \times Y$, find a neighborhood $U$ of $A$ in $X$ and
a neighborhood $V$ of $B$ in $Y$ such that
$U \times V \subset W$.\\
\linebreak
\textit{Solution:} Fix an $a \in A$. Then
for each $b \in B$, since $W$ is a neighborhood of $(a,b)$, we can
find a basis element $(a,b) \in U_b \times V_b \subset W$ with
$U_b$ and $V_b$ neighborhoods of respectively $a$ and $b$ in respectively $X$
and $Y$. Now, $\bigcup_{b \in B} V_b$ covers $B$, so as $B$ is compact, there
exists a finite subcover $V_{b_1} \cup \ldots \cup  V_{b_n}$. Now let
$V_a  = V_{b_1} \cup \ldots \cup  V_{b_n}$.
$N_a = U_{b_1} \cap \ldots \cap U_{b_n}$. Then we claim
$\left\{ a \right\} \times B \subset N_a \times V_a
\subset W$. Let $(a,b) \in \left\{ a \right\} \times B$. 
Since $V_a$ covers $B$, we can find a $V_{b_i}$ containing $b$, so
$(a,b) \in N_a \times V_{b_i} 
\subset N_a \times V_a$ giving the first inclusion.\\
Now if $(a',b') \in N_a \times V_a$, then
there exists $b_i$ such that $(a',b') \in N_a \times V_{b_i}$ and hence
$(a',b') \in U_{b_i} \times V_{b_i} \subset W$, giving the
other inclusion.\\
\linebreak
Now $\bigcup_{a \in A}  N_a$ covers $A$ with open sets as $N_a$ is a finite
intersection of open sets, hence as $A$ is compact, there exists a finite
subcover
$N_{a_1} \cup \ldots \cup N_{a_m}$. Let $V = V_{a_1} \cap \ldots \cap V_{a_m}$
and
$U = N_{a_1} \cup \ldots \cup N_{a_m}$. Both $U$ and $V$ are open as the union
of open sets and the finite intersection of open sets, and
we claim $A \times B \subset U \times V \subset W$.\\
For the first inclusion, let $(a,b) \in A \times B$. 
Then as $U$ covers $A$, there exists $a_i$ such that
$a \in N_{a_i}$, and as each $V_{a_j}$ contains $B$, 
$b \in V_{a_j}$ as well, so $(a,b) \in \left( N_{a_i} \times  V_{a_{1}} \right)
\cap \ldots \cap \left( N_{a_i} \times  V_{a_m} \right) 
= N_{a_i} \times  V \subset U \times V$.\\
\linebreak
Now, if $(u,v) \in U \times V$, then there exists $N_{a_i}$ such that
$u \in N_{a_i}$, so
$(u,v) \in N_{a_i} \times V \subset N_{a_i} \times V_{a_i} \subset W$, giving
the other inclusion.



















\end{document}
