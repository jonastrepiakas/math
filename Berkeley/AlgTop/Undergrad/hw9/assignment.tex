\documentclass[a4paper]{article}

\usepackage[margin=2.5cm]{geometry}
\usepackage[pdftex]{graphicx}
\usepackage[utf8]{inputenc}
\usepackage[T1]{fontenc}
\usepackage{textcomp}
\usepackage{babel}
\usepackage{amsmath, amssymb}
\usepackage[colorlinks=true,linkcolor=blue]{hyperref}
\usepackage{float}
\usepackage{mathrsfs}
%\usepackage{enumitem}
%% for identity function 1:
\usepackage{bbm}
%%For category theory diagrams:
%\usepackage{tikz-cd}
%%For code (e.g. python) in latex:
%\usepackage{listings}
%
%Usage: 
%\begin{lstlisting}[language=Python]
%\end{lstlisting}

\newcommand{\incfig}[2][1]{%
\def\svgwidth{#1\columnwidth}
\import{./figures/}{#2.pdf_tex}
}


% figure support
\usepackage{import}
\usepackage{xifthen}
\pdfminorversion=7
\usepackage{pdfpages}
\usepackage{transparent}

\pdfsuppresswarningpagegroup=1

\setlength\parindent{0pt}

\newcommand{\qed}{\tag*{$\blacksquare$}}
\newcommand{\qedwhite}{\hfill \ensuremath{\Box}}

%Inequalities
\newcommand{\cycsum}{\sum_{\mathrm{cyc}}}
\newcommand{\symsum}{\sum_{\mathrm{sym}}}
\newcommand{\cycprod}{\prod_{\mathrm{cyc}}}
\newcommand{\symprod}{\prod_{\mathrm{sym}}}

%Linear Algebra

\DeclareMathOperator{\Span}{span}
\DeclareMathOperator{\Ima}{Im}
\DeclareMathOperator{\diag}{diag}
\DeclareMathOperator{\Ker}{Ker}
\DeclareMathOperator{\ob}{ob}
\DeclareMathOperator{\Hom}{Hom}
\DeclareMathOperator{\sk}{sk}
\DeclareMathOperator{\Vect}{Vect}
\DeclareMathOperator{\Set}{Set}
\DeclareMathOperator{\Group}{Group}
\DeclareMathOperator{\Ring}{Ring}
\DeclareMathOperator{\Ab}{Ab}
\DeclareMathOperator{\Top}{Top}
\DeclareMathOperator{\Htpy}{Htpy}
\DeclareMathOperator{\Cat}{Cat}
\DeclareMathOperator{\CAT}{CAT}


%Row operations
\newcommand{\elem}[1]{% elementary operations
\xrightarrow{\substack{#1}}%
}

\newcommand{\lelem}[1]{% elementary operations (left alignment)
\xrightarrow{\begin{subarray}{l}#1\end{subarray}}%
}

%SS
\DeclareMathOperator{\supp}{supp}
\DeclareMathOperator{\Var}{Var}

%NT
\DeclareMathOperator{\ord}{ord}

%Alg
\DeclareMathOperator{\Rad}{Rad}
\DeclareMathOperator{\Jac}{Jac}

\DeclareMathAlphabet{\pazocal}{OMS}{zplm}{m}{n}
\newcommand{\unif}{\pazocal{U}}

\begin{document}
    \textbf{p. 109:}\\
    \textbf{25:} Show that the punctured torus deformation-retracts onto the
    one-point union of two circles.\\
    \linebreak
    \textit{Solution:} 
    Let $X = I^2 - (\frac{1}{2}, \frac{1}{2})$. Then we can
    give  $X$ the usual partition for the torus (i.e. we paste opposite sides
    of the square) inducing an identification space $X^{*}$ for the punctured
    torus where we can assume without loss of generality that the usual image
    in the torus of
     $(\frac{1}{2}, \frac{1}{2})$ is the removed point. Let $\pi  \colon X \to X^{*}$ denote
     the identification map.\\
     Now, define a map
    $F  \colon X \times I \to X$ given by
    \[
    F((x,y),t) = \left( \frac{t}{2 \max \left\{ 
    \left| x - \frac{1}{2} \right| ,
\left| y - \frac{1}{2} \right| \right\} }
+ 1 -t \right) \left(x-\frac{1}{2}, y - \frac{1}{2} \right) + \left( \frac{1}{2},
\frac{1}{2} \right).
    \] 
    This is continuous. Furthermore,
    $F\left( (x,y), 1 \right) 
    \in \partial I^2$, and
    for $(x,y) \in \partial I^2$,
    $\max \left\{ 
    \left| x- \frac{1}{2} \right| ,
\left| y - \frac{1}{2} \right| \right\} = \frac{1}{2}$, so
    \[
    F((x,y), 1)
    = (x,y).
    \] 
    Hence $F$ is a deformation retraction of $X$ onto
    $\partial I^2$. Thus
    $\pi \circ F$ is a deformation retraction of
    $X$ onto $\pi \left( \partial I^2 \right) $.\\
    \linebreak
    We wish to show that $\pi \circ F$ factors through $X^{*}$.\\
    \linebreak
    For any $x',y' \in X$ with $x'\neq y'$ such that $\pi(x') = \pi(y')$, we have
    that $x',y' \in \partial I^2$ and are symmetric about either the line
    $x = \frac{1}{2}$ or the line $y = \frac{1}{2}$. But 
    $\pi \circ F$ maps such points to the same point, so we can define
    a map $\tilde{f}  \colon X^{*} \times I \to \pi\left( \partial I^2 \right) $ such
    that
    $\pi \circ F = \tilde{f} \circ \left( \pi, \mathbbm{1} \right) $ and by theorem 4.1, 
    $\tilde{f}$ is continuous - $\left( \pi, \mathbbm{1} \right) $ is
    an identification map as a consequence of exercise 11, section 29 in
    Munkres,
    and since $I = \left[ 0,1 \right] $ is a locally compact
    Hausdorff space.\\
    \linebreak
    Thus $\tilde{f}(\pi (x),0) = \pi(x)$,
    $\tilde{f}\left( \pi (x),1 \right) 
    = \pi \left( F(x,1) \right) \in \pi \left( \partial I^2 \right) $, and
    for $\pi(x) \in \pi \left( \partial I^2 \right) $, we have
    $\tilde{f}\left( \pi (x), 1 \right) 
    = \pi \left( F(x,1) \right) 
    = \pi(x)$, so $\tilde{f}$ is a deformation retraction of
    $X^{*}$ onto $\pi \left( \partial I^2 \right) $.\\
    \linebreak
    \textbf{26:} Consider the following examples of a circle $C$ embedded in
    a surface $S$ :\\
    (a) $S =$ Möbius strip, $C=$ boundary circle;\\
    (b) $S =$ torus, $C=$ diagonal circle $=\left\{ \left( x,y \right) 
    \in S^{1} \times S^{1}  \mid x=y \right\} $ ;\\
    (c) $S =$ cylinder, $C=$ one of boundary circles.\\
    \linebreak
    In each case, choose a base point in $C$, describe generators for the
    fundamental groups of $C$ and $S$, and write down in terms of these
    generators the homomorphisms of fundamental groups induced by the inclusion
    of $C$ in $S$.\\
    \linebreak
    \textit{Solution:} \\
    (a) Consider the Möbius strip as the identification space of
    $X = I^2$ with $(0,t)$ and $(1,1-t)$ identified for all $t$. Suppose we choose
    as a base point for $C$ the point
    $p = (0,0)$ and let $\pi  \colon X \to X^{*} = S$ be the identification map.\\
    \linebreak
    Then letting $\gamma  \colon I \to X$ be given by
     \[
    \gamma(t) = \begin{cases}
        (2t,0), & t \in \left[ 0, \frac{1}{2} \right] \\
        (2t-1, 1), & t\in \left[ \frac{1}{2},1 \right] 
    \end{cases},
    \] 
    $\pi \circ \gamma  \colon I \to X^{*}$ is the boundary circle.\\
    \linebreak
    Now, consider the path $\alpha  \colon I \to X$ given by
    \[
    \alpha (t) = (t, t).
    \] 
    Since $(0,0)$ and $(1,1)$ are identified, 
    $ g := \pi \circ \alpha$ is a loop in $S$.\\
    \linebreak
    Define the map $F  \colon X \times I \to X$ by
     \[
    F((x,y),t) =
    (x,y + t (x-y)).
    \] 
    $F$ is a deformation retraction of 
    $X$ onto the image of $\alpha$. Now, since
    \begin{align*}
        \pi \circ F((0,s),t) 
        &= \pi \left( 0, s (1-t) \right) \\
   &= \pi \left( 1, 1 - (1-t)s \right)\\
   &= \pi \left( 1, 1 -s +ts \right) \\
   &= \pi \left( 1, (1-s) + t (1 - (1-s)) \right) \\
   &= \pi \circ F \left( (1, 1-s), t \right)  
    \end{align*}
    we have that $\pi \circ F$ factors through
    $X^{*} \times I$, so there exists
    a map $\tilde{f}  \colon X^{*} \times I \to X^{*}$ such that
    \[
    \pi \circ F = \tilde{f} \circ \left( \pi, \mathbbm{1} \right).
    \] 
    Again, we have that $\tilde{f}$ is continuous as in problem 25, so
    $\tilde{f}$ is a deformation retraction of
    $X^{*}$ onto the image of $\pi \circ \alpha$ which is a circle.\\
    Thus $\pi_1 \left( S, p(0,0) \right) \cong \mathbb{Z}$ and furthermore,
    since $\tilde{f} (-,1)_*$ is an isomorphism according to theorem 5.18, we
    have that
    $\tilde{f} \left( -, 1 \right)_* \left( \pi \circ \alpha \right) 
    = \pi \circ \alpha$ which is an generator of the image of
    $\tilde{f}\left( -, 1 \right)$ which is $\pi \circ \alpha$. Hence
    $\pi \circ \alpha$ is also a generator for
    $X^{*} = S$.\\
    Let $i  \colon C \to S$ be the inclusion map.\\
    Then $i \circ \pi \circ \gamma$ is a path in $S$ and
    $$\tilde{f} \left( i \circ \pi \circ \gamma (s) ,t \right) 
    = \pi \circ F \left( \gamma (s), t \right) 
    = \pi \circ 
    \begin{cases}
        \left( 2s, 2t s \right), & t\in \left[ 0, \frac{1}{2} \right] \\
        \left( 2s -1, 1 + t \left( 2s - 2 \right)  \right),& t \in \left[
        \frac{1}{2},1 \right] 
    \end{cases} $$
   is a homotopy of $\pi \circ \gamma$ to
   $\pi \circ \alpha. \alpha$. Thus
   $\langle \pi \circ \gamma \rangle 
   = \langle \pi \circ \alpha . \alpha \rangle $, so
    the generator $\pi \circ \gamma$ in $C$ is equivalent to
    winding around the Möbius strip twice. I.e. the inclusion induces the map
    $i_*  \colon \mathbb{Z} \cong \pi_1 (C, p(0,0)) \to \pi_1 \left( S, p(0,0) \right)
    \cong \mathbb{Z}$ given
    by
    $i_* (1) = 2$.\\
    \linebreak
    (b) We have $\pi_1 (S) = \pi_1 (S^{1} \times S^{1})
    \cong \pi_1 (S^{1}) \times \pi_1 \left( S^{1} \right) 
    \cong \mathbb{Z} \times \mathbb{Z}$ and
    the fundamental group of  $C$ is $\mathbb{Z}$ as it is a circle.\\
    Let $X = \mathbb{R}^2$ and identify $(a,b+t) \sim (c,d+t)$ and
    $(a+s,b) \sim (c+s,d)$  for $a,b,c,d \in Z$ and $s,t \in I$, giving the identification space of the
    torus. Let $\pi  \colon X \to X^{*} = S$ be the identification map.\\
    Let $\pi(0,0)$ be the chosen base point.\\
    Then the diagonal circle  $C$ is given by $\pi \circ \alpha$ with
    $\alpha  \colon I \to X$ given by $\alpha(t)
    = (t,t)$, and $\langle \pi \circ \alpha \rangle $ is a generator for
    $\pi_1 \left( C, p(0,0) \right) $\\
    \linebreak 
    Now, we can view the above in terms of orbit space by letting
    $\mathbb{Z} \times \mathbb{Z}$ act
    on the plane as a group of homeomorphisms. As in the proof of theorem 5.13,
    we then find that defining $\varphi  \colon G \to 
    \pi_1 \left( X /G, \pi(x_0) \right) $ by
    $\varphi (g) = \langle \pi \circ \gamma \rangle $ where 
    $x_0 \in X$ and $\gamma$ is a path joining $x_0$ to $g(x_0)$ given some $g
    \in G$ gives an isomorphism of
    $G$ and $\pi_1 \left( X /G, \pi (x_0) \right) $.
    Now, $(1,0)$ and $(0,1)$ generate $\mathbb{Z} \times \mathbb{Z}$,
    so their images generate 
    $\pi_1 \left( X /G , \pi(x_0) \right) 
    = \pi_1 \left( S , \pi(0,0) \right) $.\\
    Since $(1,0).(0,0) = (1,0)$ where the left element is a group element of
    $\mathbb{Z} \times \mathbb{Z}$ acting on the right element $(0,0)$ which is
    our base point in $\mathbb{R}^2$, and in the same way
    $(0,1).(0,0) = (0,1)$, we have that
    letting $\beta  \colon I \to X$ be the straight line connecting
    $(0,0)$ and $(0,1)$ and $\delta  \colon I \to X$ be the straight line
    connecting $(0,0)$ and $(1,0)$,
    $\varphi (0,1) = \langle \pi \circ \beta \rangle $ and
    $\varphi (1,0) = \langle \pi \circ \delta \rangle $ generate
    $\pi_1 (S)$ which correspond to the meridean and longitudinal circle on the
    torus. Furthermore, commutativity in $\mathbb{Z} \times \mathbb{Z}$ of
    $(0,1)$ and $(1,0)$ implies commutativity 
    $\langle \pi \circ \beta \rangle $ and $\langle \pi \circ \delta \rangle
    $ in
    $\pi_1 (S)$.\\
    Let $i  \colon C \to S$ denote the inclusion. We claim that
    $\langle i \circ \pi \circ \alpha \rangle 
    = \langle \pi \circ \delta \rangle \langle \pi \circ \beta \rangle $.\\
   We can define a map
   $F((x,y),t) = (x,y) + t \left( 
   \left( \frac{x+y}{2}, \frac{x+y}{2} \right) - (x,y)\right) $. This is a
   homotopy rel $\left\{ 0,1 \right\} $ of
   $\alpha$ and $\delta . \beta_1$ where
   $\beta_1(t)$ is given by $\beta (t) + (1,0)$. Now, $
   \pi \left( \beta \right) 
   = \pi \left( \beta_1 \right) $, so 
   $\langle i \circ \pi \circ \alpha \rangle 
   = \langle \pi \circ \delta. \beta_1 \rangle 
   = \langle \pi \circ \delta . \pi \circ \beta_1 \rangle 
   = \langle \pi \circ \delta . \pi \circ \beta \rangle 
   = \langle \pi \circ \delta \rangle 
   \langle \pi \circ \beta \rangle $.\\
   \linebreak
   Hence $i_*  \colon \mathbb{Z} \cong 
   \pi_1 \left( C, \pi(0,0) \right) 
   \to  \pi_1 \left( S, \pi(0,0) \right) 
   \cong \mathbb{Z} \times \mathbb{Z}$ is given by
   $i_*(1) = (1,1)$, so
   $i_*(n) = (n,n)$.\\
   \linebreak
   (c) We consider the cylinder as embedded as
   $S^{1} \times I$, so $S = S^{1} \times I$. Let
   $C = S^{1} \times  \left\{ 0 \right\} $, and let 
   $(1,0)$ be the base point. Then $\gamma  \colon
   I \to S^{1} \times I$ by
   $\gamma (t) = \left( e^{2 \pi i t},0 \right) $ defines a loop
   giving the boundary circle $C$.\\
   \linebreak
   Define $F  \colon S \times I \to S$ by
   $F\left( \left( e^{2 i \pi \theta},s \right) , t \right) 
   = \left( e^{2 i \pi \theta}, s (1-t) \right) $. This is a deformation
   retraction of $S$ onto $S^{1} \times \left\{ 0 \right\} \subset S$ (the
   boundary circle, $C$). Thus
   $\pi_1 (S, (1,0)) \cong \pi_1 (C, (1,0)) \cong \mathbb{Z}$. Now, by example
   4 on page 104, $S$ and $C$ have the same homotopy type, so
   by theorem 5.18 and its proof, we have that
   $\left( F(-,1) \right)_*$ is an isomorphism of
   $\pi_1 (S, (1,0))$ and $\pi_1 (C,(1,0))$. Now,
   $\left( F(-,1) \right)
   \left( \langle \gamma \rangle  \right) 
   = \langle F(-,1) \circ \gamma  \rangle 
   = \langle \gamma \rangle $ which is a generator of
   $\pi_1 \left( C, (1,0) \right) $ and thus
   $\langle \gamma \rangle $ is also a generator of
   $\pi_1 (S, (1,0))$ as $\left( F(-,1) \right)_*$ is an isomorphism. Thus,
   with these generators,
   $i_*  \colon \mathbb{Z} \to \mathbb{Z}$
 is given by $i_*(1) = 1$, and hence $i_*(n) = n$.\\
 \linebreak
 \textbf{p. 111:}\\
 \textbf{33:} Which of the following spaces have the fixed-point property?\\
 (a) The 2-sphere\\
 (b) the torus\\
 (c) the interior of the unit disc\\
 (d) the one-point union of two circles.\\
 \linebreak
 \textit{Solution:}\\
 (a) Let $S^{2} = 
 \left\{ (x,y,z)  \mid x^2 + y^2 + z^2 = 1 \right\} $. Define a map
 $f  \colon S^{2} \to S^{2}$ by
 $f(x,y,z) = \left( -x,-y,-z \right) $. This is continuous as each coordinate
 function is continuous, and has no fixed point since if
 $f(x,y,z)=(x,y,z)$ then $-x = x, -y=y$ and $-z = z$ imply $(x,y,z)=(0,0,0)
 \not\in S^{2}$. Thus the 2-sphere does not have the fixed-point property.\\
  \linebreak
  (b) The torus does not have the fixed-point property either. For example,
  consider the map
  $f  \colon S^{1} \times S^{1} \to S^{1} \times S^{1}$ by
  $f(e^{i \theta}, e^{i \alpha}) =
  \left( e^{i (\theta + \frac{\pi}{2})},
  e^{i \alpha} \right) $. This map is continuous as its coordinate maps are
  continuous, and has no fixed point since
  $e^{i \theta}= e^{i \left( \theta + \frac{\pi}{2} \right) 
  =i e^{i \theta}} \implies
  1 = i$ since $\left| e^{i \theta} \right| =1$ and thus dividing is
    possible.\\
    We thus derive a contradiction, so the map has no fixed point.\\
    \linebreak
    (c) The interior of the unit disc does not have the fixed point property:\\
    Consider  $D^2 = \left\{ (x,y) \in \mathbb{R}^2  \mid 
    x^2 + y^2 \le \le 1\right\} $. 
    Define the map 
    $f \colon \left( D^2 \right)^{\circ}
    \to \left( D^2 \right)^{\circ}$ by
    $f(x,y) = \left( \frac{x^2 + y^2 +1}{2},0 \right) $. This is continuous as
    the coordinate functions are contiuous. However,
    $(x,y) = f(x,y) = \left( \frac{x^2 + y^2 +1}{2},0 \right) $ implies
    $y = 0$ so $\frac{x^2 + 1}{2}= x$ which implies $x=1$, however
    $(1,0) \not\in \left( D^2 \right)^{\circ}$.\\
    \linebreak
    (d) Suppose we identify the two copies of $S^{1}$ at $1$. Define a map
    $f  \colon S^{1} \vee S^{1} \to S^{1} \vee S^{1}$ by
    sending $x \mapsto -x$ for $x$ in the first copy of $S^{1}$ (i.e. each
    point is mapped to its antipodal point in the first copy) and
    $x \mapsto -1$ for $x$ in the second copy of $S^{1}$. This map is
    continuous on each of the circles separately, and agree on $1$, so 
    $f$ is continuous and has no fixed point.
    
    
    

    
    
    
    
 





    \newpage
    
    \textbf{Lemma:} If $p  \colon X \to Y$ is an identification map and if
    $Z$ is a locally compact Hausdorff space, then the map
    \[
    \pi = p \times \mathbbm{1}_{Z}  \colon X \times Z
    \to Y \times Z
    \] 
    is a quotient map.\\
    \linebreak
    \textit{Proof:} Firstly, $\pi$ is continuous. Let
    $A \subset Y \times Z$ and
    $\pi^{-1}(A)$ be open and suppose it contains
    $x \times y$ which we will write for  $(x,y)$.
    Now, choose open sets $U_1$ and $V$ with
    $\overline{V}$ compact such that
    $x \times y \in U_1 \times V$ and
    $U_1 \times \overline{V} \subset \pi^{-1}(A)$ (we can find such a $V$ 
    because $Z$ is locally compact).\\
    Now, given 



    
    
   
    

    
    
























\end{document}
