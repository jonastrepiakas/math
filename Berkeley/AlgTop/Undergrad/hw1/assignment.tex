\documentclass[a4paper]{article}

\usepackage[margin=2.5cm]{geometry}
\usepackage[pdftex]{graphicx}
\usepackage[utf8]{inputenc}
\usepackage[T1]{fontenc}
\usepackage{textcomp}
\usepackage{babel}
\usepackage{amsmath, amssymb}
\usepackage[colorlinks=true,linkcolor=blue]{hyperref}
\usepackage{float}
\usepackage{mathrsfs}
%\usepackage{enumitem}
%% for identity function 1:
%\usepackage{bbm}
%%For category theory diagrams:
%\usepackage{tikz-cd}
%%For code (e.g. python) in latex:
%\usepackage{listings}
%
%Usage: 
%\begin{lstlisting}[language=Python]
%\end{lstlisting}

\newcommand{\incfig}[2][1]{%
\def\svgwidth{#1\columnwidth}
\import{./figures/}{#2.pdf_tex}
}


% figure support
\usepackage{import}
\usepackage{xifthen}
\pdfminorversion=7
\usepackage{pdfpages}
\usepackage{transparent}

\pdfsuppresswarningpagegroup=1

\setlength\parindent{0pt}

\newcommand{\qed}{\tag*{$\blacksquare$}}
\newcommand{\qedwhite}{\hfill \ensuremath{\Box}}

%Inequalities
\newcommand{\cycsum}{\sum_{\mathrm{cyc}}}
\newcommand{\symsum}{\sum_{\mathrm{sym}}}
\newcommand{\cycprod}{\prod_{\mathrm{cyc}}}
\newcommand{\symprod}{\prod_{\mathrm{sym}}}

%Linear Algebra

%Redeclaring Span and image
\DeclareMathOperator{\Span}{span}
\DeclareMathOperator{\Ima}{Im}
\DeclareMathOperator{\diag}{diag}
\DeclareMathOperator{\Ker}{Ker}
\DeclareMathOperator{\ob}{ob}


%Row operations
\newcommand{\elem}[1]{% elementary operations
\xrightarrow{\substack{#1}}%
}

\newcommand{\lelem}[1]{% elementary operations (left alignment)
\xrightarrow{\begin{subarray}{l}#1\end{subarray}}%
}

%SS
\DeclareMathOperator{\supp}{supp}
\DeclareMathOperator{\Var}{Var}

%NT
\DeclareMathOperator{\ord}{ord}

%Alg
\DeclareMathOperator{\Rad}{Rad}
\DeclareMathOperator{\Jac}{Jac}

\DeclareMathAlphabet{\pazocal}{OMS}{zplm}{m}{n}
\newcommand{\unif}{\pazocal{U}}



\title{Assignment 1}
\author{Jonas Trepiakas - jtrepiakas@berkeley.edu - Student ID: 3039733855}
\date{}

\begin{document}
\maketitle
\newpage

    \textbf{p. 31}\\
    \textbf{1.:} Verify each of the following for arbitrary subsets $A,B$ of
    a space $X:$ \\
    \linebreak
    (a) $\overline{A \cup B} = \overline{A} \cup \overline{B}$.\\
    (c) $\overline{ \overline{A}} = \overline{A}$.\\
    (d) $\left( A \cup B \right)^{\circ} \supseteq A^{\circ}
    \cup B^{\circ}$.\\
    \linebreak
    \textit{Solution:}\\
    (a) Let  $x$ be a limit point of $A \cup B$. If any neighborhood of 
    $x$ intersects $A - \left\{ x \right\} $, then $x \in \overline{A} \subset 
    \overline{A} \cup \overline{B}$, so assume there exists a neighborhood
    $N_A$
    of $x$ that does not intersect $A - \left\{ x \right\} $. Assume there
    exists a neighborhood $N_B$ of $x$ that does not intersect $B- \left\{
    x \right\} $. Then $x \in N_A \cap N_B$ as $x \in N_A$ and $x \in N_B$ and
    $N_A \cap N_B$ is a neighborhood of $x$ by definition 1.3.(b).\\
    Assume $y \in \left( N_A \cap N_B \right) \cap A\cup B - \left\{ x \right\}
    $. Then $y \in N_A$ so since $N_A \cap A -\left\{ x \right\}
    = \varnothing$, we have $y \not\in A -\left\{ x \right\} $. Similarly,
    since
    $y \in N_B$, we have $y \not\in B - \left\{ x \right\} $ as $N_B \cap
    B - \left\{ x \right\} = \varnothing$.\\
    Hence $y \not\in  A -\left\{ x \right\} \cup B -\left\{ x \right\} 
    = A\cup B - \left\{ x \right\} $ which contradicts
    $y \in \left( N_A \cap N_B \right) \cap A \cup B - \left\{ x \right\} $.
    Thus $(N_A \cap N_B) \cap A \cup B - \left\{ x \right\} = \varnothing$.\\
    But then $N_A \cap N_B$ is a neighborhood of $x$ that does not intersect
    $A\cup B - \left\{ x \right\} $ contradicting $x \in \overline{A \cup B}$.
    Thus no such $N_B$ can exist, so all neighborhood of $x$ intersect
    $B - \left\{ x \right\} $, hence $x \in \overline{B} \subset \overline{A}\cup 
    \overline{B}$.\\
    
    
    
    Conversely, if $x \in \overline{A}\cup \overline{B}$ then
    $x \in \overline{A}$ or $x \in \overline{B}$. Assume without loss of
    generality that $x \in \overline{A}$. Then any neighborhood of $x$ 
    intersects
    $A - \left\{ x \right\} \subset A \cup B - \left\{ x \right\} $, so
    $x \in \overline{A \cup B}$.\\
    \linebreak
    (c) By theorem 2.3, $\overline{A}$ is a closed set. Then by another
    use of theorem 2.3,
    $\overline{\overline{A}}$ is the smallest closed set containing
    $\overline{A}$ which is thus $\overline{A}$ as it is closed, so
    $\overline{\overline{A}} = \overline{A}$.\\
    Alternatively, we can note:
    $x \in \overline{\overline{A}}$ if and only if $x$ is a limit point of 
    $\overline{A}$, and this happens if and only if $x \in \overline{A}$
    since $\overline{A}$ is closed by theorem 2.3 and thus
    contains all its limit points by theorem 2.2.\\
    \linebreak
    (d) Let $x \in A^{\circ} \cup B^{\circ}$, then assume without loss of
    generality that $x \in A^{\circ}$. By definition, $x$ is thus contained
    in the union of all open sets contained in $A$ and hence is contained in
    some
    open set, say $U$, contained in $A$. Then $x \in U \subset A \subset A \cup
    B$, and by definition $U \subset \left( A \cup B \right)^{\circ}$, so
    $x \in \left( A\cup B \right)^{\circ}$.\\
    To show that equality need not hold, consider 
    the set  $X = \left\{ a,b \right\} $ with the indiscrete topology, i.e.
    $\tau = \left\{ \varnothing, X \right\} $. Then let $A = \{a\}, B = \{b\}$.
    We have $\left( A \cup B \right)^{\circ} = X^{\circ} = X$, so 
    $a \in \left( A \cup B \right)^{\circ}$. However, $A^{\circ} = \varnothing$ 
    and
    $B^{\circ} = \varnothing$, so $a \not\in A^{\circ}\cup B^{\circ}$.\\
    \linebreak
    \textbf{3:} Specify the interior, closure, and frontier of each of the
    following subsets of the plane:\\
    \linebreak
    (a) $A = \left\{ (x,y)  \mid 1 < x^2 + y^2 \le 2 \right\} $ \\
    (b) $B = \mathbb{E}^2$ with both axes removed\\
    (c) $C = \mathbb{E}^2 - \left\{ (x, \sin (\frac{1}{x}) )  \mid x > 0 \right\}
    $.\\
    \linebreak
    \textit{Solution:}\\
    \textbf{Lemma 1:} Let $J = \left\{ (a,b)  \mid a,b \in \mathbb{E}
    \right\} $ with the convention that if $b \le a$ then $(a,b) = \varnothing$.
    Then $J \times J$ is a basis for $\mathbb{E}^2$.\\
    \linebreak
    \textit{Proof:} We first show that $J \times J$ determines a basis for a
    topology on $\mathbb{E}^2$ and then show that this topology coincides with
    the standard topology on $\mathbb{E}^2$.\\
    \linebreak
    $J \times J$ is clearly nonempty. Let 
    $(a_1, b_1) \times (c_1, d_1), \ldots, (a_n, b_n) \times (c_n, d_n)$ be
    a finite number of members of $J \times J$. Then
    \[
        (a_1, b_1) \times (c_1, d_1) \cap \ldots \cap 
        (a_n, b_n) \times (c_n, d_n)
        = \left( \max \left\{ a_i \right\} , \min \left\{ b_i \right\}  \right) 
        \times  \left( \max \left\{ c_i \right\} , \min \left\{ d_i \right\}  \right) 
        \in J
    \] 
    where we again remember the if say $\min \left\{ b_i \right\} \le 
    \max \left\{ a_i \right\} $ then the intersection becomes the empty set;
    similarly, for $c_i$ and $d_j$.\\
    Furthermore, for any $(x,y) \in \mathbb{E}^2$, we have
    $(x,y) \in (x-1, x+1) \times (y-1,y+1) \in J \times J$, so by
    theorem 2.5, we get that $J \times J$ determines a topology on
    $\mathbb{E}^2$.\\
    \linebreak
    
    



    Now let $U$ be any open set in the standard topology on $\mathbb{E}^2$ and let $x \in U$.\\
    Then by definition p. 28, we can find $\varepsilon_x > 0$ such that
    $B(x, \varepsilon_x) \subset U$. Then writing $x = (x_1, x_2)$, we have
    $ V_x= \left( x_1- \frac{\varepsilon_x}{\sqrt{2} }, x_1
        + \frac{\varepsilon_x}{\sqrt{2} }
    \right) \times 
    \left( x_2 - \frac{\varepsilon_x}{\sqrt{2} }, x_2
    + \frac{\varepsilon_x}{\sqrt{2} } \right)
    \subset  B(x, \varepsilon_x)$ since for any
    $(a,b) \in V$, we have
    $\|(x_1, x_2) - (a,b)\| < \sqrt{\frac{\varepsilon_x^2}{2}
    + \frac{\varepsilon_x^2}{2}} = \varepsilon_x$. Hence
    $x \in V_x \subset B\left( x, \varepsilon_x \right) \subset U$. As $x$ was
    arbitrary, we can find such a $V_x$ for any $x \in U$. Thus
    $U = \bigcup_{x \in U} \{x\} \subset \bigcup_{x \in U} V_x \subset U$, so
    $U = \bigcup_{x \in U} V_x$. 
    Denoting the standard topology on $\mathbb{E}^2$ by $\tau$ and
    the topology induced by the basis $J \times J$ by $\tau_J$, we thus have
    $\tau \subset \tau_J$.\\
    \linebreak
    Conversely, for any open set $U \in \tau_J$, we thus have
    $U = \bigcup_{i \in  I} \left( a_i, b_i \right) \times (c_i,d_i)$. 
    Let $x = (x_1, x_2) \in U$. Then there exists
    $i_x \in I$ such that $x \in (a_{i_x}, b_{i_x}) \times (c_{i_x}, d_{i_x})$.\\
    Then for $\varepsilon_x = \min \left\{ 
    \left| b_{i_x} - x_1 \right| , \left| a_{i_x}- x_1 \right| ,
\left| d_{i_x} - x_2 \right| , \left| c_{i_x}- x_2 \right| \right\} $, we have
    $B(x, \varepsilon_x) \subset 
    \left( a_{i_x}, b_{i_x} \right) \times \left( c_{i_x}, d_{i_x} \right) 
    \subset U$. Since $x$ was arbitrary, we can write
    $U = \bigcup_{x \in U} \{x\} 
    \subset \bigcup_{x \in U} B\left( x, \varepsilon_x \right) \subset U$, so
    $U = \bigcup_{x \in X} B\left( x , \varepsilon_x \right) $. Thus
    $\tau_J \subset \tau$. Thus $J \times J$ is a basis for the standard
    topology on $\mathbb{E}^2$.\\
    \linebreak

    \textbf{Lemma 2:} For any $A \subset X$, we have 
    $\partial A = \overline{A} - A^{\circ}$, where $\partial A$ is the frontier
    of $A$ : $\partial A = \overline{A} \cap \overline{X -A}$. \\
    \linebreak
    \textit{Proof:} We have $x \in \overline{A}$ if and only if
    any neighborhood of  $x$ intersects $A - \{x\}$ if and only if
    for any neighborhood $N$ of $x$, either $x \in N \subset A$ or
    $N \cap A - \{x\} \neq \varnothing \neq N \cap (X-A - \{x\})$.
    Now the last statement gives that if there exists $N$ such that $x \in
    N \subset A$, then by definition of a neighborhood, there exists an open
    set $U$ such that $x \in U \subset N \subset A$ and hence $x \in U \subset
    A^{\circ}$; if no such $N$ exists, then for all neighborhoods $N$, we have
    $N \cap A-\{x\} \neq \varnothing \neq N \cap (X-A - \{x\})$ which is
    equivalent to $x \in \overline{A} \cap \overline{X-A} = \partial A$. Hence
    $x \in \overline{A}$ if and only if $x \in A^{\circ} \cup \partial A$ and
    furthermore, these two cases are disjoint: if $x \in A^{\circ}$ then
    $x \not\in \overline{X-A} \supset \partial A$. And conversely by
    contraposition: $x \in \partial A$ implies $x \not\in A^{\circ}$.\\
    Therefore $\partial A = \overline{A} - A^{\circ}$.\\
    \linebreak
    
    
    

    (a) We claim that $A^{\circ} = \left\{ (x,y)  \mid 1 < x^2 + y^2
    < 2 \right\} $. Let $\tilde{A}$ denote the right hand side.\\
    For $(x,y) \in \tilde{A}$, we have that for
    $\varepsilon = \min \left\{ \sqrt{2} - \|(x,y) \|, \|(x,y) \|-1 \right\} $, 
    $B((x,y), \varepsilon) \subset \tilde{A} \subset A$. Hence
    $\tilde{A}$ is open, so $\tilde{A} \subset A^{\circ}$.\\
    For $(x,y) \in A$ with $\|(x,y)\|^2 = 2$, we have that if $(x,y) \in
    A^{\circ}$ then there exists an open neighborhood $U$ of $(x,y)$ such that
    $(x,y) \in U \subset A$. However, since $J \times J$ is a basis for
    $\mathbb{E}^2$ there must thus exist $(a,b) \times (c,d)$ such that
    $(x,y) \in (a,b) \times (c,d) \subset U \subset A$, but then
    for $x' = \begin{cases}
        \frac{x+b}{2}, & x \ge 0\\
        \frac{x+a}{2}, & x<0
    \end{cases}$ and $y' = \begin{cases}
        \frac{y + d}{2}, & y \ge 0\\
        \frac{y-c}{2},& y<0
    \end{cases}$, we have
    $\| (x',y')\| > 2$ but $(x',y') \in (a,b) \times (c,d) \subset U \subset
    A$, contradiction. Hence $(x,y) \not\in A^{\circ}$, so
    $A^{\circ} \subset \tilde{A}$. Thus $A^{\circ} = \tilde{A}$.\\
    \linebreak
    We claim the closure of $A$ is $\left\{ (x,y)  \mid 1 \le x^2 + y^2 \le
    2 \right\} $.
    Let again $\tilde{A}$ denote the right hand side. By theorem 2.3, the
    closure of $A$ is the smallest closed set containing $A$, hence
    $A \subset \overline{A}$. It remains to show that
    all $(x,y)$ with $\|(x,y)\| = 1$ are limit points of $A$ and that
    all points $(x,y)$ with $\|(x,y)\|<1$ or $>2$ are not limit points of
    $A$.\\
    \linebreak
    Let $(x,y)$ be such that $\|(x,y)\| = 1$ and let $U$ be any open
    neighborhood of $(x,y)$. Then we can find a $\varepsilon > 0$ such that
    $B((x,y), \varepsilon) \subset U$, and in this ball we can find some
    $(x',y')$ with $1 < \|(x',y')\| \le 2$.\\
    To be specific, there exists a basis element
    $(a,b) \times (c,d) \in J\times J$ with
    $(x,y) \in (a,b) \times (c,d) \subset U$, and again letting say
    $x' = \begin{cases}
        \frac{x + \min\{\frac{1}{4}, b \}}{2} , & 0 \le x\\
        \frac{x + \max\{-\frac{1}{4},a\}}{2}  , & 0> x
    \end{cases}$ and
    $y' = \begin{cases}
        \frac{y + \min\left\{ \frac{1}{4},d \right\} }{2}, & 0 \le y\\
        \frac{y + \max\left\{ -\frac{1}{4},c \right\} }{2}, & 0> y
    \end{cases}$ or anything that is simply "close" to $(x,y)$ in $(a,b) \times
    (c,d)$ while having greater norm which does not exceed $\sqrt{2} $- which is possible to find as $(a,b)
    \times (c,d)$ is open -,
    we get $(x',y') \in (a,b) \times (c,d) \subset U$ while $1 < \|(x',y')\|
    \le \sqrt{2} $ and hence $U$ intersects $A$. Thus
    $\tilde{A} \subset \overline{A}$.\\
    \linebreak
    Now, for points $(x,y)$ with $\|(x,y)\| < 1$, we have
    $(x,y) \in B\left( (0,0), 1 \right) $ which is open
    and thus a neighborhood of $(x,y)$ that is disjoint from $A$. Thus
    $\overline{A} \subset \mathbb{E}^2 -  B\left( (0,0), 1 \right)$. 
    For points $(x,y)$ with $\|(x,y)\| > \sqrt{2} $, we have that
    $(x,y) \in \mathbb{E}^2 - \overline{B\left( (0,0), \sqrt{2}  \right) }$ 
    which is open as its compliment is $\overline{B \left( (0,0), \sqrt{2}
    \right) }$ which is closed. But $A \subset \overline{B \left( (0,0),
    \sqrt{2}  \right) }$, and thus $\mathbb{E}^2 - \overline{B \left(
(0,0)\sqrt{2}  \right) }$ is an open neighborhood that is disjoint from $A$, so
$(x,y)$ is not a limit point of $A$. Thus
$\overline{A} \subset \overline{B \left( (0,0), \sqrt{2}  \right) } -
B \left( (0,0),1 \right) = \tilde{A} \subset A$, hence $\overline{A}
= \tilde{A}$.\\
\linebreak
Now by lemma 2, $\partial A = \overline{A} - A^{\circ}
= \left\{ (x,y)  \mid x^2 + y^2 \in \left\{ 1,2 \right\}  \right\} $.\\
\linebreak





(b) We claim the interior is $B^{\circ} = B$. For any point $(x,y)$, we have
that $(x,y) \in B\left( (x,y), \frac{\min \left\{ |x|, |y| \right\} }{2} \right) 
\subset B$, thus $B$ is open, so $B^{\circ} = B$.\\
\linebreak
The closure of $B$ is all of $\mathbb{E}^2$ : the closure contains $B$ by
theorem 2.3. Now let $(0,x) \in \mathbb{E}^2$ for any $x \in \mathbb{E}$. Then
for any neighborhood $N$ of $(0,x)$, we can find a basis element $(a,b) \times
(c,d)$ such that $(0,x) \in (a,b) \times (c,d) \subset N$, so
letting $y' = x$ if $x \neq 0$ and otherwise $y' = \frac{d}{2}$, and letting
 $x' = \frac{b}{2}$, we get
 $(x',y') \in (a,b) \times (c,d) \subset N$ and $(x',y') \in B$, hence
 any neighborhood of $(0,x)$ intersects $B$ so
 $\{0\} \times \mathbb{E} \subset \overline{B}$.\\
 Equivalently, let $(x,0) \in \mathbb{E}^2$ for any $x \in \mathbb{E}$. Then
 for any neighborhood $N$ of $(x,0)$, we can find a basis element $(a,b) \times
 (c,d)$ such that $(x,0) \in (a,b) \times (c,d) \subset N$, so letting
 $x' = x$ if $x\neq 0$ and $x' = \frac{b}{2}$ otherwise, and $y'
 = \frac{d}{2}$, we get
 $(x',y') \in (a,b) \times (c,d) \subset N$ and $(x',y') \in B$, so
 any neighborhood of  $(x,0)$ intersects $B$ hence
 $\mathbb{E} \times \{0\} \subset \overline{B}$, thus
 $\mathbb{E}^2 = B \cup \{0\} \times \mathbb{E} \cup \mathbb{E} \times \{0\}
 \subset \overline{B} \subset \mathbb{E}^2$ and hence $\overline{B}
 = \mathbb{E}^2$.\\
 \linebreak
 By lemma 2, we now get $\partial B = \overline{B}- B^{\circ}
 = \mathbb{E} \times \{0\} \cup \{0\} \times \mathbb{E}$.\\
 \linebreak
 (c) We claim that the interior of $C$ is 
  $C - \left\{ (x,y)  \mid x=0, |y|\le 1 \right\} $.\\
  Let $f  \colon (0,\infty) \to \mathbb{E}$ be given by
  $f(x) = \sin (\frac{1}{x})$ . This is continuous as the composition of
  continuous functions. Let  $(x,y) \in C$ with $x >0$. We can by continuity
  find $\delta$ such that $B(x,\delta) \subset  f^{-1}\left( B(f(x), \varepsilon) \right) 
  $ for any $\varepsilon > 0$. Let $\varepsilon = \frac{\left| y - f(x) \right|
  }{2}$ (where $|y- f(x)| > 0$ since $(x,y) \not\in \mathbb{E}^2 - C$ ). Then $B \left( (x,y), \min\{\varepsilon, \delta\} \right) \subset C$ 
  since
  if $(x',y') \in B\left( (x,y), \min\{\varepsilon, \delta\} \right) \cap 
  \mathbb{E}^2 - C$, then $|x - x'| \le d\left( (x,y), (x',y') \right) 
  < \min\{\delta, \varepsilon\} \le  \delta$, and hence
  \[
  d \left( (x,y), (x',y') \right)  \ge |y-y'| = |y- f(x')| 
  \ge |y - f(x)| - \left| f(x) - f(x') \right| 
  = \varepsilon,
  \] 
  which is a contradiction.\\
  \linebreak
  For $x = 0$ and $|y| >1$, we can naturally choose
  $\varepsilon = |y|-1$ and get $B\left( (x,y), \varepsilon \right) \subset C$
  since any point $(x', f(x')) \in \mathbb{E}^2 - C$ has $|f(x')| \le 1$.\\
  \linebreak
  For $x < 0$, we can choose $\varepsilon = |x|$. Then
  $B\left( (x,y), \varepsilon \right) \subset C$ since any
  $(x', f(x')) \in \mathbb{E}^2 - C$ has $x' > 0$.\\
  It remains to show that the points $\left\{ (0,y)  \mid |y|\le 1 \right\}
  $ are limit points of $\mathbb{E}^2 - C$.\\
  Take any neighborhood N of a fixed point $(0,y)$ with $|y| \le 1$ and
  take a basis element $(a,b) \times (c,d) \subset N$ containing $(0,y)$. Then
  since $\sin$ is periodic with period $2\pi$, we can find $N \in \mathbb{R}_+$ such that
  $\frac{1}{N} < b$ and $\sin \left( \frac{1}{N} \right) = y$. But then
  $\left( \frac{1}{N}, y \right) \in (a,b) \times (c,d) \cap \mathbb{E}^2 - C$,
  and thus any neighborhood of $(0,y)$ intersects $\mathbb{E}^2 - C$, so
  we get $C^{\circ} = C - \left\{ (x,y)  \mid x = 0, |y| \le 1 \right\}$.\\
  \linebreak

  \textbf{Lemma 3:} If $A \subset X$ then
  $\overline{X-A} = X - A^{\circ}$.\\
  \textit{Proof:} $x \in \overline{X-A}$ if and only if any
  neighborhood of $x$ intersects $X-A$ if and only if there does not
  exist a neighborhood $N$ of $x$ such that $x \in N \subset A$ if
  and only if $x \in X- A^{\circ}$.\\
  \linebreak
  Thus $\overline{C} = \mathbb{E}^2 - 
  \left\{ (x, \sin (\frac{1}{x})  \mid x>0 \right\}^{\circ}$.
  For any point $(x',y') \in \left\{ (x, \sin (\frac{1}{x}) \mid 
  x > 0\right\} $, take a neighborhood $N$ of it and let
  $(a,b) \times (c,d)$ be a basis element with
  $(x',y') \in (a,b) \times (c,d) \subset N$. Then
  choose any $y \in (c,d) - \left\{ y' \right\} $. Now
  $(x', y) \in (a,b) \times (c,d) \subset N$, but
  $(x', y) \not\in \left\{ \left( x, \sin(\frac{1}{x}) \right)  \mid x >0
  \right\} $, so $(x',y') \not\in 
  \left\{ \left( x, \sin(\frac{1}{x}) \right) \mid x > 0  \right\}^{\circ} $,
  hence $\left\{ \left( x, \sin(\frac{1}{x}) \right)  \mid 
  x > 0 \right\}^{\circ} = \varnothing$, so
  $\overline{C} = \mathbb{E}^2$.\\
  \linebreak
  Now by lemma 2, 
  $$\partial C = \overline{C} - C^{\circ}
  = \left( \mathbb{E}^2 - C \right) \cup 
  \left\{ \left( x,y \right)  \mid x=0, |y|\le 1 \right\} 
  = \left\{ \left( x, \sin(\frac{1}{x}) \right)  \mid x>0 \right\} 
  \cup \left\{ (x,y)  \mid x=0, |y|\le 1 \right\} $$.\\
  \linebreak
  \textbf{12:} Show that if $X$ has a countable base for its topology, then
  $X$ contains a countable dense subset.\\
  \linebreak
  \textit{Solution:} Let $\left\{ \beta_i \right\}_{i \in I}$ be a basis
  for the topology on $X$ with $I$ countable. Now for each $i \in I$, choose
  an $x_i \in \beta_i$. We claim that the set
  $\left\{ x_i \right\}_{i \in I}$ is a countable dense subset of $X$.\\
  \linebreak
  Assume $x \not\in \overline{\left\{ x_i \right\}_{i \in I}}$. Then there
  exists
  a neighborhood $N$ of $x$ such that $N$ does not contain any $x_i$ for $i \in
  I$. By definition of a neighborhood, there exists an open set $U$ such that
  $x \in U \subset N$. Now since $U$ is open and $\left\{ \beta_i \right\}_{i
  \in I}$ is a basis for the topology on $X$, we can write
  $U = \bigcup_{j \in J} \beta_j$ for some $J \subset I$. But then
  for all $j \in J$, $x_j \in U \subset N$, so $N$ contains a point
  $x_i$ with $i \in I$, contradiction. Thus no such $N$ exists, so $x \in 
  \overline{\left\{ x_i \right\}_{i \in I}}$ and as $x$ was arbitrary, we get
  $X = \overline{\left\{ x_i \right\}_{i \in I}}$.














\end{document}
