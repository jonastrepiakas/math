\documentclass[a4paper]{article}

\usepackage[margin=2.5cm]{geometry}
\usepackage[pdftex]{graphicx}
\usepackage[utf8]{inputenc}
\usepackage[T1]{fontenc}
\usepackage{textcomp}
\usepackage{babel}
\usepackage{amsmath, amssymb}
\usepackage[colorlinks=true,linkcolor=blue]{hyperref}
\usepackage{float}
\usepackage{mathrsfs}
%\usepackage{enumitem}
%% for identity function 1:
\usepackage{bbm}
%%For category theory diagrams:
%\usepackage{tikz-cd}
%%For code (e.g. python) in latex:
%\usepackage{listings}
%
%Usage: 
%\begin{lstlisting}[language=Python]
%\end{lstlisting}

\newcommand{\incfig}[2][1]{%
\def\svgwidth{#1\columnwidth}
\import{./figures/}{#2.pdf_tex}
}


% figure support
\usepackage{import}
\usepackage{xifthen}
\pdfminorversion=7
\usepackage{pdfpages}
\usepackage{transparent}

\pdfsuppresswarningpagegroup=1

\setlength\parindent{0pt}

\newcommand{\qed}{\tag*{$\blacksquare$}}
\newcommand{\qedwhite}{\hfill \ensuremath{\Box}}

%Inequalities
\newcommand{\cycsum}{\sum_{\mathrm{cyc}}}
\newcommand{\symsum}{\sum_{\mathrm{sym}}}
\newcommand{\cycprod}{\prod_{\mathrm{cyc}}}
\newcommand{\symprod}{\prod_{\mathrm{sym}}}

%Linear Algebra

%Redeclaring Span and image
\DeclareMathOperator{\Span}{span}
\DeclareMathOperator{\Ima}{Im}
\DeclareMathOperator{\diag}{diag}
\DeclareMathOperator{\Ker}{Ker}
\DeclareMathOperator{\ob}{ob}


%Row operations
\newcommand{\elem}[1]{% elementary operations
\xrightarrow{\substack{#1}}%
}

\newcommand{\lelem}[1]{% elementary operations (left alignment)
\xrightarrow{\begin{subarray}{l}#1\end{subarray}}%
}

%SS
\DeclareMathOperator{\supp}{supp}
\DeclareMathOperator{\Var}{Var}

%NT
\DeclareMathOperator{\ord}{ord}

%Alg
\DeclareMathOperator{\Rad}{Rad}
\DeclareMathOperator{\Jac}{Jac}

\DeclareMathAlphabet{\pazocal}{OMS}{zplm}{m}{n}
\newcommand{\unif}{\pazocal{U}}

\title{Assignment 2}
\author{Jonas Trepiakas - jtrepiakas@berkeley.edu}
\date{}

\begin{document}
\maketitle
\newpage
    \textbf{p. 35}\\
    \textbf{17:} Let $\mathbb{R}_{fc}$ denote the set of all real numbers with the
    finite-complement topology, and define $f \colon \mathbb{R} \to
    \mathbb{R}_{fc}$ by
    $f(x) = x$. Show that $f$ is continuous, but is not a homeomorphism.\\
    \linebreak
    \textit{Solution:} Let $U$ be any open set in $\mathbb{R}_{fc}$. Then
    $f^{-1}(U) = U$ by definition of $f$, so the first part of the problem can be reformulated
    as: show that the finite-complement topology is coarser than the standard
    topology on $\mathbb{R}$.\\
    \linebreak
    Now, let $U$ be open in $\mathbb{R}_{fc}$ and let $x \in U$ be any point.\\
    By definition, the complement of $U$ in $\mathbb{R}$ is a finite set, say
    $\left\{ x_1, \ldots, x_n \right\} $ and $x \neq x_i$ for all $i
    = 1,\ldots, n$. Let $\delta_i = \left| x - x_i \right| >0$. Then
    $\delta = \min_{i} \left\{ \delta_i \right\} >0$, and
    we claim that $B(x, \delta) \subset U$ where the open ball is taken in the
    standard metric on $\mathbb{R}$.\\
    \linebreak
    Let $y \in B\left( x, \delta \right) $. Then
    \[
        \left| y- x_i \right| \ge \underbrace{\left| x - x_i \right|}_{
        \ge \delta} - \underbrace{\left| y - x \right|}_{< \delta}
        > 0,
    \] 
    so $ y \in U$. Hence $B\left( x, \delta \right) \subset U$. 
    Since $x$ was arbitrary, we can find such a $\delta_x = \delta$ for any
    $x$, so we can write
    $U \subset \bigcup_{x \in U} \left\{ x \right\} \subset 
    \bigcup_{x \in U} B\left( x, \delta_x \right) 
    \subset U$, and thus $U = \bigcup_{x \in U} B\left( x, \delta_x \right) $,
    so
    $U$ is open in the standard topology on $\mathbb{R}$. Therefore
    $f$ is continuous.\\
    \linebreak
    To see that this is not a homomorphism, note that
    $(0, \infty)$ is open in the standard topology as the union
    $\bigcup_{x \in (0,\infty)} B\left( x, |x| \right) $, however,
    $(0, \infty)^{c} = \mathbb{R} - (0, \infty) = (-\infty, 0]$ is not finite,
    so
    $(0, \infty)$ is not open in $\mathbb{R}_{fc}$, and since
    $f\left( (0, \infty) \right) = \left( 0, \infty \right) $, we find that
    $f$ does not map open sets of $\mathbb{R}$ in the standard topology to open
    sets in $\mathbb{R}_{fc}$, hence $f^{-1}$  is not continuous, so $f$ is not
    a homeomorphism.\\
    \linebreak
    \textbf{21:} 
    Define the maps
    $f  \colon I^{n} = \left[ -1,1 \right]^{n} \to D^{n} = \left\{ 
    x  \mid \|x\|\le 1 \right\} $ and
    $g  \colon D^{n} \to I^{n}$, with the euclidean norm, by
    \[
    f\left( x \right) 
    = \begin{cases}
        \frac{\max_{i}|x_i|}{\|x\|} x,& x \neq 0\\
        0,& x=0.
    \end{cases}
\]
and
\[
g(x) = 
\begin{cases}
    \frac{\|x\|}{\max_i |x_i|} x,& x\neq 0\\
    0, & x=0.
\end{cases}
\] 

    $f$ is continuous on $\left[ -1,1 \right]^{n} - \left\{ 0 \right\}
    $ since
    it is the composition of continuous functions there - where
    $\max_i |x_i| $ is the maximal norm, and
    $g$ is continuous on $D^{n} - \left\{ 0 \right\} $ since it is the
    composition
    of continuous functions there as well.\\
    \linebreak
    For $x=0$, let $\varepsilon > 0$. Then let
    $\delta = \varepsilon$. For $\|x\| < \delta$, we have
    \[
    \left| \max_i |x_i|  \right|^2
    \le \sum_{i=1}^{n} x_i^2
    \] 
    so $\left| \max_i |x_i| \right| \le 
    \|x\|$, and thus
    
    \[
        \|f(y_1, \ldots, y_n)\| = \left| \max_i |x_i|  \right| 
        \le \|x\| < \delta = \varepsilon.   
    \] 
    Thus $f$ is continuous at $0$ too.\\
    For given $\varepsilon > 0$, we can for $g$ choose
    $\delta = \frac{\varepsilon}{\sqrt{n} }$, where we then get for
    $0 < \|x\| < \delta$ that since
    $\sum_{i=1}^{n} x_i^2 \le n \max_i \{x_i^2\}$, we have
    \[
    \|g\left( x_1, \ldots, x_n \right) \|
    = \frac{1}{\max_i |x_i|  }
     \sum_{i=1}^{n} x_i^2
     = \sqrt{\frac{\sum_{i=1}^{n} x_i^2}{\max_i x_i^2}} 
     \sqrt{\sum_{i=1}^{n} x_i^2} 
     < \sqrt{n} \frac{\varepsilon}{\sqrt{n} } = \varepsilon.
    \] 
    Thus $g$ is continuous at $0$ as well.\\
    Since $f\circ g = \mathbbm{1}_{D^{n}}$ and
    $g\circ f = \mathbbm{1}_{I^{n}}$, we have that
    $f$ is a homeomorphism.\\
    \linebreak
    \textbf{p. 41:}\\
    \textbf{28:} If $A,B$ are disjoint closed subsets of a metric space, find
    disjoint open sets $U,V$ such that $A \subset U$ and $B \subset V$.\\
    \linebreak
    \textit{Solution:} By lemma 2.14, we have that since $A,B$ are disjoint
    closed subsets of a metric space $X$, there exists a 
    continuous real-valued function on $X$ which takes value $1$ on $A$ and
    $-1$ on $B$ and values strictly in $(-1,1)$ on $X - \left( A\cup B \right)
    $.\\
    Let $f$ denote this function. Then
    $A \subset f^{-1}\left( (-\infty, 0) \right) = U $ and
    $B \subset f^{-1}\left( (0, \infty) \right) = V $. Furthermore, since
    $f$ is continuous and $(-\infty, 0)$ and $(0, \infty)$ are
    open, we have that
    $U = f^{-1}\left( (- \infty, 0) \right) $ and
    $V= f^{-1}\left( (0, \infty) \right) $ are open. Assume
    $y \in U \cap V $. Then
    $f(y) \in \left( -\infty, 0 \right) \cap (0, \infty) = \varnothing$, so
    $U \cap V = \varnothing$. So $U,V$ are open disjoint sets such that
    $A \subset U$ and $B \subset V$.

    

    
    
    













    
    
    








































\end{document}
