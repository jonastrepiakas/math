\documentclass[a4paper]{article}

\usepackage[margin=2.5cm]{geometry}
\usepackage[pdftex]{graphicx}
\usepackage[utf8]{inputenc}
\usepackage[T1]{fontenc}
\usepackage{textcomp}
\usepackage{babel}
\usepackage{amsmath, amssymb}
\usepackage[colorlinks=true,linkcolor=blue]{hyperref}
\usepackage{float}
\usepackage{mathrsfs}
%\usepackage{enumitem}
%% for identity function 1:
\usepackage{bbm}
%%For category theory diagrams:
%\usepackage{tikz-cd}
%%For code (e.g. python) in latex:
%\usepackage{listings}
%
%Usage: 
%\begin{lstlisting}[language=Python]
%\end{lstlisting}

\newcommand{\incfig}[2][1]{%
\def\svgwidth{#1\columnwidth}
\import{./figures/}{#2.pdf_tex}
}


% figure support
\usepackage{import}
\usepackage{xifthen}
\pdfminorversion=7
\usepackage{pdfpages}
\usepackage{transparent}

\pdfsuppresswarningpagegroup=1

\setlength\parindent{0pt}

\newcommand{\qed}{\tag*{$\blacksquare$}}
\newcommand{\qedwhite}{\hfill \ensuremath{\Box}}

%Inequalities
\newcommand{\cycsum}{\sum_{\mathrm{cyc}}}
\newcommand{\symsum}{\sum_{\mathrm{sym}}}
\newcommand{\cycprod}{\prod_{\mathrm{cyc}}}
\newcommand{\symprod}{\prod_{\mathrm{sym}}}

%Linear Algebra

\DeclareMathOperator{\Span}{span}
\DeclareMathOperator{\Ima}{Im}
\DeclareMathOperator{\diag}{diag}
\DeclareMathOperator{\Ker}{Ker}
\DeclareMathOperator{\ob}{ob}
\DeclareMathOperator{\Hom}{Hom}
\DeclareMathOperator{\sk}{sk}
\DeclareMathOperator{\Vect}{Vect}
\DeclareMathOperator{\Set}{Set}
\DeclareMathOperator{\Group}{Group}
\DeclareMathOperator{\Ring}{Ring}
\DeclareMathOperator{\Ab}{Ab}
\DeclareMathOperator{\Top}{Top}
\DeclareMathOperator{\Htpy}{Htpy}
\DeclareMathOperator{\Cat}{Cat}
\DeclareMathOperator{\CAT}{CAT}


%Row operations
\newcommand{\elem}[1]{% elementary operations
\xrightarrow{\substack{#1}}%
}

\newcommand{\lelem}[1]{% elementary operations (left alignment)
\xrightarrow{\begin{subarray}{l}#1\end{subarray}}%
}

%SS
\DeclareMathOperator{\supp}{supp}
\DeclareMathOperator{\Var}{Var}

%NT
\DeclareMathOperator{\ord}{ord}

%Alg
\DeclareMathOperator{\Rad}{Rad}
\DeclareMathOperator{\Jac}{Jac}

\DeclareMathAlphabet{\pazocal}{OMS}{zplm}{m}{n}
\newcommand{\unif}{\pazocal{U}}

\begin{document}
    \textbf{p. 95:}\\
    \textbf{10:} Let $\gamma, \sigma$ be two paths in the space $X$ which begin
    at the point $p$ and end at $q$. As in the proof of theorem
    (5.6), these paths induces isomorphisms $\gamma_*, \sigma_*$ of
    $\pi_1 (X, p)$ with $\pi_1(X,q)$. Show that
    $\sigma_*$ is the composition of $\gamma_*$ and the inner automorphism
    of $\pi_1 (X,q)$ induced by the element $\langle \sigma^{-1}\gamma
    \rangle$.\\
    \linebreak
    \textit{Solution:}
    The isomorphisms induces by $\gamma$ and $\sigma$ are given by
    $\gamma_* \left( \langle \alpha \rangle  \right) 
    = \langle \gamma^{-1} \alpha \gamma \rangle $ and
    $\sigma_* \left( \langle \alpha \rangle  \right) 
    = \langle \sigma^{-1} \alpha \sigma \rangle $, respectively.\\
    The inner automorphism of $\pi_1 (X,q)$ induced by
    $\langle \sigma^{-1} \gamma \rangle$ is
    given by
    $\langle \alpha \rangle \to 
    \langle \sigma^{-1} \gamma \rangle \langle \alpha \rangle
    \langle \sigma^{-1} \gamma\rangle^{-1}
    = \langle \sigma^{-1} \gamma \rangle 
    \langle \alpha \rangle \langle \gamma^{-1} \sigma \rangle 
    \langle \sigma^{-1} \gamma \alpha \gamma^{-1} \sigma \rangle 
    = \sigma_* \gamma^{-1}_* \langle \alpha \rangle $, so denoting this
    inner
    automorphism by $\beta$, we get
    $\langle \alpha \rangle \stackrel{\gamma_*}{\to }
    \langle \gamma^{-1}\alpha \gamma \rangle 
    \stackrel{\beta}{\to }
    \sigma_* \gamma^{-1}_* \langle \gamma^{-1} \alpha \gamma \rangle 
    = \sigma_* \langle \gamma \gamma^{-1} \alpha \gamma \gamma^{-1} \rangle 
    = \sigma_* \langle \alpha \rangle $, hence
    $\sigma_* = \beta \circ \gamma_*$, so
    $\sigma_*$ is the composition of
    $\gamma_*$ and the inner automorphism of
    $\pi_1 \left( X, q \right) $ induces by
    $\langle \sigma^{-1} \gamma \rangle $.\\
    \linebreak
    
    \textbf{p. 102:}\\
    \textbf{18:} Let $\pi  \colon X \to Y$ be a covering map. So each point $y
    \in Y$ has a neighborhood $V$ for which $\pi^{-1}(V)$ breaks up as a union
    of disjoint open sets, each of which maps homeomorphically onto $V$ under
    $\pi$. Call such a neighborhood 'canonical'. If $\alpha$ is a path in $Y$,
    show how to find points $0 = t_0 < t_1 < \ldots < t_m = 1$ such that
    $\alpha \left( \left[ t_i, t_{i+1} \right]  \right) $ lies in a canonical
    neighborhood for $0 \le i \le m-1$. Hence lift $\alpha$ piece by piece to a
    (unique) path in $X$ which begins at any preassigned point of
    $\pi^{-1}\left( \alpha(0) \right) $.\\
    \linebreak
    \textit{Solution:} For each $y \in Y$, let
    $V_y$ denote a canonical open neighborhood for $y$ under
    $\pi$ (we can assume $V_y$ is open since $\pi$ restricting to
    a homeomorphism also gives a homeomorphic correspondence of open sets).
    By assumption, $\bigcup_{y \in Y} V_y$ covers $Y$, so
    since $\alpha$ maps $I \to Y$, we have
    $\bigcup_{y \in Y} \alpha^{-1}(V_y)$ is an open covering of $I $, so as $I$ is a compact
    metric space, there exists a Lebesgue number $ \delta > 0$
 for the covering. Subdivide $I = \left[ 0,1 \right] $ into intervals
 $0 = t_0 < t_1 \ldots < t_m = 1$ such that
 $\left| t_{i+1}-t_i \right| < \delta$ for all $i$ (for example we can choose
 some $N \in \mathbb{N}$ such that
 $\frac{1}{N} < \delta$ and let
 $t_i = \frac{i}{N}$ for $i = 0, \ldots, N$ ).\\
 \linebreak
 Now, by construction
any $\left[ t_i, t_{i+1} \right] $ lies in
some $\alpha^{-1}(V_y)$, so
$\alpha \left( \left[ t_i, t_{i+1} \right]  \right) 
\subset V_y$ which is a canonical neighborhood.\\
\linebreak
Now, let $x_0$ be an arbitrary point in $\pi^{-1}\left( \alpha (0) \right) \subset 
X$, and suppose $\alpha(\left[ 0, t_1 \right] ) \in V_{y_1}$. Then $x_0$ lies
in some $U \subset \pi^{-1} (V_{y_1})$ such that
$U$ is mapped homeomorphically by $\pi$ onto $V_{y_1}$. Let
$f$ be the inverse of $\pi|_{U}  \colon U \to V_{y_1}$.


Define
$\tilde{\alpha} (t) = f \circ \alpha(t)$ for $t \in \left[ 0, t_1 \right] $.
Then for $t \in \left[ 0, t_1 \right] $, we have
$\pi \circ \tilde{\alpha}(t) = 
\pi \circ f \circ \alpha(t) =
\alpha(t)$, so
$\tilde{\alpha}$ is a lift on
$\left[ 0, t_1 \right] $.\\
\linebreak


Suppose we have defined $\tilde{\alpha}$ on $\left[ 0,t_N \right] $ for
$1 \le N < m$. Thus
$\alpha|_{\left[ 0, t_N \right] }
= \pi \circ \tilde{\alpha}$ on $\left[ 0,t_N \right] $.
We want to define $\tilde{\alpha}$ on
$\left[ t_{N}, t_{N+1} \right] $. Suppose
$\alpha \left( \left[ t_{N}, t_{N+1} \right]  \right) 
\subset  V_y$. Then 
$\tilde{\alpha}(t_N) \in \pi^{-1}(V_y)$, so let
$\tilde{\alpha}(t_N) \in U \subset \pi^{-1}(V_y)$ such that $\pi$ maps $U$
homeomorphically onto $V_y$. Let $g$ be the inverse of
$\pi|_{U}  \colon U \to V_y$. Let
$\tilde{\alpha}(t) = g\circ \alpha(t)$ for $t \in \left[ t_N, t_{N+1} \right] $. 
Since $g \circ \alpha(t_N)$ agrees with
on $\tilde{\alpha}(t_N)$, the gluing lemma ensures continuity of
$\tilde{\alpha}$ on all
of $\left[ 0, t_{N+1} \right] $.\\
\linebreak
With this construction
$\pi \circ \tilde{ \alpha}(t) =
\pi \circ g \circ \alpha(t) = \alpha(t)$ on $\left[ t_N, t_{N+1} \right] $, so
$\tilde{\alpha}$ is a lift on $\left[ t_N, t_{N+1} \right] $.\\
\linebreak
By repeating $m$ times, we thus have constructed 
$\tilde{\alpha}$ on all of $I = \left[ 0,1 \right] $,
with $\alpha = \pi \circ \tilde{\alpha}$ and thus $\tilde{\alpha}$ is a lift starting at
$x_0$.\\
\linebreak
For uniqueness, we notice that for $x_0$, we had that
$\alpha \left( \left[ 0, t_1 \right]  \right) $ lies in some $V_{y_1}$, so
$x_0 \in \pi^{-1}(V_{y_1})$ which, by assumption, is a disjoint union of sets each
homeomorphic to $V_y$ by $\pi$. Since these are disjoint, $x_0$ is in precisely
one of them, say $U$. So supposing $\beta $ is another lift of $\alpha$ on
$\left[ 0, t_1 \right] $ starting at $x_0$, we get
$\pi \circ \beta = \alpha = \pi \circ \tilde{\alpha}$. Since
$(\pi \circ \beta) (\left[ 0, t_1 \right] ) =
\alpha \left( \left[ 0, t_1 \right]  \right) 
= \left( \pi \circ \tilde{\alpha} \right) \left( \left[ 0, t_1 \right]  \right)
$ lies in some  $V_{y_1}$ and if $f$ is the inverse of $\pi|_{U}  \colon U \to
V_{y_1}$ (which exists as $\pi$ is a homeomorphism of $U$ onto $V_{y_1}$ ),
then
$\beta =f \circ \pi \circ \beta = f \circ \pi \circ \tilde{\alpha}
= \tilde{\alpha}$ on $\left[ 0, t_1 \right] $. Hence there is a unique
extension of $\tilde{\alpha}$ over $\left[ 0, t_1 \right] $.\\
Similarly, for arbitrary  $1 \le i \le m-1$,
$\alpha \left( \left[ t_{i}, t_{i+1} \right]  \right) 
\subset V_y$ for some $y$, so, as above,
$\tilde{\alpha}(t_i) \in U \subset \pi^{-1}(V_y)$ where $\pi$ maps $U$
homeomorphically onto $V_y$. Suppose now $\beta$ is another lift of
$\alpha$ on $\left[ t_i, t_{i+1} \right] $ starting at
$\tilde{\alpha}(t_i)$. We have
$\pi \circ \beta = \alpha = \pi \circ \tilde{\alpha}$ on $\left[ t_i, t_{i+1}
\right] $, so as this takes values in $V_y$, let
$g$ be the inverse of $\pi|_{U}  \colon U \to V_y$; giving
$\beta = g \circ \pi \circ \beta = g \circ \pi \circ \tilde{\alpha}
= \tilde{\alpha}$ on $\left[ t_i, t_{i+1} \right] $. As we showed
uniqueness for the base case $i = 0$ and
this inductive step ensures
that gives a unique extension
$\tilde{\alpha}|_{\left[ 0, t_N \right] }$ there
exists a unique extension
$\tilde{\alpha}|_{\left[ 0, t_{N+1} \right] }$ 
of $\tilde{\alpha}|_{\left[ 0,t_N \right] }$
over $\left[ t_N, t_{N+1} \right] $.
Repeating this $m$ times gives uniqueness of
$\tilde{\alpha}$.\\
\linebreak
\textbf{21:} Describe the homomorphism
$f_*  \colon \pi_1 \left( S^{1},1 \right) 
\to \pi_1 \left( S^{1}, f(1) \right) $ induces by each of the following maps:\\
(a) The antipodal map $f\left( e^{i \theta} \right) 
= e^{i \left( \theta + \pi \right) }, 0 \le \theta \le 2 \pi$.\\
\linebreak
(b) $f(e^{i \theta}) =
e^{i n \theta}, 0 \le \theta \le 2 \pi$ where $n \in \mathbb{Z}$.\\
\linebreak
(c) $f\left( e^{i \theta} \right) 
= \begin{cases}
    e^{i \theta}, & 0 \le \theta \le  \pi\\
    e^{i \left( 2 \pi -\theta \right) }, & \pi \le \theta \le  2\pi
\end{cases}$.\\
\linebreak
\textit{Solution:}\\
\linebreak
In the following, let
$\langle g \rangle $ denote a generator for
$\pi_1 \left( S^{1}, 1 \right) $ with $g  \colon I \to S^{1}$ given by
$g(t) = e^{2 \pi i t}$.\\
For any induced homomorphism
$f_*  \colon \pi_1\left( S^{1},1 \right) 
\to \pi_1 \left( S^{1},f(1) \right) $, let $\gamma
 \colon I \to S^{1}$ be the loop
 $f \circ g$.\\
 Then for any $\alpha \in \pi_1(S^{1}, 1)$, we have
 that there exists an $n \in \mathbb{Z}$ such that
 $\langle \alpha \rangle 
 = \langle g \rangle^{n}
 = \langle g^{n} \rangle $, so
 $f_* \langle \alpha \rangle 
 = \langle f \circ g^{n} \rangle 
 = \langle f \circ g \rangle^{n}
 = \langle \gamma \rangle^{n}$, so 
 $\langle \gamma \rangle $ is a generator for the image
 of $f_*$.\\ Hence
 $f_*$ on the generator  $\langle g \rangle $, i.e.
 $f \circ g$, gives a complete description of the homomorphism
  $f_*  \colon \pi_1 \left( S^{1},1 \right) \to 
  \pi_1 \left( S^{1},f(1) \right) $.\\
 \linebreak
 

 (a)
We claim that the image of  $f_*$ is isomorphic to $\mathbb{Z} 
\cong \pi_1 (S^{1}, -1)$.\\
\linebreak
Let $\alpha  \colon I \to S^{1}$ be the path
$\alpha(t) = e^{i \pi t}$. Then the isomorphism induces by $\alpha$ is given by
\[
\alpha_* \langle \beta \rangle 
= \langle \alpha^{-1} \beta \alpha \rangle 
\] 
which is an isomorphism of $\pi_1 (S^{1}, 1)$ and
$\pi_1 \left( S^{1},-1 \right) $. We claim that
$f_* \left( \langle \beta \rangle  \right) 
= \alpha_* \left( \langle \beta \rangle  \right) $. For this, it suffices to
show that it is true on the generator of $\pi_1 \left( S^{1},1 \right) $ since
$f_*$ and $\alpha_*$ are homomorphisms. 
We want to produce a homotopy between
$f \circ g$ and $\alpha^{-1} g \alpha$. Define
the map


\[
F(s,t)
= 
\begin{cases}
    e^{i \pi (1-4s)}, & s \in  \left[ 0, \frac{1-t}{4} \right] \\
    e^{i \pi \left( \frac{4 (2-t)s + 4t - 2}{1+t} \right) }, & s \in \left[
    \frac{1-t}{4} , \frac{1}{2} \right] \\
        e^{i \pi (2s-1)}, & s \in \left[ \frac{1}{2},1 \right]
\end{cases}.
\] 
This is continuous and a homotopy between
$F(s,0) = \alpha^{-1} \gamma \alpha$ and
$F(s,1) = f \circ g$.





\[
F(s,t) = 
\begin{cases}
    e^{ i \pi (1-4s)}, & s \in \left[ 0, \frac{1-t}{4} \right] \\
    e^{i \pi \frac{8s - 4st + 4t - 2}{1+s}}, & s \in \left[ \frac{1-t}{4},
    \frac{1}{2} \right] \\
        e^{i \pi (2s-1)}, & s \in \left[ \frac{1}{2},1 \right] 
\end{cases}
\] 
Hence, $f_*  \colon \pi_1 \left( S^{1}, 1 \right) 
\to \pi_1 \left( S^{1},-1 \right) $ is the map sending a generator to
a generator.\\
\linebreak

 













(b) We have
\[
f \circ g (t) = 
e^{2 \pi i n t} = g(t)^{n}, \quad t \in \left[ 0,1 \right] 
\] 
which is the loop that winds around $S^{1}$ counterclockwise 
$n$ times.\\
Thus $f_*$ maps $g \to g^{n}$ - or, identifying $\pi_1 \left( S^{1},1 \right)
$ with $\mathbb{Z}$, maps $1 \to n$.\\
This generates the image group of $f_*$ which thus is
$\langle g^{n} \rangle 
\le \pi_1 \left( S^{1},1 \right) $.





(c) 
\[
\gamma(t) := f \circ g(t) =
\begin{cases}
    e^{2 \pi i t}, & t \in \left[ 0, \frac{1}{2} \right] \\
    e^{i 2 \pi (1-t)}, & t \in \left[ \frac{1}{2},1 \right] 
\end{cases}
\] 
is the loop that goes counterclockwise from $1$ to $-1$ along $S^{1}$ 
during $\left[ 0, \frac{1}{2} \right] $ and then
reversing direction going from $-1$ to $1$ clockwise during $\left[
\frac{1}{2},1 \right] $.\\
Now $\gamma$ is homotopic to the constant loop by
the homotopy
$F  \colon I \times I \to S^{1}$ given by
$F(s,t) =
\gamma(st)$, so since $\langle 1 \rangle 
= \langle \gamma \rangle $ generates the image subgroup, we find that
$f_*$ is the homomorphism mapping everything to the constant loop.




























\end{document}
