\documentclass[a4paper]{article}

\usepackage[margin=2.5cm]{geometry}
\usepackage[pdftex]{graphicx}
\usepackage[utf8]{inputenc}
\usepackage[T1]{fontenc}
\usepackage{textcomp}
\usepackage{babel}
\usepackage{amsmath, amssymb}
\usepackage[colorlinks=true,linkcolor=blue]{hyperref}
\usepackage{float}
\usepackage{mathrsfs}
%\usepackage{enumitem}
%% for identity function 1:
\usepackage{bbm}
%%For category theory diagrams:
%\usepackage{tikz-cd}
%%For code (e.g. python) in latex:
%\usepackage{listings}
%
%Usage: 
%\begin{lstlisting}[language=Python]
%\end{lstlisting}

\newcommand{\incfig}[2][1]{%
\def\svgwidth{#1\columnwidth}
\import{./figures/}{#2.pdf_tex}
}


% figure support
\usepackage{import}
\usepackage{xifthen}
\pdfminorversion=7
\usepackage{pdfpages}
\usepackage{transparent}

\pdfsuppresswarningpagegroup=1

\setlength\parindent{0pt}

\newcommand{\qed}{\tag*{$\blacksquare$}}
\newcommand{\qedwhite}{\hfill \ensuremath{\Box}}

%Inequalities
\newcommand{\cycsum}{\sum_{\mathrm{cyc}}}
\newcommand{\symsum}{\sum_{\mathrm{sym}}}
\newcommand{\cycprod}{\prod_{\mathrm{cyc}}}
\newcommand{\symprod}{\prod_{\mathrm{sym}}}

%Linear Algebra

\DeclareMathOperator{\Span}{span}
\DeclareMathOperator{\Ima}{Im}
\DeclareMathOperator{\diag}{diag}
\DeclareMathOperator{\Ker}{Ker}
\DeclareMathOperator{\ob}{ob}
\DeclareMathOperator{\Hom}{Hom}
\DeclareMathOperator{\sk}{sk}


%Row operations
\newcommand{\elem}[1]{% elementary operations
\xrightarrow{\substack{#1}}%
}

\newcommand{\lelem}[1]{% elementary operations (left alignment)
\xrightarrow{\begin{subarray}{l}#1\end{subarray}}%
}

%SS
\DeclareMathOperator{\supp}{supp}
\DeclareMathOperator{\Var}{Var}

%NT
\DeclareMathOperator{\ord}{ord}

%Alg
\DeclareMathOperator{\Rad}{Rad}
\DeclareMathOperator{\Jac}{Jac}

\DeclareMathAlphabet{\pazocal}{OMS}{zplm}{m}{n}
\newcommand{\unif}{\pazocal{U}}

\title{Assignment 4}

\author{Jonas Trepiakas - jtrepiakas@berkeley.edu - Student ID: 3039733855}
\date{}

\begin{document}
\maketitle
\newpage
    \textbf{31:} Give the set of real numbers the finite-complement topology.
    What are the components of the resulting space? Answer the same question
    for the half-open interval topology.\\
    \linebreak
    \textit{Solution:} Let $\mathbb{R}_{fc}$ denote the reals with the finite
    complement topology. We claim that $\mathbb{R}_{fc}$ is connected.\\
    Assume there exist disjoint open sets $U,V \subset \mathbb{R}_{fc}$ such
    that
    $\mathbb{R} = U \cup V$. Then by definition, 
    $\mathbb{R}-U$ and $\mathbb{R}-V$ are finite, however, as $U \cap
    V = \varnothing$,
    we have $V \subset \mathbb{R}-U$, so $V$ is finite, and hence
    $\mathbb{R} - V$ is infinite, contradicting $V$ being a nonempty open set
    - which would have to be infinite.\\
    By theorem 3.20 (c) equivalent to (a), we have that $\mathbb{R}_{fc}$ is
    connected.\\
    \linebreak
    \textbf{34:} A space $X$ is locally connected if for each $x \in X$, and
    each neighborhood $U$ of $x$, there is a connected neighborhood $V$ of $x$ 
    which is contained in $U$. Show that any euclidean space, and therefore any
    space which is locally euclidean (like a surface), is locally connected. If
    $X = \left\{ 0 \right\} \cup 
    \left\{ \frac{1}{n}  \mid n = 1,2,\ldots \right\} $ with the
    subspace topology from the real line, show that $X$ is not locally
    connected.\\
    \linebreak
    \textit{Solution:} We have the continuous map
    $f  \colon (-1,1) \to \mathbb{R}$ defined by
    $f(t) = \frac{t}{1-|t|}$ defines a homeomorphism from $(-1,1)$ to $\mathbb{R}$ 
    with inverse $f^{-1} (t) = \frac{t}{1 + |t|}$, and
    $(-1,1)$ is homeomorphism to any open interval $(a,b)$ by the map
    $t \to a + (b-a) \frac{t+1}{2}$ with inverse
    $t \to -1 + 2 \frac{t-a}{b-a}$. Hence any interval $(a,b)$ is homeomorphic
    to
    $\mathbb{R}$ which is connected and as connectedness is a topological
    property, $(a,b)$ is also connected. 
    By induction on theorem 3.26, we have that
    any generalized open cube
    \[
        (a_1, b_1) \times (a_2, b_2) \times \ldots \times (a_n, b_n) \subset
        \mathbb{R}^{n}
    \] 
    is connected.
    Now, the product topology on $\mathbb{R}^{n}$ is generated by
    the base that consists of all products
    $U_1 \times \ldots \times U_n$, such that $U_i \subset \mathbb{R}$ is
    open.\\
    Taking for each $U_i$ a basis element $(a_i, b_i) \subset U_i$, we see that
    $(a_1, b_1) \times \ldots \times (a_n , b_n) \subset U_1 \times \ldots
    \times U_n$, so the topology on $\mathbb{R}^{n}$ generated by the basis
    of all products of intervals
    $(a_1, b_1) \times \ldots \times (a_n, b_n)$ is finer than the topology
    induced by all product of open sets in $\mathbb{R}$.\\
    Conversely, since each $(a_i, b_i)$ is already an open set in $\mathbb{R}$,
    the topology generated by the base consisting of products of open sets of
    $\mathbb{R}$ is finer than the topology generated by the base consisting of
    products of open intervals. Thus, as these topologies are each finer than
    the other, the topologies are the same.\\
    \linebreak
    Now we are ready to show the result: let $x \in \mathbb{R}^{n}$ and $U$ be
    a neighborhood of $x$. Then there exists an open set 
    $V \subset U$ containing $x$. As $V$ is open, we can now find a basis
    element
    $ I= (a_1, b_1) \times \ldots \times (a_n ,b_n) \subset V$ with
    $x \in I$. As $I$ was shown to be connected, we thus find that
    $\mathbb{R}^{n}$ is locally connected.\\
    \linebreak
    Now suppose $X = \left\{ 0 \right\} \cup \left\{ \frac{1}{n} \mid 
    n = 1, 2, \ldots \right\} $ with the subspace topology from the real
    line.\\
    Let $U$ be a neighborhood of $0$, and suppose there exists a connected
    neighborhood $V$ of $0$ which is contained in $U$. As $V$ is
    a neighborhood, there exists an open neighborhood $W \subset V$ of $0$ and
    thus we can find some interval $(a,b)$ containing  $0$ such that
    $0 \in (a,b) \cap X \subset W \subset V$. Now, choose a
    $N \in \mathbb{N}$ such that $\frac{1}{N} \in (a,b)$. Then
    choosing any $q \in \left( \frac{1}{N+1}, \frac{1}{N} \right) $, we have
    that
    $ \left( (-\infty, q) \cap V \right) \cup
    \left( (q, \infty) \cap V \right) = V$, and
    $0 \in (\infty, q) \cap V$ and $\frac{1}{N} \in (q, \infty) \cap V$, so
    $(-\infty, q)\cap V$ and $(q, \infty) \cap V$ are nonempty disjoint open sets
    in $X$ separating $V$, contradicting $V$ being connected - using
    formulation (c) in theorem 3.20 for connectedness.
          





























\end{document}
