\documentclass[a4paper]{article}

\usepackage[margin=2.5cm]{geometry}
\usepackage[pdftex]{graphicx}
\usepackage[utf8]{inputenc}
\usepackage[T1]{fontenc}
\usepackage{textcomp}
\usepackage{babel}
\usepackage{amsmath, amssymb}
\usepackage[colorlinks=true,linkcolor=blue]{hyperref}
\usepackage{float}
\usepackage{mathrsfs}
%\usepackage{enumitem}
%% for identity function 1:
\usepackage{bbm}
%%For category theory diagrams:
%\usepackage{tikz-cd}
%%For code (e.g. python) in latex:
%\usepackage{listings}
%
%Usage: 
%\begin{lstlisting}[language=Python]
%\end{lstlisting}

\newcommand{\incfig}[2][1]{%
\def\svgwidth{#1\columnwidth}
\import{./figures/}{#2.pdf_tex}
}


% figure support
\usepackage{import}
\usepackage{xifthen}
\pdfminorversion=7
\usepackage{pdfpages}
\usepackage{transparent}

\pdfsuppresswarningpagegroup=1

\setlength\parindent{0pt}

\newcommand{\qed}{\tag*{$\blacksquare$}}
\newcommand{\qedwhite}{\hfill \ensuremath{\Box}}

%Inequalities
\newcommand{\cycsum}{\sum_{\mathrm{cyc}}}
\newcommand{\symsum}{\sum_{\mathrm{sym}}}
\newcommand{\cycprod}{\prod_{\mathrm{cyc}}}
\newcommand{\symprod}{\prod_{\mathrm{sym}}}

%Linear Algebra

\DeclareMathOperator{\Span}{span}
\DeclareMathOperator{\Ima}{Im}
\DeclareMathOperator{\diag}{diag}
\DeclareMathOperator{\Ker}{Ker}
\DeclareMathOperator{\ob}{ob}
\DeclareMathOperator{\Hom}{Hom}
\DeclareMathOperator{\sk}{sk}
\DeclareMathOperator{\Vect}{Vect}
\DeclareMathOperator{\Set}{Set}
\DeclareMathOperator{\Group}{Group}
\DeclareMathOperator{\Ring}{Ring}
\DeclareMathOperator{\Ab}{Ab}
\DeclareMathOperator{\Top}{Top}
\DeclareMathOperator{\Htpy}{Htpy}
\DeclareMathOperator{\Cat}{Cat}
\DeclareMathOperator{\CAT}{CAT}
\DeclareMathOperator{\Cone}{Cone}


%Row operations
\newcommand{\elem}[1]{% elementary operations
\xrightarrow{\substack{#1}}%
}

\newcommand{\lelem}[1]{% elementary operations (left alignment)
\xrightarrow{\begin{subarray}{l}#1\end{subarray}}%
}

%SS
\DeclareMathOperator{\supp}{supp}
\DeclareMathOperator{\Var}{Var}

%NT
\DeclareMathOperator{\ord}{ord}

%Alg
\DeclareMathOperator{\Rad}{Rad}
\DeclareMathOperator{\Jac}{Jac}

\DeclareMathAlphabet{\pazocal}{OMS}{zplm}{m}{n}
\newcommand{\unif}{\pazocal{U}}

\begin{document}
    \textbf{1:} Let $F = x^3 + y^3 - 2xyz$, and consider $\mathbb{V}(F)
    \subset \mathbb{P}_{\mathbb{C}}^2$.\\
    (a) The tangent line to $\mathbb{V}(F)$ at $\left[ 1 : 1 : 1 \right] $ is
    the projective closure of the tangent line to $V(f)$ at $(1,1)$ where
    $f = x^3 + y^3 - 2xy$. The tangent line to $V(f)$ at $(1,1)$ is
    $f_x(1,1) (x-1) + f_y(1,1) (y-1)$. Now
    $f_x = 3x^2 - 2y$ and $f_y = 3y^2 - 2x$, so $f_x(1,1) = 1 = f_y (1,1)$, so
    $T_{(1,1)}V(f) = V\left( x+y-2 \right) $. Hence
    the tangent line to $\mathbb{V}(F)$ becomes
    $\mathbb{V}(x+y-2z)$.\\
    \linebreak
    (b) $P$ is a singular point of $\mathbb{V}(F)$ if and only if
    $F(P) = F_x(P) = F_y(P) = F_z(P) = 0$.\\
    Now,  $F_x = 3x^2 - 2yz, F_y = 3y^2 - 2xz$ and
    $F_z = -2xy$. If $x = 0$ then $y = 0$, so $z\neq 0$. If
    $y = 0$ then $x=0$ so $z\neq 0$. Since $F_z$ must be zero at a singular
    point, either no singular point is in $U_3^{c}$.\\
    So the only singular point is
    $\left[ 0 : 0 : 1 \right]$. \\
    \linebreak
    (c) The multiplicity of $\left[ 0:0:1 \right] $ is
    the multiplicity of $(0,0)$ for $f = x^3 + y^3 - 2xy$ which is
    $2$ and the tangent cone is $V(xy) = V(x) \cup V(y)$.\\
    \linebreak
    \textbf{2:} Let $f = y^{4} + y^3 - x^2$, and consider $V(f) \subset 
    \mathbb{A}_{\mathbb{C}}^2$.\\
    (a) The homogenization of $f$ is
    $F = y^{4} + y^{3}z - x^2 z^2$, so the projective closure of
    $V(f)$ is $\mathbb{V}(y^{4}+y^3 z - x^2 z^2)$.\\
    \linebreak
    (b) $F$ is Eisenstein in $x^2$, and hence irreducible. Now
    $F_x = -2x z^2, F_y = 4y^3 + 3y^2 z$ and
    $F_z = y^3 - 2zx^2$.\\
    If $x = 0$ then $y = 0$, so $z\neq 0$. If
    $z = 0$ then $y = 0$, so $x \neq 0$. So the only singular points are
    $\left[ 0 : 0 : 1 \right] $ and $\left[ 1 : 0 : 0 \right] $.\\
    \linebreak
    (c) The multiplicity of $\left[ 0 : 0 : 1 \right] $ is the multiplicity of
    $(0,0)$ of $f$ which is $2$ and the
    tangent cone is $V(x^2) = V(x)$ so the tangent cone for
    $\left[ 0:0:1 \right] $ is
    $\mathbb{V}(x)$.\\
    The mult of $\left[ 1 : 0 : 0 \right] $ is the mult of $(0,0)$ of
    $g = y^{4} + y^3 z - z^2$ which is $2$ and the tangent cone of
    $\left[ 1 : 0 : 0 \right] $ is $\mathbb{V}(z)$.\\
    \linebreak
    \textbf{3:}\\
    (a) 
    \begin{align*}
        I_P (xy+y^3, x^2 + 2xy + y^3)
        &= I_P \left( y(x+y^2) , x(x+y) \right)\\
        &= I_P(y,x) + I_P(y,x+y) + I_P(x+y^2,x) + I_P(x+y^2 , x+y)\\
        &= 1 + 1 + 2 + I_P(y(y-1), x+y)\\
        &= 4 + \underbrace{I_P(y,x+y)}_{=1} + I_P(y-1, x+y)\\
        &= 5.
    \end{align*}
    If we instead follow the algorithm:\\
    First reduce to
    $I_P(y,x^2 + 2xy + y^3) + I_P(x + y^2, x^2 + 2xy + y^3)$ by (6).\\
    Now, the first one reduces to $I_P(y,x^2) = 2$.\\
    The second is $I_P (x+y^2, 2xy) = I_P (x+y^2, x) + I_P(x+y^2, y)
    = 2 + 1 = 3$.\\
    So we recover $I_P\left( xy + y^3, x^2 + 2xy + y^3 \right) = 5 $.\\
    \linebreak
    (b) We have
    that the tangent cone of $y^2 - x^3$ is $V(y)$ and
    the tangent cone of $xy + x^{4} + y^{4}$ is 
    $V(x) \cup V(y)$.\\
    Let $f = y^2 - x^3$ and $g = xy + x^{4} + y^{4}$. Then
    $f(x,0) = -x^3$ and $g(x,0) = x^{4}$.\\
    We have case 2 in (6), with $\deg f \le \deg g$, so write
    $h_2 = g + xf = xy + y^{4} + xy^2$. Then
    $I_P(f,g) = I_P(f,h_2)$.\\
    Now back to step (2).\\
    $h_2 = y (x+y^3 + xy)$, so  $h_2$ and $f$ have no common factor.\\
    (3)  $(0,0) \in V(f) \cap V(h_2)$.\\
    (4) Now the tangent cone of  $h_2$ is $V(xy) = V(x)\cup  V(y)$, so
    $V(y)$ is again in common. Now $h_2(x,0) = 0$, so
    $\deg h_2(x,0) \le \deg f(x,0)$. Write
    $h_2 = y h$, so $h = x+ y^3 + xy$. Now, by (6), we have
    $I_P(f,h_2) = I_P(y, f) + I_P(h,f)$, and
    $I_P(y,f) = 3$, and
    $I_P(h,f) = I_P(x+y^3 + xy, y^2 - x^3)$. Back to (2).\\
    Tangent cone of $x+y^3 + xy$ is $V(x)$ and tangent cone of
    $y^2 - x^3$ is $V(y)$. No lines in common, so
    $I_P(x+y^3 + xy, y^2 - x^3) = \text{mult}_{(0,0)}(x+y^3 + xy)
    \text{mult}_{(0,0)}(y^2 - x^3) = 2$. Thus
    $I_P(f,g) = 3+2 = 5$.\\
    \linebreak
    (c) Let $f = x^2 + y^2 + x^{4}y^{4}$ and $g = x^3 - y^3 + 3xy^{5}$.\\
    We have that the tangent cone of $f$ at $P$ is
    $V(x^2 + y^2)$ and the tangent cone of $g$ at $P$ is
    $V(x^3 - y^3) = V((x-y)(x^2 + xy + y^2)) = V(x-y) \cup V(x^2 + xy + y^2)$.
    Since they have no lines in common, we have
    $I_P(f,g) = \text{mult}_P(f) \text{mult}_P(g) = 
    2 \cdot  3 = 6$.\\
    \linebreak
    \textbf{4:} Let $\alpha = \frac{x^2 + xy}{3y^2} \in k(\mathbb{P}^{1})$.
    What is the value of $\alpha$ at $\left[ 1:3 \right] \in \mathbb{P}^{1}$?
    Where is $\alpha$ defined?\\
    \linebreak
    \textit{Solution:} Since $\alpha$ is a fraction of
    homogeneous polynomials and $3y^2$ is nonzero at
    $\left[ 1:3 \right] $, we have that $\alpha$ is defined at
    $\left[ 1:3 \right] $ and is given by
    $\frac{1^2 + 1 \cdot 3}{3 \cdot 3^2}= \frac{4}{27}$. Now  $\alpha$ is
    defined whenever $y\neq 0$, so
    $\alpha$ is defined on $U_2$.\\
    \linebreak
    \textbf{5:} It is by definition of closure an algebraic set.\\
    Suppose $\overline{\varphi(X)} =
    U \cup V$ with $U$ and $V$ algebraic sets which are both proper subsets
    of $\overline{\varphi(X)}$. We have
    $\varphi^{-1}(U \cup V) = \varphi^{-1}(U) \cup \varphi^{-1}(V) \supset X$,
    hence $X = U' \cup V'$ where $U' = \varphi^{-1}(U) \cap X$ and
    $V' = \varphi^{-1}(V) \cap X$ which are both open. Hence
    $U' = X$ without loss of generality. So
    $\varphi^{-1}(U) = X$, but then
    $\varphi(X) = \varphi\left( \varphi^{-1}(U) \right) 
    \subset U$, so $\overline{\varphi(X)} 
    \subset \overline{U}$. Now
    We know $U$ is open, and since $U \cup V = \overline{\varphi(X)}$, we have
    that $U$ is closed in $\overline{\varphi(X)}$, so since
    $\overline{\varphi(X)}$ is closed in $\mathbb{A}^{m}$, we have that
    $U$ is closed in $\mathbb{A}^{m}$ too. So $U$ is both closed and open.
    Hence $\overline{U} = U$, so
    $\overline{\varphi(X)} = U$, contradiction.\\
    \linebreak
    \textbf{6:} Let $\varphi  \colon \mathbb{A}^{1} \to \mathbb{A}^{3}$ be the
    morphism given by $t \mapsto \left( t, t^2, t^3 \right) $. Let
    $x,y,z$ be the coordinates on $\mathbb{A}^3$.\\
    (a) We have that $\Gamma(\mathbb{A}^3) = 
    k\left[ x,y,z \right] / I\left( \mathbb{A}^3 \right) 
    = k\left[ x,y,z \right] / (0) = k\left[ x,y,z \right] $, and similarly,
    $\Gamma(\mathbb{A}^{1}) = k\left[ t \right] $.\\
    Now $\varphi^{*}(x)(t) =
    x \left( \varphi(t) \right) = x(t,t^2,t^3) = t$,
    $\varphi^{*}(y)(t) = y \left( t,t^2,t^3 \right) =t^2$ and similarly,
    $\varphi^{*}(z)(t) = t^3$.\\
    \linebreak
    (b) The image of $\varphi$ is indeed closed as it is
    $V\left( y-x^2, z-x^3 \right) $. Namely, for any
    $t \in \mathbb{A}$, we have $\varphi(t) = \left( t, t^2, t^3 \right)
    $ which clearly lies in $V(y-x^2, z-x^3)$. Conversely, if
    $(x,y,z) \in V(y-x^2, z-x^3)$ then
    $y = x^2$ and $z = x^3$, so $(x,y,z) = (x, x^2, x^3) =
    \varphi(x) \in \Ima \varphi$.\\
    In fact, this also follows directly from the fact that
    $\varphi^{*}$ maps $x \mapsto t$ and is thus surjective. By a proposition,
    we then have that $\Ima \varphi$ is an algebraic set and even that
    $\varphi$ is an isomorphism of the domain onto its image.\\
     \linebreak
     (c) This follows from the last comment of (b).\\
     Alternatively,
     the map $\psi  \colon \mathbb{A}^3 \to \mathbb{A}^{1}$ by
     $(x,y,z) \mapsto x$ is a polynomial map given by
     $P \mapsto \left( T(P) \right) $ where $T(x,y,z) = x$. Furthermore,
     suppose $(x,y,z) \in \Ima \varphi$. Then
     $\varphi \circ \psi (x,y,z)
     = \varphi(x) = (x,x^2, x^3)
     = (x,y,z)$, and
     $\psi \circ \varphi(x)
     =\psi (x,x^2,x^3) = x$, so
     indeed $\varphi$ is an isomorphism.\\
     \linebreak
     (d) We have
     \[
     \varphi^{-1}\left( V(yz-x^5) \right) 
     = V\left( \varphi^{*}(yz-x^{5}) \right) 
     = V\left( t^2 t^3 - t^{5} \right) 
     = V(0) = \mathbb{A}^{1}.
     \] 
     \textbf{7:} Let $X = V\left( x^2 z, x^2 + xz + yz + y^2 \right) 
     \subset \mathbb{A}_{\mathbb{C}}^3$. If $x^2 z = 0$ then either
     $x = 0$ or $z = 0$. Suppose $x = 0$, then
     $yz + y^2 = y (z+y) = 0$, so either $y = 0$ or $z = -y$.\\
     If instead $z = 0$ then
     $x^2 + y^2 = (x+iy)(x-iy) = 0$, so either $x = -iy$ or $x = iy$. Thus
     \[
    X = V(x, z+y) \cup V(x+iy) \cup V(x-iy)
     \] 
     where $V(x,y) \subset V(x+iy)$.\\
     These are irreducible since $\Gamma(V(x, z+y)) \cong k\left[ x \right]
     $ which is an integral domain, so
     $I\left( V\left( x, z+y \right)  \right) $ is prime and hence
     $V(x, z+y)$ is irreducible. Now
     $V(x+iy)$ and $V(x-iy)$ are irreducible since
     $\Gamma(V(x+iy)) \cong k\left[ t,s \right] \cong
     \Gamma(v(x-iy))$ which is also an integral domain.\\
     \linebreak
     \textbf{8:} Let $J = \left( y^2 - x^2, y^2 + x^2 \right) 
     \subset \mathbb{C}\left[ x,y \right] $.\\
     \linebreak
     (a) If $(x,y) \in V(J)$ then $y^2 - x^2 = 0 = y^2 + x^2$ so
     $x = 0$ and hence $y = 0$. So
     $V(J) = \left\{ (0,0) \right\} $.\\
     \linebreak
     (b) 
     \[
     \dim_{\mathbb{C}} \mathbb{C}\left[ x,y \right] / J
     = \dim_{\mathbb{C}}\mathbb{C}\left[ x,y \right] / \left( y^2 - x^2 , y^2
     + x^2 \right) 
     \]
     In $\mathbb{C}\left[ x,y \right] / J$, we have that
     $y^{n} = y^{2} y^{n-2} = 0$ for $n\ge 2$ since $\frac{1}{2} \left[ y^2 -x^2 + y^2 +x^2 \right] 
     = y^2 \in J$ and
     similarly  $x^{2} = 0$. Hence
      $\mathbb{C}\left[ x,y \right] / J$ is generated by
      $\left\{ 1, x, y, xy \right\} $, so
      $\dim_{\mathbb{C}} \mathbb{C}\left[ x,y \right] /J = 4$.\\
      \linebreak
      (c) 
      \[
      I_{(0,0)}\left( y^2 - x^2, y^2 + x^2 \right) 
      = I_{(0,0)}\left( y^2 - x^2 , 2y^2 \right) 
      = I_{(0,0)}\left( -x^2 , 2y^2 \right) 
      = I_{(0,0)}(x^2 , y^2)
      = 4 I_{(0,0)}(x,y) = 4.
      \] 
      (d) We have that
      $\mathbb{V}(J)$ is the projectivization of $V(J) = \left\{ (0,0) \right\}
      $, so
      $\mathbb{V}(J) = \varnothing$.\\
      \linebreak
      \textbf{9:} Find the projective closure of
      $V(x+y^3 + z) \subset \mathbb{A}^3$ in $\mathbb{P}^3$. Is it smooth?\\
      \linebreak
      \textit{Solution:} The projective closure is the vanishing of the
      homogenization:
      \[
      \mathbb{V} \left( xw^2 + y^3 + zw^2 \right) \subset \mathbb{P}^3
      \] 
      This is smooth if it has no singular points. Now
      $x w^2 + y^3 + z w^2$ is Eisenstein in $w^2$, and hence has no repeated
      factors. So a point $P\in \mathbb{P}^3$ is singular if and only if
      for  $F = xw^2 + y^3 + zw^2$, we have
      $F(P) = F_x(P) = F_y(P) = F_z(P) = 0$. Now
      $F_x = w^2, F_y = 3y^2, F_z = w^2, F_w = 2xw + 2zw$. So
      if  $P = \left[ x,y,z,w \right] $ then
      $w = y = 0$. So
      any points of the form $\left\{ a : 0 : b : 0 \right\} $ for either
      $a\neq 0$ or $b \neq 0$ is singular. Hence the closure is not smooth.
      This can be verified since e.g. $\left[ 1 : 0 : 0 : 0 \right] 
      \in U_1$ and hence letting $f = w^2 + y^3 + z w^2$, we have that
      $f_y = 3y^2, f_z = w^2, f_w = 2w + 2zw$ all of which disappear at
      $(0,0,0)$, so
      $(0,0,0)$ is a singular point of $V(f)$ and hence
      $\left[ 1 : 0 : 0 : 0 \right] $ is a singular point of
      $\mathbb{V}\left( xw^2 + y^3 + zw^2 \right) $.\\
      \linebreak
      \textbf{10:}\\
      (1) Suppose $L$ is a finite extension of $k$. This means that
      $\left[ L : k \right] $ is finite, i.e., that
      $L$ as a $k$-vector space has finite dimension, which means that there
      exist $l_1, \ldots, l_n$ such that
      $L = \sum k l_i$, i.e., $L$ is finitely generated as a $k$-module. Now,
      $L$ is an algebraic extension over $k$ if for any $l \in L$, there
      exists some polynomial $f \in k\left[ x \right] $ such that
      $f(l) = 0$. Now,  $1, l, l^2, \ldots, l^{n}$ are linearly dependent as
      the dimension of the $k$-vector space $L$ is $n$, so there exist
      $k_0, \ldots, k_n$ such that
      $k_0 + k_1 l + \ldots + k_n l^{n} = 0$ and hence
      the polynomial $f = k_0 + k_1 x+ \ldots + k_n x^{n} \in k\left[ x \right]
      $ is a polynomial such that $f(l)=0$, so $l$ is algebraic over
      $k$. As $l$ was arbitrary, we have that $L$ is an algebraic extension of
      $k$.\\
      \linebreak
      (2) False. Not all algebraic extensions are finite.\\
      Consider $k = \mathbb{Q}$ and 
      $L = \mathbb{Q}\left[ \sqrt{2} , \sqrt[3]{2},
      \sqrt[4]{2}, \ldots \right] $. Then $L$ is clearly algebraic over
      $k$; however, $L$ is not a finite extension as a basis is
      the infinte set $\left\{ \sqrt[n]{2}  \right\}_{n \ge 2}$.\\
      \linebreak
      (3) True. Suppose $\varphi  \colon X \to Y$ is an isomorphism with
      inverse
      $\psi  \colon Y \to X$. Let first  $y \in Y$. Then
      $\varphi \left( \psi (y) \right) = y$, so
      $\varphi$ is surjective. Now
      If $x,x' \in X$ with $\varphi(x) = \varphi(x')$ then
      $x = \psi\left( \varphi(x) \right) = \psi(\varphi(x')) = x'$. So
      $\varphi$ is injective.\\
      \linebreak
      (4) Consider the map 
      $\varphi  \colon \mathbb{A}^{1} \to V\left( y^3 - x^2 \right) $ by
      $t \mapsto (t^2, t^3)$. This is a bijection on points, however, 
      $\varphi$ is not an isomorphism, since 
      \[
      \Gamma(A^{1}) = k\left[ x \right] \not \cong k\left[ x,y \right] /(y^3
      - x^2)
      \] 
      
      since the first is UFD, however, in $k\left[ x,y \right] / (y^3 - x^2)$,
      we have $y^3 = x^2$, so since
      $y$ and $x$ are irreducible, we have that $y$ is associate to $x$, which
      is a contradiction.\\
      \linebreak
      (5) False. E.g. $I$ is closed in the classical topology on
      $\mathbb{A}^{1}$, however not in the Zariski topology since an infinite
      solution set implies that the function is identically zero.\\
      \linebreak
      (6) True. The Zariski topology is coarser than the classical topology.
      This follows as polynomial maps are continuous.\\
      \linebreak
      (7) False. $(xy) \subset (x)$ but
      $(1,0) \in V((xy))$ while
      $(1,0) \not\in V(x)$, so
      $V((xy)) \not \subset V(x)$.\\
      \linebreak
      (8) True.\\
      (9) False. $\left[ 0 : 1 \right] \in 
      Y = \left\{ \left[ 1  : 0 \right] , \left[ 0 : 1 \right]  \right\} $, however
      $C(Y) = V(x) \cup V(y) \not \subset V(y)$ since
      $(0,1) \in C(Y)$ but $(0,1) \not\in V(y) = C(X)$.\\
      \linebreak
      (10) False.
      $x$ vanishes on $\mathbb{V}(x)$ and
      $y$ vanishes on $\mathbb{V}(y)$ but
      $x+y \in \mathbb{I}\left( \mathbb{V}(x) \right) +
      \mathbb{I}\left( \mathbb{V}(y) \right) =
      (x) + (y)$ does not vanish one
      $\mathbb{V}(x) \cup \mathbb{V}(y)$ since
      $\left[ 1:0 \right] \in \mathbb{V}(x) \cup \mathbb{V}(y)$ but
      $(x+y) \left( \left[ 1:0 \right]  \right) 
      = 1 \neq 0$.\\
      \linebreak
      (11) True.\\
      \linebreak
      (12) False. Let $J = (x^2 y^2)$. 
      Then $\left[ 1 : 0 \right] \in \mathbb{V}(J)$ and
      so $x \in \mathbb{I}\left( \mathbb{V}(J) \right) $, however,
      $x \not\in \left( x^2 y^2 \right) $ since we are dealing with an integral
      domain, so the degrees add.\\
      \linebreak
      (13) It suffices to show it in the affine variety, since
      the projective tangent space of a projective alg set $X \subset
      \mathbb{P}^{n}$ at $P \in X$ with $P \in U_i$ is the projective closure
      of $T_P \left( X \cap U_i \right) \subset U_i$ and
      the projective tangent cone of $X$ at $P$ is 
      the projective closure of $TC_P \left( X \cap U_i \right) $.\\
      In the affine case, if we write $f = f_1 + \ldots f_m$, then
      if $f_1 \neq 0$, we have that $T_P \left( X \cap U_i \right) 
      = TC_P\left( X\cap U_i \right) $. If
      $f_1 = 0$ then $T_P\left( X \cap U_1 \right) =\mathbb{A}^{n+1}$ and
      $TC_P\left( X \cap U_i \right) \subset \mathbb{A}^{n+1}$.\\
      \linebreak
      (14) Not true. E.g. for $x^2 \in k\left[ x,y \right] $, the tangent
      space is $\mathbb{A}^{2}$ but the tangent cone is $V(x)$ which does not
      contain all of $\mathbb{A}^2$.\\
      \linebreak
      (15) True.\\
      (16) Not true. For example, $(xy)$ is a radical ideal, however,
       it is not prime since $xy \in (xy)$ while neither $x$ nor $y$ are in
       $(xy)$.\\
       It is radical since if $f^{n} \in (xy)$ then
       $f^{n} = xy \alpha$ so
       $x, y  \mid f^{n}$ and as $x$ and $y$ are irreducible, we have
       $x,y  \mid f$ so $f \in (xy)$.\\
       \linebreak
       \textbf{11:} Let $C = V\left( (x-x_0)^2 + (y-y_0)^2 - r^2 \right) 
       \subset \mathbb{A}_{\mathbb{C}}^2$ be a circle with center
       $\left( x_0, y_0 \right) $ and radius $r$.\\
       \linebreak
       (a) Let $\overline{C} \subset \mathbb{P}_{\mathbb{C}}^2$ be the
       projective closure of $C$, given by
       $\mathbb{V}\left( (x-zx_0)^2 + (y-zy_0)^2 - (zr)^2 \right) $. Then, the
       tangent line to $\overline{C}$ at
       $\left[ 1:i:0 \right] $ is the tangent line to
       $V\left( (1-zx_0)^2 + (y-zy_0)^2 - (zr)^2 \right) $ at
       $(i,0)$. Let
       $\varphi  \colon \mathbb{A}^2 \to \mathbb{A}^2$ be the translation
       $(0,0) \mapsto (i,0)$. Then
       the tangent line $T_{(i,0)}
       V\left( (1-zx_0)^2 + (y-zy_0)^2 - (zr)^2 \right) $ is
       \[
           \varphi\left( T_{(0,0)}V
       \left( (1-zx_0)^2 + (y+i - zy_0)^2 - (zr)^2 \right) \right) 
   \]
      Inserting $(0,0)$, we get $0$, and the degree one polynomial part is
      $-2zx_0 + 2iy - 2i y_0 z
      = 2 \left( -z \left( x_0 + iy_0 \right) + iy \right) $. So the tangent
      line is
      $V\left( -z(x_0 +iy_0) + iy \right) $ which maps under $\varphi$ to
      $V\left( -z (x_0 + iy_0) + i(y-i) \right) $ whose projective closure is
      \[
      \mathbb{V} \left( -z (x_0 +iy_0) + i(y-xi) \right) 
      \] 
      Similarly, for $\left[ 1 : -i : 0 \right] $, we get
      \[
      \varphi \left( T_{(0,0)}
      V \left( (1-zx_0)^2 + (y-i-zy_0)^2 - (zr)^2 \right) \right) 
      \] 
      so the internal polynomial becomes 
      $2x_0 z-2iy_0z +2iy$ which is mapped to
      \[
      V\left( 2x_0 z - 2iy_0 z + 2i (y+i) \right) 
      \] 
      whose projective closure becomes
      \[
      \mathbb{V}\left( 2x_0 z - 2iy_0 z + 2i (y+ix) \right) 
      = \mathbb{V}\left( z (x_0 - iy_0) + i(y+ix) \right) 
      \] 
      (b) In (a) we found that the projective tangent line is not dependent on
      the radius $r$, so $\overline{C_1}$ and $\overline{C_2}$ have the
      same tangent line at $\left[ 1 : i : 0 \right] $ and
      $\left[ 1 : -i : 0 \right] $ and are thus tangent at these points.\\
      \linebreak
      (c) Let
      $f$ be the equation for the circle $C_1$ and
      $g$ the equation for the circle $C_2$. Let
      $F$ and $G$ be the homogenizations. The intersection of $C_1$ and $C_2$ in
      $\mathbb{A}_{\mathbb{C}}^2$ is empty if
      \[
      \sum_{P \in \mathbb{A}_{\mathbb{C}}^2} I_P(f, g) = 0
      \] 
      by (2) in intersection multiplicities. But
      
      \[
      \sum_{P \in \mathbb{A}_{\mathbb{C}}^2} I_P(f, g)  
      = \sum_{P \in \mathbb{P}^2} I_P\left( F,G \right) 
      \] 
      Now, Bezout's theorem gives that
      $\sum_{P \in \mathbb{P}^2}I_P\left( F,G \right) 
      = 4$ if $\mathbb{V}(F,G)$ is finite, i.e. if $F$ and $G$ share no
      components. But if
      $\alpha  \mid F,G$ then
      $\alpha  \mid G-F
      = \left( zr_F \right)^2 - \left( zr_G \right)^2
      = z^2 \left( r_F + r_G \right) (r_F - r_G)$, so
      assume wlog. that $\alpha = z$. However,
      $z \nmid F,G$ since
      $z \nmid 1 + y^2$.\\
      Thus $\sum_{P \in \mathbb{P}^2}I_P(F,G) = 4$. Now,
      we want to calculate $I_{\left[ 1 : i : 0 \right] }(F,G)$. We will
      assume without loss of generality that $P = (0,0)$. Thus
       $F$ becomes $x^2 + y^2 - (zr_F)^2$ and
       $G$ becomes $x^2 + y^2 - (zr_G)^2$. Now, we want to calculate
       \begin{align*}
    I_{(i,0)} (1 + y^2 - (zr_F)^2, 1+ y^2 - (zr_G)^2)
    &= I_{(0,0)} \left( \varphi^{*}\left( 1 + y^2 - (zr_F)^2 \right) ,
    \varphi^{*} \left( 1 + y^2 - (zr_G)^2 \right) \right) \\
    &= I_{(0,0)}\left( 1 + (y-i)^2 - (zr_F)^2, 1+ (y-i)^2 - (zr_G)^2 \right) \\
       &= I_{(0,0)}(1+(y-i)^2 - (zr_F)^2, z^2 (r_F - r_G)(r_F + r_G))\\
       &= I_{(0,0)}\left( y^2 -2iy, z^2 (r_F - r_G)(r_F + r_G) \right) \\
       &= I_{(0,0)}\left( y, z^2 \right) + \underbrace{I_{(0,0)}\left( y-2i,
       z^2 \right)}_{=0, \text{ by (2) since it does not vanish at }(0,0)}\\
       &= 2
       \end{align*}
       Hence the intersection multiplicities are each $2$, so
       in particular, by (2), no other points is a common vanishing of both
       curves.\\
       \linebreak
       \textbf{12:} Let
       \[
       F = a \left( x^2 + y^2 \right) + cxz + eyz + fz^2
       \in k\left[ x,y,z \right] .
       \] 
       Let $x' = x + \alpha z$ and
       $y' = y + \beta z$ and
       $z' = rz$ for constants $\alpha, \beta, r$. Write
       \[
       F = a' \left( x'^2 + y'^2 \right) +c'x'z' + e'y'z' + f' z'^2
       \] 
       for coefficients $a',c',e',f'$ (which are themselves polynomials in
       $a,c,e,f$ ). Prove that the map
       $\left[ a : c : e :f \right] \mapsto 
       \left[ a' : c' : e' : f' \right] $ is a projective change of
       coordinates.\\
       \linebreak
       \textit{Solution:} We must show that the map
       $(a,c,e,f) \mapsto (a',c',e',f')$ is a linear change of coordinates,
       i.e., an invertible linear transformation.\\
       We have
       \begin{align*}
           F
           &= a'\left( \left( x+\alpha z \right)^2 + 
           \left( y+\beta z \right)^2 \right) 
           + c' (x+ \alpha z) rz + e' (y+ \beta z) rz
           +f' (rz)^2\\
           &= a' \left( x^2 + y^2 \right) +
           \left( 2a \alpha + c' r \right) xz +
           \left( 2a \beta + e' r \right) yz
           + \left( a \left( \alpha^2 + \beta^2 \right) + c' \alpha r +
           e' \beta r + f' r^2 \right) z^2 \\
       \end{align*}
       Comparing coefficients, we have
       $a' = a$, so considering the map
       $T  \colon \mathbb{A}^{4} \to \mathbb{A}^{4}$, we have
       $T_1 (a,c,e,f) = a$.\\
       Assume $r \neq 0$.\\
       We have $c = 2 a \alpha + c' r$, so
       $c' = \frac{c}{r} - \frac{2 \alpha}{r} a $, so
       $T_2 \left( a,c,e,f \right) 
       = \frac{1}{r}c - \frac{2 \alpha}{r} a$.\\
       We have $e = 2 a \beta + e' r$, so
       $e' = \frac{e - 2 a \beta}{r}
       = \frac{1}{r} e - \frac{2 \beta}{r} a = T_3 (a,c,e,f)$.\\
       We have $f = a \left( \alpha^2 + \beta^2 \right) 
       + \left( \frac{1}{r} c - \frac{2 \alpha}{r} a \right) \alpha r
       + \left( \frac{1}{r} e - \frac{2 \beta}{r} a \right) \beta r
       + f' r^2$, and again we can isolate $f'$ to be a 
       linear polynomial in $a,c,e$ and $f$, namely
       \[
       f' = \frac{1}{r^2} \left[ f - a \left( \alpha^2 + \beta^2 \right) -
       \left( \frac{1}{r} c - \frac{2 \alpha}{r} a \right) \alpha r 
   - \left( \frac{1}{r} e - \frac{2 \beta}{r}a \right) \beta r \right] 
       \] 
       We have $T_4 \left( 1, 0, 0, 0 \right) 
       = - \frac{\alpha^2 + \beta^2}{r^2}
       + \frac{2 \alpha^2}{r^2} + \frac{2 \beta^2}{r^2}
       = \frac{\alpha^2 + \beta^2}{r^2}$,\\
       $T_4 \left( 0, 1, 0 ,0 \right) 
       = -\frac{1}{r^2}$,\\
       $T_4 \left( 0, 0, 1, 0 \right) 
       = - \frac{\beta}{r^2}$ and
       $T_4 \left( 0,0,0,1 \right) 
       = \frac{1}{r^2}$, so the matrix
       $\left( T_{ij} \right) $ then becomes
       \[
       T =
       \begin{pmatrix} 
           1 & 0 & 0 & 0\\
           -\frac{2 \alpha}{r} & \frac{1}{r} & 0 & 0\\
           - \frac{2 \beta}{r} & 0 & \frac{1}{r} & 0\\
           \frac{\alpha^2 + \beta^2}{r^2} & \frac{-1}{r^w} &
           \frac{- \beta}{r^2} & \frac{1}{r^2}
       \end{pmatrix} 
       \] 
       which has determinant
       $\frac{1}{r^{4}}\neq 0$.\\
       \linebreak
       If $r = 0$, we find
       \[
       F = a' \left( x'^2 + y'^2 \right) 
       = a \left( (x+ \alpha z)^2 + \left( y+ \beta z \right)^2 \right) 
       \] 
       so we find
       that $T$ does not become invertible. . . 
       So it is a projective change of coordinates if
       $r \neq 0$.
       
       

       

       
      


      
      
      
      

     
     
     
     
















\end{document}
