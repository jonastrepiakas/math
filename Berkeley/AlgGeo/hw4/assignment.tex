\documentclass[a4paper]{article}

\usepackage[margin=2.5cm]{geometry}
\usepackage[pdftex]{graphicx}
\usepackage[utf8]{inputenc}
\usepackage[T1]{fontenc}
\usepackage{textcomp}
\usepackage{babel}
\usepackage{amsmath, amssymb}
\usepackage[colorlinks=true,linkcolor=blue]{hyperref}
\usepackage{float}
\usepackage{mathrsfs}
%\usepackage{enumitem}
%% for identity function 1:
%\usepackage{bbm}
%%For category theory diagrams:
%\usepackage{tikz-cd}
%%For code (e.g. python) in latex:
%\usepackage{listings}
%
%Usage: 
%\begin{lstlisting}[language=Python]
%\end{lstlisting}

\newcommand{\incfig}[2][1]{%
\def\svgwidth{#1\columnwidth}
\import{./figures/}{#2.pdf_tex}
}


% figure support
\usepackage{import}
\usepackage{xifthen}
\pdfminorversion=7
\usepackage{pdfpages}
\usepackage{transparent}

\pdfsuppresswarningpagegroup=1

\setlength\parindent{0pt}

\newcommand{\qed}{\tag*{$\blacksquare$}}
\newcommand{\qedwhite}{\hfill \ensuremath{\Box}}

%Inequalities
\newcommand{\cycsum}{\sum_{\mathrm{cyc}}}
\newcommand{\symsum}{\sum_{\mathrm{sym}}}
\newcommand{\cycprod}{\prod_{\mathrm{cyc}}}
\newcommand{\symprod}{\prod_{\mathrm{sym}}}

%Linear Algebra

%Redeclaring Span and image
\DeclareMathOperator{\Span}{span}
\DeclareMathOperator{\Ima}{Im}
\DeclareMathOperator{\diag}{diag}
\DeclareMathOperator{\Ker}{Ker}
\DeclareMathOperator{\ob}{ob}


%Row operations
\newcommand{\elem}[1]{% elementary operations
\xrightarrow{\substack{#1}}%
}

\newcommand{\lelem}[1]{% elementary operations (left alignment)
\xrightarrow{\begin{subarray}{l}#1\end{subarray}}%
}

%SS
\DeclareMathOperator{\supp}{supp}
\DeclareMathOperator{\Var}{Var}

%NT
\DeclareMathOperator{\ord}{ord}

%Alg
\DeclareMathOperator{\Rad}{Rad}
\DeclareMathOperator{\Jac}{Jac}

\DeclareMathAlphabet{\pazocal}{OMS}{zplm}{m}{n}
\newcommand{\unif}{\pazocal{U}}

\begin{document}
    \textbf{1.} \\
    (a) Assume $i (a) = i(b)$, then $\frac{a}{1} = \frac{b}{1}$, so
    $a = a \cdot 1 = b\cdot 1 = b$.\\
    \linebreak
    (b) Let $\beta \left( \frac{a}{b} \right) = 
    \varphi(a) \varphi(b)^{-1}$.\\
    \linebreak
    Clearly, for any $a \in R$,
    $\beta \circ i (a) = \beta\left( \frac{a}{1} \right) 
    = \varphi (a) \varphi(1)^{-1} = \varphi(a)$.\\
    To show that it is well-defined:\\
    if $\frac{a}{b}= \frac{a'}{b'}$, then $ab' = ba'$, so
    $\varphi(a) \varphi(b') = \varphi(b) \varphi(a')$, hence
    $\frac{\varphi (a)}{\varphi(b)} = \frac{\varphi(a')}{\varphi(b')}$ where
    we can divide by $\varphi(b)\varphi(b')$ since
    neither is $0$ as $\varphi(0) = 0$ and $\varphi$ is injective.\\
    \linebreak
    \textbf{2:} \\
    (a) Suppose $\varphi$ is not the zero map.\\
    Let $\varphi(1) = e$. Assume $\varphi(a) = 0$. Then
    $\varphi(1) = \varphi(a a^{-1}) = \varphi(a) \varphi(a^{-1}) = 0$. However,
    then
    $\varphi(r) = \varphi(r) = \varphi (r) \varphi(1) = 0$ for all
    $r$, so $\varphi$ is the zero map. Contradiction. So $\varphi$ is
    injective.\\
    \linebreak
    (b) Assume such a $\varphi$ exists. Firstly we show that there
    exist infinitely many irreducible polynomials in $k\left[ y \right] $.
    Assume only finitely many exist, let these be $f_1, \ldots, f_n$. Then
    $f_1 \cdot \ldots \cdot f_n + 1$ is a polynomial of nonzero degree.\\
    Since this is by assumption reducible, we must have that there
    exists $f_i$ such that $f_i  \mid f_1 \ldots f_n + 1$, but this means
    $f_i  \mid 1$ which is a contradiction since $k$ is an integral domain - so
    degrees don't match up.\\
    \linebreak
    Now if such a $\varphi$ existed, then for any irreducible polynomial
    $f$ in $k\left[ x \right] $, we would have that there exists
    $f' \in k\left[ x_1, \ldots, x_n \right] $ such that
    $\varphi(f') = \frac{1}{f}$. Choose $f$ irreducible such that
    $f$ does not divide any denominator of $\varphi(x_i)$ for any $i$. Then
    $ \frac{1}{f} = k_1 \varphi(x_1) + \ldots + k_n \varphi(x_n)$, but
    multiplying with the product of the denominators of each
    $\varphi(x_n)$, we get that the right hand side is in
    $k\left[ y \right] $ while the left hand side is not. Contradiction.\\
    \linebreak
    \textbf{3:} We claim that
    $k\left[ x \right] / (f) = \left\{k_0 \overline{1} + k_1 \overline{x}+
     \ldots + k_{n-1} \overline{x}^{n-1}  \mid k_i \in k \right\} $.\\
     $(\subset )$ : Let $g$ be any polynomial and assume
     there does not exist a representative in its equivalence class
     in $k\left[ x \right] / (f)$ of the form on the right hand side.
     Then $\overline{g}$ is of degree $m \ge n$. Now let
     $f = a_0 + \ldots + a_n x^{n}$. Then letting
     $\overline{g} = b_{m} \overline{x}^{m} +\ldots + b_0 \overline{1}$, we 
     get 
     $\overline{g} = b_m \overline{x}^{m} + \ldots + b_0 \overline{1}
     - \frac{b_m}{a_n} \overline{f}$ which has degree
     $< m$, contradicitng the assumption on $g$.\\
     \linebreak
     $\left( \supset \right) $ : This is clear.\\
     \linebreak
     (b) We claim that the dimension is
     $\frac{(d+1)(d+2)}{2}$.\\
     \linebreak
     We claim that $k\left[ x,y \right] /I$ is generated
     by the basis
     \begin{align*}
         &1, x, \ldots, x^{d}\\
          &y, yx, \ldots, yx^{d-1}\\
          &\vdots \\
               &y^{d-1}, y^{d-1}x\\
               &y^{d}.
     \end{align*}
     For any $\overline{f} \in k\left[ x,y \right] /I$, if there is a term
     of degree $\ge d$, then it is of the form
     $\sum_{i=0}^{m} a_i x^{i}y^{m-i}$, for $m\ge d$. But then
     this any term in this is of the form
     $a_i x^{i} y^{m-i}$ for $m\ge d$, but then if $i\le d-1$, we have
     $x^{i} y^{m-i} = x^{i} y^{d-i} y^{m-d} = 0$, and if 
     $i\ge d$ then $x^{i} y^{m-i} = x^{d} x^{i-d} y^{m-i} = 0$, so
     \[
     \sum_{i=0}^{m} a_i x^{i}y^{m-i} = 0.
     \] 
     Conversely, any linear combination of the basis is clearly in
     $k\left[ x,y \right] / I$. Thus the result follows since
     the basis is precisely
     $\frac{(d+1) (d+2)}{2}$ elements.\\
     \linebreak
     (c) The number of ways to put $k$ objects into $n$ bins is 
     given by the ball-and-urn formula:
     \[
         \binom{n+k-1}{n-1}.
     \] 
     We can think of the dimension of $I$ as the number of ways to put
     $d$ objects into $n+1$ bins which is
     \[
         \binom{d+n}{n}
     \] 
     We see that for $n = 2$ this corresponds, as we saw in (b), to
     $\frac{(d+2)(d+1)}{2}$.\\
     \linebreak
     \textbf{4:} 
     For points $P_1, P_2$, we can define the polynomials
     $f_1(x) = \frac{x-P_2}{P_1 - P_2}$ and
     $f_2 (x) = \frac{x - P_1}{P_2 - P_1}$. These satisfy the condition.\\
     Fix some $j$.
     Now we thus can find for each $i$ a function $F_{ij}$ s.t. 
     $F_{ij}(P_i) = 0$ and $F_{ij}(P_j) = 1$.\\
     Then $\Pi_{i \neq j} F_{ij}$ vanishes on $P_i$ for $i\neq j$ and
     is $1$ at $P_j$.\\
     \linebreak
     \textbf{5:}\\
     (a) There exist polynomials
     $T_1, \ldots, T_m \in k\left[ x_1, \ldots, x_n \right] $ such that
     $\varphi (P) = \left( T_1(P), \ldots, T_m(P) \right) $ for
     all $P \in X$. Similarly, there exist $S_1, \ldots, S_r \in k\left[ x_1,
     \ldots, x_m \right] $ such that
     $\psi (P) = \left( S_1 (P), \ldots, S_r(P) \right) $ for all
     $P \in Y$. Hence
     $\psi \circ \varphi (P) = 
     \left( S_1 \left( T_1 (P), \ldots, T_m(P) \right) ,
     \ldots, 
 S_r \left( T_1 (P), \ldots, T_m(P) \right) \right) $ for all $P \in X$, and
 since compositions of polynomials is a polynomial, we find that
 $\psi \circ \varphi$ is a polynomial map.\\
 \linebreak
 (b) This is clear as
 $\left( \psi \circ \varphi \right)^{*}(f) (P)
 = f\left( \psi \circ \varphi(P) \right) 
 = \psi^{*} \left( f \left( \varphi \right)  \right) (P)
 = \psi^{*} \left( \varphi^{*} f \right) (P)
 = \psi^{*} \circ \varphi^{*} (f) (P)$.

     


      
     

 













\end{document}
