\documentclass[a4paper]{article}

\usepackage[margin=2.5cm]{geometry}
\usepackage[pdftex]{graphicx}
\usepackage[utf8]{inputenc}
\usepackage[T1]{fontenc}
\usepackage{textcomp}
\usepackage{babel}
\usepackage{amsmath, amssymb}
\usepackage[colorlinks=true,linkcolor=blue]{hyperref}
\usepackage{float}
\usepackage{mathrsfs}
%\usepackage{enumitem}

\newcommand{\incfig}[2][1]{%
\def\svgwidth{#1\columnwidth}
\import{./figures/}{#2.pdf_tex}
}


% figure support
\usepackage{import}
\usepackage{xifthen}
\pdfminorversion=7
\usepackage{pdfpages}
\usepackage{transparent}

\pdfsuppresswarningpagegroup=1

\setlength\parindent{0pt}

\newcommand{\qed}{\tag*{$\blacksquare$}}
\newcommand{\qedwhite}{\hfill \ensuremath{\Box}}

%Inequalities
\newcommand{\cycsum}{\sum_{\mathrm{cyc}}}
\newcommand{\symsum}{\sum_{\mathrm{sym}}}
\newcommand{\cycprod}{\prod_{\mathrm{cyc}}}
\newcommand{\symprod}{\prod_{\mathrm{sym}}}

%Linear Algebra

%Redeclaring Span and image
\DeclareMathOperator{\Span}{span}
\DeclareMathOperator{\Ima}{Im}
\DeclareMathOperator{\diag}{diag}
\DeclareMathOperator{\Ker}{Ker}

%Row operations
\newcommand{\elem}[1]{% elementary operations
\xrightarrow{\substack{#1}}%
}

\newcommand{\lelem}[1]{% elementary operations (left alignment)
\xrightarrow{\begin{subarray}{l}#1\end{subarray}}%
}

%SS
\DeclareMathOperator{\supp}{supp}
\DeclareMathOperator{\Var}{Var}

%NT
\DeclareMathOperator{\ord}{ord}

%Alg
\DeclareMathOperator{\Rad}{Rad}
\DeclareMathOperator{\Jac}{Jac}

\DeclareMathAlphabet{\pazocal}{OMS}{zplm}{m}{n}
\newcommand{\unif}{\pazocal{U}}


\title{Assignment 1}
\author{Jonas Trepiakas - 3039733855 - jtrepiakas@berkeley.edu}
\date{}

\begin{document}
\maketitle
\newpage
    \textbf{1:} \\
    (a) 6.\\
    (b) 9.\\

    \textbf{2:} Any line $L \subset \mathbb{A} $ can be written as a graph $y
    = ax +b$ and is thus the hypersurface defined by
    $y - (ax + b)$. We thus have
    \[
    L \cap C = V\left( y - (ax+b) \right) \cap V(f)
    = V \left( y - (ax+b), f \right) 
    \] 
    So for any pair $(x_1,y_1) \in L \cap C$, we have
    $y_1 = ax_1 + b$ and thus $0 = f(x_1,y_1) = f(x_1, ax_1+b)$, so
    $x_1 \in V\left( f(x, ax+b) \right) $. However, $f(x,ax+b)$ is a
    polynomial in $k[x]$ and as such has either $\mathbb{A}^{1}$ as
    its zero locus or finitely many zeros - in particular, since roots of
    $f(x, ax+b)$ correspond to linear factors and $k[x]$ is an integral domain,
    we have that $f(x, ax+b)$ has at most $d$ roots in $k$ (Dummit and Foote,
    proposition 17). Since 
    $L \not \subset C$, we must have that 
     $f(x,ax+b)$ has at most $d$ roots, and since each $y_i$ is uniquely
     determined by the corresponding $x_i$, there exists at most $d$ roots
     of
     $f(x,y)$ in $L \cap C$.\\
     \linebreak
     \textbf{3:} \\
     (a) $A = \left\{ \left( t, \sin t \right)  \colon t \in \mathbb{R}
     \right\} \subset \mathbb{A}_{\mathbb{R}}^2$ is not an algebraic set.
     Assume for contradiction that $A = V(S)$ for some
     $S \subset k[x,y]$. Let $f$ be any polynomial in $S$.
     Let $L = V\left( y \right) $ - i.e. a line as in problem \textbf{2} with
     $a,b = 0$. Then by problem  \textbf{2} $L \cap V(f)$ is finite, however,
      \[
 \left\{ \left( \pi k,0 \right)  \colon
k \in \mathbb{Z} \right\} =
     \left\{ \left( t, \sin t \right)  \colon
     t \in \mathbb{R} \right\} \cap \left\{ \left( x,0 \right) x
 \in \mathbb{R} \right\} = L \cap A = L \cap \bigcap_{g \in S} V(g)
 \subset L \cap V(f)
     \] 
     and $\left\{ \left( \pi k, 0 \right)  \colon k \in \mathbb{Z} \right\}
     $ is an infinite set hence $L \cap V(f)$ is an infinite set
     contradicting problem $\textbf{2}$.\\
     \linebreak
     (b) Similarly to (a), let $B = \left\{ \left( x,y \right) 
     \in \mathbb{A}_{\mathbb{C}}^2  \colon x= 0, y\neq 0 \right\} $ 
     and $L = V\left( x \right) $. Then $L \not \subset B$ since
     $\left( 0,0 \right)  \not \in B$. \\
     Assume $B = V(S)$ for some $S \subset k[x,y]$ and let $f \in S$.
     Then
     \[
     B = L \cap B = L \cap V(S) = L \cap \bigcap_{g \in S} V(g)
     \subset L \cap V(f).
     \] 
     Since $B$ is an infinite set, so is $L \cap V(f)$, so by contraposition of
     problem \textbf{2}, we find $B$ is not an algebraic set.
     \\
     \linebreak
     (c) Similarly to (a) and (b), let $
     C = \left\{ \left( x,y \right)  \in \mathbb{A}_{\mathbb{R}}^2
      \colon y = \left| x \right| \right\} $ and
      $L = V \left( y-x \right) $. Then $L \not \subset C$ since
      for example  $(-1,-1) \not\in C$ but is in $L$.
      Assume $C = V(S)$ where $S \subset k[x,y]$ and
      let $f \in S$. Then
      \[
      \left\{ (x,y) \in \mathbb{A}_{\mathbb{R}}^2  \colon
      y = x, x\ge 0 \right\} \subset L \cap C
     = L \cap V(S) = L \cap \bigcap_{g \in S} V(g) 
     \subset L \cap V(f)
      \] 
      Since $\left\{ (x,y) \in \mathbb{A}_{\mathbb{R}}^2
       \colon y = x , x\ge 0 \right\} $ is an infinite set, so
       is $L \cap V(f)$, so by contraposition of problem \textbf{2},
       $C$ is not an algebraic set. 
      \\
      \linebreak
      (d) $ D= \left\{ \left( t, t^2, t^3 \right) 
      \in \mathbb{A}_{k}^3  \colon t\in k \right\} $ is an algebraic set
      since $D = V \left( y-x^2, z - x^3 \right) $ :\\
      \linebreak
       $\left( \subset  \right) $ : for any  $(t,t^2,t^3) \in D$, we
       have $(t^2) - (t)^2 = 0$ and
       $(t^3) - (t)^3 = 0$.\\
       $(\supset)$: For an arbitrary  $(x,y,z) \in D$, we have
       $y = x^2$ and $z = x^3$ so $(x,y,z)=(x,x^2,x^3)$ giving
       the inclusion as $x$ ranges over $k$.
     
     
    

\end{document}
