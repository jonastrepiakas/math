\documentclass[a4paper]{article}

\usepackage[margin=2.5cm]{geometry}
\usepackage[pdftex]{graphicx}
\usepackage[utf8]{inputenc}
\usepackage[T1]{fontenc}
\usepackage{textcomp}
\usepackage{babel}
\usepackage{amsmath, amssymb}
\usepackage[colorlinks=true,linkcolor=blue]{hyperref}
\usepackage{float}
\usepackage{mathrsfs}
%\usepackage{enumitem}
%% for identity function 1:
%\usepackage{bbm}
%%For category theory diagrams:
%\usepackage{tikz-cd}
%%For code (e.g. python) in latex:
%\usepackage{listings}
%
%Usage: 
%\begin{lstlisting}[language=Python]
%\end{lstlisting}

\newcommand{\incfig}[2][1]{%
\def\svgwidth{#1\columnwidth}
\import{./figures/}{#2.pdf_tex}
}


% figure support
\usepackage{import}
\usepackage{xifthen}
\pdfminorversion=7
\usepackage{pdfpages}
\usepackage{transparent}

\pdfsuppresswarningpagegroup=1

\setlength\parindent{0pt}

\newcommand{\qed}{\tag*{$\blacksquare$}}
\newcommand{\qedwhite}{\hfill \ensuremath{\Box}}

%Inequalities
\newcommand{\cycsum}{\sum_{\mathrm{cyc}}}
\newcommand{\symsum}{\sum_{\mathrm{sym}}}
\newcommand{\cycprod}{\prod_{\mathrm{cyc}}}
\newcommand{\symprod}{\prod_{\mathrm{sym}}}

%Linear Algebra

%Redeclaring Span and image
\DeclareMathOperator{\Span}{span}
\DeclareMathOperator{\Ima}{Im}
\DeclareMathOperator{\diag}{diag}
\DeclareMathOperator{\Ker}{Ker}
\DeclareMathOperator{\ob}{ob}
\DeclareMathOperator{\ob}{ob}


%Row operations
\newcommand{\elem}[1]{% elementary operations
\xrightarrow{\substack{#1}}%
}

\newcommand{\lelem}[1]{% elementary operations (left alignment)
\xrightarrow{\begin{subarray}{l}#1\end{subarray}}%
}

%SS
\DeclareMathOperator{\supp}{supp}
\DeclareMathOperator{\Var}{Var}

%NT
\DeclareMathOperator{\ord}{ord}

%Alg
\DeclareMathOperator{\Rad}{Rad}
\DeclareMathOperator{\Jac}{Jac}

\DeclareMathAlphabet{\pazocal}{OMS}{zplm}{m}{n}
\newcommand{\unif}{\pazocal{U}}

\begin{document}
   \textbf{1.30:}\\
   (a) In $\mathbb{R}$, $x^2 + y^2 + 1 = 0$ has no solutions, so
   $V \left( x^2 + y^2 + 1 \right) = \varnothing$ and hence
   $I \left( V\left( x^2 + y^2 +1 \right)  \right) = I\left( \varnothing \right) 
   = k\left[ x, y \right] = (1)$.\\
   \linebreak
   (b) Let $V$ be an algebraic subset of $\mathbb{A}^2 \left( \mathbb{R}
   \right) $.
   By corollary $2$, we have that the algebraic sets are precisely
   $\mathbb{A}^2 \left( \mathbb{R} \right) , \varnothing$, points and
   irreducible plane curves $V(F)$ where $F$ is an irreducible polynomial.
   Since $\mathbb{A}^2 \left( \mathbb{R} \right) = V(0), \varnothing = V(1)$
   and
   for any point $\left( a,b \right) \in \mathbb{A}^2 (\mathbb{R})$,
    $(a,b) = V\left( (x-a)^2 + (y-b)^2 \right) $. Thus for any collection
    of points $\left\{ \left( a_1, b_1 \right) ,\ldots
    , \left( a_n, b_n \right) \right\} $, it is the zero locus
    of 
    \[\left( (x-a_1)^2 + (y-b_1)^2 \right)\cdot  \left( (x-a_2)^2 + (y-b_2)^2 \right) 
    \cdots \left( (x-a_n)^2 + (y-b_n)^2 \right) \]


\textbf{2.14:}\\
(b) Assume $V = V(F_1)$.\\
Now $V^{T} = V \left( I(V(F_1))^{T} \right) 
= V\left( F_1 \circ T \right)  = V\left( F_1 (T_1, \ldots, T_n) \right) $.
Write $F_1 = \sum a_i x_i + a_0 $. We can
let $T_i = \frac{1}{a_i} x_i - \frac{a_0}{l a_i} $ where
$l$ is the amount of $a_i$ that are nonzero - if $a_i = 0$, let
$T_i = x_i$. Then
$F_1 \left( T_1, \ldots, T_n \right) =
\sum_{a_i \neq 0} x_i $. \\
\linebreak











\textbf{Practice for intersection multiplicities:}\\
\linebreak
Consider the case $P = (0,0)$ and 
$f= (x^2 + y^2)^3 - 4x^2 y^2$ and
$g = (x^2 + y^2)^3 +3x^2 y -y^3$. Find
$I_P (f,g)$.\\
\linebreak
\textit{Solution:} We follow the algorithm:\\
(1) $P$ is indeed $(0,0)$.\\
(2) $f$ and $g$ have no common factors which can be checked by computer.\\
(3) Indeed $(0,0) \in V(f) \cap V(g)$.\\
(4) We have
\begin{align*}
    TC_P V(f) &= V(-4x^2 y^2) = V(x) \cup V(y)\\
    TC_P V(g) &= V(3x^2 y - y^3)
    = V(y) \cup V(\sqrt{3} x-y) \cup V\left( \sqrt{3} x + y \right).
\end{align*}
So they have the line $V(y)$ in common.\\
(5) Since the line is already $V(y)$, we find
$f(x,0) = x^{6}$ and $g(x,0) = x^{6}$.\\
\linebreak
(6) As $r \neq 0$, we let $h = f-g = y^3 - 3x^2 y - 4x^2 y^2 
= y \left( y^2 - 3x^2 - 4x^2 y \right) $. Then
$I_P(f,g) = I_P(g,h) = I_P(f,h)$. We will see that
$I_P(f,h)$ is easier to compute.\\
\linebreak
We repeat from step 2:\\
(2) Again,  $h$ and $g$ have no common divisors.\\
(3) $P$ is still a common vanishing point.\\
(4) We now have
\[
TC_P V(h) = V\left( y^3 - 3x^2 y \right) 
= V(y) \cup V\left( y^2 - 3x^2 \right) 
= V(y) \cup V(y - \sqrt{3} x) \cup V\left( y + \sqrt{3} x \right).
\] 
Again $V(y)$ is a common tangent cone line.\\
(5) We now have $h(x,0) = 0$ and $g(x,0) = x^6$.\\
(6) We have
\[
I_P(h,g) = I_P(y, g) + I_P (y^2 - 3x^2 - 4x^2 y, g)
\] 
Now since  $g = x^{6} + y B$, we find
$I_P(y,g) = 6$.\\
Now let $h_2 = y^2 - 3x^2 - 4x^2 y$.\\
Again (2) and (3) are satisfied, and
\[
TC_P V(h_2) = V\left( y^2 - 3x^2 \right) 
= V(y - \sqrt{3} x) \cup V(y + \sqrt{3} x).
\] 
Here we see the problem. These tangent cone lines are also tangent cone lines
for $g$, however, they are not for $f$.\\
Now, does it matter that we chose $f$ and not $g$?\\
No, we only care about the vanishings in each case which remain in both cases.
(5) is still satisfied the same, so we find
 \[
I_P (h,g) = I_P(h,f) = 6 + I_P (y^2 - 3x^2 - 4x^2 y, f)
= 6 + mult_P (h_2) mult_P(f) = 6 + 2\cdot 4 = 14
\] 


\textbf{To show that single points are projective algebraic sets:} 
We can write the point with last coordinate $1$ if it does not live in
$\mathbb{V}(z)$: $\left[ a_1 : \ldots : a_n : 1 \right] \in \mathbb{P}^{n}$.\\
The homogeneous ideal
$\left( x_n - a_n x_{n+1}, x_{n-1}-a_{n-1} x_{n+1}, \ldots
, x_1 - a_1 x_{n+1} \right) $
vanishes at $\left[ a_1 : \ldots : a_n : 1 \right] $ precisely.\\
If last coordinate is $0$, we can proceed similarly.\\
\linebreak
\begin{align*}
    v_{2,2}^{-1} \left( \mathbb{V}\left( a_1 x_1 + \ldots + 
    a_6 x_{6} \right)  \right) 
    &= \left\{ \left[ x:y:z \right]  \colon
    a_1 x^2 + a_2 xy + \ldots + a_6 + z^2 = 0 \right\} \\
    &= \mathbb{V} \left( a_1 x^2 + \ldots + a_6 z^{2} \right) 
\end{align*}




















































\end{document}
