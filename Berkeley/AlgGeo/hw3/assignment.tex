\documentclass[a4paper]{article}

\usepackage[margin=2.5cm]{geometry}
\usepackage[pdftex]{graphicx}
\usepackage[utf8]{inputenc}
\usepackage[T1]{fontenc}
\usepackage{textcomp}
\usepackage{babel}
\usepackage{amsmath, amssymb}
\usepackage[colorlinks=true,linkcolor=blue]{hyperref}
\usepackage{float}
\usepackage{mathrsfs}
%\usepackage{enumitem}
%% for identity function 1:
%\usepackage{bbm}
%%For category theory diagrams:
%\usepackage{tikz-cd}
%%For code (e.g. python) in latex:
%\usepackage{listings}
%
%Usage: 
%\begin{lstlisting}[language=Python]
%\end{lstlisting}

\newcommand{\incfig}[2][1]{%
\def\svgwidth{#1\columnwidth}
\import{./figures/}{#2.pdf_tex}
}


% figure support
\usepackage{import}
\usepackage{xifthen}
\pdfminorversion=7
\usepackage{pdfpages}
\usepackage{transparent}

\pdfsuppresswarningpagegroup=1

\setlength\parindent{0pt}

\newcommand{\qed}{\tag*{$\blacksquare$}}
\newcommand{\qedwhite}{\hfill \ensuremath{\Box}}

%Inequalities
\newcommand{\cycsum}{\sum_{\mathrm{cyc}}}
\newcommand{\symsum}{\sum_{\mathrm{sym}}}
\newcommand{\cycprod}{\prod_{\mathrm{cyc}}}
\newcommand{\symprod}{\prod_{\mathrm{sym}}}

%Linear Algebra

%Redeclaring Span and image
\DeclareMathOperator{\Span}{span}
\DeclareMathOperator{\Ima}{Im}
\DeclareMathOperator{\diag}{diag}
\DeclareMathOperator{\Ker}{Ker}
\DeclareMathOperator{\ob}{ob}


%Row operations
\newcommand{\elem}[1]{% elementary operations
\xrightarrow{\substack{#1}}%
}

\newcommand{\lelem}[1]{% elementary operations (left alignment)
\xrightarrow{\begin{subarray}{l}#1\end{subarray}}%
}

%SS
\DeclareMathOperator{\supp}{supp}
\DeclareMathOperator{\Var}{Var}

%NT
\DeclareMathOperator{\ord}{ord}

%Alg
\DeclareMathOperator{\Rad}{Rad}
\DeclareMathOperator{\Jac}{Jac}

\DeclareMathAlphabet{\pazocal}{OMS}{zplm}{m}{n}
\newcommand{\unif}{\pazocal{U}}

\begin{document}
    \textbf{1:}\\
    (a) This is the fourth isomorphism theorem for rings which can be proven as
    follows:
    by the fourth isomorphism theorem for groups, we have that there
    is a bijective correspondence
    \[
        \left\{ \text{additive subgroups of } R \text{ containing } I \right\} 
        \to \left\{ \text{additive subgroups of } R /I \right\} 
    \] 
    Now let $J$ be an additive subgroup of $R$ containing $I$.\\
    \linebreak
    We first show that $J$ is a ring in $R$ if and only if $J /I$ is a ring in
    $R /I$.\\
    Let $a,b \in J$. Then we have $\pi (a) \pi(b) = \pi(ab)$ and this is in
    $J /I$ if and only if $ab \in J$.\\
    \linebreak
    We have our correspondence if we can show that $J$ is an ideal if and only
    if $J /I$ is an ideal.\\
    
    Let $r \in R$ and $j \in J$.
    In $R/I$ we have $\pi (r) \pi(j) = \pi(rj)$ and $\pi(j) \pi(r) = \pi(jr)$,
    and this is in $J /I$ if and only if $rj$ and $jr$ are in  $J$.
    Thus $J$ is an ideal containing $I$ if and only if $J /I$ is an ideal.\\
    \linebreak
    

    (b) The correspondence in (a) was given by the canonical homomorphism
    $\pi$.\\
    Now let $J$ be a radical ideal in $R$ containing $I$.
    Assume $\pi(r^{k}) = \pi (r)^{k} \in \pi(J)$. Then
    $r^{k} \in \pi^{-1} \left( \pi (J) \right) = J$, so $r \in J$ as it is
    radical, and thus $\pi(r) \in \pi(J)$, so $\pi(J)$ is radical.\\
    Similarly, for any radical ideal $\pi (J)$ in $R /I$, we have
    that if $r^{k} \in J$ then $\pi(r)^{k} = \pi(r^{k}) \in  \pi(J)$ so
    $\pi(r) \in \pi(J)$ as it is a radical ideal, and thus $r \in \pi^{-1} \left(
    \pi (J) \right) = J$, so $J$ is a radical ideal.\\
    We thus find that the bijective correspondence induces a bijection between
    radical ideal in $R$ containing $I$, and radical ideals in $R /I$.\\
    \linebreak
    (c) Let $M \subset R$ be a maximal ideal containing $I$. 
    Assume that there exists an ideal $\pi(S)$ in $R /I$ such that
    $\pi(M) \subset \pi(S) \subset R /I$ - where $S$ is an ideal in $R$ 
    containing $I$ which we can assume by (a).
    Then applying $\pi^{-1}$, we have
    $M \subset S \subset R$.
    Therefore  $S = M$ or $S = R$, so $\pi(S) = \pi(M)$ or $\pi(S) = R /I $. 
    So $\pi (M)$ is a maximal ideal.\\
    \linebreak
    Conversely, let $\pi(M)$ be a maximal ideal in $R /I$ with $I \subset M$ 
    and $M$ ideal.
    Assume there exists an ideal $S$ in $R$ containing $I$ such that
    $M \subset S \subset R$. Then
    $\pi(M) \subset \pi(S) \subset R /I$, and $\pi(S)$ is an ideal, so
    it must either be $\pi(M)$ or $R /I$. By $A$ we then get $S$ is $M$ or $R$,
    so
    $M$ is a maximal ideal.\\
    \linebreak
    Thus the bijection in (a) induces a bijection between maximal ideals in $R$ 
    containing $I$ and maximal ideals in $R /I$.\\
    \linebreak
    \textbf{2.}\\
    (a) We assume $I \neq (1) = k \left[ x_1, \ldots, x_n \right] $ - i.e. $I$ 
    is proper.\\
    According to a lemma from lecture notes 4, we have that $I$ is contained in
    a maximal ideal - so the below intersection over maximal ideals containing
    $I$ is nonempty.\\
    Assume $I \subset k\left[ x_1, \ldots, x_n \right] $ is a radical
    ideal with $k$ algebraically closed. Then
     \[
         I \subset \bigcap_{I \subset A, A \text{ max ideal}} A = \mathcal{A}
     \] 
     Let $f \in  \bigcap_{I \subset A, A \text{ max ideal}} A$.\\
     Since $k$ is algebraically closed, we get by weak Nullstellensatz 2 that
     all the maximal ideals $A$ containing $I$ are of the form
     $\left( x_1 - a_1, \ldots, x_n - a_n \right) = I \left( a_1, \ldots, a_n
     \right)  $.
     Let now $\alpha \in V(I)$ and assume $\alpha \not\in V(f)$. 
     Write $\alpha = \left( \alpha_1, \ldots, \alpha_n \right) $. Then
     $I \subset \left( x_1 -\alpha_1, \ldots, x_n - \alpha_n \right) $ which is
     a maximal set, hence
     $f \in \bigcap_{I \subset A, A \text{ max ideal}} A \implies f
     \in \left( x_1 - \alpha_1, \ldots, x_n - \alpha_n \right) $ but then
      $\alpha \in  V(f)$, contradiction. Thus
      $V(I) \subset \bigcap_{f \in \mathcal{A}} V(f)
      = V \left( \bigcup_{f \in \mathcal{A}} (f)  \right) 
      = V\left( \mathcal{A} \right) $ so 
       $$\mathcal{A} \subset \sqrt{\mathcal{A}} = I\left( V\left( \mathcal{A} \right)  \right) 
       \subset I \left( V\left( I \right)  \right) = I$$

       Thus $I = \mathcal{A}$.\\
       \linebreak
       (b)
       
       We find $V(I)$: if $y^3 - y = 0$ we have $y=0$ or $y = \pm 1$. If
       $y = 0$ then $x^2 - 2xy^{4} + y^{6} = x^2 = 0$ if and only if $x = 0$.\\
       If $y = \pm 1$, then $x^2 - 2x y^{4} + y^{6} = x^2 - 2x + 1 = 
       (x-1)^2 = 0$ if and only if 
       $x= 1$.\\
       Thus
       $V(I) = \left\{ (0,0), (1,1), (-1,1) \right\} $ and so
       $\sqrt{I} = I\left( V\left( I \right)  \right) 
       = I\left( \left\{ (0,0), (1,1), (-1,1) \right\}  \right) 
       = I\left( (0,0) \right) \cap I\left( (1,1) \right) \cap I\left( (-1,1) \right) 
       = \left( x,y \right) \cap \left( x-1, y-1 \right) \cap (x+1, y-1)$. Each
       of
       these is a maximal ideal by weak Nullstellensatz 2 as they are one point
       sets and $\mathbb{C}$ is algebraically closed -
       or alternatively as they are the kernels of the evaluation maps.\\
       \linebreak
       \textbf{3:} Assume $x^2 - yz = 0$ and $xz -x = 0$. Then $x = 0$ or
       $z=1$.\\
       If $x = 0$, we have $yz = 0$ which implies $y=0$ or $z=0$ as
       $ \mathbb{C}$ is an integral domain.\\
       If $z = 1$ we get $x^2 = y$.\\
       Conversely, each of these solution sets satisfies the system. Thus
       we find
       \[
       V\left( x^2 - yz, xz - x \right)  
       = V(y) \cup V(z) \cup V(x^2 - y)
       \] 
      Now $I \left( V\left( y \right)  \right) 
      = \sqrt{(y)} = (y)$ which is a prime ideal since
      \[
      k\left[ x, y, z \right] /(y) \cong k\left[ x,z \right] 
      \] 
      which is a integral domain and hence $y$ is a prime ideal and thus also
      $ \sqrt{(y)} = (y)$.
      Interchanging the roles of $y$ and $z$, we find that
      $(z)$ is also a prime ideal. It follows from proposition 1 in section
      5 that
      $V(y), V(z)$ are irreducible.\\
      Now, let $\varphi \colon k\left[ x,y,z \right] \to k\left[ x,z \right] $ 
      be the map sending $y \to x^2$. It has kernel $(x^2 -y)$, and
      is trivially surjective as any $f\in k\left[ x,z \right] $ 
      is of the form $\sum_{i,j} c_{i,j}x^{i}z^{j}$ and 
      $\varphi\left( \sum_{i,j} c_{i,j}x^{i}z^{j} \right) 
      =  \sum_{i,j} c_{i,j}x^{i}z^{j}$. Thus
      \[
      k\left[ x,y,z \right] / (x^2 - y) \cong k\left[ x,z \right] 
      \] 
      which is an integral domain and thus $(x^2 - y)$ is prime.
      Therefore $\sqrt{\left( x^2 -y \right) } = (x^2 - y)$ and
      $V(x^2 - y)$ is irreducible by proposition 1 section 5 since
      $I\left( V\left( x^2 - y \right)  \right) = \sqrt{\left( x^2 - y \right) } 
      = \left( x^2 - y \right) $ is prime.\\
      \linebreak
      \textbf{4:} \\
      (a) Let $\mathcal{A}$ be the set of elements in $L$ that are algebraic
      over $k$. Since for any $\alpha \in k$, $\alpha$ is the root of
      $x- \alpha \in k\left[ x \right] $, we have $k \subset \mathcal{A}$ and
      thus also $0,1 \in \mathcal{A}$.\\
      If $\alpha, \beta \in \mathcal{A}$, then there exist
      $f,g \in k[x]$ such that $f(\alpha) = 0 = g(\beta)$.\\
      By the corollary in section 9, we have that
      the elements of $L$ that are algebraic over $k$ form a subring
      of $L$ containing $k$. To show that this subring is, in fact, a subfield,
      we must show that for any $\alpha$ algebraic over $k$, $\alpha^{-1}$ is
      also algebraic over $k$.\\
      \linebreak
      We use the hint: suppose $v \neq 0$ is algebraic over $k$. Then there
      exists $f \in k\left[ x \right] $, say $f(x) = \sum_{i=1}^{n} a_i x^{i}$,
      such that $f(v) = 0$ - we can assume that
      $a_n = 1$ as $a_n \in k$ so we can divide out by $a_n^{-1}$ since $k$ is
      a field. Thus assume $f$ is monic.\\
      In particular, this means that
      $v^{n} + a_{n-1} v^{n-1} + \ldots + a_n = 0$. If $a_0 = 0$, then
      $v \left( v^{n-1} + \ldots + a_1 \right) = 0 $, so since $v$ is not
      a zero divisor in $L$, we have $v^{n-1} + \ldots + a_1 = 0$. Thus we have
      reduced the degree once. We can continue this if $a_1 = 0$ until we get
      some $a_i \neq 0, i>1$ - this must eventually occur as after at $n-1$ 
      steps, we would get $v + a_{n-1} = 0$ and thus $a_{n-1} \neq 0$ as $v\neq
      0$. Assume thus without loss of generality that
      $a_n \neq 0$. Then $v \left( v^{n-1} + \ldots + a_1 \right) = -a_n$. Then
      dividing through by $v^{-n}(-a_n)^{-1}$ we get
      \[
          (-a_n)^{-1} + \ldots + (-a_n)^{-1}a_1 v^{n-1} + v^{-n} = 0
      \]
      so $v^{-1}$ is the root of a monic polynomial in  $k[x]$ and thus
      algebraic over $k$.\\
      \linebreak
      (b) We must show that any nonzero element of $R$ has an inverse.
      Let $r \in R$ be nonzero. Since $L$ is a finite extension of $k$ it is also an
      algebraic extension by the claim in lecture notes 5. Hence there exists
      a monic polynomial  $f \in k\left[ x \right] $ such that $f(\frac{1}{r}) = 0$. Let
      $f (x) = \sum_{i\le n} a_i x^{i}$, so
      $0 = f\left( \frac{1}{r} \right) = \sum_{i\le n} a_i r^{-i}$. Multiplying
      by $r^{n-1}$, we get $\frac{1}{r} =
      - \frac{1}{a_n} \sum_{i\le n-1} a_i r^{n-1-i} \in R$. Hence each $r \in
      R$ has
      an inverse, so $R$ is a field.\\
      \linebreak
      \textbf{5:} If  $\alpha \in L'$ is algebraic over $L$,
      then there exist $c_i \in L$ such that\\
      $a^{n} = c_0 + c_1 \alpha
      + \ldots + c_{n-1}\alpha^{n-1}$.
       Let
      $ S = k \left[ c_0, c_1, \ldots, c_{n-1}, \alpha \right] = \left\{ 
      \sum a_{(i)} c_0^{i_0} \ldots c_{n-1}^{i_{n-1}} \alpha^{i_{n}} \right\}
      $.
     Define the homomorphism $\varphi  \colon k\left[ x_0, \ldots, x_{n-1},
      x_n \right] 
      \to S$ by sending $x_i \to c_i$ and $x_n \to \alpha$. Then
      $S$ a subfield of $L'$ that is $k$ ring-finite. By Zariski, $S$ is
      module-finite and hence algebraic over $k$. Thus there exists
      $f \in k[x]$ such that $f\left( \alpha \right) = 0$\\
      \linebreak

      Alternative, we can show it by repeated uses of proposition 3, section 9:
      \\
      since $c_0 ,\ldots, c_n$ are in $L$, they
      are algebraic, so 
      $S' = k\left[ c_0, \ldots, c_n \right] $ is
      modulo finite by repeated applications of proposition 3 with
      problem 1.45.(a) - transitivity of module-finiteness.\\
      Then $S = S'\left[ \alpha \right] $ is
      module finite over $S'$ by the first part of the problem.
      Since $k\left[ \alpha \right] \subset S$, we have
      that $k\left[ \alpha \right] $ is finite and thus
      $\alpha$ is algebraic over $k$.
     

    
























\end{document}
