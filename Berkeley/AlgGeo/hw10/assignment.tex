\documentclass[a4paper]{article}

\usepackage[margin=2.5cm]{geometry}
\usepackage[pdftex]{graphicx}
\usepackage[utf8]{inputenc}
\usepackage[T1]{fontenc}
\usepackage{textcomp}
\usepackage{babel}
\usepackage{amsmath, amssymb}
\usepackage[colorlinks=true,linkcolor=blue]{hyperref}
\usepackage{float}
\usepackage{mathrsfs}
%\usepackage{enumitem}
%% for identity function 1:
\usepackage{bbm}
%%For category theory diagrams:
%\usepackage{tikz-cd}
%%For code (e.g. python) in latex:
%\usepackage{listings}
%
%Usage: 
%\begin{lstlisting}[language=Python]
%\end{lstlisting}

\newcommand{\incfig}[2][1]{%
\def\svgwidth{#1\columnwidth}
\import{./figures/}{#2.pdf_tex}
}


% figure support
\usepackage{import}
\usepackage{xifthen}
\pdfminorversion=7
\usepackage{pdfpages}
\usepackage{transparent}

\pdfsuppresswarningpagegroup=1

\setlength\parindent{0pt}

\newcommand{\qed}{\tag*{$\blacksquare$}}
\newcommand{\qedwhite}{\hfill \ensuremath{\Box}}

%Inequalities
\newcommand{\cycsum}{\sum_{\mathrm{cyc}}}
\newcommand{\symsum}{\sum_{\mathrm{sym}}}
\newcommand{\cycprod}{\prod_{\mathrm{cyc}}}
\newcommand{\symprod}{\prod_{\mathrm{sym}}}

%Linear Algebra

\DeclareMathOperator{\Span}{span}
\DeclareMathOperator{\Ima}{Im}
\DeclareMathOperator{\range}{range}
\DeclareMathOperator{\diag}{diag}
\DeclareMathOperator{\Ker}{Ker}
\DeclareMathOperator{\ob}{ob}
\DeclareMathOperator{\Hom}{Hom}
\DeclareMathOperator{\sk}{sk}


%Row operations
\newcommand{\elem}[1]{% elementary operations
\xrightarrow{\substack{#1}}%
}

\newcommand{\lelem}[1]{% elementary operations (left alignment)
\xrightarrow{\begin{subarray}{l}#1\end{subarray}}%
}

%SS
\DeclareMathOperator{\supp}{supp}
\DeclareMathOperator{\Var}{Var}

%NT
\DeclareMathOperator{\ord}{ord}

%Alg
\DeclareMathOperator{\Rad}{Rad}
\DeclareMathOperator{\Jac}{Jac}

\DeclareMathAlphabet{\pazocal}{OMS}{zplm}{m}{n}
\newcommand{\unif}{\pazocal{U}}

\begin{document}
    \textbf{1:}\\
    (a) 
    Suppose $J \subset k\left[ x_1, \ldots, x_{n+1} \right] $ is an ideal
    and that $\left( x_1, \ldots, x_{n+1} \right) 
    \subset \sqrt{J} $. Then by definition of radical ideals,
    for each $i \in \left\{ 1, \ldots, n+1 \right\} $,
    there exists $m_i \in \mathbb{N}$ such that
    $x_{i}^{m_i} \in J$. Then letting $m = \max \left\{ m_1, \ldots, m_{n+1}
    \right\} $, we have
    \[
        \left( x_1 , \ldots, x_{n+1} \right)^{(n+1)m}
        = \left\{ x_1^{k_1} x_2^{k_2}\ldots x_{n+1}^{k_{n+1}} \mid 
        k_1 + \ldots + k_{n+1} = (n+1)m,
    k_i \ge 0\right\}.
    \] 
    Let
    $x = x_1^{k_1} x_2^{k_2} \ldots x_{n+1}^{k_{n+1}} \in 
    \left( x_1, \ldots, x_{n+1} \right)^{(n+1)m}$ be an arbitrary element.\\
    By the pigeonhole principle, there must exist a 
    $k_i$ such that $k_i \ge m \ge m_i$, so
    $x_i^{k_i} \in J$, and since $J$ is a double-sided ideal,
    $x_1^{k_1}\ldots x_i^{k_i} \ldots x_{n+1}^{k_{n+1}} \in J$.\\
    Since $x$ was arbitrary, we get
    $(x_1, \ldots, x_{n+1})^{(n+1)m} \subset J$.\\
    Thus $(n+1)m = N$ works as an  $N$.\\
    \linebreak
    (b) $X$ is a projective algebraic set if
    there exists a set of homogeneous polynomials $S \subset k\left[ x_1,
    \ldots, x_{n+1} \right] $ such that
    $\mathbb{V}(S) = X$.\\
    \linebreak
    We have that
    $\mathbb{I} (X) = I\left( C(X) \right) $ is prime if
    and only if $C(X)$ is irreducible. Now, suppose $X$ is reducible as
    $X = X_1 \cup X_2$. So there
    exists sets of homogeneous polynomials
    $T,R \subset k\left[ x_1,\ldots, x_{n+1} \right] $ such that
    $X_1 = \mathbb{V}(T)$ and $X_2 = \mathbb{V}(R)$. We claim
    $C(X_1 \cup X_2) = C(X_1) \cup C(X_2)$.\\
    \linebreak
    We have $ (0,\ldots, 0) \neq (x_1, \ldots, x_{n+1}) \in C(X_1 \cup X_2)$ if
    and only if
    $\left[ x_1  : \ldots : x_{n+1} \right] \in X = X_1 \cup X_2$ if and only
    if
    $\left[ x_1 : \ldots : x_{n+1} \right] \in X_1$
    or $\left[ x_1 : \ldots : x_{n+1} \right] \in X_2$ if and only if
    $(x_1, \ldots, x_{n+1}) \in C(X_1)$ or
    $(x_1 ,\ldots, x_{n+1}) \in C(X_2)$.\\
    If $(0,\ldots,0) = (x_1, \ldots, x_{n+1})$, then
    $ (x_1,\ldots,x_{n+1)} \in C(X_1 \cup X_2), C(X_1), C(X_2)$ by
    definition.\\
    Now, $C(X) 
    = \left\{ (x_1, \ldots, x_{n+1})  \colon
    \left[ x_1: \ldots : x_{n+1} \right] 
\in X = \mathbb{V}(S) \lor (x_1, \ldots, x_{n+1}) = (0,\ldots,0)\right\} 
= V(S)$ which is an algebraic set, and similarly
$C(X_1) = V(T)$ and $C(X_2) = V(R)$ are algebraic sets.\\
Thus we get that
$V(S) = C(X) = C(X_1 \cup X_2) = C(X_1 ) \cup C(X_2)
= V(T) \cup V(R)$, and then if $X$ is irreducible, then
$V(S)$ is irreducible, so $V(S) = V(T)$ or $V(S) = V(R)$. But then
$C(X) = V(S) = V(T) = C(X_1)$ or
$C(X) = V(S) = V(R)= C(X_2)$, so
$C(X)$ was not irreducible and hence
$\mathbb{I}(X) = I\left( C(X) \right) $ is not prime by the affine case.\\
\linebreak
Conversely, if $\mathbb{I}(X) = I(C(X))$ is prime then
$C(X)$ is, as we showed above, an irreducible algebraic set. However, if
$X = X_1 \cup X_2$ with $X_1$ and $X_2$ algebraic sets equaling
$V(T)$ and $V(R)$ as above, respectively. Then
$C(X) = C(X_1) \cup C(X_2)$, so either
$V(T) = V(S)$ or $V(R) = V(S)$, and then we again get
$C(X) = C(X_1)$ or $C(X) = C(X_2)$. Projectivizing this last bit gives
$X = X_1$
 or $X = X_2$, showing that $X$ is irreducible.\\
 \linebreak
 \textbf{2:}\\
 (a) Suppose $X$ is closed in $\mathbb{P}^{n}$, so by definition it is
 a projective algebraic set, i.e. there exists
 a subset of homogeneous polynomials $S \subset k\left[ x_1, \ldots, x_{n+1} \right] $
 such that
 $X = \mathbb{V}(S)$. Define
 \[
 S_i = \left\{ 
 f \in k\left[ x_1, \ldots, x_n \right]  \mid 
 \exists g \in S  \colon
 f(x_1, \ldots, x_n) =
 g(x_1, \ldots, x_{i-1}, 1, x_{i+1},\ldots, x_n), 
 \forall (x_1,\ldots, x_n) \in \mathbb{A}^{n}
\right\}. \]
Then we claim
$X \cap U_i = V(S_i) \subset \mathbb{A}^{n}$.
Now, if $(a_1, \ldots, a_n) \in V(S_i)$ then
for all  $g \in S$ we have
$g \left( a_1, \ldots, a_{i-1},1,a_{i+1},\ldots, a_n \right) = 0$, so
since $g$ is homogeneous,
$\left[ a_1: \ldots : a_{i-1}: 1: a_{i+1}: \ldots a_n \right] 
\in \mathbb{V}(g)$. Hence
$\left[ a_1: \ldots : 1 : \ldots : a_n \right] 
\in \bigcap_{g \in S} \mathbb{V}(g) 
= \mathbb{V}\left( S \right) =X  $, so
$(a_1, \ldots, a_n) \in \mathbb{V}(S) \cap U_i$.\\
\linebreak
Conversely, suppose $(a_1, \ldots, a_n) \in X \cap U_i$. Then
for all $g \in S$, 
$g \left( a_1, \ldots, a_{i-1},1,a_{i+1},\ldots a_n \right) =0$, so
by definition of $S_i$, $(a_1, \ldots, a_n) \in V(S_i)$.\\
\linebreak
Hence $X \cap U_i = V(S_i) \subset \mathbb{A}^{n}$ which is an algebraic
set in $\mathbb{A}^{n}$ and thus closed in the Zariski topology
on $\mathbb{A}^{n} \cong U_i$ for all $i$.\\
\linebreak
(b) If $W \subset U_i$ is open in the Zariski topology on
$U_i \cong \mathbb{A}^{n}$, then
$\mathbb{A}^{n}- (W \cap U_i)$ is closed, so 
there exist polynomials
$S \subset k\left[ x_1, \ldots, x_n \right] $ such that
$\mathbb{A}^{n} - (W \cap U_i) = V(S)$. Now define

\begin{align*}
S_i = \left\{ 
f \text{ homogeneous} \in k\left[ x_1, \ldots, x_{n+1} \right] 
 \mid  \exists g \in k\left[ x_1,\ldots,x_n \right]
\in S  
\colon 
f\left( x_1, \ldots, x_{i-1},1,x_{i+1},\ldots , x_{n+1} \right)
= g(x_1, \ldots, x_n)\right\}.
\end{align*}
Since $W \subset U_i$, $S_i$ is not empty, and hence
$\mathbb{P}^{n} - W = \mathbb{V}(S_i)$, so $W$ is open in $\mathbb{P}^{n}$.\\
\linebreak
(c) We show that $\mathbb{P}^{n} - X = \bigcup_{i} U_i - \left( U_i \cap
X \right) $.\\
\linebreak
( $\subset $ ):  If $\left[ x_1 : \ldots :x_{n+1} \right] 
\in \mathbb{P}^{n} - X =
\bigcup_{i} (U_i) - X$, then there exists
a $U_i$ such that $\left[ x_1: \ldots :x_{n+1} \right] \in U_i$ and
as it is not in  $X$, $\left[ x_1 : \ldots : x_{n+1} \right] 
\not\in U_i \cap X$, so
$\left[ x_1  :\ldots: x_{n+1} \right] 
\in U_i - \left( U_i \cap X \right) 
\subset \bigcup_{i} U_i - \left( U_i \cap X \right) $.\\
Conversely, if $\left[ x_1:\ldots:x_{n+1} \right] 
\in \bigcup_{i} U_i - \left( U_i \cap X \right) $ then
there exists $i$ such that
$\left[ x_1: \ldots : x_{n+1} \right] 
\in U_i - \left( U_i \cap X \right) $. Thus
$\left[ x_1: \ldots :x_{n+1} \right] \not\in  X$ and
$\left[ x_1: \ldots: x_{n+1} \right] 
\in U_i \subset \bigcup_{i} U_i = \mathbb{P}^{n}$, so
$\left[ x_1: \ldots : x_{n+1} \right] \in \mathbb{P}^{n} - X$.\\
\linebreak
Now, if each $X \cap U_i$ is closed on each $U_i \cong \mathbb{A}^{n}$ then
each $U_i - \left( U_i \cap X \right) $ is open
in $\mathbb{P}^{n}$, so
$\mathbb{P}^{n}- X = \bigcup_{i} U_i - \left( U_i \cap X \right)$ is open, so
$X = \mathbb{P}^{n}-\left( \mathbb{P}^{n}-X \right) $ is closed.\\
\linebreak
(d) Suppose $X \subset \mathbb{P}^{n}$ is closed. Then by (a),
$X \cap U_i$ is closed for each $i$ in the Zariski topology on each
$U_i \cong \mathbb{A}^{n}$.\\
Conversely, if each  $X \cap U_i$ is closed in the Zariski topology on each
$U_i \cong \mathbb{A}^{n}$, then by (c),
$X$ is closed in the Zariski topology on $\mathbb{P}^{n}$.\\
Now, if $X \subset \mathbb{P}^{n}$ is open, then
$\mathbb{P}^{n} - X$ is closed, so by (a),
$(\mathbb{P}^{n} - X) \cap U_i =
\left( \mathbb{P}^{n} \cap U_i \right) - X =U_i - X
$ is closed in $U_i$, so
$U_i - \left( U_i - X \right) 
= U_i \cap \left( U_i \cap X^{c} \right)^{c}
= U_i \cap \left( U_i^{c} \cup X \right) 
= U_i \cup X$ is open in $U_i$ for each $i$.\\
Conversely, if $X \cap U_i$ is open for each $U_i$, then
$U_i - \left( X \cap U_i \right) = U_i - X
= U_i \cap X^{c}$ is closed
for each $U_i$, so by (c), $X^{c}$ is closed in $\mathbb{P}^{n}$, and thus
$\mathbb{P}^{n} - X^{c} = X$ is open in $\mathbb{P}^{n}$.\\
\linebreak
\textbf{3:}\\
(b) Suppose $J \subset k\left[ x_1, \ldots, x_{n+1} \right] $ is a radical
homogeneous ideal. By (a), we have that
$J'$ is an ideal. Suppose now $f \in \sqrt{J'} $, so
$f^{n} \in J'$.\\
\textbf{Lemma:} if $f,g \in k\left[ x_1, \ldots, x_n \right] $ then
$H(fg) = H(f) H(g)$.\\
\linebreak
\textit{Proof:} Suppose $f = \sum_{\alpha_1 + \ldots + \alpha_n}
a^{(\alpha)} x_1^{\alpha_1} \ldots x_n^{\alpha_n}$ and
$g = \sum_{\beta_1 + \ldots + \beta_n \le M}
b^{(\beta)} x_1^{\beta_1} \ldots x_n^{\beta_n}$.\\
Then dropping summation sign and using Einstein notation
$H(f) = a^{(\alpha)}
x_1^{\alpha_1} \ldots x_n^{\alpha_n} x_{n+1}^{N- \sum \alpha_i}$ and
$H(g) = b^{(\beta)}
x_1^{\beta_1}\ldots x_n^{\beta_n}
x_{n+1}^{M - \sum \beta_i}$. Then
$fg = a^{(\alpha)}
b^{(\beta)} x_1^{\alpha_1+ \beta_1} \ldots x_n^{\alpha_n + \beta_n} $, so
$$H(fg) = a^{(\alpha)}b^{(\beta)}
x_1^{\alpha_1 + \beta_1} \ldots x_{n}^{\alpha_n + \beta_n}
x_{n+1}^{N+M - \sum_i \left( \alpha_i + \beta_i \right) }
= \left[ a^{(\alpha)}
x_1^{\alpha_1} \ldots x_n^{\alpha_n}
x_{n+1}^{N - \sum_i \alpha_i} \right] 
\left[
    b^{(\beta)}x_1^{\beta_1} \ldots x_n^{\beta_n}
    x_{n+1}^{M - \sum_i \beta_i}
\right]
    = H(f) H(g)
$$

We thus find that
since $f^{n} \in J'$, $H(f)^{n} \stackrel{\text{lemma}}{=} H(f^{n}) \in J$, so
since $J$ is radical, $H(f) \in J$. Dehomoginizing, we find
$f \in J'$, so $\sqrt{J'} \subset J' \subset \sqrt{J'} $, hence
$\sqrt{J'}  = J'$, so  $J'$ is a radical ideal.\\
\linebreak
(c) Suppose $I \subset k\left[ x_1, \ldots, x_n \right] $ is radical.
Suppose $F \in \sqrt{H(I)} $, so
$F^{n} \in H(I)$. There exists an $f \in I$ such that
$H(f) = F^{n}$.\\ Now, denote by
$D(f)$ the dehomogenized polynomial of $f$ - i.e. the polynomial $f$ with
$x_{n+1} = 1$. We have as a direct corollary from the definitions that
$D(H(f)) = f$ and $H(D(f)) = f$, where $H(f)$ gives the smallest degree
homogenization of $f$.\\
\textbf{Lemma:} For $f,g \in k\left[ x_1, \ldots, x_{n+1} \right] $, 
$D(fg) = D(f)D(g)$.\\
\textit{Proof:} Again, using Einstein summation, suppose
$f = a^{(\alpha)} x_1^{\alpha_1}\ldots x_{n+1}^{\alpha_{n+1}}$ and
$g = b^{(\beta)} x_1^{\beta_1}\ldots x_{n+1}^{\beta_{n+1}}$ then
$fg = a^{(\alpha)}b^{(\beta)}x_1^{\alpha_1}x_1^{\beta_1}\ldots
x_{n+1}^{\alpha_{n+1}}x_{n+1}^{\beta_{n+1}}$, so
\[
D(fg) = a^{(\alpha)}b^{(\beta)}
x_1^{\alpha_1}x_1^{\beta_1}\ldots x_{n}^{\alpha_{n}}x_{n}^{\beta_{n}}
= D(f) D(g).
\] 
Hence
dehomogenizing $H(f) = F^{n}$, we get
 $\left( D(F) \right)^{n} = D(F^{n}) = D(H(f)) = f \in I$, so
 $D(F) \in I$ since $I$ is radical, and hence
 $F = H(D(F)) \in H(I)$. Thus $\sqrt{H(I)} \subset H(I) \subset \sqrt{H(I)} $,
 so
 $\sqrt{H(I)} = H(I)$, so $H(I)$ is radical.\\
 \linebreak
 \textbf{4:}\\
Firstly, we give a trivial solution:\\
By theorem 3.26 in Linear Algebra Done Right by Axler, a homogeneous system of
linear equations with more variables than equations has nonzero solutions.\\
\linebreak
Now we show the above:

Consider the matrix $\left( a_{i,j} \right) $ where the $i,j$ entry is
$a_{i,j}$. The matrix is of dimension $m \times (n+1)$. It represents the
linear transformation $T  \colon \mathbb{A}^{n+1} \to \mathbb{A}^{m}$ given by
\[
T(x_1, \ldots, x_{n+1}) 
= \left( \sum_{k=1}^{n} a_{1,k} x_k,\ldots
, \sum_{k=1}^{n} a_{m,k} x_k \right) 
\] 

Since
$m \le n < n+1$, by the fundamental theorem of
linear maps (3.22 in Linear Algebra done right by Axler), we have $n+1= \dim \mathbb{A}^{n+1} =
\dim \ker T + \dim \range T$, and as $\dim \range T \le
\dim \mathbb{A}^{n} \le n$, we have
$\dim \ker T \ge 1$, so $T$ has nontrivial solutions.\\
Suppose
 $0\neq (b_1, \ldots, b_{n+1}) \in \ker T$, so
  \begin{align*}
      (0,\ldots,0)
     &=T(b_1, \ldots, b_{n+1})\\
     &= \left( \sum_{k=1}^{n} b_{1,k}x_k, \ldots,
     \sum_{k=1}^{n} b_{m,k}x_k \right) 
 \end{align*}
Hence 
$\left[ b_1, \ldots, b_{n+1} \right] 
\in \mathbb{V}(a_{1,1}x_1+ \ldots + a_{1,n+1}x_{n+1})
\cap \ldots \cap \mathbb{V}\left( a_{m,1}x_1 + \ldots + a_{m,n+1}x_{n+1} \right) 
= \Lambda_1 \cap \ldots \cap \Lambda_m$, so
$\Lambda_1 \cap \ldots \cap \Lambda_m \neq \varnothing$.\\
\linebreak
\textbf{5:}\\
(a) Suppose $F \in S_d$. Then there exists
a homogeneous polynomial of degree $d$, $f \in k\left[ x_1, \ldots, x_{n+1} \right] $ with
$\overline{f} = F$. Now, we claim that the set of homogeneous polynomials of
degree $d$ in $k\left[ x_1,\ldots, x_{n+1} \right] $ is precisely
$A = (x_1, \ldots, x_{n+1})^{d}$. Any monomial of degree $d$ is in $A$, and
hence any homogeneous polynomial of degree $d$ is in $A$. Conversely, any
polynomial in $A$ is by definition a sum of monomials of degree $d$ and thus
a homogeneous polynomial of degree $d$.\\
Since
$A = \sum_{\alpha_1 + \ldots + \alpha_{n+1} = d} k
x_1^{\alpha_1} \ldots x_{n+1}^{\alpha_{n+1}}$, it is finite-dimensional over
$k$. Thus $f \in A$, so
$ F=\overline{f} \in \pi_I (A) =
\sum_{\alpha_1 + \ldots + \alpha_{n+1} = d}
k \overline{x}_1^{\alpha_1} \ldots \overline{x}_{n+1}^{\alpha_{n+1}}$. Now,
if $f \in I$, then $F = 0$. Thus because $\pi_I$ is surjective,
 \[
\pi_I (A) = S_d = \left\{ a \overline{x}^{\alpha_1} \ldots \overline{x}_{n+1}^{\alpha_{n+1}}
      \mid \alpha_1 + \ldots + \alpha_{n+1} = d,
      x_1^{\alpha_1} \ldots x_{n+1}^{\alpha_{n+1}} \not\in  I
\right\} \]
 which is generated by the basis
\[
    \left\{ \overline{x}_1^{\alpha_1} \ldots \overline{x}_{n+1}^{\alpha_{n+1}}
 \mid \alpha_1 + \ldots + a_{n+1} = d,
x_1^{\alpha_1}\ldots x_{n+1}^{\alpha_{n+1}}\not\in  I \right\} 
\] 
and is thus finite dimensional.\\
We check also the requirements for a vector space:\\
If  $F,G \in S_d$, then there exist
homogeneous polynomials of degree $d$ 
 $f,g \in k\left[ x_1, \ldots, x_{n+1} \right] $ such that
 $\overline{f} = F$ and $\overline{g}=G$. So
 $F+G = \overline{f} + \overline{g} = \overline{f+g}
 \in \pi_I(A) = S_d$, since
 $f+g$ is homogeneous of degree $d$.\\
 For scalar multiplication, let $F \in S_d$, then
 as above $\overline{f} = F$, so
 for  $\lambda \in k$, $\lambda F = \lambda \overline{f}
 = \overline{\lambda f} \in \pi_I (A) = S_d$ since
 $\lambda f$ is homogeneous of degree $d$.\\
 Furthermore, we can consider $0$ as a form of any degree, and thus
 $0 \in S_d$ for any $d$, so we have an additive identity. The
 rest of the requirements are seen trivially as inherited from
 $k\left[ x_1, \ldots, x_{n+1} \right] $.\\
 \linebreak
 (b) One upper bound is the upper bound in
 $k\left[ x_1, \ldots, x_{n+1} \right] $. The subspace
 $A$ is generated by the basis given in (a). For this, we use the
 balls-and-urns formula: we must partition $d$ objects into
 $n+1$ urns, which we can do in
 $\binom{(n+1)+d-1}{(n+1)-1} = \binom{n+d}{n}$ ways. The image of the basis
 under
 $\pi_I$ is a basis for $\pi_I (A) = S_d$. However, $\pi_I$ may collapse some
 basis elements to $0$ if they are contained in $I$, so the basis for
 $S_d$ has an upper bound of
 $\binom{n+d}{n}$.







\end{document}
