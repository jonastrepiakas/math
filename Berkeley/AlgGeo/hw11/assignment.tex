\documentclass[a4paper]{article}

\usepackage[margin=2.5cm]{geometry}
\usepackage[pdftex]{graphicx}
\usepackage[utf8]{inputenc}
\usepackage[T1]{fontenc}
\usepackage{textcomp}
\usepackage{babel}
\usepackage{amsmath, amssymb}
\usepackage[colorlinks=true,linkcolor=blue]{hyperref}
\usepackage{float}
\usepackage{mathrsfs}
%\usepackage{enumitem}
%% for identity function 1:
\usepackage{bbm}
%%For category theory diagrams:
%\usepackage{tikz-cd}
%%For code (e.g. python) in latex:
%\usepackage{listings}
%
%Usage: 
%\begin{lstlisting}[language=Python]
%\end{lstlisting}

\newcommand{\incfig}[2][1]{%
\def\svgwidth{#1\columnwidth}
\import{./figures/}{#2.pdf_tex}
}


% figure support
\usepackage{import}
\usepackage{xifthen}
\pdfminorversion=7
\usepackage{pdfpages}
\usepackage{transparent}

\pdfsuppresswarningpagegroup=1

\setlength\parindent{0pt}

\newcommand{\qed}{\tag*{$\blacksquare$}}
\newcommand{\qedwhite}{\hfill \ensuremath{\Box}}

%Inequalities
\newcommand{\cycsum}{\sum_{\mathrm{cyc}}}
\newcommand{\symsum}{\sum_{\mathrm{sym}}}
\newcommand{\cycprod}{\prod_{\mathrm{cyc}}}
\newcommand{\symprod}{\prod_{\mathrm{sym}}}

%Linear Algebra

\DeclareMathOperator{\Span}{span}
\DeclareMathOperator{\rank}{rank}
\DeclareMathOperator{\Ima}{Im}
\DeclareMathOperator{\diag}{diag}
\DeclareMathOperator{\Ker}{Ker}
\DeclareMathOperator{\ob}{ob}
\DeclareMathOperator{\Hom}{Hom}
\DeclareMathOperator{\sk}{sk}
\DeclareMathOperator{\Vect}{Vect}
\DeclareMathOperator{\Set}{Set}
\DeclareMathOperator{\Group}{Group}
\DeclareMathOperator{\Ring}{Ring}
\DeclareMathOperator{\Ab}{Ab}
\DeclareMathOperator{\Top}{Top}
\DeclareMathOperator{\Htpy}{Htpy}
\DeclareMathOperator{\Cat}{Cat}
\DeclareMathOperator{\CAT}{CAT}


%Row operations
\newcommand{\elem}[1]{% elementary operations
\xrightarrow{\substack{#1}}%
}

\newcommand{\lelem}[1]{% elementary operations (left alignment)
\xrightarrow{\begin{subarray}{l}#1\end{subarray}}%
}

%SS
\DeclareMathOperator{\supp}{supp}
\DeclareMathOperator{\Var}{Var}

%NT
\DeclareMathOperator{\ord}{ord}

%Alg
\DeclareMathOperator{\Rad}{Rad}
\DeclareMathOperator{\Jac}{Jac}

\DeclareMathAlphabet{\pazocal}{OMS}{zplm}{m}{n}
\newcommand{\unif}{\pazocal{U}}

\begin{document}
    \textbf{1:} Choose a representation for
    each $P_i = (a_{i1}, a_{i2}, a_{i3})$ and
    for each $Q_i = (b_{i 1}, b_{i 2}, b_{i 3})$.
    Since $P_1, P_2$ and $P_3$ (resp. $Q_1$, $Q_2$ and $Q_3$ ) do not lie on
    a line, three points lying in the corresponding subset in $\mathbb{A}^3$ 
    for
    $P_1, P_2$ and $P_3$ span
    $\mathbb{A}^3$, so choosing these as basis for
    $\mathbb{A}^3$, we can define a linear map
    sending $P_i \to Q_i$. This is clearly invertible with
    the linear map sending $Q_i \to P_i$, and
    sends $0 \to 0$, so it induces a projective change
    of coordinates. And since
    $T(\lambda P_i) = \lambda T(P_i) =
    \lambda Q_i$, we have that it maps
    the line through $P_i$ to the line through $Q_i$, and thus
    that the induced map $\mathbb{P}^2 \to \mathbb{P}^2$ maps
    the points $P_i \to Q_i$.\\
    \linebreak

    Similarly, if $P_1, \ldots, P_{n+1}$ are points in 
    $\mathbb{P}^{n}$ not lying on a hyperplane, points on the represented lines
    in
    $\mathbb{A}^{n+1}$ are linearly independent, so
    they define a basis for $\mathbb{A}^3$. We can again define a map
    sending  $P_i \to Q_i$ which is invertible with
    $Q_i \to P_i$ and $0 \to 0$. Thus it induces a projective
    change of coordinates mapping
    $P_i \to Q_i$.\\




    \textbf{2. Duals:}\\
    (a) We first show that $\left( P^{*} \right)^{*} = P$.\\
    As stated in the problem,
    assigning $\left[ a_1 : \ldots : a_{n+1} \right] 
    \in \mathbb{P}^{n}$ to
    $\Lambda = \mathbb{V}\left( a_1 x_1 +\ldots
    + a_{n+1} x_{n+1} \right) $ sets up a one-to-one correspondence between
    $\left\{ \text{hyperplanes in }\mathbb{P}^{n} \right\} $ 
    and $\mathbb{P}^{n}$ - since
    $(a_1, \ldots, a_{n+1})$ is determined by $\Lambda$ up to scaling since it
    must be perpendicular to the vector
    $(x_1, \ldots, x_n)$ and thus can be freely scaled.\\
    \linebreak
    Now,
    $P^{*} = \mathbb{V}\left( 
    a_1  : \ldots : a_{n+1}\right) $ which we assign
    to $\left[ a_1 : \ldots : a_{n+1} \right] = P$ by construction.\\
    \linebreak
    For $\left( \Lambda^{*} \right)^{*} = \Lambda$, note that
    $\Lambda^{*} = \left[ a_1  : \ldots : a_{n+1} \right] $, so
    $\left( \Lambda^{*} \right)^{*} = 
    \mathbb{V} \left( a_1 x_1 + \ldots
     + a_{n+1} x_{n+1} \right) = \Lambda$.\\
     \linebreak
     (b) Suppose 
     $P = \left[ a_1 : \ldots : a_{n+1} \right] $ and
     $\Lambda = \mathbb{V}
     (l_1 x_1 + \ldots + l_{n+1} x_{n+1})$. Then
     $P \in \Lambda$ if and only if
     $l_1 a_1 + \ldots + l_{n+1}a_{n+1} = 0$ if and only if
     $\Lambda^* =
     \left[ l_1 : \ldots : l_{n+1} \right] 
     \in \mathbb{V}\left( a_1 x_1 + \ldots +
     a_{n+1} x_{n+1} \right) 
     = P^{*}$.\\
     \linebreak
     \textbf{3:} Write $P_i = \left[ p_{i 1} :
     p_{i 2} : p_{i 3} \right] $. Then the lines
     passing through $P_i$ in $\mathbb{P}^2$ is precisely
     the hyperplane corresponding to $P_i$, i.e.
     $P_i^{*}$.\\
     Now, by problem 2.(b),
     $\forall i  \colon P_i \not\in \Lambda \iff
     \forall i  \colon \Lambda^* \not\in  P_i^{*}
     \iff \Lambda^{*} 
     \not\in P_1^{*} \cup \ldots \cup 
     P_r^{*}$.\\
     Thus, if we can show that there exists
     a point $\Lambda^{*}$ not in the union of hyperplanes
     $P_1^{*} \cup \ldots \cup P_r^{*}$, then this gives the existence of the
     line $\Lambda = \left( \Lambda^{*} \right)^{*}$ not passing through any of
     the points $P_i$.\\
     \linebreak
     Now
     \[
     P_1^{*} \cup \ldots \cup P_r^{*}
     = \mathbb{V}\left( p_{1 1} x +
     p_{1 2} y + p_{1 3}z\right) \cup 
     \ldots \cup 
     \mathbb{V}\left( p_{r 1}x+ p_{r 2}y + p_{r 3}z \right) 
     = \mathbb{V}\left( 
     \Pi_i \left( p_{i 1}x + p_{i 2}y + p_{i 3}z \right)   \right) 
     \] 
     
     So if there does not exist a point $\Lambda^{*}$, then
     \[
     \mathbb{P}^2 =
     \mathbb{V} \left( \Pi_i
     \left( p_{i 1} x+ p_{i 2}y + p_{i 3}z \right) \right) 
     \] 
     Consider the polynomial
     $f \in k\left[ x,y,z \right] $ given by
     $f = \Pi_i
     \left( p_{i 1}x + p_{i 2}y + p_{i 3}z \right) $.
    We have
    $\mathbb{V} (f) = \mathbb{P}^2$ if and only if
    $V(f) = \mathbb{A}^3$, so
    $f$ is constant zero. Thus one of the points
    $\left( p_{i 1}, p_{i 2}, p_{i 3} \right) $ must be
    $0$, however
    then $P_i = \left[ 0:0:0 \right] \not\in  \mathbb{P}^2$, contradiction.
    Thus such a $\Lambda^{*}$ exists, and hence
    for all $i$, we have
    $P_i \not\in \Lambda$, so
     $\Lambda$ is a line in $\mathbb{P}^2$ not passing through any
     of the $P_i$.\\
     \linebreak
     

 

     \textbf{4:}\\
     (a) We must give an inverse to
     $v_{1,3}  \colon \mathbb{P}^{1} \to \mathbb{P}^3$ by
     $\left[ s : r \right] \to 
     \left[ s^3 : s^2 t : s t^2 : t^3 \right] $.\\
     Let $Y$ denote the image under
     $v_{1,3}$. Consider the map
     $\varphi  \colon Y \to \mathbb{P}^{1}$ by
     \[
     \begin{cases}
         \left[ s^3 : s^2 t \right], & s \neq 0 \text{ (i.e. in } U_1)\\
         \left[ s t^2 : t^3 \right], & t\neq 0
         \text{ (i.e. in } U_4)
     \end{cases}
     \] 
     
     This map covers its image by definition and 
     also agrees on $U_1 \cap U_4$ since
     then
     $\left[ s^3 : s^2 t \right] 
     = \left[ s : t \right] 
     = \left[ s t^2 : t^3\right] $.\\
     \linebreak
     Now for an arbitrary element in the
     image $Y$, there exist $s,t \in k$ such that
     $v_{1,3} [s:t] = 
     \left[ s^3 : s^2 t : s t^2 : t^3 \right] $ represents
     the element and hence
     \[
     v_{1,3} \circ \varphi
     (\left[ s^3 : s^2 t : s t^2 : t^3 \right] )
     = \begin{cases}
         v_{1,3} \left[ s^3 : s^2 t \right] & s\neq 0\\
         v_{1,3} \left[ s t^2  : t^3 \right] & t\neq 0
     \end{cases}
     = v_{1,3} \left[ s : t \right] 
     = \left[ s^3 : s^2 t : s t^2 : t^3 \right]     
     \] 
     and
     \[
     \varphi \circ v_{1,3} \left[ s:t \right] 
     = \varphi \left[ s^3 : s^2 t : s t^2 : t^3 \right] 
     = \begin{cases}
         \left[ s^3 : s^2 t \right] & s\neq 0\\
         \left[ s t^2 : t^3 \right] & t\neq 0
     \end{cases}
     = \left[ s:t \right] 
     \] 
     where either $s \neq 0$ or $t \neq 0$ since
     $\left[ 0 : 0 \right]  \not\in  \mathbb{P}^2$.\\
     Hence $\varphi$ is the inverse to $v_{1,3}$, so
     $v_{1,3}$ is an isomorphism onto its image.\\
     \linebreak
     (b) By writing out relations such as
     $z_1 z_6 - z_2 a_3 = 0$ which are clear by definition, we
     find the matrix
     \[
         A = \begin{pmatrix} z_1 & z_3 & z_4\\
         z_2 & z_3 & z_7\\
     z_3 & z_8 & z_9\\
 z_4 & z_9 & z_{10}
\end{pmatrix} 
     \] 
     Every $2\times 2$ minor in this matrix vanishes, while 
     some $1\times 1$ minor does not, so
     the rank of $A$ is $1$. Consider
     \[
     Y = \left\{ \left[ x_1 : \ldots : x_{10} \right] 
      \colon 
  \rank A \le 1 \right\} 
     \] 
    Now, by definition
    $v_{2,3} (\mathbb{P}^3) \subset 
    Y$.\\
    Furthermore, for an arbitrary
    $\left[ z_1 : \ldots : z_{10} \right] \in Y$, either
    $z_1, z_5, z_8$ or $z_{10}$ is nonzero since
    otherwise all entries would be  $0$, but $
    \left[ 0 : \ldots : 0 \right] \not\in  \mathbb{P}^{9}$.\\
    If $z_1 \neq 0$ then
    $v_{2,3} \left( \left[ z_1 : z_2 : z_3 : z_4 \right]  \right) 
    = \left[ z_1 : \ldots : z_{10} \right] $. If
    $z_5 \neq 0$, then
    $v_{2,3}\left( \left[ z_2 : z_5 : z_6 : z_7 \right]  \right) 
    = \left[ z_1 : \ldots : z_{10} \right] $.\\
    If $z_8 \neq 0$ then
    $v_{2,3} \left( \left[ z_3 : z_6 : z_8 : z_9 \right]  \right) 
    = \left[ z_1 :\ldots : z_{10} \right] $.\\
    If $z_{10} \neq 0$, then
    $v_{2,3} \left( \left[ z_4 : z_7 : z_9 : z_{10} \right]  \right) 
    = \left[ z_1 : \ldots : z_{10} \right] $, so
    $Y \subset v_{2,3} \left( \mathbb{P}^3 \right) $. Hence
    $Y$ is precisely the image of $v_{2,3}$.
     

     \newpage
     (c) Since $z_1 = x_1^2,
     z_3 = x_1 x_3, z_7 = x_2 x_4$ and
     $z_9 = x_3 x_4$, we have
     \[
     v_{3,2}^{-1}\left( 
     \mathbb{V} \left( z_1 + 4z_3 - 2z_7+ 5 z_9 \right) \right) 
     = \mathbb{V}
     \left( x_1^2 - 4x_1 x_3 - 2
     x_2 x_4 + 5 x_3 x_4\right) 
     \] 

     \textbf{5:}\\
     (a) 
     \linebreak
     By the comment in the beginning on lecture note 22, 
     we have that
     $\Gamma_h (\overline{X}) \cong
     \Gamma_h (\overline{Y})$, so
     $\Gamma(C(\overline{X})) =
     \Gamma \left( C(\overline{Y}) \right) $. Hence
     $C(\overline{X}) \cong C(\overline{Y})$ by the lemma on lecture note 8. Thus there
     exist morphisms $\varphi  \colon C(\overline{X}) \to C(\overline{Y})$ and
     $\psi  \colon C\left( \overline{Y} \right) \to 
     C\left( \overline{X} \right) $ with
     $\varphi \circ \psi = \mathbbm{1}$ and
     $\psi \circ \varphi = \mathbbm{1}$.\\
     There exist polynomials $T_1, \ldots, T_{n+1} \in k\left[ x_1, \ldots,
     x_{n+1} \right] $ 
     such that
     $\varphi (P ) = \left( T_1(P),\ldots, T_{n+1}(P) \right) $ 
     and polynomials 
     $S_1, \ldots, S_{n+1} \in k\left[ x_1, \ldots, x_{n+1} \right] $ such that
     $\psi(P) = \left( S_1 (P), \ldots, S_{n+1}(P) \right) $.
     The image
     $\varphi \left( \overline{X}\cap U_{n+1} \right) $ is isomorphic to
     $Y$ as this is
     $\varphi(\overline{X}) \cap \varphi (U_{n+1}) = \overline{Y} 
     \cap \varphi(U_{n+1}) $.\\
     Restricting to $P$ having final coordinate $1$ then gives us an
     isomorphism
     $X \to Y$ with the polynomials $T_1, \ldots, T_{n}$ and
     $S_1, \ldots, S_n$.\\
     As $\overline{X}$ and $\overline{Y}$ are equivalent, there exists
     $G  \colon \mathbb{P}^{n} \to \mathbb{P}^{n}$ restricting to an
     isomorphism $\overline{X} \to \overline{Y}$. 
     
\newpage
     
     (b) We have $V(x - y^2) \simeq V(x)$ as affine plane
     curves. However,
     their projective closures are
     $Y =\mathbb{V}(xz-y^2)$ and $\mathbb{V}(x)$. Now
     $\Gamma_h (Y) = k\left[ x,y,z \right] /(xz-y^2)$ and
     $\Gamma_h (\mathbb{V}(x)) = k\left[ x,y,z \right] /(x)
     \simeq k\left[ y,z \right] 
     $
     Now, $k\left[ x,y,z \right] / (xz-y^2)$ is not UFD while
     $k\left[ y,z \right] $ is UFD, so
     $\Gamma_h (Y)$ and $\Gamma_h (\mathbb{V}(x))$ are not isomorphic. However,
     by
     page 8 on lecture note 21, we have that
     if  $\overline{X}$ and $\overline{Y}$ are projectively equivalent, then
     $\Gamma_h (\overline{X})$ and $\Gamma_h (\overline{Y})$ are isomorphic, so
     by contraposition, $\overline{X}$ and $\overline{Y}$ are not projectively
     equivalent.
     
     
















\end{document}
