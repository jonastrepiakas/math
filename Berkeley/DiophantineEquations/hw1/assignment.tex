\documentclass[a4paper]{article}

\usepackage[margin=2.5cm]{geometry}
\usepackage[pdftex]{graphicx}
\usepackage[utf8]{inputenc}
\usepackage[T1]{fontenc}
\usepackage{textcomp}
\usepackage{babel}
\usepackage{amsmath, amssymb}
\usepackage[colorlinks=true,linkcolor=blue]{hyperref}
\usepackage{float}
\usepackage{mathrsfs}
%\usepackage{enumitem}
%% for identity function 1:
%\usepackage{bbm}
%%For category theory diagrams:
%\usepackage{tikz-cd}
%%For code (e.g. python) in latex:
\usepackage{listings}
%
%Usage: 
%\begin{lstlisting}[language=Python]
%\end{lstlisting}

\newcommand{\incfig}[2][1]{%
\def\svgwidth{#1\columnwidth}
\import{./figures/}{#2.pdf_tex}
}


% figure support
\usepackage{import}
\usepackage{xifthen}
\pdfminorversion=7
\usepackage{pdfpages}
\usepackage{transparent}

\pdfsuppresswarningpagegroup=1

\setlength\parindent{0pt}

\newcommand{\qed}{\tag*{$\blacksquare$}}
\newcommand{\qedwhite}{\hfill \ensuremath{\Box}}

%Inequalities
\newcommand{\cycsum}{\sum_{\mathrm{cyc}}}
\newcommand{\symsum}{\sum_{\mathrm{sym}}}
\newcommand{\cycprod}{\prod_{\mathrm{cyc}}}
\newcommand{\symprod}{\prod_{\mathrm{sym}}}

%Linear Algebra

%Redeclaring Span and image
\DeclareMathOperator{\Span}{span}
\DeclareMathOperator{\Ima}{Im}
\DeclareMathOperator{\diag}{diag}
\DeclareMathOperator{\Ker}{Ker}
\DeclareMathOperator{\ob}{ob}


%Row operations
\newcommand{\elem}[1]{% elementary operations
\xrightarrow{\substack{#1}}%
}

\newcommand{\lelem}[1]{% elementary operations (left alignment)
\xrightarrow{\begin{subarray}{l}#1\end{subarray}}%
}

%SS
\DeclareMathOperator{\supp}{supp}
\DeclareMathOperator{\Var}{Var}

%NT
\DeclareMathOperator{\ord}{ord}

%Alg
\DeclareMathOperator{\Rad}{Rad}
\DeclareMathOperator{\Jac}{Jac}

\DeclareMathAlphabet{\pazocal}{OMS}{zplm}{m}{n}
\newcommand{\unif}{\pazocal{U}}

\begin{document}

\textbf{Problem 1.1.7 - (a) and (b) - (6 points):}

\begin{lstlisting}[language=Python]
def bezout(n,m):
    #returns [x,y] s.t. nx+my = 1 if a,b are any coprime integers
    n_buffer = False

    s_n = 1
    s_m = 1

    if n < 0:
        s_n = -1
        n = abs(n)
    if m < 0:
        s_m = -1
        m = abs(m)

    if m > n:
        n_buffer = n
        n = m
        m = n_buffer

    a = [n, m]
    q = []

    n = 0
    while a[len(a)-1] != 0:
        a.append(a[n] % a[n+1])
        q.append(a[n] // a[n+1])
        n += 1

    x = len(a) * [0]

    x[n] = 1
    x[n+1] = 0

    i = n-1
    while i >= 0:
        x[i] = x[i+2]
        x[i+1] = x[i+1] - q[i] * x[i+2]
        i -= 1
    if n_buffer:
        return(s_m * x[1], s_n * x[0])
    else:
        return(s_n * x[0], s_m * x[1])
\end{lstlisting}
Running it on $(n,m) = (12345678901, 10987654321)$ gives
$(x,y) = (3733873449, -4195363388)$.\\
\linebreak



\textbf{Problem 1.1.5 - (3 points):}
\begin{lstlisting}[language=Python]
    def euclid(a,b):

    while True:
        if a%b == 0:
            return(b)
        else:
            b_buffer = a%b
            a = b
            b = b_buffer
            continue

euclid(1027,1738)            
\end{lstlisting}
The output was $79$.  Now we have $\frac{1}{79}\left( 1027,1738 \right) 
= (13, 22)$ and running
bezout(13,22) from the previous problem we get $(-5,3)$ so
$13 \cdot (-5) + 22 \cdot 3 = 1$ and then
\[
    (13 \cdot 79) \cdot (-5) + (22 \cdot 79) \cdot 3 
    = 1027 \cdot (-5) + 1738 \cdot 3 = 79 = \gcd(1027,1738)
\] 
so
$(\gcd(1027,1738), x, y) = (79, -5, 3)$ works.\\
\linebreak
\textbf{Problem 1.1.1 - (1 point):}
Looking at the quadratic residues modulo $4$ we find
$$0^2 \equiv 0, 1^2 \equiv 1, 2^2 \equiv, 3^2 \equiv 1 \pmod{4}$$
so for any $x,y \in \mathbb{Z}$, we have $x^2,y^2 \in \{0,1\} \pmod{4}$, so
\[
    x^2 + y^2 \in \left\{ 0,1,2 \right\} \pmod{4}
\] 
and hence $x^2 + y^2 \not \equiv 3 \equiv n \pmod{4}$.


\end{document}
