\documentclass[a4paper]{article}

\usepackage[margin=2.5cm]{geometry}
\usepackage[pdftex]{graphicx}
\usepackage[utf8]{inputenc}
\usepackage[T1]{fontenc}
\usepackage{textcomp}
\usepackage{babel}
\usepackage{amsmath, amssymb}
\usepackage[colorlinks=true,linkcolor=blue]{hyperref}
\usepackage{float}
\usepackage{mathrsfs}
%\usepackage{enumitem}
%% for identity function 1:
%\usepackage{bbm}
%%For category theory diagrams:
%\usepackage{tikz-cd}
%%For code (e.g. python) in latex:
\usepackage{listings}
%
%Usage: 
%\begin{lstlisting}[language=Python]
%\end{lstlisting}

\newcommand{\incfig}[2][1]{%
\def\svgwidth{#1\columnwidth}
\import{./figures/}{#2.pdf_tex}
}


% figure support
\usepackage{import}
\usepackage{xifthen}
\pdfminorversion=7
\usepackage{pdfpages}
\usepackage{transparent}

\pdfsuppresswarningpagegroup=1

\setlength\parindent{0pt}

\newcommand{\qed}{\tag*{$\blacksquare$}}
\newcommand{\qedwhite}{\hfill \ensuremath{\Box}}

%Inequalities
\newcommand{\cycsum}{\sum_{\mathrm{cyc}}}
\newcommand{\symsum}{\sum_{\mathrm{sym}}}
\newcommand{\cycprod}{\prod_{\mathrm{cyc}}}
\newcommand{\symprod}{\prod_{\mathrm{sym}}}

%Linear Algebra

%Redeclaring Span and image
\DeclareMathOperator{\Span}{span}
\DeclareMathOperator{\Ima}{Im}
\DeclareMathOperator{\diag}{diag}
\DeclareMathOperator{\Ker}{Ker}
\DeclareMathOperator{\ob}{ob}


%Row operations
\newcommand{\elem}[1]{% elementary operations
\xrightarrow{\substack{#1}}%
}

\newcommand{\lelem}[1]{% elementary operations (left alignment)
\xrightarrow{\begin{subarray}{l}#1\end{subarray}}%
}

%SS
\DeclareMathOperator{\supp}{supp}
\DeclareMathOperator{\Var}{Var}

%NT
\DeclareMathOperator{\ord}{ord}

%Alg
\DeclareMathOperator{\Rad}{Rad}
\DeclareMathOperator{\Jac}{Jac}

\DeclareMathAlphabet{\pazocal}{OMS}{zplm}{m}{n}
\newcommand{\unif}{\pazocal{U}}

\begin{document}
     \textbf{1.2.4:}\\
     (a) We show it by induction.\\
     For $\left[ a_0 \right] = \left[ b_0 \right] $, we have $a_0 = b_0$.\\
     For $\left[ a_0; a_1 \right] = \left[ b_0; b_1 \right] $, we have
     $\frac{h_1}{k_1} =\frac{a_0 a_1 + 1}{a_1} = \left[ a_0; a_1 \right] 
     = \left[ b_0; b_1 \right] = \frac{b_0 b_1 + 1}{b_1} = \frac{h_1'}{k_1'}$, and by corollary
     1.36, $\frac{a_0 a_1 + 1}{a_1}$ and $\frac{b_0 b_1 + 1}{b_1}$ are in
       reduced form with $a_1, b_1 \ge 1$, so since they are equal, we have
       $a_1  \mid \left( a_0 a_1 +1 \right) b_1 \implies a_1 | b_1$ and
       similarly, $b_1 | a_1$ so $b_1 = a_1$. And then $a_0 = b_0$ too.\\
       \linebreak
       Now assume $\left[ a_0, \ldots, a_{n-1} \right] = \left[ b_0, \ldots,
       b_{n-1} \right] $. Then
       we have
       \[
           a_0 + \underbrace{\frac{1}{\left[ a_1; \ldots, a_n \right] }}_{\in
           (0,1]}
       = \left[ a_0; a_1, \ldots, a_n \right] 
       = \left[ b_0; b_1, \ldots, b_n \right] 
       = b_0 + \underbrace{\frac{1}{\left[ b_1; b_2, \ldots, b_n \right]
       }}_{\in (0,1]}
       \] 
       where the element in $(0,1]$ follows since
       $\left[ a_1; \ldots, a_n \right] $ and $\left[ b_1; \ldots, b_n \right]
       $ are
       positive since each $a_i, b_j$ is positive. Since $a_0$ and $b_0$ are
       integers, we therefore have
       \[
       \left| a_0 - b_0 \right| = \left| \frac{1}{\left[ a_1; \ldots, a_n \right] }
       - \frac{1}{\left[ b_1; \ldots, b_n \right] }\right| \in [0,1)
       \implies a_0 = b_0
       \] 
       since $a_0- b_0 \in \mathbb{Z}$.\\
       \linebreak
       We then get
       \[
       \left[ a_1; \ldots, a_n \right] = \left[ b_1; \ldots, b_n \right] 
       \] 
       and by the inductive hypothesis, we then have
       $a_i = b_i$ for all $i = 0, \ldots, n$.

\textbf{1.2.5 - (5 points):}

\begin{lstlisting}[language=Python]
A = [2, 1, 2, 1, 1, 4, 1, 1, 6, 1]

def conv(S, n):
    H = [0, 1]
    K = [1, 0]
    for i in range(1,n+1):
        H.append(S[i-1] * H[i] + H[i-1])
        K.append(S[i-1] * K[i] + K[i-1])
    return [H[-1], K[-1]]

for i in range(1, len(A)+1):
    print(f"{i}: {conv(A,i)}")
\end{lstlisting}

The output was:\\
1: [2, 1]\\
2: [3, 1]\\
3: [8, 3]\\
4: [11, 4]\\
5: [19, 7]\\
6: [87, 32]\\
7: [106, 39]\\
8: [193, 71]\\
9: [1264, 465]\\
10: [1457, 536]\\
\linebreak
Where nth: [$h_n, k_n$] gives the
$nth$ convergent as $\frac{h_n}{k_n}$.

       
       
\end{document}
