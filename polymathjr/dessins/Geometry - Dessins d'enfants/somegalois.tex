\documentclass[10pt]{article}  
\usepackage{amssymb}  
\usepackage{amsthm}
\usepackage{amsmath}
\usepackage{mathtools}
\synctex=1

\usepackage[left=1.25in,right=1.25in,bottom=1in,top=1in]{geometry}


\newcommand{\Z}{\mathbb Z}
\newcommand{\Q}{\mathbb Q}
\newcommand{\R}{\mathbb R}
\newcommand{\C}{\mathbb C}


\newtheorem{theorem}{Theorem}
\newtheorem{proposition}[theorem]{Proposition}
\newtheorem{lemma}[theorem]{Lemma}
\newtheorem{corollary}[theorem]{Corollary}

\theoremstyle{definition}
\newtheorem{definition}[theorem]{Definition}

\theoremstyle{remark}
\newtheorem{remark}[theorem]{Remark}

\theoremstyle{definition}
\newtheorem{exercise}[theorem]{Exercise}

\newenvironment{solution}{\noindent\textbf{Solution.}}{\qed}

\renewcommand{\labelenumi}{(\alph{enumi})}


\begin{document}  

\title{Some galois theory}
\date{\vspace{-5ex}}
\author{\vspace{-5ex}}
\maketitle

\pagestyle{empty}   
\thispagestyle{empty}   
\noindent Note: here we consider only fields of characteristic $0$.
\section{The field extensions perspective}

\textbf{Primitive Element Theorem.} Let $E/F$ be a finite extension. Then $E = F(\alpha)$ for some $\alpha \in E$.\\

This leads to the following:\\\\
\noindent\textbf{Proposition.} If $E/F$ is a finite extension, then $|\mbox{Aut}(E/F)| \leq [E:F]$.
\begin{proof}
By the PET, $E = F(\alpha)$ for some $\alpha \in E$. If $f \in F[x]$ be the minimal polynomial of $\alpha$, then $[E:F] = \mbox{deg}(f)$.\\

\noindent Let $Z$ be the set of roots of $f$ in $E$. Since $E$ has characteristic $0$, these roots are distinct. So $|Z| \leq \mbox{deg}(f)$.\\

\noindent $\mbox{Aut}(E/F)$ acts freely on $Z$ since $\alpha$ generates $E$ over $F$. Moreover, $\mbox{Aut}(E/F)$ is transitive on $Z$ since $f$ is irreducible. So we have $|\mbox{Aut}(E/F)| = |Z|\leq \deg(f)=[E:F]$.
\end{proof}

\noindent\textbf{Definition.} A finite extension $E/F$ is Galois if $|\mbox{Aut}(E/F)| = [E:F]$. Note: recall from GGD that a covering $f:S_1\to S_2$ is Galois if and only if $|\mbox{Mon}(f)| = \mbox{deg}(f)$.\\

The central idea of Galois theory that we covered in this class relates to subgroups of $\mbox{Aut}(E/F)$ and intermediate field extensions between $E$ and $F$.\\

\noindent\textbf{Theorem.} Let $G < \mbox{Aut}(E)$ be a finite subgroup, and let $E^G = \{x \in E:gx = x \mbox{ for all }g\in G\}$. Then, the extension $E/E^G$ is Galois, and $\mbox{Aut}(E/E^G) = G$.\\
 
\noindent\textbf{Theorem.} Let $E/F$ be a Galois extension with an intermediate field $M$. That is, $F\subset M\subset E$. Then, the extension ``from the top" $E/M$ is also Galois, and $M = E^H$ where $H = \mbox{Aut(E/M)}$.\\

Given a Galois extension $E/F$ with $G = \mbox{Aut}(E/F)$, these two theorems provide a bijection$$\{\mbox{subgroups }H\leq G\}\quad \longleftrightarrow\quad \{\mbox{intermediate fields }M,\,F\subset M\subset E\}$$where $H$ maps to its fixed-field $E^H$, and $M$ maps to $\mbox{Aut}(E/M)$. In other words,$$M\mapsto \mbox{Aut}(E/M)\mapsto E^{\mbox{Aut}(E/M)} = M$$and$$H\mapsto E^H\mapsto\mbox{Aut}(E/E^H) = H$$Moreover, the extension ``from the bottom" $M/F$ is Galois if and only if $\mbox{Aut}(E/M) \triangleleft G$.

\section{The polynomial perspective}
\textbf{Definition.} Consider a polynomial $f \in F[x]$. The minimal field $E$ in which $f$ splits into linear factors $f = (x-\alpha_1)\cdots(x-\alpha_n)$ is the splitting field of $f$. Note that $E$ is generated by the roots of $f$.\\

\noindent\textbf{Theorem.} Consider $f\in F[x]$ with an isomorphism $\psi:F\to F'$, and let $\psi(f) = f' \in F'[x]$. Let $E$ and $E'$ be the splitting fields of $f$ and $f'$ over $F$ and $F'$ respectively. Then, $E \simeq E'$.\\

Note: the specific isomorphism $\phi:F\to F'$ does not matter. If we consider the identity map, then that theorem tells us that splitting fields of $f$ are unique up to isomorphism.\\

\noindent\textbf{Theorem.} Let $E$ be a splitting field of $f \in F[x]$. The extension $E/F$ is Galois.\\

Along with the previous theorem, we can now define the Galois group of a polynomial $f\in F[x]$ to be $\mbox{Aut}(E/F)$ where $E$ is some splitting field of $f$. If $E$ and $E'$ are two splitting fields of $f$, then $E \simeq E'$, and hence $\mbox{Aut}(E/F) \simeq \mbox{Aut}(E'/F)$. Therefore, we can let $G_f = \mbox{Aut}(E/F)$ denote the Galois group of $f$ over $F$.

\end{document}