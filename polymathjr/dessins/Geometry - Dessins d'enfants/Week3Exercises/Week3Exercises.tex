% document setup
\documentclass{article}
\usepackage[utf8]{inputenc}

% notational packages
\usepackage{amsfonts}
\usepackage{amsmath}
\usepackage{amssymb}
\usepackage{amsthm}
\usepackage{physics}
\usepackage{wasysym}

% other useful packages
\usepackage{graphicx}
\usepackage{natbib}
\usepackage{graphicx}
\usepackage[margin=1in]{geometry}
\usepackage[colorlinks=true,linkcolor=magenta]{hyperref}%
\usepackage{url}
\usepackage{tikz}
\usetikzlibrary{graphs,graphs.standard}
\usepackage{caption}
\usepackage{subcaption}
\usetikzlibrary{decorations.markings}
\usepackage[bottom]{footmisc}
\usepackage{pgfplots}
\usepgfplotslibrary{polar}
\pgfplotsset{compat=newest}
\usetikzlibrary{calc}
\usepackage{quiver}

% basic theorem environment
\theoremstyle{definition}
\newtheorem{theorem}{Theorem}
\newtheorem{definition}{Definition}
\newtheorem{lemma}{Lemma}
\newtheorem{corollary}{Corollary}

% custom theorems
\newtheorem{probleminner}{Problem}
\newenvironment{problem}[1]{%
  \renewcommand\theprobleminner{#1}%
  \probleminner
}{\endprobleminner}
\newtheorem{exercise}{Exercise}
\newtheoremstyle{claim}% name
  {\topsep}% space above
  {\topsep}% space below
  {}% body font
  {}% indent amount
  {\itshape}% theorem head font
  {:}% punctuation after theorem head
  {.5em}% space after theorem head
  {\thmname{#1}\thmnumber{ #2}\thmnote{ (#3)}}% theorem head spec
\theoremstyle{claim}
%\newtheorem{claim}{Claim}
\newtheorem*{claim*}{Claim}
\newtheorem*{example*}{Example}


\newtheoremstyle{named}{}{}{\itshape}{}{\bfseries}{.}{.5em}{\thmnote{#3}#1}
\theoremstyle{named}
\newtheorem*{namedtheorem}{}


% title info
\title{Polymath Jr.\ 2024 Dessins Project - Exercises 2 - Solutions}
\author{Ajmain Yamin, Meiling Laurence}
\date{July 3 2024}

\begin{document}

\maketitle

\section{Wiman Surfaces}

\begin{exercise}  Let $p$ and $q$ be two distinct points on the boundary of $P_m$ which are related by a side pairing.
\begin{enumerate}
\item[(a)] Prove that $p$ is a vertex of $P_m$ if and only if $q$ is.  Moreover, prove that if $p$ is the $j^\text{th}$ vertex $p = v_j$, then $q$ must either be the $(j+m/2-1)^\text{th}$ vertex or the $(j-m/2+1)^\text{th}$ vertex.
\item[(b)] Prove that if $p$ is not a vertex of $P_m$ then $q = T_j(p)$ where $j$ is the unique index $j\in \mathbb{Z}/m\mathbb{Z}$ such that $p$ is contained in the $j^\text{th}$ side $s_j$ of $P_m$.
\end{enumerate}
\end{exercise}

\begin{proof}[Solution to \textup{(a)}]
Suppose that \( p \) is a vertex. Then, \( p = v_j \) for some \( j  \in \{0, 1, \dots, m-1\}\). It is given that \(q = p + \lambda_\ell\), where \(\lambda_\ell = v_{\ell + \frac{m}{2} - 1} - v_\ell\) for some \( \ell  \in \{0, 1, \dots, m-1\}\).  So, \( q = v_j + v_{\ell + \frac{m}{2} - 1} - v_\ell\). Note that for any point $p$ located on the $j^\text{th}$ side, the $j + 1^\text{th}$ side-pairing translation is the translation that will take $p$ to another point on the boundary. Since $p$ is located on both the $j^\text{th}$ side and the $j - 1^\text{th}$ side, we have either \(\ell = j\) or \(\ell = j + 1\). Then, either \(q = v_j + v_{j  + \frac{m}{2} - 1} - v_j = v_{j + \frac{m}{2} - 1}\) or \(q = v_j + v_{j + \frac{m}{2}} - v_{j + 1} = v_{j + \frac{m}{2} + 1} = v_{j - \frac{m}{2} + 1}\). So, if $p$ is the $j^\text{th}$ vertex $p = v_j$, then $q$ must either be the $(j+m/2-1)^\text{th}$ vertex or the $(j-m/2+1)^\text{th}$ vertex. Conversely, by a symmetric argument, if q is a vertex, then p is a vertex.
\end{proof}

\begin{proof}[Solution to \textup{(b)}]
TYPE IN HERE
\end{proof}

\begin{exercise}
Describe all equivalence classes in $P_m/{\sim}$. Prove that $P_m/{\sim}$ is a topological surface.
\end{exercise}

\begin{exercise}
Using this map $M_k :R_k \hookrightarrow P_{2k}/{\sim}$, we can compute the genus of topological surface $P_{m}/{\sim}$.
\begin{enumerate}
    \item[(a)] Prove that $M_k$ is indeed a topological map.  How many faces does it have?  How many sides per face?
    \item[(b)] Compute the genus of $P_{2k}/{\sim}$ by applying the \emph{Euler characteristic} formula, which states that for any topological map $M$, with vertex set $V$, edge set $E$, and face set $F$, the genus $g$ of $M$ is determined by $$\abs{V} - \abs{E} + \abs{F} = 2 - 2g.$$
    \item[(c)] Conclude that the genus of $P_m/{\sim}$ is $g$ if and only if $m = 4g$ or $m=4g+2$.
\end{enumerate}
\end{exercise}

\begin{exercise}\label{exer:charts}
Let's construct a Riemann surface structure on the topological surface $S = P_m/{\sim}$.  

Write $\pi$ to denote the quotient map $\pi : P_m \rightarrow P_m/{\sim}$.
Write $F$ to denote the interior face of $P_m$.  Let $\phi_F: \pi(F) \rightarrow F$ be the function given by the identity function on $F$.
For each side $s_j$ of $P_m$, $j\in\mathbb{Z}/m\mathbb{Z}$, let $U_j$ denote the open subset of $P_m$ given by $$U_j = \{\rho \cdot p \mid p\in s_j\setminus\{v_j,v_{j+1}\} \text{ and } 0 < \rho \leq 1 \}.$$ 
%and $$V_j = \{t\cdot p \mid p\in s_j\setminus\{v_j,v_{j+1}\} \text{ and } t \in (-1,0)\cup(0,1)\}$$ 
If $s_j$ and $s_{j'}$ are opposite sides of $P_m$, let $\phi_j: \pi(U_j \sqcup U_{j'}) \rightarrow \Im(\phi_j)$ denote the function given by 
$$\phi_j(z) = \begin{cases}
    z & \text{if } z\in U_j\\
    z + \lambda_{j'} & \text{if } z \in U_{j'}
\end{cases}.$$

\begin{itemize}
    \item[(a)] Prove that $\phi_F, \phi_0,\phi_1,\dots , \phi_{m-1}$ are all charts on $S$.  Prove that the transition function between any two of these charts is given by translations, and is thus holomorphic.
    \item[(b)] Define one or two more charts to give $S$ the structure of a Riemann surface.
\end{itemize}
\end{exercise}

\begin{exercise}\label{exer:cyclicauts}
Consider the action of $\mathbb{Z}/m\mathbb{Z}$ on $P_m/{\sim}$ given by rotating $P_m$ counterclockwise by multiples of $2\pi/m$ radians. Verify that these are biholomorphisms of the Riemann surface constructed in Exercise \ref{exer:charts}.
\end{exercise}

\begin{exercise}
Let $\mathcal{D}_k$ be the \emph{bipartification dessin} of the of the map $M_k : R_k \hookrightarrow P_{2k}/{\sim}$. 
\begin{enumerate}
    \item[(a)] Prove that $C_g^\text{I} : y^2 = x^{2g+1} - 1$, $C_g^\text{II} : y^2 = x(x^{2g} - 1)$ define Riemann surfaces isomorphic to $W_g^\text{I}$, $W_g^\text{II}$.% resp.
    \item[(b)] Determine rational functions $\beta_g^\text{I}$ and $\beta_g^\text{II}$ on $C_g^\text{I}$ and $C_g^\text{II}$ such that $(C_g^\text{I}, \beta_g^\text{I})$ and $(C_g^\text{II}, \beta_g^\text{II})$ are \emph{Belyi pairs} for $\mathcal{D}_{4g+2}$ and $\mathcal{D}_{4g}$, respectively.
\end{enumerate}
\end{exercise}

\end{document}
