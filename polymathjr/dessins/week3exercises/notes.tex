\documentclass{amsart}
\usepackage{amscd, amssymb, amsmath, amsthm}
\usepackage{amsfonts,enumerate}
\usepackage{latexsym,,txfonts}
\usepackage{graphics,graphicx,psfrag}
\usepackage{hyperref}
\usepackage[utf8]{inputenc}
\usepackage[T1]{fontenc}
\usepackage{textcomp}
\usepackage{babel}



\pdfsuppresswarningpagegroup=1

\newtheorem{theorem}{Theorem}[section]
\newtheorem{lemma}[theorem]{Lemma}
\newtheorem{prop}[theorem]{Proposition}
\newtheorem{cor}[theorem]{Corollary}
\newtheorem{conj}[theorem]{Conjecture}

\theoremstyle{definition}
\newtheorem{definition}[theorem]{Definition}
\newtheorem{example}[theorem]{Example}
\newtheorem{exercise}[theorem]{Exercise}
\newtheorem{problem}[theorem]{Problem}
\newtheorem{question}[theorem]{Question}

\theoremstyle{remark}
\newtheorem*{remark}{Remark}

\newcommand{\SL}{{\mathrm{SL}}}
\newcommand{\mobgp}{{\mathrm{PSL}_2(\mathbb{C})}}
\newcommand{\Hom}{{\mathrm{Hom}}}
\newcommand{\id}{{\mathrm{id}}}
\newcommand{\Mod}{{\mathrm{Mod}}}
\newcommand{\ud}{{\mathrm{d}}}
\newcommand{\Vol}{{\mathrm{Vol}}}
\newcommand{\Area}{{\mathrm{Area}}}
\newcommand{\diam}{{\mathrm{diam}}}

\newcommand{\reg}{{\mathtt{reg}}}
\newcommand{\geo}{{\mathtt{geo}}}

\newcommand{\tori}{{\mathcal{T}}}
\newcommand{\cpn}{{\mathtt{c}}}
\newcommand{\pat}{{\mathtt{p}}}




\begin{document}
    \begin{exercise}[4.b]
        Define one or two more charts to give $S$ the structure of a Riemann
        surface.
    \end{exercise}
    \begin{proof}
        Take a small ball around each vertex of $P_m$ and intersect it
        with $P_m$. Let $V_i$ denote the open set at vertex $v_i$.
        Define a map $\psi_i \colon V_i \to \mathbb{C}$ by
        $\psi_i (z) = f_i \left( z - v_i \right) $, where
        $f_j \colon \mathbb{C} \to \mathbb{C}$ is given by
        \[
        f_j(re^{i \theta} ) = r e^{i \frac{2 \pi r_j}{n}} \left[ 
        e^{- i \left( \frac{(n-2 + 2j)\pi}{n} \right) } e^{i \theta}\right]^{\frac{2}{n-2}}
        \] 
        where $r_j$ is the smallest positive integer such that
        $r_j \left( \frac{m}{2}-1 \right) \equiv j \mod{m} $
    \end{proof}












    %\bibliography{../refs.bib}
\end{document}
