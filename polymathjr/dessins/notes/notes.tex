\documentclass[reqno]{amsart}
\usepackage{amscd, amssymb, amsmath, amsthm}
\usepackage{amsfonts,enumerate}
\usepackage{latexsym,,txfonts}
\usepackage{graphics,graphicx,psfrag}
\usepackage{hyperref}
\usepackage[utf8]{inputenc}
\usepackage[T1]{fontenc}
\usepackage{textcomp}
\usepackage{babel}



\pdfsuppresswarningpagegroup=1

\newtheorem{theorem}{Theorem}[section]
\newtheorem{lemma}[theorem]{Lemma}
\newtheorem{prop}[theorem]{Proposition}
\newtheorem{cor}[theorem]{Corollary}
\newtheorem{conj}[theorem]{Conjecture}

\theoremstyle{definition}
\newtheorem{definition}[theorem]{Definition}
\newtheorem{example}[theorem]{Example}
\newtheorem{exercise}[theorem]{Exercise}
\newtheorem{problem}[theorem]{Problem}
\newtheorem{question}[theorem]{Question}

\theoremstyle{remark}
\newtheorem*{remark}{Remark}

\newcommand{\SL}{{\mathrm{SL}}}
\newcommand{\mobgp}{{\mathrm{PSL}_2(\mathbb{C})}}
\newcommand{\Hom}{{\mathrm{Hom}}}
\newcommand{\id}{{\mathrm{id}}}
\newcommand{\Mod}{{\mathrm{Mod}}}
\newcommand{\ud}{{\mathrm{d}}}
\newcommand{\Vol}{{\mathrm{Vol}}}
\newcommand{\Area}{{\mathrm{Area}}}
\newcommand{\diam}{{\mathrm{diam}}}
\newcommand{\Aut}{{Aut}}
\newcommand{\Homeo}{{Homeo}}

\newcommand{\reg}{{\mathtt{reg}}}
\newcommand{\geo}{{\mathtt{geo}}}

\newcommand{\tori}{{\mathcal{T}}}
\newcommand{\cpn}{{\mathtt{c}}}
\newcommand{\pat}{{\mathtt{p}}}




\begin{document}
    \begin{problem}[]
        Write down explicit $2 \times 2$ matrices that are generators for
        a Fuchsian group uniformizing the Wiman surfaces of Type I and Type
        II from the exercises from week 3.
    \end{problem}

    \begin{proof}
        We will construct the transformation as a composition of
        complex conjugation, followed by rotation and then inversion.\\
        Firstly, we wish to find the point of inversion: $\alpha = 
        a e^{i \theta}$.
        We have
        \begin{align*}
            \frac{L}{\sin \left( \frac{\pi}{2n} \right) }
            &= \frac{a}{\sin \left( \frac{(n+1) \pi}{2n} \right) }, \quad
            a^2 = 1 + L^2\\
            \implies a &= \sqrt{\frac{\sin^2 \left( \frac{(n+1)\pi}{2n} \right) }{
            \sin^2 \left( \frac{(n+1) \pi}{2n} \right) -
    \sin^2 \left( \frac{\pi}{2n} \right) }} \tag{$\Omega$}
        \end{align*}
        And clearly $\theta = \frac{\pi}{2n}$. Now the inversion
        at $\alpha$ is given by
        \[
        \rho (z) = \frac{\alpha \overline{z} - 1}{\overline{z} - 
        \overline{\alpha}}
        \] 
        so all together we get the hyperbolic translation to be
        \[
        T(z) = \rho \left( e^{- \frac{(n-1)\pi i}{n}} \overline{z} \right) 
        = \frac{\alpha e^{\frac{(n-1) \pi i}{n}} z - 1}{
        e^{\frac{(n-1) \pi i}{n}} z - \overline{\alpha}}
        = \frac{a e^{\frac{\pi i}{2n}} e^{\frac{(n-1) \pi i}{n}} z - 1}{
        e^{\frac{(n-1) \pi i}{n} }z - a e^{- \frac{i \pi}{2n}}}
        = \frac{-a z - e^{\frac{i \pi}{2n}}}{-e^{\frac{- i \pi}{2n}} z-
        a}
        \] 
        Hence
        \[
        M(T) = 
        \begin{pmatrix} 
            a & e^{\frac{i \pi}{2n}}\\
            e^{- \frac{i \pi}{2n}} & a
        \end{pmatrix} 
        \] 
        Since $T_j = e^{\frac{(j-1) i \pi}{n}} T\left( 
        e^{- \frac{(j-1) i \pi}{n}} \right) $ we have
        \begin{align*}
            M(T_j) 
            &= \begin{pmatrix} e^{\frac{(j-1) i \pi}{n}} & 0 \\
            0 & 1\end{pmatrix} 
            \begin{pmatrix} a & e^{\frac{i \pi}{2n}}\\
            e^{- \frac{i \pi}{2n}} & a\end{pmatrix} 
                \begin{pmatrix} e^{- \frac{(j-1) i \pi}{n}} & 0\\
                0 & 1 \end{pmatrix} \\
            &= \begin{pmatrix} a & e^{\frac{(2j-1) i \pi}{2n}}\\
                e^{- \frac{(2j-1) i \pi}{2n}} & a
            \end{pmatrix} 
        \end{align*}
        So $\Gamma_n = \left< T_1, \ldots, T_n \right>$ is a Fuchsian group,
        and
        we have
        $W_{2n} \approx \mathbb{D} / \Gamma_n$.
    \end{proof}













    \newpage

        Let $T (z) = \frac{az + b}{cz +d}$.\\
        Let $f \colon H \to D$ be the map
        $z\mapsto \frac{z-i}{z+i}$ and let
        $f^{-1}(z) = - \frac{i \left( 1+z \right) }{z-1}$ be the inverse.
        It is indeed easy to check that $f^{-1}$ takes
        $\partial D$ to $\mathbb{R} \cup  \left\{ \infty \right\} $.
        Now $T(z) = z$ if and only if
        $c z^2 + (d-a) z - b = 0$. We want the fixed points to be
        $f^{-1}\left( e^{\frac{i \pi}{2n}} \right) =-
        \frac{i \left( 1 + e^{\frac{i \pi}{2n}} \right) }{ e^{\frac{i
        \pi}{2n}}-1} 
         $ and
        $f^{-1}\left( e^{\frac{(2n+1) i \pi}{2n}} \right) 
        =- \frac{i \left( 1+ e^{\frac{(2n+1) i \pi}{2n}} \right) }{
            e^{\frac{(2n+1) i \pi}{2n}}-1}$
            (which lie in $\mathbb{R}$ ), so we have the equation
            \begin{align*}
                0 &= \left( z+ \frac{i \left( 1 + e^{\frac{i \pi}{2n}} \right) }{
                         e^{\frac{i \pi}{2n}}-1} \right) \left( z+
        \frac{i \left( 1 + e^{\frac{i \pi}{2n}} \right) }{
                        e^{\frac{i \pi}{2n}}-1}
    \right)\\
                  &=
                  z^2 +i \left[  \frac{ \left( 1 + e^{\frac{i \pi}{2n}} \right) }{
                        e^{\frac{i \pi}{2n}}-1} + 
                        \frac{ \left( 1+ e^{\frac{(2n+1) i \pi}{2n}} \right) }{
            e^{\frac{(2n+1) i \pi}{2n}}-1}  \right] z
            - \frac{ \left( 1 + e^{\frac{i \pi}{2n}} \right) }{
                        e^{\frac{i \pi}{2n}}-1}  
                        \frac{ \left( 1+ e^{\frac{(2n+1) i \pi}{2n}} \right) }{
            e^{\frac{(2n+1) i \pi}{2n}}-1}\\
                  &= 
                  z^2 +i \left[  \frac{ \left( 1 + e^{\frac{i \pi}{2n}} \right) }{
                        e^{\frac{i \pi}{2n}}-1} + 
                        \frac{ \left( 1+ e^{\frac{(2n+1) i \pi}{2n}} \right) }{
            e^{\frac{(2n+1) i \pi}{2n}}-1}  \right] z
            -1
                    \end{align*}
            Now, we must choose $a,b,c,d$ suitably so that
            $ad-bc = 1$. Now we have the system of equations
            \begin{align*}
                ad 
                &= 1 + b^2\\
                d-a 
                &=  b i 
                \left[  \frac{ \left( 1 + e^{\frac{i \pi}{2n}} \right) }{
                        e^{\frac{i \pi}{2n}}-1} + 
                        \frac{ \left( 1+ e^{\frac{(2n+1) i \pi}{2n}} \right) }{
            e^{\frac{(2n+1) i \pi}{2n}}-1}  \right]
            \end{align*}
            Denote the right hand side of the last equation
            by $b C$. 
            \begin{align*}
                a^2 C 
                &= 2 \implies a =\sqrt{\frac{2}{C}} \\
                d &= C + \sqrt{\frac{2}{C}} 
            \end{align*}
            Hence
            \[
            a =  \sqrt{\frac{2}{C}} , \quad
            d = C + \sqrt{\frac{2}{C}} 
            \] 
            Hence we have
            \[
            T(z) = \frac{  \sqrt{\frac{2}{C}} z + 1}{z + C \pm 
            \sqrt{\frac{2}{C}} }, \text{where} \quad
            C =
            i 
                \left[  \frac{ \left( 1 + e^{\frac{i \pi}{2n}} \right) }{
                        e^{\frac{i \pi}{2n}}-1} + 
                        \frac{ \left( 1+ e^{\frac{(2n+1) i \pi}{2n}} \right) }{
            e^{\frac{(2n+1) i \pi}{2n}}-1}  \right]
            \] 
            Put in a different way, if we let
            $ r_1 = f^{-1}\left( e^{ \frac{i \pi}{2n}} \right)  =-
\frac{i \left( 1 + e^{\frac{i \pi}{2n}} \right) }{
                         e^{\frac{i \pi}{2n}}-1}
            $ and
            $r_2 = f^{-1} \left( e^{\frac{(2n+1)i \pi}{2n}} \right)  =
- \frac{i \left( 1+ e^{\frac{(2n+1) i \pi}{2n}} \right) }{
            e^{\frac{(2n+1) i \pi}{2n}}-1}
            $, then
            $r_1 + r_2 = -C$, so
            \[
            T(z) = \frac{ \sqrt{- \frac{2}{ r_1 + r_2}}z +1}{
            z  \sqrt{-\frac{2}{r_1 + r_2}}- r_1 - r_2 },
            \quad G = \begin{pmatrix} 
                 \sqrt{-\frac{2}{r_1 +r_2}} & 1 \\
                1 &  \sqrt{-\frac{2}{r_1 + r_2}} -r_1 -r_2 
            \end{pmatrix} 
            \] 
            To convert it to a transformation of the unit disk instead,
            we have
            \[
            S(z) = f \circ T \circ f^{-1} (z)
            \] 
            so since
            $M(T) = G$, $M(f) = \begin{pmatrix} 1 & -i \\ 1 & i \end{pmatrix} $
            and
            $M \left( f^{-1} \right) =
            \begin{pmatrix} -i & -i\\ 1 & -1 \end{pmatrix} $, we have
            \begin{align*}
                M(S) 
                &= M(f) M(T) M(f^{-1})\\
                &= \begin{pmatrix} 1 & -i\\ 1 & i \end{pmatrix} 
                \begin{pmatrix} 
                     \sqrt{- \frac{2}{r_1+r_2}} & 1\\
                    1 & \sqrt{- \frac{2}{r_1+r_2} } -r_1 -r_2 
                \end{pmatrix} 
                \begin{pmatrix} -i & -i\\ 1 & -1 \end{pmatrix} \\
                                   &=
                \begin{pmatrix} 
                    -i \left(  2 \sqrt{- \frac{2}{r_1+r_2}} -r_1-r_2 \right) &
                    -i \left( r_1+r_2 \right) -2\\
                - i \left( r_1+r_2 \right) +2 & -i \left( 
                2 \sqrt{- \frac{2}{r_1+r_2}} - r_1 - r_2\right) 
                \end{pmatrix}\\
            \end{align*}
            Now all of the other side pairings $S_2, \ldots, S_n$ are
            obtained by conjugation with rotations:
            \begin{align*}
                M(S_i) 
                &= 
            \begin{pmatrix} e^{\frac{(i-1) \pi i}{n}} & 0 \\
            0 & 1 \end{pmatrix} 
\begin{pmatrix} 
                    -i \left( 2 \sqrt{- \frac{2}{r_1+r_2}} -r_1-r_2 \right) &
                    -i \left( r_1+r_2 \right) -2\\
                - i \left( r_1+r_2 \right) +2 & -i \left( 
                2 \sqrt{- \frac{2}{r_1+r_2}} - r_1 - r_2\right) 
                \end{pmatrix}
            \begin{pmatrix} e^{-\frac{(i-1) \pi i}{n}} & 0 \\
            0 & 1 \end{pmatrix} \\
            &= \begin{pmatrix} -i \left( 2 \sqrt{- \frac{2}{r_1+r_2}} 
            - r_1 - r_2\right) & e^{\frac{(i-1) \pi i}{n}} \left[ 
        -i \left( r_1+r_2 \right) -2\right] \\
            e^{- \frac{(i-1) \pi i}{n}} \left[ -i \left( r_1+r_2 \right) +2 \right] 
                               & -i \left( 
                           2 \sqrt{- \frac{2}{r_1+r_2}} -r_1-r_2\right) 
            \end{pmatrix} 
            \end{align*}



            \newpage
            Hence the other side pairing translations are given by
            $T_2, T_3, \ldots, T_{n}$ where
            \[
            T_i (z) = f^{-1} \left( e^{\frac{(i-1) \pi i}{n}}
            T \left( e^{- \frac{(i-1) \pi i}{n}} f(z)\right) \right)
            \] 
            and since the association
            $GL(2,\mathbb{C}) \to \Homeo \left( \mathbb{C}^{*} \right) $
            is a group homomorphism, it suffices to look at
            the associated matrices. Now
            $M(T) = G$, $M(f) = \begin{pmatrix} 1 & -i \\ 1 & i \end{pmatrix} $
            and
            $M \left( f^{-1} \right) =
            \begin{pmatrix} -i & -i\\ 1 & -1 \end{pmatrix} $, so
            \begin{align*}
                M(T_i) 
                &= 
                \begin{pmatrix} -i & -i \\ 1 & -1 \end{pmatrix} 
                \begin{pmatrix} \sqrt{- \frac{2}{r_1+r_2}} 
                & e^{\frac{(i-1) \pi i}{n}} \\
                1 & e^{\frac{(i-1) \pi i}{n}} \left[ 
            \sqrt{- \frac{2}{r_1 +r_2}}  - r_1 - r_2 \right]   \end{pmatrix} 
                \begin{pmatrix} 1 & -i \\ 1 & i \end{pmatrix}
            \end{align*}
            where again
$r_1 = f^{-1}\left( e^{\frac{i \pi}{2n}} \right) =-
        \frac{i \left( 1 + e^{\frac{i \pi}{2n}} \right) }{ e^{\frac{i
        \pi}{2n}}-1} 
         $ and
        $r_2 = f^{-1}\left( e^{\frac{(2n+1) i \pi}{2n}} \right) 
        =- \frac{i \left( 1+ e^{\frac{(2n+1) i \pi}{2n}} \right) }{
            e^{\frac{(2n+1) i \pi}{2n}}-1}$. Then
            $\Gamma_n = \left< T_1, \ldots, T_n \right>$ is a Fuchsian group
            such that
            $\mathbb{H} / \Gamma_n \approx W_{2n}$.





    %\bibliography{../refs.bib}
\end{document}
