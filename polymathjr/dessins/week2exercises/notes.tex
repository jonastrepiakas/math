\documentclass{amsart}
\usepackage{amscd, amssymb, amsmath, amsthm}
\usepackage{amsfonts,enumerate}
\usepackage{latexsym,,txfonts}
\usepackage{graphics,graphicx,psfrag}
\usepackage{hyperref}
\usepackage[utf8]{inputenc}
\usepackage[T1]{fontenc}
\usepackage{textcomp}
\usepackage{babel}



\pdfsuppresswarningpagegroup=1

\newtheorem{theorem}{Theorem}[section]
\newtheorem{lemma}[theorem]{Lemma}
\newtheorem{prop}[theorem]{Proposition}
\newtheorem{cor}[theorem]{Corollary}
\newtheorem{conj}[theorem]{Conjecture}

\theoremstyle{definition}
\newtheorem{definition}[theorem]{Definition}
\newtheorem{example}[theorem]{Example}
\newtheorem{exercise}[theorem]{Exercise}
\newtheorem{problem}[theorem]{Problem}
\newtheorem{question}[theorem]{Question}

\theoremstyle{remark}
\newtheorem*{remark}{Remark}

\newcommand{\SL}{{\mathrm{SL}}}
\newcommand{\mobgp}{{\mathrm{PSL}_2(\mathbb{C})}}
\newcommand{\Hom}{{\mathrm{Hom}}}
\newcommand{\id}{{\mathrm{id}}}
\newcommand{\Mod}{{\mathrm{Mod}}}
\newcommand{\ud}{{\mathrm{d}}}
\newcommand{\Vol}{{\mathrm{Vol}}}
\newcommand{\Area}{{\mathrm{Area}}}
\newcommand{\diam}{{\mathrm{diam}}}

\newcommand{\reg}{{\mathtt{reg}}}
\newcommand{\geo}{{\mathtt{geo}}}

\newcommand{\tori}{{\mathcal{T}}}
\newcommand{\cpn}{{\mathtt{c}}}
\newcommand{\pat}{{\mathtt{p}}}




\begin{document}
 \begin{problem}[]
     Show that $K_5$ is not planar.
 \end{problem}   
\begin{proof}
    We'll make use of the theorem
    \begin{theorem}[]
        $\left| V \right| - \left| E \right| + \left| F \right|  = 2- 2g$.
    \end{theorem}
        If $K_5$ is planar, then $g = 0$, so
        $\left| V \right| - \left| E \right|  + \left| F \right|  = 2$.
        We count the number of vertices and edges:
        \begin{align*}
            \left| V \right|  &= 5\\
            \left| E \right| &= \begin{pmatrix} 5 \\2 \end{pmatrix} = 
            10
        .\end{align*}
        Hence $\left| F \right|  = 2 - 5 + 10 = 7$.
        Note that in a cycle of $\rho^{*}$, we must have at least $3$ sides,
        since if
        $\rho^{*} \left( v,w \right) = \left( w, \rho_w (v) \right) $, and
        $\rho^{*} \left( w, \rho_w (v) \right)  = 
        \left( \rho_w(v) , \rho_{\rho_w (v)}\left( \rho_w(v) \right)  \right) 
        = \left( v,w \right) $, so $\rho_w(v) = v$, but $\rho_w$ cannot
        fix any vertices as it is a cyclic permutation. So
        if each face of $K_5$ must have at least $3$ edges, then
        there must be at least $7\cdot 3 = 21$ sides.
\end{proof}

\begin{proof}
    \[
    \sigma^{*} \left( gx, gy \right) = \left( gy,
    \sigma_{gy} (gx) \right) = 
    \left( gy, \rho_{gy}^{(g)}(gx) \right) 
    = \left( gy, g \rho_{y} g^{-1} \left( gx \right)  \right) 
    = \left( gy, g g_y (x) \right) = \left( gy, gz \right) 
    \] 
    where $\left( x,y,z, \ldots \right) $ occurs in a face.\\
    The second part of the exercise follows from noting that
    $\sigma^{\left( g^{-1} \right) } = \rho$.
\end{proof}


    %\bibliography{../refs.bib}
\end{document}
