\documentclass[reqno]{amsart}

\usepackage[margin=2.5cm]{geometry}
\usepackage[pdftex]{graphicx}
\usepackage[utf8]{inputenc}
\usepackage[T1]{fontenc}
\usepackage{textcomp}
\usepackage{babel}
\usepackage{amsmath, amssymb, amsthm, amscd}
\usepackage[colorlinks=true,linkcolor=blue]{hyperref}
\usepackage{float}
\usepackage{mathrsfs}
%\usepackage{enumitem}
%% for identity function 1:
\usepackage{bbm}
%%For category theory diagrams:
\usepackage{tikz-cd}
%%For code (e.g. python) in latex:
%\usepackage{listings}
%
%Usage: 
%\begin{lstlisting}[language=Python]
%\end{lstlisting}

\newcommand{\incfig}[2][1]{%
\def\svgwidth{#1\columnwidth}
\import{./figures/}{#2.pdf_tex}
}


\theoremstyle{plain}% default
\newtheorem{theorem}{Theorem}[section]
\newtheorem{lemma}[theorem]{Lemma}
\newtheorem{proposition}[theorem]{Proposition}
\newtheorem{corollary}[theorem]{Corollary}


\theoremstyle{definition}
\newtheorem{definition}[theorem]{Definition}
\newtheorem{example}[theorem]{Example}
\newtheorem{exercise}[theorem]{Exercise}
\newtheorem{problem}[theorem]{Problem}


\theoremstyle{remark}
\newtheorem*{remark}{Remark}
\newtheorem*{note}{Note}
\newtheorem*{solution}{Solution}






% figure support
\usepackage{import}
\usepackage{xifthen}
\pdfminorversion=7
\usepackage{pdfpages}
\usepackage{transparent}

\pdfsuppresswarningpagegroup=1

\setlength\parindent{0pt}

\newcommand{\qedwhite}{\hfill \ensuremath{\Box}}

%Inequalities
\newcommand{\cycsum}{\sum_{\mathrm{cyc}}}
\newcommand{\symsum}{\sum_{\mathrm{sym}}}
\newcommand{\cycprod}{\prod_{\mathrm{cyc}}}
\newcommand{\symprod}{\prod_{\mathrm{sym}}}

%Linear Algebra

\DeclareMathOperator{\Span}{span}
\DeclareMathOperator{\Ima}{Im}
\DeclareMathOperator{\diag}{diag}
\DeclareMathOperator{\Ker}{Ker}
\DeclareMathOperator{\ob}{ob}
\DeclareMathOperator{\sk}{sk}
\DeclareMathOperator{\Vect}{Vect}
\DeclareMathOperator{\Set}{Set}
\DeclareMathOperator{\Group}{Group}
\DeclareMathOperator{\Ring}{Ring}
\DeclareMathOperator{\Ab}{Ab}
\DeclareMathOperator{\Top}{Top}
\DeclareMathOperator{\hTop}{hTop}
\DeclareMathOperator{\Htpy}{Htpy}
\DeclareMathOperator{\Cat}{Cat}
\DeclareMathOperator{\CAT}{CAT}
\DeclareMathOperator{\Cone}{Cone}
\DeclareMathOperator{\dom}{dom}
\DeclareMathOperator{\cod}{cod}
\DeclareMathOperator{\Aut}{Aut}
\DeclareMathOperator{\Mat}{Mat}
\DeclareMathOperator{\Fin}{Fin}
\DeclareMathOperator{\rel}{rel}
\DeclareMathOperator{\Int}{Int}
\DeclareMathOperator{\sgn}{sgn}
\newcommand{\SL}{{\mathrm{SL}}}
\newcommand{\mobgp}{{\mathrm{PSL}_2(\mathbb{C})}}
\newcommand{\Hom}{{\mathrm{Hom}}}
\newcommand{\id}{{\mathrm{id}}}
\newcommand{\Mod}{{\mathrm{Mod}}}
\newcommand{\ud}{{\mathrm{d}}}
\newcommand{\Vol}{{\mathrm{Vol}}}
\newcommand{\Area}{{\mathrm{Area}}}
\newcommand{\diam}{{\mathrm{diam}}}

\newcommand{\reg}{{\mathtt{reg}}}
\newcommand{\geo}{{\mathtt{geo}}}

\newcommand{\tori}{{\mathcal{T}}}
\newcommand{\cpn}{{\mathtt{c}}}
\newcommand{\pat}{{\mathtt{p}}}


%Row operations
\newcommand{\elem}[1]{% elementary operations
\xrightarrow{\substack{#1}}%
}

\newcommand{\lelem}[1]{% elementary operations (left alignment)
\xrightarrow{\begin{subarray}{l}#1\end{subarray}}%
}

%SS
\DeclareMathOperator{\supp}{supp}
\DeclareMathOperator{\Var}{Var}

%NT
\DeclareMathOperator{\ord}{ord}

%Alg
\DeclareMathOperator{\Rad}{Rad}
\DeclareMathOperator{\Jac}{Jac}

\DeclareMathAlphabet{\pazocal}{OMS}{zplm}{m}{n}
\newcommand{\unif}{\pazocal{U}}

\begin{document}
\begin{problem}[2.2. Algebra Structure on $C_p^{\infty}$]
        Define carefully addition, multiplication, and scalar
        multiplication on $C_p^{\infty}$. Prove that addition
        in $C_p^{\infty}$ is commutative.
\end{problem}

\begin{solution}
    Suppose $\left[ \left( f,U \right) \right], \left[ \left( g,V \right)
    \right]
    \in C_{p}^{\infty}$. Define
    $\left[ \left( f,U \right) \right] + \left[ 
    \left( g,V \right) \right]
    = \left[ \left( f+g , U \cap V \right) \right] ,
    \left[ \left( f,U \right)  \right] 
    \cdot \left[ \left( g,V \right)  \right] 
    = \left[ \left( f \cdot g, U \cap V \right)  \right]$ and
    $\lambda \left[ \left( f,U \right)  \right] 
    = \left[ \left( \lambda f, U \right)  \right] $
    for $\lambda \in \mathbb{R}$ as scalar.\\
    We must show that these is well-defined. Firstly, 
    since $p \in U ,V$ we have $p \in U \cap V$. Now, since
    $f  \colon U \to \mathbb{R}$ and
    $g  \colon V \to \mathbb{R}$ are smooth, we have that both are
    smooth on  $U \cap V$ as this is an open subspace of
    $U$ and $V$. Hence $f+g, f\cdot g
    \in C^{\infty}\left( U \cap V \right) $ and
    $\lambda f \in C^{\infty} \left( U \right) $ as
    $C^{\infty}\left( W \right) $ is a ring for any open set $W$. Hence
    $\left( f+g, U \cap V \right)  $ is in some equivalence class
    in $C_p^{\infty}$.\\
    Suppose now $\left[ \left( f,U \right)  \right] 
    = \left[ \left( \tilde{f}, \tilde{U} \right)  \right] $ and
    $\left[ \left( g, V \right)  \right] = 
    \left[ \left( \tilde{g}, \tilde{V} \right)  \right] $. 
    Then $f = \tilde{f}$ on some $W \subset U \cap \tilde{U}$ and
    $g = \tilde{g}$ on some $S \subset V \cap \tilde{V}$. Thus
    $f+g = \tilde{f}+ \tilde{g}$ and
    $f \cdot g = \tilde{f} \cdot \tilde{g}$ on 
    $W \cap S \subset U \cap \tilde{U} \cap V \cap \tilde{V}$, and
    $W \cap S$ is open.
    By definition then
    $\left[ \left( f+g, U \cap V \right)  \right] 
    = \left[ \left( \tilde{f}+ \tilde{g},
    \tilde{U} \cap \tilde{V} \right)  \right] $ and
    $\left[ \left( f\cdot g, U \cap V \right)  \right] 
    = \left[ \left( \tilde{f}\cdot \tilde{g},
    \tilde{U} \cap \tilde{V} \right)  \right] $, so
    addition and multiplication are well-defined.\\
    And similarly since $f = \tilde{f}$ on some $W 
    \subset U \cap \tilde{U}$, we have
    $\lambda f = \lambda \tilde{f}$ on
    $W$, so by definition
    $\left[ \left( \lambda f, U \right)  \right] 
    = \left[ \left( \lambda \tilde{f}, \tilde{U} \right)  \right] $.\\
    \linebreak
    Commutativity of addition (and even multiplication) follows
    from the commutativity of these in $\mathbb{R}$ and the above.
\end{solution}

\begin{problem}[Transformation rule for a wedge product of coverctors,
    3.7.]
    Cuppose two sets of covectors on a vector space
    $V$, $\beta^{1},\ldots, \beta^{k}$ and
    $\gamma^{1},\ldots, \gamma^{k}$ are related by
    \[
    \beta^{i} = \sum_{j=1}^{k} a_{j}^{i} \gamma^{j}, \quad
    i = 1,\ldots,k,
    \] 
    for a $k \times k$ matrix
    $A = \left[ a_{j}^{i} \right] $. Show that
    \[
    \beta^{1} \wedge \ldots \wedge \beta^{k} =
    \left( \det A \right) \gamma^{1} \wedge \ldots \wedge
    \gamma^{k}.
    \] 
\end{problem}

\begin{solution}
    By proposition 3.27, we have
    \[
    \beta^{1} \wedge \ldots \wedge \beta^{k}
    \left( v_1 , \ldots, v_k \right) =
    \det \left[ \beta^{i} \left( v_j \right)  \right] .
    \] 
    Now since 
    $\gamma^{1} \wedge \ldots \wedge \gamma^{k}
    \left( v_1, \ldots, v_k \right) 
    = \det \left[ \gamma^{i} \left( v_j \right)  \right] $,
    we must show that
    $ \left[ \beta^{i} \left( v_j \right)  \right] 
    = \left[ a_{j}^{i} \right] \left[ \gamma^{i}\left( v_j \right)  \right] 
    $. I.e., we must show
    $\beta^{i}\left( v_j \right) 
    = \sum_{r=1}^{k} \alpha^{i}_{r} \gamma^{r}\left( v_j \right) $, but
    this is precisely the definition of $\beta^{i}$. Hence
    $\det \left[ \beta^{i} (v_j) \right] 
    = \det \left( \left[ a_{j}^{i} \right] 
    \left[ \gamma^{i} \left( v_j \right)  \right] \right) 
    = \det A \det \left[ \gamma^{i} (v_j) \right] 
    = \left( \det A \right)  \gamma^{1} \wedge \ldots \wedge
    \gamma^{k} \left( v_1, \ldots, v_k \right) $.
    This shows the desired equality.
\end{solution}

\begin{exercise}[4.4 (Wedge product of a $2$-form with a 
    $1$-form)]
    Let $\omega $ be a $2$-form and $\tau$ a $1$-form
    on $\mathbb{R}^3$. If $X, Y, Z$ are vector fields on $M$, find
    an explicit formula for
    $\left( \omega  \wedge \tau \right) \left( X,Y,Z \right) $ in
    terms of the values of $\omega $ and $\tau$ on the vector
    fields $X,Y,Z$.
\end{exercise}

\begin{solution}
    Let $X = x_1, Y = x_2, Z = x_3$. Then
    \begin{align*}
        \left( \omega  \wedge \tau \right) \left( X,Y,Z \right) 
        &= \frac{1}{2} \sum_{\sigma \in S_3} \left( \sgn \sigma  \right) 
        \omega \left( x_{\sigma 1}, x_{\sigma  2} \right) 
        \tau \left( x_{\sigma  3} \right)\\
        &= \omega \left( X,Y \right) \tau (Z) +
        \omega \left( Y,Z \right) \tau (X)
        + \omega \left( Z,X \right) \tau(Y)
    \end{align*}
\end{solution}

\begin{exercise}[4.4 (Exterior calculus)]
    Suppose the standard coordinates on $\mathbb{R}^3$ are called
    $\rho , \varphi$ and $\theta$. If $x = \rho \sin
    \varphi \cos \theta, y = \rho  \sin \varphi \sin \theta$ and
    $z = \rho  \cos \varphi$, calculate $dx, dy, dz$ and
    $dx \wedge dy \wedge dz$ in terms of $d \rho ,
    d \varphi$ and $d \theta$.
\end{exercise}

\begin{solution}
    We have
    \begin{align*}
        dx &= \frac{\partial x}{\partial \rho }  d \rho  +
    \frac{\partial x}{\partial \varphi}  d \varphi + 
    \frac{\partial x}{\partial \theta} d \theta
    = \sin \varphi \cos \theta d \rho  + 
    \rho  \cos \varphi \cos \theta d \varphi -
    \rho  \sin \varphi \sin \theta\\
    dy
    &= \sin \varphi \sin \theta d \rho  + 
    \rho  \cos \varphi \sin \theta d \varphi + 
    \rho \sin \varphi \cos \theta d \theta\\
    dz 
    &= \cos \varphi d \rho - \rho  \sin \varphi d \varphi
    \end{align*}
    Hence
    \begin{align*}
        dx \wedge dy \wedge dz
        &=  \sin^3 \varphi \cos^2 \theta \rho^2 d \rho \wedge
        d \varphi \wedge d \theta +
        \rho^2  \cos^2 \varphi \cos^2 \theta \sin \varphi
        d \rho  \wedge d \varphi \wedge d \theta\\
        &+ \rho^2 \cos^2 \varphi \sin \varphi \sin^2 \theta 
        d \rho  \wedge d \varphi \wedge d \theta +
        \rho^2  \sin^3 \varphi \sin^2 \theta 
        d \rho  \wedge d \varphi \wedge d \theta\\
        &= \left(  \rho ^2 \sin^3 \varphi + 
        \rho^2 \cos^2 \varphi \sin \varphi \right) d \rho \wedge
        d \varphi \wedge d \theta\\
        &= \rho^2 \sin \varphi d \rho  \wedge d \varphi \wedge
        d \theta
    \end{align*}
\end{solution}


\begin{exercise}[Wedge product and cross product]
    The correspondence between differential forms and
    vector fields on an open subset of $\mathbb{R}^3$ in
    subsection 4.6 also makes sense pointwise. Let
    $V$ be a vector space of dimension 3 with basis
    $e_1, e_2, e_3$ and dual basis $\alpha^1, \alpha^2, \alpha^3$.
    To a $1$-covector $\alpha = a_1 \alpha^1 + 
    a_2 \alpha^2 + a_3 \alpha^3$ on $V$, we associate vector
    $v_{\alpha} = \langle a_1, a_2, a_3 \rangle 
    \in \mathbb{R}^3$. To the $2$-covector
    \[
    \gamma = c_1 \alpha^2 \wedge \alpha^3 + c_2 \alpha^3 \alpha^1
    + c_3 \alpha^1 \wedge \alpha^2
    \] 
    on $V$, we associate the vector $v_{\gamma} =
    \langle c_1, c_2, c_3 \rangle \in \mathbb{R}^3$. Show that
    under this correspondence, the wedge product of
    $1$-covectors corresponds to the cross product of
    vectors in $\mathbb{R}^3$ : if
    $\alpha = a_1 \alpha^1 + a_2 \alpha^2 + a_3 \alpha^3$ and
    $\beta = b_1 \alpha^1 + b_2 \alpha^2 + b_3 \alpha^3$, then
    $v_{\alpha \wedge \beta} = 
    v_{\alpha} \times v_{\beta}$.
\end{exercise}

\begin{proof}
    We have
    \begin{align*}
        \alpha \wedge \beta
        &= \left( a_1 b_2 - a_2 b_1 \right) \alpha^1 \wedge
        \alpha^2 + \left( a_3 b_1 - a_1 b_3 \right) \alpha^3 \wedge
        \alpha^1 + \left( a_2 b_3 - b_2 a_3 \right) \alpha^2 \wedge
        \alpha^3
    \end{align*}
    which corresponds to
    \[
    v_{\alpha \wedge \beta} 
    =\begin{pmatrix} 
        a_2 b_3 - b_2 a_3 \\
        a_3 b_1 - a_1 b_3\\
        a_1 b_2 - b_1 a_2\\
    \end{pmatrix} 
    = \det 
    \begin{pmatrix} 
        e_1 & e_2 & e_3\\
        a_1 & a_2 & a_3\\
        b_1 & b_2 & b_3
    \end{pmatrix} 
    =
    \begin{pmatrix} 
        a_1\\
        a_2\\
        a_3
    \end{pmatrix} \times 
    \begin{pmatrix} 
        b_1 \\b_2 \\b_3
    \end{pmatrix} 
    = v_a \times v_b
    \] 
    
\end{proof}

\begin{example}[Smoothness of a projection map]
    Let $M$ and $N$ be manifolds and $\pi  \colon
    M \times N \to M, \pi\left( p,q \right) =p$ the projection
    to the first factor. Prove that $\pi$ is a $C^{\infty}$ map.
\end{example}

\begin{proof}
    Suppose $\left( p,q \right) \in M \times N$. 
    Choose a chart  $\left( U, \varphi \right) $ around $p$ in
    $M$ and a chart $\left( V, \psi \right) $ around $q$ in $N$.
    Then $\left( U \times V, \varphi \times \psi \right) $ is
    a chart around $\left( p,q \right) $ in $M \times N$.
    Now $\varphi \circ \pi \circ \left( \varphi \times \psi \right)^{-1}
     \colon \varphi (U) \times \psi (V) \to \mathbb{R}^{m}$ is the
     projection map onto the first coordinate on 
     $\varphi (U) \times \psi(V)$ which is $C^{\infty}$. Hence
     $\pi$ is $C^{\infty}$.
\end{proof}



%bibliography{refs}
\end{document}
