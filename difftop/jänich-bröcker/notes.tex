\documentclass[reqno]{amsart}
\usepackage{amscd, amssymb, amsmath, amsthm}
\usepackage{graphicx}
\usepackage[colorlinks=true,linkcolor=blue]{hyperref}
\usepackage[utf8]{inputenc}
\usepackage[T1]{fontenc}
\usepackage{textcomp}
\usepackage{babel}
%% for identity function 1:
\usepackage{bbm}
%%For category theory diagrams:
\usepackage{tikz-cd}


\setlength\parindent{0pt}

\pdfsuppresswarningpagegroup=1

\newtheorem{theorem}{Theorem}[section]
\newtheorem{lemma}[theorem]{Lemma}
\newtheorem{proposition}[theorem]{Proposition}
\newtheorem{corollary}[theorem]{Corollary}
\newtheorem{conjecture}[theorem]{Conjecture}

\theoremstyle{definition}
\newtheorem{definition}[theorem]{Definition}
\newtheorem{example}[theorem]{Example}
\newtheorem{exercise}[theorem]{Exercise}
\newtheorem{problem}[theorem]{Problem}
\newtheorem{question}[theorem]{Question}

\theoremstyle{remark}
\newtheorem*{remark}{Remark}
\newtheorem*{note}{Note}
\newtheorem*{solution}{Solution}



%Inequalities
\newcommand{\cycsum}{\sum_{\mathrm{cyc}}}
\newcommand{\symsum}{\sum_{\mathrm{sym}}}
\newcommand{\cycprod}{\prod_{\mathrm{cyc}}}
\newcommand{\symprod}{\prod_{\mathrm{sym}}}

%Linear Algebra

\DeclareMathOperator{\Span}{span}
\DeclareMathOperator{\im}{im}
\DeclareMathOperator{\diag}{diag}
\DeclareMathOperator{\Ker}{Ker}
\DeclareMathOperator{\ob}{ob}
\DeclareMathOperator{\Hom}{Hom}
\DeclareMathOperator{\Mor}{Mor}
\DeclareMathOperator{\sk}{sk}
\DeclareMathOperator{\Vect}{Vect}
\DeclareMathOperator{\Set}{Set}
\DeclareMathOperator{\Group}{Group}
\DeclareMathOperator{\Ring}{Ring}
\DeclareMathOperator{\Ab}{Ab}
\DeclareMathOperator{\Top}{Top}
\DeclareMathOperator{\hTop}{hTop}
\DeclareMathOperator{\Htpy}{Htpy}
\DeclareMathOperator{\Cat}{Cat}
\DeclareMathOperator{\CAT}{CAT}
\DeclareMathOperator{\Cone}{Cone}
\DeclareMathOperator{\dom}{dom}
\DeclareMathOperator{\cod}{cod}
\DeclareMathOperator{\Aut}{Aut}
\DeclareMathOperator{\Mat}{Mat}
\DeclareMathOperator{\Fin}{Fin}
\DeclareMathOperator{\rel}{rel}
\DeclareMathOperator{\Int}{Int}
\DeclareMathOperator{\sgn}{sgn}
\DeclareMathOperator{\Homeo}{Homeo}
\DeclareMathOperator{\SHomeo}{SHomeo}
\DeclareMathOperator{\PSL}{PSL}
\DeclareMathOperator{\Bil}{Bil}
\DeclareMathOperator{\Sym}{Sym}
\DeclareMathOperator{\Skew}{Skew}
\DeclareMathOperator{\Alt}{Alt}
\DeclareMathOperator{\Quad}{Quad}
\DeclareMathOperator{\Sin}{Sin}
\DeclareMathOperator{\Supp}{Supp}
\DeclareMathOperator{\Char}{char}
\DeclareMathOperator{\Teich}{Teich}
\DeclareMathOperator{\GL}{GL}
\DeclareMathOperator{\tr}{tr}


%Row operations
\newcommand{\elem}[1]{% elementary operations
\xrightarrow{\substack{#1}}%
}

\newcommand{\lelem}[1]{% elementary operations (left alignment)
\xrightarrow{\begin{subarray}{l}#1\end{subarray}}%
}

%SS
\DeclareMathOperator{\supp}{supp}
\DeclareMathOperator{\Var}{Var}

%NT
\DeclareMathOperator{\ord}{ord}

%Alg
\DeclareMathOperator{\Rad}{Rad}
\DeclareMathOperator{\Jac}{Jac}

%Misc
\newcommand{\SL}{{\mathrm{SL}}}
\newcommand{\mobgp}{{\mathrm{PSL}_2(\mathbb{C})}}
\newcommand{\id}{{\mathrm{id}}}
\newcommand{\Mod}{{\mathrm{Mod}}}
\newcommand{\PMod}{{\mathrm{PMod}}}
\newcommand{\SMod}{{\mathrm{SMod}}}
\newcommand{\ud}{{\mathrm{d}}}
\newcommand{\Vol}{{\mathrm{Vol}}}
\newcommand{\Area}{{\mathrm{Area}}}
\newcommand{\diam}{{\mathrm{diam}}}
\newcommand{\End}{{\mathrm{End}}}


\newcommand{\reg}{{\mathtt{reg}}}
\newcommand{\geo}{{\mathtt{geo}}}

\newcommand{\tori}{{\mathcal{T}}}
\newcommand{\cpn}{{\mathtt{c}}}
\newcommand{\pat}{{\mathtt{p}}}

\let\Cap\undefined
\newcommand{\Cap}{{\mathcal{C}}ap}
\newcommand{\Push}{{\mathcal{P}}ush}
\newcommand{\Forget}{{\mathcal{F}}orget}




\begin{document}
\section*{Chapter 1}

\begin{exercise}[4]
    Let $M$ be a differentiable manifold and
    $\tau \colon M \to M$ be a fixed point free involution, i.e.,
    a diffeomorphism with $\tau \circ \tau = \id_M$ and
    $\tau(x) \neq x$ for all $x$.
    Show that the quotient space $M / \tau$ is 
    a topological manifold possessing a unique differentiable
    structure making the projection
    $M \to M/\tau $ locally diffeomorphic.
\end{exercise}

\begin{proof}
    \textit{Hausdorff:} let $x,y \in M / \tau$ be distinct.
    The preimage contains points $x_1,x_2$ and
    $y_1,y_2$. Choose open disjoint subsets $U_1, U_2$ containing
    $x_1$ and $x_2$ respectively which is disjoint from
    $y_1,y_2$. We can modify 
    $U_1$ to be $U_1 \cap \tau (U_2)$, and
    then let $U_2 = \tau (U_1)$. 

    Let $V_1, V_2$ be disjoint open neighborhoods of
    $y_1,y_2$ respectively, disjoint from $U_1,U_2$.
    Replace $V_1$ by $V_1 \cap \tau (V_2)$ and
    let $V_2 = \tau(V_1)$. Then
    $U :=U_1 \cup U_2$ is saturated with respect to
    $\tau$ and $V := V_1 \cup  V_2$ is also, hence
    they descend to open sets
    in $M / \tau$ which are again disjoint. 
    Indeed, suppose $\pi \colon M \to M/\tau$ is the quotient map
    and $\overline{z} \in \pi U \cap \pi V$. Then
    there exist  $z_1 \in U$ and $z_2 \in V$ such that
    $\tau \left( z_1 \right) = z_2$. But this contradicts
    $U \cap V = \varnothing$.\\

    \textit{Second-countable:} Take a countable
    open cover $\mathcal{B}$ of $M$. Now take the countable
    cover $\mathcal{B}' :=
    \left\{ U \cup  \tau (U) \colon U \in \mathcal{B} \right\} $.
    This descends to a countable cover on $M / \tau$.\\

    \textit{Locally-homeomorphic to $\mathbb{R}^{n}$ :}

    Let $\overline{x} \in M / \tau$. Choose
    a chart $\left( \varphi_x, U_x \right) $ around
    $x \in M$. If necessary, intersect
    $U_x$ with $U_1$ constructed from before. This
    then implies that $\tau(U_x) \cap U_x = \varnothing$.
    Then $U_x \cup \tau \left( U_x \right) $ is a 
    saturated open neighborhood descending to an
    open neighborhood $\overline{U_x}$ of $\overline{x}$.
    Now $\pi|_{U_x} \colon U_x \to \overline{U_x}$ is a
    homeomorphism, so we can define a chart for $\overline{x}$ 
    as $\left( \varphi_x \circ \left( \pi|_{U_x} \right)^{-1},
    \overline{U_x} \right) $. This constitutes an
    atlas for $M / \tau$.

    For good measure, we prove the lemma

    \begin{lemma}[]
        $\pi|_{U_x} \colon U_x \to \overline{U_x}$ is a
        homeomorphism.
    \end{lemma}
    \begin{proof}
        Suppose $\pi(y) = \pi(z)$ for $y,z \in U_x$ distinct.
        That forces $y = \tau(z)$. However, then
        $\tau(U_x) \cap U_x \neq \varnothing$, which
        is a contradiction by construction. Thus $\pi$ is
        injective on $U_x$.
        
        Now, an injective quotient map
        is a homeomorphism, giving the desired result.
    \end{proof}

    Lastly, we must prove that
    \textit{there exists a structure making the quotient
    $\pi \colon M \to  M / \tau$ a local diffeomorphism and it
    is the unique such structure:} 
    by construction of the charts on
    $M / \tau$, we can choose transition maps
    giving the identity as a coordinate representation.
    Hence the structure we constructed indeed gives
    a local diffeomorphism $M \cong M / \tau$.

    Any other structure making it a local diffeomorphism
    would necessarily give a local diffeomorphism
    of $M / \tau$ in the two structures, thus
    forcing all charts to be compatible in the two structures,
    and by maximality, it forces the structures
    to be the same.
\end{proof}

\begin{exercise}[5]
    Show that $\mathbb{R}\mathbb{P}^{1} \cong S^{1}$.
\end{exercise}
\begin{proof}
    This now follows by applying the
    previous exercise to $S^{1}$ with
    $\tau \colon S^{1} \to S^{1}$ being
    the antipodal map
    $\tau(x) = -x$. Indeed $S^{1} / \tau \cong \mathbb{R}\mathbb{P}^{1}
    $ is the usual quotient construction for $\mathbb{R}\mathbb{P}^{1}$.
\end{proof}



\begin{exercise}[9]
    Prove that if $M$ is a non-empty, $n$-dimensional 
    smooth manifold
    and $k\le n$, then there is an embedding
    $\mathbb{R}^{k} \to M$.
\end{exercise}

\begin{proof}
    Let $x \in M$ and $\left( U, \varphi  \right) $ be
    a chart centered around $x$. Take some
    open ball $B\left( 0,r \right) \subset 
    \varphi (U)$. Then we define
    $f \colon \left( -r,r \right)^{k} 
    \to M$ as $\varphi^{-1}$. Since
    $\left( -r,r \right)^{k} $ is diffeomorphic to
    $\mathbb{R}^{k}$, we get a diffeomorphism
    $\mathbb{R}^{k} \to f\left( (-r,r)^{k} \right) $ if and
    only if $f$ is a diffeomorphism onto its image which
    it trivially is as the restriction of a diffeomorphism
    to a submanifold. Indeed, it is a $k$-submanifold
    as, if we let 
    $U' = \varphi^{-1} \left( B(0,r) \right) $, we
    get $\left( \varphi|_{U'}, U' \right) $ as a chart in
    the atlas, and letting
     $N = \im f$, we have
     $\varphi \left( N \cap U' \right) 
     = \varphi (N) = \left( -r,r \right)^{k}
     = B\left( 0,r \right) \cap \mathbb{R}^{k}$.
\end{proof}



    %\bibliography{../refs.bib}
\end{document}
