\documentclass[reqno]{amsart}
\usepackage{amscd, amssymb, amsmath, amsthm}
\usepackage{graphicx}
\usepackage[colorlinks=true,linkcolor=blue]{hyperref}
\usepackage[utf8]{inputenc}
\usepackage[T1]{fontenc}
\usepackage{textcomp}
\usepackage{babel}
%% for identity function 1:
\usepackage{bbm}
%%For category theory diagrams:
\usepackage{tikz-cd}


\setlength\parindent{0pt}

\pdfsuppresswarningpagegroup=1

\newtheorem{theorem}{Theorem}[section]
\newtheorem{lemma}[theorem]{Lemma}
\newtheorem{proposition}[theorem]{Proposition}
\newtheorem{corollary}[theorem]{Corollary}
\newtheorem{conjecture}[theorem]{Conjecture}

\theoremstyle{definition}
\newtheorem{definition}[theorem]{Definition}
\newtheorem{example}[theorem]{Example}
\newtheorem{exercise}[theorem]{Exercise}
\newtheorem{problem}[theorem]{Problem}
\newtheorem{question}[theorem]{Question}

\theoremstyle{remark}
\newtheorem*{remark}{Remark}
\newtheorem*{note}{Note}
\newtheorem*{solution}{Solution}



%Inequalities
\newcommand{\cycsum}{\sum_{\mathrm{cyc}}}
\newcommand{\symsum}{\sum_{\mathrm{sym}}}
\newcommand{\cycprod}{\prod_{\mathrm{cyc}}}
\newcommand{\symprod}{\prod_{\mathrm{sym}}}

%Linear Algebra

\DeclareMathOperator{\Span}{span}
\DeclareMathOperator{\Ima}{Im}
\DeclareMathOperator{\diag}{diag}
\DeclareMathOperator{\Ker}{Ker}
\DeclareMathOperator{\ob}{ob}
\DeclareMathOperator{\Hom}{Hom}
\DeclareMathOperator{\sk}{sk}
\DeclareMathOperator{\Vect}{Vect}
\DeclareMathOperator{\Set}{Set}
\DeclareMathOperator{\Group}{Group}
\DeclareMathOperator{\Ring}{Ring}
\DeclareMathOperator{\Ab}{Ab}
\DeclareMathOperator{\Top}{Top}
\DeclareMathOperator{\hTop}{hTop}
\DeclareMathOperator{\Htpy}{Htpy}
\DeclareMathOperator{\Cat}{Cat}
\DeclareMathOperator{\CAT}{CAT}
\DeclareMathOperator{\Cone}{Cone}
\DeclareMathOperator{\dom}{dom}
\DeclareMathOperator{\cod}{cod}
\DeclareMathOperator{\Aut}{Aut}
\DeclareMathOperator{\Mat}{Mat}
\DeclareMathOperator{\Fin}{Fin}
\DeclareMathOperator{\rel}{rel}
\DeclareMathOperator{\Int}{Int}
\DeclareMathOperator{\sgn}{sgn}
\DeclareMathOperator{\Char}{char}

%Row operations
\newcommand{\elem}[1]{% elementary operations
\xrightarrow{\substack{#1}}%
}

\newcommand{\lelem}[1]{% elementary operations (left alignment)
\xrightarrow{\begin{subarray}{l}#1\end{subarray}}%
}

%SS
\DeclareMathOperator{\supp}{supp}
\DeclareMathOperator{\Var}{Var}

%NT
\DeclareMathOperator{\ord}{ord}

%Alg
\DeclareMathOperator{\Rad}{Rad}
\DeclareMathOperator{\Jac}{Jac}

%Misc
\newcommand{\SL}{{\mathrm{SL}}}
\newcommand{\mobgp}{{\mathrm{PSL}_2(\mathbb{C})}}
\newcommand{\id}{{\mathrm{id}}}
\newcommand{\Mod}{{\mathrm{Mod}}}
\newcommand{\ud}{{\mathrm{d}}}
\newcommand{\Vol}{{\mathrm{Vol}}}
\newcommand{\Area}{{\mathrm{Area}}}
\newcommand{\diam}{{\mathrm{diam}}}


\newcommand{\reg}{{\mathtt{reg}}}
\newcommand{\geo}{{\mathtt{geo}}}

\newcommand{\tori}{{\mathcal{T}}}
\newcommand{\cpn}{{\mathtt{c}}}
\newcommand{\pat}{{\mathtt{p}}}


\title{Direct and semidirect product}

\begin{document}
\maketitle
\newpage

\section{Motivating direct product}

In all fields of math, we often get a better understanding of objects by
reducing them to a composition of smaller structures. Likewise, we now ask
whether we can reduce groups to smaller constituents. We have seen
that we can do this with finitely generated abelian groups using direct product, so how do we
recognize direct product generally? 

\begin{definition}[Characteristic in]
    A subgroup $H \le G$ is \textit{characteristic} in $G$, written
    $H \Char G$, if  $\sigma(H) = H$ for all 
     $\sigma \in \Aut(G)$.
\end{definition}

 Recall that
$HK = \left\{ hk  \mid h \in H, k \in K \right\} $.

\begin{proposition}[]
    Let $G$ be a group, let $x,y \in G$ and let $H \le G$. Then
    \begin{enumerate}
        \item $xy = yx \left[ x,y \right] $.
        \item $H \trianglelefteq G $ if and only if
            $\left[ H, G \right] \le H$. 
        \item $\sigma \left[ x,y \right] = 
            \left[ \sigma (x), \sigma (y) \right] $ for any 
            $\sigma \in \Aut \left( G \right) $, $\left[ G,G \right]  \Char G$ and
            $G / \left[ G,G \right] $ is abelian.
        \item $G / \left[ G,G \right] $ is the largest abelian quotient
            of $G$ in the sense that if $H \trianglelefteq G$ and $G /H$ is
            abelian, then $\left[ G,G \right] \le H$. Conversely,
            if $\left[ G,G \right] \le H$, then $H \trianglelefteq G$ and
            $G / H$ is abelian. (Note, in particular, that choosing
            $H = \left[ G,G \right] $, we get $\left[ G,G \right]
            \trianglelefteq 
            G$).
        \item If $\varphi \colon G \to A$ is any homomorphism of $G$ into an
            abelian group $A$, then $\varphi $ factors through $\left[ G,G \right] $, i.e.,
            $\left[ G,G \right]  \le \ker \varphi$ and the following diagram commutes:
            \begin{equation*}
            \begin{tikzcd}
                G \ar[r] \ar[dr, "\varphi"'] & G / \left[ G,G \right]  \ar[d]\\
                                     & A
            \end{tikzcd}
            \end{equation*}
    \end{enumerate}
\end{proposition}

\begin{proposition}[]
    Let $H,K \le G$. The number of distinct ways of writing each element
    of the set $HK$ in the form $hk$ for some $h \in H$ and $k \in K$ is
    $\left| H \cap K \right| $. In particular, if $H \cap K = 1$, then
    each element of $HK$ can be written uniquely as a product  $hk$ for
    some  $h \in H$ and $k \in K$.
\end{proposition}

\begin{theorem}[]
    Suppose $G$ is a group with subgroups $H$ and $K$ such that
    \begin{enumerate}
        \item $H$ and $K$ are normal in $G$, and
        \item $H \cap K = 1$.
    \end{enumerate}
    Then $HK \cong H \times K$.
\end{theorem}

\begin{definition}[Internal and external direct product]
    If $G$ is a group and $H$ and $K$ are normal subgroups of $G$ with
    $H \cap K = 1$, we call $HK$ the \textit{internal direct product} of
    $H$ and $K$. We shall (when emphasis is called for) call
    $H \times K$ the \textit{external direct product} of $H$ and $K$.
\end{definition}





\section{Motivating semidirect product}
If we have a group $G$ with subgroup $H$ and $K$ such that
\begin{itemize}
    \item $H \trianglelefteq G$ (but $K$ is not necessarily normal in $G$ ), and
    \item $H \cap K = 1$,
\end{itemize}
then $HK$ is a subgroup of $G$ and there is an isomorphism of \underline{sets}
$HK \cong H \times K$ given by
$(h,k) \mapsto hk$. However, this isomorphism
may \underline{not} be a group isomorphism.

In the group $HK \le G$, the operation is as follows

\begin{align*}
    \left( h_1 k_1 \right) \left( h_2 k_2 \right) 
    &= h_1 k_1 h_2 \left( k_1^{-1} k_1 \right) k_2\\
    &= h_1 \left( k_1 h_2 k_1^{-1} \right) k_1 k_2 \label{eq:Omega} \tag{$\Omega$}\\
    &= h_3 k_3
\end{align*}
where $h_3 = h_1 \left( k_1 h_2 k_1^{-1} \right) $ and
$k_3 = k_1 k_2$ - note that since $H \triangleleft G$, $k_1 h_2 k_1$, so
$h_3 \in H$ and $k_3 \in K$.\\
\linebreak
These calculations were predicated on the assumption that there already exists
such a group $G$ containing subgroups $H$ and $K$ with
$H \triangleleft G$ and $H \cap K = 1$.


\begin{question}
    Suppose we are given  two (abstract) groups $H$ and $K$.
    Can we construct a group $G$ containing isomorpic copies of
    $H$ and $K$ such that $H \triangleleft G$ and
    $H \cap K = 1$? That will ensure that $HK$ is an isomorphic copy of 
    a subgroup of $G$ and
    each element of $HK$ can be written uniquely as a product $hk$.
\end{question}

Note that in \eqref{eq:Omega}, we  multiplied $k_1$ and $k_2$ in $K$ and
$h_1$ and $k_1h_2k_1^{-1}$ in $H$. So if we can understand how to obtain
$k_1 h_2 k_1^{-1}$ without reference to an operation in $G$, we can define
the product of $\left( h_1,  k_1 \right) \left( h_2 ,k_2 \right) $ to
be $\left( h_1 \left( k_1 h_2 k_1^{-1} \right) , k_1k_2 \right)$ whatever
$k_1 h_2 k_1^{-1}$ is determined to mean.

Here is one thing to notice: $H$ is normal in $G$, so $K$ acts on $H$ by
conjugation:
\[
k \cdot  h = k h k^{-1}, \quad \text{for } h \in H, k \in K.
\] 
Thus
\[
    \left( h_1 k_1 \right) \left( h_2 k_2 \right) =
    \left( h_1 k_1 \cdot  h_2 \right) \left( k_1 k_2 \right) 
\] 
Let us introduce some new notation

\begin{definition}[Conjugation]
    We define the conjugate $y^{x} := x^{-1} y x$.
    In a group, this satisfies the following properties of exponents:
    \begin{align*}
        \left( y_1 y_2 \right)^{x} &= y_1^{x} y_2^{x}\\
        y^{x_1 x_2} &= \left( y^{x_1} \right)^{x_2}
    \end{align*}
\end{definition}

Thus we can rewrite
\[
    \left( h_1 k_1 \right) \left( h_2k_2 \right) 
    = \left( h_1 h_2^{k_1} \right) \left( k_1 k_2 \right) 
\] 

So assuming $G$ exists, the action of $K$ on $H$ by conjugation gives
a homomorphism $\phi \colon K \to \Aut (H)$. Thus multiplication in $HK$
depends only on multiplication in $H$, multiplication in $K$ and the
homomorphism $\phi \colon K \to \Aut(H)$.\\
\linebreak
Now we assume these things and reverse the construction

\begin{theorem}[]\label{semi-direct-prod}
    Let $H$ and $K$ be groups and let $\varphi \colon K \to \Aut (H)$ be
    a homomorphism. Let $h^{k}$ denote the action of $k$ on $h$ determined by
    $\varphi$ for arbitrary $k$ and $h$. Let $G$ be the set of ordered
    pairs $\left( h,k \right) $ with $h \in H$ and $k \in K$ and define the
    following multiplication on $G$ :
    \[
        \left( h_1, k_1 \right) \left( h_2, k_2 \right) =
        \left( h_1 h_2^{k_1}, k_1k_2 \right).
    \] 
    \begin{enumerate}
        \item This multiplication makes $G$ into a group of order
            $\left| G \right| = \left| H \right| \left| K \right| $.
        \item The sets $\left\{ \left( h,1 \right) \mid h \in H  \right\} $
            and $\left\{ \left( 1,k \right)  \mid k \in K \right\} $ are
            subgroups of $G$ and the maps $h \mapsto (h,1)$ for
            $h \in H$ and $k \mapsto \left( 1,k \right) $ for $k \in K$ 
            are isomorphisms of these subgroups with the groups $H$ and $K$,
            respectively:
            \[
            H \cong \left\{ \left( h,1 \right)  \mid h \in H \right\} ,
            \quad \text{and} \quad K \cong \left\{ \left( 1,k \right)  \mid 
            k \in K \right\} .
            \] 
            Identifying $H$ and $K$ with their isomorphic copies in $G$ 
            described in (2), we have
        \item $H \trianglelefteq G$ 
        \item $H \cap K = 1$ 
        \item for all $h \in H$ and $k \in K$, $k h k^{-1} = h^{k}
            = \varphi (k) (h)$.
    \end{enumerate}
\end{theorem}

Here, part (5) is carried out using the identifications from (2) as follows:
\begin{align*}
    (1,k) (h,1) (1,k)^{-1}
    &=
    \left( \left( 1,k \right) (h,1) \right) (1,k)^{-1}\\
    &= \left(  h^{k}, k \right) \left( 1,k^{-1} \right)\\
    &= \left( h^{k} 1^{k}, 1 \right) \\
    &= \left( h^{k}, 1 \right) 
\end{align*}
so indeed $khk^{-1} = h^{k}$ under the identifications.



\begin{definition}[Semidirect product]
    Let $H$ and $K$ be groups and $\varphi $ be a homomorphism from $K$ into
    $\Aut(H)$. The group described in \ref{semi-direct-prod} is called
    the \textit{semidirect product} of $H$ and $K$ with respect to
    $\varphi $ and will be denoted by $H \rtimes_\varphi  K$ (when there
    is no danger of confusion, we shall simply write $H \ltimes K$ ).
\end{definition}


\begin{proposition}[]\label{Prop:JSAO2}
    Let $H$ and $K$ be groups and let $\varphi  \colon K \to \Aut(H)$ be
    a homomorphism. Then the following are equivalent:
    \begin{enumerate}
        \item the identity (set) map between 
            $H \rtimes K$ and $H \times K$ is a group homomorphism (hence an
            isomorphism)
        \item $\varphi $ is the trivial homomorphism from $K$ into $\Aut(H)$ 
        \item $K \trianglelefteq H \rtimes K$.
    \end{enumerate}
\end{proposition}

\begin{corollary}
    When $H$ is abelian, $H \rtimes K \cong H \oplus K = H \times K$.
\end{corollary}

\begin{proof}
    When $H$ is abelian, 
    $h = k h k^{-1} = h^{k} = \varphi (k) (h)$, so
    $\varphi (k) = \id$ for all $k$, so
    $\varphi \colon K \to \Aut (H)$ is the trivial homomorphism.
    Now Proposition \ref{Prop:JSAO2} gives that
    $H \rtimes K \cong H \times K = H \oplus K$.
\end{proof}
    
    %\bibliography{../refs.bib}
\end{document}
